\documentclass{novela}
\title{Robinson Crusoe}
\author{Daniel Defoe}
\AnotherAuthor{Traducción: Julio Cortázar}
\date{}

\begin{document}
	\maketitle
	\tableofcontents
	%\part{}
    1. PRIMERAS AVENTURAS DE ROBINSON

    Nací en el año 1632 en la ciudad de York, de buena familia aunque no del país, pues mi padre, oriundo de Bremen, se había dedicado al comercio en Hull, donde logró una buena posición. Desde entonces, y luego de abandonar su trabajo, se radicó en York, donde casó con mí madre; ésta pertenecía a los Robinson, una distinguida familia de la región, y de ahí que yo fuera llamado Robinson Kreutznaer, aunque por la habitual corrupción de voces en Inglaterra se nos llama Crusoe, nombre que nosotros mismos nos damos y escribimos y con el cual me han conocido siempre mis compañeros.
    Siendo el tercero de los hijos, y no preparado para ninguna carrera, mi cabeza empezó a llenarse temprano de desordenados pensamientos. Mi anciano padre me había dado la mejor educación que el hogar y una escuela común pueden proveer, y me destinaba a la abogacía; pero yo no ansiaba otra cosa que navegar y mi inclinación a los viajes me hizo resistir tan fuertemente la voluntad y las órdenes de mi padre, así como las persuasiones de mi madre y mis amigos, que se hubiera dicho que existía algo de fatal en esa tendencia que me arrastraba directamente hacia un destino miserable.
    Mi padre, hombre prudente y serio, trató con sus excelentes consejos de hacerme abandonar el intento que había adivinado en mí. Una mañana me llamó a su habitación, donde lo retenía la gota, para hacerme cordiales advertencias sobre mis proyectos. Con su tono más afectuoso me rogó que no cometiera una chiquillada y me precipitara a desdichas que la naturaleza y mi posición en la vida parecían propicias a evitarme; no tenía yo necesidad de ganarme el pan puesto que él me ayudaría con su impulso a obtener la situación acomodada que me había destinado; en fin, si no lograba una posición en el mundo sería sólo por culpa mía o del destino, sin que tuviera él que rendir cuentas de ello, ya que cumplía con su deber al prevenirme contra actitudes que sólo redundarían en mi desgracia; en una palabra, me aseguró que haría mucho por mí si me quedaba en casa, pero que no quería tener participación alguna en mis desventuras alentándome a partir. Para terminar me señaló el ejemplo de mi hermano mayor, con el cual había empleado el mismo género de persuasiones a fin de evitar que fuera a las guerras de Flandes, no pudiendo sin embargo impedir que sus juveniles impulsos lo llevaran a la lucha donde encontró la muerte. Me aseguró que no dejaría de rogar por mí, pero que se aventuraba a decirme que si me dejaba arrastrar por mi impulso Dios no me acompañaría, quedándome sobrado tiempo para lamentar haber desoído los consejos paternales y ello cuando ya nadie pudiera acompañarme en mi arrepentimiento.
    Sus palabras me afectaron profundamente, como es natural, y resolví abandonar toda idea de viajes estableciéndome en casa de acuerdo con la voluntad paterna. Mas, ¡ay!, muy pocos días disiparon los buenos propósitos, y unas semanas después me decidí a evitar lo que consideraba importunidades de mi padre yéndome de su lado. Sin embargo, no permití que el calor de mi resolución me arrastrara. Y acudiendo a mi madre un día en que la creí de mejor humor que otras veces le confié que mis deseos de conocer el mundo eran tan irresistibles que jamás podría dedicarme a cosa alguna que me lo impidiera, y agregué que mi padre haría mejor en darme su consentimiento que obligarme a partir sin él. Ya tenía yo dieciocho años, edad demasiado avanzada para entrar de aprendiz en cualquier comercio o como pasante en un bufete, y si me forzaban a ello estaba seguro de escapar de mi amo a toda costa y lanzarme al mar. Por fin le aseguré que si convencía a mi padre de que me dejara partir y a mi regreso encontraba yo que el viaje no me había gustado, le prometía no volver a intentarlo jamás y rescatar, con todo celo y diligencia, el tiempo perdido.
    Todo esto sólo sirvió para encolerizar a mi madre. Me dijo que era vano hablar a mi padre del asunto, que lo sabía demasiado seguro de cuál era el camino provechoso para dar un consentimiento que sólo sería mi desgracia, y se maravilló de que pudiera insistir después de la conversación que había tenido con él y las tiernas y bondadosas frases que había empleado conmigo; en fin, si yo estaba dispuesto a perderme no había manera de impedirlo, pero jamás mi intención lograría el consentimiento de ambos; por su parte no estaba dispuesta a colaborar en mi ruina y nunca podría decirse de ella que había obrado contra la voluntad de su esposo.
    Aunque se cuidó de decir todo esto a mi padre, vine a saber más tarde que le contó lo ocurrido y que el anciano, tras de mostrar gran preocupación, dijo suspirando:
    —El muchacho sería dichoso si se quedara en casa, pero si se lanza a viajar será el hombre más infeliz que haya pisado la tierra. No puedo darle mi consentimiento.
    Sólo un año después de todo esto dejé mi casa, aunque entretanto me mantuve sordo a toda proposición que se me hizo de dedicarme al comercio, y discutía frecuentemente con mis padres sobre lo que yo consideraba su empecinamiento contra mis más ardientes inclinaciones. Pero un día, hallándome casualmente en Hull y sin la menor intención de escaparme en esa oportunidad, encontré un amigo que se embarcaba para Londres en el barco de su padre y que me instó a que lo acompañara, valiéndose del cebo habitualmente empleado por los marinos, esto es, que el pasaje no me costaría nada. Sin consultar a mis padres ni comunicarles mi partida, dejándolos que se enteraran como pudiesen; sin pedir la bendición de Dios ni la de mi padre y sin cuidado alguno de las circunstancias y las consecuencias de mi acción, en un día aciago como Dios sabe, el primero de septiembre de 1651 me embarqué en aquel navío rumbo a Londres. No creo que las desgracias de ningún muchacho aventurero hayan comenzado tan pronto y durado tanto. Apenas habíamos salido del Humber cuando se desató el viento y las olas empezaron a encresparse horriblemente; yo, que jamás había estado en el mar, sufrí a la vez el padecimiento del cuerpo y el terror del alma. Me puse a pensar seriamente en lo que había hecho, y con qué justicia me castigaba el cielo por mi perversa conducta al abandonar la casa de mi padre y mi deber.
    Entretanto la tormenta crecía y el mar, aún desconocido para mí, parecía levantarse, aunque nunca en la forma en que lo vi más adelante; no, nunca como lo vi unos días después. Pero entonces bastaba para impresionar a un joven marino que no tenía noción alguna al respecto. Me parecía que cada ola iba a tragarnos, y que cada vez que el barco se hundía, en lo que a mí me daba la impresión de ser el fondo del mar, jamás volvería a surgir a la superficie. En tal estado de terror hice solemnes promesas y adopté la resolución de que si Dios llevaba su bondad a perdonarme la vida y me permitía desembarcar a salvo, iría directamente a la casa de mis padres para no volver a pisar la cubierta de una nave en lo que me quedara de vida. Prometí también que seguiría el consejo paterno sin precipitarme nunca más en tan miserables andanzas; veía claramente ahora la justeza de sus palabras acerca de una cómoda medianía en la vida, cuan fácil y confortable había transcurrido para él la existencia, lejos de toda tempestad en el mar y conflicto en la tierra; y decidí volver, como el hijo pródigo, a casa de mis padres.
    Mis prudentes y sosegados pensamientos duraron lo que la tormenta y hasta un poco más; pero al día siguiente el viento había amainado, el mar estaba menos revuelto y yo comencé a habituarme a ambos. No obstante me mantuve serio todo el día, a lo que hay que sumar un resto de mareo, pero hacia la tarde el tiempo aclaró completamente, el viento cesó en absoluto y tuvimos un hermoso crepúsculo. Con igual claridad que al ponerse se levantó el sol a la siguiente mañana; soplaba apenas una brisa, el mar estaba terso y el sol, brillando sobre las aguas, componía el más hermoso de los espectáculos que me fuera dado ver.
    Habiendo dormido profundamente me sentía ya libre del mareo, y lleno de ánimo miraba maravillado el mar tan terrible el día anterior y capaz de mostrarse tan sereno y agradable muy poco después. Entonces, como para impedir que continuaran mis buenas resoluciones, el camarada que me había impulsado a embarcarme se me acercó y me dijo, palmeándome el hombro:
    —Y bien, Bob... ¿cómo lo has pasado? Apuesto a que te diste un buen susto anoche, y eso que no sopló más que una ráfaga.
    — ¿Le llamas ráfaga? —exclamé—. ¡Pero si fue una terrible tormenta!
    — ¡Tormenta! —dijo mi amigo—. ¿Le llamas tormenta a eso, gran tonto? ¡Pero si no fue nada! Con un buen barco y mar abierto no nos preocupamos por un viento como ése. Es que tú eres marino de agua dulce, Bob. Ven, apuremos un jarro de ponche y nos olvidaremos de todo. ¿No ves qué hermoso tiempo hace ahora?
    Para abreviar esta lamentable parte de mi relato, diré que seguimos el camino de todos los marinos; el ponche fue servido, yo me embriagué con él y en el desorden de aquella noche abandoné todo arrepentimiento, mis reflexiones sobre el pasado y mis resoluciones acerca del futuro. En algunos momentos de meditación, empero, aquellos pensamientos parecían esforzarse por retornar a mí, pero me apresuraba a rechazarlos y me salía de ellos como de una enfermedad. Así, dedicándome a beber y a alternar con los camaradas, pronto dominé aquellos ataques —como yo los llamaba— y en cinco o seis días logré la más completa victoria sobre la conciencia que pudiera desear un muchacho resuelto a no escucharla. Pero otra prueba me esperaba, y la Providencia, tal como lo hace en casos así, resolvió dejarme esta vez sin la menor excusa en mi futura conducta; porque si el primer episodio podía no parecerme una advertencia, el siguiente fue tal que el peor y más empedernido miserable entre nosotros hubiera admitido a la vez el peligro y la gracia.
    Al sexto día de navegación entramos en la rada de Yarmouth; con viento contrario y tiempo sereno, habíamos avanzado muy poco desde la tormenta. Nos vimos obligados a anclar en la rada y quedarnos allí, mientras el viento soplaba continuamente del sudoeste, por espacio de siete u ocho días, durante los cuales muchos barcos provenientes de Newcastle entraron en la rada, puerto común donde los navíos podían aguardar viento favorable para remontar el río.
    Sin embargo, no hubiéramos permanecido tanto tiempo allí sin remontar el río de no levantarse un viento que, entre el cuarto y quinto día, empezó a soplar con furia. Con todo, aquellas radas eran consideradas tan seguras como un puerto y estábamos muy bien y sólidamente anclados, por lo cual nuestros hombres no se preocupaban, en un todo ajenos al peligro, y pasaban el tiempo en diversiones y descanso como todo marino. Pero en la mañana del octavo día el viento arreció, y fue necesario que toda la tripulación se lanzara a calar los masteleros y aligerar lo bastante para que el buque se mantuviera fondeado lo mejor posible. A mediodía creció el mar, y el castillo de proa se hundía mientras las olas barrían la cubierta, al extremo de que llegamos a creer que el ancla se había cortado y el capitán mandó echar el ancla de esperanza, con lo cual el barco se mantuvo con dos anclas y los cables tendidos hasta las bitas.
    Esta vez era verdaderamente un terrible temporal, y yo comencé a ver señales de espanto hasta en el rostro de los marinos. El capitán atendía las maniobras para preservar el barco, pero mientras entraba y salía de su cabina y pasaba cerca de mí le oí decir varias veces:
    — ¡Dios se apiade de nosotros, nos ahogaremos todos, estamos perdidos!
    Durante los primeros momentos, yo permanecí en mi camarote de proa como petrificado, y no podría describir lo que pasaba por mí. Me dolía recordar mi primer arrepentimiento, del que aparentemente me había sido tan fácil librarme y contra el cual me había endurecido; pensaba que no había peligro de muerte y que el temporal amainaría como el otro. Pero cuando el capitán pasó cerca de mí y le oí decir que estábamos todos perdidos me espanté horriblemente y levantándome de mi cucheta me asomé fuera. Jamás había visto un espectáculo tan espantoso; el mar se hinchaba como si fueran montañas y nos barría a cada instante; cuanto veían mis ojos en torno era desolación. En dos barcos anclados cerca de nosotros habían cortado los mástiles por exceso de arboladura, y nuestros marineros gritaban que un navío fondeado a una milla del nuestro acababa de naufragar. Otros dos barcos que habían perdido sus anclas eran arrebatados de la rada hacia el mar, librados a su suerte. Los barcos livianos resistían mejor el embate, pero dos o tres de ellos pasaron desmantelados frente a nosotros, huyendo con sólo la botavara al viento.
    Hacia la tarde, el piloto y el contramaestre pidieron al capitán que les dejara cortar el palo de trinquete. Aunque se negó al principio, las protestas del contramaestre que aseguraba que el buque se hundiría en caso contrario lo llevaron a consentir; pero cuando cayó el mástil se vio que el palo mayor quedaba suelto y sacudía de tal manera el barco que fue necesario cortarlo a su vez y dejar la cubierta arrasada.
    Cualquiera puede inferir en qué estado de ánimo estaría yo a todo esto, siendo un novato en el mar y habiendo pasado poco antes tanto miedo por una simple ráfaga. Pero —sí me es posible describir ahora los pensamientos que me asaltaban entonces— recuerdo que sentía diez veces más miedo por haber abominado de mis anteriores resoluciones y recaído en los malos designios que por la idea de la muerte. Eso, agregado al espanto de la tormenta, me ocasionó un estado de ánimo que jamás podría narrar. Y sin embargo lo peor no había sobrevenido aún; el temporal continuaba con tal furia que los mismos marineros aseguraban no haber visto jamás uno semejante. Teníamos un buen barco, pero excesivamente cargado y calaba tanto que los marineros esperaban verlo irse a pique a cada momento. El único alivio que se me brindó entonces fue ignorar el sentido de la expresión «irse a pique», hasta que lo supe más tarde. Pude entonces ver en medio de la furia de la tormenta algo que no es frecuente: al capitán, al contramaestre y algunos otros más cuerdos que el resto, elevando sus ruegos mientras el navío parecía zozobrar a cada instante. A mitad de la noche, y para colmo de nuestras desventuras, uno de los marineros que descendiera de intento para observar la cala volvió gritando que el barco hacía agua; otro hombre aseguró que ya había cuatro pies en la bodega. De inmediato se llamó a todos a las bombas, y cuando oí esa palabra el corazón pareció dejar de latirme en el pecho y caí de espaldas sobre la cucheta donde había estado sentado. Pronto, sin embargo, los marineros vinieron a decirme que si hasta entonces no había sido capaz de ayudar en nada, bien podía hacerlo en una bomba como cualquier otro. Me levanté y obedecí poniendo todas mis fuerzas en el trabajo. Entretanto el capitán había divisado algunos barcos carboneros que, incapaces de resistir anclados la tormenta, se veían obligados a salir de la rada y lanzarse al mar; como habían de pasar cerca de nosotros, ordenó el capitán disparar un cañonazo en demanda de socorro. Yo no sabía lo que eso significaba y me sorprendí tanto que me pareció que el barco se había partido en dos o que acababa de ocurrir alguna otra cosa tremenda. Para decirlo en una palabra, me desmayé. En aquella hora cada uno tenía su propia vida que cuidar, y naturalmente nadie se preocupó por lo que pudiera haberme ocurrido; otro marinero que vino a la bomba me hizo a un costado con el pie, creyendo seguramente que había muerto, y pasó un largo rato antes de que recobrara el sentido.
    Trabajábamos más y más, pero el agua crecía en la bodega y era evidente que terminaríamos por hundirnos; aunque la tormenta había decrecido un poco no parecía probable que pudiéramos sostenernos a flote hasta entrar en puerto, por lo cual el capitán siguió disparando cañonazos. Un barco pequeño que estaba anclado justamente delante de nosotros osó enviar un bote en nuestro auxilio. Fue harto afortunado que el bote pudiera acercarse, pero nos resultaba imposible transbordar a él así como al bote mantenerse al costado, hasta que los remeros, con un supremo esfuerzo en el que exponían sus vidas para salvar las nuestras, consiguieron alcanzar el cable que por la popa les tiramos con una boya al extremo, y después de infinitas dificultades los remolcamos hasta nuestra popa y pudimos así transbordar. No era su propósito volver al navío de donde partieran, de modo que estuvimos de acuerdo en dejarnos llevar por el viento y solamente encaminar en lo posible el bote hacia tierra firme; nuestro capitán, por su parte, aseguró que si la embarcación se averiaba al tocar la costa, él indemnizaría a su dueño y con eso, remando algunos y otros dirigiendo el rumbo, fuimos hacia el norte sesgando la costa casi a la altura de Winterton Ness.
    Mientras los hombres se inclinaban sobre los remos tratando de acercar el bote a tierra, y en los momentos en que éste, al montar sobre una ola, nos permitía la visión de la costa, podíamos distinguir una gran cantidad de gentes corriendo por ella con intención de ayudarnos. Pero avanzábamos con gran lentitud y no pudimos alcanzar la costa hasta más allá del faro de Winterton, donde hace una entrada hacia el oeste en dirección a Cromer y, por tanto, la misma tierra protege al mar contra la violencia del viento. Allí desembarcamos no sin bastantes dificultades, y fuimos a pie hacia Yarmouth donde nuestra desgracia fue aliviada por la generosidad de todos, desde los magistrados de la ciudad que nos dieron buen alojamiento hasta los comerciantes y propietarios de barcos, que nos facilitaron suficiente dinero para ir a Londres o retornar a Hull, según nuestra voluntad.
    Si hubiera tenido entonces bastante sensatez para volver a Hull y a mi hogar, habría encontrado allí la felicidad, y mi padre, como un emblema de la parábola de Nuestro Señor, habría matado para mí el ternero cebado; en verdad, al enterarse de la desgracia ocurrida en la rada de Yarmouth al barco en el cual yo había huido, pasó largo tiempo inquieto hasta asegurarse de que no me había ahogado.
    Pero mi mala estrella seguía impulsándome con una fuerza que nada podía resistir, y aunque muchas veces me sentí agobiado por el pensamiento y la voluntad de volver a casa, no encontré fuerza suficiente para hacerlo. Ignoro qué nombre debo dar a esto, ni pretendo que se trate de una secreta predestinación que nos lleva a ser instrumentos de nuestra propia ruina, aun cuando la estemos viendo y corramos hacia ella con los ojos abiertos. Por cierto que sólo una desdicha inevitablemente destinada a mí, y de la cual me era imposible escapar, podía haberme arrastrado contra todo sensato razonamiento y las persuasiones de mi propia meditación, máxime teniendo en cuenta las dos evidentes advertencias que acababa de recibir en mi primera tentativa.
    El camarada que me había empujado en mi decisión, y que era el hijo del capitán, parecía ahora mucho menos animoso que yo. La primera vez que me habló en Yarmouth, es decir, dos o tres días más tarde, porque nos alojábamos en lugares distintos, me dio la impresión de que estaba cambiado, y luego de preguntarme con aire melancólico y moviendo la cabeza cómo estaba mi salud, se volvió hacia su padre y le dijo quién era yo y cómo había intentado ese viaje a manera de prueba para más distantes expediciones. Su padre se volvió a mí con un aire a la vez grave y afectuoso, para decirme:
    —Joven, no os embarquéis nunca más. Lo que ha ocurrido debe bastaros como indudable signo de que no estáis destinado a ser marino. Estad seguro de que si no volvéis al hogar, en cualquier sitio adonde vayáis encontraréis desastres y decepciones, hasta que las palabras de vuestro padre se hayan cumplido en vos.
    Nos separamos al rato, sin que yo le hubiera contestado gran cosa, y no sé qué fue más tarde de él. Por lo que a mí respecta, dueño de algún dinero, me fui por tierra a Londres y allí, lo mismo que en el curso del viaje, sostuve duras luchas conmigo mismo para decidir cuál debería ser mi camino, si volvería a casa o al mar. De ir a casa me detenía la vergüenza, opuesta a mis mejores impulsos; se me ocurría que todos iban a reírse de mí, que no sólo me humillaría presentarme ante mis padres sino a los vecinos y amigos; y puedo decir que desde entonces he observado cuan absurdo e irracional es el carácter de los hombres, en especial en los jóvenes, que los lleva a no avergonzarse de sus faltas y sí de su arrepentimiento, que no se reprochan los actos por los cuales merecen el nombre de insensatos mientras que los humilla el retorno a la verdad que les valdría en cambio la reputación de hombres prudentes.
    Tuve suerte al hallarme a poco de mi llegada a Londres en muy buena compañía, cosa no muy frecuente en jóvenes tan libres y mal encaminados como lo era yo entonces, ya que el diablo no tarda en prepararles sus trampas. En primer lugar conocí al capitán de un barco que venía de Guinea y que, habiendo tenido allá muy buena fortuna, estaba resuelto a volver. Mi conversación, que en aquel entonces no era del todo torpe, le agradó mucho y oyéndome decir que ansiaba conocer el mundo me propuso hacer el viaje con él sin que me costara nada; sería su compañero de mesa y su camarada, sin contar que, llevando alguna cosa conmigo para comerciar, tendría todas las ventajas del intercambio y tal vez eso acrecentara mi decisión.
    Acepté la propuesta y habiéndome hecho muy amigo del capitán, que era hombre simple y honesto, emprendí viaje con él llevando conmigo una modesta pacotilla que, gracias a la desinteresada probidad de mi compañero aumentó considerablemente. Había comprado por valor de cuarenta libras las baratijas y chucherías que el capitán me aconsejaba llevar, y ese dinero fue el producto de la ayuda de algunos parientes con los cuales me mantenía en contacto, de donde infiero que mi padre, o por lo menos mi madre, contribuyeron con ello a mi primera aventura.
    Aquél fue el único viaje que puedo llamar excelente entre todas mis andanzas, y lo debo a la honesta integridad de mi amigo el capitán junto al cual adquirí además un discreto conocimiento de las matemáticas y las reglas de navegación, aprendí a llevar un diario de ruta, calcular la longitud y latitud para determinar la posición del buque y, en resumen, comprender aquellas cosas que deben ser conocidas r por un marino. Es verdad que así como él tenía placer en enseñarme yo lo tenía en aprender; y en realidad aquel viaje hizo de mí a la vez un comerciante y un marino. Traje de regreso cinco libras y nueve onzas de oro en polvo a cambio de mi pacotilla, y ello me reportó en Londres no menos de trescientas libras, terminando de llenarme de ambiciosos proyectos que desde entonces me han traído a la ruina.
    Y con todo, aun en aquel viaje tuve inconvenientes, por ejemplo, una continua enfermedad, producto de la elevada temperatura del clima que me producía calenturas; comerciábamos en la costa, desde los 15° hasta el mismo ecuador.
    Podía considerarme ya un comerciante de Guinea, y cuando para desdicha mía a poco de desembarcar falleció mi amigo, me resolví a emprender nuevamente el viaje y embarqué en el mismo barco capitaneado ahora por el que había sido piloto en la anterior travesía. Nadie hizo nunca un viaje menos afortunado, pues aunque sólo llevé conmigo cien libras de mi nueva fortuna, dejando las doscientas restantes en manos de la viuda de mi amigo, que las guardó celosamente, las desgracias llovieron sobre mí. La primera ocurrió cuando nuestro barco navegaba hacia las islas Canarias o, mejor, entre aquéllas y la costa africana, pues fuimos sorprendidos una mañana por un corsario turco de Sallee que empezó a perseguirnos con todas las velas desplegadas. De inmediato soltamos cuanto trapo eran capaces de soportar los mástiles, pero nuestra esperanza de ganar distancia se vio pronto desmentida por el avance de los piratas, por lo cual nos dispusimos a la lucha contando con doce cañones contra los dieciocho que tenía el buque pirata. A las tres de la tarde se puso a tiro, pero en vez de soltarnos su andanada por la popa como parecía dispuesto vino sesgando para alcanzarnos más de lleno, permitiéndonos asestarle ocho cañones de ese lado y enviarle una andanada que lo obligó a alejarse, no sin antes responder a nuestro fuego agregando a los cañones una nutrida fusilería de los doscientos hombres que tenía a bordo. Por suerte no habían herido a nadie y nuestros hombres se mantenían a cubierto. Vimos que se preparaba a atacar nuevamente, pero esta vez se aproximó por la otra borda lanzándose al abordaje contra el castillo de proa, donde unos sesenta piratas que consiguieron saltar se precipitaron con hachas y cuchillos a cortar los mástiles y aparejos. Los recibimos con fusilería, atacándolos también con bayonetas y granadas de mano, hasta conseguir despejar por dos veces la cubierta. Pero resumiendo esta triste parte de mi relato, después que nuestro barco quedó desmantelado, con tres marineros muertos y ocho heridos, no tuvimos otro remedio que rendirnos y los piratas nos condujeron prisioneros a Sallee, puerto que pertenecía a los moros.





    2. CAUTIVERIO Y EVASIÓN




    El trato que me dieron en Sallee no resultó tan duro como yo había esperado; ni siquiera me llevaron al interior del país con destino a la corte del emperador como les ocurrió a mis compañeros, sino que el capitán pirata me conservó como su parte en el botín, considerándome un esclavo joven y listo y por lo tanto apropiado para esa clase de andanzas.
    Mi nuevo amo me había conducido a su casa, donde yo vivía en la esperanza de que me llevara consigo cuando volviera a embarcarse, confiando que el destino lo hiciera caer tarde o temprano prisionero de algún marino español o portugués y eso me valiera la libertad. Pronto, sin embargo, tuve que abandonar mi esperanza, porque cuando el pirata se embarcó me puso al cuidado del jardín y a cargo del resto de las tareas que son propias de los esclavos; y cuando volvió de su viaje me hizo subir a bordo para que me quedara vigilando el barco. Yo no hacía más que pensar en mi fuga y la manera de llevarla a cabo, pero no se me presentaba la más mínima ocasión y para mayor desgracia no tenía a nadie a quien participar mis intenciones y convencer de que se embarcara conmigo. Así pasaron dos años, en los que mi imaginación no descansó un momento, pero en los cuales jamás tuve oportunidad de utilizar mis ideas.
    Pasados los dos años se presentó una ocasión bastante curiosa que volvió a animar en mí la esperanza de escaparme. Hacía mucho tiempo que mi amo permanecía en su casa sin alistar el barco para hacerse a la mar, según oí, por falta de dinero; dos veces a la semana, cuando el tiempo estaba bueno, acostumbraba salir de pesca en la pinaza del barco. En aquellas ocasiones me llevaba consigo, así como a un joven morisco, para que remáramos; ambos le placíamos mucho, en especial yo por mi habilidad en la pesca, tanto que terminó por enviarme algunas veces con un moro pariente suyo y el joven morisco a fin de que pescáramos para su mesa.
    Aconteció que estando en la pinaza una mañana de mucha calma, se levantó tan espesa niebla que a media legua de la costa no podíamos verla, y remábamos sin saber en qué dirección; así pasamos todo el día y toda la noche hasta que al despuntar la mañana encontramos que habíamos salido al mar en vez de volver a tierra, de la que nos separaban por lo menos dos leguas. Con gran trabajo pudimos retornar, ya que el viento arreciaba y estuvimos en peligro, pero lo que más molestaba era el hambre.
    Nuestro amo, advertido por la aventura, resolvió ser más precavido en el futuro, y disponiendo de la chalupa del buque inglés que había apresado se decidió a no salir de pesca sin llevar una brújula y algunas provisiones, ordenando al carpintero del barco —que era también un esclavo inglés— que le construyera una pequeña cabina en el centro de la chalupa, como las que tienen las falúas, con bastante espacio atrás para dirigir el timón y halar la vela mayor, y delante para que un marinero o dos pudiesen maniobrar el velamen.
    Con esta chalupa salíamos frecuentemente y mi amo no me dejaba nunca en tierra porque apreciaba mi destreza en la pesca. Ocurrió que habiendo invitado a bordo, con intenciones de paseo o de pesca, a dos o tres moros distinguidos, hizo llevar provisiones en cantidad extraordinaria, ordenando que por la noche se cargara la chalupa con todo lo necesario y mandándome que alistara las tres escopetas que había a bordo con las correspondientes balas y pólvora, ya que les agradaba tanto cazar como pescar.
    Hice todo lo que me había indicado y a la mañana siguiente esperaba con la chalupa perfectamente limpia, su bandera y gallardetes enarbolados y todo lo necesario para recibir a los huéspedes, cuando vino mi amo a decirme que sus amigos habían renunciado al paseo a causa de imprevistos negocios, por lo cual me mandaba que saliera con el moro y el muchacho que eran mis acompañantes habituales a pescar para la cena, ya que aquellos amigos comerían en su casa; agregó que tan pronto hubiera pescado lo bastante me apresurara a llevarlo a la casa, todo lo cual me dispuse a ejecutar.
    Fue entonces cuando mis contenidas ansias de libertad me asaltaron con renovada violencia al darme cuenta de que tendría a mi disposición un pequeño barco, y cuando mi amo se alejó me apresuré a proveerme, no para una partida de pesca sino para un viaje; cierto que no sabía hacia dónde iba a encaminar mi rumbo, pero ni siquiera me detuve a pensarlo; cualquier camino que me llevara lejos de allí era mi camino.
    Mi primera medida fue convencer al moro de que necesitábamos embarcar con nosotros algunas provisiones para no sentir hambre durante la pesca, y aduje que no correspondía que tocáramos los alimentos que el amo había almacenado en la chalupa. A él le pareció bien y pronto vino trayendo un gran canasto de galleta o bizcochos y tres tinajas de agua. Yo sabía dónde guardaba mi amo sus licores, encerrados en una caja que, por el aspecto, era indudablemente de fabricación inglesa, sin duda botín de algún navío apresado; mientras el moro estaba en tierra llevé la caja a bordo como para hacer creer que el amo lo había ordenado así anteriormente. Llevé también un gran pedazo de cera que pesaba más de cincuenta libras, un rollo de bramante, una hachuela, una sierra y un martillo, todo lo cual nos sería muy útil más adelante, especialmente la cera para hacer velas. Equipados con todo lo necesario salimos del puerto a pescar, y los guardianes del castillo que defiende el puerto nos conocían tan bien que no nos molestaron, por lo que seguimos más de una milla fuera hasta encontrar sitio donde arriar las velas y principiar la tarea. El viento soplaba del N-NE, y por tanto no me convenía, mientras que viniendo del sur me hubiera llevado con seguridad a la costa española y a la bahía de Cádiz. Pero mi resolución estaba tomada; soplara de donde soplase yo me fugaría de aquel horrible lugar dejando el resta en manos del destino.
    Estuvimos largo rato sin pescar nada, pues cuando yo sentía picar no alzaba el anzuelo, hasta que dije al moro:
    —Este lugar es malo y si nos quedamos en él nuestro amo no será servido como se merece; tenemos que alejarnos más.
    Sin sospechar nada, el moro asintió y se puso a tender las velas mientras yo piloteaba la chalupa hasta una legua más allá donde nos detuvimos como para pescar; entonces, dejando el timón al muchacho, me fui hasta donde estaba el moro y fingiendo inclinarme para levantar algo a sus espaldas lo tomé de las piernas y lo precipité por la borda al mar. Salió inmediatamente a la superficie porque nadaba como un pez y me suplicó lo dejara subir a bordo asegurándome que iría conmigo a cualquier parte. Nadaba tan rápidamente detrás de la chalupa que pronto la hubiera alcanzado, ya que apenas había viento, de modo que corrí a la cabina y tomando una de las escopetas le apunté diciéndole que no le deseaba ningún mal y que si desistía de subir a bordo no tiraría sobre él.
    —Sabes nadar lo bastante como para llegar a tierra —agregué— y el mar está tranquilo, de modo que vuélvete ahora mismo; si insistes en subir a la chalupa te tiraré a la cabeza, porque estoy dispuesto a recuperar mi libertad.
    Oyendo estas palabras giró en el agua y lo vimos volverse hacia la costa, adonde no dudo habrá llegado fácilmente, pues ya he dicho lo bien que nadaba.
    Hubiera preferido tener al moro a mi lado y tirar por la borda al muchacho, pero no me fiaba de aquél. Cuando se hubo alejado me volví hacia mi compañero, que se llamaba Xury y le dije:
    —Xury, si me eres fiel tendrás una gran recompensa; pero si no te golpeas la cara (es decir, si no juraba por Mahoma y la barba de su padre) tendré que tirarte también al agua.
    El muchacho, sonriendo con inocencia, dijo tales palabras y me hizo tales juramentos de que iría conmigo hasta el fin del mundo, que no me quedó ninguna desconfianza.
    Mientras estuvimos al alcance de la mirada del moro, que seguía nadando, mantuve la chalupa al pairo inclinándola más bien a barlovento para que me creyera encaminado hacia la boca del estrecho. Pero tan pronto como oscureció cambié el rumbo y puse proa al sudeste, ligeramente hacia el este para no perder de vista la costa; con buen viento y el mar en calma navegamos tanto que a las tres de la tarde del día siguiente, cuando calculé la posición, deduje que habíamos recorrido no menos de ciento cincuenta millas al sur de Sallee, mucho más allá de los dominios del emperador de Marruecos y probablemente de todo otro imperio, ya que en la costa no se veía a nadie.
    Pero era tal el miedo que me inspiraban los moros y desconfiaba tanto de caer en sus manos que no quise detenerme para bajar a tierra, ni siquiera anclar, sino que aprovechando el buen viento seguimos navegando por espacio de cinco días; entonces el viento cambió al cuadrante sur y como yo sabía que aquello perjudicaba igualmente a todo buque perseguidor, me aventuré a acercarme a la costa y anclamos en la desembocadura de un riacho tan desconocido como la latitud, el país y los habitantes. Por cierto que prefería no ver a nadie, siendo única razón del desembarco la necesidad de agua dulce. Llegamos por la tarde al riacho, decidiendo nadar de noche hasta la costa y explorar los alrededores, pero así que oscureció empezamos a oír tan horribles rugidos, ladridos y aullidos de los animales salvajes que el pobre Xury se moría de miedo y me rogó que no bajase a tierra hasta que viniera el día.
    Yo estaba tan asustado como el pobre muchacho, pero nuestro espanto creció cuando oímos a uno de aquellos enormes animales que venía nadando hacia la chalupa. No alcanzábamos a verlo, pero comprendíamos por sus resoplidos que debía ser un animal enorme y furioso. Xury sostenía que se trataba de un león —lo que acaso era cierto— y me rogaba que levantáramos anclas y huyéramos.
    —No, Xury —le dije—. Podemos soltar el cable con la boya y dejarnos llevar hacia el mar; los animales no osarán nadar tanta distancia.
    Apenas había dicho esto cuando vi al monstruo (fuera lo que fuese) a dos remos de distancia de la chalupa. Venciendo mi sorpresa tomé una de las escopetas de la cabina y tiré sobre él, viéndolo girar de inmediato en el agua y volverse hacia la costa.
    Seria imposible describir los horribles sonidos, el aullar y rugir que se elevó en la costa y desde muy adentro del país como un eco a mi disparo, ruido que probablemente aquellas bestias oían por vez primera. Aquello me convenció de que sería insensato desembarcar de noche, pero también durante el día. Caer en manos de salvajes era tan desastroso como caer en las garras de tigres y leones; ambas cosas nos parecían igualmente funestas.
    Sea lo que fuese, necesitábamos obtener agua de alguna manera, puesto que no teníamos ni una pinta. Pero ¿cómo? Fue entonces que Xury me rogó que lo dejara desembarcar con una de las tinajas para buscar y traerme agua. Le pregunté por qué quería ir él en vez de quedarse esperándome en la chalupa. La respuesta del muchacho me hizo quererlo profundamente desde ese momento.
    —Si hombres salvajes venir —dijo— ellos comerme a mí, vos salvaros.
    —Muy bien, Xury —le contesté—, entonces iremos los dos y si vienen los salvajes los mataremos para que no nos coman.
    Le di un pedazo de galleta y un trago del licor que saqué de la caja ya mencionada, y tras de acercar la chalupa todo lo posible a la costa desembarcamos sin otra defensa que nuestros brazos y dos tinajas para el agua.
    No me atrevía a perder de vista la chalupa por miedo a que los salvajes salieran del río en canoas y la abordaran; entretanto el muchacho había visto un terreno bajo a una milla aproximadamente y corrido hacia él, hasta que de improviso lo vi volver a toda carrera. Pensé que algún salvaje lo perseguía o que había tenido miedo de las fieras, por lo que fui en su ayuda, pero cuando estuvo más cerca vi que traía algo colgando del hombro, un animal que acababa de cazar parecido a una liebre, pero de patas más largas y distinto color. Nos alegramos mucho y su carne nos pareció excelente, aunque la mayor alegría de Xury fue hacerme saber que había encontrado agua potable y ningún salvaje en los alrededores.
    Yo había navegado por aquellas costas y sabía que las islas Canarias así como las de Cabo Verde no podían estar muy distantes. Me faltaban sin embargo instrumentos para calcular la latitud; no recordaba con precisión la de las islas, de manera que no sabía si continuar en una u otra dirección para encontrarlas; salvo esto, hubiera sido muy simple tocar tierra en ellas. Mi esperanza estaba en seguir la línea de la costa hasta las regiones donde comercian los ingleses, y dar con alguno de sus barcos mercantes que nos rescatara de nuestras desdichas.
    Una o dos veces me pareció ver el Pico de Tenerife, la cresta culminante de las montañas de Tenerife en las Canarias, y por dos veces intenté llegar a las islas, pero los vientos contrarios me lo impidieron, así como un mar demasiado agitado para nuestro barquichuelo; entonces me resigné a proseguir el viaje sin perder de vista la costa.
    Muchas veces nos vimos obligados a desembarcar en procura de agua dulce, y recuerdo una ocasión en que anclamos muy temprano al pie de un promontorio bastante alto, esperando que la marea nos llevara aún más adentro. Xury, que tenía mejor vista que yo, me llamó de pronto para decirme que haríamos mejor en levar anclas cuanto antes.
    —Mirad allá —agregó— ese horrible monstruo que duerme en la ladera de la colina.
    Seguí la dirección que me apuntaba y vi ciertamente al monstruo: un enorme león tendido sobre la playa y protegiéndose del sol por una proyección rocosa de la colina.
    —Xury —dije al muchacho—, irás a la tierra y lo matarás.
    Me miró aterrado.
    — ¿Yo matarlo? ¡El comerme de un boca! —exclamó, queriendo significar un bocado.
    No le dije más nada, pero indicándole que se quedara quieto tomé la escopeta más grande, cuyo calibre era casi el de un mosquete, y la cargué con suficiente pólvora y dos pedazos de plomo; metiendo dos balas en otra escopeta, puse en la tercera cinco plomos pequeños. Apunté lo mejor posible con la primera arma, buscando darle en la cabeza, pero como dormía con una pata tapándole parcialmente la nariz los plomos le alcanzaron la rodilla y le rompieron el hueso. Se levantó gruñendo, pero al sentir la pata rota volvió a caer para enderezarse luego sobre tres patas y exhalar el más horroroso rugido que haya escuchado en mi vida. Me sorprendía no haberle acertado en la cabeza, por lo cual le apunté con la segunda escopeta y, aunque se movía de un lado a otro, tuve el placer de verlo desplomarse ya sin rugir, pero todavía luchando en su agonía. Xury, que había recobrado los ánimos, me pidió que lo dejara desembarcar y cuando se lo consentí saltó al agua, con una escopeta en la mano y nadando con la otra hasta llegar junto al león, y apoyándole el caño en la oreja le disparó el tiro de gracia.
    Todo ello nos había divertido un buen rato, pero sin darnos alimentos, tanto que empecé a lamentar haber desperdiciado aquella pólvora y balas en un animal que de nada nos servia. Xury quería conservar algo de él y cuando vino a bordo me pidió permiso para llevar el hacha a tierra.
    — ¿Para qué la quieres, Xury? —pregunté.
    —Yo cortarle cabeza —me contestó. Pero aunque hizo lo posible no pudo cortársela y se conformó con una pata que trajo a bordo y que era monstruosamente grande. Entonces se me ocurrió que la piel del león podía sernos de alguna utilidad y resolvimos desollarlo. Xury fue mucho más hábil que yo en esta tarea que me resultaba muy difícil. Trabajamos el día entero, pero al fin le sacamos la piel y la pusimos sobre el techo de la cabina, donde el sol la secó en un par de días, tras de lo cual me sirvió para dormir sobre ella.
    Nuevamente embarcados, seguimos hacia el sur sin interrupción durante diez o doce días, tratando de ahorrar las provisiones que disminuían rápidamente y bajando a tierra sólo cuando la sed nos obligaba. Mi intención era llegar hasta el río Gambia o Senegal —es decir, a la altura de Cabo Verde— donde confiaba encontrar algún barco europeo; de no tener esa suerte ignoraba qué iba a ser de mí, ya fuera buscando las islas o pereciendo a mano de los negros. Sabia que todos los barcos que navegan de Europa a la costa de Guinea, Brasil o las Indias Orientales, tocan en el Cabo o en aquellas islas; en una palabra, depositaba mi entera suerte en el hecho de encontrar un barco y de lo contrario sólo podía esperar la muerte.
    Mientras trataba de poner en práctica esa intención, y en el transcurso de aquellos diez días, empecé a notar que la tierra estaba habitada; en dos o tres lugares vimos en las playas gentes que nos miraban pasar, advertimos que eran negros y que estaban completamente desnudos. Me inclinaba yo a trabar relación con ellos, pero Xury era mi mejor consejero y repetía:
    —No, no ir, no ir.
    Acerqué sin embargo la chalupa a distancia suficiente para hablar, pero los negros echaron en seguida a correr por la playa. Noté que no llevaban armas, salvo uno que tenía una especie de largo bastón que Xury dijo ser una lanza que aquellos salvajes arrojan con gran puntería y a larga distancia. Me mantuve, pues, alejado, pero traté de entenderme con ellos por signos haciendo aquellas señales que se refieren al acto de comer. Me contestaron a su modo que anclara la chalupa y que me darían alimentos, y mientras yo arriaba la vela y quedaba a la espera, dos de ellos fueron tierra adentro, de donde regresaron a la media hora trayendo consigo dos grandes pedazos de carne seca y grano como el que produce su país.
    Aunque no teníamos idea de lo que podían ser tales alimentos los aceptamos de inmediato, pero el problema estaba en cómo recibirlos, pues ni yo me animaba a desembarcar ni ellos a llegarse hasta la chalupa; pronto vi, sin embargo, que habían encontrado un procedimiento satisfactorio para ambos, ya que dejaron la carne y los granos en la playa, se alejaron a gran distancia y me dieron tiempo de ir a buscarlos, tras lo cual volvieron a acercarse.
    Teníamos, pues, provisiones y agua, y separándonos de aquellos cordiales negros seguimos navegando otros once días aproximadamente sin volver a arrimar a la costa, hasta que un día vi una tierra que penetraba profundamente en el mar a una distancia de cuatro o cinco leguas de donde estábamos; como el día era sereno, dimos una gran bordada para llegar a ella, y por fin, cuando doblamos la punta a unas dos leguas de la costa, distinguimos con toda claridad tierras al otro lado, mirando hacia el mar. Supuse que la tierra más próxima era Cabo Verde y la otra las islas que llevan su mismo nombre. Desgraciadamente estaban a una enorme distancia y no me decidía a lanzarme en su dirección por miedo a que una borrasca me sorprendiera a mitad de camino y sin poder llegar a una ni otra.
    En este dilema me fui a la cabina a pensarlo mejor, dejando a Xury en el timón, cuando repentinamente le oí gritar:
    — ¡Señor, señor, un barco con vela!
    El pobre muchacho estaba mortalmente asustado, temiendo que se tratara de algún navío enviado por el moro para perseguirnos y sin pensar que ya estábamos demasiado lejos de su alcance. Salté de la cabina y conocí de inmediato que el barco era portugués y que se dirigía sin duda a Guinea en procura de negros. Con todo, observando la ruta que seguía, me convencí de que el barco iba a otra parte y no mostraba intenciones de acercarse a tierra, por lo que saqué la chalupa mar afuera, resuelto a hablar con aquellos marinos si estaba a mi alcance.
    Soltando todo el trapo que teníamos, vine a descubrir que no sólo era imposible acercarnos al navío sino que éste se alejaría antes de que me fuera posible hacerle señal alguna; pero mientras yo, después de haber intentado todo lo imaginable, empezaba a desesperar, parece que ellos alcanzaron a ver la chalupa con ayuda de su anteojo descubriendo que se trataba de un bote europeo, por lo cual imaginaron que un barco había naufragado y se apresuraron a arriar velamen para que yo pudiera ganar terreno. Esto me llenó de alegría, y como conservaba la bandera de mi antiguo amo la enarbolé en señal de socorro y disparé un tiro de escopeta, cosas ambas que observaron desde el barco, pues más tarde me dijeron que habían visto el humo aunque no les llegó el ruido del disparo. Tales señales los determinaron a detener el barco y esperarme; tres horas después subía yo a bordo.
    Me hicieron muchas preguntas que no entendí, hablándome en portugués, español y francés, hasta que un marinero natural de Escocia se dirigió a mí y pude explicarle que era inglés y cómo me había fugado de los moros en Sallee, siendo de inmediato muy bien recibido a bordo con todos mis efectos.
    Es fácil de comprender la inmensa alegría que tuve al considerarme librado de tan desdichada situación; de inmediato ofrecí cuanto tenía al capitán como compensación por mi rescate, pero él no quiso aceptar nada y me dijo generosamente que todo lo mío me sería devuelto cuando llegásemos al Brasil.
    —Al salvar vuestra vida —me aseguró— he procedido tal como quisiera ser tratado yo mismo si alguna vez me encontrara en las mismas circunstancias. Además si os llevara a un lugar tan lejano de vuestra patria y os privara de lo que es vuestro, seria como condenaros a perecer de hambre y quitaros así la misma vida que acabo de salvar. No, no, señor inglés, os llevaré allá sin recibir nada, y lo que poseéis os servirá para vivir en el Brasil y pagar el pasaje de retorno.
    Pronto comprendí que sus actos se ajustaban celosamente a sus promesas; ordenó a los marineros que nadie tocara lo mío, lo puso bajo su propia responsabilidad y mandó hacer un inventario que me entregó, donde se incluían hasta las tres tinajas de barro.
    Cuando vio mi chalupa, que era excelente, quiso comprármela para incorporarla a su barco y me preguntó en cuánto estimaba yo su valor. Le contesté que había sido tan generoso conmigo que no me correspondía fijar el precio sino que lo dejaba en sus manos. Me propuso entonces librarme una letra pagadera en el Brasil por valor de ochenta piezas de a ocho, y que si al llegar allí alguien ofrecía más por la chalupa él compensaría la diferencia. Me ofreció también sesenta piezas de a ocho por Xury, pero me desagradaba recibirlas, no porque me preocupara la suerte del muchacho junto al capitán sino porque me dolía vender la libertad de quien tan fielmente me ayudara a lograr la mía. Cuando dije esto al capitán me contestó que era muy justo, pero que para tranquilizarme se comprometía a firmar una obligación por la cual Xury sería libre al cabo de diez años siempre que se hiciera cristiano. Satisfecho con esto, y más cuando el mismo Xury me manifestó su conformidad, se lo cedí.
    Tuvimos buen viaje al Brasil y a los veintidós días llegamos a la bahía de Todos los Santos. Nuevamente me había salvado de la más miserable situación en que puede verse un hombre, y otra vez debía enfrentar el problema de mi futuro destino.





    3. LA PLANTACIÓN. EL NAUFRAGIO




    Nunca estaré bastante agradecido al generoso comportamiento del capitán. Sin aceptar nada por mi pasaje, me dio veinte ducados por una piel de leopardo y cuarenta por la de león, ordenando que todo cuanto tenía yo a bordo me fuera entregado al detalle; me compró aquellas cosas que yo quería vender, como la caja de licores, dos escopetas y lo que quedaba del pedazo de cera con el cual había fabricado muchas velas. En resumen, me encontré en posesión de unas doscientas veinte piezas de a ocho y con esta suma desembarqué en el Brasil.
    No llevaba mucho tiempo allí cuando fui recomendado por el capitán a un hombre de su misma honestidad que poseía un «ingenio», como llaman ellos a una plantación y fábrica de azúcar. Allí viví cierto tiempo, en cuyo transcurso aprendí a plantar y obtener el azúcar, y reparando en la agradable vida que llevaban los colonos y con cuánta facilidad se enriquecían resolví que si obtenía permiso para radicarme entre ellos me dedicaría a las plantaciones, tratando entretanto de recobrar los fondos que había dejado en Londres. Con este fin solicité y obtuve una especie de carta de naturalización y gasté el dinero que poseía en comprar tierra inculta, trazando los planes para una plantación y establecimiento de acuerdo con la cantidad que esperaba recibir de Inglaterra.
    Era mi vecino un portugués de Lisboa, hijo de padres ingleses y de apellido Wells. Como se encontraba en condiciones semejantes a las mías y su plantación era lindera, yo le llamaba vecino y llegamos a ser buenos amigos. Ambos teníamos poco capital y plantábamos para comer, más que para otra cosa; pero poco a poco empezamos a progresar, y nuestras tierras a rendir provecho. El tercer año plantamos tabaco, y a la vez despejamos un gran pedazo de tierra para plantar caña de azúcar al año siguiente. Nos faltaban brazos que nos ayudaran, y fue entonces cuando advertí el error cometido al separarme de Xury.
    Me encontraba ya avanzado en la tarea de mejorar la plantación cuando mi salvador y buen amigo el capitán decidió hacerse a la vela, pues su barco había permanecido tres meses completando el cargamento y alistándose. Cuando le conté lo del pequeño capital que tenía en Londres, me dio este amistoso y sincero consejo:
    —Señor inglés —porque siempre me llamaba así—, si me libráis cartas y poder en debida forma, con orden a la persona que tiene vuestro dinero en Londres para que lo transfiera a quien yo designe, en Lisboa, os lo traeré si Dios quiere a mi regreso en diversos artículos que tengan fácil salida en este país. Como todo lo humano está sujeto a desastres y cambios, os aconsejo que sólo libréis órdenes por cien libras esterlinas, que según me decís es la mitad de vuestro capital; si las cosas resultan bien podréis rescatar el resto en la misma forma, y de lo contrario os quedará siempre esa reserva.
    El consejo era tan sano y amistoso que comprendí que debía seguirlo, de manera que inmediatamente escribí cartas a la dama depositaría de mis fondos y entregué un poder al capitán. Conté a la viuda del capitán inglés todas mis aventuras, la esclavitud, mi fuga y cómo había conocido al capitán portugués; le narré su generoso comportamiento y en qué circunstancias me encontraba en ese momento, agregando las instrucciones necesarias para la transferencia de los fondos. Cuando el capitán llegó a Lisboa hizo que alguno de los comerciantes ingleses allí establecidos enviaran a Londres la orden y además el entero relato de lo que me había ocurrido, de tal modo que la viuda no solamente entregó sin vacilar el dinero sino que de su propio bolsillo envió un presente al capitán portugués, como homenaje a su generoso y humano proceder.
    El corresponsal en Londres invirtió mis cien libras esterlinas en artículos ingleses tal como el capitán se lo había mandado, y los remitió a Lisboa, de donde mi amigo los trajo felizmente al Brasil. Entre aquellas mercancías, y sin que yo las hubiera pedido, pues era aún demasiado inexperto en la plantación para pensar en ello, venían, por encargo del capitán, herramientas, instrumentos y utensilios necesarios para el trabajo, que me fueron de gran utilidad.
    Cuando llegó el cargamento creí que mi fortuna estaba hecha, tanto me maravilló aquello. Mi servicial amigo el capitán había empleado las cinco libras que le regalara la viuda en contratar por seis años un criado que él mismo me trajo, y no quiso aceptar la menor retribución salvo una pequeña cantidad de tabaco que, por ser de mi plantación, logré al fin que aceptara.
    No todo concluyó allí: las mercancías inglesas tales como paños, tejidos y bayetas eran sumamente solicitadas en el país, de modo que pronto las vendí con tal ganancia que puede asegurarse que cuadripliqué el valor de mi primer cargamento, dejando pronto atrás a mi pobre vecino en el progreso de la plantación; lo primero que hice fue comprar un esclavo negro y obtener los servicios de otro criado europeo, fuera del que el capitán me había traído de Lisboa.
    ¡Cuántas veces la excesiva prosperidad es el más seguro medio de precipitarnos en la mayor desgracia! Así ocurrió conmigo. Al año siguiente la plantación me dio gran cosecha y recogí cincuenta fardos de tabaco fuera de la cantidad destinada a cambiar a los vecinos por otros productos. Cada rollo pesaba más de cien libras, y luego de prepararlos convenientemente los dejé en depósito hasta que volviera el convoy de Lisboa. Entretanto, próspero en negocios y riqueza, empecé a dejarme llevar por proyectos y ambiciones superiores a mis medios, fantasías que terminan por ser la ruina de los comerciantes más expertos.
    Podéis imaginar que llevando casi cuatro años en el Brasil y dirigiendo una floreciente plantación, no sólo había aprendido el idioma sino que sostenía relaciones con los demás plantadores y los comerciantes de San Salvador, que era nuestro puerto. En diversas ocasiones les había narrado mis dos viajes a la costa de Guinea, la forma de comerciar con los negros y qué fácil es conseguir no solamente oro en polvo, granos, colmillos de elefante, sino también negros para el servicio de las plantaciones, a cambio de insignificancias como cuentas de vidrio, cuchillos, tijeras, hachuelas, pedazos de cristal y otras chucherías.
    Escuchaban mis narraciones con gran atención, que se acrecentaba más cuando mencionaba yo la forma de comprar negros, ya que en aquel entonces la trata de esclavos no sólo estaba muy restringida sino que existía un monopolio a cargo de los «asientos» o permisos de los reyes de España y Portugal, lo cual hacía que los negros fueran escasos y exageradamente caros.
    Ocurrió que después de haber estado en compañía de algunos comerciantes y plantadores de mi relación hablando de aquellas cosas con mucho detenimiento, vinieron a verme a la mañana siguiente tres de ellos para decirme que habían reflexionado mucho sobre lo que yo les contara y qué tenían una proposición que hacerme, siempre que les guardara el secreto. Me confiaron que estaban dispuestos a fletar un buque a Guinea, ya que teniendo plantaciones al igual que yo, nada les preocupaba tanto como la falta de brazos, pero que como no podían procurárselos, ya que estaba prohibida la venta pública de esclavos negros, intentaban realizar un viaje secreto en busca de ellos, traerlos a tierra sin despertar sospechas y repartirlos entre sus plantaciones. En una palabra, se trataba de saber si aceptaría ir como encomendero en el barco para dirigir la compra de negros, a cambio de lo cual me ofrecían igual participación que la de ellos en el reparto de los esclavos sin contribuir en nada a los gastos del flete.
    Preciso es confesar que aquélla hubiera sido una excelente proposición para cualquiera que no estuviese ya radicado con una plantación próspera a cuidar, crecientes ganancias y un buen capital. Para mí, ya establecido y sin otra tarea que continuar tres o cuatro años más lo que había iniciado, agregando a ellos las cien libras que debían enviarme de Londres; para mí, que en ese momento y con aquella adición poseía no menos de tres o cuatro mil libras esterlinas, en camino a aumentar todavía, el solo pensar en aquel viaje representaba las más descabellada idea que un hombre en tales circunstancias pudiera concebir.
    Pero yo había nacido para ser causa de mi propia desgracia y no pude resistir aquella oferta, como no había logrado impedir mis primeros planes aventureros a pesar de los consejos de mi padre. Les dije que partiría sin vacilar siempre que se encargasen de velar por mi plantación mientras durara mi ausencia y cumplieran mi voluntad en todo si me ocurría una desgracia. Se comprometieron formalmente y lo rubricaron por escrito, tras lo cual hice testamento declarando mi legatario universal al capitán portugués que me había salvado la vida, y dejándole la mitad de mis bienes con la condición expresa de que enviaría la otra mitad a Inglaterra.
    En resumen, tomé todas las medidas para salvaguardar mis propiedades y la plantación. Si hubiera sido capaz de emplear sólo la mitad de esa prudencia en velar por mis verdaderos intereses y meditar serenamente lo que debía o no debía hacer, jamás habría renunciado a una situación tan próspera dejando todo al azar de las circunstancias y lanzándome a un viaje por mar con lo mucho que tiene de azaroso, para no mencionar las razones que yo tenía para prever especiales catástrofes. Pero cediendo a mis impulsos obedecí ciegamente los dictados del capricho y no los de la razón. Cuando estuvo alistado el barco, el cargamento a bordo y todo perfectamente dispuesto por mis socios, me embarqué en un día aciago, el primero de septiembre de 1659, justamente al cumplirse el octavo aniversario de mi abandono del hogar de mi padre en Hull, cuando me rebelé a su autoridad para hacer el tonto a mis expensas.
    Nuestro barco era de unas ciento veinte toneladas, tenía seis cañones y catorce hombres fuera del capitán, su asistente y yo. No llevábamos gran cargamento fuera de las baratijas necesarias para el intercambio con los negros, tales como cuentas de vidrio, trozos de cristal, conchas y diversas chucherías, en especial pequeños espejos, cuchillos, tijeras y hachuelas.
    Nos hicimos a la vela el mismo día en que embarqué, costeando hacia el norte para luego rumbear al África cuando estuviéramos a los diez o doce grados de latitud norte, que era el camino seguido en aquellos tiempos. A los doce días cruzamos la línea y nos encontrábamos, según la última observación que alcanzamos a hacer, a unos siete grados veintidós minutos norte cuando un violento tornado o huracán nos privó completamente de referencias. Empezó a soplar del sudeste, luego del noroeste, hasta fijarse en el cuadrante noreste, de donde nos azotó con tal furia que por espacio de doce días no pudimos hacer otra cosa que dejarnos llevar a la deriva y, arrastrados por su violencia, ser impulsados hacia donde el destino y la fuerza del viento lo quisieran. Sería ocioso decir que en aquellos momentos cada uno de nosotros esperaba ser devorado por el mar, y que nadie guardaba la menor esperanza de salvar su vida.
    Fuera de la furia de la borrasca, tuvimos la desgracia de que uno de los hombres muriera de calenturas y que otro, juntamente con el muchacho asistente, fuera arrebatado por el mar. Hacia el duodécimo día el tiempo mejoró un poco y el capitán pudo hacer una precaria observación, según la cual nos encontrábamos sobre la costa de Guinea o bien sobre la del norte de Brasil, más allá de las bocas del Amazonas y cerca del Orinoco, llamado también Río Grande. Consultó conmigo qué camino deberíamos tomar, puesto que el buque estaba averiado y navegaba difícilmente, por lo cual creía conveniente ganar lo antes posible la costa del Brasil.
    Me negué de plano a esta sugestión, y mirando juntos los mapas de la costa americana descubrimos que no existía región habitada donde pudiéramos hallar socorro hasta entrar en el círculo de las islas Caribes, y por lo tanto pusimos proa hacia las Barbados para alcanzarlas desde alta mar y evitarnos así la entrada de la bahía o Golfo de México; confiábamos en llegar a ellas en unos quince días, ya que de ninguna manera podíamos proseguir viaje a la costa africana sin las reparaciones que el barco necesitaba.
    Decidido esto cambiamos el rumbo y tomamos el de O-NO, tratando de alcanzar alguna de las islas inglesas donde nos auxiliarían; pero nuestro viaje estaba predestinado a ser distinto, pues una segunda tormenta cayó sobre nosotros arrastrándonos hacia el oeste y tan lejos de toda ruta comercial que aun logrando salvarnos de la furia del océano estábamos más próximos a ser devorados por salvajes que volver alguna vez a nuestro país.
    Mientras padecíamos angustiados la furia de los vientos, oímos de mañana gritar « ¡Tierra!» a uno de los marineros. No habíamos acabado de salir de las cabinas para tratar de distinguir a qué regiones habíamos arribado cuando el barco encalló en las arenas y de inmediato el oleaje empezó a azotarlo con tal furia que tuvimos la impresión de que pereceríamos allí mismo y nos refugiamos en los camarotes para guarecernos del agua y las espumas.
    No es fácil para uno que jamás se ha visto en tal situación concebir la angustia que sentíamos en esas circunstancias. Ignorábamos dónde habíamos encallado, si era el continente o una isla, si habitada o desierta; y como el viento seguía azotando, bien que con menos fuerza que al comienzo, no nos cabía duda de que el barco iba a destrozarse en contados minutos a menos que un milagro calmara la tempestad. Nos mirábamos unos a otros esperando la muerte a cada instante, y tratábamos de prepararnos para la otra vida, ya que comprendíamos que poco nos quedaba por hacer en ésta. Algo nos consolaba que el navío hubiera resistido hasta ese instante, y el capitán sostenía que el viento estaba amainando un poco; pero aunque fuera así, el buque encallaba profundamente en las arenas y parecía demasiado hundido para pensar en sacarlo de su posición, de manera que seguíamos en terrible peligro y sólo nos quedaba tratar de salvar la vida de cualquier manera. Había un bote en la popa antes de que estallara la borrasca, pero se destrozó al chocar incesantemente contra el timón y luego de partirse fue arrebatado por las olas, de manera que no contábamos con él; quedaba otro bote a bordo, ¿pero podríamos echarlo al agua? Sin embargo, no había nada que discutir, pues estábamos seguros de que el barco iba a partirse en pedazos de un momento a otro, y ya algunos aseguraban que estaba destrozado.
    En esta confusión, el piloto se decidió a asegurar el bote y con ayuda de la tripulación consiguió hacerlo pasar sobre la borda; inmediatamente embarcamos, once en total, y nos confiamos a la merced de Dios en aquel mar embravecido, que, aunque había amainado el viento, seguía encrespándose horrorosamente.
    Al punto comprendimos que estábamos perdidos; el oleaje era tan alto que el bote no podía resistirlo y no pasaría mucho antes de ahogarnos. Otra vez confiamos nuestras almas a la Providencia, y como el viento nos arrastraba hacia la costa apresuramos nuestra destrucción remando con toda la fuerza posible hacia tierra.
    ¿Cómo era la costa? ¿Rocosa o arenosa, abrupta o de suave pendiente? No lo sabíamos; nuestra única sombra de esperanza era la de ir a parar a un golfo o bahía, quizá las bocas de un río donde nuestro bote, a cubierto por el sotavento de la tierra, encontrara aguas tranquilas. Pero nada de esto parecía probable y mientras nos acercábamos a la costa la encontrábamos aún más espantosa que el mismo mar.
    Después de remar, o mejor, de dejarnos llevar, aproximadamente una legua y media, una gigantesca ola, como rugiente montaña líquida, se precipitó súbitamente sobre nosotros, dándonos la impresión de que era el «coup de gráce». Nos cayó con tal violencia que el bote se dio vuelta en un instante, y separándonos de él como de nosotros mismos, sin darnos tiempo a decir: « ¡Mi Dios!»,'nos engulló a todos.
    No podría describir el estado de ánimo que tenía cuando me sentí hundir en las aguas, porque aunque sabía nadar muy bien no conseguía librarme de la fuerza de las olas y ascender a respirar, hasta que después de arrastrarme interminablemente en dirección a la playa, la ola rompió allí y al retroceder me dejó en tierra firme, medio muerto por el agua que había tragado. Me quedaban suficiente aliento y presencia de ánimo como para advertir que estaba más cerca de la playa de lo que había supuesto, y enderezándome traté de correr hacia ella con toda la velocidad posible antes de que otra ola me arrebatara. Pero de inmediato supe que aquello era imposible porque vi crecer el mar a mis espaldas como una montaña y con la furia de un enemigo que me superaba infinitamente en fuerzas. Mi salvación estaba en retener el aliento y sostenerme a flote todo lo posible, tratando en esa forma de nadar hacia la playa; pero me aterraba pensar que acaso el oleaje, después de sumirme profundamente en el mar, no me devolvería a la costa en su retorno.
    La ola que me cayó encima me hundió veinte o treinta pies en su seno, y otra vez me sentí arrastrado con una salvaje violencia y velocidad hacia la tierra, pero contuve la respiración y traté de nadar hacia adelante con todas mis fuerzas. Me parecía que iba a estallar por falta de aire, cuando me sentí levantado y de pronto tuve la cabeza y las manos fuera del agua; aunque esto solamente duró un segundo, me permitió recobrar el aliento y nuevo valor. Otra vez me tapó el agua, pero no tanto como para hacerme perder las energías, y cuando advertí que estaba en la playa y que la ola iba a volver, luché por sostenerme hacia adelante y toqué tierra con los pies. Me estuve quieto un momento para recobrar la respiración y mientras el agua se retiraba eché a correr con toda la velocidad posible hacia la costa. Pero ni esto me libró de la furia del mar y por dos veces consecutivas volví a ser arrebatado y devuelto otra vez a la playa, que era sumamente suave.
    La segunda vez estuvo a punto de serme fatal porque el oleaje, después de llevarme mar adentro, me proyectó con violencia contra una roca y tal fue la fuerza del golpe que me privó de los sentidos, dejándome indefenso contra su furia. El golpe me había magullado el pecho y el costado, privándome por completo de la respiración; estoy seguro de que si el mar hubiera vuelto inmediatamente habría perecido ahogado. Pero recuperé los sentidos un momento antes del retorno de la ola, y viendo que otra vez iba a ser arrastrado por ella me aferré con todas mis fuerzas a la roca, luchando por contener el aliento hasta que el agua retrocediera. Las olas ya no eran tan altas como antes, por la proximidad de la costa, y pude por lo tanto resistir el embate hasta que cesó, y entonces eché a correr hacia tierra con tal fortuna que la siguiente ola, aunque me alcanzó, ya no pudo arrancarme de donde estaba y en una segunda carrera me libré totalmente de su rabia, encaramándome sobre los acantilados hasta desplomarme sobre la hierba, libre de todo peligro y a salvo del mar.
    Cuando comprendí con claridad el riesgo del que acababa de salvarme, elevé mis ojos a Dios y le agradecí que hubiera perdonado una vida que segundos antes no conservaba la menor esperanza. Me paseaba por la playa alzando no sólo las manos sino todo mi ser en acción de gracias por mi rescate, haciendo mil ademanes que no podría describir y reflexionando sobre mis camaradas que se habían ahogado, siendo yo el único que había conseguido pisar tierra; nunca volví a verlos, ni siquiera encontré señales de ellos, salvo tres sombreros, una gorra y dos zapatos de distinto par.
    Fijé los ojos en el barco encallado, al que la distancia y la furia del mar apenas me permitían divisar, y me maravillé.
    — ¡Oh, Señor! —prorrumpí—. ¿Cómo he podido llegar a tierra?
    Después de alegrar mi espíritu con el lado feliz de mi aventura, empecé a reconocer el lugar en torno mío para averiguar qué clase de sitio era y cuáles medidas debía tomar. Mas pronto cesó mi contento al comprender que de nada me servía la salvación. Estaba empapado, sin ropa que cambiarme y nada para comer y beber; la perspectiva más probable era la de morir de hambre o ser devorado por animales feroces. Lo que más me afligía era no tener armas con que matar un animal para alimentarme o como defensa contra cualquier bestia que quisiera hacerlo a costa mía. En una palabra, sólo tenía un cuchillo, una pipa y un poco de tabaco en una cajita. Al comprender la miseria en que me encontraba sentí crecer en mí tal desesperación que eché a correr como un loco. La noche se acercaba y en mi angustia me pregunté si en aquel país habría bestias salvajes, sabiendo de sobra que aquellas eligen las tinieblas para acechar sus presas. Todo lo que se me ocurrió fue treparme a un frondoso árbol, especie de abeto pero con espinas, y allí me propuse estarme la noche entera y decidir, a la mañana siguiente, cuál sería mi muerte; porque ya no veía esperanza alguna de seguir viviendo.
    Anduve primero en busca de agua dulce, que con gran alegría encontré a un octavo de milla aproximadamente; después de beber y mascar un poco de tabaco para adormecer el hambre, trepé a mi árbol, tratando de hallar una posición de la cual no me cayera si el sueño me vencía. Había cortado un sólido garrote para defenderme, y era tal mi extenuación que pronto quedé dormido con un sueño profundo y tranquilo como no creo que nadie haya podido disfrutar en semejantes circunstancias.





    4. LA ISLA DESIERTA




    Era pleno día cuando desperté; el tiempo estaba despejado y sin huellas del temporal, por lo que el mar aparecía muy tranquilo. Lo que más me sorprendió fue advertir que la marea había zafado el barco de las arenas donde encallara y traído hasta junto a la roca donde por poco me matan las olas al golpearme contra ella. Apenas una milla me separaba del barco, y notando que éste se mantenía a flote se me ocurrió ir a bordo en procura de aquellas cosas que me fueran necesarias.
    Bajando del árbol, dirigí la vista en torno y no tardé en descubrir el bote que el viento y las olas habían arrojado a las arenas dos millas a mi derecha. Fui hacia él para asegurarlo, pero encontré un brazo de mar ancho de media milla entre el bote y yo, y volviéndome por el mismo camino busqué acercarme al barco, donde esperaba encontrar alimentos.
    Poco después de mediodía el mar se puso como un espejo y la marea bajó tanto que pude acercarme a un cuarto de milla del barco; ya entonces sentía renovarse mi desesperación al comprender que si nos hubiéramos quedado a bordo todos estaríamos a salvo y en tierra, sin verme yo reducido a una absoluta soledad, huérfano de socorro y alivio. Derramé nuevamente lágrimas, pero como de nada me servían resolví si era posible llegar al barco. Hacía mucho calor, por lo cual me quité parte de la ropa antes de tirarme al agua, y nadando hasta el buque empecé a buscar un modo de trepar a cubierta. La dificultad estaba en que el buque se mantenía derecho, sin punto alguno de apoyo para intentar escalarlo. Nadé dos veces en torno a él, y a la segunda advertí un cabo de cuerda que colgaba de los portaobenques de mesana. Asombrado de no haber reparado antes en ella, así su extremo después de muchos esfuerzos y me encaramé al castillo de proa. El barco tenía una vía de agua y estaba parcialmente inundado; encallado en un banco de arena muy dura —o más bien de tierra—, la popa se levantaba sobre aquél mientras la proa casi tocaba el agua. Era de alegrarse que toda la popa estuviera sobre el nivel del banco, ya que cuanto contenía se encontraba intacto, cosa que de inmediato me apresuré a verificar. Las provisiones de a bordo no habían sufrido absolutamente nada, y de inmediato pude satisfacer mi gran apetito llenándome los bolsillos de galleta y comiendo a la vez que revisaba el resto del barco para no perder tiempo. Hallé un poco de ron en la cabina del capitán, y bebí un buen trago para fortalecerme ante la tarea que me esperaba. Ahora solamente me hacía falta un bote para llenarlo con todo aquello que presentía iba a serme de gran necesidad.
    Era inútil sentarse a esperar lo imposible, y la dificultad aguzó mi ingenio. Había a bordo muchas verjas sueltas, dos o tres perchas o berlingas y uno o dos masteleros de juanete. Me resolví a emplearlos y levantándolos por la borda los arrojé al agua no sin antes atarlos con sogas para que el mar no los llevase lejos. Hecho esto me descolgué por el costado del buque y atrayendo los palos cerca de mí empecé a atar juntamente cuatro de ellos, sujetándolos por ambos extremos para formar una especie de balsa; cruzando los palos menores para reforzarla comprobé que me sostenía muy bien sobre el agua pero que no sería capaz de soportar un gran peso por la fragilidad de la madera. Subiendo a bordo corté con la sierra del carpintero un mastelero de juanete en tres partes, que incorporé a mi balsa no sin gran esfuerzo y fatiga; pero la esperanza de proveerme de aquello que tanto iba a necesitar me movió a hacer más de lo que me hubiera creído capaz en otro momento.
    Ahora mi balsa era lo bastante resistente para llevar una carga razonable. Se presentaba el problema de elegir lo indispensable y al mismo tiempo preservarlo de los golpes del mar. Ante todo puse en la balsa todas las planchas y tablas que pude reunir y después de pensar bien lo que me hacía falta busqué tres arcones de marinero y vaciándolos los puse en la balsa. Al primero lo llené de provisiones, como ser arroz, pan, tres quesos de Holanda, cinco trozos de carne seca de cabra —que había sido nuestro alimento habitual a bordo— y un pequeño sobrante de granos que fuera embarcado para alimentar las aves que llevábamos y que ya habíamos comido. Recordé la existencia de alguna cantidad de cebada y de trigo candeal, pero con gran disgusto mío las ratas lo habían devorado. Hallé muchas cajas de botellas de licor, pertenecientes al capitán, y además unos cinco o seis galones de la bebida llamada arak. Llevé las cajas a la balsa, no habiendo necesidad de meterlas en los arcones donde, por otra parte, no cabían.
    Mientras me ocupaba en esto advertí que la marea empezaba a subir aunque muy lentamente, y tuve la mortificación de ver mi saco, camisa y chaleco que dejara en la playa, arrastrados por el agua; había nadado hasta el barco con los calzones, que eran de lienzo y abiertos hasta la rodilla, y los calcetines. Lo ocurrido me hizo pensar en la necesidad de ropas, y aunque había mucha a bordo sólo tomé las indispensables por el momento, puesto que otras cosas reclamaban mi interés con mayor fuerza; sobre todo herramientas para trabajar en tierra. Después de mucho buscarlo di con el arcón del carpintero, que me parecía más valioso que todo un cargamento de oro. Lo llevé tal como estaba a la barca, sin perder tiempo en abrirlo, puesto que tenía una idea aproximada de su contenido.
    Mi inmediata tarea fue procurarme armas y municiones. Había dos magníficas escopetas de caza en la cabina del capitán, y dos pistolas; las cogí, así como algunos frascos de pólvora, un saquito de balas y dos viejas espadas enmohecidas. Recordaba que a bordo había tres barriles de pólvora, pero no el lugar donde los tenía el artillero. Tras mucho buscar di con ellos, y aunque uno se había mojado los restantes parecían secos y me los llevé todos a la balsa. Mi cargamento me llenaba de satisfacción, pero el problema estaba en llegar con él a la playa no teniendo vela, remo ni timón; el más pequeño golpe de viento hubiera acabado con mis esperanzas. Tenía, sin embargo, tres razones para sentirme confiado. En primer lugar la tranquilidad del océano, luego la marea alta que se movía hacia la costa y por fin el leve viento que soplaba en dirección de tierra. Encontré dos o tres remos rotos que habían sido del bote, y tras de hallar en cubierta algunas otras herramientas tales como dos sierras, un hacha y un martillo, bajé todo a la balsa y con tal cargamento me hice a la mar. Por espacio de una milla aproximadamente mi balsa navegó muy bien, sólo desviándose un poco del sitio donde tocara primeramente tierra, lo que me hizo suponer alguna corriente marina; acaso, pensé, hallaría cerca algún arroyo o ensenada que pudiera servirme de puerto para desembarcar mi cargamento.
    Ocurrió como lo imaginaba; pronto vi a la distancia una entrada en la tierra hacia donde la fuerte corriente de la marea se precipitaba, e hice por mantener mi balsa en el centro de la corriente. Fue entonces cuando estuve a punto de sufrir un segundo naufragio que, estoy seguro de ello, hubiera concluido conmigo. Ignorante por completo de la costa no pude impedir que mi balsa chocara contra un bajío, y no tocando tierra por el otro extremo faltó poco para que el cargamento resbalara hacia ese lado y cayera al agua. Apoyé la espalda contra los arcones para mantenerlos en su lugar, pero me fue imposible desencallar la balsa a pesar de mis esfuerzos, y por otra parte no me animaba casi a moverme y así, sosteniendo los arcones con todas mis fuerzas, permanecí cerca de media hora hasta que el ascenso de la marea niveló un poco la balsa. Un rato después se zafó sola, y aprovechando un remo la hice entrar en el canal y avanzando un poco me hallé en la boca de un riachuelo, con tierra a ambos lados y la fuerte corriente de la marea remontando la balsa. Miré hacia las orillas en procura de un buen sitio para desembarcar, ya que no quería internarme demasiado sino establecerme junto a la costa para esperar el paso de algún buque.
    Luego de elegir una pequeña caleta en la orilla derecha de la ensenada conduje con gran trabajo la balsa hasta allí y me puse tan cerca que clavando el remo pude hacer que tocara la tierra. Pero por segunda vez estuvo mi cargamento a punto de caer al agua, porque la orilla era muy abrupta y la balsa sólo tocaba en ella con uno de sus extremos, de modo que el otro quedaba a un nivel inferior y ponía en peligro mis cosas. Sólo me quedó esperar que la marea creciera aún más, empleando el remo como ancla para mantener la balsa junto a un sitio llano donde confiaba que alcanzaría la marea. Así fue, y tan pronto vi que había fondo suficiente para que la balsa pudiera moverse la llevé hasta esa plataforma llana sujetándola por medio de los remos clavados en el suelo, uno a cada extremo; y ahí quedamos hasta que la marea empezó a bajar y nos dejó en tierra firme.
    Mi inmediata tarea era la de reconocer el lugar en busca de un sitio adecuado para instalarme y almacenar mis efectos con toda seguridad. Ignoraba por completo dónde me encontraba. ¿Era el continente o una isla, estaba o no habitado, habría bestias salvajes en los alrededores? A una milla de donde me hallaba vi una colina alta y escarpada, que parecía sobrepasar a otras que continuaban la cordillera hacia el norte. Tomando una escopeta, una pistola y suficiente pólvora, me encaminé hacia la cumbre de la colina adonde llegué después de duras dificultades. Apenas dirigí la mirada en torno cuando tuve conciencia de mi triste destino: estaba en una isla, enteramente rodeada por el mar y sin tierras próximas, excepción hecha de algunos lejanos escollos y dos pequeñas islas, menores que ésta, a unas tres leguas hacia el oeste.
    Evidentemente la tierra era inculta y, como podía suponerse, solamente habitada por animales salvajes, de los que sin embargo no vi ninguno. Había una diversidad de aves cuyas especies eran desconocidas para mí; me pregunté si su carne sería o no comestible. Mientras regresaba, maté un gran pájaro que posaba en un árbol junto a un bosque. Pienso que aquél fue el primer disparo hecho en aquella tierra desde la creación del mundo, pues como respuesta al estampido se levantó una bandada inmensa de aves de toda clase produciendo una algarabía confusa en la que distinguí los gritos de cada especie, pero ninguno me resultó familiar. El pájaro muerto era parecido a un halcón, sobre todo en el pico y el color, pero no tenía las garras que son propias de este pájaro; en cuanto a su carne resultó imposible de comer.
    Satisfecho con lo que había investigado me volví a la balsa y empecé a trasladar el cargamento, lo que me ocupó el resto del día. Cuando vino la noche me pregunté dónde la pasaría, pues desconfiaba quedarme en el suelo por temor a alguna bestia salvaje, temor que como descubrí más adelante era infundado. Hice una barricada con los arcones y las tablas, en forma de tosca cabaña, y en ella me parapeté para pasar la noche. Aún ignoraba de qué manera iba a alimentarme, puesto que solamente había visto dos o tres animales parecidos a liebres en el bosque donde matara al pájaro.
    Se me ocurrió que aún podría sacar muchas cosas útiles del barco, en especial aparejos, velas y todo lo que pudiera ser transportado a tierra, y me decidí a hacer otro viaje a bordo. No ignoraba que la próxima tormenta acabaría con el barco, de modo que me pareció mejor dejar de lado toda tarea hasta concluir con aquélla. Consulté conmigo mismo si llevaría la balsa, pero me pareció poco apropiado y preferí irme otra vez nadando en cuanto bajara la marea. Así lo hice, abandonando la choza sin más que una camisa, calzoncillos y zapatos livianos.
    Cuando estuve a bordo construí una segunda balsa, pero aprovechando la experiencia de la primera no la hice tan pesada ni la cargué tanto. Muy pronto hallé cantidad de cosas útiles; primeramente, en el cuarto del carpintero, dos o tres cajas de clavos y tornillos, un gran barreno, una o dos docenas de hachuelas, y lo más precioso de todo, una piedra de afilar. Reuní todo juntamente con varios objetos pertenecientes al artillero, como ser algunas palancas de hierro, dos barriles de balas de mosquete, siete mosquetes y otra escopeta, con alguna pequeña cantidad de pólvora; hallé también una gran caja llena de perdigones y un pedazo de plomo, pero este último pesaba tanto que no pude hacerlo pasar por la borda. Fuera de esto reuní todas las ropas que pude encontrar, una vela sobrante de la cofa de trinquete, una hamaca, colchones y ropa de cama. Cargué mi segunda balsa con todo aquello y conseguí llevarla sin dificultad a tierra.
    Durante el tiempo que pasé a bordo había sentido el temor de que en mi ausencia mis provisiones fueran devoradas, pero cuando desembarqué pude ver que no había señales de vida en torno, salvo la presencia de un animal semejante a un gato montes que se había subido a uno de los arcones y que al verme se alejó un poco. Noté que no tenía miedo, y que me miraba fijamente como mostrando su intención de trabar relaciones amistosas. Le apunté con la escopeta, pero como no comprendió la razón del gesto se quedó inmóvil y sin mostrar deseos de alejarse, por lo que le ofrecí un pedazo de galleta que, dicho sea de paso, no me sobraba como para andar regalándola. Sin embargo, le di aquel trozo, que el gato comió después de olfatearlo; pareció quedar muy satisfecho y. desear aún más, pero ya entonces no repetí el obsequio y al rato lo vi marcharse.
    Trasladando mi segundo cargamento a tierra, me vi en la obligación de abrir los barriles de pólvora y dividir el contenido, porque el excesivo peso no me dejaba moverlos; me puse en seguida a construir una pequeña tienda con la vela y algunas estacas que corté; puse en la tienda todo aquello que podría estropearse con la lluvia o el sol, y por fuera hice una barricada con los arcones y los barriles vacíos, a manera de fortificación contra cualquier ataque de hombre o animal.
    Terminado esto bloqueé la puerta por dentro con tablones y por fuera con un arcón parado; tendiendo uno de los colchones, con las dos pistolas cerca de la mano y la escopeta a mi alcance, me metí en cama por primera vez y dormí plácidamente toda la noche porque estaba rendido hasta la extenuación, habiendo dormido muy poco la noche anterior y trabajado el día entero en lo que ya he descrito.
    Era dueño del más completo y variado surtido de efectos que jamás fuese reunido por un solo hombre, pero no me sentía aún satisfecho, ya que estando el barco a mi alcance me pareció necesario extraer de él todo lo posible. Por lo tanto iba diariamente a bordo aprovechando la marea baja y sacaba una y otra cosa del navío; en especial la tercera vez que fui traje todos los aparejos que pude reunir, las cuerdas y jarcias, con un pedazo de lona que servía para remendar las velas, y hasta el barril de pólvora que se había mojado. Por fin saqué del barco todas las velas, aunque me vi obligado a cortarlas en pedazos para llevarlas juntas; pero no me importaba, ya que en adelante sólo servirían como lonas.
    Lo que más me alegró en aquellos viajes fue que después de estar a bordo cinco o seis veces, y cuando ya no esperaba encontrar nada que valiera la pena de mover de su sitio, descubrí un gran barril de galleta, tres pipas de ron o aguardiente, una caja de azúcar y un barril de harina flor; me quedé admirado, porque ya no creí hallar provisión alguna, salvo las que estaban estropeadas por el agua. Vacié el barril de galleta, haciendo paquetes pequeños con los pedazos de velas, y pronto arribé felizmente a tierra con todo.
    Al otro día hice un nuevo viaje. El barco estaba ya despojado de todo lo que podía moverse y transportarse fácilmente, de modo que la emprendí con los cables cortando el más grueso en trozos que pudieran llevarse, y así preparé dos cables y una guindaleza, junto con todo el herraje que pude juntar y que puse en una gran balsa hecha con los trozos de vergas de cebadera y de mesana. Pero mi buena suerte empezó a abandonarme, porque la balsa era tan pesada y tenía tanta carga que apenas habíamos llegado a la pequeña caleta que me servía de desembarcadero cuando por una falsa maniobra se hundió arrojándome al agua con todos aquellos efectos. Yo no corría peligro puesto que estaba junto a la costa, pero el cargamento se perdió en gran parte, especialmente el hierro, que tan útil me hubiera sido. Cuando bajó la marea pude salvar buena parte de los pedazos de cable y algo de los herrajes, aunque con gran esfuerzo, porque tenía que zambullirme para buscarlo y pronto me agotó la tarea. Pese a todo seguí yendo al barco y trayendo lo que encontraba aprovechable.
    Llevaba ya trece días en la playa y había hecho once viajes al barco, en cuyo transcurso retiré todas aquellas cosas que un par de manos pueden mover, tanto que de haber continuado el buen tiempo estoy seguro que hubiera terminado por traerme el barco pieza por pieza a la costa. Cuando me disponía a mi duodécimo viaje empezó a soplar viento, pero aproveché la marea baja para intentar otra expedición. Me parecía haber saqueado completamente la cabina del capitán, y sin embargo hallé todavía un armario con cajones, en uno de los cuales había dos o tres navajas, un par de tijeras largas, y casi una docena de excelentes cuchillos y tenedores; en otro cajón hallé un valor de treinta y seis libras esterlinas en monedas europeas, brasileñas y algunas piezas de a ocho de oro y plata. Sonreí a la vista de aquel dinero. — ¡Ah, metal inútil! —exclamé—. ¿Para qué me sirves? No mereces que me moleste en recogerte; cualquiera de esos cuchillos vale más que tú. ¡En nada podría emplearte y mejor es que te quedes donde estás y te hundas como un ser cuya vida no vale la pena salvar!
    Pero luego lo pensé mejor y tomé el dinero, envolviendo todo con una pieza de lona y pensando ya en construir otra balsa; mas cuando salí a cubierta el cielo se había encapotado, el viento crecía y al cuarto de hora soplaba fuerte desde la costa. Comprendí que era inútil hacer una nueva balsa soplando viento de tierra, y que me convenía alejarme de allí antes de que comenzara el reflujo impidiéndome alcanzar la orilla. Me arrojé inmediatamente al agua y nadé hacia el canal con gran dificultad, en parte por el peso del bulto que llevaba y en parte por el fuerte oleaje que el viento levantaba cada vez con más violencia hasta convertirse en tempestad.
    Pude llegar con fortuna a mi pequeña tienda, donde me refugié con todas mis riquezas bien aseguradas. La tormenta arreció aquella noche, y a la mañana siguiente encontré que el barco había desaparecido. Me afligió un poco, pero mi consuelo fue reflexionar que no había perdido tiempo ni escatimado esfuerzos para retirar de él todo lo que pudiera serme útil, siendo bien poco lo que podía haber quedado a bordo.
    Desentendiéndome, pues, del barco y su recuerdo, sólo me ocupé de aquellos pedazos que la tormenta había arrastrado a la playa, pero pronto supe que serían de muy poca utilidad. Mis pensamientos estaban consagrados ahora a encontrar los medios de asegurarme contra los salvajes o las bestias que pudiera haber en la isla; vacilé mucho acerca de las medidas que debía tomar, si me convenía construir una choza o cavar un abrigo en la profundidad de la tierra. Por fin, luego de meditarlo bien, me resolví por ambas cosas. Y se me ocurre que puede ser interesante la descripción de cómo las llevé a cabo.
    Había advertido que el lugar en que estaba no era conveniente para establecerme, en especial porque se hallaba sobre terrenos pantanosos e insalubres próximos al mar, y cerca de allí no había agua dulce. Me resolví, por tanto, a buscar un sitio más saludable y apropiado para construir mi vivienda.
    Calculé aquello que necesitaba de manera indispensable: en primer lugar agua dulce y aire saludable, como ya he dicho; luego abrigo de los ardores solares y seguridad contra posibles atacantes, fueran hombres o animales. Finalmente quería tener frente a mí el horizonte marino, para que, si Dios me enviaba algún barco por las cercanías, no perdiera yo esa oportunidad de salvarme, ya que tal esperanza no había perecido todavía en mí.
    En busca del lugar que reuniera tales condiciones, hallé una pequeña explanada al costado de una colina cuya ladera era tan escarpada como un muro y me evitaba, por tanto, todo peligro de ese lado. En un lugar de la roca había un hueco, semejante a la entrada de una caverna, pero en realidad no se trataba de ninguna cueva ni entrada. Decidí instalar mi choza en la explanada, justamente delante de ese hueco; noté que la parte llana tenía unas cien yardas de ancho y el doble de largo, y que se extendía como un parque delante de mi puerta, descendiendo luego irregularmente hacia las tierras bajas del lado del mar. Estaba hacia el N-NO de la colina, de modo que me protegía de los calores diurnos hasta que el sol descendiera al O cuarto SO, que en aquellas latitudes ocurre casi al crepúsculo.
    Antes de principiar mi tienda tracé un semicírculo delante de la parte hueca, cuyo diámetro a partir de la roca era de unas diez yardas, y veinte en el diámetro total desde uno a otro extremo. En este semicírculo clavé dos hileras de fuertes estacas, hundiéndolas en tierra hasta que quedaron absolutamente firmes, sobresaliendo de la tierra hasta unos cinco pies y medio, y las agucé en la punta. Las dos hileras no estaban separadas más de seis pulgadas entre sí. Tomando entonces los pedazos de cable que me había procurado en el barco, los apilé en el interior del círculo apretándolos hasta que cubrieron el espacio entre las estacas, y sostuve mi empalizada con otras estacas de unos dos pies y medio que coloqué inclinadas por el lado de adentro, a manera de puntales. Tan fuerte quedó el vallado que ningún animal o ser humano hubiera podido derribarlo, ni siquiera pasar por encima. Tuve mucho que trabajar en él, especialmente cortando la madera de los bosques, llevándola al lugar y clavándola en tierra.
    Decidí que la entrada no sería una puerta sino una corta escalera para trepar a la empalizada, puesta de tal modo que una vez dentro fuera fácil retirarla, con lo cual me encontraba perfectamente amurallado y defendido contra todo el mundo y podía dormir sin temor a enemigos, aunque más tarde vine a saber que mis precauciones no eran necesarias.
    Con infinito trabajo reuní todos mis afectos en la fortaleza, provisiones, armas y demás cuya lista es ya conocida, y armé una gran tienda; para que me preservara de las lluvias, que en cierta estación caen allí con violencia, hice una tienda doble, es decir, una pequeña y otra mayor tendida por encima, cubriendo esta última con una tela embreada que traje del barco juntamente con las velas. Ya no dormía en el colchón sino que instalé la hamaca que había pertenecido al piloto del barco y era excelente.
    Puse en la tienda todas las provisiones y aquello que pudiera estropearse con las lluvias, y habiendo comprobado que mis bienes estaban a salvo cerré la entrada que hasta ese momento dejara abierta en la empalizada y desde entonces utilicé la escalera para entrar y salir.
    A partir de ese día principié a excavar la roca, de la que arranqué gran cantidad de piedras y tierra que fui apilando al pie de mi empalizada a manera de terraplén de pie y medio de alto. Pronto tuve, pues, una nueva cueva justamente detrás de mi tienda, que me servía de bodega y despensa.
    Todo aquello me llevó mucho tiempo y grandes fatigas, y en ese transcurso ocurrieron varias cosas que me preocuparon. En los días en que trazaba los planes para armar mi tienda y excavar la roca, ocurrió que en medio de una violenta tormenta que acababa de desatarse vi caer un rayo, seguido inmediatamente de un terrible trueno. No me asustó tanto el rayo como el pensamiento que de inmediato cruzó por mi mente:
    — ¡La pólvora!
    Creí que mi corazón cesaba de latir al pensar que en un segundo mi pólvora podía arder, privándome no sólo de defensa sino del alimento que contaba lograr con ella. Ni siquiera sentí miedo por mí mismo, porque sabía bien que si la pólvora estallaba no me daría tiempo a pensar de dónde procedía la catástrofe.
    Tal impresión me causó lo sucedido que después de la tormenta dejé de lado mis tareas —la tienda, la fortificación— y me apliqué a fabricar cajas y bolsas donde separar la pólvora para impedir que ardiera toda, y al mismo tiempo distanciarla lo bastante entre sí para que el incendio de una parcela no determinara el de las restantes. El trabajo llevó una quincena, pero por fin la pólvora, que alcanzaba a unas doscientas cuarenta libras, quedó dividida en no menos de cien paquetes. Por lo que respecta al barril que se había mojado no me inspiraba temor, de modo que lo puse en mi caverna, a la que yo llamaba «la cocina»; el resto lo distribuí en agujeros entre las rocas, cuidando que ninguna humedad llegara a los paquetes, y marcando exactamente el sitio donde los dejaba.
    Mientras me ocupaba en todo esto no dejé de salir por lo menos una vez al día con mi escopeta, en parte para distraerme y en parte para ver si cazaba algo comestible, a la vez que exploraba las posibilidades de la isla. El primer día que salí tuve gran satisfacción al encontrar que había cabras en los alrededores, pero pronto me desalentó lo tímidas, astutas y ágiles que se mostraban, al extremo de que era casi imposible acercarse a ellas. Sin descorazonarme me dije que la ocasión se presentaría de alcanzar alguna con mis disparos, como efectivamente ocurrió una vez que hube localizado los lugares que frecuentaban. Noté que si me acercaba viniendo por el valle, las cabras huían aterradas, aunque estuviesen al abrigo de las altas rocas, pero que si triscaban en el valle y yo venía por las alturas ni siquiera reparaban en mi presencia, de lo cual deduje que la posición de sus ojos era tal que no veían sino aquello que estaba a su nivel o por debajo. Adopté de inmediato la costumbre de encaramarme a las rocas más altas, y desde allí me fue bastante simple abatir alguna. El primer disparo que hice mató una cabra cuyo cabrito todavía se amamantaba, lo cual me produjo mucha pena, ya que al caer la madre vi que no se movía de su lado, incluso cuando me acerqué a él. Mientras llevaba la cabra sobre mis hombros, el cabrito me siguió hasta la empalizada, y allí lo tomé en los brazos y lo hice pasar al interior con la esperanza de domesticarlo. Pero se negó a comer y al fin me vi precisado a matarlo para comerlo yo. Aproveché aquella carne durante bastante tiempo, pues me alimentaba con mucha prudencia y trataba de economizar las provisiones, especialmente la galleta.





    5. EL DIARIO DE ROBINSON




    Ahora que me toca iniciar la melancólica narración de una vida solitaria, tal como acaso nunca fuera imaginada en el mundo, quiero hacerlo desde su comienzo y proseguir ordenadamente. Según mis cálculos, había arribado en la forma narrada a tan hórrida isla un 30 de setiembre, cuando el sol en su equinoccio otoñal estaba casi sobre mi cabeza, de donde calculé que me hallaba a una latitud de nueve grados veintidós minutos norte.
    Después de vivir allí diez o doce días se me ocurrió que por falta de calendarios, así como de papel y tinta, perdería la cuenta del tiempo y no sería capaz de distinguir los días de fiesta de los de trabajo. Para evitarlo hice un poste en forma de cruz, que clavé en el sitio donde por primera vez había tocado tierra, y grabé en él con mi cuchillo y en letras mayúsculas:


    LLEGUE A ESTA PLAYA EL
    30 DE SETIEMBRE DE 1659


    Sobre los lados del poste practicaba diariamente un corte, y cada siete una marca algo mayor; el primer día del mes hacía una señal aún más grande, y en esa forma llevé mi calendario de semanas, meses, años.
    Entre lo mucho que había traído del barco encallado en los viajes arriba mencionados se encontraban diversas cosas muy útiles para mí, aunque menos que las otras, por lo cual no las describí antes. En particular plumas* tinta y papel, y objetos pertenecientes al capitán, piloto, artillero y carpintero, tales como tres o cuatro compases, instrumentos matemáticos, cuadrantes, anteojos de larga vista, mapas y libros de navegación, etc., todo lo cual traje a tierra sin saber si me serviría o no. Encontré también tres excelentes Biblias que vinieran de Inglaterra con mi cargamento y que yo había cuidado de llevar conmigo; algunos libros portugueses, entre ellos dos o tres libros católicos de oraciones y varios otros que conservé cuidadosamente. No debo olvidarme de señalar que teníamos a bordo un perro y dos gatos, de cuya importante historia habré de ocuparme en su justo lugar. Había traído conmigo los dos gatos, y en cuanto al perro se arrojó él mismo al agua y vino nadando a mi lado el día siguiente a mi primer viaje al barco; desde entonces estuvo conmigo y fue un fiel compañero por muchos años. No me interesaba lo que pudiera apresar para mí, ni la compañía que me hacía; hubiera solamente deseado oírle hablar, y por desgracia eso era lo imposible.
    Como antes he dicho encontré plumas, tinta y papel, e hice lo indecible por economizarlos; mientras duró la tinta pude llevar una crónica muy exacta, pero cuando se terminó me hallé imposibilitado de continuarla, ya que no pude hacer tinta a pesar de todo lo que probé. Esto vino a demostrarme que necesitaba muchas cosas fuera de las que había acumulado; así como tinta, debo citar la falta que me hacían una azada, pico y pala para roturar la tierra, y también agujas, alfileres e hilo; en cuanto al lienzo, pronto me pasé fácilmente sin él.
    Tal falta de utensilios tornaba fatigosa toda tarea que emprendía, y transcurrió casi un año antes de que hubiera terminado mi empalizada y las demás obras. Las estacas, que eran tan pesadas como podía encontrar, llevaba mucho tiempo cortarlas y aguzarlas en el bosque y otro tanto moverlas hasta la explanada. A veces pasaba dos días entre cortar y trasladar uno de aquellos postes y un tercer día en hundirlo firmemente en el suelo, para lo cual me valía de una pesada maza de madera hasta que se me ocurrió emplear una de las palancas de hierro; asimismo me daba mucho trabajo asegurar aquellos postes.
    Pero ¿por qué había de preocuparme el mucho tiempo que insumían estas cosas? Bien claro estaba que me sobraba tiempo, y si mis trabajos hubieran terminado antes me habría quedado sin saber qué hacer, salvo explorar la isla en busca de alimento, cosa que llevaba a cabo casi diariamente.
    Empecé así a meditar seriamente sobre la condición en que me hallaba y las circunstancias a que me veía reducido, y redacté por escrito mis pensamientos, no tanto por dejarlos a mis herederos, que por lo visto serían pocos, sino para aliviar a mi espíritu de llevarlos constantemente consigo hasta la aflicción. Mi razón empezaba a dominar mis desfallecimientos, veía de consolarme lo mejor posible y a oponer el bien al mal para que mi situación no me pareciera tan desesperada en comparación a otras mucho peores. Todo eso fue escrito imparcialmente, a manera de un debe y haber, señalando los consuelos que me habían sido dados a cambio de las desgracias que sufría, en la siguiente forma:





LO MALO

He sido arrojado a una isla desierta sin la menor esperanza de rescate.
He sido excluido del resto del mundo, a solas con mi miseria.








LO BUENO

Pero vivo, sin haberme ahogado como mis compañeros.
Pero también he sido excluido de la muerte, al contrario de toda la tripulación del barco; y El, que me salvó milagrosamente de tal muerte, puede salvarme igualmente de esta condición en que me hallo.

LO MALO

Vivo separado de la humanidad, solitario y desterrado de toda sociedad. No tengo ropas para cubrirme.
Carezco de toda defensa contra los animales y los hombres.
No tengo a nadie con quien hablar, a nadie que me consuele.













LO BUENO

Pero no he muerto de hambre en un lugar desierto, privado de toda subsistencia.
Pero estoy en un clima cálido donde las ropas me servirían de poco.
Pero me encuentro en una isla donde no he visto animales feroces que me amenacen, como los viera en la costa de África. ¿Y si hubiera naufragado allá? Pero Dios envió milagrosamente el barco cerca de la costa para que pudiera sacar de él multitud de cosas necesarias que suplen mis necesidades o me permitirán hacerlo mientras viva.


    Habiendo conseguido acostumbrar un poco mi espíritu a su actual condición y abandonando la costumbre de mirar el mar por si divisaba algún navío, me apliqué desde entonces a organizar mi vida y a hacerla lo más confortable posible.
    He descrito ya mi vivienda, que era una tienda junto a la ladera rocosa, rodeada de un fuerte vallado de estacas y cables al que puedo llamar ahora muro porque del lado exterior le puse una base de tierra con césped que alcanzaba a dos pies de alto; más tarde —pienso que un año y medio después— agregué unas vigas y cabrias que iban de la empalizada hasta las rocas, e hice un techo con ramas de árbol y todo aquello que pudiera protegerme mejor de las lluvias, que en ciertas épocas del año caían con gran violencia.
    Ya he dicho que había puesto todos mis efectos dentro de la empalizada y en la caverna. Al principio estaban tan revueltos, apilados sin orden ni cuidado, que ocupaban casi todo mi sitio, no dejándome lugar libre. Me puse entonces a agrandar la caverna, siéndome fácil porque se trataba de una roca arenosa que cedía fácilmente. Ya en aquel entonces estaba seguro de que no había fieras en la isla, y ahondando la cueva hacia la derecha hice un túnel que formaba una salida más allá de la empalizada, lo cual me permitiría salir y entrar de lo que llamaríamos la parte trasera de mi casa y a la vez depósito de efectos.
    Pude luego dedicarme a fabricar aquellas cosas que más falta me hacían, como por ejemplo una mesa y una silla, sin las cuales no podría gozar de las pocas comodidades que tenía en el mundo, ya que era difícil escribir o comer agradablemente sin una mesa. Nunca había manejado una herramienta en mi vida, pero con tiempo, aplicación y perseverancia descubrí que si hubiera tenido los elementos necesarios habría podido fabricar cuanto me faltaba. Así y todo hice muchas cosas sin herramienta alguna, y otras con la sola ayuda de una azuela y un hacha, aunque con infinitas dificultades. Si, por ejemplo, necesitaba un tablón, no me quedaba otro remedio que derribar un árbol, ponerlo en un caballete y hacharlo por ambos lados hasta darle el espesor de un tablón, y lo pulía luego convenientemente con la azuela. Con este método sólo sacaba un tablón por árbol, pero como no encontraba otra manera de lograrlo me armaba de paciencia ante la enormidad de tiempo que me llevaba la sola obtención de una tabla. Cierto que mi tiempo y mi trabajo nada valían allí, y tanto me daba emplearlos de un modo que de otro.
    Así fabriqué en primer lugar una mesa y una silla, aprovechando los pedazos de tabla que trajera del barco. Después, cuando obtuve algunos tablones de la manera ya descrita, hice estantes de pie y medio de ancho, uno sobre otro, a lo largo de las paredes de mi cueva, que servían para poner mis herramientas, clavos y herrajes teniendo todo clasificado y puede decirse que al alcance de la mano. Clavé soportes en las paredes para colgar mis escopetas y lo que en esa forma quedara cómodo, tanto que si alguien hubiera podido ver mi cueva le hubiera parecido un depósito general de objetos necesarios. Tenía todo tan al alcance de la mano que me encantaba ver cada cosa en orden y, más que nada, descubrir que mi provisión era tan abundante.
    Fue entonces cuando empecé a llevar un diario de mis tareas cotidianas. En un principio había estado demasiado ocupado, no solamente con mi trabajo sino con los confusos pensamientos que pasaban por mi mente, y mi diario hubiera aparecido lleno de cosas torpes y melancólicas. Pero habiendo superado en alguna medida ese estado de ánimo y sintiéndome seguro en mi casa, dueño de una mesa y silla y con todo lo que me rodeaba aceptablemente bueno, empecé a llevar mi diario, del cual he de dar aquí una copia —aunque a veces resulte repetición de lo ya dicho— hasta el punto en que, por falta de tinta, hube de interrumpirlo.


    FRAGMENTOS DEL DIARIO


    4 de noviembre — Empecé esta mañana a reglar mis horas de trabajo, de salidas, sueño y esparcimiento. Todas las mañanas partía con mi escopeta por espacio de dos o tres horas, siempre que no lloviera; luego trabajaba hasta las once más o menos, comía, y me echaba a dormir de doce a dos por ser intolerable el calor a tales horas. A la tarde volvía a trabajar. La tarea de este día y del siguiente fue dedicada enteramente a la construcción de la mesa, sintiéndome todavía muy torpe como carpintero, aunque con el tiempo llegué a ser tan diestro como cualquier otro.


    5  de noviembre — Salí con la escopeta y mi perro, matando un gato montes. Piel muy suave, pero carne imposible de comer. Desollaba los animales que había cazado para aprovechar después las pieles. Volviendo por la playa vi toda clase de aves marinas desconocidas para mí; me sorprendieron y casi asustaron a dos o tres focas, y mientras trataba de verificar qué clase de animales eran las vi huir en el mar.


    6  de noviembre — Después del paseo matinal seguí trabajando en la mesa y la terminé, aunque no a mi gusto; muy pronto me di cuenta de cómo podía mejorarla.


    7  de noviembre — El buen tiempo parece mantenerse firme. Pasé desde el 7 hasta parte del 12 (salvo el domingo 11) trabajando en la construcción de una silla, y logré por fin darle una forma aceptable aunque no de mi gusto; varias veces la deshice a mitad de trabajo.


    NOTA: Pronto descuidé la observancia del domingo, porque olvidándome de señalarlos en el poste con una marca mayor perdí la noción de los mismos.


    17 de noviembre — Empecé a excavar la roca detrás de mi tienda para tener más lugar de almacenamiento.


    NOTA: Tres cosas me hacían gran falta en esta tarea: un azadón, pala y una carretilla o espuerta, de manera que desistí de mi trabajo y busqué la manera de procurarme aquellas herramientas necesarias o sus equivalentes. A manera de azadón utilicé una de las barras de hierro, que aunque muy pesadas daban buen resultado, pero subsistía la cuestión de la pala. Esto me era tan necesario que sin ella no podía seguir la excavación, aunque ignoraba cómo podría fabricar una.


    18 de noviembre — Explorando los bosques encontré un árbol de esa madera que en Brasil llaman palo de hierro por su gran dureza; si no era el mismo se le parecía mucho, tanto que con enorme trabajo y estropeando la hoja de mi hacha conseguí cortar un pedazo y traerlo a casa no sin esfuerzos porque pesaba enormemente.
    La extraordinaria dureza de esta madera me obligó a perder mucho tiempo mientras poco a poco le iba dando la forma de una pala, con un mango igual al que usamos en Inglaterra; desgraciadamente, como no tenía hierro para guarnecer la extremidad más ancha, bien poco habría de durarme su filo.
    Todavía me faltaba la espuerta. No sabía cómo arreglármelas para hacer una canasta careciendo de varillas de mimbre lo bastante flexibles, y no habiendo descubierto todavía su existencia en la isla. Para fabricar una carretilla encontraba la dificultad en la rueda, ya que no veía modo de construirla, sin contar que tampoco podría arreglármelas para forjar los soportes y el eje que debería sostener la rueda. Renunciando a esa idea me conformé con transportar lo que sacaba de la caverna en una especie de artesa como las que los albañiles emplean para llevar el mortero.


    23 de noviembre — Volví a mi excavación, dueño ya de herramientas suficientes para ello, y trabajé dieciocho días consecutivos, tanto como me lo permitían mis fuerzas y el tiempo disponible, en ensanchar y profundizar la caverna para que mis efectos cupieran adecuadamente.


    NOTA: Durante todo este tiempo ensanché la caverna con intención de que me sirviera al mismo tiempo de almacén, cocina, comedor y bodega. Preferí seguir durmiendo en la tienda, salvo cuando en la estación de las lluvias los chaparrones eran tan fuertes que terminaban por mojarme, lo que me llevó más adelante a techar el espacio dentro de la empalizada con largas pértigas que apoyaban contra la roca, y que fui cubriendo con espadañas y grandes hojas de árboles, como un techo de paja.


    10 de diciembre — Empezaba a considerar concluida mi caverna cuando debido acaso a la excesiva anchura se produjo un hundimiento lateral, cayendo tanta piedra que llegó a atemorizarme y no sin motivo, pues si hubiera estado debajo no habría necesitado sepulturero. Este desastre me obligó a reanudar el duro trabajo, sacando fuera lo que se había desplomado y asegurando el techo para que no volara.


    11 de diciembre — De acuerdo con lo decidido coloqué dos puntales contra el techo con dos tablas cruzadas en su extremidad. Terminé la tarea al día siguiente, y agregando luego otros postes con tablones que sostuvieron el techo pude asegurarlo firmemente una semana más tarde. Como los postes habían sido dispuestos en hileras, me sirvieron para establecer distintas habitaciones en mi nueva casa.


    17 de noviembre — Desde la fecha hasta el veinte estuve fijando estanterías y poniendo clavos en los postes para colgar diversas cosas; ya empiezo a encontrar ordenada mi casa.


    20 de diciembre — Llevé al interior de la cueva todas mis cosas, y comencé a amueblarla poniendo algunos tablones a modo de aparador para las vituallas. Empiezan a faltarme tablas, pero aún alcanzaron para hacer otra mesa.


    24 de diciembre — Llovió todo el día y toda la noche, sin que pudiera salir.


    25 de diciembre — Llovió todo el día.


    26 de diciembre — Cesó la lluvia, refrescando la tierra de un modo muy agradable.


    27 de diciembre — Maté una cabra y herí a otra, alcanzando a apresarla y llevarla a casa sujeta con una cuerda. Allí le entablillé y vendé la pata rota.


    NOTA: Tanto la cuidé que se mejoró, quedándole la pata igual que antes. A causa de mis cuidados se domesticó, comía del césped en torno a mi casa y no se alejaba mucho. Por primera vez pensé en la posibilidad de criar animales domésticos para que no me faltaran alimentos el día en que se concluyera la pólvora.
    28, 29 y 30 de diciembre — Fuertes calores y ninguna brisa, de modo que apenas salía al atardecer en busca de alimentos. Pasé este tiempo ordenando mis cosas.


    1.o de enero — Todavía muy caluroso, por lo que salía temprano y al anochecer con la escopeta, descansando a mitad del día. Al entrar esta tarde en los valles que conducen al centro de la isla hallé gran cantidad de cabras, aunque tan asustadizas que era difícil acercarse. Se me ocurrió  L que acaso mi perro fuera capaz de echarlas hacia mi lado.


    2  de enero — Llevé al perro y lo solté a las cabras, pero contra lo que esperaba le hicieron frente, y él advirtió el peligro sin animarse a avanzar.


    3  de enero — Empecé el muro o empalizada, y en previsión de algún posible ataque traté de darle una extraordinaria solidez y tamaño.


    NOTA: Como esto ha sido ya narrado, omito todo lo que a su respecto contiene el diario. Basta observar que la tarea me llevó desde el 3 de enero hasta el 14 de abril, y durante este tiempo construí, terminé y mejoré aquel vallado que sólo tenía sin embargo veinticuatro yardas de largo y formaba un semicírculo desde un punto de la pared rocosa hasta otro situado a ocho yardas más allá, con la entrada de la caverna en el justo medio.


    En todo este tiempo trabajé intensamente a pesar de estorbármelo la lluvia muchos días y a veces semanas enteras; me perseguía la idea de que no iba a estar bien seguro hasta que concluyera la empalizada. Es difícil creer lo que me costó cada cosa, en especial cortar madera del bosque y clavarla en tierra, ya que había hecho estacas más grandes de lo que hubiera sido necesario.
    Concluido el vallado, su parte exterior doblemente protegida por un terraplén de tierra con césped de bastante altura, me persuadí de que si alguien desembarcaba en la isla no se daría cuenta de que era una habitación humana; y tal cosa me fue harto útil, como lo comprobé más adelante.
    Diariamente iba al bosque de caza, salvo cuando llovía, y con frecuencia realizaba algún descubrimiento ventajoso. Una vez hallé una especie de palomas silvestres que no anidaban en los árboles como la torcaz sino que formaban especie de palomares en los agujeros de las rocas. Traté de domesticar algunos pichones y lo conseguí, pero cuando fueron mayores no pude impedir que se volaran, probablemente por falta de alimento, que yo no tenía para darles; con todo, iba frecuentemente a sus nidos y me apoderaba de los pichones, cuya carne era excelente.
    A medida que atendía mis cosas fui descubriendo todo lo que me faltaba y que a primera vista me parecía imposible de hacer o procurarme. Por ejemplo, comprendí que no podría construir un tonel con aros; tenía uno o dos barriles pequeños, como ya he dicho, pero jamás pude aprovecharlos de modelo para uno mayor, aunque pasé semanas probando. Era imposible colocar los fondos y unir las duelas con suficiente justeza para que no dejaran escapar el agua, de manera que por fin abandoné la tentativa.
    En segundo término carecía de velas; la falta de luz me obligaba a acostarme apenas oscurecía, lo que allí ocurre a eso de las siete. Me acordaba del pedazo de cera con el cual hice velas durante mi aventura en Africa, pero ahora el único remedio a mi alcance era aprovechar la grasa de las cabras que mataba; fabriqué un platillo de arcilla que puse a cocer al sol, y agregándole un pabilo de estopa conseguí hacer una lámpara que daba una luz mucho más débil y vacilante que la de una vela.
    Mientras me ocupaba en todo esto, encontré al registrar entre mis cosas un pequeño saco que, como ya lo he dicho antes, había contenido granos para el alimento de las aves que teníamos a bordo, pero que acaso había sido llenado en el viaje anterior, cuando el navío vino de Lisboa. Lo poco que quedaba en el saco aparecía devorado por las ratas, y sólo encontré polvo y cáscaras, de manera que precisando el saco para otro uso —creo recordar que para poner pólvora en él cuando me asustó el episodio del rayo— fui a sacudir las cáscaras a un lado de la empalizada, junto a las rocas. Esto sucedía un poco antes de las grandes lluvias ya citadas, y muy pronto olvidé que había vaciado allí los restos del saco, cuando aproximadamente un mes más tarde vi surgir de la tierra unos tallos verdes que me parecieron de una planta desconocida; pero mi asombro fue inmenso al notar poco después que las plantas echaban diez o doce espigas que reconocí ser de cebada, el mismo tipo de cebada que se cultiva en Europa, sobre todo en Inglaterra.
    Podéis imaginar cómo habré cuidado aquellas espigas, que recogí a su debido tiempo, es decir a fines de junio. Me resolví a sembrar todo el grano, confiando que con el tiempo tendría bastante para hacer pan, pero recién al cuarto año pude permitirme separar algo de la cosecha para alimentarme, y esto con mucha prudencia, como relataré luego, pues perdí casi todo lo que sembrara la primera vez, no habiendo calculado bien la época adecuada; lo hice antes de la estación de sequía, por lo cual se malogró todo o casi todo, como contaré en su debido tiempo.
    Además de la cebada habían crecido allí veinte o treinta tallos de arroz que cuidé con la misma atención, pensando que de su grano podría hacer pan u otro alimento, y descubrí el modo de cocerlo sin necesidad de horno, aunque más adelante lo tuve. Pero volvamos a mi diario.
    Trabajé hasta la extenuación durante esos tres o cuatro meses para terminar la empalizada, y el 14 de abril quedó cerrada, y podía entrar y salir de ella por una escalera que no dejaba huellas exteriores de que allí hubiera una habitación humana.


    16 de abril — Terminé la escalera con la cual trepaba a la empalizada, retirándola luego y dejándola del lado de adentro. Esto me aislaba totalmente y nadie podía llegar hasta mí a menos que escalara la pared.


    Al día siguiente de concluir el trabajo estuve a punto de que todo se malograra y hasta me vi en peligro de muerte. Ocurrió que trabajando detrás de mi tienda, justo delante de la entrada de la cueva, me espantó de improviso algo horroroso: el material que formaba el techo de mi caverna empezó a desplomarse mientras tierra y piedras de la ladera de la colina caían sobre mí; dos de los postes que pusiera como puntales se quebraron con un ruido terrible. Me asusté mucho, pero en ese instante no tuve la visión de lo que verdaderamente sucedía, y me pareció tan sólo que el techo de la caverna se desplomaba, como ya había ocurrido parcialmente antes; temeroso de ser alcanzado corrí entonces a la escalera y pasé por encima de la empalizada, temiendo a cada instante que las rocas de la colina cayeran sobre mí aplastándome. Tan pronto pisé suelo firme me di cuenta de que se trataba de un violento terremoto; tres veces tembló la tierra con intervalo de ocho minutos, y sus sacudidas eran tales que hubiera derribado el más sólido edificio de la tierra. Vi que un trozo de roca, a una media milla de donde me hallaba, caía hacia el mar con el ruido más espantoso que haya oído en mi vida. El mar estaba también revuelto por el cataclismo, y me parece que las sacudidas eran aún más fuertes allí que en la isla.


    Después de la tercera conmoción hubo un rato de calma y empecé a cobrar valor: sin embargo, no me animaba a trasponer la empalizada por miedo a ser enterrado vivo, y me senté en el suelo profundamente abatido y desconsolado, sin saber qué hacer. En ningún momento tuve el menor pensamiento religioso, salvo la común imploración: « ¡Apiádate de mí, Señor!», y cuando cesó el terremoto también dejé de pronunciarla.


    Mientras permanecí allí reparé en que el cielo se encapotaba como si fuera a llover. Pronto comenzaron ráfagas cada vez más violentas, y media hora más tarde se desencadenaba un terrible huracán. El océano estaba cubierto de espumas, rompía con violencia en la playa, eran arrancados los árboles de raíz y aquella horrorosa tormenta duró casi tres horas antes de calmar, y sobrevino una profunda tranquilidad tras la cual principió a llover copiosamente.
    Me vi obligado a volver a la cueva, aunque lleno de temor porque me parecía que iba a desplomarse sobre mí. La lluvia era tan violenta que para evitar que la acumulación del agua dentro de mi fortificación concluyera por inundar la cueva, tuve que hacer un agujero en la muralla como vía de escape. Me quedé allí cobrando más coraje a medida que pasaba el tiempo y los temblores no se repetían. Buscando reanimar mis ánimos, que por cierto lo necesitaban mucho, fui a mi pequeño almacén y bebí un poco de ron, cosa que hacía siempre con mucha prudencia, sabedor de que no podría reemplazarlo cuando se concluyera.
    Llovió toda esa noche y gran parte del día siguiente, de modo que no pude salir, pero sintiéndome ya repuesto medité lo que me convendría hacer, llegando a la conclusión de que si la isla estaba sujeta a tales terremotos no me convenía vivir en una caverna; era mejor levantar mi choza en un sitio abierto que circundaría con una empalizada como lo hiciera aquí, para asegurarme contra bestias o seres humanos; porque si osaba quedarme en la cueva terminaría por morir enterrado vivo.
    Me resolví, pues, a mover mi tienda del sitio en que estaba, justamente debajo de la escarpada ladera de la colina, ya que indudablemente sería sepultado al producirse un nuevo terremoto. Pasé los días siguientes —19 y 20 de abril— en estudiar dónde y cómo mudaría mi habitación.
    El miedo de ser aplastado por un alud no me dejaba dormir tranquilo, pero menos aún quería hacerlo en sitio descubierto y sin la protección de la empalizada. Cuando miraba en torno y veía cuan ordenadas estaban mis cosas, lo bien ocultas y a salvo que se encontraban, me dolía mucho la idea de abandonar el sitio.
    Se me ocurrió entonces que me llevaría mucho tiempo la nueva instalación, y que mientras tanto era mejor correr el riesgo de seguir viviendo allí hasta que hubiera encontrado un lugar apropiado y puesto en condiciones de defensa para mudarme a él. Ya resuelto, decidí que empezaría con toda la rapidez posible a levantar una empalizada circular en el sitio elegido, haciéndola con estacas y cables como la primera, y que una vez concluida pondría dentro mi tienda; pero entretanto decidí arriesgarme a permanecer en mi primera morada. Esto sucedía el veintiuno.


    22 de abril — A la mañana siguiente me dispuse a poner en práctica mis decisiones, pero el gran problema lo constituían las herramientas. Tenía tres grandes hachas, abundancia de hachuelas (que habíamos llevado en cantidad para el intercambio con los negros), pero de tanto cortar madera dura estaban llenas de muescas y sin filo. Tenía una piedra de afilar, pero no era posible hacerla dar vueltas al mismo tiempo que aplicaba las hojas; este problema me ocupó tanto tiempo como a un hombre de estado resolver una difícil situación política o a un juez la vida o muerte de un hombre. Por fin armé la rueda con un cable que la pusiera en movimiento con el impulso del pie, dejándome ambas manos libres.


    NOTA: Jamás había visto mecanismo igual en Inglaterra, o por lo menos no había observado su funcionamiento, aunque más tarde vine a saber que allí era muy común. Aparte de eso, el gran tamaño y peso de la piedra dificultaba mi tarea, de modo que perfeccionar la máquina me llevó una semana de trabajo.


    28 y 29 de abril — Pasé estos días afilando mis herramientas y tuve la alegría de que la máquina funcionara muy bien.





    6. EL DIARIO DE ROBINSON (II)




    Primero de mayo — Mirando hacia la playa de mañana a la hora del reflujo, vi un objeto bastante grande y semejante a un barril. Me acerqué y hallé un pequeño tonel y dos o tres pedazos del barco que el reciente huracán había tirado a la costa. Mirando hacia el casco mismo, me pareció que emergía del agua más que en días anteriores. Examiné el barril y vi que contenía pólvora, pero tan mojada que estaba dura como piedra. Lo hice rodar para alejarlo de las olas y me acerqué cuanto pude por la playa a fin de examinar el casco más de cerca.
    Cuando llegué a su lado noté que había cambiado extrañamente de posición. El castillo de proa, antes enterrado en la arena, estaba ahora a seis pies de elevación; la popa, que se había partido y separado del resto por la violencia del mar —poco después que yo cesara de explorarla—, estaba tumbada de lado y la arena se acumulaba de tal manera en aquella parte, hasta la popa, que pude llegar caminando a ella cuando antes debía nadar cerca de un cuarto de milla. Al principio me maravillé, pero pronto deduje que el cambio se debía al terremoto. Y como a causa de esto el barco estaba más destrozado que antes, diariamente llegaban objetos a la playa que el viento y el oleaje sacaban del navío y depositaban en tierra.
    Esta novedad apartó mis pensamientos del proyecto de mudanza, y empecé a buscar la manera de introducirme en el barco; pero mi desilusión fue grande al comprobar que el casco estaba lleno de arena. Decidí, sin embargo, sacar todos los pedazos que pudiera, ya que sin duda me serían de utilidad.


    3 al 17 de mayo — Fui diariamente al casco, extraje gran cantidad de madera, planchas y tablones, así como unas trescientas libras de hierro.


    24 de mayo — Trabajé hasta hoy en el casco del barco, aflojando con la palanca diversas partes que flotaron en cuanto se levantó viento, pero como por desgracia soplaba de la costa nada llegó a tierra salvo algunas maderas y un barril que contenía salazón de cerdo del Brasil, tan estropeado por el agua que no era de ningún provecho.
    Seguí trabajando en el casco hasta el 15 de junio, salvo los momentos dedicados a cazar, que elegía a las horas de marea alta para tener tiempo libre durante el reflujo. Ya entonces había obtenido suficiente madera y herraje como para construir un buen bote si hubiera sabido cómo. También saqué poco a poco y en muchos pedazos casi cien libras de plomo.


    16 de junio — Yendo hacia la playa encontré una enorme tortuga. Era la primera que veía, más por mala suerte que por otra cosa, ya que si hubiera ido al otro lado de la isla habría encontrado cientos de ellas, como lo descubrí más tarde; pero acaso me hubiera salido aquello demasiado caro.


    17 de junio — Pasé el día cocinando la tortuga, dentro de la cual había sesenta huevos. Su carne me pareció en esa ocasión la más deliciosa que hubiera probado en mi vida, ya que desde mi arribo a tan triste lugar mi único alimento habían sido las cabras y las aves.


    18  de junio — Llovió el día entero y me quedé dentro. Esta vez encontré que el agua era muy fría y sentí escalofríos, lo que me pareció muy raro en estas latitudes.


    19 de junio — Muy enfermo y temblando como si hiciese mucho frío.


    20 de junio — No dormí en toda la noche; terrible dolor de cabeza, fiebre.


    21  de junio — Muy enfermo, mortalmente asustado con la idea de sentirme tan mal y no tener ayuda alguna. Rogué a Dios .por primera vez desde la tempestad en Hull, pero apenas recuerdo lo que dije y por qué lo dije. Mis pensamientos eran confusos.


    22  de junio — Algo mejor, pero lleno de aprensiones por mi enfermedad.


    23  de junio — Otra vez muy mal; tiritando de frío y luego con una fuerte jaqueca.


    24 de junio — Mucho mejor.


    25 de junio — Violenta calentura. La crisis duró siete horas, con alternancias de calor y frío y luego una copiosa transpiración.


    26 de junio — Mejor. No teniendo qué comer salí con la escopeta, sintiéndome muy débil. Con todo maté una cabra, la traje penosamente a casa y luego de cocer un pedazo lo comí. Hubiera preferido hervirlo y hacer un poco de caldo, pero no tenía olla.


    27  de junio — Tan violenta calentura que estuve el día entero en cama sin comer ni beber. Me parecía que iba a morir de sed, sintiéndome demasiado débil para levantarme en busca de agua. Rogué otra vez a Dios, pero en mi delirio e ignorando lo que debía decir sólo atinaba a implorar: « ¡Señor, apiádate! ¡Señor, protégeme! ¡Ten compasión de mí, Señor!» Estuve así continuamente por dos o tres horas hasta que la calentura cedió y quedé dormido; me desperté ya entrada la noche. Me sentía mejor, pero muy débil y con una sed continua. No tenía agua en mi habitación de modo que hube de esperar hasta la mañana, durmiendo entretanto. Mientras dormía tuve un sueño terrible.
    Soñé que estaba sentado en el suelo, más allá de la empalizada, donde permanecí mientras la tormenta arreciaba después del terremoto, y que veía un hombre que se acercaba en una oscura nube, envuelto en un halo de fuego que iluminaba el terreno; brillaba de tal manera que apenas podía soportar su presencia. Su aspecto era tan imponente que no hay palabras para describirlo. Cuando posé los pies en tierra creí que el suelo temblaba con un nuevo terremoto, y el aire entero pareció llenarse, para mi mayor espanto, de ígneas lenguas. Apenas había descendido cuando se adelantó hacia mí, con una lanza en la mano para matarme; de pie en una eminencia, oí que me hablaba con voz tan terrible que es imposible tratar de describir el espanto que me produjo. Todo lo que alcancé a entender fue esto: «puesto que lo que has visto no te ha movido a arrepentirte, ahora morirás». Y levantó la lanza para atravesarme con ella.
    Nadie que lea este relato esperará que yo sea capaz de describir el espanto que pasó mi alma ante tan terrible visión. Aunque solamente se tratara de un sueño, soñé también el espanto que me produjo. Y menos aún podría dar una idea de la impresión que quedó en mí una vez que hube despertado y comprendido que se trataba de un sueño.
    No tenía, ¡ay!, instrucción religiosa; de todo lo que la bondad de mi padre me había inculcado apenas quedaba nada tras ocho años de errantes extravíos y continuo contacto con aquellos que, como yo, eran malos y profanos en máximo grado. No recordaba haber tenido en todo aquel tiempo un solo pensamiento que tendiera a la contemplación de Dios o a un examen severo de mi propia conducta.
    Es verdad que cuando me salvé del naufragio y tuve la certeza de que todos habían muerto salvo yo, pasé por un momento de éxtasis y por tales transportes que, de haberme asistido la gracia de Dios, me hubieran llevado a una verdadera gratitud; pero todo terminó donde había principiado —un simple arranque de alegría por sentirme aún vivo— sin que eso me moviera a reflexionar sobre la bondad de la mano que, preservándome, había guardado mi vida mientras perecían todos los demás. No se me ocurrió pensar por qué la Providencia había sido generosa conmigo; tuve sólo la vulgar alegría que todo marino salvado de un naufragio se apresura a ahogar en un vaso de ponche para olvidarla de inmediato; y toda mi vida había sido así.
    Aun el terremoto, bien que nada podía ser más terrible que sus manifestaciones o más revelador del invisible poder que rige tales fuerzas, no me había impresionado más que mientras duró, y lo olvidé casi en seguida. Tenía tan poca noción de Dios y su justicia, olvidaba a tal punto que mi miserable condición podía ser obra de Su mano, que se hubiera creído que estaba viviendo en la prosperidad. Pero cuando enfermé y los temores de la muerte se presentaron a mis ojos; cuando mis ánimos cedieron ante la fuerza de tan grave mal y mi resistencia se agotó por la fiebre, la conciencia tanto tiempo dormida empezó a despertarse y a hacerme reproches sobre mi pasada vida, por la cual había provocado a la justicia de Dios para que me abatiera con tan duros golpes, siendo mi empecinada maldad la causa de su severo castigo.
    «Ahora —dije en voz alta— van a cumplirse las palabras de mi querido padre. La justicia de Dios me ha fulminado y no tengo nadie que me ayude o me escuche. Rechacé la gracia de la Provincia que generosa me había colocado en una condición de vida donde habría tenido comodidad y calma; pero no fui capaz de verlo, ni siquiera a través de lo que me decían mis padres. Rehusé su ayuda y asistencia que me hubieran hecho adelantar en la vida, dándome todo lo que podía necesitar; y ahora me veo precisado a luchar contra fuerzas que la misma naturaleza no podría vencer, sin compañía, sin socorro, sin consuelo, sin consejo...» Y grité con todas mis fuerzas: « ¡Señor, ayúdame en mi aflicción!»
    Esta fue la primera plegaria, si así puedo llamarla, que elevaba al Cielo en muchos años. Pero vuelvo a mi diario.


    28 de junio — Algo aliviado por el profundo sueño, y encontrando que había pasado el acceso, conseguí levantarme todavía aterrado por el recuerdo de lo que había soñado; alcancé, sin embargo, a pensar que la calentura volvería al siguiente día y que ahora era momento de procurarme agua y alimentos que necesitaría después. Llené de agua una gran damajuana y la puse sobre la mesa, al alcance de mi lecho; para quitarle lo que pudiera causarme más fiebre mezclé en ella un cuarto de pinta de ron. Luego asé un pedazo de carne de cabra, pudiendo comer unos pocos bocados. Estaba tan débil que apenas pude moverme; me agobiaban la tristeza y el temor de que la calentura volviera al día siguiente. Por la noche cené tres huevos de tortuga que cocí enteros en las cenizas; y hasta donde alcanzo a recordar ésa fue la primera comida para la cual solicité la bendición de Dios.
    Quise caminar un poco después de la cena, pero apenas podía sostener la escopeta, que jamás abandonaba al salir; a poco de andar me senté en tierra, mirando hacia el mar que se extendía sereno a lo lejos. Y allí se me ocurrieron estos pensamientos: que todo cuanto me ocurría era por la voluntad de Dios; que había sido llevado a tan miserable situación por Su decisión, puesto que El tenía poder no sólo sobre mí sino sobre todo cuanto ocurría en el universo. De inmediato me pregunté: « ¿Por qué Dios ha hecho esto conmigo? ¿Cuál ha sido mi culpa para ser tratado así?»
    Mi conciencia me impidió seguir más adelante en tales interrogaciones, como si fueran blasfemias, y me pareció que hablaba dentro de mí una voz: « ¡Miserable!», decía. « ¿Preguntas lo que has hecho? ¿Por qué no miras tu vida malgastada y te preguntas más bien qué es lo que no has hecho? ¿Por qué no preguntas la razón de no haber perecido mucho antes, por qué no te ahogaste en la rada de Yarmouth o te mataron en la pelea cuando el pirata de Sallee apresó tu barco? ¿Por qué las bestias salvajes no te devoraron en la costa africana, o te ahogaste aquí donde pereció toda la tripulación? ¿Y te atreves todavía a preguntar qué has hecho?»
    Quedé tan abatido por semejantes reproches que, sin encontrar una sola palabra que responder, me levanté triste y pensativo y volví a mi tienda como dispuesto a dormir; pero me sentía demasiado confundido para que el sueño me venciera, de modo que me dejé caer en la silla y encendí la lámpara, pues ya era casi de noche. El temor de que volviera la fiebre me asaltaba, y entonces recordé que los brasileños no toman otra medicina que su propio tabaco para cualquier clase de enfermedades; yo guardaba un pedazo en uno de los arcones, ya curado, y otros que todavía estaban verdes.
    Fui al arcón, sin duda guiado por el Cielo, ya que allí encontré a la vez remedio para el cuerpo y para el alma. Al abrirlo en busca del tabaco hallé los pocos libros que salvara del naufragio, y entre ellos una de las Biblias que antes mencionara y que hasta ese momento no había mirado por falta de tiempo y de inclinación. Tomándola, la traje juntamente con el tabaco a mi mesa.
    Ignoraba la manera de emplear el tabaco para curarme y ni siquiera estaba seguro de que me hiciera bien; pero con la idea de acertar en alguna forma me propuse tomarlo de distintos modos. Ante todo corté un pedazo para mascar, lo que me produjo gran embotamiento, ya que el tabaco era fuerte y yo no tenía el hábito. Puse otra porción en una cantidad, de ron para beberlo al acostarme; y finalmente, quemando algunas hojas sobre el fuego, me incliné sobre él y aspiré profundamente el humo, resistiendo lo más posible el calor y la sofocación.
    En los intervalos de este tratamiento había abierto mi Biblia y empezado a leer, pero tenía la cabeza demasiado mareada por el tabaco para seguir con atención la lectura; al abrir el libro al azar, las primeras palabras que vieron mis ojos fueron: «Invócame en los días de aflicción, y yo te libraré, y tú me alabarás.»
    Como era tarde y los efectos del tabaco se sentían con fuerza, noté que el sueño me vencía, de manera que dejando encendida la lámpara en la cueva por si necesitaba algo durante la noche me fui a la cama. Pero antes hice lo que no había hecho nunca; me arrodillé para rogar a Dios que cumpliera su promesa de ayudarme si yo lo invocaba en los días de aflicción. Cuando hube terminado mi torpe y simple plegaria, bebí el ron donde había puesto tabaco; la bebida era tan fuerte y su gusto tan desagradable que apenas pude tragarla y caí en el lecho. De inmediato noté que la poción me mareaba, y dormí tan profundo sueño que no desperté hasta las tres de la tarde del siguiente día. Incluso llegué a pensar más adelante que en realidad había dormido todo ese día y la noche, hasta la tarde del tercero, porque de otro modo no me explico cómo pude saltar un día en la cuenta que llevaba, error que descubrí años "después. Sin duda, de haber hecho mal las líneas o puesto una sobre otra habría perdido más de un día. En fin, me faltó uno en la cuenta y nunca supe cómo.
    Al despertarme me sentía otro hombre, aliviado y con el espíritu animoso. Noté al levantarme que estaba mucho más fuerte que el día anterior, tenía apetito y la fiebre no mostraba señales de volver. En suma, no tuve acceso al día siguiente y continuó la mejoría; esto era el 29 del mes.
    El 30 estaba ya restablecido y me animé a salir con la escopeta aunque sin alejarme mucho. Maté una o dos aves marinas parecidas a los ánades y las traje a casa aunque sin mucha disposición para comerlas, de manera que me sustenté con algunos huevos de tortuga, que eran excelentes. Por la noche renové la medicina que imaginaba me había hecho tanto bien, o sea la infusión de tabaco con ron, pero bebí una dosis más pequeña, sin masticar aparte tabaco ni aspirar el humo. Al día siguiente ya no me sentía tan bien como esperaba, porque tuve algunos escalofríos de fiebre, aunque en pequeña escala.
    4 de julio — Por la mañana abrí la Biblia empezando a leer el Nuevo Testamento, dispuesto a hacer lo mismo todas las mañanas y las noches, sin proponerme un determinado número de capítulos sino llegar cada vez hasta donde mis pensamientos fueran claros.
    Empecé a interpretar el pasaje ya mencionado —«Invócame y te libraré»— en un sentido distinto del que antes le diera; porque hasta ese momento mi concepto de la liberación se refería únicamente al cautiverio en que me hallaba. Cierto que vivía libre en una isla, pero para mí era una cárcel en el más duro sentido de la palabra. Ahora principié a imaginar otro modo de libertad, al contemplar con horror mi pasada vida, y mis pecados surgieron tan terriblemente ante mí que mi alma sólo ansiaba de Dios liberación de ese insoportable peso de culpas que la privaba de toda alegría.
    Pronto mi espíritu se sintió más aligerado, aunque las condiciones de mi vida fueran las mismas de antes; llevados mis pensamientos por la lectura de la Biblia y la oración a regiones que jamás habían alcanzado en su vuelo, un profundo alivio fue surgiendo en mí como jamás lo conociera anteriormente. Y pues al mismo tiempo recobraba la salud y mis fuerzas volvían, me consagré a procurarme todas las cosas necesarias, haciendo a la vez una vida tan regular como fuera posible.
    Del 4 al 14 de julio di pequeños paseos con mi escopeta sin alejarme mucho, como convenía a un hombre que recobra lentamente las energías después de una enfermedad; difícil es imaginar lo débil que me sentía al comienzo. El tratamiento que he descrito más arriba era ciertamente original y acaso nunca habría curado antes una calentura; es por eso que no recomiendo a nadie que lo intente a su turno. Evidentemente me libró de la fiebre, pero acaso contribuyó a debilitarme tanto, pues incluso sufrí cierto tiempo de convulsiones nerviosas y musculares.
    Llevaba ya más de diez meses en tan triste isla, y toda posibilidad de rescate parecía imposible; yo estaba firmemente convencido de que jamás un pie humano había pisado antes ese sitio. Fue entonces cuando, después de terminar mi vivienda del modo que me pareció más adecuado, se me ocurrió hacer una exploración completa de la isla para descubrir aquellos productos naturales que me resultaran útiles.
    Desde el 15 de julio principié a recorrer la isla con tal fin, yendo ante todo a la ensenada donde, como he contado, mis balsas fondearon con su cargamento. Descubrí que a dos millas corriente arriba la marea ya no alcanzaba a penetrar y sólo había un arroyuelo de aguas límpidas y frescas; pero como estábamos en la estación seca apenas traja aguas para formar una corriente.
    A la orilla de este riacho encontré hermosas sabanas, vastas llanuras cubiertas de verdes pastos; en las partes más elevadas, ya cerca de las mesetas donde se hubiera supuesto que jamás alcanzaba el agua, hallé una gran cantidad de tabaco que crecía vigorosamente, así como otras diversas plantas desconocidas para mí, que acaso fueran de gran utilidad, aunque no podía aprovecharlas por mi ignorancia.
    Busqué entre ellas alguna raíz de cazabe o yuca, con la cual los indios de aquellas latitudes fabrican su pan, pero no vi ninguna. Había grandes plantas de áloe, cuya utilidad desconocía entonces, y mucha caña de azúcar en estado silvestre y, por lo tanto, poco aprovechable. Me contenté ese día con tales descubrimientos y volví meditando cómo podría arreglármelas para conocer las virtudes de las plantas y frutos que iba descubriendo. Desgraciadamente no arribé a ninguna conclusión, porque tan poco observador había sido mientras viví en el Brasil que no sabía nada de sus productos naturales, o tan poco que apenas podía ayudarme en mi presente desgracia.
    Al día siguiente," 6 de julio, seguí el mismo camino y avanzando más allá encontré que el arroyo y las sabanas se iban perdiendo y que la región era más boscosa. Hallé diferentes frutos, en especial melones en abundancia y uvas entre los árboles. Las viñas habían crecido entremezcladas en los árboles, y magníficos racimos ya maduros pendían de las ramas. Tan extraordinario descubrimiento me llenó de alegría, pero tuve cuidado de no excederme en la cantidad de uvas que comía porque recordaba lo sucedido en Berbería, donde muchos ingleses esclavos perecieron a causa de las fiebres y disentería que les produjo comer demasiada cantidad de esta fruta. Se me ocurrió que la mejor manera de aprovecharlas era ponerlas a secar al sol para conservar las pasas que tan grato me resultaría comer en las épocas en que ya no hubiera uvas maduras en las viñas. Pasé allí la noche sin regresar a mi casa, cosa que ocurría por primera vez desde que estaba en la isla. Recordando mi anterior precaución, trepé a un árbol apenas oscureció y dormí perfectamente; por la mañana mi exploración me llevó cuatro millas más allá, según juzgué por el largo del valle que se extendía hacia el norte, con una cresta de colinas al sur y al norte de donde yo me hallaba.
    Al fin de este recorrido vine a dar a un espacio abierto, y allí el terreno parecía descender hacia el oeste; un arroyuelo que nacía en las laderas de la colina bajaba en dirección contraria, es decir, al este; la región eran tan fresca, tan fértil y florida, que al ver ese derroche de vegetación se la hubiera tomado por un jardín en primavera.
    Exploré un lado de aquel delicioso valle, observándolo todo con secreto placer en el cual se mezclaba sin embargo la aflicción, y pensando que aquello era mío. Podía considerarme dueño y señor de esas tierras, con derechos incontestables, incluso el de legarlas si me parecía bien, al igual que cualquier lord de Inglaterra. Noté la abundancia de cocoteros, naranjos, limoneros y cidras, todos ellos silvestres y con muy pocos frutos. Las limas que recogí no sólo eran agradables de comer sino de buen tamaño, y desde entonces mezclé su jugo con agua y bebí con agrado ese refresco sano y tonificante.
    Tenía ya bastante que llevar a mi morada y proyectaba acumular suficientes uvas, limas y limones para que no faltaran en la próxima estación de las lluvias. Junté, pues, una cantidad considerable de uvas, otra más pequeña al lado y un montón de limas y limones en otro sitio; tomando luego una poca cantidad de cada clase me volvía a mi casa, resuelto a regresar con un saco o cosa parecida para llevarme el resto.
    El camino de vuelta insumió tres días, de modo que cuando llegué a casa (como debo llamar a mi tienda y mi cueva) las uvas se habían echado a perder, debido a su completa madurez y al peso de los racimos. Tuve, pues, que tirarlas sin probar ni una; las limas eran excelentes, pero por desgracia había traído muy pocas.
    Al día siguiente, 19 de julio, volví a ponerme en camino con dos pequeños sacos destinados a acarrear mi cosecha. Allá me esperaba la desagradable sorpresa de encontrar mis racimos de uva, tan perfectos y hermosos cuando los cortara, completamente estropeados, esparcidos por el suelo y en la mayor parte devorados. Comprendí que había animales salvajes en los alrededores, pero cuáles y cómo eran no pude imaginármelo.
    En la alternativa de juntar racimos para que se secaran al sol o llevármelos de inmediato en los sacos, y seguro de que en el primer caso los animales los devorarían y en el otro iban a aplastarse por su propio peso, tuve la siguiente idea: cortando gran cantidad de racimos los colgué en las ramas exteriores de los árboles para que se secaran al sol. Luego llené mis sacos con todas las limas y limones que eran capaces de contener y los traje conmigo.
    Cuando estuve de regreso de mi expedición, recordaba continuamente y con agrado la fertilidad de aquel valle y el excelente lugar en que estaba situado, a cubierto de tempestades, con agua dulce y bosques. No me faltó más para concluir que de los rincones de la isla, aquel donde yo tenía mi morada era desde todo punto de vista el peor. ¿Por qué no —pensé— cambiar de lugar, irme a un sitio tan seguro como el que ahora tenía, pero situado en aquella fértil y hermosa planicie de la isla?
    La idea me tentó durante mucho tiempo y no podía olvidar la belleza del valle, pero cuando lo pensé con más calma comprendí que iba a cometer un error. Me encontraba junto a la orilla del mar, donde había por lo menos una posibilidad de recibir algún socorro; la misma desgracia que me arrojara a la costa podía traer a otros infelices de la misma manera; y aunque tal cosa era improbable, si yo me alejaba e iba a encerrarme entre las colinas y los bosques del centro de la isla, me condenaba definitivamente a mi destino, tornando imposible lo que sólo habría sido improbable. Comprendí, pues, que era preciso quedarme donde vivía.
    Con todo, tan encantado estaba de aquellos lugares, que pasé en ellos buena parte de mi tiempo hasta final de julio, y aunque seguía firmemente decidido a no mudarme, levanté para mi comodidad una especie de enramada protegida a distancia por un fuerte vallado hecho de una doble hilera de altas estacas bien clavadas y con maleza en medio. Allí, sintiéndome seguro, solía quedarme dos o tres noches, empleando el mismo sistema de escalera para entrar y salir. Me complacía pensar que era dueño de una casa de campo, así como de otra junto al mar; y en aquellos trabajos pasé hasta principios de agosto.
    Había terminado mi empalizada y principiaba a cosechar los agradables frutos de ese trabajo cuando vinieron las lluvias obligándome a volver a mi primera morada, porque aunque en el valle tenía una tienda semejante a ésta, fabricada con una vela y bien tendida, me faltaba el abrigo de la colina para protegerme de tempestades, y la cueva para refugiarme cuando las lluvias fueran demasiado copiosas. He dicho que hacia principios de agosto terminé la enramada y principié a utilizarla. El 3 encontré que los racimos se habían secado perfectamente y eran ya muy buenas pasas, apresurándome a descolgarlos con el justo tiempo para impedir que las lluvias los estropearan privándome de lo mejor de mi alimentación invernal. Era dueño de más de doscientos magníficos racimos. Tuve el tiempo de llevarlos a mi cueva cuando se precipitaron las lluvias, y desde ese día, 14 de agosto, hasta mediados de octubre, llovió diariamente con más o menos fuerza y a veces con violencia tan extraordinaria que me obligaba a estarme en la cueva por espacio de muchos días.
    Durante aquel tiempo tuve la sorpresa de ver aumentar mi familia. Había lamentado mucho la pérdida de uno de mis gatos, al que suponía muerto o perdido, y no tuve la menor noticia de él hasta que para mi gran asombro apareció hacia fin de agosto acompañado de tres gatitos. Desgraciadamente con el tiempo se multiplicaron tanto que eran una calamidad, y me vi forzado a matarlos sin lástima o arrojarlos lejos de mi casa.
    Desde el 14 hasta el 26 de agosto llovió incesantemente, por lo que me cuidé de salir, temeroso de mojarme. Comenzó a faltar alimento en mi prisión, pero aventurándome a salir dos veces, la primera maté una cabra y la segunda, justamente el 26, encontré una enorme tortuga que fue un regalo para mí. Regulaba mis comidas en esta forma: un racimo de pasas de desayuno, como almuerzo un pedazo de carne de cabra o tortuga, asado, pues por desgracia no tenía olla para hervirlo o guisarlo, y dos o tres huevos de tortuga a modo de cena.
    Mientras estuve confinado por la lluvia trabajé varias horas diarias ensanchando mi cueva, cavando un túnel que desviaba poco a poco hacia un lado, hasta que vine a salir a la ladera de la colina y tuve una puerta que daba fuera de mi empalizada y me resultaba muy cómoda. Pero no me sentía con la tranquilidad de antes, porque hasta entonces mi morada había sido un recinto completamente cerrado, mientras que ahora cualquiera podía entrar por aquella puerta. No comprendía aún que el mío era un temor infundado, ya que el animal de mayor tamaño en la isla era la cabra.


    30 de setiembre — Llegó finalmente el triste aniversario de mi naufragio. Conté las marcas en el poste y vi que llevaba en la isla trescientos sesenta y cinco días. Consideré ese día como de ayuno, y lo dediqué a meditaciones religiosas.
    Hasta entonces no había observado nunca el domingo, ya que olvidándome de señalarlos con una línea más grande, perdí la cuenta de cuáles eran los días.
    Al sumarlos y comprobar que llevaba un año en tierra, dividí ese tiempo en semanas y consideré domingo el séptimo día de cada una, bien que hacia el final de mis cálculos di con un error de uno o dos días.
    Poco después de esto noté que la tinta escaseaba, de modo que preferí emplearla con cuidado y sólo escribir los acontecimientos dignos de mención, sin llevar una prolija crónica de cada cosa.
    La estación de las lluvias y la de sequía se alternaban regularmente y me habitué a dividirlas para tomar las necesarias precauciones.
    Esto lo aprendí a costa de una de las experiencias más desalentadoras que puedan imaginarse. Ya he dicho que había recogido las pocas espigas de cebada y arroz que tan maravillosamente crecieran sin el menor cuidado y al azar; había unas treinta espigas de arroz y veinte de cebada, que consideré conveniente sembrar una vez pasadas las lluvias, cuando el sol se aleja del cenit declinando hacia el sur.
    Cavé lo mejor que pude con mi azadón de madera un buen trozo de tierra, y dividiéndolo en dos partes planté el grano; mientras lo estaba haciendo se me ocurrió que no convenía sembrar todo porque acaso no era el momento propicio, de manera que planté sólo dos terceras partes de cada clase, conservando cuidadosamente el puñado restante.
    Pronto tuve razones para alegrarme de mi prudencia porque no germinó ni un solo grano de los sembrados, pues sucediendo el período de sequía al de las lluvias aquella tierra no recibía suficiente humedad; sin embargo, cuando cambió la estación, vi nacer los cereales como si acabara de plantarlos.
    Al advertir que la semilla no germinaba por causa de la sequía, busqué un terreno más húmedo para intentar otra siembra, y hallándolo cerca de mi nueva enramada lo cavé y en el mes de febrero, poco antes del equinoccio de primavera, le confié el resto de mi semilla. Con las lluvias de marzo y abril germinó muy bien, dándome excelente cosecha, pero como por temor a que también esos granos se perdieran había puesto aparte otra pequeña reserva, el producto fue muy escaso, apenas medio celemín de cada clase. Me bastaba, sin embargo, para saber exactamente cuándo debía sembrar el grano, y la posibilidad de obtener fácilmente dos cosechas anuales.
    Mientras crecían mis sembrados hice un pequeño descubrimiento que después me fue harto útil. Tan pronto cesaron las lluvias y el tiempo se estabilizó —cosa que ocurría en noviembre— fui a visitar mi enramada, de la que faltaba desde hacía varios meses, encontrando todo en orden y tal como lo dejara al abandonarla. El círculo o doble empalizada no sólo estaba tan firme como antes sino que las estacas que cortara de algunos árboles de las inmediaciones habían echado raíces y ramas, iguales a las que produce el sauce al primer año de haber sido podado. Ignoraba el nombre de aquellos árboles que me habían provisto de tales estacas y me sorprendí grandemente, pero a la vez me alegró ver crecer los arbolillos, y traté de que todos se desarrollaran igualmente, podándolos lo mejor posible. Es difícil describir lo hermosos que se pusieron al cabo de tres años, tanto que a pesar de tener el círculo un diámetro de casi veinticuatro yardas, las copas se encontraban en su centro y daban una espesa sombra, bastante para abrigarme durante toda la estación de sequía.
    Esto me decidió a cortar más estacas de la misma clase y formar con ellas un semicírculo arbolado en torno a mi empalizada, me refiero a la de mi primera residencia, cosa que hice en una doble hilera situada a ocho yardas de la antigua empalizada. Pronto crecieron tanto y cubrieron tan bien mi morada que hasta me sirvieron más adelante de defensa, como se verá.





    7. VIAJES Y TRABAJOS




    Había yo observado que las estaciones del año no se dividían como en Europa en invierno y verano, sino en estación seca y lluviosa. Luego de experimentar en carne propia los inconvenientes de las lluvias, tuve buen cuidado de proveerme por adelantado de lo más necesario a fin de no tener que salir para nada, y durante los meses de lluvia hacía todo lo posible por quedarme a cubierto.
    No estaba sin embargo ocioso mientras duraba mi encierro, efectuando toda clase de trabajos aplicables a esa circunstancia, tales como diversos objetos necesarios que sólo con gran paciencia y dedicación podían ser fabricados. Intenté muchas veces tejer un canasto, pero los mimbres que a tal efecto ensayaba eran tan quebradizos que de nada servían. Fue entonces que me resultó de gran utilidad el haber observado siendo joven a un cestero de mi pueblo natal, siguiendo con atención su modo de tejer el mimbre; como todo muchacho dispuesto a ayudar y lleno de curiosidad por la forma en que se fabricaban aquellos cestos, y a veces participando en la tarea, llegué a conocer bastante bien los procedimientos usuales, faltándome ahora sólo el material suficiente. Se me ocurrió que acaso los tallos de aquel árbol del que había sacado las estacas que prendían fueran tan resistentes como los del sauce o mimbre, y me propuse averiguarlo.
    Al día siguiente fui a mi casa de campo, como me agradaba llamarla, y cortando algunos de los tallos más tiernos descubrí que se adaptaban admirablemente a mi propósito; volví, pues, la vez siguiente armado de una hachuela para cortar gran cantidad, lo que era fácil por la abundancia de árboles. Los puse a secar dentro del vallado, y cuando estuvieron listos los traje a la cueva; allí, durante la estación de las lluvias me entretuve en fabricar toda clase de canastos tanto para acarrear tierra como para poner en ellos distintas cosas. Cierto que no estaban muy bien terminados, pero servían pasablemente para lo que yo los destinaba. Desde entonces me preocupé de que no faltaran, y a medida que los veía estropearse con el uso los iba reemplazando con otros mejores, en especial unos grandes cestos que hice para depositar el grano de la cosecha en vez de meterlos en sacos.
    Superada aquella dificultad y puesto mucho tiempo en lograrlo, empecé a buscar el modo de suplir dos grandes necesidades. No tenía vasijas para líquidos a excepción de dos barrilitos llenos de ron y algunas botellas, ya de tamaño común o bien las cuadradas que se emplean en guardar licores y bebidas. Carecía de ollas para guisar o hervir alimentos, salvo una enorme marmita que salvé del naufragio y que era demasiado grande para hacer en ella caldo o guisar un trozo de carne. Y la segunda cosa que deseaba intensamente era una pipa. Pero no hallaba la manera de fabricarme una hasta que al fin pude dar con el procedimiento.
    Me ocupé en plantar la segunda hilera de estacas y tejer cestas durante toda la estación seca, cuando una nueva tarea se presentó para demandarme mucho más tiempo del que imaginaba dedicarle. He dicho que sentía el deseo de explorar íntegramente la isla, y que llegando hasta el arroyuelo y siguiendo aguas arriba había desembocado en el valle donde ahora estaba mi enramada, y desde el cual podía verse un paso que conducía a la costa opuesta y al mar. Resolví franquear esa distancia que aún me faltaba conocer, y llevando una hachuela, la escopeta y mi perro, así como suficiente cantidad de pólvora y balas, provisto de dos galletas y un gran racimo de pasas para comer en camino, empecé la jornada. Luego de atravesar el valle donde estaba mi casa, llegué por el oeste a la vista del mar, y como era un día excepcionalmente diáfano pude ver tierra a lo lejos, aunque sin distinguir si se trataba de un continente o una isla. Era una tierra muy alta, extendiéndose del oeste al O-SO a una enorme distancia que, según mis cálculos, no podía ser menos de quince o veinte leguas.
    Vi abundancia de papagayos, y me hubiera gustado apresar uno vivo para domesticarlo y enseñarle a que me dirigiera la palabra. Con gran trabajo alcancé a darle con el bastón a uno muy joven, aturdiéndolo; pero una vez en casa pasaron varios años antes de que aprendiera a hablar. Por fin supo llamarme con mucha familiaridad por mi nombre, y el episodio a que esto dio lugar, aunque sea una insignificancia, divertirá cuando sea narrado en su debido momento.
    Lo pasé muy bien en aquel viaje. En las tierras bajas había animales parecidos a liebres y zorros, pero tan distintos de las otras clases que ya conocía en la isla que no me animé a probar su carne aunque había matado unos cuantos. ¿Para qué arriesgarme si tenía suficiente comida y de la mejor, tal como la carne de cabra, pichones de paloma y tortugas? Sumando mis pasas, el mismo mercado de Leadenhall no hubiera podido abastecer tan bien una mesa en proporción al número de comensales. Aunque mi situación era deplorable, tenía razones para estar agradecido y yo no padecía necesidades sino que hasta me sobraban las cosas. Nunca hice más de dos millas en línea recta mientras cumplía este viaje, pero eran tantas mis vueltas y revueltas en procura de nuevos descubrimientos que llegaba rendido hasta el sitio que elegía por campamento nocturno; allí trepaba a un árbol o me protegía rodeándome con un círculo de estacas que clavaba en el suelo —y a veces tendía de árbol a árbol— para estar seguro de que ningún animal salvaje se acercaría sin despertarme antes. Tan pronto arribé a la orilla del océano tuve la sorpresa de comprobar que había elegido el peor lado de la isla para vivir, ya que aquí la playa estaba cubierta por innumerables tortugas, mientras que en la costa opuesta sólo había visto tres en año y medio. También descubrí infinidad de aves de toda clase, muchas que ya había encontrado y otras desconocidas. La mayor parte tenía una carne exquisita, pero ignoraba su nombre, salvo el de los pájaros llamados pingüinos.
    Era fácil matar gran cantidad de estos animales, mas me interesaba economizar todo lo posible la pólvora y el plomo, por lo cual preferí dedicarlos a las cabras que me daban alimento más duradero; vi aquí todavía más cabras que del lado donde vivía, pero la dificultad para cazarlas era mayor por la regularidad del terreno que les permitía divisarme desde muy lejos.
    Confieso que este lado de la isla me pareció harto más bueno que el otro, pero no me sentí movido a cambiar de vivienda. Ya me había habituado a mi casa, me parecía algo natural y propio, tanto que ahora en todo momento sentía la impresión de hallarme de viaje, y proveniente de mi casa. Seguí la costa marina hacia el este, unas doce millas según presumo, y después de clavar un palo a modo de señal me decidí a regresar a casa, quedando resuelto que en la próxima exploración saldría de ella en dirección al otro lado de la isla costeándola hasta dar con el palo que a propósito dejaba.
    Para volver elegí otro camino, pensando que me sería fácil recordar la topografía de la isla y que era más agradable regresar viendo cosas nuevas. Pero pronto me encontré perdido, erré de un sitio a otro y por fin tuve que volver a la costa buscando la señal, y enderezar hacia el camino ya andado. Regresé haciendo etapas muy cortas porque el calor era excesivo y cuanto yo llevaba —escopeta, municiones, hacha— resultaba muy pesado.
    En esos días sorprendió mi perro a un cabrito y, saltándole encima, me dio tiempo a que llegara corriendo y lo salvara de sus colmillos. Tuve otra vez la idea de llevarlo a casa y me pregunté si no sería posible domesticar uno o dos cabritos a fin de irme procurando un hato que supliera mi falta de alimentos cuando no tuviese más pólvoras ni balas.
    Tejí un collar para el animalito y atándolo con una cuerda que siempre llevaba conmigo lo arrastré, aunque no sin dificultades, hasta mi enramada, donde lo dejé encerrado porque me sentía impaciente por arribar a mi casa, de la que faltaba desde hacía casi un mes.
    No puedo expresar con cuánta satisfacción penetré en la vieja tienda y me dejé caer en la hamaca. Aquella exploración, sin lugar fijo de residencia, me había resultado tan poco grata que mi propia casa —como me gustaba llamarla— era una morada perfecta comparada a lo anterior. Tan confortable me parecía tener mis cosas a mi alrededor que prometí no alejarme nunca más tan lejos mientras mi suerte me tuviera encadenado a aquella isla.
    Descansé una semana de las fatigas del viaje, entreteniéndome en la importante tarea de fabricar una jaula para mi papagayo, que se domesticaba rápidamente. Me acordé luego del pobre cabrito que dejara encerrado en la enramada, y me apresuré a ir en su busca o por lo menos a llevarle alimentos. Estaba donde lo había dejado, ya que le era imposible escaparse, pero medio muerto de hambre. Cortando follaje de árbol y ramas de arbustos tiernos se los di a comer y después que se hubo satisfecho lo até para llevarlo a casa, pero se había amansado tanto con el hambre que no era necesaria esta precaución, pues me seguía como un perro. Como continuara alimentándolo, el cabrito se volvió tan dócil y tan cariñoso que desde entonces permaneció conmigo y formó parte de mi familia sin abandonarme jamás.
    Venía la estación de las lluvias del equinoccio otoñal, y celebré el 30 de setiembre de la misma solemne manera. Se cumplía el segundo aniversario de mi arribo a aquellas tierras y en todo ese tiempo no había tenido la menor posibilidad de ser rescatado. Pasé el día en humilde y reconocido agradecimiento de los muchos y admirables beneficios que aliviaran mi desgracia, y sin los cuales hubiera sido infinitamente miserable.
    Fue entonces cuando empecé a sentir claramente cuánto más feliz era esta vida, con todos sus rigores, que la perversa, maldita y abominable existencia en que había dejado deslizarse mis años pasados. Tanto mis alegrías como tristezas eran muy distintas de las antiguas; mis deseos cambiaron, así como mis afectos, y la alegría que ahora era capaz de experimentar tenía razones totalmente opuestas a las que sentía a mi llegada a la isla o en los dos años que acababan de cumplirse.
    En tal disposición de espíritu principié mi tercer año de soledad, y aunque no fatigaré al lector con la detallada crónica de mis trabajos en este período debo decirle que muy pocas veces estuve ocioso ya que dividí regularmente mi tiempo de acuerdo con las tareas que debía efectuar cotidianamente. Estas eran, ante todo, mis deberes para con Dios y la lectura de la Biblia, a la que dedicaba un rato tres veces al día; luego salía de caza, lo que me llevaba unas tres horas por la mañana salvo que lloviera; tercero, me ocupaba en preparar y cocer la carne así obtenida. En esta forma se iba buena parte del día, sin contar que hacia las doce, cuando el sol estaba en el cenit, el calor era tan intenso que no se podía salir, de manera que recién reanudaba el trabajo a eso de las cuatro; a veces, como único cambio en este orden de vida, alteraba las horas de caza y de labor, haciendo ésta de mañana y cazando al atardecer.
    Al tiempo empleado en el trabajo es preciso agregar la extraordinaria dificultad de cada tarea y las muchas horas que, por falta de herramientas, ayuda y habilidad, me llevaba cada cosa que hacía. Por ejemplo, pasé cuarenta y dos días para hacer un tablón que me sirviera de estante en la cueva, mientras que dos serradores con herramientas apropiadas hubieran cortado seis tablones del mismo árbol en medio día.
    El trabajo era el siguiente: buscaba ante todo un árbol lo bastante grande para sacar un tablón ancho. Tres días se iban hachando el árbol, más dos para quitarle el follaje y reducirlo a una sola pieza de madera. A fuerza de hachazos y tajos lo iba rebajando de ambos lados hasta que el menor peso me permitía moverlo; lo daba vuelta y trataba de alisar un lado para que quedase como el tablón que quería, y volviéndolo a cambiar de posición alisaba el lado opuesto hasta conseguir la plancha deseada, de unas tres pulgadas de espesor y bien nivelada. Cualquiera puede imaginar lo que significa una labor semejante, pero la paciencia y el trabajo me permitían al fin lograr mi propósito. Si he traído este ejemplo ha sido para mostrar cómo una sola tarea podía insumir tal cantidad de tiempo, y que algo tan simple de hacer con herramientas adecuadas y alguna ayuda se convertía en una empresa harto compleja y reclamaba largo tiempo a un hombre solo y sin más instrumentos que sus manos. Pese a todo, con paciencia y perseverancia conseguí triunfar en muchas empresas y en todo cuanto me era necesario en tales circunstancias, como se verá en el siguiente relato.
    Venían noviembre y diciembre, y esperaba yo mi cosecha de cebada y arroz. El terreno donde los sembrara no era muy grande, pues ya he dicho que sólo tenía medio celemín de cada semilla, habiendo perdido el resto por sembrarlo en la estación seca. Mi cosecha prometía ser excelente cuando de improviso me encontré en peligro de volver a perderla por causa de algunos enemigos difíciles de combatir. En primer término las cabras y aquellos animales que yo llamaba liebres, los cuales habiendo advertido que los tallos eran tiernos venían noche y día a comerlos, estropeándolos de tal modo que temí que no echaran espiga.
    Para esto no hallé más remedio que fabricar un vallado en torno a mi terreno, lo que me dio un gran trabajo, ya que además de su extensión era necesario apresurarse. Por suerte mi tierra arada no era mucha, y en tres semanas de continua labor la cerqué completamente; matando algunos de esos animales durante el día y atando el perro a una estaca por la noche, para que ladrara cuando alguno se acercaba, conseguí que los enemigos abandonaran el sitio. El grano creció fuerte y sano y lo vi madurar admirablemente.
    Pero así como los animales mencionados estuvieron a punto de estropear mi cosecha cuando aún no levantaba del suelo, ahora fueron los pájaros quienes acudieron para hacerlo cuando surgieron las espigas. Yendo a visitar mi tierra la encontré rodeada de aves que no parecían esperar otra cosa sino que me fuera de allí. De inmediato les disparé un tiro, ya que jamás me separaba de mi escopeta, y apenas sonó el disparo vi levantarse una nube de pájaros que hasta entonces no había sospechado y que estaban posados entre el grano. Esto me preocupó seriamente, porque preví que en pocos días aquellas aves devorarían mis espigas y yo estaría condenado a perderlo todo y pasar hambre.
    Decidí hacer cuanto pudiera para impedir el daño, aunque tuviese que estar noche y día de centinela. Ante todo entré en el sembrado para verificar los daños causados, y en verdad que los pájaros habían destruido buena parte de las espigas; pero como había muchas aún demasiado verdes para su gusto, podía confiar en que el resto de la cosecha me compensaría.
    Me detuve a cargar la escopeta, y alejándome un poco observé que los ladrones estaban posados en los árboles circundantes como a la espera de que me marchara. Y así fue, pues apenas hube andado un poco más fingiendo no tener intención de volver, los vi precipitarse uno a uno sobre el sembrado. Tanto me indignó esto que no tuve paciencia para esperar a que los otros hicieran lo mismo, ya que me parecía que cada grano que devoraban equivalía un pan entero en el futuro. Volviendo, pues, hacia el sembrado, disparé sobre ellos consiguiendo matar tres. Esto era lo que deseaba, para emplear las víctimas tal como lo hacemos en Inglaterra con los bandoleros a quienes se deja colgados del cadalso a fin de que sirvan de escarmiento a los demás.
    Es casi imposible imaginar que mi procedimiento tuviera buen éxito, pero no sólo los pájaros cesaron por completo de volar sobre mi sembrado sino que huyeron de esa parte de la isla y jamás volví a verlos por ese lado mientras los espantapájaros colgaron allí.
    Esto me alegró sobremanera, y hacia la última mitad de diciembre, época de la segunda recolección anual, coseché el grano.
    Mi problema era fabricar una hoz o guadaña a tal efecto, y lo resolví bien que mal con ayuda de uno de los sables que trajera del barco. Como la primera cosecha era harto menguada no me costó mucho su recolección y la hice a mi modo, cortando solamente las espigas, que iba poniendo en un gran cesto y desgranaba luego entre mis manos. Cuando hube terminado encontré que aquellos medios celemines se habían convertido en casi dos fanegas de arroz y dos y media de cebada, según calculé aproximadamente, ya que me faltaba con qué medirlos.
    Esto me llenó de entusiasmo y llegué a imaginar que con el tiempo Dios me concedería tener pan. Algo sin embargo me preocupaba, porque no sabía cómo moler el grano para hacer harina, ni siquiera limpiarlo y cernirlo. Luego, aunque obtuviera la harina, ¿cómo arreglármelas para hacer pan si no tenía horno? Agregándose tales cosas al deseo que experimentaba de guardar suficiente cantidad de grano en depósito y asegurarme la subsistencia en lo venidero, resolví reservar aquella cosecha para sembrarla entera en la siguiente estación, y dedicar entretanto todo mi ingenio y mis horas de trabajo a la fabricación de lo indispensable para que mi sueño de tener pan se cumpliera al fin.
    Mientras llovía y yo estaba encerrado en la cueva, hallé oportunidad de dedicarme a esas tareas y a la vez me entretenía mucho hablando a mi papagayo para enseñarle a hacer lo mismo. Pronto aprendió el nombre que le pusiera y por fin dijo en alta voz: «Poli». Aquélla fue la primera palabra pronunciada en la isla por otra boca que la mía. Mi tarea no consistía sin embargo en eso, que era sólo diversión; tenía por delante un complicado trabajo que venía meditando desde tiempo atrás. Se trataba de ver si era posible fabricar vasijas de barro, tan necesarias para mí. Pensé que dada la elevada temperatura de aquel clima bastaría encontrar una arcilla conveniente y moldear con ella los recipientes, los que pronto se secarían al sol con dureza bastante para resistir el manejo y contener aquellas cosas que requerían un sitio seco y seguro; igualmente pensé que iban a ser útiles para guardar el grano y la harina que yo proyectaba acumular en mi casa. De modo que busqué la arcilla y me puse a fabricar recipientes a manera de grandes tinajas que almacenaran mis productos.
    El lector se apiadaría de mí, o acaso le produjeran risa los raros procedimientos que puse en práctica para dar forma a aquella pasta, las grotescas, deformes y feísimas vasijas que hice. ¡Cuántas se aplastaban o se abrían porque la arcilla no era bastante espesa para resistir su propio peso! ¡Cuántas estallaban a causa del violento calor del sol, demasiado pronto expuestas a sus rayos! ¡Cuántas se rompían de sólo tocarlas, antes o después de secas! En fin, luego de trabajar duramente para descubrir buena arcilla, extraerla en cantidad, mezclarla antes de llevarla a casa y moldear los recipientes, al cabo de dos meses de trabajo sólo conseguí fabricar dos grandes y desgarbadas cosas que no me atrevo a llamar tinajas.
    Habiéndolas secado muy bien al sol las levanté con todo cuidado y las puse dentro de unos cestos que a tal propósito había tejido, temeroso de que se rompieran en otra forma. Como entre los cestos y las vasijas quedaba un espacio hueco, lo rellené con la paja de la cebada y el arroz y notando que estaban perfectamente secas creí que serían depósito adecuado para el grano y, tal vez, la harina, una vez molido aquél.
    Aunque fracasé casi completamente en mi intención de fabricar grandes recipientes, con los más pequeños no me fue tan mal; hice potecitos, platos playos, cántaros y todo lo que se me ocurría; el calor del sol los coció bastante bien. Sin embargo nada de esto me consolaba de no poder tener una olla que soportara el fuego para cocer los alimentos, pues aquellos cacharros no me servían. Tiempo después, habiendo encendido un gran fuego para asar mi comida encontré al apagarlo un pedazo de cacharro que inadvertidamente había quedado entre las llamas, perfectamente cocido, duro como una piedra y del color de una teja. Me llenó de alegría descubrir tal cosa, y me dije que si los cacharros se cocían estando rotos igualmente lo harían enteros. El problema estaba en cómo disponer el fuego para tal fin. No tenía idea alguna de lo que era un horno de alfarero, ni del barniz de plomo que se pone a los cacharros, aunque disponía del metal suficiente. Puse tres grandes pucheros y dos o tres potes uno sobre otro, y encendí fuego a su alrededor cuidando de que gran número de brasas estuvieran colocadas directamente debajo de la pila. Renovaba continuamente el combustible, tratando de mantener avivado el fuego hasta que los cacharros empezaron a ponerse al rojo vivo sin que ninguno se quebrara. Cuando los vi así, sostuve el fuego por cinco o seis horas más, hasta que de pronto uno de los cacharros empezó a fundirse debido al excesivo calor que derretía la arena que yo mezclara con la arcilla. De haber continuado habría visto convertirse en vidrio aquella pasta, pero disminuí gradualmente el fuego hasta que los cacharros perdieron su rojo vivo, y quedándome despierto toda la noche para que el fuego no disminuyera demasiado bruscamente, me encontré de mañana dueño de tres excelentes —si no bellos— pucheros, así como de dos potes, todos ellos cocidos como pudiera desearse, y uno de ellos bonitamente barnizado por la arena derretida.
    Después de tal experimento está de más decir que tuve todos los cacharros necesarios. Su único defecto era la forma irregular y tosca, ya que carecía de medios para hacerlos mejor y trabajaba como los niños cuando hacen pasteles de barro o una cocinera que sin saber amasar quisiera hacer una tarta. Pocas veces hubo alegría tan desproporcionada a la insignificancia de su objeto como la que yo sentí al descubrir el modo de fabricar una olla que resistiera el calor del fuego. Tuve que contener mi impaciencia mientras se enfriaba, y apenas estuvo lista volví a ponerla al fuego con agua, en al que herví carne, viendo con júbilo que la olla resistía perfectamente la prueba. Con un trozo de carne de cabrito hice caldo, aunque me faltaba harina de avena y demás ingredientes necesarios que le dieran el debido sabor.
    Mi inmediata tarea fue fabricar una especie de mortero para moler el grano, ya que construir un molino era tarea inaccesible para un simple par de manos. Anduve varios días buscando una piedra lo bastante grande para excavarla en el centro y darle forma de mortero, mas no encontré ninguna salvo las rocas, cuya dureza impedía todo intento. Las piedras sueltas de la isla eran de una sustancia arenosa que se disgregaba fácilmente, y no hubieran resistido el golpe de otra piedra o llenado de arena el grano molido. Después de buscar inútilmente durante mucho tiempo resolví abandonar la tarea y elegir en cambio un trozo de madera suficientemente dura, cosa que me fue muy fácil. Llevando a casa el pedazo más grande que pude mover, lo redondeé exteriormente con ayuda del hacha, y luego por medio del fuego —aunque con infinito trabajo— pude vaciarlo interiormente a la manera como los indios del Brasil fabrican sus canoas. De un pedazo de palo de hierro hice la mano del mortero, y así equipado me dispuse a esperar la próxima cosecha cuyo grano había decidido moler —o más bien machacar— para hacer pan.
    Otra dificultad era la de procurarme un cedazo o tamiz para cerner la harina y separarla del salvado, sin lo cual me parecía imposible obtener el pan. Esto resultó lo más difícil de todo ya que carecía de lo indispensable para construirlo, es decir, una tela de trama bastante abierta para tamizar la harina. Tal cosa me detuvo durante muchos meses, y al final estaba enteramente desorientado; tenía algo de género de hilo, pero reducido a andrajos; también guardaba pelo de cabra, pero ¿cómo hilarlo y tejerlo si ignoraba el procedimiento y aun habiéndolo sabido carecía de todo instrumento adecuado? Por fin encontré una solución transitoria al recordar que entre las ropas de marinero que salvara del barco había algunas corbatas de zaraza o muselina, y con sus pedazos pude hacer tres pequeños tamices que me prestaron excelente servicio durante muchos años. Más adelante habré de narrar cómo los reemplacé.
    El problema de la cocción venía en seguida. ¿Podría hacer pan una vez que tuviera harina? Ante todo me faltaba levadura, y aunque esto último no me preocupa grandemente me afligía la carencia de horno adecuado. Por fin inventé un procedimiento que consistió ante todo en fabricar vasijas de arcilla, muy anchas pero no profundas, es decir, de unos dos pies de diámetro y apenas nueve pulgadas de hondo; las cocí en el fuego al igual que las anteriores, y las puse aparte. Luego, cuando deseaba hornear, encendía una gran hoguera sobre mi fogón, que estaba recubierto de tejas cuadradas —si puedo darles ese nombre— que yo mismo fabricara. Una vez que el fuego se había reducido a brasas, las disponía de modo que cubrieran enteramente el fogón y las dejaba hasta que lo hubiera recalentado; sacando luego las brasas ponía mis panes sobre las tejas y cubriéndolos con las vasijas mencionadas los rodeaba por fuera con brasas que mantuvieran el calor. Así, como en el mejor horno del mundo, mi pan de cebada cocía maravillosamente y pronto fui un excelente pastelero, ya que me animé a hornear en la forma descrita varias tortas de arroz y budines; no pude hacer pasteles porque no tenía nada con que rellenarlos, salvo carne de aves o de cabra.
    No habrá de causar asombro el que estas tareas se llevaran la mayor parte de mi tercer año en la isla, ya que además de ellas tenía en los intervalos que ocuparme en mi nueva cosecha y la labranza. Hice a su debido tiempo la recolección del grano y lo llevé a casa como pude, guardando las espigas en las grandes tinajas hasta tener tiempo para desgranarlas a mano, pues carecía de lugar para trillarlas así como de los necesarios instrumentos.
    Como mi provisión de cereales iba en aumento, empecé a ver la necesidad de construir mayores graneros. Quería un sitio donde tenerlos bien guardados, porque la cosecha había sido tan buena que me dio cerca de veinte fanegas de cebada y otro tanto de arroz, de modo que me resolví a emplearlos sin hacer economía. Mi provisión de pan se había agotado y debía renovarla; además quise calcular qué cantidad de semilla iba a bastarme para todo un año, a fin de sembrar anualmente una sola vez.
    Llegué a calcular que las cuarenta fanegas de arroz y cebada excedían en mucho a lo que podía gastar en un año, y por tanto me propuse sembrar cada vez una cantidad igual a la de mi última plantación, confiando en que, de esa manera, tendría bastante para hacer mi pan y otras comidas.





    8. EXPEDICIÓN TEMERARIA




    Mientras atendía a todas esas cosas, podéis imaginar que muchas veces mis pensamientos tornaban hacia aquella tierra que divisara desde el lado opuesto de la isla, y que sentía nacer en mí la secreta esperanza de alcanzar alguna vez sus playas, pensando que acaso fuese tierra continental a donde pudiera trasladarme más adelante y hallar, por fin, el camino de la liberación.
    En ningún momento dejé de ver claramente los peligros que aquello suponía; lo peor era caer en manos de salvajes, sobre todo de aquellos que tenía razones para suponer peores que los leones y tigres del África. Sabía que de ser apresado lo más probable era que me asesinaran e incluso comieran; había oído decir que los caribes eran antropófagos e imaginaba, por la latitud, que no podía hallarme a mucha distancia de aquellas costas. Aun suponiendo que no fueran caníbales, lo mismo me matarían como a tantos europeos víctimas de su salvajismo; recordé que hasta grupos de veinte o treinta hombres habían sucumbido a los ataques de los salvajes y poca esperanza de defenderme podía abrigar con tal comparación. Como se ve, puse todo en la balanza; pero aquellas consideraciones que más tarde influyeron sobre mí no podían impedir ahora que mi imaginación volviera una y otra vez a su fantasía de llegar a aquellas tierras.
    Hubiese querido tener conmigo a Xury y la chalupa con la vela triangular que me había llevado más de mil millas por la costa de África. Pensé luego en utilizar el bote de nuestro barco, que como ya he narrado fuera arrojado sobre la costa. Lo encontré casi en el mismo sitio pero tumbado por la violencia constante del oleaje y el viento, cubierto casi por la arena gruesa de la playa y completamente en seco.
    De haber tenido ayuda para reflotar el bote y reparar sus averías, no dudo que me hubiera prestado buen servicio y acaso llevado hasta el Brasil. Pero me bastó estudiar su posición para darme cuenta de que tan difícil sería darlo vuelta como mover la isla entera. Hice sin embargo todo lo posible; corté troncos que sirvieran de rodillos y palancas, cobrando ánimos con la esperanza de enderezar el bote, lo que me permitiría repararlo y ponerlo en condiciones de navegar con seguridad.
    Tres o cuatro semanas empleé vanamente en este trabajo sin recompensa. Por fin, cuando estuve seguro de que mis pobres fuerzas no bastaban para dar vuelta el casco, imaginé excavar la arena que tenía debajo para que su peso hiciera lo que yo no había podido; poniendo adecuadamente los rodillos y palancas, imaginaba que podía guiar su caída. Cuando conseguí eso me resultó imposible levantar el bote del hueco en que estaba sumido, y mucho menos moverlo en dirección al mar, de modo que por fin abandoné la tarea. Pero mientras mis esperanzas se disipaban en ese sentido, las de llegar a las lejanas tierras se acrecentaban como si aquel fracaso las estimulara.
    Fue entonces cuando se me ocurrió la idea de intentar la construcción de una canoa o piragua, como la llaman los nativos, que son capaces de construirlas sin herramientas y hasta podría decirse que sin manos, empleando grandes troncos de árbol. Pensé que aquello no solamente era posible sino fácil, y cobré ánimos con la idea de que llevarlo a cabo me sería más simple que a los indios, pues disponía de mayores recursos que ellos. Olvidaba, sin embargo, cosas que me faltaban y que los naturales tienen en abundancia, por ejemplo, brazos para trasladar las canoas al mar cuando están concluidas; esta sola dificultad era para mí mucho mayor que la falta de herramientas para ellos. ¿De qué iba a servirme elegir un buen árbol en el bosque, derribarlo con gran trabajo, afanarme con mis herramientas en darle por fuera el perfil y las dimensiones de una canoa y ahuecarlo por dentro con mis instrumentos o el fuego, si al final tendría que dejarlo donde estaba por no tener fuerzas suficientes para llevarlo hasta el mar?
    Es como para creer que no había pesado la más insignificante de estas reflexiones mientras construía una canoa, ya que de inmediato hubiera advertido el problema de botarla al agua. Pero mis pensamientos estaban a tal punto absorbidos con la idea de intentar el viaje que en ningún momento consideré seriamente el problema. No fui capaz de advertir que me resultaría más fácil hacer navegar mi canoa cuarenta y cinco millas marinas que moverla las cuarenta y cinco toesas que la separaban del agua.
    Al dedicarme a su construcción hice la más grande locura que pueda cometer un hombre cuerdo. Me engañé a mí mismo con el proyecto, sin pararme a medir si era posible cumplirlo. No es que a veces no me preocupara la dificultad de botar la canoa al mar, pero de inmediato atajaba el hilo de mis pensamientos con una insensata respuesta que yo mismo me daba: «Hagámosla primero: de seguro encontraré después un medio u otro para ponerla a flote.»
    Era el sistema más absurdo que pueda concebirse, pero la intensidad de mi capricho prevaleció y me puse a la tarea. Haché un cedro tan hermoso que me pregunto si Salomón tuvo alguno tan grande para la construcción del Templo de Jerusalén. Medía cinco pies diez pulgadas de diámetro inferior, y cuatro pies once pulgadas en el superior, a una distancia de veintidós pies de altura, a partir de la cual adelgazaba un poco y se dividía luego en ramas. Con infinito trabajo derribé este coloso, tardando veinte días en hacharlo por la base y catorce en cortar las ramas y troncos menores hasta separar de él su vasta copa, en medio de fatigas indescriptibles. El mes siguiente lo pasé dando a la parte exterior del tronco la forma y las proporciones aproximadas de un bote que navegara pasablemente. Luego me llevó tres meses ahuecar el interior, hasta que tuvo la exacta apariencia de un bote. No empleé para ello el fuego sino escoplo y martillo, con agotador trabajo, hasta conseguir que el todo se pareciera bastante a una piragua y fuese capaz de llevar a bordo veintiséis hombres, cosa equivalente a mi persona y todo mi cargamento.
    ¡Qué alegría sentí cuando hube terminado el trabajo! Él bote era en verdad mucho mayor que todas las canoas o piraguas hechas de troncos que yo viera en mi vida. Muchos hachazos me había costado por cierto, y ahora sólo faltaba botarlo al agua; de haberlo conseguido hubiera yo emprendido a su bordo el más alocado e imposible viaje de que se tenga memoria alguna.
    Todas las tentativas de llevar el bote al agua fracasaron una tras otra, aunque cada una me costaba enorme trabajo. No había más de cien yardas hasta el agua, pero el primer inconveniente fue que el terreno se iba elevando hacia el lado del arroyo. Proyecté entonces excavar el suelo para formar un declive. Lo llevé a cabo aunque con prodigiosas dificultades, pero, ¿quién repara en eso cuando tiene a la vista su liberación? Y con todo, terminada la pendiente resultó imposible mover la canoa y fracasó como antes con el bote de la playa.
    Resuelto a todo, medí la distancia y resolví abrir una especie de dique o pequeño canal para traer el agua hasta la canoa, ya que no podía llevar la canoa al agua. Empecé el trabajo. Cuando había avanzado algo y me di clara cuenta de la profundidad y anchura que había que cavar, así como la arena que sería preciso extraer, hallé que teniendo por toda ayuda mis solas manos pasarían diez o doce años antes de que el canal estuviera terminado, porque la playa iba subiendo y en su parte superior el corte debía tener por lo menos veinte pies de profundidad... Con infinito disgusto, después de vacilar mucho, tuve que abandonar aquella postrera tentativa.
    Tal cosa me deprimió hondamente. Aunque demasiado tarde advertía la locura de empezar un trabajo sin calcular antes su costo y si mis fuerzas eran capaces de sobrellevarlo.
    A mitad de esta tarea se cumplió mi cuarto aniversario en la isla y celebré el día con la misma devoción y aún más recogimiento que los anteriores; el constante estudio y mi aplicación a interpretar la palabra de Dios, así como el socorro a Su gracia, me habían dado un conocimiento más hondo que el de otros tiempos. Ahora veía las cosas de muy distinta manera. El mundo se me aparecía como algo remoto, que en nada me concernía y del que nada debía esperar o desear. En una palabra, me hallaba del todo aislado de él y como si ello hubiera de durar siempre; me habitué a considerarlo en la forma en que acaso lo hacemos cuando ya no estamos en él; un lugar en el cual se ha vivido pero al que ya no se pertenece. Y bien podía decir como el patriarca Abraham al hombre rico: «Entre tú y yo hay un abismo.»
    Vivía ahora de un modo mucho más confortable que al comienzo, y con una tranquilidad harto mayor tanto para el alma como para el cuerpo. Me sentaba a comer sintiéndome lleno de gratitud, y admiraba la providencia de Dios que así tendía mi mesa en la soledad. Aprendí a estar reconocido a la parte buena de mi situación y a olvidar en lo posible la mala; prefería tener más en cuenta lo que me daba placer que las privaciones; y esto me hacía experimentar a veces tan secreto júbilo que no podría expresarlo, y si lo menciono aquí es solamente para llamar la atención de aquellos que no saben gozar alegremente lo que Dios les ha dado porque sólo ven y envidian lo que El no ha querido concederles. Toda nuestra aflicción por lo que no tenemos nace de la falta de gratitud hacia lo que nos ha sido dado.
    Pasaba muchas horas y hasta días en tratar de representarme con todo detalle lo que habría sido de mí si no hubiera podido sacar nada del barco. Me hubiera faltado alimento en primer lugar, salvo tal vez peces y tortugas; pero sí no hubiera podido encontrarlos pronto habría perecido de hambre. Aun así, ¿no hubiera sido mi existencia la de un salvaje? Suponiendo que la suerte me hubiera ayudado a capturar un pájaro o una cabra, no habría tenido cómo desollar o abrir esos animales, separar la carne de los huesos y las entrañas, sino que a modo de las bestias habría desgarrado y comido con mis uñas y dientes.
    Tales reflexiones me llenaban de gratitud por la bondad de la Providencia y a pesar de todos mis temores y dificultades estaba reconocido por lo que me había sido dado. Si mi vida era lamentable en un sentido, en otro mostraba claramente las señales de la bondad divina; llegué a desear, no una existencia confortable, sino que mi reconocimiento por la gracia de Dios, su protección incesante en tan duro trance, fueran mi diario consuelo. Y cuando hube alcanzado tal serenidad de espíritu ya no volví a sentirme nunca triste.
    Llevaba tanto tiempo en la isla, que diversas cosas que había traído del barco estaban agotadas o poco menos. La tinta, como he dicho ya, se consumió casi enteramente salvo una porción que fui salvando con pequeñas dosis de agua que le agregaba hasta que se puso tan débil que apenas se veían los trazos sobre el papel. Mientras duró, la empleé en llevar la crónica de las cosas más importantes que me ocurrían mensualmente. Fue entonces cuando, releyendo los acontecimientos del pasado, vi que existía una extraña coincidencia de fechas en los sucesos que me habían ocurrido y que, de ser supersticioso al extremo de considerar días afortunados y días nefastos, hubiera tenido harto motivo para sentirme extrañado.
    En primer lugar noté que el mismo día en que abandonando a mis padres y amigos salí de Hull, fue tiempo más tarde el de mi captura y esclavitud a manos del pirata de Sallee.
    El mismo día que tuve la fortuna de salvarme del naufragio en la rada de Yarmouth, huí del cautiverio de Sallee en la chalupa.
    El mismo día de mi nacimiento, es decir, el 30 de setiembre, salvé milagrosamente mi vida veintisiete años más tarde al quedar como único sobreviviente en la isla: de manera que mi vida de errores y mi vida solitaria principiaron el mismo día.
    Excepto la tinta, lo que me faltó después de un tiempo fue la galleta que había conseguido sacar del barco. Aunque la había economizado en extremo, limitándome a una por día, me encontré privado de ella casi un año antes de que lograra hacer pan con mi grano; y de esto último tenía sobrada razón para mostrarme agradecido: ya he contado la manera casi milagrosa en que pude obtener las primeras plantas.
    Luego fueron mis vestidos los que se deterioraron. Casi no tenía ropa blanca salvo algunas camisas a cuadros que encontré en los arcones de los marineros y que tuve cuidado de guardar, ya que solamente en algunas épocas toleraba la camisa sobre el cuerpo, siendo una gran suerte para mí tener casi tres docenas de ellas. Había también varios capotes de marino, pero me resultaban demasiado abrigados para usarlos. Aunque el clima era cálido y allí la ropa no era demasiado necesaria en modo alguno, en realidad no podía vivir completamente desnudo aun cuando me hubiera agradado estar así, cosa que no ocurría pese a encontrarme solo.
    La razón por la cual no podía vivir allí desnudo era que la violencia del sol se amortiguaba un poco llevando algo de ropa; de lo contrario me hubiera lacerado la piel, mientras que con una camisa liviana, el aire que circundaba y se movía entre ella y mi cuerpo era siempre más fresco que el ambiente. De la misma manera hubiese sido imposible andar bajo el sol sin un sombrero o gorro. Son tan fuertes los rayos del sol en esa latitud que producen de inmediato violentas jaquecas si no se los contrarresta con un sombrero, y yo había notado que bastaba ponérmelo para que el dolor cesara en seguida.
    Empecé a preocuparme por salvar los pocos harapos que me quedaban y que yo llamaba ropas. Había gastado todas las chaquetas que tenía y pensaba ahora en la posibilidad de hacer algunas con los pesados capotes marineros y cualquier otro material a mi alcance. Comencé, pues, a hacer el sastre, o mejor el remendón porque sólo conseguí resultados deplorables. Me las arreglé, con todo, para cortar dos o tres chaquetas que esperaba me serian muy útiles: en cuanto a calzoncillos y pantalones los resultados fueron aún peores hasta que pude hallar una mejor solución.
    He mencionado que guardaba las pieles de todos los animales que cazaba; luego de desollarlos tendía las pieles al sol entre estacas, por lo cual algunas se pusieron tan secas y duras que no me sirvieron, pero otras en cambio parecían útiles. Lo primero que obtuve de esas pieles fue un gorro grande, con el pelo hacia afuera para preservarme de la lluvia; y tan bien me resultó que me animé a cortarme un traje hecho enteramente de dichas pieles, es decir, una chaqueta y calzones abiertos en las rodillas, todo muy holgado, ya que se trataba de preservarme más del calor que del frío. No debo omitir que estaban atrozmente mal hechos, pues si era mal carpintero resulté aún peor sastre. Pero me sirvieron admirablemente, y cuando la lluvia me sorprendía en un viaje, el pelo hacia afuera de mi chaqueta y gorra me mantenía perfectamente seco.
    Pasé también bastante tiempo y no poco trabajo fabricándome una sombrilla. Me era muy necesaria y ansiaba tener una, pues recordaba la que había visto en el Brasil, donde son de gran utilidad contra los calores, siendo aquí en verdad imprescindible, ya que la isla se hallaba aún más cerca del ecuador. Aparte de eso, como tenía que alejarme con frecuencia de mi casa, la sombrilla podría ser igualmente útil contra los calores y las lluvias. Me dio bastante trabajo y pasaron muchos días antes de que pudiera fabricar algo parecido a lo que quería; incluso estropeé dos o tres veces mi obra antes de estar seguro de haber conseguido lo que deseaba, pero por fin obtuve un quitasol adecuado a mis necesidades. La mayor dificultad consistiría en cerrarlo, ya que abrirlo era fácil, pero si luego no podía plegarlo resultaba demasiado incómodo de llevar, salvo sobre mi cabeza, donde a veces no era necesario. Por fin di con la manera de manejarlo fácilmente y lo cubrí de pieles, con el pelo hacia arriba para que me protegiera de las lluvias como un tejadillo y a la vez atajara los rayos del sol; con él podía andar sin preocupaciones en el tiempo más caluroso como en el más frío, y cuando no me hacía falta lo plegaba para llevarlo bajo el brazo.
    De esa manera aumenté mis comodidades de vida, mientras mi espíritu se resignaba cada vez más a la voluntad de Dios y se entregaba enteramente a lo que su Providencia dispusiera.
    No puedo decir que a partir de entonces y durante cinco años me haya ocurrido nada de extraordinario. Vivía en la forma ya narrada, en el mismo sitio y tal como antes. Mis ocupaciones dominantes eran la siembra anual de la cebada y el arroz, la preparación de uvas en cantidad suficiente para tener, entre granos y pasas, lo bastante para alimentarme un año entero. Al margen de esta labor anual, y la cotidiana de salir de caza con la escopeta, tenía otra, que era la de hacerme una canoa; por fin la terminé, y cavando un canal de unos seis pies de ancho y cuatro de profundidad, la llevé hasta la ensenada, distante casi media milla. Se recordará que mi primera canoa, excesivamente pesada por haber sido construida sin la conveniente reflexión acerca de cómo la botaría al mar, quedó en el sitio; incapaz de llevarla hasta el agua o conducir el agua hasta ella, me vi obligado a abandonarla en el lugar donde la construyera como un claro ejemplo para ser más sensato otra vez. Por lo tanto, en esta segunda tentativa, aunque no pude conseguir un árbol tan bueno ni un lugar donde el agua estuviera a menos de la media milla ya mencionada, comprendí que la tarea era practicable y me entregué a ella de lleno. Estuve ocupado durante dos años, pero jamás renuncié un solo día a mi labor con la esperanza de tener al fin un bote que me permitiera hacerme a la mar.
    Sin embargo, cuando mi pequeña piragua estuvo concluida, supe que su tamaño no se prestaba para la intención que yo había tenido en cuenta al construir la primera, es decir, lanzarme hacia «tierra firme» cuarenta millas más allá. La fragilidad de esta canoa me obligaba a desistir de esa esperanza y no pensar más en ella; pero ya que tenía un barquichuelo, proyecté dar con él la vuelta a la isla. El viaje que hiciera a la costa del lado opuesto, atravesando por tierra y haciendo tantos interesantes descubrimientos, me impulsaba a explorar otras regiones de la costa. Era dueño de un bote; ¿por qué no navegar en torno a la isla?
    A tal fin equipé discretamente mi canoa, comenzando por fijar en ella un pequeño mástil cuya vela hice aprovechando pedazos que había traído del barco y que había conservado cuidadosamente en depósito.
    Emplazados el mástil y la vela, descubrí después de algunas pruebas que el bote navegaba muy bien. Le adapté entonces varias cajas donde poner provisiones, pólvora y balas, así como otros efectos que debían conservarse secos tanto de la lluvia como de la espuma del mar; hice luego en el interior de la canoa una larga y profunda ranura donde cabía mi escopeta, protegida por una lona contra toda humedad.
    Puse mi sombrilla en la popa, como un segundo mástil cuya vela se extendiera sobre mi cabeza protegiéndome del sol a manera de toldo. Así equipado emprendí una y otra vez pequeños recorridos por el mar, aunque sin alejarme mucho de la costa conocida y de la caleta. Por fin, cada vez más deseoso de hacer la circunnavegación de mi pequeño reino, me resolví a él. Puse a bordo suficientes vituallas para el viaje; dos docenas de panes de cebada (a los que debería llamar más bien galletas), un puchero de barro con arroz tostado, alimento del que hacía gran consumo, una botellita de ron, la mitad de una cabra, sin contar pólvora y balas en cantidad suficiente para cazar y dos de los abrigos de marinero que ya he mencionado como provenientes de los arcones que salvé del naufragio; contaba con ellos para que me sirvieran de colchón y de frazada.
    El dieciséis de noviembre, en el sexto año de mi reino o mi cautiverio, como se quiera llamarle, inicié el viaje, que me resultó mucho más extenso de lo que había imaginado; en verdad la isla no era muy grande, pero cuando llegué a su extremo oriental vi una gran cadena de rocas que penetraban más de dos leguas en el mar, algunas emergiendo del agua y otras submarinas, y más allá un banco de arena, casi en seco, otra media legua hacia adentro. Era preciso contornear ese cabo para seguir viaje.
    Cuando descubrí el accidente estuve a un paso de dar por finalizada la empresa y volverme, no sabiendo cuánto tendría que internarme en el océano y, sobre todo, preguntándome si me sería posible volver; en la duda me decidí a anclar allí mismo, cosa simple, pues me había fabricado una especie de ancla con un pedazo de cloque roto que traje del barco.
    Ya fondeado el bote, tomé la escopeta y trepé a una colina que me parecía adecuada para tener una visión panorámica del lugar; desde allí vi el largo total del cabo rocoso, y decidí aventurarme.
    Observando el mar desde la colina noté una fuerte y violenta corriente que corría hacia el este y pasaba casi rozando el cabo; la estudié detenidamente porque advertía el peligro de que si mi bote era envuelto por ella podría ser arrastrado mar afuera y sin posibilidad de regresar a la isla. Hasta pienso que si no hubiera tenido el cuidado de subir antes a la colina, algo de eso me habría ocurrido porque la misma corriente se desplazaba en el lado opuesto de la isla, sólo que se dirigía hacia afuera a mayor distancia. Entre ambas advertí la presencia de un fuerte remolino junto a la costa, y deduje que si me las arreglaba para zafarme del impulso de la primera corriente encontraría ventaja en la protección de aquel remolino.
    Con todo me quedé allí dos días, porque el viento soplaba fuerte del E-SE y siendo contrario a la citada corriente producía un violento oleaje contra el cabo rocoso; no era posible navegar cerca de las rocas por temor a ese oleaje, ni más lejos porque allí podía arrastrarme la corriente.
    En la mañana del tercer día observé que el viento había cedido durante la noche y que el mar estaba sereno, por lo cual emprendí viaje. Que lo que sigue sea un ejemplo para todos los pilotos ignorantes o temerarios: apenas había llegado al extremo del cabo, sin apartarme de él más que la longitud de mi canoa, cuando me hallé en un agua tan profunda y una corriente tan violenta como la compuerta de un molino. Fue tal la fuerza con que me arrastró que no pude mantenerme cerca de la orilla y pronto fui llevado más y más lejos del remolino que quedaba a mi mano izquierda. No soplaba viento alguno que pudiera ayudarme, y la fuerza de los remos no servía de nada. Entonces me creí perdido, ya que la corriente marina bordeaba la isla por ambos lados y venía a unirse algunas leguas más allá, hacia donde sería arrastrado con mayor fuerza. Carecía de medios para evitarlo; ante mí se extendía solamente la visión de la muerte, no por naufragio, ya que reinaba absoluta calma, sino por el hambre. Verdad que había encontrado una tortuga en la playa, tan pesada que apenas pude levantarla y meterla en el bote, y que también llevaba un cacharro grande lleno de agua dulce. ¿Pero de qué iba a servirme aquello cuando me encontraba perdido en el vasto océano donde sin duda no había tierra firme, ni siquiera una isla, por lo menos en mil leguas a la redonda?
    Puede entonces advertir cuan fácil le es a la divina Providencia tornar aún peor la más miserable condición humana. Miraba hacia mi desolada isla como si se tratara del más bello lugar de la tierra, y cuanta felicidad podía desear mi corazón era volver allá.
    Hice lo que estaba a mi alcance, hasta casi caer extenuado luchando por mantener el bote rumbo al norte, es decir, hacia el lado de la corriente donde había divisado el remolino. A eso de mediodía, cuando el sol alcanzaba el cénit, me pareció sentir en la cara un aleteo de brisa soplando del S-SE. Aquello me animó no poco, y más aún cuando media hora después la brisa se transformó en viento. Estaba terriblemente alejado de la isla, y la menor nube o niebla que hubiese surgido bastaba para mi perdición; no tenía brújula, y perder de vista la tierra un solo instante hubiera bastado para no dar ya nunca con el rumbo.
    El tiempo sin embargo seguía despejado, y aprovechando mi mástil y vela traté de mantenerme hacia el norte para salir de la corriente que desviaba al noroeste. Una hora más tarde había conseguido llegar a una milla de la costa y allí, aprovechando el agua tranquila, pronto toqué tierra.
    Apenas me sentí a salvo caí de rodillas para dar gracias a Dios por su bondad, y me propuse abandonar toda idea de salir de la isla en el bote. Luego me alimenté con lo que tenía a bordo, y dejando el bote anclado en una pequeña caleta que había descubierto bajo la protección de los árboles, dormí profundamente, agotado por las fatigas de aquel viaje.
    Me preocupaba ahora la idea de cómo volver a casa con el bote. Había corrido demasiado peligro y conocía bien las dificultades de regresar por el mismo lado que viniera; en cuanto a lo que pudiese haber en el lado opuesto —es decir, el oeste— lo ignoraba y no me sentía con deseos de arriesgarme más. Resolví, pues, a la mañana siguiente, ir por el oeste siguiendo la costa, y buscar algún arroyo o caleta donde dejar la canoa bien asegurada por si me hacía falta más adelante. A unas tres millas, costeando la playa, vine a dar con una pequeña bahía o ensenada, a una milla más allá, que se iba estrechando hasta la desembocadura de un arroyuelo, donde descubrí el fondeadero más indicado para mi bote, y en el que podría quedar como si fuera un puerto hecho a su medida. Allí dejé la canoa después de amarrarla sólidamente, y volví a la costa para explorar el sitio en que me hallaba.
    No me costó mucho advertir que apenas había pasado el lugar adonde llegara en mi anterior viaje a pie por la costa. Sacando del bote la escopeta y la sombrilla, porque hacía mucho calor, empecé a recorrer el camino conocido, que me resultó bastante agradable después de un viaje como el que acababa de hacer; por la tarde llegué al sitio donde se levantaba mi enramada, encontrando todo en orden, lo que me complació, ya que aquella era mi casa de campo y me gustaba que se conservara en buen estado. Escalé la empalizada y tendiéndome a descansar de tan fatigosa jornada pronto me venció el sueño. Pero juzguen los que leen esta historia cuál habrá sido mi sorpresa cuando entre sueños oí una voz que me llamaba repetidas veces por mi nombre:
    — ¡Robin, Robin, Robin Crusoe! ¡Pobre Robin Crusoe! ¿Dónde estás, Robin Crusoe? ¿Dónde estás? ¿Dónde has estado?
    Al principio yacía tan profundamente dormido, con la fatiga de remar toda la mañana y más tarde hacer a pie el resto del viaje, que no me desperté de una vez, sino que entre sueños me pareció que aquella voz llamándome era producto de mi fantasía. Pero como continuaba repitiendo « ¡Robin Crusoe, Robin Crusoe!», por fin me arrancó del sueño haciéndome pasar a un mortal terror. Mas apenas había abierto los ojos divisé a Poli, mi papagayo, posado en el borde de la empalizada, y comprendí que era él quien me había estado hablando en ese lenguaje lastimero que yo le había enseñado ex profeso. Tan bien lo había aprendido que era capaz de estarse largo rato posado en mi dedo y con su pico contra mi cara, diciéndome:
    — ¡Pobre Robin Crusoe! ¿Dónde estás? ¿Dónde has estado? ¿Cómo has venido aquí?
    Aun después que me hube convencido de que se trataba del papagayo y de nadie más, pasó un rato antes que recobrase la serenidad. Me asombraba que el animal hubiera llegado hasta allí, y que en vez de irse a otro lugar estuviera como esperándome en la enramada. Pero cuando estuve bien convencido de que era mi buen Poli, deseché mis preocupaciones, y tendiéndole la mano lo llamé por su nombre. El cariñoso pájaro voló a mi lado y luego de posarse como le gustaba hacerlo en mi pulgar, prosiguió hablándome como antes:
    — ¡Pobre Robin Crusoe! —me decía, y me preguntaba cómo había llegado allí y dónde había estado, igual que si hubiera sentido una inmensa alegría al verme otra vez. Entonces lo llevé conmigo a casa.
    Bastante tenía ahora de correrías por el mar, y suficiente tema para quedarme meditando muchos días acerca del peligro pasado. No me gustaba tener el bote del otro lado de la isla y tan alejado de mí, pero tampoco hallaba manera segura de traerlo. Por el lado oriental no quería ni pensar en la posibilidad de aventurarme a bordearlo por segunda vez; a la sola idea sentía paralizárseme el corazón y helarse mi sangre. En cuanto al lado opuesto ignoraba sus características, pero suponiendo que la corriente tuviera en aquella costa la misma fuerza que ya había yo experimentado en la parte opuesta, intentar el viaje equivalía a correr los mismos riesgos de ser arrastrado mar afuera. Terminé por resignarme a no tener el bote conmigo, aunque tantos meses de duro trabajo me había costado entre hacerlo y lanzarlo al mar.





    9. LA PISADA EN LA ARENA




    En tal disposición de ánimo viví cerca de un año, haciendo una vida retirada y tranquila como puede imaginarse; mis pensamientos estaban tan adaptados a mi presente condición, y había llegado a resignarme tanto a los designios de la Providencia, que hasta me consideré un hombre feliz en todos los aspectos salvo el de la compañía.
    Mi ingenio seguía aplicándose a las labores mecánicas que debía realizar para suplir tantas cosas necesarias, y pienso que llegué a ser un excelente carpintero, sobre todo si se tiene en cuenta la escasez de herramientas en que me encontraba. Aparte de esto, mi experiencia como alfarero se acrecentó también y pude por fin moldear la arcilla con una rueda, lo que permitía obtener más fácilmente cacharros de buena forma, mientras que los antiguos apenas podían ser mirados. Pero nada creo que me haya ocasionado mayor satisfacción, haciéndome sentir tan orgulloso de mi habilidad, como el día en que llegué a construirme una pipa. Cierto que era muy tosca y fea, cocida al fuego como los otros objetos de arcilla, pero resultó fuerte y el humo tiraba perfectamente. Mucho me alegré porque me gustaba en extremo fumar; a bordo había pipas, pero al principio no las busqué, ya que ignoraba la existencia de tabaco en la isla, y cuando volví más tarde al casco del barco no pude encontrarlas.
    También hice grandes progresos en cestería, tejiendo muchos canastos según mi gusto; aunque de no muy buena apariencia, resultaban extremadamente útiles para guardar efectos o acarrear diversas cosas a casa. Por ejemplo, si mataba lejos una cabra, podía colgarla allí mismo de un árbol y luego de haberla desollado y cortado en trozos los traía en uno de los canastos. Lo mismo si atrapaba una tortuga; allí mismo extraía los huevos y algunos pedazos de carne que me bastaban trayéndolos en mi cesto y dejando el resto en la playa. Los canastos más grandes y profundos eran mi depósito de granos, pues me apresuraba a desgranar las espigas apenas estaban secas y guardaba la semilla en la forma indicada.
    Pronto me di cuenta de que la pólvora disminuía considerablemente, y como de ninguna manera sería posible reemplazarla con los medios a mi alcance, me di a pensar qué haría para procurarme carne de cabra cuando ya no tuviese medios de cazarlas. Se recordará que durante mi tercer año en la isla apresé un cabrito que se crió muy manso, tanto que jamás pude decidirme a matarlo y lo dejé que viviera hasta que murió de viejo. Ahora, al cumplirse el undécimo año de mi residencia en la isla, y advirtiendo que las municiones disminuían, me puse a pensar algún medio de tender trampas a las cabras para atraparlas vivas.
    A tal fin tejí algunas redes en las que estoy seguro que cayeron varias cabras, pero como las cuerdas no eran solidad y yo no tenía alambre, las encontraba siempre rotas y el cebo comido. Por fin probé una trampa distinta; luego de hacer varios pozos profundos en aquellos sitios que frecuentaban las cabras, los disimulé con haces entretejidos que yo mismo había fabricado, sobre los cuales puse un gran peso; esparciendo espigas de cebada y arroz aunque sin alistar las trampas, observé que los animales acudían a esos lugares, como me lo probaron las huellas de sus patas. Una noche apresté tres trampas, y al acudir por la mañana vi que los haces estaban removidos y que faltaba el grano; pero las cabras habían evitado la celada. Esto me descorazonó bastante y me puse a rehacer las trampas; por fin, y para abreviar, yendo una mañana z revisarlas encontré en una un viejo macho cabrío y en la otra tres cabritos.
    Con respecto al macho cabrío no encontraba qué hacer con él, porque era tan fiero que no me atrevía a bajar al pozo para sacarlo vivo, lo que me hubiera agradado mucho. Por fin lo dejé escapar, y huyó a tal velocidad que parecía haberse vuelto loco de espanto. Yo había olvidado lo que una vez aprendiera, y es que el hambre amansa al mismo león; si hubiera dejado al macho tres o cuatro días en la trampa sin darle de comer, y le hubiera llevado después un poco de agua y algo de grano, se hubiera domesticado lo mismo que los cabritos, ya que son animales sagaces y tratables cuando se los cría convenientemente.
    Ignorando todo eso, lo dejé escapar; después, sacando uno a uno los cabritos del pozo, los até con sogas y no sin trabajo pude llevarlos a casa.
    Pasó bastante tiempo antes de que aceptaran lo que les daba de comer, pero terminé por tentarlos con granos maduros y pronto vi que se amansaban. Ya para ese entonces había decidido que si quería contar con carne de cabra el día en que se concluyera mi pólvora, criar un rebaño al lado de mi casa era la única solución posible.
    Meditando en esto, advertí la conveniencia de mantener separados los ya mansos de los salvajes en libertad, pues si los dejaba juntarse no tardarían aquéllos en hacerse tan salvajes como éstos. No veía otro remedio que elegir un buen pedazo de tierra y rodearlo de una empalizada, a fin de que la separación fuera absoluta y para siempre.
    Un par de manos era harto poco para semejante tarea, pero como advertía su urgente necesidad me apresuré a elegir terreno adecuado donde hubiese suficiente hierba para pastar, agua dulce y protección contra los calores solares.
    Para empezar resolví construir la empalizada en torno a un área de unas ciento cincuenta yardas de largo por cien de ancho; como no me faltaban tierras aptas en torno, podría más adelante ensanchar el vallado si mi rebaño aumentaba mucho. La tarea no me pareció excesiva, y la comencé con decisión. Durante tres meses estuve cercando el corral, y en este plazo tuve a las cabritas en la mejor parte, cuidando de alimentarlas lo más cerca posible de mí para que se amansaran bien. Con frecuencia les llevaba algunas espigas de cebada o un puñado de arroz y se los ofrecía en mi mano, por lo cual después que el vallado rodeó el terreno y pude soltarlas dentro, corrían detrás de mí balando por un poco de grano.
    Todo resultó como lo había deseado, y un año y medio más tarde era dueño de un rebaño de unas doce cabras, incluyendo los cabritos; dos años después ascendía a cuarenta y tres, fuera de las muchas que había matado para alimentarme. Aparte cerqué cinco corrales menores para que pastaran, con portillos que comunicaban a mi gusto unos con otros, y especie de pequeñas jaulas donde las hacía entrar para apresarlas fácilmente.
    No fue esto todo, porque además de la carne necesaria para comer disponía asimismo de leche, cosa que no se me había ocurrido pensar al principio, pero que me llenó de agradable sorpresa cuando comprendí lo simple que era. De inmediato monté una lechería, y diariamente ordeñaba un galón o dos de leche. La naturaleza, que da los medios de alimentarse a toda criatura, parece enseñarle al mismo tiempo cómo debe aprovechar ese alimento; yo que jamás había ordeñado una vaca y mucho menos una cabra, ni había visto preparar manteca o queso, llegué a hacer todo eso de la manera más natural y simple, aunque no sin muchos ensayos y fracasos. Desde entonces tuve tanta manteca y queso como podía desearlo.
    Hasta un estoico se hubiera reído al verme comer rodeado de mi pequeña familia. Yo era allí la majestad y el poder, príncipe y señor de la isla entera; la vida de mis súbditos estaba librada a mi arbitrio; podía ahorcar, descuartizar, conceder libertad y privar de ella. No había rebeldes entre mis súbditos.
    Solo como un rey, comía atendido por mis sirvientes.
    Poli, a manera de un favorito, era el único con derecho a dirigirme la palabra. Mi perro, ya muy viejo y chocho, se tendía a mi derecha mientras dos gatos, uno a cada lado de la mesa, esperaban que les cediera uno que otro bocado, como una prueba de especial favor.
    Rodeado de tal corte, y con tanta liberalidad, transcurría mi vida. Nada podía desear, como no fuera la compañía de mis semejantes; y por cierto que poco tiempo después la logré en exceso.
    Ya he dicho que muchas veces me volvía la idea de tener el bote conmigo, aunque no me impulsara el deseo de correr nuevos peligros a su bordo. En algunas ocasiones me ponía a pensar el modo de traerlo de este lado de la isla, pero otras veces me conformaba fácilmente con su ausencia. Poco a poco, sin embargo, predominaron aquellos deseos, y sobre todo el de llegar al punto de la isla donde, como ya he narrado, trepé a una colina para ver desde allí la línea de la costa y la dirección de las corrientes marinas. El deseo aumentó diariamente hasta que decidí irme a pie, recorriendo la costa. Así lo hice y si algún inglés hubiera podido en aquel entonces tropezar conmigo se hubiera asustado mucho o por el contrario reído a morirse. Yo mismo, cuando a veces me contemplaba, no podía menos de sonreír a la idea de atravesar Yorkshire con semejantes ropas y el correspondiente equipo. Que el lector juzgue por el siguiente esbozo:
    Llevaba un gran gorro sin forma alguna, hecho de piel de cabra, con una pieza de piel colgando detrás para que me protegiera de los rayos del sol y a la vez impidiera a la lluvia entrarme por el cuello, porque pocas cosas dañan tanto en aquellos climas como la lluvia entre los vestidos y la piel.
    Usaba una corta chaqueta también de piel de cabra, cuyos faldones me llegaban a la mitad de los muslos, y un par de calzones cortos del mismo material. Estos calzones habían sido cortados de la piel de un viejo macho cabrío y el pelo era tan largo que colgaba, a manera de pantalón, hasta la mitad de la pantorrilla. Me faltaban medias y zapatos, pero me había ingeniado para fabricarme unos borceguíes, si es que puedo darles algún nombre, altos de pierna y que se anudaban a los lados como las polainas; es de imaginar la forma que tendrían, al igual que el resto de mi atavío.
    Como cinturón usaba una larga tira de piel de cabra que se ajustaba con dos tiras más pequeñas en lugar de hebillas; a cambio de la espada o el puñal que se lleva en el cinturón, tenía un hacha y una pequeña sierra. Poseía además un segundo cinturón, especie de tahalí que me cruzaba el hombro, y en su extremo, bajo el brazo izquierdo, había colgado dos sacos de piel de cabra; en uno estaba la pólvora y en el otro las balas. Con una cesta en la espalda y la escopeta al hombro, sostenía sobre la cabeza una fea y pesada sombrilla también hecha de piel, que después de la escopeta era el objeto más necesario para mí. En cuanto a mi rostro, no lo tenía tan atezado como se hubiera podido suponer de un hombre que en modo alguno lo cuidaba y que vivía dentro de los diecinueve grados de latitud. Al principio toleré el crecimiento de mi barba hasta que tuvo casi un cuarto de yarda, pero como tenía tijeras y navajas, la recorté, salvo el bigote, que me complacía en retorcer a la manera de las patillas mahometanas (como había visto que lo usaban los turcos que conociera en Sallee, ya que los moros lo cortan de diferente modo).
    De mis bigotes o patillas no diré que fuesen lo bastante largos para colgar en ellos el sombrero, pero tenían suficiente longitud y espesor como para resultar espantosos en Inglaterra.
    Todo esto carece de importancia: tan poco me ocupaba de mi aspecto que no le concedía la más insignificante atención, de modo que nada más diré al respecto. Con tal traza empecé mi viaje, que duró cinco o seis días. En primer término seguí la costa hasta el lugar donde había anclado el bote para encaramarme a la colina. No teniendo ahora canoa de la cual preocuparme, busqué la vía más corta para subir a la misma altura que la vez anterior, y cuando estuve en la cumbre miré el cabo rocoso que penetraba en el océano y que en aquel terrible día intenté bordear a bordo de la canoa. ¡Cuál no sería mi asombro al descubrir que el mar estaba allí profundamente tranquilo, sin oleaje, ni movimiento, ni corriente!
    No podía comprender cómo había cambiado de esa manera; resolví por lo tanto quedarme algún tiempo observándolo, para estudiar lo que ocurría con las distintas mareas.
    El detallado estudio me demostró pronto que la única precaución a tomar consistía en tener presente el flujo y reflujo de la marea, y que no había dificultad alguna en llevar el bote al otro lado de la isla. Pero cuando pensé en llevar esto a la práctica, me invadió un terror tan grande con el recuerdo del peligro que había pasado la otra vez, que ni siquiera fui capaz de imaginar esa posibilidad. Preferí adoptar una segunda resolución, más segura aunque mucho más trabajosa: construir otra canoa o piragua, a fin de poseer una en cada lado de la isla.
    Es preciso tener en cuenta que para entonces disponía yo de dos fundos —si puedo llamarlos así— en la isla; el primero era la tienda con su fortificación de empalizada y la cueva a sus espaldas, que había profundizado y dividido en varios departamentos que comunicaban entre sí. Uno de estos depósitos, el mayor y menos húmedo, con una salida que daba más allá de la empalizada, estaba ocupado con las tinajas más grandes de que ya he hablado y además catorce o quince canastos capaces cada uno de contener cinco o seis fanegas. Allí acumulaba mis reservas de alimentos, especialmente el grano, del que una parte estaba aún en espiga y el resto había sido desgranado a mano.
    En cuanto a la empalizada, hecha con los troncos que ya he descrito, se había convertido en una muralla de árboles tan grandes y extendidos que no dejaban sospechar en modo alguno la existencia de una habitación humana.
    Cerca de mi morada, pero hacia el interior de la isla y sobre tierras más bajas, estaban mis dos plantaciones que cuidaba y araba para cosechar anualmente el grano en su punto; si hubiera deseado más semilla, disponía de abundante tierra a continuación de aquélla.
    En segundo término era dueño de mi residencia de campo, que constituía por cierto un fundo bastante pasable. Ante todo la enramada, que tenía buen cuidado de arreglar podando el cerco circundante para mantenerlo siempre a la misma altura, con la escalera del lado de adentro. En cuanto a los árboles, que al principio no eran más que estacas pero crecían ahora con gran lozanía, los podé como para que su copa se desarrollara espesa y amplia, dándome la sombra más agradable que pueda imaginarse. En el centro estaba la tienda, hecha con un gran trozo de vela sostenido por pértigas; era tan firme y sólida que nunca necesitaba reparación alguna. Bajo ella había armado una especie de cama con pieles de los animales que cazaba y otras cosas blandas, colocadas sobre un colchón salvado del naufragio, y si era necesario me cubría con un gran capote de marinero. Toda vez que me ausentaba de mi morada principal tenia, pues, este refugio en pleno campo.
    A esto hay que agregar los corrales del ganado, es decir, las cabras. Tanto como me costara rodear el terreno con un vallado, me costaba ahora cuidar que no se rompiera y escaparan por allí los animales; no abandoné mis esfuerzos hasta rodear el exterior del cerco con gran cantidad de pequeñas estacas, tan juntas que era difícil pasar por entre ellas una mano. Cuando aquellas estacas echaron raíces, cosa que sucedió en la estación lluviosa, el cerco se puso más fuerte que cualquier pared.
    Esto dará testimonio de que no pasaba mi tiempo sin hacer nada y que no escatimaba energías en lo que consideraba necesario para mi comodidad; estaba seguro de que criar aquellos animales al alcance de mi mano equivalía a tener un almacén de carne, leche, manteca y queso para toda mi vida, aunque durase cuarenta años más; pero, por otra parte, criar las cabras cerca de mí exigía perfeccionar de tal modo los cercos que de ninguna manera pudieran escaparse; y como he dicho obtuve tan buen éxito con el procedimiento de las pequeñas estacas que cuando crecieron vine a descubrir que eran demasiadas y tuve que entresacar algunas.
    Mis viñedos crecían también en la enramada, y contaba principalmente con ellos para tener pasas durante el invierno. Cuidé por tanto de conservarlos bien, ya que me parecían los más agradables entre mis alimentos y porque reunían virtudes medicinales que las tornaban muy refrescantes y saludables.
    Como la enramada venía a estar a mitad de camino entre mi casa y el sitio donde dejé fondeado el bote, habitualmente pernoctaba en ella en mi viaje hacia allá. Me gustaba mucho visitar la caleta y ver si el bote se mantenía en buenas condiciones. Algunas veces salí con él para distraerme, pero sin intentar nunca un verdadero viaje; navegaba a uno o dos tiros de piedra de la costa, de miedo a ser otra vez arrastrado por el viento o la violencia de las corrientes.
    Y llego ahora a una nueva etapa de mi vida. Cierta mañana, a eso del mediodía, yendo a visitar mi bote, me sentí grandemente sorprendido al descubrir en la costa la huella de un pie descalzo que se marcaba con toda claridad en la arena.
    Me quedé como fulminado por el rayo, o como en presencia de una aparición. Escuché recorriendo con la mirada en torno mío; nada oí, nada se dejaba ver. Trepé a tierras más altas para mirar desde allí; anduve por la playa, inspeccionando cada sitio, pero nada encontré como no fuera esa única huella. Empecinado, me puse a buscar otra vez preguntándome si no me estaría dejando llevar por una fantasía. Pero pronto hube de desechar esa idea: la huella era exactamente la de un pie humano, con su talón, dedos y forma característica. No podía imaginarme la procedencia de aquel pie, y después de debatir en mí mismo innumerables y confusos pensamientos, regresé a mi fortificación sin sentir, como suele decirse, el suelo que pisaba; tanto era el terror que me había invadido. A cada paso me daba vuelta a mirar en torno, confundía los arbustos y árboles y creía ver un hombre en cada tronco. Imposible es describir las distintas formas en que la imaginación sobreexcitada me hacía ver las cosas, las extrañas ideas que cruzaban por mi mente y hasta qué punto me dejé arrebatar por sus enfermizas fantasías mientras hice el camino de regreso.
    Al llegar a mi castillo —como creo que le llamé a partir de entonces— entré en él como un perseguido. Si lo hice mediante la escalera en la forma ya descrita, o entré por la abertura de la cueva, es cosa que no recuerdo. ¡Nunca una liebre corrió a su cueva ni un zorro a la suya con mayor espanto que el mío al entrar en mi morada!
    No dormí en toda la noche. Cuanto más tiempo transcurría desde el descubrimiento mayores eran mis aprensiones, al contrario de lo que parecería natural en tal circunstancia, sobre todo teniendo en cuenta la habitual reacción de los hombres ante el miedo. Tan aplastado quedé por el peso de mis fantasías en torno a lo que había descubierto, que a cada instante éstas iban en aumento aunque ya era tiempo de serenarme. De pronto se me ocurría que la huella era del diablo, y hasta encontraba apoyo razonable a tal suposición, porque ¿cómo podía haber llegado otra criatura con forma humana a la isla? ¿Dónde estaba el barco que la trajo? ¿Por qué no había otras señales de su paso? ¿De qué manera había podido un hombre llegar allí? Pero casi de inmediato me ponía a pensar lo contrario. ¿Por qué iba Satanás a adoptar forma humana en aquella playa donde nada había que pudiera interesarle? ¿Y por qué dejar su única huella en un sitio donde no había seguridad ninguna de que yo alcanzara a verla? Nada de eso tenía consistencia. Me dije que el diablo conocía infinidad de maneras más efectivas para aterrorizarme —si se lo hubiera propuesto— que dejar una señal en la playa; por otra parte, habitando yo en el extremo opuesto de la isla, ¿no hubiera sido más lógico que estampara allí la huella y no en un sitio donde había diez mil probabilidades contra una de que no la viera? ¿Y por qué en la arena, donde el primer embate del mar la borraría sin dejar rastro? Todo esto parecía incoherente ante el hecho mismo y la idea que habitualmente nos formamos de la sutileza del demonio.
    Estos argumentos me ayudaron a desterrar la idea de que fuera el diablo, y por ellos llegué a la conclusión de que se trataba de algo peor, es decir, algunos de los salvajes del continente próximo que, navegando en sus canoas, hubieran sido arrastrados por las corrientes o vientos contrarios hasta la costa, donde después de recorrerla habían vuelto a embarcarse quizá, tan poco deseosos de quedar en la desolada isla como yo de que lo hicieran.
    Mientras tales reflexiones ocupaban mi mente, me sentí profundamente reconocido por la fortuna que había tenido de no estar justamente en aquella parte de la isla, y que los salvajes no hubieran visto mi bote por el cual habrían descubierto la presencia de habitantes y acaso intentado su búsqueda. De ahí pasé a imaginarme con mortal terror que acaso habían dado con el bote, y que adivinando que la isla estaba poblada volverían en gran número para devorarme; aun suponiendo que lograra esconderme, lo mismo descubrirían mi vivienda, destruirían mis plantaciones, llevándose todas las cabras y dejándome morir al fin de inanición.
    Mis esperanzas en lo divino parecían disiparse bajo la fuerza del miedo. Toda mi confianza en Dios, fundada en las prodigiosas pruebas que había tenido de Su bondad, se desvanecieron. ¡Como si El, que hasta entonces me había alimentado milagrosamente, no tuviera poder suficiente para preservar los bienes que su bondad me había concedido!
    Me reproché no haber sembrado más semilla que la necesaria para sustentarme hasta la siguiente estación, como si nada pudiera suceder que me impidiera cosechar cada vez el grano. Tan fundado me pareció este reproche que decidí para el futuro acumular semilla suficiente para dos o tres años, a fin de no morir de hambre viniera lo que viniese.
    Reflexionando luego que Dios no sólo era justo sino todopoderoso, deduje que así como había dispuesto castigarme y afligirme, lo mismo podía salvarme si lo quería; y que si no era esa Su voluntad, mi deber estaba en someterme absoluta y enteramente a esa voluntad, al mismo tiempo que poner en ella toda mi esperanza, rogar al Señor y someterme a los dictados y decretos de Su providencia.
    Estos pensamientos me absorbieron durante horas y días, y hasta puedo decir semanas y meses. No debo omitir uno de ellos en particular; cierta mañana, mientras meditaba en mi lecho sobre los peligros que me acechaban a causa de los salvajes, me sentí hondamente afligido; pero en ese momento surgió en mi mente la palabra de la Escritura: Invócame en los días de aflicción, y yo te libraré, y tú me alabarás.
    En medio de estas meditaciones, terrores y conjeturas, se me ocurrió un día que acaso era víctima de las quimeras de mi imaginación. ¿No habría marcado yo mismo la huella en la arena el día en que desembarqué del bote en aquella playa? Esto me animó un poco y empecé a persuadirme de que sufría una ilusión y que aquel pie en la arena era el mío. ¿Acaso no podía haber andado por ese camino al salir de la piragua, cuando para volver a ella tomaba por ahí? Me dije que de ninguna manera podía recordar con exactitud el lugar por donde caminara aquella vez, y que si al final resultaba que la huella era mía, estaba haciendo lo que esos tontos que cuentan historias de fantasmas y apariciones y terminan por ser los primeros en asustarse de ellas.
    Esto me devolvió algo de coraje, y me puse a hacer pequeñas excursiones por los alrededores; llevaba tres días con sus noches sin salir del castillo y me faltaban alimentos porque no tenía a mano más que algunas galletas de cebada y un poco de agua. Recordé que debía ordeñar mis cabras, lo que antes era mi entretenimiento vespertino. Las pobres bestias habían padecido mucho por falta de cuidado, y a algunas se les había secado la leche.
    Alentándome con la creencia de que la huella provenía de mi propio pie, y que en realidad me había asustado de mi sombra, volví a salir y fui a mi casa de campo para ordeñar las cabras. ¡Pero con qué miedo avanzaba, cuan a menudo me daba vuelta para mirar a mis espaldas y cómo me aprontaba a arrojar la canasta a la primera alarma y correr para salvar la vida! Cualquiera que hubiese podido verme habría pensado que el remordimiento me perseguía, o que acababa de pasar por un miedo espantoso, lo que en verdad era así.
    Con todo, después que hube hecho el viaje dos o tres veces sin ver nada de inquietante, empecé a sentirme más animoso y a persuadirme de que todo aquello era simple producto de la imaginación. Nada de esto bastaba sin embargo para calmarme enteramente; era necesario volver a la playa, buscar la huella y comparándola con mi pie adquirir el convencimiento de que coincidía con mi pisada y era por lo tanto mía. Pero me bastó llegar allí para darme cuenta, en primer término, que al desembarcar del bote no había podido alejarme en dirección hacia donde estaba la huella; y luego, al compararla con mi pisada, descubrí que la misteriosa señal era mucho más grande. Ambas revelaciones volvieron a hundirme en el fantaseo más desatinado, y tal fue su violencia que me estremecía con escalofríos como si tuviese calentura. Volví a casa plenamente convencido de que un hombre, o muchos, habían desembarcado en aquella costa, salvo que en realidad la isla estuviera habitada, lo cual me exponía a ser atacado antes de volver de mi sorpresa. Tenía que defenderme a toda costa. Pero ¿cómo?
    Tal confusión de pensamientos me tuvo despierto la noche entera, aunque de mañana conseguí dormirme; la agitación de mi mente así como la angustia de mi espíritu me habían fatigado de tal manera que dormí profundamente y al despertar me sentí mucho mejor y más animado que antes. Principié a pensar serenamente, y luego de profundas reflexiones llegué a la conclusión de que esta isla tan hermosa, tan fértil y relativamente cercana al continente, no estaba abandonada como yo había supuesto hasta entonces. Cierto que no vivían en ella residentes fijos, pero era probable que con cierta frecuencia arribaran canoas a su costa, acaso deliberadamente o tal vez arrastradas por vientos contrarios. Llevaba yo allí quince años y jamás había visto la sombra de un ser humano, por lo que podía inferir que a poco de llegar a tierra volvían a embarcarse, sin haber mostrado hasta ahora la menor intención de permanecer en la isla. El peligro que podía amenazarme radicaba, pues, en algún desembarco occidental de esos errantes pueblos de mar, desembarco que ocurriría ciertamente contra su voluntad, lo que era fácil de advertir en su prisa por volverse al océano, permaneciendo sólo una noche en la costa hasta que la marea y la luz del día los ayudaban a reanudar el viaje. En vista de todo eso no me quedaba más que buscar algún sitio seguro donde refugiarme si los salvajes tocaban tierra.
    Me arrepentí inmediatamente de haber hecho la cueva tan profunda que la salida daba más allá de la empalizada que constituía mi fortificación. Medité el modo de evitar este peligro y resolví levantar una segunda línea de defensa, también en semicírculo, justamente donde doce años atrás plantara una doble hilera de árboles. Tan juntos los había puesto que me bastó intercalar unas pocas estacas entre ellos para dar al conjunto una extraordinaria solidez.
    Tenía, pues, una doble muralla de defensa; la exterior estaba reforzada con tablones, cables viejos y todo lo que sirviera para darle más resistencia y en ella había practicado siete orificios grandes como para pasar el brazo. Del lado interior acumulé tierra que extraía de la cueva, apisonándola fuertemente hasta lograr en la base un espesor de diez pies; luego puse en los mencionados orificios siete mosquetes que, como ya he narrado, había podido sacar del barco. Estaban sostenidos por horcones que hacían de cureñas como en los cañones, de manera que resultaba posible disparar toda la artillería en unos dos minutos. Me llevó muchos meses terminar aquella empalizada, pero no me sentí seguro hasta que la vi concluida.
    Hecho esto planté más allá de la muralla y en una gran extensión multitud de estacas de un árbol parecido al sauce mimbrero, que crece con gran prontitud y es muy sólido. Creo que puse cerca de veinte mil estacas, cuidando de dejar un claro entre ellas y la muralla para tener visibilidad del enemigo y evitar al mismo tiempo que se protegiera entre los árboles para asaltar la empalizada.
    A los dos años tenía formado un tupido seto, y cinco o seis años más tarde se había convertido en un verdadero bosque delante de mi morada, tan espeso y compacto que resultaba absolutamente intransitable. Ningún ser humano, sea quien fuere, podría haber imaginado que detrás de aquella selva había una vivienda. En cuanto a la manera de entrar y salir, cuidé de no dejar señal ni paso alguno. Colocaba una escalera hasta la parte baja de la roca donde había lugar para apoyar una segunda, de manera que cuando había retirado las dos escaleras nadie hubiese podido llegar hasta mí sin destrozarse; y aun llegando, se habría encontrado fuera de mi muralla exterior.
    Había, pues, adoptado todas las precauciones que la prudencia humana podía aconsejar para mi propia seguridad, y pronto se verá que no estaban del todo injustificadas, bien que en aquel entonces sólo preveía vagamente lo que mi miedo me insinuaba.





    10. LOS CANÍBALES




    Mientras me ocupaba en las cosas ya descritas, no descuidé sin embargo el resto de mis trabajos. Lo que más me preocupaba era la cuestión del pequeño rebaño de cabras. Para ese entonces, no solamente me daban carne en abundancia y proveían a mis necesidades sin tener que gastar pólvora y balas, sino que me eximían de la difícil caza de las cabras salvajes. Es por eso que sentía profunda inquietud ante la idea de perder aquellas ventajas y verme obligado a principiar nuevamente la domesticación.
    Pensándolo mucho, no vi más que dos caminos en ese sentido. Uno de ellos era encontrar sitio adecuado para excavar una caverna bajo tierra a fin de recoger allí las cabras por la noche; el otro consistía en cercar dos trozos de terreno, lejos uno del otro y lo más ocultos posible, donde pudiera yo criar una media docena de cabras. En esa forma, si alguna desgracia le ocurría al rebaño mayor, podría renovarlo pronto sin mucha fatiga. Cierto que esto último requería gran trabajo, pero me pareció la mejor de las dos soluciones.
    Pasé, pues, un tiempo en buscar sitio adecuado en los lugares más remotos de la isla y por fin di con uno que reunía todas las condiciones que podía desear. Era una porción de tierra húmeda, en medio del profundo y espeso bosque donde, como ya he contado, me perdí una vez cuando trataba de volver del lado oriental de la isla. Eran unos tres acres, tan rodeados de bosque que parecía provisto de cerco por la misma naturaleza. Gracias a eso la tarea de hacer el vallado no sería tan fatigosa como en los otros lugares elegidos anteriormente por mí.
    Inmediatamente me puse a trabajar, y en menos de un mes lo había cercado de tal manera que mis cabras, que eran mucho menos salvajes de lo que podría imaginarse, estuvieron en lugar seguro. Llevé ahí diez cabras jóvenes y dos machos cabríos, sin querer perder más tiempo, y cuando los tuve allí me dediqué a perfeccionar el vallado hasta que quedó tan seguro como el otro, el cual había sido levantado con menos prisa y empleando mucho más tiempo.
    ¡Pensar que todas estas fatigas tenían por única causa la huella de un pie humano sobre la arena! Hasta ese momento no había encontrado otra señal de presencia extraña en la isla. Dos años llevaba viviendo bajo esa preocupación constante que, como es de imaginar, tornó mi vida mucho menos apacible de lo que era antes; cualquiera que haya vivido obsesionado por el terror al hombre puede concebirlo. Aunque me duela decirlo, la confusión de mi espíritu era tanta que hasta se reflejaba sobre el lado religioso de mis pensamientos; el horror de caer en manos de salvajes y caníbales era tal que raramente me sentía en disposición de elevarme hacia Dios, por lo menos con aquella calma y resignación de espíritu necesarias a tal fin.
    Pero prosigamos. Luego de haber asegurado la existencia de mi pequeño rebaño, empecé a explorar nuevamente para descubrir otro sitio análogo. Me hallaba en una ocasión en el lado occidental de la isla cuando, al mirar hacia el océano, me pareció distinguir una embarcación a gran distancia. Tenía uno o dos anteojos que había encontrado en los arcones de marinero salvados del naufragio, pero no llevaba ninguno conmigo, y el barco, si lo era, estaba a una distancia que no me permitía distinguirlo bien, aunque miré con tal fijeza que mis ojos se fatigaron. Ignoro si se trataba o no de un barco, pero como al descender de la colina ya no lo divisé más, no quise seguir pensando en ello; con todo me propuse no volver a salir sin uno de los anteojos.
    Descendiendo la colina hacia la extremidad de la isla —donde jamás había estado anteriormente— me convencí de que la huella de un pie humano en la costa no era una cosa tan extraña como me había parecido al principio. Si la providencia no me hubiera hecho la merced de depositarme en la parte de la costa donde jamás desembarcaban los salvajes, hubiera advertido en seguida que nada era más frecuente para aquellas canoas arrastradas mar afuera que tocar tierra en este lado y procurarse refugio. Asimismo, como los tripulantes de las piraguas frecuentemente se abordaban y combatían entre sí, los vencedores traían a sus prisioneros a la costa donde, de acuerdo con sus horrorosas costumbres de antropófagos, los mataban y comían como se verá a continuación.
    Apenas había descendido de la colina a la playa, en la parte SO de la isla, cuando me sentí presa del espanto. ¿Cómo traducir la confusión y el terror de mi mente al ver la costa sembrada de cráneos, manos, pies y otros huesos humanos? A un lado se veían señales de que habían hecho fuego, y en su torno una especie de círculo como el corral de las luchas de gallos, en el cual sin duda se habían sentado aquellos salvajes para efectuar sus inhumanos festines con la carne de sus semejantes.
    Tan aterrado permanecía mirando aquellas cosas que ni siquiera pensé que pudiera encontrarme en peligro. Todas mis aprensiones desaparecieron a la vista de semejante colmo de monstruosa, infernal brutalidad, ante el horror de la degeneración humana llegada a tal punto. Muchas veces había oído hablar de los caníbales, pero nunca me había sido dado ver una cosa semejante. Por fin aparté el rostro de tan atroz espectáculo, y trepando rápidamente la colina me volví de inmediato a casa.
    Cuando me hube alejado algo de esa parte de la isla, me detuve como paralizado; entonces, recobrando mis sentidos, miré hacia el cielo con profundo reconocimiento y dejé que corrieran mis lágrimas mientras daba gracias a Dios por haberme hecho nacer en un lugar del mundo tan diferente del de aquellos espantosos seres.
    Lleno de gratitud volví a mi castillo y empecé a sentirme mucho más seguro bajo tales circunstancias que unos años antes. Comprendía que aquellos salvajes jamás arribaban a la isla en procura de algo; probablemente no esperaban encontrar gran cosa en ella, y si habían explorado como era muy natural la parte boscosa de la misma, debían sentirse desilusionados al no hallar nada que les conviniera. Me animaba la idea de que llevaba allí casi dieciocho años sin haber visto jamás la menor presencia humana, y que por lo tanto podría vivir otros dieciocho años tan oculto como hasta ahora, salvo que me dejara descubrir o sorprender por los salvajes; mi ocupación primordial debía consistir por lo tanto en mantenerme oculto, salvo que la suerte trajera a aquella tierra otras gentes mejores que los caníbales.
    Pese a estas ideas conservé una repugnancia tan grande hacia los salvajes, y me causaba tal horror su costumbre de devorarse unos a otros, que seguí pensativo y melancólico, casi sin salir de mis fundos por espacio de dos años. Me refiero a mi castillo, la casa de campo o enramada, y el corral oculto en los bosques. A este último sólo iba para cuidar de las cabras, ya que la aversión que sentía hacia aquellos diabólicos salvajes era tal que tenía miedo de encontrarme con ellos como con el demonio.
    El tiempo y la seguridad de que no sería descubierto lograron quitarme poco a poco aquella ansiedad, y llegué a vivir de la misma manera que antes, con la única diferencia que me mostraba más precavido y nunca salía sin tomar medidas para no ser sorprendido por algún salvaje. Cuidaba de modo especial no disparar inútilmente la escopeta, por temor a que oyeran el tiro si acertaban a hallarse en la isla. Me alegraba profundamente haber tenido la precaución de domesticar un rebaño dé cabras, cosa que tornaba innecesaria toda caza en los bosques. Si capturaba algunas a veces, era mediante las trampas que me habían permitido iniciar mi rebaño, y creo que por espacio de dos años a partir de lo narrado no disparé una sola vez la escopeta aunque la llevaba siempre conmigo. A mi armamento agregué las tres pistolas que salvara del barco, o por lo menos dos, que llevaba sujetas a mi cinturón de piel de cabra. También me colgué al cinto, con ayuda de un tahalí, uno de los grandes machetes que encontrara a bordo, de manera que mi aspecto debía ser formidable cuando emprendía cualquier viaje si a la descripción ya hecha de mi indumentaria y equipo se agregan ahora las dos pistolas y el gran sable colgando sin vaina a mi costado.
    A medida que pasaba el tiempo, y aparte de las precauciones mencionadas, volvía yo a mi antigua vida apacible y sosegada. Todo ello servía para mostrarme, más que nunca, qué lejos estaba mi condición de ser desesperada en comparación a la de otros, y cómo Dios, de haberlo querido, me hubiera reducido a una miseria infinitamente peor. Reflexioné entonces cuan pocas protestas habría entre los hombres de cualquier condición si tuvieran la prudencia de comparar sus vidas con otras más desdichadas, y sentirse agradecidos en vez de mirar a aquellos que se hallan por encima y creerse así con derecho a murmurar y quejarse.
    En mi actual situación no carecía de nada que me fuera indispensable, pero era tal el miedo y la inquietud que me produjeran los salvajes, como la necesidad de ocuparme de mi seguridad, que llegué a pensar que mi ingenio para procurarme nuevas cosas se había agotado. Abandoné un proyecto que anteriormente me preocupara mucho: intentar la transformación en malta de una parte de mi cebada, a fin de obtener cerveza.
    Mi ingenio, sin embargo, se explayaba en otro sentido; no dejé de pensar un momento en el modo de destruir a algunos de esos monstruos cuando estuvieran entregados a su sangriento festín, y si fuera posible salvar a la víctima que iban a inmolar. Llenaría un volumen mucho mayor que el presente el relatar todas las ideas que se me ocurrieron, y que rumiaba incesantemente, para destruir a aquellos salvajes o al menos aterrarlos de tal modo que jamás volvieran a aproximarse a la isla. Pero ninguna me parecía aceptable. Además, ¿qué podía hacer un hombre contra tantos, si acaso desembarcaban veinte o treinta armados de sus dardos, o arco y flechas, con los cuales podían tirar tan eficazmente como yo con mi escopeta?
    Una vez se me ocurrió hacer una excavación debajo del sitio donde encendían la hoguera y poner allí cinco o seis libras de pólvora, con lo cual apenas se dispusieran a comer volarían todos en pedazos. Pero, en primer lugar, me disgustaba la idea de gastar en ellos tanta pólvora, ya que apenas me quedaba un barril, y luego no estaba seguro de que la explosión se produciría en el momento debido para sorprenderlos; acaso alcanzara a aturdirlos y aterrarlos, pero sin fuerza suficiente como para que abandonaran el lugar.
    Deseché, pues, el proyecto y me propuse en cambio emboscarme en algún sitio conveniente con las tres escopetas y doble carga en cada una, esperando que estuvieran congregados para su sangriento festín; entonces podría disparar sobre ellos con la certeza de que cada tiro mataría o dejaría mal heridos a dos o tres, lanzándome finalmente al asalto con las pistolas y el machete. Tenía la seguridad de que en esa forma era posible dar cuenta hasta de veinte salvajes, y esta fantasía me complació tanto que la abrigué durante semanas; me absorbía a tal punto que hasta soñaba con ella, y frecuentemente me parecía que ya iba a lanzarme sobre la horda de caníbales.
    Tan lejos llevé el deseo de poner en práctica mi idea que anduve buscando los lugares indicados para emboscarme y espiar sus movimientos; volví muchas veces a aquel sitio, que ya me iba resultando familiar; y especialmente cuando mi cerebro estaba inflamado con ideas de venganza que me movían a exterminar sin piedad a veinte o treinta de ellos, el horror que me inspiraba ese sitio, con todos los restos de aquellos espantosos festines, apenas si atemperaba mi cólera.
    Por fin encontré un apostadero a un lado de la colina donde me pareció posible esperar a cubierto que alguna canoa se aproximase a la costa; desde allí, y antes de que los salvajes hubieran tenido tiempo de desembarcar, podía deslizarme sin ver visto entre los árboles hasta una concavidad que me cubría completamente; era un excelente puesto para tomar posición, observar en detalle sus sangrientos preparativos y hacer puntería sobre sus cabezas cuando estuvieran congregados, con tal precisión que no dudaba alcanzaría a dos o tres con cada disparo.
    Resolví, pues, fijar allí mi escondite, y de acuerdo con el plan preparé convenientemente dos mosquetes y mi escopeta de caza. Cargué los mosquetes con un puñado de pedazos de plomo y cuatro o cinco balas de pistola; a la escopeta le puse abundantes balines de grueso calibre, y finalmente cargué las pistolas con cuatro balas. Así artillado, y teniendo abundante munición para una segunda y tercera carga, completé los preparativos para el ataque.
    Luego de haber planeado los detalles y hasta haberlos puesto en práctica en mi imaginación, diariamente me iba a la cresta de la colina que quedaba a unas tres millas de mi castillo, para otear el océano y descubrir si había alguna embarcación que se aproximara a la isla. A los dos o tres meses de este cansador ejercicio empecé a fatigarme de él, ya que regresaba sin haber descubierto nada, no solamente en la isla sino en la vasta extensión del mar hacia el cual se dirigían mis ojos y mi catalejo.
    Mientras practiqué diariamente el viaje de reconocimiento a la colina, mantuve vivo el deseo de poner mi plan en práctica; me parecía absolutamente natural matar veinte o treinta salvajes desnudos por un crimen que no había entrado a discutir, dejándome llevar por el horror que me producían las monstruosas costumbres de aquellos pueblos.
    Pero cuando lo medité con más serenidad, necesariamente tenía que llegar a la conclusión de que estaba equivocado. Aquellos salvajes no eran más asesinos, en el sentido que me llevara antes a condenarlos mentalmente, que aquellos cristianos que frecuentemente sentencian a muerte prisioneros apresados en la batalla; o aquellos otros que, en tantas ocasiones, pasan a cuchillo batallones enteros sin querer darles cuartel a pesar de haber rendido las armas.
    En segundo término se me ocurrió que, aunque se devoraban unos a otros, nada de eso debía importarme. ¿Qué injurias me habían hecho aquellas gentes? Si atentaban contra mí, si yo veía que para preservarme de su ataque era conveniente caer sobre ellos, entonces se justificaría mi acción; pero hasta ahora me hallaba a salvo y ni siquiera mi existencia les era conocida, por lo cual no era justo precipitarme como lo proyectaba.
    Estas consideraciones me hicieron vacilar al principio y después me detuvieron completamente en mis planes; poco a poco los abandoné convenciéndome a la larga que había estado equivocado al resolverme a exterminar a los salvajes. No me correspondía mezclarme en sus asuntos si no me atacaban primero, y mi deber era solamente tratar de impedir esto; si de todos modos el ataque se producía, entonces quedaba en libertad de acción para repelerlo.
    Por otra parte llegué a darme cuenta de que mi proyecto no era precisamente un modo de asegurarme la tranquilidad, sino, por el contrario, acarrearme la peor de las catástrofes a menos que tuviese la seguridad de matar, no solamente a los que estuviesen en tierra en ese instante, sino a los que pudieran venir más tarde; porque estaba claro que si uno solo conseguía escapar se apresuraría a ir con la noticia a su pueblo, y pronto invadirían por millares la isla a fin de tomarse venganza por la muerte de sus semejantes. Comprendí que era atraerme la destrucción, mientras que hasta el presente nada tenía que temer de aquellos caníbales.
    En fin, por un doble motivo, moral y práctico, vi la conveniencia de mantenerme al margen de sus vidas. Mi tarea consistía en ocultarme a su vista por todos los medios, no dejando la menor señal que les permitiese sospechar en la isla la existencia de un ser humano.
    Unida aquí la religión a la prudencia, pronto adquirí la convicción de que había estado en un perfecto error cuando tramaba mis sangrientas venganzas contra aquellos seres inocentes (inocentes en lo que a mí respecta). Con sus culpas y crímenes personales nada tenía yo que ver; eran cuestiones concernientes a sus hábitos nacionales, y yo debía librarlos a la justicia de Dios, que es el Gobernador de las naciones y sabe cómo, con castigos adecuados, penar a quienes ofenden Su ley y juzgar públicamente y de acuerdo con Sus designios a quienes también públicamente han cometido las ofensas.
    Aclarados mis pensamientos al respecto, viví durante otro año con tan pocos deseos de estorbar a aquellos miserables que en todo ese tiempo no fui ni una sola vez a la cresta de la colina para observar si habían desembarcado o si estaban a la vista; temía no poder resistir la tentación de renovar mi cólera o sentirme arrastrado por las circunstancias a caer sobre ellos. Me ocupé en cambio de llevar a otra parte mi canoa, y sacándola de su caleta la conduje hasta el extremo oriental de la isla, donde la dejé a cubierto en una pequeña ensenada al abrigo de las rocas, seguro de que los salvajes, por temor a las corrientes, jamás sé atreverían a acercarse a un sitio semejante.
    Con el bote me llevé todo lo que había dejado cerca de él y que le pertenecía, tal como el mástil y la vela especialmente construida para impulsarlo, y una especie de ancla que no sé si merecía llamarse así o solamente rezón. Todo eso fue ocultado de manera que no quedase ni sombra que guiara a descubrirlo, así como la menor apariencia de bote o de habitación humana en la isla entera.
    Aparte de eso continué haciendo una vida todavía más retirada que antes; salía solamente para mis tareas cotidianas, es decir, ordeñar las cabras y cuidar del pequeño rebaño que tenía en los bosques y que, hallándose en el otro extremo de la isla, se encontraba perfectamente a salvo. Estaba seguro de que los salvajes, pese a acercarse a veces a la isla, no lo hacían con la esperanza de hallar nada en ella y por tanto cuidaban de no alejarse de la costa; tampoco me cabía duda de que habían vuelto varias veces a tierra después que mi descubrimiento me tornara tan cauteloso. A veces pensaba con espanto en lo que hubiera sido de mí al darme inesperadamente de boca con ellos, en la época en que sin más defensa que la escopeta —y ésta apenas con algunos balines— me paseaba sin cuidado por mis dominios. ¿Qué hubiera podido hacer si en vez de descubrir la huella de un pie humano me hubiese encontrado de pronto frente a quince o veinte salvajes que, a la velocidad que son capaces de correr, me hubieran apresado inmediatamente?
    Confío en que el lector de esta narración no hallará extraño que le confiese hasta qué punto aquellas ansiedades, ese constante peligro en que vivía ahora y las muchas preocupaciones que se cernían sobre mí, agotaron mi capacidad inventiva para las tantas cosas que antaño proyectara en busca de mayor comodidad. Necesitaba ahora mis manos más para procurarme seguridad que alimentos; no me atrevía a clavar un clavo o a cortar un pedazo de madera por miedo a que el ruido fuera escuchado. Mucho menos me atrevía a disparar la escopeta y, por sobre todo ello, buscaba no encender fuego por temor a que el humo, visible de día a gran distancia, me traicionara. Trasladé, pues, aquellas tareas que requerían el empleo del fuego, tal como la cocción de cacharros y tinajas, al abrigo de los bosques, donde después de estar cierto tiempo hallé con indescriptible alegría una enorme caverna natural en la entraña de la tierra, que parecía extenderse profundamente y donde me atrevería a decir que ningún salvaje se hubiera aventurado nunca a penetrar; incluso era capaz de aterrar a cualquiera, salvo a mí, que tanto la necesitaba como escondite.
    La boca de la caverna daba al pie de un gran peñasco donde se hubiera dicho que por casualidad (si no tuviera yo bastante motivo para considerar tales cosas como obra de la Providencia) me encontraba un día cortando algunas ramas gruesas para hacer carbón de leña. Quiero, antes de proseguir, explicar por qué hacía carbón y la razón es simple: evitar a toda costa que el humo me denunciara. Como no me era posible vivir sin hornear el pan, cocer mis alimentos y demás, me ingenié entonces en quemar leña bajo tierra como lo había visto hacer en Inglaterra, hasta que se carbonizara; luego, apagando el fuego, retiraba el carbón y lo llevaba a casa, donde podía utilizarlo sin peligro de humo.
    Pero dejemos esto. Cortando leña un día, observé que detrás de una espesa ramazón de arbustos bajos había como un hundimiento en el peñasco. La curiosidad me movió a acercarme, y cuando tras no poca dificultad llegué delante de aquella boca vi que era muy honda y lo bastante alta para estar de pie en el interior un hombre de mi estatura o aún más alto. Debo confesar que salí de allí con más apuro del que había entrado al divisar en la absoluta oscuridad del interior unos ojos brillantes clavados en mí, ojos que no sabía si eran del diablo o de un ser humano y que brillaban como dos estrellas, al reflejar la luz de la abertura.
    Reuniendo todo mi valor y tratando de darme ánimo con la idea de que el poder y la presencia de Dios están en todas partes y me protegerían, avancé unos pasos alumbrándome con una tea que sostenía por encima de mi cabeza; en el suelo yacía un enorme y espantoso macho cabrío, respirando anhelante y haciendo ya, como suele decirse, su testamento, pues estaba en las últimas a fuerza de viejo.
    Lo hostigué para ver si conseguía echarlo de la cueva, pero aunque hizo esfuerzos por levantarse no lo consiguió; no quise entonces molestarlo pensando que si tanto me había asustado aterraría aún más a cualquier salvaje que osara acercarse a la boca de la cueva mientras el animal se conservara con vida.
    Ya curado de mi temor empecé a reconocer la caverna, que era muy pequeña; tendría unos doce pies de diámetro, pero no es posible hablar de su forma, ya que no era ni cuadrada ni circular, siendo en un todo la obra de la Naturaleza. Reparé en que hacia el lado más profundo aparecía una segunda abertura, pero para pasar por allí hubiese sido necesario arrastrarme sobre pies y manos y yo ignoraba hacia dónde me llevaría. Renunciando por el momento a reconocer el segundo compartimiento, me propuse retornar al día siguiente con algunas velas y un yesquero que había sacado de la llave de un mosquete, pensando emplear el mixto de la cazoleta para encenderlo.
    Volví, pues, al otro día provisto de seis grandes velas hechas con cebo de cabra y que alumbraban muy bien; penetrando por la segunda abertura, tuve que arrastrarme por espacio de unas diez yardas, cosa que dicho sea de paso era harto aventurada, ya que no sabía hacía dónde me llevaba el pasadizo ni lo que encontraría al final. Por fin noté que el techo se elevaba hasta cerca de veinte pies, y me vi frente al espectáculo más hermoso que jamás contemplara en la isla. Iluminadas por la luz de dos velas, las paredes de la caverna, así como el techo, devolvían la luz en mil reflejos maravillosos. ¿Qué había en la roca? ¿Diamantes, piedras preciosas, acaso oro como me parecía sospechar? No podía decirlo a ciencia cierta.
    El lugar en que me encontraba era una admirable cavidad o gruta, aunque absolutamente oscura. El suelo, seco y llano, aparecía cubierto de una ligera capa de arena suelta, sin que en parte alguna se vieran animales venenosos; mirando hacia las paredes tampoco noté en ellas la menor huella de humedad. La única dificultad era la entrada, pero meditando que aquella caverna podía ser el sitio indicado para estar a salvo de los salvajes, me pareció que resultaba una ventaja. Profundamente regocijado con mi descubrimiento me resolví sin perder tiempo a trasladar a la gruta las cosas cuya seguridad me interesaba de modo especial; en primer término mis reservas de pólvora y todas las armas que no empleaba, es decir, dos escopetas y tres de los ocho mosquetes. En el castillo dejé cinco montados en las ya descritas cureñas, listos para tirar desde la empalizada; también podían servirme en cualquier expedición que emprendiera.
    En oportunidad de llevar mis municiones a la caverna, se me ocurrió abrir el barril que había salvado del mar y cuya pólvora estaba mojada. Al hacerlo comprobé que el agua había penetrado tres o cuatro pulgadas en la masa de pólvora y que la porción mojada, endureciéndose como una costra, había preservado del agua el resto como si fuera el corazón de un fruto. Tenía, pues, a mi disposición cerca de sesenta libras de excelente pólvora que extraje del centro del casco. Muy agradable sorpresa fue para mí en las circunstancias en que me encontraba, y llevándome todo a la gruta dejé en el castillo apenas dos o tres libras de pólvora para evitar cualquier sorpresa. Igualmente puse a salvo el plomo que me quedaba para hacer balas.
    Me complacía ahora imaginarme como uno de aquellos gigantes legendarios que moraban en cavernas y grutas a las cuales nadie podía llegar; estaba persuadido de que aunque quinientos salvajes anduvieran tras de mí, jamás descubrirían mi paradero y en el peor de los casos no se atreverían a atacarme en mi refugio.
    El viejo macho cabrío agonizante murió a la mañana siguiente de mi descubrimiento. Me pareció más simple excavar una sepultura en la misma cueva y cubrirlo bien de tierra, que arrojarlo al exterior.
    Se cumplían ya los veintitrés años de mi residencia en la isla; tan habituado me sentía a ella y a mi manera de vivir, que de haber tenido la certidumbre de que los salvajes no vendrían a estorbarme hubiera aceptado pasar en ella el resto de mi existencia, aunque al fin tuviese que tenderme en el suelo y esperar la muerte como el viejo macho cabrío de la caverna. Hasta había llegado a imaginar algunas diversiones y entretenimientos que me ayudaban a pasar el tiempo de modo mucho más agradable que en otras épocas. En primer término ya he contado que enseñé a hablar a Poli, y llegó a hacerlo tan bien, me hablaba tan familiarmente y con tanta claridad, que resultaba encantador; estuvo a mi lado nada menos que veintiséis años, e ignoro si vivió todavía más. En el Brasil afirman que estos animales alcanzan una existencia de un siglo, y tal vez mi Poli sigue aún viviendo en la isla, llamando al pobre Robin Crusoe. No deseo a ningún inglés la mala suerte de andar por ahí y escucharlo hablar, porque con seguridad creerá hallarse en presencia del mismo demonio.
    Mi perro fue un excelente y cariñoso compañero por espacio de dieciséis años, hasta que murió de viejo. En cuanto a los gatos, ya he dicho que se habían multiplicado tanto que tuve que matar a muchos para impedir que devoraran cuanto tenía; después, cuando murieron los dos más viejos que traje del barco, hostigué tanto a los otros sin darles el menor alimento que terminaron por huir al bosque y hacerse salvajes, excepto dos o tres favoritos que conservé a mi lado y cuyas crías me apresuraba a ahogar apenas nacidas. Fuera de estos animales tenía siempre conmigo dos o tres cabritos mansos a los que había enseñado que comieran de mi mano. Tenía también otros dos papagayos a quienes enseñé a decir mi nombre, pero ninguno podía compararse a Poli; cierto que no me tomé con ellos el trabajo que había dedicado a mi primer papagayo. En mi casa había varios pájaros marinos domesticados, cuyos nombres ignoro y que había capturado en la costa, cortándoles las alas. Las pequeñas estacas que plantara delante del castillo se habían convertido en un espeso seto, y allí vivían mis pájaros anidando entre los árboles más bajos, lo cual me agradaba mucho. Como puede apreciarse, con todo aquello había llegado a considerar mi vida como muy pasable, si sólo hubiera logrado desechar el temor a los salvajes.
    Pero mi suerte disponía otra cosa, y acaso no sea inútil para los que lean esta historia la observación que sigue. ¡Cuántas veces, en el curso de nuestra vida, el mal que con más empeño tratamos de evitar y que nos parece, cuando se precipita sobre nosotros, la más horrible cosa, resulta al fin la verdadera áncora de nuestra salvación, la única puerta por la cual podemos salir de la aflicción que nos embargaba! Muchos ejemplos de esto podría dar en el transcurso de mi extraña vida, pero donde más se manifestó fue en las circunstancias que rodearon mis últimos años de residencia en la isla.





    11. UN NAUFRAGIO Y UN SUEÑO




    Transcurría el mes de diciembre de mi vigésimo tercer año de soledad, y era la época del solsticio austral (porque no puedo darle el nombre de invierno) en la que me ocupaba yo de la recolección del grano, viéndome obligado a permanecer gran parte de mi tiempo en las plantaciones. Una mañana, cuando aún no era día claro y empezaba mi tarea, me sorprendió ver la luz de un fuego en la costa, a unas dos millas hacia el extremo donde primeramente advirtiera la huella de los salvajes, y al mirar con atención comprobé que no se trataba del lado opuesto de la isla, sino de la parte donde yo residía.
    Fue tal el azoramiento que se apoderó de mí que no me atrevía a salir de la enramada por miedo a que me sorprendieran, pero tampoco podía quedarme allí por temor a que los salvajes, errando por los alrededores, encontraran mis sembrados, las parvas de grano o cualquiera de mis otros trabajos, lo que les demostraría de inmediato la existencia de habitantes en el lugar. No dudaba que inmediatamente se pondrían a buscarme sin descanso, de manera que armándome de valor volví al castillo, levanté la escalera una vez que hube pasado, y traté de que todo tuviera el aspecto más salvaje y natural posible.
    Inmediatamente me apresté a la defensa. Cargando lo que yo llamaba mis cañones, es decir, los mosquetes montados sobre horcones, y alistando las pistolas, me resolví a defenderme hasta el último aliento, sin olvidar encomendarme con fervor a la protección divina y rogar ardientemente a Dios que me salvara de las manos de aquellos bárbaros. Así me quedé por espacio de unas dos horas, lleno de impaciencia por saber lo que ocurría más allá y careciendo de exploradores o espías que fuesen a buscar novedades.
    Después de estarme quieto, pensando qué debía hacer en la emergencia, no pude resistir por más tiempo la inactividad, de manera que coloqué la escalera haciéndola llegar como ya he descrito hasta el sitio donde la roca formaba una especie de plataforma; levantando luego la escala y volviéndola a colocar en dicho apoyo, me encaramé a la cresta de la colina. Me había tirado de boca contra el suelo, y con ayuda del anteojo que trajera ex profeso empecé a buscar el sitio donde ardía el fuego. Pronto descubrí que había nueve salvajes desnudos que rodeaban una hoguera, no para calentarse, ya que ninguna falta les hacía el calor en ese clima ardiente, sino probablemente para entregarse a alguno de sus horribles banquetes de carne humana que habrían traído consigo, aunque no alcanzaba a distinguir a los posibles prisioneros.
    Vi dos canoas que habían arrastrado fuera del agua; y como la marea estaba baja, parecían a la espera del flujo para embarcarse nuevamente. No es fácil describir mi estado de ánimo contemplando aquella escena, sobre todo al darme cuenta de que ocurría de este lado de la isla y tan cerca de mí. Pero al comprender que probablemente los desembarcos acontecían en el momento del reflujo, me tranquilicé un poco pensando que me sería posible salir con toda tranquilidad siempre que al empezar la marea no hubiese visto antes aproximarse las canoas. Esto me permitió proseguir con más calma las tareas de la cosecha.
    Ocurrió tal como lo esperaba. Tan pronto creció la marea vi a los salvajes embarcarse y remar (o más bien palear) hacia fuera. Olvidaba decir que durante la hora y media que precedió a su marcha estuvieron bailando en la playa, y que con ayuda de los anteojos pude ver perfectamente sus movimientos y ademanes.
    Tan pronto se alejaron me eché dos escopetas a la espalda, y con dos pistolas al cinto y la gran espada sin vaina al costado, corrí con toda la rapidez posible a la colina donde por primera vez había tenido noticia de los salvajes. Cuando llegué allá, después de dos horas de fatigosa marcha, cargado como estaba con tantas armas, descubrí que en ese lugar habían atracado otras tres piraguas; mirando hacia el mar alcancé a verlas todavía mientras se internaban en el océano.
    Aquello era espantoso de ver, pero algo peor me esperaba cuando descendí a la playa y encontré los restos que después del atroz festín habían quedado diseminados; sangre, huesos, trozos de carne humana que aquellos monstruos habían devorado en medio de danzas y júbilo. Tan lleno de indignación me sentí a la vista del horrendo espectáculo que empecé inmediatamente a premeditar la destrucción de los que desembarcasen una próxima vez en la isla, sin importarme su número.
    Transcurrieron con todo un año y tres meses antes de que volviera a ver a los salvajes, como contaré en su lugar. Es probable sin embargo que vinieran una o dos veces, pero se quedaron muy poco tiempo o yo no tuve noticia de su presencia. En el mes de mayo, según creo recordar, y en el año vigésimo cuarto de mi residencia, tuve un extraño encuentro con ellos que narraré en su debido momento.
    Durante ese intervalo de quince o dieciséis meses, la perturbación de mi espíritu fue grande. Dormía mal, despertándome en medio de terribles pesadillas y sobresaltado. Como de día no abrigaba más que esa constante preocupación, tal inquietud se reflejaba en mis sueños, donde me veía matando salvajes o preguntándome cuál era el motivo para hacerlo. Pero, dejando esto por el momento, diré que a mediados de mayo, creo que el dieciséis según los inseguros datos de mi calendario de madera que yo trataba de mantener al día; el dieciséis, digo, se levantó una gran tormenta de viento, con relámpagos y truenos, y la noche que siguió fue tempestuosa. No recuerdo exactamente las circunstancias, pero sí que me encontraba leyendo la Biblia y meditando seriamente en mi presente condición cuando escuché, viniendo del mar, un sonido semejante al de un cañonazo.
    Sentí una sorpresa muy distinta de las que había experimentado hasta entonces, porque las ideas que aquel cañonazo despertaron en mí eran de naturaleza harto diferente. Me lancé como un rayo fuera de mi tienda, y en un santiamén puse la escalera contra la roca, la retiré, volví a colocarla en el segundo apoyo y me encaramé a la cumbre de la colina en el preciso instante en que un destello me anunciaba el segundo cañonazo, cosa que efectivamente escuché medio minuto más tarde; y por el sonido deduje que venía del lado del mar hacia donde una vez la corriente me arrastrara con el bote.
    De inmediato comprendí que se trataba de un navío en peligro, que tal vez disparaba los cañonazos en demanda de socorro a otro navío que navegaba cerca. Tuve presencia de ánimo para pensar que aunque yo nada podía hacer por ellos, acaso ellos pudiesen hacer mucho por mí, de manera que juntando toda la leña seca que había a mi alcance y encendiéndola, iluminé con una gran hoguera la cumbre de la colina. Aunque el viento era muy fuerte, la madera seca ardió de inmediato, dándome la certeza de que si en verdad un barco navegaba en las cercanías tendría que enterarse de mi presencia. Y no dudo que así fue, porque apenas había alzado la hoguera cuando resonó otro cañonazo y después varios seguidos provenientes del mismo lugar. Mantuve encendido el fuego toda la noche; cuando fue día claro y despejado alcancé a divisar algo a una gran distancia en el mar, hacia el lado este de la isla, aunque no podía decir si era un casco o una vela. Ni siquiera con ayuda del anteojo pude reconocerlo a esa distancia, ya que aún persistía una cierta niebla.
    Miré todo el día en aquella dirección, y no tardé en darme cuenta de que no se movía; evidentemente era un barco fondeado. Ansioso por saciar mi curiosidad, tomé la escopeta y corrí hacia el sur de la isla buscando aquellas rocas donde la corriente me había arrebatado con la canoa. El tiempo estaba muy claro, y trepando a la altura pude ver con toda nitidez y profunda aflicción que el barco había naufragado durante la noche en aquellas rocas ocultas que prolongaban el cabo y que yo había visto desde mi bote; las mismas rocas que, oponiéndose a la violencia de la corriente y haciendo una especie de contracorriente o remolino, me salvaran de la más desesperada situación en que jamás me viera antes.
    Lo que salva a un hombre puede perder a otro. Estaba claro que aquellos marinos, ignorantes de la costa y de los arrecifes, habían sido arrastrados hacia ellos por el fuerte viento que toda la noche soplara del este y E-NE. De haber visto la isla —cosa al parecer muy improbable— lo más lógico era que hubiesen intentado llegar a tierra embarcándose en la chalupa; pero aquellos cañonazos en demanda de auxilio, especialmente después de haber visto, según yo suponía, mi hoguera, me llenaban de ideas contradictorias. Pensé primero que tras de divisar mi fuego se habrían embarcado en el bote del barco y puesto rumbo a la costa, pero que estando el mar embravecido los habría arrastrado lejos. Luego imaginaba que habrían perdido la chalupa antes de encallar, como tantas veces ocurre, en especial cuando el oleaje barre la cubierta y obliga a los marineros a soltar el bote o romperlo para precipitarlo sobre la borda. Después pensé que otro navío, escuchando aquellas llamadas, se habría acercado y recogido a los náufragos. Por fin imaginé a la tripulación mar afuera en la chalupa, arrastrada por la gran corriente marina que la llevaría hacia la desolada extensión del océano donde sólo reina la muerte. Acaso en este instante empezaban a sentir hambre, y pronto estarían en estado de comerse los unos a los otros.
    Todas aquellas eran conjeturas, pero en la situación en que me encontraba yo, ¿qué otra cosa podía hacer sino meditar sobre la desgracia de aquellos hombres y apiadarme de ellos? Una vez más pude comparar por su suerte lo que debía agradecer a Dios, que tanto y tan bien me había asistido en mi desdicha. De dos enteras tripulaciones ahora perdidas en esta región del mundo, ninguna vida se había salvado más que la mía. Aprendí nuevamente que es muy raro que la Providencia de Dios nos abandone a una vida tan baja y miserable como para no tener oportunidades de mostrarnos agradecidos, aunque sólo sea viendo a otros en peores condiciones que nosotros.
    No puedo expresar con ningún lenguaje la ansiedad que se apoderó de mí, la violencia de mis deseos al contemplar el triste espectáculo que me obligó a prorrumpir en exclamaciones:
    — ¡Oh, que por lo menos se hayan salvado uno o dos, aunque solamente sea uno! ¡Que pueda yo tener un compañero, un semejante con el cual hablar, con el cual vivir!
    En todos aquellos años de vida solitaria nunca había sentido una necesidad tan grande de tener compañía; y nunca su falta se tradujo en una melancolía más honda.
    Así estaba dispuesto. Su destino o el mío, acaso ambos, lo prohibían; hasta el último año de mi permanencia en la isla ignoré si alguno se había salvado de la catástrofe. Tuve con todo el dolor de encontrar en la playa, algunos días más tarde, el cadáver de un grumete ahogado. Yacía en la parte próxima al sitio del naufragio y por ropas tenía una chaqueta de marino, un par de calzones abiertos y una camisa de tela azul; no llevaba nada que me permitiera conocer su nacionalidad. Encontré en sus bolsillos dos piezas de a ocho y una pipa, que para mí valía diez veces más que el dinero.
    Había vuelto la calma, y sentí deseos de aventurarme en mi canoa hasta el casco encallado, con la seguridad de encontrar a bordo cosas que me fueran útiles. Lo que más me impulsaba a hacerlo era la esperanza de que en la nave pudiese haber quedado alguien con vida y no sólo me alentaba el deseo de salvar esa vida sino que imaginaba lo que para mí significaría adquirir en esa forma un compañero. Tanto me torturó la idea que no encontraba un instante de paz, ni de día ni de noche, y me repetía que era necesario arriesgarme y llegar hasta el casco. Tan fuerte era mi ansiedad que terminé por encomendarme a la Providencia Divina y pensar que aquel impulso provenía de lo alto, que me equivocaba al resistirlo y que cometería una falta si dejaba transcurrir más tiempo.
    Dominado por una fuerza superior a mí, me apresuré a regresar al castillo y hacer los preparativos del viaje, reuniendo buena cantidad de pan, una tinaja de agua dulce, brújula, una botella de ron del que me quedaba buena cantidad y un canasto de pasas. Cargado con todo aquello fui al sitio donde fondeaba mi bote, achiqué el agua que contenía y después de depositar el cargamento volví en procura de más. Este consistió en un saco grande de arroz, la sombrilla para fijar en la popa, otra tinaja de agua y dos docenas de panecillos de cebada, a lo que agregué también una botella de leche de cabra y un queso. Con gran trabajo pude llevar todo hasta el bote, y rogando a Dios que dirigiera mi rumbo me embarqué de inmediato. Ayudado por los remos y sin apartarme de la costa, llegué por fin al punto extremo de la isla, es decir, al noroeste. Ahora se trataba de penetrar en el océano, de aventurarse o no en la empresa. Miré las rápidas corrientes que corrían a ambos lados de la isla y que tanto terror me producían al recordar el peligro en que estuviera; sentí que mi corazón me abandonaba, porque estaba seguro de que llevado por cualquiera de ellas me internaría de tal modo en el mar que la isla quedaría fuera de mi vista y de mi alcance. Sólo con que se levantara una simple brisa, mi pequeño bote naufragaría irremisiblemente.
    Tanto me angustiaron estos pensamientos que pensé en abandonar la empresa. Llevando el bote hasta una pequeña caleta en la playa, desembarqué y sentado en una eminencia me puse a pensar, abatido y ansioso a la vez, luchando entre el miedo y el deseo. En esta perplejidad advertí que cambiaba la marea y que empezaba el flujo, de manera que mi posible viaje se tornaba impracticable durante muchas horas.
    Decidí entonces trepar al terreno más alto de las inmediaciones para tratar de ver en qué dirección y cómo se movían las corrientes de la marea, a fin de saber a ciencia cierta si, en caso de que mi bote fuese arrastrado mar afuera, la misma marea no podría traerme otra vez a la costa con igual rapidez y fuerza. Apenas había pensado en esta posibilidad cuando ya me encaramaba a una pequeña colina lo bastante elevada para tener visión completa del mar y sus movimientos, buscando calcular qué rumbo debería seguir a mi retorno del casco. Descubrí que así como la corriente de reflujo pasaba rozando el extremo sur de la isla, la motivada por el flujo lo hacía contra la costa del norte, de manera que cuidando de llevar el bote hacia allá podría volverme a tierra sin peligro.
    Animado por mi descubrimiento decidí embarcarme con la marea matinal, y después de haber pernoctado en la canoa al abrigo del capote de marino que ya he mencionado, zarpé temprano. Al comienzo puse rumbo al norte hasta que empecé a sentir la fuerza de la corriente que me arrastró un buen trecho hacia el este, aunque no con la terrible violencia que lo hiciera la corriente austral en la anterior ocasión que me privó de todo gobierno de la canoa. Con ayuda de los remos pude encaminar el bote hacia el sitio del naufragio, y en menos de dos horas me encontraba junto al casco encallado.
    ¡Lamentable espectáculo para mis ojos! El barco, que por sus líneas parecía español, estaba como encajado entre dos rocas, la popa y buena parte de su casco destrozadas por el oleaje; el castillo de proa, incrustado en las rocas, había recibido tal golpe que el palo mayor y el trinquete se quebraron en la base. Sin embargo el bauprés estaba entero y el esperón parecía firme.
    Al acercarme, vi a un perro en la borda que al divisarme aulló y ladró. Apenas lo hube llamado cuando se arrojó al mar y pronto estuvo a bordo casi muerto de hambre y de sed. Le di una galleta y la devoró como un lobo salvaje que llevara dos semanas en la nieve sin comer. Le ofrecí después agua dulce, y bebió tanta que de haberlo dejado hacer su gusto hubiera reventado.
    Subí a bordo; lo primero que alcanzaron a ver mis ojos fueron dos hombres ahogados en la cocina, sobre el castillo de proa; estaban estrechamente abrazados, y comprendí por su actitud que al encallar el buque en medio de la tempestad, tan alto había sido el oleaje y de tal modo barría la cubierta que aquellos infelices no habían podido resistirlo, ahogándose a bordo lo mismo que si hubieran estado bajo el agua. Fuera del perro, nada quedaba con vida en aquel navío; y por lo que alcancé a ver el cargamento estaba averiado. Descubrí algunos cascos de licor, ignoro si vino o aguardiente, que se apilaban en la sentina y eran visibles con la marea baja; pero mis fuerzas no bastaban para moverlos de su lugar. Había también numerosos arcones, pertenecientes sin duda a los tripulantes; eché dos de ellos en mi bote, sin perder tiempo en examinar el contenido.
    Si al encallar el barco se hubiera destrozado la proa en vez de la popa, estoy seguro de que mi viaje habría resultado fructífero, ya que de acuerdo con lo que encontré en los dos arcones el navío tenía muchas riquezas a bordo. Calculando por el rumbo que llevaba en el momento de naufragar, supuse que había sido fletado desde Buenos Aires, o el Río de la Plata, en la parte austral de América más allá del Brasil, y que su destino era La Habana, en el Golfo de México, o tal vez España. Llevaba un gran tesoro a bordo que de nada serviría ya, y el destino de su tripulación era entonces para mí un misterio.
    Aparte de los arcones encontré un pequeño barril de licor de unos veinte galones, que con no poco trabajo puse en el bote. Había muchos mosquetes en una cabina y un frasco de pólvora conteniendo no menos de cuatro libras. Los mosquetes no me eran necesarios, pero sí la pólvora, por lo cual la tomé, así como una pala y tenazas, que me hacían muchísima falta. Di con un par de ollítas de cobre, una chocolatera y unas parrillas, y con ese cargamento, además del perro, emprendí el regreso aprovechando la marea que empezaba a subir. Esa misma tarde, ya entrada la noche, alcancé la isla, donde desembarqué melancólico y fatigado hasta la extenuación.
    Pasé la noche en el bote, y por la mañana decidí guardar mis nuevos efectos en la gran caverna en vez de conducirlos al castillo. Después de alimentarme puse el cargamento en tierra y empecé a examinarlo con detalle. El casco de licor contenía una especie de ron, pero no como el que se bebe en el Brasil, que es harto más bueno. Sin embargo los arcones me consolaron porque contenían diversas cosas de gran utilidad. Por ejemplo encontré una caja de extraordinaria forma, llena de botellitas conteniendo cordiales de excelente calidad y delicado sabor; cada botella tenía unas tres pintas de licor y estaba cerrada con tapón de plata. Había también dos frascos de frutas en almíbar o confitadas, tan bien cerrados que el agua de mar no los había dañado; otros dos, en cambio, estaban averiados. Encontré algunas excelentes camisas que fueron un verdadero regalo, y una docena y media de pañuelos de hilo blanco, así como corbatas de color. Los pañuelos me llenaron de contento, ya que me serían muy útiles para enjugarme el rostro en los días calurosos. Luego, al mirar en el fondo del cofre, vi tres grandes sacos conteniendo piezas de a ocho, lo que daba unas mil piezas en total; en uno de los sacos y envueltos en papel hallé seis doblones y algunas barritas de oro. En conjunto creo que pesaban cerca de una libra.
    El otro arcón contenía también algunas ropas, pero de poco valor; supuse que el cofre era el perteneciente al oficial de artillería, pero aunque busqué pólvora sólo pude dar con tres frascos pequeños conteniendo una pólvora muy fina y brillante, que sin duda se reservaba para cargar las escopetas de caza. En realidad mi expedición al barco me fue de poco provecho; por lo que respecta al dinero no tenia oportunidad de usarlo, y me importaba tanto como la tierra que pisaban mis pies. Lo hubiera dado íntegramente a cambio de tres o cuatro pares de zapatos ingleses y de medias, que mucha falta me hacían desde varios años atrás. Cierto que era dueño de dos pares de zapatos que quité a los marineros ahogados que viera en el puente, y en uno de los cofres encontré otros dos pares; pero no eran como nuestros zapatos ingleses, ni por su solidez ni por su comodidad, mereciendo más el nombre de escarpines. En uno de los cofres hallé unas cincuenta piezas de a ocho en reales, pero no oro. Presumo que el arcón pertenecía a uno de los marineros, mientras el otro debió ser de un oficial.
    De todos modos me llevé el dinero a la caverna, donde lo puse junto al que extrajera de mi propio barco. Era una verdadera lástima que la parte más importante del buque no hubiera estado a mi alcance, ya que tengo la seguridad de que habría podido llenar de oro mi canoa varias veces, y acumulado riquezas suficientes en la gruta para llevármelas conmigo si alguna vez conseguía escapar de la isla.
    Asegurado el cargamento sólo me quedaba volver a mi bote y remar en él hasta dejarlo fondeado en su vieja ensenada; de allí, luego de asegurarlo bien, volví a mi morada donde todo estaba en orden y sin novedad. Tras de haber descansado lo bastante reanudé mi existencia habitual cuidando de mis intereses domésticos; por un buen espacio de tiempo viví sin inquietudes, sólo que ponía más cuidado en mis movimientos y no me alejaba tan a menudo de mi casa. Si algún paseo emprendía era hacia el lado oriental de la isla, donde contaba con la seguridad de que los salvajes no desembarcarían nunca. Eso me evitaba adoptar tantas precauciones y llevar conmigo un enorme peso en armas y municiones, absolutamente necesario cuando me encaminaba en dirección opuesta.
    Dos años más transcurrieron en tales condiciones; pero mi malhadada imaginación, siempre dispuesta a recordarme que yo había nacido para hacer de mí un desdichado, estuvo todo ese tiempo fraguando proyectos y planes para escapar de la isla; a veces me sugería la conveniencia de hacer otro viaje hasta el casco encallado, aunque la razón me decía claramente que nada quedaba allí que me sirviera; otras veces me insinuaba navegar hacia un lado o hacia otro. En fin, estoy convencido de que si hubiera tenido a mi disposición la chalupa con la cual huí de Sallee me habría aventurado a cruzar el mar, con rumbo desconocido y destino incierto.
    En todas las circunstancias de mi vida yo he sido una especie de aviso para aquellos que también sufren la más grande plaga de la humanidad, plaga de la cual proviene por lo menos la mitad de sus desdichas; me refiero a los que no se sienten satisfechos con aquello que Dios y Naturaleza les han concedido.
    Una lluviosa noche de marzo, en el vigésimo cuarto año de mi existencia solitaria, reposaba en mi lecho o hamaca, despierto aunque sin sentir la menor molestia; mi salud era excelente, no tenía dolores ni la preocupación de mi mente era mayor que otras veces, y sin embargo no conseguía de ningún modo cerrar los ojos; me fue imposible dormir un solo instante en toda la noche.
    Sería tan difícil como inútil tratar de describir la innumerable multitud de pensamientos que se precipitaban a través de ese vasto camino del cerebro que es la memoria. Volvía a ver la entera historia de mi vida, aunque en miniatura o compendiada, hasta mi arribo a la isla; y también la siguiente etapa solitaria de mi existencia.
    Mi mente se detuvo un cierto tiempo a considerar las costumbres de aquellos miserables salvajes, y me pregunté cómo podía ocurrir en este mundo que el sabio Rector de todas las cosas hubiera podido dejar caer alguna de sus criaturas hasta semejante grado de inhumanidad, algo todavía por debajo de la brutalidad, como lo es devorar a sus semejantes. Pero terminando aquellas ideas en inútiles consideraciones, se me ocurrió de pronto preguntarme en qué parte del mundo vivían aquellos monstruos. ¿Estaba muy lejos la costa desde donde venían? ¿Por qué se aventuraban a apartarse tanto de su tierra? ¿Qué clase de canoas tenían? Y por primera vez encaré la posibilidad de lanzarme a un viaje que me llevase hasta el país de los salvajes, así como ellos eran capaces de llegar al mío.
    No sentí en ese momento la menor preocupación por lo que me esperaría al arribar allá. Ignoraba qué iba a ser de mí si era apresado por los salvajes, o cómo me las arreglaría para impedirlo. Tampoco se me ocurrió la manera de llegar hasta sus playas sin que me alcanzaran antes con sus piraguas, cosa de la que me sería imposible defenderme. Y luego, aun si me salvaba de sus manos, ¿cómo evitar morirme de hambre, cuál debería ser mi rumbo en tierra firme? Nada de todo eso, lo repito, cruzó entonces por mi cerebro; demasiado absorbido estaba con la esperanza de llegar al continente. Me limitaba a considerar mi actual situación como la más miserable que pudiera imaginarse, y creía que nada, salvo la muerte, podría parecerme peor que ella. Traté de animarme con la idea de que, ya en tierra firme, encontraría pronto algún socorro o bien podría ir costeando el continente como ya una vez lo hiciera en África, hasta dar con un país habitado donde me auxiliaran. Tal vez en mi camino encontrara algún buque cristiano que me recibiera a bordo, y si venía lo peor, sólo tenía la muerte por delante, lo cual era una manera de terminar de una vez con todas aquellas desdichas.
    Estos pensamientos se agitaron en mí por espacio de dos horas o más, con tal violencia que mi sangre parecía arder y mi pulso latía como si estuviera bajo la acción de la fiebre. ¡Tal era la fuerza de mi imaginación y su poder! Pero la Naturaleza, como si quisiera rescatarme de tan gran fatiga, terminó por sumirme en un profundo sueño. Se podría pensar que mis sueños siguieron el curso de aquellas ideas de la vigilia, pero nada de ello ocurrió, sino algo muy distinto.
    Soñé que, al salir como todas las mañanas del castillo, veía dos canoas en la costa y once salvajes que desembarcaban arrastrando a otro que sin duda se disponían a asesinar y comer; repentinamente, el salvaje prisionero se desasió de un salto y confió su vida a la velocidad de la carrera. Me parecía en mi sueño que se acercaba hasta ocultarse entre el espeso seto delante de mis fortificaciones; entonces, viendo que estaba solo y que sus enemigos no lo buscaban de ese lado, me mostré a él sonriéndole bondadosamente para darle ánimo. El salvaje cayó de rodillas ante mí, pareciéndome que me rogaba auxilio. Le mostré la escalera, haciéndolo entrar por ella en el castillo, y escondiéndolo en la cueva pronto se transformó en mi criado. Tan pronto como en sueños me sentía dueño de ese hombre, me decía: «Ahora puedo aventurarme sin temor al continente; este salvaje me servirá de piloto, me indicará qué debo hacer, si me conviene o no desembarcar en procura de provisiones, en fin, me irá evitando todo peligro de ser apresado y comido.»
    Me desperté bajo esa impresión y tal había sido el rapto de mi alegría ante la posibilidad de escapar de la isla, que el desencanto subsiguiente lo igualó en intensidad, sumiéndome en una profunda melancolía.
    Aquel sueño, sin embargo, me llevó a la conclusión de que mi única probabilidad de escapar de la isla estaba en apoderarme de algún salvaje, en lo posible algún prisionero traído a la isla para ser muerto y devorado. La dificultad del proyecto consistía en que no iba a ser fácil llegar a tal fin sin atacar antes a toda la pandilla de caníbales y matarlos. La tentativa podía muy bien fracasar, y a la vez se renovaban en mí los escrúpulos acerca de mi derecho a hacer una cosa semejante; mi corazón se estremecía a la idea de derramar tanta sangre aunque fuera para mi salvación. No necesito repetir todos los argumentos que acerca de esto se me ocurrieron, ya que son los mismos expuestos antes. Hasta había llegado a acumular nuevas excusas, como la de que aquellos salvajes eran un peligro para mi vida, pues si me echaban mano me devorarían; que mi proceder contra ellos equivalía a una defensa propia en su más extremo grado, ya que con él obtendría la liberación de esta existencia peor que la muerte; que si me adelantaba a atacarlos procedía con el mismo derecho que si ellos hubieran abierto el asalto, y otras cosas parecidas. Pero aunque todo aquello argüía en defensa de mis planes, la idea de verter sangre humana como precio de mi libertad se me antojaba terrible y durante mucho tiempo no pude conciliar ambas cosas en mi conciencia.
    Por fin, después de muchas y renovadas disputas conmigo mismo en las que pasaba por extraordinarias perplejidades, ya que los argumentos luchaban y se debatían en mi cerebro, las incontenibles ansias de libertad dominaron toda reserva y me decidí, costara lo que costase, a tratar de apoderarme de alguno de los salvajes.
    De inmediato se planteó el problema de llevar esto a la práctica, y no creo que haya tenido otro más arduo. No hallando por el momento solución plausible, me dediqué a hacer de centinela a la espera de que llegaran a tierra, dejando el resto confiado a los acontecimientos que por sí mismos me dictarían el caminó a seguir.
    Adoptadas estas resoluciones, principié a vigilar la costa casi de continuo y con tal intensidad que llegué a hartarme de ello, pues transcurrió más de un año y medio en espera, durante el cual casi diariamente iba yo hasta el extremo oeste o al ángulo sudoeste de la isla en busca de posibles canoas que jamás arribaban. La inacción era descorazonante, y empezó a torturarme con violencia, porque contrariamente a la vez anterior, en que el tiempo calmó mi irritación contra los salvajes, ahora parecía como si su ausencia exacerbara mi ansiedad por descubrirlos. Así como años atrás me mostraba deseoso de no tener contacto con aquellas gentes y evitaba hasta espiarlos, ahora me desvivía por las ganas de verlos desembarcar.
    Había pensado que quizá pudiera apoderarme no sólo de uno, sino de dos o tres de ellos, y confiaba en convertirlos en esclavos que no solamente me obedecieran en todo sino que resultaran incapaces de hacerme el menor daño. Imaginaba constantemente el modo de lograrlo, pero entretanto la isla continuaba desierta. Todos mis proyectos empezaron a sucumbir y pasó mucho tiempo sin que los salvajes se aproximaran a tierra.





    12. VIERNES




    Llevaba así un año y medio, y después de haber abrigado tantos planes los veía desvanecerse en el aire por falta de ocasión para ejecutarlos. Una mañana, sin embargo, me sorprendió la presencia de unas cinco canoas en la costa, cuyas tripulaciones habían desembarcado y estaban fuera de mi vista. Todas mis previsiones se derrumbaron al pensar en el número de aquellos salvajes, porque viendo tantas canoas y seguro de que en cada una venían cuatro o cinco tripulantes, no se me ocurría la manera de atacar a veinte o treinta hombres con mis solas fuerzas. Perplejo y desilusionado permanecí en el castillo, pero adopté, al igual que la vez anterior, las necesarias precauciones en caso de ataque, y pronto estuve listo para repelerlo. Aguardé un rato tratando de oír si hacían algún ruido, mas como mi impaciencia crecía por momentos puse las escopetas al pie de la escalera y me encaramé a la cumbre de la colina con el procedimiento ya descrito, teniendo sin embargo buen cuidado de que mi cabeza no sobrepasara el nivel de la roca y quedara completamente oculta a las miradas de aquellos hombres. Con ayuda del anteojo vi que no eran menos de treinta, que acababan de encender una hoguera y aderezaban allí sus alimentos. No pude distinguir qué clase de carne era aquélla y de qué modo la cocían; pero los vi bailar como locos en torno al fuego, haciendo toda clase de contorsiones y bárbaros ademanes.
    Mientras los observaba, mi anteojo me mostró de pronto a dos miserables prisioneros que eran arrastrados desde las canoas y conducidos al sacrificio. Vi a uno de ellos caer inmediatamente, y supongo que lo golpearon con una maza o cachiporra, como es su costumbre habitual. Inmediatamente dos o tres salvajes se precipitaron sobre el caído y empezaron a descuartizarlo, mientras el otro desgraciado permanecía inmóvil y a la espera de que le llegara el turno. Pero en ese mismo instante, como el infeliz había sido descuidado por sus captores y el instinto le inspirara una esperanza de vida, echó a correr con velocidad increíble a lo largo de la playa, justamente en dirección al lugar donde se hallaba mi morada.
    Confesaré que el espanto se apoderó de mí al verle tomar esa dirección, y sobre todo cuando la pandilla entera se lanzó en su persecución. Pensé que mi sueño iba a cumplirse y que el salvaje se ocultaría en el bosquecillo; pero no contaba con que el resto del sueño se cumpliera igualmente, es decir, que los salvajes renunciaran a seguirlo por esos lados. Permanecí inmóvil y a la espera, y pronto recobré algo de ánimo al advertir que solamente tres hombres perseguían al prisionero, y más aún comprobando que su rapidez en la carrera era muy superior a la de aquéllos, con lo cual si conseguía mantenerla por una media hora jamás se pondría de nuevo a su alcance.
    Entre el lugar hasta donde habían llegado y mi castillo se encontraba la ensenada que he citado en la primera parte de mi relato, cuando desembarqué los efectos del buque. El perseguido debía necesariamente nadar a través de ella, o lo apresarían en la orilla. Lo vi llegar a toda carrera y, sin preocuparse de que la marea estaba alta, zambullirse y lanzarse a la otra orilla sin perder un segundo; en unas veinte brazadas alcanzó el lado opuesto y allí siguió corriendo aún con más rapidez y energía que antes. Los tres perseguidos llegaron de inmediato a la ensenada, pero solamente dos sabían nadar; el otro, luego de mirar al fugitivo, no se animó a tirarse al agua y poco después se volvió lentamente hacia atrás, lo que fue para su propio bien.
    Desde mi apostadero pude observar que los dos perseguidores emplearon el doble de tiempo que el perseguido en cruzar la ensenada. Entonces me invadió el impulso irresistible de procurarme allí mismo el criado, o tal vez el compañero y ayudante que necesitaba, y pensé que la Providencia me había designado para salvar la vida de aquel infeliz. Descendí a toda velocidad por la escalera, tomé las armas que, como he dicho, había dejado al pie de ésta, y volví a subir a la cresta de la colina.
    Marchando en dirección al mar, como había un camino de atajo que descendía bruscamente de la colina a la playa, pronto me hallé entre el perseguido y los perseguidores, llamando a aquél en alta voz. Cuando, al mirar hacia atrás, me vio distintamente, tuvo más miedo de mí que de los otros, pero le hice señas con la mano de que se acercara; entretanto avancé sigiloso hacia los dos salvajes, y saltando bruscamente sobre el que venía adelante lo .derribé de un culatazo. No me atrevía a disparar el arma por temor a que el resto oyera el ruido, aunque a tan gran distancia no era fácil, máxime que tampoco podrían ver el humo y orientarse por él. Ya en el suelo el salvaje, el otro que iba más atrás se detuvo como aterrado; me acerqué lentamente, pero entonces vi que tenía un arco y flechas, que estaba armando para atravesarme, por lo cual no quedó otro remedio que disparar sobre él y lo derribé muerto al primer tiro.
    El pobre salvaje fugitivo, que ante mi actitud permanecía inmóvil a cierta distancia, vio a sus enemigos caídos y muertos, pero tuvo un terror tan grande al oír el estampido de la escopeta que se quedó como piedra, incapaz de avanzar o retroceder y, sin embargo, con más ganas de seguir huyendo que de venir hacia mi. Lo llamé otra vez, haciéndole signos de que se aproximara, lo que entendió fácilmente. Dio unos pasos, se detuvo, luego caminó otro trecho y volvió a pararse; advertí que temblaba a la idea de sufrir el mismo destino que sus perseguidores. Insistí en hacerle señas de que se acercara, tratando de demostrarle en toda forma que no le haría nada, para animarlo; fue aproximándose lentamente, pero cada diez o doce pasos se arrodillaba en señal de reconocimiento por haberle salvado la vida. Le sonreí de la manera más cariñosa, haciéndole seña de que se adelantara aún más, y por fin llegó a mi lado.
    Entonces, dejándose caer de rodillas, besó el suelo y apoyó en él su cabeza, y tomando mi pie lo puso sobre ella, lo que sin duda significaba su voluntad de hacerse mi esclavo por toda la vida.
    Lo levanté, acariciándolo y tratando de devolverle el coraje en todo lo posible. Sin embargo aún había tarea que realizar, porque de pronto advertí que el salvaje que golpeara con la culata no estaba muerto sino solamente desmayado y daba señales de recobrar los sentidos. Le apunté con la escopeta mientras hacía señas a mi salvaje para que reparara en su enemigo; comprendiendo, me habló algunas palabras que, aunque carentes para mí de sentido, fueron muy dulces de oír, ya que era el primer sonido humano que escuchaba yo en aquella isla después de veinticinco años.
    Pero no había tiempo ahora para reflexiones: el salvaje se recobraba poco a poco de su desmayo, lo vi que se sentaba en el suelo y advertí que mi compañero principiaba a asustarse otra vez, por lo cual le ofrecí la otra escopeta por si quería emplearla. Pero él me señaló por el contrario la espada que yo llevaba desnuda en la cintura, y se la alcancé. Apenas lo había hecho cuando lo vi precipitarse sobre su enemigo y cortarle la cabeza de un solo tajo con tal destreza que el mejor verdugo de Alemania no lo hubiese hecho más pronto ni mejor.
    Aquello me asombró en un hombre que, según imaginaba yo, jamás había visto antes una espada, salvo las de madera que usan esos pueblos. Más tarde, sin embargo, vine a saber que fabrican sus espadas con una madera tan dura como pesada, y que el filo es tan agudo que con ellas pueden decapitar de un golpe, e incluso tajar un brazo entero.
    En cuanto a él, después de matar a su enemigo, vino hacia mi riendo en señal de triunfo, y con abundancia de ademanes que no entendí depositó a mis pies la espada ensangrentada y la cabeza del salvaje.
    Lo que más lo pasmaba era la forma en que yo había matado al otro indio, y señalándolo parecía pedirme permiso para ir a examinarlo, lo que le concedí lo mejor que pude. Cuando llegó junto al cadáver se quedó como helado, mirándolo por todas partes; lo dio vuelta a un lado, después al otro, observó la herida de la bala, que había alcanzado a darle en el pecho, haciendo un orificio del cual manaba muy poca sangre, ya que la muerte se había producido por hemorragia interna. Por fin tomó el arco y las flechas y vino junto a mí, que me disponía a regresar. Le hice señas de que me siguiera, tratando de atemorizarlo a la vez con la idea de que otros salvajes podían presentarse de improviso.
    Como entendiera muy bien, me hizo señas de que lo dejara enterrar los cuerpos en la arena para que el resto de la pandilla no los encontrara. Cuando asentí se puso a cavar un hoyo con las manos, y pronto fue lo bastante grande para enterrar a uno de los muertos; repitió el procedimiento y un cuarto de hora más tarde los dos estaban sepultados. Llamándolo entonces lo llevé conmigo, no al castillo, sino a la gruta que quedaba en el otro extremo de la isla; de modo que no dejé cumplirse el sueño en aquella parte, según la cual el salvaje había buscado refugio en mi soto.
    Le di pan y un racimo de pasas, así como agua, de la que estaba muy necesitado después de aquella carrera.
    Luego que se hubo refrescado le hice signos de que se acostara a dormir, señalándole un sitio donde había un colchón de paja de arroz cubierto con una manta, que yo empleaba a veces para descansar allí. El pobre obedeció y pronto estuvo dormido.
    Era un individuo bien parecido, muy bien formado y fuerte, no demasiado alto pero de gran esbeltez, que contaría según calculé unos veintiséis años. Tenía un rostro agradable, sin ninguna fiereza ni ferocidad, aunque advertí que sus facciones eran muy varoniles; cuando sonreía, encontraba yo en su rostro toda la suavidad y la dulzura de los europeos. Su largo y negro cabello no se encrespaba como lana; la frente era ancha y despejada, y había vivacidad e inteligencia en su mirada. La piel no era negra sino atezada, pero sin ese desagradable matiz amarillento de los naturales del Brasil, Virginia y otros lugares americanos, sino más bien un aceitunado oscuro que resultaba muy agradable de ver aunque no sea fácil describirlo. La cara era redonda y llena, con una nariz pequeña y no aplastada como la de los negros, una boca firme de labios pequeños y dientes tan perfectos y blancos como marfil.
    Luego que hubo dormitado, más que dormido, una media hora, se levantó y saliendo de la gruta fue hacia donde estaba yo terminando de ordeñar las cabras que guardaba en ese sitio. Cuando me divisó vino corriendo a arrodillarse otra vez a mis plantas, con fervientes demostraciones de reconocimiento y humildad, haciendo mil gestos para que yo comprendiera. Por fin apoyó la cabeza contra el suelo junto a mi pie, y volvió a levantar mi otro pie y colocárselo encima, tras lo cual hizo todos los ademanes posibles de sumisión y servidumbre para darme a entender que sería mi esclavo por siempre. Comprendí bastante todo esto, y traté de demostrarle que me sentía muy contento con él. Poco después empecé a hablarle, a fin de que aprendiera a contestarme poco a poco. Ante todo le hice saber que su nombre sería Viernes, ya que en este día lo salvé de la muerte y me pareció adecuado nombrarlo así. A continuación le enseñé a que me llamara amo y a que contestara SÍ' o no, precisándole la significación de ambas cosas. Llené de leche un cacharro que puse en sus manos, mostrándole primero cómo se bebía aquello y mojando mi pan en la leche; de inmediato hizo lo mismo, dando señales visibles de que le gustaba mucho.
    Lo tuve conmigo aquella noche, y a la mañana siguiente le indiqué que me siguiera, haciéndole comprender que le daría algunas ropas para que se vistiera, ya que estaba completamente desnudo. Cuando cruzamos el lugar donde había enterrado a los dos salvajes me señaló con precisión el sitio, mostrándome las marcas que había hecho para encontrarlos otra vez, y comprendí por sus signos que me invitaba a desenterrarlos y comerlos. A esto me mostré encolerizado, dándole a entender la repugnancia que me producía la sola idea, e hice como si su intención me causara náuseas, ordenándole que se alejara de allí al punto, cosa que hizo con gran sumisión. Lo llevé conmigo hasta la cumbre de la colina, para observar si sus enemigos habían vuelto a embarcarse; con ayuda del anteojo recorrí la costa y aunque encontré el lugar donde se habían congregado no descubrí la menor señal de su presencia, lo que indicaba evidentemente que se habían marchado sin inquietarse en lo más mínimo por la suerte de sus dos compañeros.
    No contento con este descubrimiento, y como el mayor coraje aumentaba en igual grado mi curiosidad, confié a Viernes mi espada así como el arco y flechas que llevaba a la espalda y que sabía usar diestramente; le di también una escopeta para mí, y llevando yo otras dos, nos encaminamos hacia la costa donde habían pernoctado los salvajes. Cuando estuvimos allí la sangre se me heló en las venas y me pareció que mi corazón se detenía; ¡tan atroz era el espectáculo! Me quedé inmóvil de espanto, aunque Viernes no parecía conmovido en lo más mínimo. El lugar estaba cubierto de huesos humanos, el suelo tinto en sangre; grandes trozos de carne aparecían diseminados aquí y allá, devorados a medias y carbonizados; en fin, eran los testimonios del banquete triunfal con que aquellos salvajes habían celebrado la victoria sobre sus enemigos. Encontré tres cráneos, cinco manos y los huesos de tres o cuatro piernas y pies, así como abundancia de otras porciones de carne humana.
    Por medio de signos, Viernes me dio a entender que habían traído cuatro prisioneros para devorar, que aquellos restos pertenecían a tres y que él —se apuntaba con la mano— era el cuarto. Me explicó del mismo modo que había habido una gran batalla entre aquellos salvajes y los súbditos de un rey vecino, del cual parecía ser vasallo, y que habiendo resultado vencedores los otros, habían tomado gran número de prisioneros que fueron conducidos a distintos lugares para servir de pasto en el bárbaro festín de la victoria; un grupo de aquellos miserables era el que había desembarcado en mi isla.
    Ordené a Viernes que reuniera los cráneos, huesos y demás restos e hiciera con ellos una pirámide y le pegara fuego hasta que se calcinaran. Observé que se mostraba harto dispuesto a comerse parte de aquella carne, y que seguía siendo caníbal en su naturaleza; pero di tantas señales de repugnancia a la sola idea de semejante cosa, que no se atrevió a manifestar sus verdaderos instintos, ante todo porque yo le había dado a entender que si cedía a ellos no vacilaría en matarlo.
    Terminada la tarea volvimos al castillo, donde empecé a trabajar para mi criado Viernes. Ante todo le di unos calzoncillos de lienzo que encontrara en el arcón del pobre artillero y que rescaté del naufragio; con pequeñas modificaciones, le sentaron muy bien. Luego hice una chaqueta de piel de cabra, lo mejor que me fue posible, ya que era un discreto sastre; le di una gorra de piel de liebre, muy cómoda y pasablemente elegante, con lo cual quedó bastante presentable y me pareció satisfecho de verse igual que su amo. Cierto que al comienzo se sentía incómodo con aquellas ropas; los calzoncillos le estorbaban enormemente, y las mangas de la chaqueta le lastimaban los hombros y la piel de los brazos. Pero cuando se quejó de ello le hice los retoques convenientes y pronto se habituó sin la menor dificultad.
    Al siguiente día de tenerlo conmigo empecé a considerar dónde alojaría a mi criado. Se necesitaba un sitio que fuera cómodo para él y conveniente para mí, de modo que terminé levantando una pequeña tienda en el espacio libre que quedaba entre las dos fortificaciones, es decir, en el interior de la segunda y el exterior de la primera. Como justamente allí estaba la abertura que permitía entrar en la cueva, construí una verdadera puerta, clavando tablas sólidas en un marco del tamaño conveniente y fijándola en el interior del pasaje a la cueva. La puerta se abría hacia adentro, y de noche la aseguraba sólidamente teniendo también la precaución de retirar las escaleras, con lo cual nunca hubiera podido Viernes llegar hasta mí sin hacer mucho ruido que me hubiera despertado de inmediato.
    Es de recordar que mi primera empalizada tenía ahora un verdadero techo, formado por largas pértigas que cubrían enteramente la tienda y se apoyaban en la roca; sobre ellas había colocado troncos finos en lugar de vigas, y todo estaba cubierto espesamente con paja de arroz, tan sólida como si fuese caña. En el agujero que dejé para salir por la escalera había instalado una especie de trampa, que, al intentar abrirla desde afuera, hubiese caído, con gran estrépito. En cuanto al armamento, lo guardaba todas las noches conmigo.
    Sin embargo ninguna de estas precauciones resultó necesaria, porque nunca hombre alguno tuvo un sirviente tan fiel, amante y sincero como lo fue Viernes conmigo. Sin violencias, enojos o mala intención, se mostraba profundamente adicto y dispuesto; su afecto por mí parecía más bien el de un hijo por su padre, y me atrevo a decir que hubiera sacrificado voluntariamente su vida para salvar la mía en cualquier ocasión. Muchos testimonios me dio de ello, y pronto me convencí de que era inútil emplear con él aquellas excesivas precauciones
    Esto me dio oportunidad de pensar frecuentemente y con no poca maravilla que si Dios, en su Providencia y en el gobierno de su Creación, había decidido privar a tantas de sus criaturas del mejor empleo de sus facultades y sentimientos, sin embargo los había dotado de las mismas disposiciones, la misma razón, iguales afectos y sentimientos de humildad y devoción, así como de las mismas pasiones y resentimientos ante las ofensas, sentido de gratitud, sinceridad, fidelidad y todo el poder de hacer el bien y recibirlo que diera a sus demás criaturas. Y que cuando a Dios le placía ofrecerles oportunidades de ejercitar aquellas cualidades, estaban tan dispuestos y hasta parecían más capaces que nosotros en emplearlas para el bien.
    Pero volvamos a mi nuevo compañero. Me sentía muy contento con él, y traté de enseñarle en seguida aquellas cosas que lo tornarían útil, capaz y diestro. Mi mayor deseo era enseñarle a hablar, y que entendiera lo que yo le decía. Nunca se encontró mejor alumno que él; se mostraba tan contento, tan aplicado y daba muestras de tal alegría cuando alcanzaba a comprenderme o lograba que yo lo entendiera a él, que resultaba un placer hablarle. Mi vida se tornó tan placentera que con frecuencia me decía que de no mediar el peligro de los salvajes no me hubiera afligido tener que quedarme allí para siempre.
    Después de llevar dos o tres días en el castillo pensé que para alejar a Viernes de su horrible costumbre de comer carne humana y hacerle perder el hábito adquirido lo mejor sería darle a probar otras carnes. Una mañana, pues, lo llevé conmigo a los bosques dispuesto a matar para él una de las cabras que tenía en cautiverio y traer su carne al castillo. Pero en el camino di de pronto con una cabra que descansaba rodeada por dos cabritos. Detuve inmediatamente a Viernes.
    — ¡Quieto! —le ordené, haciéndole seña de que no se moviera.
    Inmediatamente apunté y el tiro alcanzó a uno de los cabritos. Viernes, que me había visto matar a distancia a uno de sus perseguidores, pero seguía sin comprender cómo era posible tal cosa, se puso a temblar y me miraba con un aire tan horrorizado que me pareció que iba a caer desvanecido. Ni siquiera se dio cuenta de que yo había tirado contra un cabrito y que allí yacía muerto, sino que empezó a tantearse la chaqueta como si quisiera descubrir alguna herida. Debió pensar, como me di cuenta en seguida, que intentaba matarlo, porque precipitándose a mis pies me abrazó las piernas, mientras decía gran cantidad de cosas que no entendí, aunque evidentemente me rogaba que no le arrebatase la vida.
    No me costó mucho convencerlo de que no le haría daño alguno, y tomándolo de la mano lo obligué a levantarse y le señalé el cabrito muerto, ordenándole que fuera a buscarlo, lo que hizo enseguida. Mientras él observaba maravillado la forma en que fuera herido el animalito, cargué de nuevo la escopeta y pronto vi un gran pájaro semejante a un halcón posado en un gran árbol cercano. Para darle a entender a Viernes cuál era mi intención le mostré el pájaro —que era en realidad un papagayo y no un halcón— y después señalé la escopeta y el sitio debajo del árbol donde estaba el animal, para que comprendiese que no dejase de mirar el papagayo, e inmediatamente lo vio caer. Se quedó de nuevo como petrificado, a pesar de mis explicaciones, y advertí que como no me había visto cargar otra vez el arma pensaba sin duda que había en ella un inagotable caudal de muerte y destrucción para todo hombre, animal o pájaro que se pusiera a distancia de tiro.
    El pasmo que esto le causaba era tal que transcurrió un tiempo antes de que se recuperara y creo que de haberlo dejado me hubiese adorado tanto a mí como a la escopeta. Durante muchos días no se atrevió a tocar el arma, pero a veces, cuando estaba solo con ella, le hablaba y parecía esperar una respuesta. Más tarde me confesó que le había suplicado con insistencia que no lo matara.
    Luego que se le pasó el primer susto al ver cómo mataba yo al pájaro, obedeció mi orden y fue a recogerlo, pero como el papagayo no estaba más que herido revoloteó para alejarse y Viernes tuvo que correr tras él hasta que al fin pudo alcanzarlo; entretanto yo aproveché su ausencia para cargar otra vez la escopeta, pues había advertido cómo lo asombraba este detalle del arma y la guardaba lista para un nuevo disparo si se presentaba la ocasión. No la hubo, sin embargo, y volvimos trayendo el cabrito, que desollé aquella misma tarde. En una de las ollas herví y guisé una cantidad de carne, obteniendo un excelente caldo. Luego de haberlo probado le di una porción a mi criado, que pareció gustar muchísimo de él. Lo que sin embargo lo maravillaba era verme comer la carne con sal, e hizo señas de que la sal no era sabrosa, y poniéndose un poco en la boca pareció sentir una viva repugnancia, escupiéndola en seguida y yendo a enjuagarse la boca con agua fresca. A mi vez me llevé a la boca un trozo de carne sin sal haciendo toda clase de demostraciones de repugnancia, para convencerlo de que así no debía comerse; pero no obtuve resultado alguno, y Viernes siguió comiendo su carne y bebiendo el caldo sin sal; más tarde empezó a salar su comida, pero apenas.
    Habiéndolo alimentado con el caldo y la carne hervida, me propuse ofrecerle al día siguiente un cuarto de cabrito asado. A tal fin colgué el trozo de una cuerda, tal como había visto hacerlo a mucha gente en Inglaterra; fijando dos estacas a ambos lados del fuego y un palo atravesado sobre ellas, sujeté la cuerda en este último cuidando de hacer girar continuamente el trozo de carne. Viernes admiró mucho estos preparativos, pero aún más maravillado se mostró al probar la carne, empleando tales gestos y ademanes para indicarme cuánto le había gustado que hubiese sido imposible no advertirlo. Por fin me dio a entender que jamás volvería a comer carne humana, lo que me produjo una gran alegría.
    Al otro día lo puse a trillar grano, así como a cernirlo de la manera que ya he contado; pronto aprendió a hacerlo tan bien como yo, especialmente cuando hubo advertido cuál era el objeto de ese trabajo, es decir, la obtención de pan. Le mostré cómo se preparaba y se cocía el pan, y en poco tiempo Viernes fue tan hábil en efectuar aquellos trabajos como pudiera haberlo sido yo mismo.
    Había que tener ahora en cuenta que éramos dos bocas para alimentar en vez de una, de modo que urgía preparar más tierras y sembrar mayor cantidad de grano que hasta entonces. Luego de elegir una superficie conveniente me di a la tarea de hacer un vallado igual al anterior, y Viernes no solamente me ayudó con habilidad y tesón sino que parecía mostrar verdadero entusiasmo. Le di a entender para qué trabajábamos, que ahora era necesaria una mayor cosecha a fin de disponer de más pan, ya que ambos teníamos que alimentarnos.
    Comprendió con suma inteligencia mi razonamiento, y me significó que él se daba clara cuenta de que mis tareas aumentaban mucho con su presencia, pero que estaba dispuesto a trabajar con todas sus fuerzas si yo le enseñaba el modo de hacerlo.
    Aquél fue el más agradable año de todos los que viví en la isla. Viernes empezaba a hablar bastante bien, entendía los nombres de casi todas las cosas que yo podía pedirle y de los lugares adonde lo enviaba. Como hablábamos mucho, volví a tener ocasión de emplear el idioma que durante tanto tiempo me había sido inútil, por lo menos para conversar. Fuera del gusto que me daban estas charlas, me sentía cada vez más atraído hacia el muchacho; su sencilla y franca manera de ser se me revelaba cada día con más claridad, y llegué a quererlo profundamente. Pienso también que él sentía por mí un cariño que jamás había experimentado en su vida.
    Una vez se me ocurrió comprobar si Viernes guardaba alguna nostalgia de su país. Como le había enseñado bastante inglés para que pudiese contestar casi todas mis preguntas, le interrogué sobre si su nación era capaz de triunfar en las guerras. A esto se sonrió y me dijo:
    —Sí, sí, nosotros siempre en pelea mejores.
    Quería significar que combatían mejor que los otros pueblos. Entonces mantuvimos el siguiente diálogo:
    — ¿Así que vosotros peleáis mejor? —dije yo—. ¿Y cómo es que te tomaron prisionero, Viernes?
    VIERNES: Mi nación vencer muchos por eso.
    AMO: ¿Cómo vencer? Si tu nación venció, ¿cómo es que fuisteis apresados?
    VIERNES: Ellos más que mi nación en sitio donde yo estar; mi nación apresar uno, dos, muchos mil.
    AMO: ¿Y por qué, entonces, los de tu nación no acudieron a rescataros de las manos del enemigo?
    VIERNES: Ellos meter uno, dos, tres y yo en canoa; mi nación no tener entonces canoa.
    AMO: Dime, Viernes, ¿qué hacen los de tu nación con los enemigos que toman prisioneros? ¿Se los llevan para comerlos, como tus enemigos?
    VIERNES: Sí, mi nación también comer hombres; comerlos todos.
    AMO: ¿Adonde los llevan?
    VIERNES: Otros sitios, donde gustarles.
    AMO: ¿Vienen a esta isla?
    VIERNES: Sí, sí, venir aquí; venir muchas partes.
    AMO: ¿Viniste aquí con ellos alguna vez?
    VIERNES: Sí, yo estar allí. (Y señalaba el lado noroeste de la isla que, al parecer, era su sitio preferido).
    A través de este diálogo descubrí que mi criado había formado anteriormente parte de las partidas de salvajes que desembarcaban en el extremo de la isla, haciendo con otros lo mismo que ahora habían pretendido hacer con él. Más adelante, cuando tuve ánimo suficiente para llevarlo conmigo a aquel lugar que ya he descrito, conoció inmediatamente el sitio y me dijo que allí mismo habían devorado en una ocasión veinte hombres, dos mujeres y un niño. No sabía decir «veinte» en inglés, pero se puso a alinear piedras y me suplicó que las contase.
    He narrado este episodio porque explicará lo que sigue, y es que luego de oír a Viernes hablar de su nación, le pregunté a qué distancia quedaba nuestra isla de aquellas costas y si las canoas no se perdían con frecuencia. Me respondió que no había peligro y que jamás se perdían las canoas, ya que apenas salidas mar afuera encontraban siempre una misma corriente y un mismo viento, en una dirección por la mañana y en otra por la tarde.
    Comprendí que se refería simplemente a las mareas alternas, pero más tarde vine a descubrir que se trataba de los grandes movimientos y el reflujo del enorme río Oroonoko,1 ya que como terminé por saber nuestra isla se hallaba en el gran golfo de su desembocadura. La tierra que se alcanzaba a ver hacia el O y el NO era la gran isla Trinidad, en la parte norte de las bocas del río. Hice mil preguntas a Viernes sobre el país, los habitantes, el mar, la costa, y cómo se llamaban las naciones próximas. Con toda buena voluntad me dijo cuanto sabía. Le pregunté los nombres de las distintas tribus de su raza, pero sólo supo responder: «Caríb». Me fue fácil deducir que se trataba de los caribes, que nuestros mapas colocan en la parte de América que va desde las bocas del Oroonoko hasta la Guayana y Santa Marta. También me dijo que mucho más allá de la luna —queriendo significar el poniente de la luna, hacia el oeste de su nación— vivían hombres de barba blanca como yo (y señalaba mis bigotes y patillas ya mencionados). Agregó que los blancos habían matado «mucho hombre», según sus palabras, por lo cual comprendí que se refería a los españoles, cuyas crueldades en América se han difundido en el mundo entero al punto de ser recordadas y transmitidas de padres a hijos en cada nación.
    Pregunté a Viernes si podía decirme el modo de salir de la isla y llegar al país de los hombres blancos.
    —Sí, sí —replicó—. Poder ir en dos canoas.
    No supe qué quería significar ni conseguí que me describiera su pensamiento, hasta que al fin y con gran dificultad vine a darme cuenta de que al decir «dos canoas» quería indicar un bote que tuviese un tamaño equivalente al doble de una piragua.
    Estas afirmaciones de Viernes me agradaron mucho, y desde entonces volví a abrigar la esperanza de que alguna vez hallaría oportunidad de fugarme de la isla, y que aquel pobre salvaje sería para mí una valiosa ayuda.
    En el ya largo tiempo que Viernes llevaba a mi lado, cuando fue capaz de hablar y comprender lo suficiente, no descuidé de sembrar en su alma los fundamentos del conocimiento religioso. En una ocasión le pregunté quién lo había creado, pero el pobre muchacho no fue capaz de comprender el sentido de mi pregunta, de modo que busqué otra manera y le pregunté quién había creado el mar, la tierra sobre la cual andábamos, las colinas y los bosques. Me dijo que el creador era el anciano Benamuki, que vivía más allá de todo. Era incapaz de decirme nada acerca de él, sino que Benamuki era viejo, mucho más viejo que el mar y la tierra, que la luna y las estrellas. Le pregunté por qué si aquel anciano era el creador de todas las cosas, no era adorado por ellas. Me miró gravemente y luego, con una absoluta inocencia, dijo:
    —Todas las cosas dicen « ¡Oh!» a Benamuki.
    Lo interrogué sobre si los hombres que morían en su nación iban a alguna parte.
    —Sí —me contestó—. Ellos ir a Benamuki.
    — ¿También los que son devorados?
    —Sí —dijo Viernes.
    Partiendo de esas conversaciones principié a instruirlo en el conocimiento del verdadero Dios. Le dije que el Hacedor de todas las cosas vivía en lo alto, y le señalé el cielo; que gobierna el mundo con el mismo poder y providencia de que se valió para crearlo; que era omnipotente, pudiendo hacer todo por nosotros, darnos o quitarnos todo; y así gradualmente fui iluminando su inteligencia. Escuchaba con gran atención, y aceptó con placer la idea de que Jesucristo había sido enviado a la Tierra para redimirnos, así como la forma de rogar a Dios y la seguridad de que El escuchaba las plegarias desde el cielo. Un día me dijo Viernes que si nuestro Dios podía escucharnos desde más allá del sol era necesariamente un dios más grande que Benamuki, que apenas vivía algo más lejos de la tierra y solamente escuchaba a los hombres cuando subían a lo alto de las montañas donde moraba para invocarlo. Le pregunté si alguna vez había subido a hablarle; me dijo que no, que los jóvenes jamás lo hacían sino que era privilegio de los ancianos a quienes llamaban Uwokaki, queriendo significar, según me explicó, los sacerdotes o ascetas; aquellos eran los que subían a decir « ¡Oh!» —evidentemente, a elevar sus plegarias— y a su regreso manifestaban la voluntad de Benamuki. De ahí deduje que aun entre los más ciegos, ignorantes y paganos habitantes del mundo existe la superchería, y que la astucia de crear una religión secreta a fin de mantener la veneración popular se practica acaso en todas las religiones del mundo, incluso las de los más embrutecidos y bárbaros salvajes.
    Hice lo posible por explicarle a Viernes ese fraude, y le dije que la artimaña de los ancianos al subir a las montañas para decir « ¡Oh!» al dios Benamuki era un engaño, y mucho más su pretensión de ser los portadores de mensajes divinos. Que si alguna palabra recibían en lo alto era proveniente del espíritu del mal, y de ahí nos internamos en una larga conversación sobre el diablo, su origen, su rebelión contra Dios, el odio que le profesa y las causas del mismo, su residencia en los lugares más sombríos de la tierra para que allí se lo adore como si fuese Dios, y las muchas estratagemas de que es capaz para precipitar en la ruina a la humanidad.
    Le mostré cómo el diablo tiene secreto acceso a nuestras pasiones y nuestros afectos, y la astucia con que tiende sus trampas aprovechando nuestras inclinaciones a fin de que nosotros mismos nos tentemos y nos hundamos voluntariamente en la destrucción.
    Le decía yo cómo el diablo es el enemigo de Dios en el corazón de los hombres y emplea allí toda su malicia y su destreza para impedir los buenos designios de la Providencia, a fin de ocasionar la ruina del reino de Cristo, cuando Viernes me interrumpió.
    —Bueno —me dijo—. ¿Vos decir Dios tan grande, tan fuerte, mucho más que diablo?
    —Sí, sí —afirmé yo—. Dios es más fuerte que el diablo, Viernes. Dios está por encima del diablo, por eso rogamos a Dios que nos permita pisotear al diablo, resistir sus tentaciones y apagar el fuego de sus dardos.
    —Pero —declaró él— si Dios más fuerte, si Dios más poderoso que diablo, ¿por qué no matar Dios al diablo, y éste así no hacer más daño?
    Aquella pregunta me sorprendió grandemente, pues aunque en aquel entonces era yo hombre maduro, mi capacidad teológica no excedía a la de un novicio y de ninguna manera podía dármelas de casuista para solucionar tales dificultades.
    Me encontré sin saber qué contestar, y fingiendo que no le había entendido le pedí que me repitiera su pregunta. Demasiado inteligente era para haber olvidado su duda, y me la repitió con las mismas pintorescas palabras. Ya entonces había recobrado un poco la serenidad, y le contesté:
    —Dios lo castigará severamente al fin; ha sido reservado para el juicio final, y será precipitado en los abismos sin fondo, donde lo consumirá un fuego eterno.
    Esto no satisfizo a Viernes, sino que volvió a la carga empleando mis propias palabras:
    — ¡Reservado al fin! Yo no entender. ¿Por qué no matar ya diablo? ¿Por qué no desde antes?
    —Lo mismo podrías preguntarme —le dije— por qué Dios no nos mata a nosotros cuando cometemos pecados que lo ofenden. El nos reserva la oportunidad de arrepentirnos y ser perdonados.
    Meditó un rato esta observación, y entonces me dijo de pronto y muy emocionado:
    —Bien, bien, eso muy bien; entonces vos, yo, diablo, todos malos, todos ser reservados, arrepentirse, Dios perdonar a todos.
    Interrumpí entonces el diálogo, levantándome bruscamente como si me llamara alguna tarea urgente; y luego de haber enviado lejos a Viernes elevé mis plegarias a Dios para que me hiciera capaz de instruir convenientemente a aquel pobre salvaje, y que mi enseñanza de la Palabra de Dios fuera tal que su conciencia se abriera a ella, sus ojos vieran la luz y se salvara su alma. Cuando volvió Viernes, le di una extensa explicación acerca de cómo habían sido redimidos los hombres por el Salvador del mundo, y le enseñé la doctrina del Evangelio dictada por el mismo Cielo, insistiendo en la noción del arrepentimiento y en la fe hacia nuestro bendito señor Jesucristo. Le expliqué después lo mejor posible por qué el santo Redentor no había adoptado la naturaleza y forma de los ángeles sino la estirpe de Abraham; y cómo, por eso, los ángeles caídos no participaban de la redención; en fin, le narré que El había venido solamente a salvar la oveja descarriada de la casa de Israel y así las restantes nociones religiosas.
    Continuando del mismo modo en todo momento libre, las conversaciones que mantuvimos Viernes y yo fueron tales que aquéllos tres años que vivimos juntos en la isla me parecieron absolutamente felices y venturosos, como si en verdad fuera posible la dicha total en algún sitio sublunar. Ya entonces el salvaje era un excelente cristiano, mejor por cierto que yo; tengo razón para creer y esperar, Dios sea bendito por ello, que ambos estábamos arrepentidos y que el consuelo divino nos había alcanzado ya. Con nosotros estaba la Palabra de Dios que podíamos leer, y no nos sentíamos más lejos de la ayuda de su gracia que si hubiésemos vivido en Inglaterra.
    Todas las disputas, riñas, debates y cuestiones que la religión ha suscitado en el mundo, ya por discrepancias sutiles de doctrina o cismas en el gobierno de la Iglesia, nos eran totalmente ajenos, así como a mi entender lo han sido para el resto del mundo. Poseíamos la más segura guía del Cielo, es decir, la Palabra de Dios, claras nociones del Espíritu Divino que Su Palabra nos enseñaba, conduciéndonos seguramente hacia la verdad e inculcándonos la voluntad y obediencia a sus dictados. No alcanzo a entender la utilidad que hubiese podido darnos el más profundo conocimiento de los puntos discutibles de la religión, por los cuales tantas confusiones acontecen en la Tierra. Pero debo ya proseguir con el relato de nuestra existencia y ordenar sus distintos episodios.





    13. BATALLA CON LOS CANÍBALES




    Después que Viernes y yo hubimos intimado, y que él fue capaz de entender casi todo lo que le decía así como hablarme en un inglés chapurreado, empecé a hacerle saber mi historia, por lo menos la parte referente a mi existencia en la isla. Le conté cómo y cuánto había podido vivir allí, lo introduje en los misterios —pues tales eran para él— de la pólvora y las balas, y hasta le enseñé a tirar. Le regalé un cuchillo, lo que le causó una inmensa alegría, y le hice un cinturón con una presilla como los que empleamos en Inglaterra para colgar machetes, dándole una hachuela para que la llevase allí, ya que no sólo era un arma excelente sino que servía muy bien para diversos usos.
    Hice a Viernes una descripción de Europa, y en especial de mi patria, Inglaterra; cómo vivimos, adoramos a Dios, nos conducimos en nuestra vida social y comerciamos en todos los mares del mundo. Le describí el naufragio por cuya causa arribara a la isla, y traté de indicarle con precisión el sitio donde estaba el casco; pero tanto se había roto la estructura del barco que nada quedaba a la vista.
    Entonces señalé a Viernes los restos del bote que había naufragado mientras estábamos a su bordo y que en vano había tratado yo de mover del sitio en que encallara; ahora aparecía destruido y deshecho. Al verlo, Viernes se quedó silencioso y permaneció largo rato pensativo. Le pregunté en qué estaba meditando, y por fin me dijo:
    —Yo ver bote igual ese llegar a mi nación.
    Al principio no le entendí, pero después de interrogarlo mucho supe que un bote semejante al mío había arribado a las costas caribes; de acuerdo con sus referencias, la violencia de un temporal lo precipitó a tierra. Imaginé de inmediato que algún barco europeo se habría estrellado cerca y que el bote, soltándose, había llegado solo y vacío a la costa; pero tan ocupado estaba en estos pensamientos que no cruzó por mi mente la idea de que algunos tripulantes podían haberse salvado de la catástrofe, y menos aún su procedencia, de manera que sólo pedí a Viernes detalles sobre el bote.
    Lo describió lo mejor posible, pero pronto me llamó a la realidad al decirme con bastante entusiasmo:
    —Nosotros salvar hombres blancos de ahogarse.
    — ¿Entonces había hombres blancos en el bote? —me apresuré a preguntar.
    —Sí, bote lleno hombres blancos.
    Le pregunté cuántos, y contó hasta diecisiete con sus dedos. ¿Qué había sido de ellos?
    —Vivir —contestó—. Vivir en mi nación.
    Esto me sumió en nuevos pensamientos, a la sola idea de que aquellos hombres pudieran provenir del navío que había encallado a poca distancia de mi isla. Tal vez, después de comprobar la destrucción del navío en los arrecifes y comprendiendo que estaban perdidos si no se alejaban del lugar del naufragio, habían embarcado en el bote yendo a dar a aquellas salvajes tierras.
    Con mayor detalle interrogué a Viernes sobre el destino de aquellos hombres. Me repitió que vivían allí, llevando cuatro años de residencia, que los salvajes los dejaban solos y les daban vituallas. Le pregunté qué razón había para no haberlos matado y comido.
    —No —repuso Viernes—. Ellos hacer hermanos.
    Supuse que significaba alguna tregua o alianza, ya que agregó:
    —Solamente comer hombres cuando luchar en guerra.
    Comprendí entonces que la costumbre caribe era la de devorar solamente a los prisioneros de las batallas.
    Mucho tiempo después de esto nos hallábamos un día en la cumbre de la colina situada en el lado este de la isla, justamente donde, como ya he dicho, había descubierto en un día claro el perfil del continente americano. El tiempo estaba muy despejado y Viernes se había puesto a mirar intensamente en aquella dirección, cuando de pronto empezó a saltar y bailar como en un rapto de entusiasmo, llamándome a gritos y haciendo que acudiera a preguntarle qué le pasaba.
    — ¡Oh, alegría! —gritaba Viernes—. ¡Contento! ¡Allí ver mi país, ver mi nación!
    Una expresión de intenso júbilo se pintaba en su fisonomía; le chispeaban los ojos y en su aspecto se traslucía la vehemente ansiedad de retornar alguna vez a su país. Tan extraña exaltación me llenó de preocupaciones, haciéndome perder la confianza que hasta ahora había sentido por Viernes; me pareció que si conseguía volverse a su pueblo no sólo olvidaría cuanto le había enseñado de religión sino también los deberes que tenía para conmigo. Probablemente hiciera a sus compatriotas una relación de mi persona y tal vez volviese con cien o doscientos de ellos para devorarme, de lo cual se sentiría tan satisfecho como cuando mataban a los enemigos apresados en la batalla.
    Al pensar así cometía una injusticia con aquel pobre muchacho, y más adelante me arrepentí de ello; pero como este estado de ánimo me dominó durante algunas semanas, me mostré más circunspecto hacia Viernes, sin manifestarme tan familiar y amable hacia él como antes. Mi error fue lamentable, porque aquel agradecido y sencillo salvaje no tenía ya entonces otros pensamientos que aquellos nacidos de los excelentes principios cristianos y de su gratitud amistosa, como más tarde pude comprobarlo a mi entera satisfacción.
    Mientras duró mi desconfianza es de imaginar que constantemente lo sondeaba para tratar de descubrir alguno de los sentimientos que imaginaba habían cobrado cuerpo en él; pero como lo que decía o manifestaba era tan honesto y candoroso, tales sospechas cedieron por falta de razones, y a pesar de mi secreta duda terminé por confiarme otra vez por completo a Viernes; en cuanto a él, en ningún momento se dio cuenta de mi estado de ánimo y por lo tanto no podía sospecharlo de falsedad.
    Paseando un día por la misma colina, pero con tiempo tan nublado que no se divisaba el continente, pregunté a Viernes:
    —Dime, ¿no te gustaría volver a tu país, a tu nación?
    —Sí —repuso él—. Yo muy contento, ¡oh mucho!, volver mi nación.
    — ¿Que harías allí? —insistí—. ¿Te volverías salvaje de nuevo, comerías carne humana lo mismo que cuando te encontré?
    Viernes me miró con aire grave y movió negativamente la cabeza.
    —No, no. Viernes decirles a ellos vivir bien; decirles rogar a Dios, decirles comer pan, carne de cabra, leche, no comer más hombre.
    —Pero entonces te matarán a ti.
    Se puso aún más grave.
    —No —dijo luego—, ellos no matarme, ellos aprender gustando.
    Quería decir que aprenderían gustosos. Agregó en seguida que ya habían aprendido muchas cosas de los hombres llegados en el bote. Le pregunté si le gustaría regresar allá y lo vi sonreírse al responder que no era capaz de nadar hasta tan lejos. Cuando le propuse construirle una canoa para que se volviera me contestó que lo haría si yo lo acompañaba.
    — ¿Yo? —dije—. ¡Me devorarán apenas llegue a tu país!
    —No, no —insistió Viernes—. Yo hacer ellos no coman vos, yo hacer ellos amaros mucho.
    Evidentemente se proponía narrarles cómo había matado a sus enemigos para salvarle la vida, y contaba ganar con eso el afecto de su pueblo. Inmediatamente se puso a explicarme lo bondadosos que eran con aquellos diecisiete hombres blancos (u hombres barbudos, como él les llamaba) que habían llegado indefensos a la costa.
    Desde ese momento confieso que sentí el impulso de aventurarme en el mar y ver si era posible dar con esos hombres que, a mi juicio, debían ser españoles o portugueses. No dudaba de que en su compañía sería posible intentar una fuga de aquellas regiones continentales, cosa más simple que salir sin ayuda y completamente solo de una isla situada por lo menos a cuarenta millas de tierra firme. Días más tarde volví a sondear el ánimo de Viernes y le dije que le daría un bote para que pudiese volver a su nación. Llevándolo hasta el otro lado de la isla, donde fondeaba mi canoa, la saqué del agua, ya que habitualmente la tenía sumergida, y luego de achicarla se la mostré y nos embarcamos en ella.
    Vi en seguida que era muy diestro en la maniobra, capaz de pilotear el bote con la misma rapidez y habilidad que yo. Mientras estaba a bordo le dije:
    —Bueno, Viernes, ¿nos vamos a tu nación?
    Noté que la pregunta le causaba un efecto desagradable, probablemente porque el bote le parecía demasiado pequeño. Le dije entonces que tenía otro más grande, y al día siguiente lo llevé al lugar donde construyera la chalupa y fracasara luego de mi tentativa de botarla. Viernes afirmó que era suficientemente grande, pero yo vi que el abandono en que la chalupa había quedado por espacio de veintidós o veintitrés años la había resquebrajado y destruido mucho, tanto que apenas parecía aprovechable. Viernes insistía en que un bote de ese tamaño era el adecuado para el viaje, y que llevaría «muchos bastantes víveres, bebida, pan», según su pintoresco lenguaje.
    Tan ardiente era ya entonces mi deseo de navegar con él hasta el continente, que le propuse construirle una chalupa tan grande como aquélla y darle libertad para que se volviera a su tierra. No contestó una palabra, pero se puso muy pensativo y triste. Le pregunté qué le ocurría, y a esto me contestó con otra pregunta:
    — ¿Por qué amo enojado con Viernes? ¿Qué haber hecho?
    Quise saber qué significaba aquello, y le aseguré que no estaba en modo alguno enfadado con él.
    — ¡No enfadado, no enfadado! —exclamó, repitiendo varias veces la palabra—. ¿Por qué mandar Viernes entonces a su nación?
    —Pero, Viernes, ¿no decías que estabas ansioso de volver con los tuyos?
    —Sí, sí —dijo él—. Los dos allá, no desear Viernes allá, amo acá.
    En suma, que no quería pensar en irse sin mí.
    — ¿Ir yo allá, Viernes? —le dije—. ¿Y qué podría hacer allá, dime?
    Me respondió vivamente:
    —Vos hacer bien mucho, enseñar hombres salvajes ser buenos, amigos; enseñarles conocer Dios, rogar Dios, vivir nueva vida.
    — ¡Ah, Viernes! —exclamé yo—. No sabes lo que dices. Yo soy un pobre ignorante.
    —Sí, sí, vos enseñarme bien, vos enseñar bien ellos.
    —Oh, no, Viernes —repetí—. Vete solo a tu pueblo; déjame aquí viviendo como antes.
    Cuando le dije eso pareció quedarse confuso y aturdido. Luego, corriendo a tomar una de las hachuelas que habitualmente llevaba, la trajo y me la presentó.
    — ¿Qué quieres que haga con ella? —me pregunté.
    —Amo matar a Viernes.
    — ¿Por qué habría de matarte? •
    — ¿Por qué enviar Viernes lejos? —le replicó rápidamente—. Matar Viernes, no enviar lejos.
    Estaba tan profundamente emocionado que le vi lágrimas en los ojos, y entonces tuve la prueba del profundo cariño que me tenía. Tan resuelto estaba que me apresuré a decirle muchas veces que jamás lo alejaría de mi lado si su voluntad era acompañarme en la isla.
    Aquel episodio no solamente sirvió para demostrarme el profundo afecto de Viernes y su voluntad de no separarse de mi lado, sino que la verdadera razón de sus deseos de volver a su pueblo se fundaba en el cariño que le tenía y su esperanza de que yo pudiera hacerle mucho bien. En cuanto a esto, seguro de mis pocas fuerzas, no me sentía inclinado en lo más mínimo a intentarlo; pero mis ansias de libertad se veían reforzadas por la existencia en el continente de aquellos diecisiete hombres blancos. Sin querer perder más tiempo empecé a buscar con ayuda de Viernes un árbol lo bastante grande para construir una piragua o canoa que soportara la travesía. En la isla había árboles en cantidad como para construir una pequeña flota, no ya de piraguas sino de barcos mayores. Con todo, necesitaba dar con un árbol que estuviera lo bastante cerca del agua para que luego, al tratar de botar la canoa, no ocurriera lo mismo que la primera vez.
    Por fin Viernes señaló el indicado; yo había advertido que era más capaz que yo de reconocer las maderas apropiadas, aunque me sería imposible decir aquí cuál era el árbol que derribamos, excepto que se asemejaba bastante al que llamamos fustete, y también al palo de Nicaragua, al que se parecía por el olor y el tono. Viernes pretendía quemar el centro del tronco para darle la forma de canoa, pero le enseñé a hacerlo con ayuda de herramientas, y cuando hubo aprendido trabajó con habilidad extremada. Un mes más tarde terminamos la canoa, que quedó muy bien; usando nuestras hachas, Viernes y yo habíamos cortado la parte exterior dándole la exacta forma de un bote. Todavía nos quedaba tarea, y pasó una quincena hasta que pudimos llevarla al agua, arrastrándola pulgada a pulgada sobre gruesos rodillos; pero cuando estuvo a flote vimos que era capaz de contener a bordo veinte hombres con toda facilidad.
    Ya en el agua, y aunque se trataba de un bote grande, me maravilló observar con qué destreza y rapidez lo gobernaba Viernes, haciéndole variar el rumbo y empleando los remos. Le pregunté entonces si se animaba a que intentáramos el viaje en la canoa.
    —Sí —dijo—. Navegar muy bien, aunque viento fuerte sople.
    Tenía yo un proyecto del que Viernes no estaba enterado, y era construir un mástil y una vela, así como proveer de ancla a nuestra embarcación. Era fácil encontrar un tronco para mástil, pues había muchos hermosos cedros en la isla y elegí uno joven y absolutamente recto que crecía cerca del mar. Ordené a Viernes que lo cortara, y le di todas las instrucciones necesarias. En cuento a la vela, su construcción me preocupaba seriamente; cierto que tenía viejas velas, o más bien cantidad de pedazos; pero aquellos trozos llevaban veintiséis años conmigo y como no los había cuidado mayormente por no imaginar jamás que llegaría a utilizarlos en esta forma, no dudaba que se habrían estropeado.
    En efecto, me bastó examinarlos para comprobar que la mayoría no estaba en condiciones de uso. Con todo hallé dos pedazos que se conservaban fuertes, y me puse con ellos al trabajo. No sin grandes fatigas y tediosas costuras —para las cuales carecía de verdaderas agujas— conseguí por fin hacer una tosca vela triangular como las que en Inglaterra llamamos «espalda de carnero», con un botalón en la base y una corta botavara en lo alto, tal como la llevan habitualmente las chalupas de nuestros navíos. Esa clase de vela me era familiar y sabía cómo manejarla, ya que una igual tenía la chalupa a cuyo bordo escapé de Berbería, como ha sido contado en la primera parte de esta historia.
    Casi dos meses me ocupó la tarea, es decir, fijar el mástil y montar la vela; además, para completar la arboladura, agregué un pequeño estay al mástil, y a él fijé una vela menor, especie de trinquete que ayudaría a tomar el viento. Finalmente, y esto fue lo más importante, puse un timón en la popa de la canoa; aunque pésimo carpintero naval, como me había dado cuenta de la utilidad y hasta la necesidad de dicho gobernalle, hice cuanto pude para que resultara bien y al fin lo conseguí; pero teniendo en cuenta las diversas tentativas que fracasaron sucesivamente, estoy seguro de que sólo el timón me costó más que la embarcación entera.
    Ya todo listo, faltaba adiestrar a Viernes en las maniobras del pilotaje, porque aunque era muy hábil en dirigir una canoa de remo ignoraba completamente lo referente a las velas y el timón. Se asombraba enormemente al verme dirigir el bote con una u otra dirección con el gobernalle, variar la posición de las velas para modificar el rumbo; su admiración no tenía entonces límites. Pronto, sin embargo, aquello se tornó familiar para él y al poco tiempo era un buen marinero, salvo para la brújula, que no conseguí hacerle entender sino imperfectamente. Cierto que en aquellas latitudes había muy pocos días nublados o con niebla, de manera que poca aplicación tenía la brújula cuando las estrellas servían de guía por la noche y la línea de la costa durante el día. En la estación lluviosa, por otra parte, nadie pensaba en navegar y ni siquiera hacer viajes por tierra firme.
    Se iniciaba el vigésimo séptimo año de mi cautiverio en la isla, aunque pienso que los tres últimos, estando en compañía de Viernes, deberían ser puestos fuera de la cuenta, ya que durante ellos mi vida tuvo un carácter muy diferente de la anterior. Celebré el aniversario de mi arribo con igual reconocimiento que en otras ocasiones por las bondades de Dios. En verdad que si entonces no me faltaban motivos para mostrarme agradecido, ahora debía estarlo aún más con las nuevas pruebas que tenía de la bondad de la Providencia y las esperanzas que en mí renacían de verme pronto liberado de aquella soledad. Día a día se acentuaba en mí el pensamiento de que mi libertad no tardaría en llegar y que ni siquiera alcanzaría a estar otro año en la isla. Cuidé sin embargo de proseguir mis tareas domésticas tales como plantar, cercar y ocuparme de la casa al igual que antes. Coseché y sequé mis uvas, y atendía como siempre a cada cosa necesaria.
    Vino la estación lluviosa, obligándome a permanecer a cubierto buena parte del tiempo. Fue preciso entonces cuidar de la chalupa, y la llevamos a la ensenada donde, como he contado, llegara con las balsas trayendo el cargamento del barco. Después de vararla en la costa aprovechando la marea alta, hice que Viernes cavara una pequeña rada lo bastante grande para contenerla y que aún flotara en ella; luego, al descender la marea, levantamos un fuerte dique en el extremo de la rada a fin de impedir que el agua volviera hasta allí, y la canoa quedó en seco, libre del mar. Para preservarla de las lluvias la cubrimos con tal cantidad de ramas de árbol que quedó como techada, y dejándola a la espera de noviembre y diciembre, tiempo en el que sería posible intentar la aventura.
    Cuando comenzó a manifestarse el buen tiempo, y como si el deseo de ejecutar mis planes creciera con él, diariamente hacía yo preparativos de viaje; lo primero fue almacenar cantidad suficiente de provisiones, calculando que nos alcanzaran para la travesía. Una semana o quince días más tarde esperaba derribar el dique y poner a flote la embarcación.
    Una mañana me ocupaba en estas tareas, cuando se me ocurrió llamar a Viernes y mandarlo a que fuera a la costa en busca de una tortuga, cosa que hacíamos generalmente una vez por semana para comer su carne y los huevos. No llevaba Viernes mucho tiempo ausente cuando lo vi volver corriendo y saltar el vallado como uno que no toca el suelo con los pies. Antes que hubiera podido hablarle, gritó: — ¡Oh amo, amo! ¡Desgracia! ¡Pena! — ¿Qué te ocurre, Viernes? — ¡Allá, allá! —exclamó—. ¡Una, dos, tres canoas! ¡Una, dos, tres!
    Por su manera de expresarse deduje que eran seis canoas, pero al interrogarlo vi que sólo eran tres.
    —Bueno, Viernes —le dije—, no te asustes.
    Traté de animarlo lo mejor posible, pero me di cuenta de que el pobre muchacho estaba mortalmente aterrado. Parecía convencido de que los salvajes venían exclusivamente en su busca, dispuestos a descuartizarlo y a comérselo; temblaba de tal manera que no sabía qué hacer con él. Traté de conformarlo y le dije que también yo estaba en peligro, ya que si nos capturaban sería igualmente devorado.
    —Por eso, Viernes —agregué—, tenemos que resolvernos a pelear. ¿Sabes tú pelear?
    —Yo tirar —dijo él—, pero ellos venir gran número.
    —Eso no importa, Viernes; nuestras escopetas asustarán a los que no hieran.
    Le pregunté entonces si estaba dispuesto a defenderme come yo a él, y si permanecería a mi lado obedeciendo las órdenes que le diera.
    —Yo morir cuando vos mandar —dijo.
    Busqué entonces ron y le di a beber un buen trago; por fortuna había cuidado tanto el licor que me quedaba todavía mucho. Luego que hubo bebido, le di las dos escopetas que llevábamos siempre, cargadas con munición muy gruesa, casi como balines de pistola. Tomé cuatro mosquetes, cargándolos con dos plomos y cinco balines cada uno. A las dos pistolas les puse un puñado de balines y dando a Viernes su hachuela me colgué a la cintura mi sable desnudo.
    Así pertrechados, tomé el anteojo y ascendí a la cumbre de la colina para observar a los enemigos. Me bastó fijar sobre ellos el anteojo para descubrir que había veintiún salvajes, tres prisioneros y tres canoas, y que su intención allí no era otra que proceder a un banquete triunfal con los cuerpos de sus víctimas. Fiesta bárbara, ciertamente, pero sin nada que la distinguiera de las que se llevaban a cabo habitualmente entre ellos.
    Noté también que no habían desembarcado en el sitio que lo hicieran cuando Viernes pudo escaparse, sino más cerca de mi ensenada, donde la costa era más baja y el espeso bosque llegaba casi hasta el mar. Esto, más el horror que la inhumana costumbre de aquellos monstruos me producía, me llenó de tal indignación que descendí a buscar a Viernes y le anuncié que estaba dispuesto a caer sobre los salvajes y matarlos, por lo cual quería saber si contaba con él. Ya se le había pasado el susto y el ron había estimulado sus ánimos, de modo que parecía bien dispuesto y me repitió que moriría si yo se lo mandaba.
    Sin poder contener la furia repartí entre los dos las armas que ya había cargado. Di a Viernes una pistola para llevar en el cinturón, y tres escopetas que se colgó al hombro; tomé la otra pistola y las escopetas restantes y así nos pusimos en marcha. Llevaba yo una botellita de ron en el bolsillo y di a Viernes un saco con bastante pólvora y balas; le ordené que se mantuviera constantemente a mi lado y que no se moviera o tirara hasta recibir una indicación mía; entretanto era preciso no pronunciar una sola palabra. Tomando por la derecha, describimos un círculo de cerca de una milla a fin de llegar a la ensenada por la parte alta cubierta de bosque, y hallarnos a tiro antes de que pudieran descubrirnos; de acuerdo con lo que me había mostrado mi observación con el anteojo, esto era bastante fácil de llevar a cabo.
    Mientras nos acercábamos sigilosos, mis pensamientos empezaron a perder su primitivo ardor. No es que tuviera miedo al número de salvajes, puesto que sabiéndolos desnudos y casi desarmados me sentía superior a ellos, incluso estando solo. Pero volvía a preguntarme qué razón, qué motivo y, lo que es más importante, qué necesidad podía impulsarme a correr hacia esas gentes y bañar mis manos en su sangre, atacándolos sin que me hubiesen hecho daño alguno o tuvieran intención maligna hacia mí. Me dije que si a Dios le parecía justo, El mismo tomaría la venganza en Sus manos y castigaría en conjunto a aquellas gentes, como a una nación, por sus crímenes nefandos; pero en el Ínterin nada de eso me concernía. Viernes, por su parte, tenía una justificación al atacar, puesto que era enemigo declarado de aquellos salvajes, su pueblo estaba en guerra con el de ellos y era legal que los atacara si podía; pero yo no estaba en las mismas circunstancias.
    Tanto me oprimieron aquellas meditaciones mientras nos acercábamos por el bosque, que por fin resolví apostarme solamente en las cercanías de la playa y observar su bárbaro festín, actuando entonces según creyera que Dios me lo ordenaba; hasta entonces, y mientras no recibiera un impulso que me sirviera de suficiente justificativo, estaba dispuesto a no intervenir en lo que ocurría.
    Así resuelto penetramos en el bosque, y andando con toda la cautela y silencio posibles, Viernes pegado a mis espaldas, llegamos hasta el borde arbolado en la parte más próxima al sitio donde estaban reunidos, y del que sólo nos separaba una franja de bosque. Llamando en voz baja a Viernes, le mostré un gran árbol que formaba justamente la saliente del bosque y le dije que fuese hasta allá a observar lo que estaban naciendo los salvajes. Volvió un momento después diciéndome que desde allí se los veía muy bien, que estaban en torno a la hoguera comiendo la carne de uno de los prisioneros, y que otro yacía en la arena, un poco más lejos, esperando su turno. Pero lo que me encendió el alma de coraje fue enterarme de que aquel prisionero no era un caribe sino uno de los hombres barbudos que, según Viernes me contara, habían llegado a la costa en un bote. Sentí que el horror me dominaba a la sola mención de un hombre blanco en tal estado y yendo hasta el árbol pude divisar, con ayuda del anteojo, que efectivamente se trataba de un semejante mío, tirado en la arena con las manos y los pies atados con cuerdas o juncos, y que indudablemente se trataba de un europeo por las ropas que tenía puestas.
    Vi otro árbol, con un matorral adyacente, a unas cincuenta yardas más cerca de los salvajes que el lugar en que ahora estábamos, y al que era fácil llegar con un pequeño rodeo: allí nos pondríamos a medio tiro de escopeta solamente. Reprimiendo, pues, mi furor, aunque estaba encolerizado hasta el límite, retrocedí unos veinte pasos y luego me deslicé por entre los arbustos, que me ocultaron hasta poder apostarme en aquel árbol. Llegué así a una pequeña eminencia del suelo, desde donde tenía una vista total de la escena a menos de ochenta yardas.
    No había un solo momento que perder, pues diecinueve de aquellos horribles monstruos permanecían unos contra otros rodeando el fuego mientras los dos restantes acababan de levantarse con intención de matar al infeliz cristiano y conducirlo, probablemente ya descuartizado, al fuego. Vi que se inclinaban a desatarle las cuerdas de los pies, y me volví a Viernes.
    —Haz lo que te mande —dije, y cuando él asintió agregué—: Pues bien, imítame en todo lo que me veas hacer, y no vaciles ante nada.
    Puse en tierra uno de los mosquetes y la escopeta, y Viernes repitió mis actitudes; tomando luego el otro mosquete, apunté a los salvajes indicándole que me imitara. Luego, al preguntarle si estaba listo y contestarme él que sí, ordené:
    — ¡Fuego, entonces!
    Viernes había apuntado mucho mejor que yo, pues del lado de su tiro vi caer dos salvajes muertos y tres heridos, mientras que yo alcancé a matar a uno y herir a dos. Es de imaginarse la confusión que reinaba entre ellos. Los que no habían recibido heridas saltaron precipitadamente, pero no sabían hacia dónde huir o qué hacer, ya que ignoraban de dónde les llegaba la muerte. Viernes tenía los ojos puestos en los míos para imitar todos mis movimientos, como se lo ordenara. Tan pronto como hubimos disparado, dejé caer el mosquete y tomé la escopeta, cosa que él repitió al punto. Al mismo tiempo amartillamos y apuntamos las armas.
    — ¿Estas listo, Viernes? —pregunté.
    —Sí —repuso.
    — ¡Fuego, entonces, en nombre de Dios!
    Y por segunda vez descargamos las armas sobre los aterrados salvajes. En esta ocasión, como las escopetas tenían por carga balines pequeños de pistola, solamente cayeron dos enemigos, pero tantos resultaron heridos que los vimos correr enloquecidos, aullando y cubiertos de sangre, la mayoría con múltiples heridas; otros tres fueron cayendo luego, aunque no muertos.
    —Ahora, Viernes —mandé dejando en tierra la pieza y levantando el otro mosquete cargado—, ¡sígueme!
    Con gran valor se levantó para obedecerme, y nos precipitamos fuera del bosque exponiéndonos a la vista de los salvajes. Tan pronto como advertí que me habían descubierto lancé un terrible alarido, mientras Viernes hacía lo mismo, y avanzamos a la carrera —no demasiado rápida por el peso de las armas que llevábamos— en dirección donde yacía la pobre víctima, tendida como he dicho en la arena entre la hoguera y el mar. Los dos carniceros que se disponían a descuartizar al prisionero acababan de abandonarlo con el terror de los disparos, huyendo a toda carrera hacia el mar, donde saltaron a una canoa, seguidos por otros tres. Mandé a Viernes que disparara sobre ellos, y comprendiendo en seguida corrió hasta situarse a unas cuarenta yardas y desde allí descargó el arma sobre los que huían. Pensé que los había matado a todos porque cayeron en montón dentro de la piragua, pero dos de ellos se enderezaron al instante. Con todo había muerto a dos y herido a un tercero, que yacía en el fondo de la canoa como fulminado.
    Mientras Viernes se entendía con ellos, extraje el cuchillo y corté los lazos que ataban a la pobre víctima. Lo ayudé a incorporarse, mientras le preguntaba en portugués quién era. Me contestó en latín: «Christianus», pero estaba tan débil que apenas podía hablar o moverse. Le di a beber un trago de ron que había traído en una botella haciéndole señales que bebiera para reanimarse, y también saqué del bolsillo un trozo de pan, que comió. Al preguntarle a qué nación pertenecía, me contestó.
    —Español.
    Ya un poco recobrado de su postración, me dejó entender con toda suerte de signos y ademanes lo reconocido que me estaba por haberlo salvado.
    —Señor —le dije en el mejor español que recordaba—, luego hablaremos, pero ahora es preciso pelear. Si os quedan fuerzas tened esta pistola y esta espada y ved de emplearlas.
    Las recibió con gratitud y apenas las hubo empuñado cuando pareció que con ellas recobraba todo su vigor, pues se lanzó como una furia sobre los asesinos y en un instante mató a dos a estocadas. La verdad es que aquellos infelices estaban tan espantados con la sorpresa que les habíamos dado y el estampido de las armas que el miedo los tenía como atontados y carecían de inteligencia para escapar o combatir en defensa de la vida. Eso era justamente lo sucedido en la canoa sobre la cual Viernes había disparado; aunque sólo tres de los cinco cayeron por efecto de las heridas, los otros dos lo habían hecho a causa del espanto sufrido.
    Mantuve el mosquete listo, sin dispararlo, queriendo reservar la carga porque había dado mi pistola y el sable al español. Llamé a Viernes, ordenándole que corriera hasta el árbol y trajera aquellas armas que habían quedado allí descargadas. Lo hizo a gran velocidad, y mientras él me escudaba con el mosquete me puse a cargar las armas, gritando a mis compañeros que acudiesen a buscarlas a medida que las necesitaran. Mientras me ocupaba en esto observé que se desarrollaba una terrible lucha entre el español y uno de los salvajes, que lo atacaba con una pesada espada de madera, justamente la misma arma que habrían empleado para descuartizarlo si yo no lo hubiera impedido. El español, que era tan osado y valiente como pueda imaginarse, había luchado sin ceder terreno a pesar de su extrema debilidad, y ya había herido dos veces al salvaje en la cabeza; pero aprovechando su falta de fuerzas el astuto y robusto enemigo acabó por acortar distancias y luego, derribando al español, parecía a punto de arrebatarle mi espada de la mano. Fue entonces cuando el español tuvo la inteligencia de abandonar la espada mientras sacaba de la cintura la pistola que le diera, y disparándole un tiro a quemarropa dejó muerto al salvaje antes de que yo pudiera llegar en su ayuda.
    Viernes, librado a su criterio, se había puesto a perseguir a los restantes sin más arma que su hachuela. Con ella acabó de matar a los tres que primeramente habían caído heridos, luego a todos los que pudo alcanzar. El español vino a mí en busca de un arma y le entregué una escopeta, con la cual logró herir a dos salvajes, pero como no tenía fuerzas para correr en su persecución se refugiaron en el bosque donde fue a buscarlos Viernes y mató a uno. El otro era sin embargo demasiado ágil para él, y, aunque herido, logró zambullirse en el mar y reunirse, nadando rápidamente, a los dos sobrevivientes de la canoa. Esos tres salvajes, más uno herido, que ignoramos si murió o no, fueron los únicos que se salvaron sobre veintiuno. La suerte de los restantes fue la siguiente:


Muertos por nuestro primer disparo desde el árbol   .......     3 Muertos por el segundo disparo    ...................................     2 Muertos por Viernes en la canoa.......................................     2 Heridos primero y muertos después por él mismo   .........      2 Muerto por él mismo en el bosque   ................................     1 Muertos por el español   ..................................................     3 Muertos a  causa de las heridas, o rematados por Viernes     4 Escapados en la canoa, de los cuales uno herido o muerto          4 TOTAL   .................................................................................   21

    Los que se salvaron en la canoa huyeron a toda velocidad para escapar a nuestras balas, y aunque Viernes les hizo dos o tres disparos no creo que alcanzara a ninguno. El muchacho quería que tomásemos una de las canoas y los persiguiéramos, lo que me pareció bien, ya que me inquietaba mucho su fuga por temor a que consiguieran llegar a sus playas y avisaran de lo ocurrido a sus compañeros, quienes podían volver en gran número y terminar por apoderarse de nosotros y devorarnos. Corriendo, pues, a una de las canoas, salté en ella y Viernes hizo lo mismo. Pero grande fue mi sorpresa al encontrar en el fondo de la piragua otra víctima atada de pies y manos, destinada al sacrificio lo mismo que el español y casi muerta de miedo por no darse cuenta clara de lo que ocurría. Tan fuertemente estaba atado el pobre hombre que le había sido imposible enderezarse para mirar por la borda de la canoa, y me dio la impresión de que en realidad le quedaba poca vida.
    Inmediatamente corté los lazos o juncos que le sujetaban los miembros y traté de ayudarlo a incorporarse, pero él no podía ni sostenerse ni hablar, y solamente se quejaba con voz lastimera, creyendo probablemente que lo desataba para asesinarlo y comerlo.
    Cuando Viernes llegó a mi lado le ordené que hablase al salvaje y le dijera que estaba libre. Sacando la botella le hice beber un trago, lo cual, junto a la noticia de que se había salvado, lo reanimó bastante y dióle fuerzas para sentarse en la canoa. Pero cuando Viernes se le acercó para hablarle y le vio la cara, de improviso empezó a abrazarlo, a besarlo, estrechándolo en sus brazos con fuerza y haciendo tales demostraciones de alegría que nadie hubiera podido contener las lágrimas al presenciar tal escena. Reía, lanzaba exclamaciones de entusiasmo, saltaba y bailaba, luego se puso a cantar, llorando de emoción, retorció sus manos, se golpeó la cara y la cabeza, en fin, hizo tales demostraciones y dio tales saltos y gritos que parecía haber perdido enteramente el juicio. Pasó un largo tiempo antes de que consiguiera hacerlo hablar con claridad, pero cuando logré al fin que se calmara un poco, me dijo que aquel salvaje era su padre.
    No es fácil de expresar la emoción que me produjo aquel rapto de júbilo y de afecto filial que acababa de presenciar en el pobre salvaje a la vista de su padre librado de la muerte; ni siquiera puedo describir las formas extravagantes que adoptaba su entusiasmo, porque tan pronto saltaba a la canoa como fuera de ella. Por fin se sentó junto a su padre, y abriéndose la chaqueta apoyó la cabeza del anciano contra su pecho, teniéndolo así más de media hora para ayudarlo a recobrar las fuerzas; luego empezó a frotarle las muñecas y los tobillos, que estaban entumecidos por la fuerza de las ligaduras, y como advertí su intención le alcancé la botella para que frotara con ron los miembros agarrotados, lo que hizo gran bien al salvaje.
    Todo esto nos distrajo de la persecución de la otra canoa, que estaba ya casi pérdida en la distancia. Y fue fortuna para nosotros no embarcarnos tras ella, porque unas dos horas después, cuando la piragua no podía haber navegado más que un cuarto de la travesía, empezó a soplar viento fuerte que continuó toda la noche del noroeste, es decir, contra ellos, de manera que resulta difícil creer que se salvaran de un naufragio o que pudiesen volver a sus tierras.
    Mas retornemos a Viernes. Estaba tan ocupado con su padre que no tuve valor para alejarlo de allí, pero cuando pensé que podía abandonarlo por su instante lo llamé y vino saltando y riendo, con todas las demostraciones de la felicidad más completa. Le pregunté si había dado algo de pan a su padre.
    —No —repuso moviendo la cabeza—. Yo perro ruin comerme todo.
    Le di un pan de los que llevaba conmigo en un saquito, y le ofrecí un trago de ron, pero no quiso probarlo y se lo llevó a su padre. Como tenía en mis bolsillos algunos racimos de pasas, le di un puñado, y lo vi correr con todo eso hasta donde estaba el anciano, y de pronto alejarse de él y escapar como si estuviera alucinado. Jamás, por otra parte, he visto a nadie que tuviera velocidad comparable en la carrera. Con tal rapidez se alejó que en un momento estuvo fuera de mi vista, y aunque lo llamé a gritos fue como si no lo hiciese, pues siguió su marcha. Un cuarto de hora más tarde lo vi regresar, aunque ya no con la misma velocidad, pues parecía cuidar algo que tenía en la mano.
    Cuando pasó a mi lado vi que había ido hasta casa en busca de un jarro de agua dulce, y que traía además otros dos panes. Me entregó el pan y fue a dar el agua a su padre, pero antes bebí yo un trago porque me sentía sediento. El agua hizo más bien al anciano que el ron que yo le diera antes a tomar, porque la sed lo devoraba.
    Luego que hubo bebido llamé a Viernes para saber si quedaba un poco de agua. Como me contestara afirmativamente le ordené que la llevase al pobre español, que también la necesitaba mucho, así como uno de los panes. El español se sentía muy débil y reposaba a la sombra de un gran árbol, tendido en el césped; sus miembros estaban aún paralizados y tenían claras señales de la fuerza con que habían sido atados. Cuando vi que aceptaba el agua que Viernes le ofrecía, así como el pan que comió inmediatamente, me acerqué a él y le di un puñado de pasas. Alzó su mirada hacia mí con la más profunda expresión de reconocimiento que pueda pintarse en un rostro humano, pero estaba tan débil —pese a haber tenido aún fuerzas para combatir— que no le fue posible sostenerse en pie. Trató de hacerlo dos o tres veces, pero sus piernas no le sostenían y los tobillos le dolían mucho. Le dije entonces que se estuviera quieto, mientras Viernes le friccionaba los miembros con ron, tal como lo hiciera con su padre.
    Noté que el pobre y afectuoso muchacho miraba cada dos minutos o menos hacia el sitio donde había dejado a su padre, para saber si seguía allí en la misma actitud en que él lo colocara. De pronto, al no verlo, se enderezó y fue hacia allí con tal velocidad que sus pies apenas tocaban el suelo. Pero como el anciano solamente se había tendido en el suelo para reposar, Viernes volvió a nosotros; dije entonces al español que Viernes lo ayudaría a caminar hasta el bote, a fin de trasladarlo luego a la morada en que podríamos atenderlo convenientemente. Mi criado, que era muy fuerte, levantó al español y cargándolo sobre sus espaldas lo llevó hasta la canoa, donde lo dejó en el borde con mucho cuidado, los pies vueltos hacia el interior; levantándolo luego otra vez, lo hizo entrar del todo y lo puso al lado de su padre. Saltando de la canoa, la empujó para botarla al agua, y aunque el viento soplaba con fuerza la llevó a remo más pronto de lo que yo podía ir por la costa, y corrió por la playa en busca de la otra piragua. Cuando pasó junto a mí le pregunté adonde iba, y me respondió:
    —Buscar más canoa.
    Con la velocidad del viento lo vi alejarse, y por cierto que nunca caballo u hombre corrieron como él. Llegó tripulando la otra canoa casi al mismo tiempo que yo arribaba a la ensenada, y después de pasarme en ella al otro lado se puso a ayudar a nuestros nuevos huéspedes para que salieran de la piragua. Pero cuando estuvieron en tierra, como no les era posible dar un paso, el pobre Viernes no sabía qué hacer.
    Principié a pensar el modo de llevarlos a casa, y luego de ordenar a Viernes que los dejara cómodamente sentados en la playa, entre los dos construimos una especie de angarillas de tal modo que cupieran ambos, y así iniciamos la marcha. El problema se presentó al llegar a la fortificación exterior del castillo, ya que de ningún modo aquellos hombres tenían fuerzas para montar sobre el vallado ni yo estaba dispuesto a romperlo por su causa. Volví, pues, a ponerme a trabajar, y en un par de horas hicimos, Viernes y yo, una confortable tienda, cubierta con pedazos de velas y por encima ramas de árbol; la instalamos en el espacio abierto que había entre la empalizada exterior y el bosque que yo plantara. Pusimos finalmente allí dos camas hechas con el mismo material que teníamos a mano, es decir, paja de arroz, cubiertas con una manta a modo de colchón y otra por encima para abrigo.

    .


    14. UN ESPAÑOL Y ALGUNOS INGLESES




    Mi isla estaba ahora poblada y, de pronto, me encontré rodeado de muchos súbditos; frecuentemente afirmaba yo en broma que de veras parecía un rey. Ante todo, la tierra era de mi absoluta propiedad, lo cual me aseguraba un indiscutible derecho de dominio. Segundo, mi pueblo estaba formado por sumisos vasallos, de los cuales era señor y juez; todos me debían la vida, y estaban dispuestos a entregarla por mí si la ocasión se presentaba. Lo más digno de notarse era que mis tres súbditos pertenecían a religiones distintas. Mi criado Viernes era protestante; su padre, pagano y caníbal, y el español, católico. Dicho sea de paso, yo había asegurado la libertad de conciencia en todos mis dominios.
    Tan pronto como los rescatados prisioneros estuvieron bajo techo y con suficiente abrigo, me puse a pensar en la forma de alimentarlos. Lo primero que hice fue ordenar a Viernes que eligiera del rebaño un cabrito como de un año y lo matara; saqué de él un cuarto trasero que corté en pequeños trozos, que Viernes hirvió y guisó; en esta forma obtuvimos un buen plato de carne y caldo, al que agregamos algo de cebada y arroz. Como cocinaba al exterior porque no quería encender fuego dentro de la empalizada, llevé todo a la nueva tienda y poniendo allí una mesa nos, sentamos a comer en compañía. Traté de animar a los huéspedes y darles coraje. Viernes era mi intérprete ante su padre, e incluso ante el español, ya que éste hablaba muy bien el idioma de los salvajes.
    Luego de haber comido o más bien cenado, ordené a Viernes que fuera en una de las canoas a recoger los mosquetes y demás armas que por falta de tiempo dejáramos tirados en la costa; también le mandé al día siguiente enterrar los cadáveres de los salvajes que yacían bajo el sol y pronto ofenderían el olfato. Le dije que hiciera lo mismo con los numerosos restos del bárbaro festín, ya que jamás habría podido emprender yo una tarea semejante; ni siquiera me animaba a mirar aquello cuando pasaba cerca. Viernes cumplió puntualmente mis indicaciones, y borró de tal modo la huella de la presencia de los caníbales que cuando volví al sitio apenas reconocí el escenario de la lucha, salvo por el extremo del bosque que apuntaba hacia allí.
    Ya entonces empezaba a sostener alguna conversación con mis dos nuevos súbditos. Pedí a Viernes que preguntara a su padre lo que pensaba sobre la fuga de los salvajes sobrevivientes, y si era de esperar que volvieran pronto en número tan grande que tornasen imposible toda defensa. Su primera opinión fue que los salvajes no eran capaces de resistir la violencia del temporal de aquella noche; si no se habían ahogado habrían derivado hacia las costas del sur, donde sin duda otras tribus caníbales los devorarían. En cuanto a lo que pudieran hacer si llegaban sanos y salvos a su tierra, no lo sabía, pero su opinión fue que sin duda estaban tan aterrados por aquel ataque inexplicable, las detonaciones y el fuego, que sin duda dirían a sus compatriotas que no era una mano humana sino el mismo rayo quien había exterminado a los restantes. A sus ojos, Viernes y yo debimos aparecer ante ellos como la encarnación misma de los dioses infernales presentándose para destruirlos sin armas humanas. Agregó que estaba seguro de ello porque tales cosas los oyó gritarse unos a otros en su lenguaje, que entendía bien; ninguno parecía concebir la posibilidad de que un hombre pudiera arrojar fuego, hablar tronando y matar a tal distancia sin siquiera levantar la mano, como había ocurrido. Pronto vimos que el viejo salvaje estaba en lo cierto, porque como llegué a enterarme más tarde por otros testimonios, jamás un caníbal se atrevió a pisar de nuevo aquella isla. Tan aterrados quedaron con el relato hecho por los cuatro sobrevivientes (que, por lo visto, se salvaron de la tempestad) que desde entonces vivieron convencidos de que cualquier osado que desembarcara en la isla sería destruido por el fuego de los dioses.
    Naturalmente yo ignoraba entonces todo aquello, y durante mucho tiempo viví bajo una continua aprensión y sin descuidar las precauciones que nuestro pequeño ejército adoptaba. Ahora éramos cuatro y, sin temor, nos hubiéramos enfrentado en cualquier momento con un centenar de salvajes en campo abierto.
    Como pasara el tiempo y las canoas enemigas no volvían, empecé a perder el miedo y otra vez invadieron mi mente los proyectos de escapar de la isla. Tenía ahora como un nuevo incentivo la garantía formalmente dada por el padre de Viernes asegurándome que si yo iba con ellos hasta su nación sería muy bien tratado bajo su responsabilidad.
    Con todo, mis proyectos se enfriaron algo después de sostener una conversación con el español, por el cual vine a saber que dieciséis de sus connacionales, así como portugueses vivían en paz con los salvajes después de haberse salvado de un naufragio que los arrojó a aquellas costas; pero su existencia era muy penosa, y pasaban inmensas privaciones en las cuales hasta su vida estaba amenazada. Le pedí datos sobre su viaje y supe que los náufragos pertenecían a un barco fletado desde el Río de la Plata con destino a La Habana, donde debían desembarcar su cargamento, compuesto principalmente de pieles y plata, y retornar trayendo manufacturas europeas que pudieran encontrarse en aquel puerto. Me contó que llevaban a bordo cinco marinos portugueses a quienes recogieron en el mar, y que, cuando naufragaron más tarde, otros cinco españoles perecieron ahogados, mientras los que consiguieron salvarse habían llegado después de infinitas aventuras y peligros, casi extenuados de hambre a la costa de los caníbales, donde esperaban de un momento a otro ser devorados.
    Me dijo que tenían algunas armas con ellos, pero que de nada les servían, pues el agua de mar les había arrebatado o estropeado completamente la pólvora, de la que sólo salvaron una pequeña porción que fue empleada para procurarse algún alimento cuando desembarcaron.
    Pregunté al español qué sería de ellos allí, y si no tenían algún proyecto de fuga. Me dijo que muchas veces lo habían pensado, pero como carecían de embarcación y de toda herramienta para construirla, sus conciliábulos terminaban siempre en lágrimas y desesperación.
    Quise saber de qué manera recibirían una propuesta mía que pudiera ayudarlos a escapar, y si él veía alguna posibilidad de fuga desde mi isla una vez que todos consiguiéramos reunimos en ella. Le advertí con toda franqueza que mi mayor temor era el de que me traicionaran una vez que hubiese puesto mi vida en sus manos; sabía de sobra que la gratitud no es una cualidad inherente a la naturaleza humana, y que con frecuencia obran los hombres más de acuerdo a las ventajas que esperan obtener que por los favores que hayan recibido. Le manifesté que me resultaría harto cruel ser instrumento de su liberación para que luego me llevaran prisionero a Nueva España, donde todo inglés está seguro de ser sacrificado cualesquiera sean los motivos que lo hayan conducido allí. Por cierto que prefería mucho más caer en manos de los salvajes y ser devorado vivo por ellos, que en las garras despiadadas de los frailes y la Inquisición. Le participé mi seguridad de que si tantos hombres conseguíamos reunimos sería posible construir una embarcación capaz de llevarnos hacia el sur, es decir, al Brasil, o bien hasta las islas españolas del norte, pero que si después de haberlos ayudado y puesto armas en sus manos me arrastraban por fuerza a su tierra, de nada me habría valido mi generosidad para con ellos y mi situación sería mucho peor que la presente.
    Con una gran sinceridad y franqueza me contestó el español que la condición de sus compañeros era tan triste y de tal modo conocían sus miserias, que de ninguna manera le parecía admisible que llegaran a traicionar a quien les ofrecía la libertad. Agregó que si yo lo autorizaba él iría a ellos en compañía del anciano salvaje y les plantearía la proposición para volver de inmediato con la respuesta. Sólo iba a tratar con ellos previo solemne juramento de que se pondrían incondicionalmente a mis órdenes en mi carácter de jefe y capitán; tal juramento sería realizado sobre los Santos Sacramentos y el Evangelio, comprometiéndose a serme fieles, dirigirse al país cristiano que a mí me pareciera bien y, en una palabra, obedecer total y absolutamente mis órdenes hasta que hubiésemos arribado con felicidad al país que yo designara. Finalmente me prometió que todo aquello sería redactado por escrito para mayor seguridad, que él iba a ser el primero en pronunciar el juramento y que jamás se apartaría de mi lado mientras yo no dispusiera otra cosa. Hasta la última gota de sangre estaba dispuesto a verter por mí si advirtiera la menor señal de mala fe entre sus camaradas.
    De acuerdo con lo que el español me dijo, los náufragos eran hombres honrados y buenos que se encontraban en la peor de las miserias, privados de armas y ropas, casi sin comer y enteramente a merced de los salvajes; no conservaban ninguna esperanza de regresar algún día a su país, de modo que si yo emprendía su liberación se sentirían tan agradecidos como para consagrarme su vida e incluso morir por mí.
    Asegurado de tales cosas, me resolví a tentar la aventura de librarlos si era posible, y enviar ante todo al viejo salvaje y al español con mis propuestas. Todo estaba ya listo para el viaje cuando el español me señaló una objeción, hecha con tal prudencia por un lado y tanta sinceridad por otro que la encontré atinadísima, llevándome a postergar la liberación de sus camaradas por otros seis meses como mínimo.
    He aquí por qué: durante el tiempo que llevaba el español a mi lado, alrededor de un mes, le había mostrado yo los recursos que poseía para alimentarme con ayuda de la Providencia. Había podido observar mis depósitos de cebada y arroz, harto abundantes para mí, pero que debimos economizar ahora que nuestra familia se elevaba a cuatro miembros. Como es natural mucho menos podrían bastarnos cuando los sobrevivientes del naufragio que sumaban dieciséis hombres en ese momento, llegaran a la isla. Todavía menos podían alcanzarnos esas reservas para avituallar el navío que pensábamos construir y que debía llevarnos hasta alguna de las colonias cristianas de América. El español me manifestó que le parecía preferible cultivar y sembrar nuevas tierras, y que yo destinara a ello todo el grano de que pudiera desprenderme; luego esperaríamos otra cosecha, a fin de tener cantidad suficiente para cuando sus compatriotas desembarcaran. La falta de alimentos podía, en caso contrario, ser causa de disgustos o de que aquellos hombres no se sintieran libertados en modo alguno sino simplemente movidos de una dificultad a otra no menos grave.
    —Bien sabéis —agregó— que aunque los hijos de Israel se regocijaron al principio por haber sido salvados del cautiverio en Egipto, acabaron por rebelarse contra el mismo Dios, su salvador, cuando les faltó pan en el desierto.
    Su prevención era tan razonable, su consejo tan excelente, que no podía yo sino sentirme satisfecho de haberlo recibido, así como me satisfacía su fidelidad hacia mí. Nos pusimos pues los cuatro a cavar la tierra, todo lo bien que nuestras herramientas de madera nos lo permitían, y un mes después, en la época propicia, habíamos dispuesto suficiente extensión para sembrar veintidós fanegas de cebada y dieciséis tinajas de arroz, lo que constituía toda la cantidad de que disponíamos para una siembra. Nos quedó apenas grano para alimentarnos durante los seis meses en que debíamos esperar la nueva cosecha, contando a partir del día en que apartamos el grano para sembrarlo, puesto que en aquellas latitudes la recolección se hace antes de ese plazo.
    Como entonces nos sentíamos acompañados mutuamente y nuestro número era bastante para alejar todo temor de los salvajes —salvo que la isla fuera invadida por un enorme ejército —andábamos libremente en cuanto se nos presentaba ocasión. La esperanza de alcanzar alguna vez la libertad nos asaltaba continuamente, por lo menos a mí. A tal efecto busqué varios árboles que me parecieron adecuados, y dije a Viernes y a su padre que los cortaran. En cuanto al español, que estaba al tanto de mis proyectos, se encargó de dirigir la tarea. Les mostré con qué indescriptibles fatigas había logrado yo hacer tablones de un enorme árbol, y los puse a la misma tarea hasta que hubieron obtenido una docena de grandes tablones de buen roble, de dos pies de ancho por treinta y cinco de largo y dos o cuatro pulgadas de espesor. Es de imaginarse el prodigioso trabajo que costó obtenerlos.
    Al mismo tiempo trataba de aumentar en todo lo posible mi rebaño de cabras. Un día enviaba al español con Viernes a cazar, y al siguiente íbamos Viernes y yo, alternándonos en la tarea; pronto capturamos así más de veinte cabritos que pusimos con las demás cabras en los corrales. Cuando encontrábamos cabras salvajes, matábamos una madre y nos apoderábamos inmediatamente de sus pequeños. Pero lo notable fue la cantidad de pasas que obtuvimos poniendo uvas a secar al sol cuando vino la época en que maduraron los racimos; creo que si hubiésemos estado en Alicante, donde se preparan las pasas, hubiéramos podido llenar sesenta u ochenta barriles. Junto con el pan, aquellas pasas eran lo principal de nuestro alimento, y por cierto muy agradable a la vez que de un gran valor alimenticio.
    Llegó el tiempo de la siega y nuestro grano espigó muy bien. No era la cosecha más productiva que viera yo en la isla, pero evidentemente alcanzaba para nuestros fines; las veintidós fanegas de cebada se convirtieron, después de la siega y trilla, en más de doscientas veinte fanegas, y la misma proporción se obtuvo de arroz, todo lo cual nos daba suficiente provisión hasta la siguiente cosecha, contando con nosotros a los dieciséis españoles. Incluso si hubiéramos estado ya listos para viajar, aquel grano hubiese sido bastante para avituallar el navío y alimentarnos en la travesía hasta cualquier punto de la costa americana.
    Concluida la tarea concerniente a la cosecha y almacenamiento del grano, nos pusimos a trabajar en cestería, haciendo grandes canastos donde pudiéramos guardar con seguridad la semilla. El español era muy diestro y habilidoso en dicho trabajo, y con frecuencia me reprendía por no haber empleado esta clase de tejido para mi defensa; pero yo no lo veía necesario.
    Ya dueño de suficiente cantidad de alimentos para todos los huéspedes que esperábamos di mi consentimiento al español para que cruzara el mar y fuese en busca de sus compañeros. Le hice estrictas recomendaciones, mandándole que no trajera consigo ningún hombre que previamente no jurara, en presencia suya y del anciano salvaje, no ofender en modo alguno, luchar o atacar a la persona que encontrara en la isla y que era quien enviaba en busca de ellos; que prometieran con el mismo juramento apoyar y defenderme contra toda tentativa hostil, así como someterse entera y absolutamente a mis órdenes. Todo aquello debía ser consignado por escrito y firmado de puño y letra. Cómo se haría eso, cuando era de imaginarse que carecían en absoluto de pluma y tinta, es cosa que a ninguno se nos ocurrió.
    Provistos de dichas instrucciones, el español y el anciano padre de Viernes partieron en una de las canoas con las cuales arribaran o, mejor dicho, fueron traídos prisioneros por los salvajes.
    A cada uno le entregué un mosquete y ocho cargas de pólvora y balas, encargándoles que las economizaran en lo posible y que sólo hicieran uso de ellas en ocasiones urgentes.
    Todo esto me llenaba de contento, pues eran las primeras medidas tomadas por mí en procura de mi libertad después de veintisiete años y algunos días de mi permanencia en la isla. Di a los viajeros suficiente cantidad de pan y pasas para una travesía de varios días, así como para alimentar a sus compatriotas una semana. Luego de desearles buen viaje los dejé partir, conviniendo antes una señal por la cual pudiera reconocerlos a su vuelta y desde muy lejos.
    Partieron con excelente viento, un día de plenilunio que correspondía, según mi cuenta, al mes de octubre. Mi calendario no era exacto, ya que después de haber perdido un día al comienzo nunca pude encontrar el error; ni siquiera estaba ya seguro del número de años, aunque cuando más tarde pude cotejar mis cálculos con la realidad vi que los años coincidían exactamente.
    No habían pasado ocho días de la partida de la canoa cuando se produjo un extraño e imprevisto acontecimiento que acaso no tenga equivalente en la historia. Dormía yo una mañana en mi tienda cuando Viernes llegó corriendo y gritando:
    — ¡Amo, amo, ellos venir, ellos venir!
    Sin mirar el peligro salté del lecho y tan pronto me hube vestido, atravesé el soto que, dicho sea de paso, era ya entonces un espeso bosque. Sin atender al peligro, repito, corrí desprovisto de armas, lo que nunca hacía, pero mi sorpresa fue grande cuando, al mirar el océano, vi un bote a legua y media de distancia que navegaba en procura de la costa, con una vela de las llamadas «espalda de carnero» y el viento a favor. Observé de inmediato que el bote no venía del lado más abierto de la costa sino de la parte sur de la isla.
    Llamé entonces a Viernes ordenándole que se mantuviera a cubierto, pues aquellas no eran las gentes que esperábamos y aún no podíamos saber si se trataba de amigos o enemigos. Apenas había alcanzado a poner mi pie en la cumbre cuando mis ojos vieron con toda claridad un navío fondeado a unas dos leguas y media de mi apostadero, hacia el S-SE, pero no a más de legua y media de la costa. Me bastó verlo para descubrir que era un barco inglés y el bote una lancha de igual procedencia.
    Es imposible expresar la confusión que experimenté. La alegría que me invadió al ver un barco que por todas las señales era tripulado por compatriotas, es decir, por amigos, no es de las que pueden ser descritas. Con todo, algunas dudas me asaltaban sin que me fuera posible comprender cuál era su motivo, y me mantuve oculto y en guardia. Ante todo se me ocurrió preguntarme qué clase de comercio podía traer a un barco inglés por estas regiones del mundo, ya que la isla estaba alejada de todo lugar donde la marina británica tuviese intercambio alguno. Por otra parte, no había venido arrastrado por una tormenta, y si verdaderamente se trataba de ingleses acaso no arribaran con buenas intenciones a esta tierra, de manera que era preferible continuar como hasta ahora y no caer en manos de probables ladrones y asesinos.
    Llevaba poco tiempo en la colina cuando vi acercarse la chalupa a tierra, costeándola en busca de alguna ensenada que facilitara el desembarco. Como no se alejaron mucho no les fue posible descubrir la pequeña caleta donde yo había llevado mis primeras balsas, sino que arrastraron la chalupa sobre la arena, a una media milla de mi puesto. Esto me alegró, porque de haber seguido la costa habrían terminado por arribar prácticamente a la puerta de mi casa, como podría decirse. Sin duda hubiesen descubierto el castillo y acaso saqueado todos mis bienes.
    Cuando desembarcaron tuve la satisfacción de reconocer en ellos a ingleses, por lo menos la mayor parte, aunque uno o dos me parecieron holandeses, en lo que estaba equivocado. Eran once en total, de los cuales tres se veían desarmados y al parecer atados, porque cuando los cinco o seis primeros pisaron tierra los hicieron salir como si se tratara de prisioneros. De los tres, uno parecía verdaderamente desesperado y mostraba tales señales de aflicción y de angustia que llegaba por momentos a la extravagancia; los otros dos, aunque a veces alzaban las manos al cielo como si estuvieran profundamente doloridos, se mostraban mucho más tranquilos que el primero.
    Aquel espectáculo aumentó mi confusión, sin que me fuera posible adivinar lo que iba a ocurrir. Entonces vino Viernes a hablarme en su chapurreado lenguaje.
    — ¡Oh, amo! ¡Ved hombres ingleses comer prisioneros igual salvajes comer hombres!
    — ¿Es que piensas, Viernes —repliqué yo—, que van a devorarlos?
    —Sí —insistió él—, ellos comer hombres.
    —Te equivocas —le dije—. Temo mucho que los asesinen, pero ya verás que no los comen.
    Entretanto seguía en la más grande ignorancia sobre lo que verdaderamente acontecía en la costa, y temblaba de horror a la idea de que aquellos tres prisioneros fueran asesinados de un momento a otro. Incluso vi a uno de los malvados alzar un gran machete o una espada para matar a los desdichados; me pareció que alguno iba a caer a cada instante, y mi sangre se heló en las venas.
    ¡Cuánto hubiera dado por tener ahora conmigo al español y al padre de Viernes! Ansiaba encontrar un medio de ponerme a tiro de aquellos individuos sin ser descubierto, a fin de rescatar a los tres infelices; había observado que no traían armas de fuego, pero pronto vi que las cosas cambiaban. Después de la insolente conducta del marinero hacia sus víctimas, todos ellos se dispersaron por la costa, como queriendo reconocer la tierra. Los tres prisioneros quedaron igualmente en libertad, pero se dejaron caer en tierra pensativos y con todo el aspecto de la desesperación más profunda.
    Esto me recordó el momento de mi llegada a la isla, cuando principié a mirar en torno mío sintiéndome perdido y sin esperanzas; me acordaba de la aprensión y el miedo con que había reconocido las inmediaciones, y cómo pasé la primera noche en un árbol por miedo a que me devoraran animales salvajes.
    Así como entonces yo ignoraba el socorro que la Providencia iba a enviarme al arrastrar el buque cerca de la costa y concederme extraer de él todo lo que me permitió alimentarme y vivir en adelante, así también aquellos pobres hombres angustiados ignoraban lo cerca de ellos que estaba la salvación y cómo en realidad podían escapar al peligro en el mismo instante en que se imaginaban abandonados a la muerte.
    Los tripulantes habían arribado a la costa justamente en el momento en que culminaba la pleamar, y en el tiempo que emplearon con los prisioneros y en recorrer más tarde las inmediaciones descuidaron tanto la chalupa que al producirse el reflujo quedó varada en tierra.
    Había, sin embargo, dos hombres a bordo, pero como sin duda habían bebido demasiado aguardiente estaban dormidos y ajenos a lo que pasaba. Uno de ellos despertó de improviso, y viendo que sus fuerzas no bastaban para empujar la chalupa al agua llamó a los demás que andaban sin rumbo fijo, y pronto se reunieron para moverla. No obstante, la tarea era superior a sus fuerzas, pues se trataba de una embarcación grande y la playa, en esa parte, tenía una arena suave y cenagosa; siendo casi arena movediza.
    Viéndose en tal situación, y como verdaderos marinos —que tal vez de todos los hombres sean los menos previsores— abandonaron la tarea y se pusieron otra vez a vagabundear. Oí que uno de ellos gritaba a otro que todavía permanecía junto a la chalupa:
    — ¡Eh, Jack, déjala quieta! ¡Ya flotará con la marea! Me bastó escuchar eso para comprobar cuál era su nacionalidad.
    Durante todo este tiempo me había mantenido muy oculto, sin animarme a salir del castillo más allá de mi puesto de observación en la cumbre de la colina, y sintiéndome harto satisfecho por las sólidas fortificaciones.
    Me dispuse entretanto para una posible batalla, aunque con mayor cuidado, pues tenía que habérmelas con otra clase de enemigos. Viernes era ya entonces un excelente tirador, y le ordené que se equipara convenientemente. Tomé dos escopetas para mí y puse tres mosquetes en sus manos. Por cierto que mi aspecto debía ser impresionante: tenía mi formidable chaqueta de piel de cabra y el gran gorro ya mencionado, una espada desnuda en la cintura, dos pistolas en el cinturón y una escopeta sobre cada hombro.
    Mi intención era no hacer ningún movimiento hasta que oscureciera; pero a eso de las dos de la tarde, a la hora de mayor calor, descubrí que todos se habían internado en los bosques y probablemente dormían tirados en el suelo. Los tres infelices prisioneros, demasiado ansiosos y apenados para encontrar descanso alguno, habían buscado la sombra de un gran árbol que se alzaba a un cuarto de milla de mi apostadero y, según me pareció, lejos de las miradas de los otros marinos.
    Resolví entonces mostrarme a ellos y averiguar qué les pasaba. Con la apariencia ya descrita me encaminé hacia el árbol seguido a cierta distancia por Viernes, a quien sus armas también hacían formidable, pero que no tenía el aspecto fantasmagórico de mi persona. Acercándome todo lo posible sin ser descubierto, les dirigí de pronto la palabra en español:
    — ¿Quiénes sois, caballeros?
    Se levantaron al oírme, pero su espanto creció diez veces al verme. No solamente no me contestaron sino que advertí su intención de escapar a la carrera.
    —Caballeros —les dije entonces en inglés—. No os espantéis de mí; acaso sea para vosotros el amigo que sin duda no esperabais.
    —Ese amigo —me contestó entonces uno de ellos gravemente, a tiempo que se quitaba el sombrero— debe haber sido enviado por el Cielo, porque en verdad nuestra situación está por encima del auxilio humano.
    —Todo auxilio viene del Cielo, señor —repliqué—. Os ruego que expliquéis a un extraño lo que os ocurre a fin de que pueda intentar ayudaros, pues me dais la impresión de hallaros en un gran apuro. Os he visto desembarcar, y mientras observaba que dirigíais súplicas a aquellos malvados que os han traído, vi a uno de ellos alzar su espada como para mataros.
    El desdichado, que me escuchaba con lágrimas en los ojos, se puso a temblar como alguien que no vuelve de su sorpresa.
    — ¿Estoy hablando con Dios o con un hombre? —dijo—. ¿Sois un ser humano o un ángel?
    —Desechad todo cuidado, caballero —le contesté—. Si Dios os hubiera enviado un ángel para salvaros, sin duda estaría mejor vestido que yo y con un armamento muy distinto. Dejad vuestros temores, soy inglés como vosotros y dispuesto a ayudaros según veis. Sólo dispongo de un criado, pero tengo armas y municiones. Decidme francamente: ¿puedo seros útil? ¿Qué os pasa?
    —Lo que nos pasa, caballero —me replicó entonces—, es demasiado largo de contar ahora que nuestros asesinos andan cerca: pero en resumen os diré que soy el capitán de aquel navío, mis hombres se han amotinado y apenas han podido contener sus deseos de asesinarme. Por fin han resuelto abandonarme en esta isla desolada junto con estos dos compañeros, uno mi piloto y el otro un pasajero. Esperábamos morir de hambre, creyendo que el lugar estaba totalmente deshabitado e incapaces de abrigar la menor esperanza al respecto.
    — ¿Dónde están esos miserables enemigos vuestros? —pregunté—. ¿Sabéis hacia dónde han ido?
    —Duermen allí, señor —contestó señalando un bosquecilio cercano—. Mi corazón tiembla de miedo al pensar que acaso nos han visto y os han oído hablar, porque en ese caso seguramente nos asesinarán a todos. — ¿Tienen armas de fuego?
    Me contestó que había dos piezas, una de las cuales había quedado en la chalupa.
    —Muy bien —repuse entonces—. Dejad el resto en mis manos. Ya veo que esos hombres duermen, y sería cosa fácil matarlos, salvo que prefiráis tomarlos prisioneros.
    El capitán me dijo entonces que entre ellos había dos desalmados a los cuales era imposible tratar con piedad, pero que eliminados ellos el resto volvería tal vez a su deber. A esa distancia le era imposible describirme su aspecto, pero se manifestó dispuesto a obedecer mis órdenes en todo lo que le mandase.
    —Entonces —dije— alejémonos en primer lugar de las cercanías para evitar ser vistos u oídos, y luego deliberaremos.
    Me siguieron de inmediato, hasta que los bosques nos ocultaron.
    —Ahora bien, caballero —dije al capitán—. Si os ayudo a recobrar vuestra libertad, ¿estáis dispuesto a cumplir dos condiciones que os fijaré?
    Se anticipó a mi propuesta diciéndome que tanto él como su barco, si era recobrado, quedarían totalmente a mis órdenes a partir de entonces, y que en caso de que el buque se perdiese, lo mismo permanecería a mi lado en cualquier sitio del mundo donde yo lo dispusiera; los otros dos hombres afirmaron lo mismo.
    —Muy bien —dije—. Mis condiciones son las siguientes. En primer lugar, que mientras estéis conmigo en esta isla no pretendáis aquí la menor autoridad; si pongo armas en vuestras manos, ellas me serán devueltas cuando así lo disponga y nada se cometerá en mi territorio que resulte en perjuicio mío. Segundo, que si el barco es recobrado, me llevaréis a mí y a mi criado a Inglaterra sin gastos de pasaje.
    De inmediato me dio todas las seguridades que la buena fe y el ingenio hayan podido inventar, asegurándome que me debería la vida y que esa deuda sería reconocida eternamente por él en cuanta ocasión se presentara.
    —Entonces —agregué— aquí hay tres mosquetes para vosotros, con pólvora y balas; decidme ahora qué consideráis conveniente hacer.
    El capitán siguió manifestándome calurosamente su gratitud, pero en cuanto a la conducta a seguir quiso que yo los guiara en un todo.
    Manifesté entonces que la tentativa me parecía peligrosa, pero a mi parecer lo más sensato era hacer fuego de inmediato sobre aquellos hombres en el sitio en que se encontraban; si alguno se salvaba y ofrecía rendirse, le perdonaríamos la vida, encomendándonos a la Providencia para que nuestros disparos fuesen certeros.
    Contestóme con mucha moderación que le repugnaba matar a aquellos hombres y que hubiera preferido evitarlo, pero que los dos incorregibles villanos habían sido los promotores del motín y que si llegaban a escaparse estaríamos perdidos, ya que eran capaces de regresar del barco en compañía de los demás tripulantes y no cejar hasta encontrarnos.
    —Muy bien, entonces —dije yo—; ya veis que la necesidad sanciona mi consejo, y que no hay otro modo de salvar nuestras vidas.
    Pese a todo, y viendo cuánta repugnancia le causaba derramar sangre humana, le di permiso para que procediera junto con sus compañeros del modo que creyese mejor.
    —Bueno —dije—, dejadlos entonces escapar; la Providencia parece haberlos despertado a tiempo para salvarse. Ahora, si el resto huye, la culpa será vuestra.
    Animado con mis palabras tomó el mosquete que le había dado, se puso una pistola en el cinto y sus camaradas lo imitaron, cada uno con un arma en la mano. Al avanzar, estos últimos hicieron algún ruido y uno de los marineros ya despiertos se dio vuelta y al verlos en tal actitud gritó un alerta a los demás. Pero ya era tarde porque en el mismo instante los dos hombres dispararon sobre ellos, mientras el capitán reservaba prudentemente su carga. Tan bien habían apuntado que uno de los marineros cayó instantáneamente muerto y el otro, muy mal herido, apenas podía incorporarse en el suelo pidiendo auxilio a los demás. El capitán se le acercó de inmediato, diciéndole que ya era tarde para pedir auxilio, y que pidiera perdón a Dios por su villanía; dicho eso le descargó en la cabeza la culata del mosquete, dejándolo exánime. De los restantes, solamente uno estaba ligeramente herido, pero como yo llegué en ese momento y comprendieron que no estaban en condiciones de resistir, pidieron perdón al punto. El capitán expresó que les salvaría la vida si le daban absoluta seguridad de su arrepentimiento por la abominable traición cometida y si juraban serle fieles, ayudarlo a recobrar el barco y tripularlo hasta Jamaica, que era su procedencia. Todos ellos hicieron abundantes protestas de sinceridad y él parecía dispuesto a creerles y salvar así sus vidas, cosa a la que yo no me opuse aunque le exigí que tuviera a esos hombres atados de pies y manos mientras permanecieran en la isla.
    En tanto que esto ocurría, mandé a la costa a Viernes con el pilo para que aseguraran la chalupa, ordenándoles que sacaran los remos y la vela; mientras se ocupaban en ello, tres marineros que habían andado vagabundeando por la isla, separados para suerte suya de los otros tripulantes, se acercaron a nosotros atraídos por los disparos. Pero viendo al capitán, un rato antes su prisionero y ahora otra vez el amo, se sometieron de inmediato y nuestra victoria fue completa.





    15. ROBINSON LOGRA SU LIBERTAD




    Aún nos quedaba al capitán y a mí contarnos mutuamente nuestras aventuras. Principié narrándole toda mi historia, que escuchó con una atención vecina al asombro, especialmente las circunstancias maravillosas por las cuales había llegado a proveerme de armas y vituallas. En verdad que siendo mi vida una serie de episodios extraordinarios lo impresionó profundamente. Luego, cuando reflexionó sobre sí mismo y cómo parecía que yo hubiese sido preservado en aquella isla para salvarle más tarde la vida, lágrimas brotaron de sus ojos y no pudo pronunciar una sola palabra.
    Luego de concluir mi narración llevé a los tres hombres a mi morada, haciéndolos entrar por el mismo sitio que empleaba para salir, es decir, la plataforma en lo alto; allí les di de comer los alimentos que había podido llevar mostrando todas las invenciones que había podido llevar a cabo en mi larguísima permanencia en la isla.
    Todo cuanto les mostré, todo cuanto les dije, los dejaba pasmados; pero el capitán admiró por sobre todo mi fortificación, en especial la forma en que había ocultado mi castillo con un soto que, plantado veinte años atrás y formado por árboles que crecen aquí mucho más pronto que en Inglaterra, era ahora un bosquecillo tan espeso que no había manera de atravesarlo salvo por el angosto pasaje trazado por mí. Dije al capitán que aquél era mi castillo y mi residencia, pero que al igual que muchos príncipes poseía una finca en el campo donde me gustaba pasar temporadas y que le mostraría en otra ocasión, pues de momento nuestro problema era considerar el modo de hacernos dueños del navío.
    Convino en ello, pero agregó que no tenía la menor idea de cómo proceder, ya que a bordo quedaban todavía veintiséis hombres que, entregados a tan perversa conspiración y sabiendo que la ley la penaría en sus vidas, procederían arrastrados por la desesperación y dejándose llevar a cualquier extremo, seguros de que si eran reducidos su destino sería la horca apenas llegaran a Inglaterra o a cualquier colonia inglesa. Era por lo tanto imposible pretender atacarlos siendo nosotros tan pocos.
    Medité largo tiempo lo que me dijo, encontrándolo muy atinado; urgía sin embargo encontrar algún camino ya fuera tendiendo una celada a los de a bordo o impidiéndoles desembarcar a toda costa para evitar ser masacrados.
    Entonces pensé que aquellos hombres, asombrados por el retraso de sus camaradas de la chalupa, vendrían a tierra tripulando la otra chalupa del barco, sin duda armados y en gran superioridad de número. El capitán encontró esto muy probable.
    Opiné que nuestra primera medida debía consistir en inutilizar la chalupa que quedara varada, sacando de ella todos los implementos y tornándola inútil para navegar. Nos apresuramos a bajar a la playa y retiramos las armas que habían quedado en la chalupa así como otras cosas que hallamos en ella tales como una botella de aguardiente, otra de ron, algunas galletas, un frasco de pólvora y un gran pedazo de azúcar envuelto en tela, que debía pesar cinco o seis libras.
    Todo esto fue muy bien recibido por mí, especialmente el aguardiente y el azúcar, de los cuales carecía desde muchos años atrás.
    Cuando hubimos llevado todo esto a tierra (ya he dicho nup anteriormente habían sido retirados el mástil, vela y timón de la chalupa) practicamos un gran agujero en el fondo a fin de que si los marineros venían en número suficiente para dominarnos no pudieran sin embargo llevarse la embarcación.
    En verdad apenas pasó por mi pensamiento que pudiéramos apoderarnos del barco, pero mi proyecto era que si aquellos individuos se marchaban sin llevarse la chalupa la pondríamos nuevamente en condiciones para navegar hasta las islas de sotavento, donde podríamos recoger de paso a nuestros amigos españoles, ya que en ningún momento los había olvidado.
    Preparábamos entretanto los planes. Con todas nuestras fuerzas movimos la chalupa para alejarla lo más posible del mar a fin de que la marea alta no pudiera ponerla a flote; luego practicamos un agujero en el fondo del casco, lo bastante grande para que no fuese fácil repararlo, y nos habíamos sentado cerca meditando qué debíamos hacer a continuación cuando oímos un cañonazo disparado de a bordo, mientras con la bandera hacían señales de que la chalupa debía retornar al barco.
    Como ninguna respuesta llegó de la isla, repitieron varias veces las señales y los cañonazos.
    Por fin, cuando se convencieron de que ni una cosa ni otra daba resultado alguno, los vimos con ayuda de mi catalejo arriar otra chalupa y embarcarse rumbo a tierra; a medida que se aproximaban pudimos distinguir que había a bordo no menos de diez hombres, y que venían provistos de armas de fuego.
    Como el barco estaba fondeado a unas dos leguas de la costa, veíamos muy bien la chalupa mientras se acercaba, incluso el rostro y aspecto de sus tripulantes. La marea los arrastraba un poco hacia el este del sitio donde varara la otra chalupa, de modo que remaban tratando de remontar la cosa hasta dar con el fondeadero de la primera embarcación.
    Gracias a eso, repito, tuvimos oportunidad de distinguirlos muy bien, y el capitán por su parte conocía al dedillo el aspecto y carácter de cada uno de los tripulantes de la chalupa. Me dijo que había entre ellos tres hombres honestos que indudablemente habían sido arrastrados al motín por la fuerza o el miedo. En cuanto al contramaestre, que parecía ser jefe absoluto de la conspiración, y los demás marineros, eran tan malvados como el resto de la tripulación y sin duda harían desesperados esfuerzos para dominarnos, por lo cual el capitán tenía fundados temores de que fuesen más fuertes que nosotros.
    Sonreí al escucharlo, y le dije que hombres en nuestras circunstancias debían sentirse más allá del miedo; cualquier situación imaginable sería mejor que aquella en la que estábamos colocados en ese momento, y sus posibles consecuencias, fuesen la vida o la muerte, equivalían de todas maneras a una liberación. Le pedí que considerara mi propia vida, y si una posibilidad de rescate no merecía correr el riesgo.
    — ¿Y qué se ha hecho, señor —agregué—, esa creencia que teníais de que mi vida había sido preservada deliberadamente para salvar alguna vez la vuestra? ¿No estabais tan animado hace un rato? Por lo que a mí respecta, sólo veo un inconveniente en la forma en que se presentan los sucesos.
    — ¿Cuál es él? —preguntó el capitán. —Pues vuestra afirmación de que en esa chalupa hay tres o cuatro hombres honestos cuya vida debería respetarse; si toda esa tripulación hubiera estado constituida por malvados, yo habría supuesto que la Providencia Divina la había escogido para ponerla en vuestras manos. Porque estad seguro de que cada hombre que desembarque en esta costa nos pertenece, y vivirá o morirá según se comporte.
    Como le dije estas cosas con voz animosa y rostro decidido, le devolví grandemente el coraje y proseguimos nuestra tarea.
    Por cierto que apenas habíamos advertido la partida de la segunda chalupa rumbo a la costa convinimos en separar nuestros prisioneros y asegurarlos convenientemente.
    Dos de ellos, a quienes el capitán tenía menos confianza que al resto, fueron llevados por Viernes y uno de los compañeros del capitán a la caverna que estaba bastante lejos y oculta para ser descubierta, y tan pérdida entre los bosques que de nada les habría valido a aquellos individuos escaparse de su prisión.
    Allí los dejaron atados, pero con provisiones suficientes y la promesa de que si se quedaban quietos recobrarían su libertad en un día o dos, pero que si pretendían escaparse serían muertos sin lástima. Aseguraron reiteradamente que soportarían con paciencia su prisión y se mostraron muy agradecidos de que les dejáramos provisiones y luz, ya que Viernes les dio algunas de las velas hechas por nosotros a fin de que lo pasaran mejor. Además, el marinero se quedó de centinela a la entrada de la cueva, para mayor seguridad.
    Los restantes prisioneros recibieron mejor trato. Dos de ellos, sin embargo, siguieron atados, porque el capitán no sentía plena confianza a su respecto, pero los otros fueron puestos a mis órdenes por recomendación de su amo y luego de haber jurado solemnemente vivir y morir a nuestro lado. Con ellos, más los tres rescatados por mí, éramos siete hombres bien armados y no me cabía duda de que podríamos luchar con los diez sublevados que venían en la chalupa, máxime que entre ellos el capitán había reconocido a tres o cuatro hombres honestos.
    Tan pronto como arribaron a la costa se apresuraron a varar la chalupa en la playa y desembarcaron para remolcarla fuera del agua, lo que me alegró mucho porque había temido que la dejaran anclada a cierta distancia de la costa, a cargo de algunos hombres, cosa que nos hubiera impedido apoderarnos de la embarcación.
    Ya en tierra, lo primero que hicieron fue correr a la otra chalupa y es de imaginarse la profunda sorpresa que tuvieron al encontrarla desmantelada y con un enorme agujero en el fondo.
    Luego de discutir un rato en torno a la chalupa se pusieron a dar grandes gritos, repitiéndolos con todas sus fuerzas para llamar la atención de sus perdidos compañeros. Como no obtuvieran respuesta alguna se reunieron en círculo y dispararon al aire sus armas, cosa que oímos perfectamente y que los bosques repitieron como un eco. Pero tampoco les sirvió la descarga, ya que los prisioneros en la caverna no podían escucharla, y aquellos en nuestro poder, aunque la oyeron, no se hubieran atrevido a contestarla.
    Los de la chalupa se quedaron tan asombrados ante su fracaso que, como nos lo dijeron más tarde, resolvieron embarcarse inmediatamente y volver lo antes posible al navío, seguros de que así como la chalupa estaba averiada sus tripulantes habían sido asesinados al desembarcar. De acuerdo con eso, botaron su lancha al agua y se embarcaron al punto.
    El capitán se mostró entonces sorprendido y luego desconcertado, seguro de que apenas llegaran a bordo se harían a la mar dejando abandonados a sus camaradas, con lo cual el navío se perdería para él, que tanto había confiado en rescatarlo.
    Pero pronto tuvo un nuevo motivo para aterrarse. Apenas habían botado aquellos hombres la chalupa cuando lo vimos que cambiaban de idea y retornaban a la costa, con la diferencia de que ahora dejaron tres hombres al cuidado de la chalupa mientras los restantes se internaban en procura de sus compañeros.
    Aquello nos causó una gran decepción, porque ignorábamos la mejor conducta a seguir; apoderarnos de los siete marineros en tierra no nos daba ninguna ventaja si dejábamos escapar la chalupa, ya que sus tripulantes irían de inmediato a bordo, induciendo a los otros a hacerse a la vela, y el rescate del buque se tornaría así imposible. Advertimos de inmediato que lo más prudente era quedar a la espera, a fin de que el curso de los acontecimientos nos dictara el camino a seguir. Los siete marinos bajaron a tierra, mientras los tres restantes mantenían la chalupa a buena distancia de la costa, andándola para quedarse esperando.
    Vimos entonces que tratar de apoderarnos de la chalupa era una empresa imposible.
    Los que habían desembarcado tuvieron buen cuidado de mantenerse unidos, encaminándose hacia las alturas de la pequeña colina bajo cuya ladera estaba mi morada; aunque no podían divisarnos, advertíamos claramente sus movimientos. Nos hubiera agradado verlos acercarse de modo de poder disparar sobre ellos, o bien que se alejaran lo bastante para poder salir de nuestro escondite.
    Cuando llegaron a lo alto de la colina desde donde tenían un amplio panorama de los valles y los bosques que se extendían hacia el noreste, donde la isla era más baja, se pusieron a gritar y hacer señales hasta que estuvieron exhaustos. No pareciendo dispuestos a alejarse mucho más de la costa, así como a separarse entre ellos, terminaron por reunirse bajo un árbol para discutir la situación. De haber decidido dormir allí, como lo hiciera antes el otro grupo, la tarea hubiese sido muy simple para nosotros, pero estaban demasiado inquietos y llenos de aprensiones para aventurarse a dormir, pese a que parecían incapaces de determinar qué clase de peligro los acechaba.
    El capitán me hizo entonces una juiciosa proposición mientras los marineros continuaban deliberando; suponía que iban a resolverse a hacer una nueva descarga cerrada para llamar la atención de sus camaradas, oportunidad que podíamos aprovechar precipitándonos sobre ellos en el preciso momento en que sus armas estuviesen descargadas, con lo cual los obligaríamos a rendirse sin necesidad de ningún derramamiento de sangre.
    Me pareció un excelente y atinado plan, ya que estábamos bastante cerca de ellos para sorprenderlos antes de que hubieran tenido el tiempo de cargar otra vez sus armas.
    Sin embargo la descarga no se produjo, y nos quedamos largo tiempo allí, sin resolvernos a emprender otra cosa. Por fin opiné que nada podría hacerse hasta llegada la noche, y que si entonces los hombres no habían vuelto aún a la chalupa tal vez encontraríamos un medio de situarnos entre ellos y la costa, así como una estratagema para convencer a los de la chalupa que no se acercaran a tierra.
    Impacientes en extremo, permanecimos sin embargo a la espera, pero nos invadió el temor al ver que, luego de largas consultas y deliberaciones, los hombres se levantaban y se ponían en marcha hacia la costa. Como si la aprensión del lugar fuese demasiado para su valor, habían resuelto al parecer retornar lo antes posible al navío, dar a sus compañeros por perdidos y reanudar de inmediato el viaje.
    Tan pronto los vi encaminarse de nuevo hacia la playa comprendí que cesaban la búsqueda, y el capitán, cuando le comuniqué mis temores, pareció desmayarse de angustia.
    Pero entonces se me ocurrió una estratagema para obligarlos a retornar, que me pareció perfectamente realizable. Ordené en consecuencia a Viernes y al piloto que se encaminaran hacia la pequeña ensenada del oeste, en la zona donde los caníbales habían desembarcado cuando Viernes huyó de ellos, y tan pronto llegaran a un altozano, a una media milla de distancia, se pusieran a dar grandes gritos para llamar la atención de los marineros. Les dije que apenas oyesen una respuesta volvieran a gritar para conseguir que fuesen hacia allí, y entonces, internándose cada vez más en la isla y si era posible entre los bosques, fueran describiendo un círculo que los trajera hasta nosotros por un camino que les señalé.
    Los marineros estaban ya embarcándose cuando Viernes y el piloto dejaron oír sus gritos. Tan pronto oyeron el llamamiento lo contestaron a coro y echaron a correr hacia el oeste en dirección de donde venían las voces. A mitad de camino tropezaron con la ensenada que, estando la marea alta, no podían vadear, por lo cual hicieron señales a los de la chalupa que vinieran a pasarlos, que era lo que yo estaba esperando.
    Apenas hubieron pasado al otro lado, advertí que la chalupa se había internado bastante en la ensenada que formaba una especie de seguro puerto en el interior de la isla, por lo cual los marineros se llevaron consigo a uno de los tres que cuidaban la embarcación, quedando los otros a bordo, luego de asegurar la chalupa al tronco de un árbol que crecía en la misma orilla.
    Esto era lo que yo deseaba, y dejando a Viernes y al piloto que prosiguieran su labor avancé con mis compañeros cruzando la ensenada a cubierto de los dos marineros desprevenidos.
    Antes de que pudieran reaccionar caímos sobre ellos; uno yacía descansando en tierra y el otro permanecía en la chalupa. El primero, que estaba dormitando, trató de incorporarse, pero el capitán, que iba delante, lo alcanzó de un culatazo derribándolo, y luego ordenó al otro que se rindiera o era hombre muerto.
    Pocos argumentos fueron necesarios para decidir a un individuo solo contra cinco bien armados que ya habían dado cuenta de su compañero; además, era uno de aquellos que no habían participado voluntariamente en el motín, y por lo tanto no sólo se rindió de inmediato sino que estuvo luego de nuestra parte con toda buena fe.
    Entretanto, tan bien habían cumplido Viernes y el piloto el papel que debían desempeñar, que con sus gritos y respuestas fueron llevando al resto de los amotinados de colina en colina y de bosque en bosque, hasta que quedaron tan agotados que de ninguna manera hubiesen podido volver a la chalupa antes de que anocheciera.
    Viernes y el piloto, ya de regreso entre nosotros, se mostraban también fatigadísimos.
    Sólo nos quedaba esperar ahora su regreso en la oscuridad y caer sobre ellos con la seguridad de dominarlos.
    Varias horas después de la vuelta de Viernes el grupo de los amotinados pudo alcanzar el sitio donde dejara la chalupa. Mucho antes de que llegaran a la ensenada oímos a los que venían delante dar gritos a los retrasados para que se apresuraran, y escuchábamos las voces de los otros quejándose de lo cansados y rendidos que estaban y de que no podían andar más ligero; todo lo cual nos llenó de alegría. Por fin llegaron a la chalupa, pero es imposible describir la confusión que experimentaron al encontrar la embarcación profundamente internada en la caleta, en seco por el reflujo y sus dos tripulantes desaparecidos. Oímos que se llamaban los unos a los otros con acento desgarrador, diciéndose que habían desembarcado en una isla encantada. Estaban seguros de que si había habitantes en ella iban a presentarse de improviso para asesinarlos a todos, y si solamente había espíritus y demonios en torno, los arrebatarían para devorarlos.
    Volvieron a gritar a coro, llamando muchas veces por sus nombres a los dos camaradas desaparecidos, pero no recibieron respuesta. A la débil luz alcanzamos a verlos, corriendo como enloquecidos y retorciéndose las manos en su desesperación; algunos entraban a descansar en la chalupa, volvían luego a salir como si no pudieran hallar reposo, y esto se repetía constantemente.
    Mis hombres hubieran querido que los dejara caer en la oscuridad sobre sus enemigos, pero yo prefería emplear algún otro recurso que evitara tener que matar a tantos hombres; en especial quería proteger la vida de mis compañeros, sabiendo de sobra que los otros estaban muy bien armados. Me resolví a esperar, confiando en que terminarían por separarse; y entretanto, para estar más seguro de ellos, decidí aproximar aún más nuestras líneas, por lo que mandé al capitán y a Viernes que se arrastraran sobre pies y manos hasta ponerse lo más cerca posible sin ser descubiertos, antes de abrir el fuego.
    No llevaban mucho tiempo en su nueva posición cuando el contramaestre, que había sido el principal promotor del motín y que ahora se mostraba el más cobarde y desesperado de todos, vino en dirección a ellos seguido de otros dos de los suyos.
    El capitán estaba tan excitado al comprender que tenía a aquel miserable villano en su poder, que apenas conservó paciencia para esperar que se acercara lo bastante, ya que hasta entonces sólo habían oído su voz. Pero cuando estuvo casi junto a ellos, el capitán y Viernes se enderezaron a un tiempo e hicieron fuego.
    El contramaestre quedó muerto instantáneamente, y de los otros dos uno recibió un balazo en el cuerpo y cayó junto a su jefe, aunque sólo murió una o dos horas más tarde; el otro huyó a toda carrera.
    Al oír los disparos avancé de inmediato con todo mi ejército, del cual era generalísimo, y que contaba con ocho hombres: Viernes —mi teniente general—, el capitán con sus dos compañeros y los tres amotinados que se plegaron a nosotros y a quienes habíamos dado armas.
    En las tinieblas avanzamos sobre ellos, de manera que no podían saber a cuántos ascendíamos. Ordené al marinero que capturáramos en la chalupa, y que ahora era de los nuestros, llamar a los amotinados por su nombre a fin de parlamentar con ellos y tratar de reducirlos sin trabarnos en batalla. Era de imaginarse que, en la situación en que se encontraban, no tardarían en rendirse voluntariamente.
    Con todas sus fuerzas, el marinero llamó a uno de sus compañeros:
    — ¡Tom Smith, Tom Smith!
    — ¿Quién me llama? —repuso Smith—. ¿Eres tú, Robinson?
    El otro, cuya voz había sido así reconocida, replicó de inmediato:
    —Sí, soy yo. En nombre de Dios, Tom Smith, arrojad vuestras armas y rendios o sois hombres muertos en un minuto.
    — ¿A quién tenemos que rendirnos? ¿Dónde están? —gritó Smith.
    —Aquí están. Es nuestro capitán y cincuenta hombres que os han estado acechando todo el tiempo. El contramaestre ha muerto, Will Frye está herido y yo soy prisionero. Rendios o estáis perdidos todos.
    —Si nos rendimos —preguntó Smith—, ¿nos darán cuartel?—Iré a preguntarlo, si prometéis entregaros —repuso Robinson.
    El capitán, que había escuchado el diálogo, se adelantó entonces.
    —Ya conoces mi voz, Smith —gritó—. Si entregáis de inmediato las armas y os sometéis, os garantizo la vida a todos menos a Will Atkins.
    — ¡Por Dios, capitán, dadme cuartel! —gritó entonces la voz de Will Atkins—. ¿Qué he hecho yo? ¿No han sido los otros tan culpables como yo?
    Nada de eso era cierto, porque este Will Atkins había sido el que primero se apoderó del capitán cuando se amotinaron, tratándolo bárbaramente, amarrándole las manos y vociferando toda suerte de injurias. Con todo, el capitán le gritó que debía rendirse a discreción y confiarse a la clemencia del gobernador, con lo cual se refería a mí, ya que todos me daban ese título.
    Momentos después los villanos habían rendido las armas y pedían por sus vidas. Envié entonces al hombre que había parlamentado con ellos y a otros dos para que los ataran sólidamente. Y luego, mi gran ejército de cincuenta hombres, que sumando los tres mencionados se elevaba a ocho soldados, avanzó hacia el enemigo y se apoderó de él y de su chalupa, quedándome yo con otro compañero fuera de su vista por razones de estado.
    Nuestra tarea inmediata consistía en reparar la chalupa y tratar de apoderarnos del navío. En cuanto al capitán, ahora que estaba en libertad para hablar con sus hombres, los apostrofó severamente sobre la villanía que habían cometido, haciéndoles ver la maldad de su designio y cómo sólo les hubiera servido para arrastrarlos finalmente a la peor de las miserias, cuando no a la horca.
    Todos ellos se mostraron muy arrepentidos y suplicaron se les perdonara la vida. A esto les replicó que no eran prisioneros suyos sino del comandante de la isla; y que aunque habían pensado al principio que esa tierra estaba deshabitada, Dios había dirigido sus pasos a un lugar poblado cuyo gobernador era inglés. Agregó que si le parecía bien podía colgarlos a todos sin vacilar, pero que como les había concedido cuartel suponía que el gobernador iba a enviarlos prisioneros a Inglaterra, donde serían juzgados por los tribunales, salvo Atkins, a quien el gobernador enviaba a decir por su intermedio que se preparase a morir, pues sería ahorcado por la mañana.
    Aunque todo esto era una invención, tuvo el efecto que el capitán deseaba. Atkins cayó de rodillas, suplicando al capitán que intercediera ante el gobernador para salvarle la vida; y todo el resto se unió a sus lamentaciones rogando encarecidamente que no los enviaran a Inglaterra.
    Se me ocurrió entonces que la hora de nuestra libertad había llegado y que sería cosa fácil lograr que aquellos individuos colaboraran con nosotros en apoderarnos del barco. Oculto como estaba a sus miradas, a fin de que no descubrieran qué clase de gobernador tenía la isla, llamé en alta voz al capitán. Como lo hacía desde buena distancia, uno de mis hombres tenía la orden de repetir el llamado y decir:
    —Capitán, el gobernador quiere veros.
    —Decid a Su Excelencia que voy de inmediato —replicó entonces el capitán.
    La escena resultó muy bien, y los prisioneros quedaron convencidos de que el gobernador andaba con sus cincuenta hombres por las inmediaciones.
    Cuando llegó a mi lado, expuse al capitán mi proyecto para apoderarnos del barco, el que le pareció excelente, y dispusimos llevarlo a ejecución a la siguiente mañana. En orden a que todo resultara sin tropiezos y con la mayor seguridad, dije a mi compañero que debíamos dividir a los prisioneros de modo que Atkins y otros dos entre los peores fueran enviados sólidamente sujetos a la caverna donde ya estaban los otros. Viernes y los dos compañeros del capitán se ocuparon de cumplir ese cometido.
    Fueron, pues, conducidos a la cueva que hacía de prisión, y que, por cierto, era un lugar espantoso para individuos en el estado en que se encontraban aquéllos. A los otros los conduje a mi enramada, de la que he hecho ya una descripción completa; como había allí empalizada y los hombres seguían atados, el sitio resultaba bastante seguro, máxime cuando la suerte de aquellos dependía de su conducta.
    Por la mañana envié al capitán a conferenciar con los prisioneros de la enramada, a fin de indagarlos y hacerme saber si le parecían dignos de confianza para acompañarnos en la expedición contra el buque. El capitán volvió a hablarles de la injuria que le habían hecho, de la situación en que se encontraban, y les dejó entrever que aunque el gobernador les concedía cuartel por el momento, apenas fueran llevados prisioneros a Inglaterra serían ahorcados con toda seguridad; pero agregó que si estaban dispuestos a unirse a nosotros para tratar de reconquistar el buque, acaso fuera posible lograr formalmente el perdón del gobernador.
    Es de imaginar con cuánta rapidez habrá sido aceptada semejante proposición por hombres que se encontraban en semejante alternativa. Cayeron de rodillas ante el capitán y le prometieron, con las demostraciones más sinceras, que le serían fieles hasta último momento; puesto que iban a deberle la vida, estaban dispuestos a acompañarlo a todas partes, y mientras vivieran lo considerarían como su padre. —Entonces —dijo el capitán— iré a decir al gobernador lo que acabo de oír y trataré por todos los medios de que acceda.
    Vino a mí con el relato de lo hablado y me participó su impresión de que aquellos individuos le serían fieles. Con todo, y para asegurarnos bien le dije que volviera a ellos y apartara a cinco hombres, que serían sus asistentes en la empresa, diciéndoles que el gobernador conservaría en su poder a los otros dos, así como a los tres prisioneros en la caverna, en calidad de rehenes para garantizar la fidelidad de aquellos cinco; y que si alguno mostraba la menor señal de traición, los rehenes serían ahorcados inmediatamente en la playa.
    Todo esto los impresionó, dándoles la seguridad de que el gobernador procedía con severidad. No les quedaba sin embargo otro recurso que aceptar aquellas condiciones, y a partir de ese instante los cinco rehenes, además del capitán, se empeñaron en persuadir a los otros para que cumplieran su deber al pie de la letra.
    Nuestras fuerzas estaban ahora listas para la expedición según el siguiente detalle:
    1.°) el capitán, su segundo y el pasajero;
    2.°) los dos prisioneros de la primera partida a los que, de acuerdo con la recomendación del capitán, habíamos dado libertad y confiado armas;
    3.°) los otros dos, que hasta entonces habíamos tenido en la enramada, atados, pero que ahora pusimos en libertad por pedido del capitán;
    4.°) los cinco recién libertados. Eran, pues, doce en total, aparte de los cinco que mantuvimos prisioneros en la caverna en calidad de rehenes.
    Pregunté al capitán si estaba dispuesto a aventurarse con aquel número en la empresa de abordar el navío; por lo que a mí y a Viernes respecta, pensé que no nos convenía movernos por el momento, teniendo siete hombres que cuidar en tierra, ya que bastante tarea suponía vigilarlos y darles suficiente alimento. Con respecto a los cinco de la caverna decidí mantenerlos atados, pero Viernes iba dos veces por día a llevarles vituallas; los otros fueron empleados en acarrear provisiones hasta cierta distancia de la cueva, donde iba Viernes a tomarlas.
    Me presenté entonces a los dos rehenes en compañía del capitán, quien les dijo que yo era la persona designada por el gobernador para vigilarlos, y que sus órdenes eran que no se apartaran un solo momento del sitio donde yo me encontrara; si lo hacían serían llevados de inmediato al castillo y encadenados. Naturalmente aquellos hombres, como no habían visto nunca al gobernador, me tomaron por su representante, y yo hablaba a cada momento de Su Excelencia, de la guarnición, el castillo y demás cosas parecidas. El capitán no tenía ahora otra dificultad que la de aparejar las dos chalupas, reparar la avería en una de ellas y tripularlas con sus hombres. Hizo a su pasajero capitán de una de las embarcaciones, y puso cuatro hombres a sus órdenes; en persona, y acompañado de su segundo y cinco hombres, se embargó en la otra; aprovecharon la mejor hora y navegaron en dirección al navío a eso de medianoche. Tan pronto estuvieron al alcance de la voz, el capitán ordenó al llamado Robinson que gritara a los del navío, diciéndoles que traían los hombres y la chalupa, pero que les había llevado muchísimo tiempo dar con ellos; con esas y otras   parecidas  charlas  debía  entretener   su  atención mientras los demás se acercaban al navío.
    Entonces, abordando el buque, el capitán y su segundo sorprendieron y derribaron a culatazos al segundo y al carpintero, siendo fielmente ayudados por los hombres que iban con ellos. Asegurando rápidamente a los restantes marineros que estaban en el puente y la popa, trancaron las escotillas para aislar a los que quedaban abajo. Entretanto la otra chalupa, desembarcando a sus tripulantes en los portaobenques del trinquete, aseguró la posesión del castillo de proa y la escotilla que daba a la cocina, donde fueron apresados otros tres hombres.
    Cumplido esto, y dueño del puente, el capitán ordenó al piloto que tomara por asalto la toldilla donde se encontraba el capitán sublevado; éste, despierto y alerta, se encontraba en compañía de dos hombres y un grumete armados con fusiles. Cuando el piloto forzó la puerta con una palanca, el nuevo capitán y sus hombres dispararon a quemarropa sobre los. atacantes, hiriendo al piloto de un tiro de mosquete que le atravesó el brazo, así como a dos de sus hombres, pero sin matar a ninguno.
    Mientras pedía auxilio a gritos entró el piloto, herido como estaba, en la toldilla y con su pistola atravesó la cabeza del capitán rebelde; la bala entró por la boca y salió detrás de una oreja haciéndolo caer sin tiempo de pronunciar una palabra.
    Al ver esto, los otros se rindieron de inmediato y el buque fue tomado sin que resultara necesario sacrificar más vidas.
    Tan pronto estuvo el barco en su poder el capitán ordenó que se dispararan siete cañonazos, señal convenida conmigo para hacerme saber el buen resultado de la empresa; es de imaginar la alegría con que los escuché, habiendo esperado novedades en la playa hasta las dos de la madrugada.
    Escuchada la señal, me dejé caer en la arena y rendido por las muchas fatigas de aquel día dormí profundamente hasta que el sonido de otro cañonazo me despertó sobresaltado. Mientras me incorporaba, oí a alguien gritando:
    — ¡Gobernador, gobernador!
    Reconocí inmediatamente la voz del capitán, y subiendo a la cumbre de la colina lo encontré; al verme señaló en dirección del navío, y viniendo a mí me estrechó en sus brazos mientras exclamaba:
    — ¡Amigo mío, mi salvador! ¡Ahí tenéis vuestro barco que os pertenece, así como nosotros, y todo lo que a bordo existe os pertenece también!
    Miré en dirección al mar y vi el navío a una media milla de la costa. Supe entonces que apenas dueños de la situación se habían apresurado a levar anclas y acercarse, aprovechando la calma que reinaba, hasta la boca de la pequeña ensenada. Estando alta la marea, el capitán había venido con la pinaza hasta el mismo sitio donde yo fondeara mis primeras balsas, y puede decirse que acababa de llegar a la puerta de mi casa.
    La emoción me embargó al extremo de que estuve a punto de desplomarme. Veía ahora con toda claridad la liberación al alcance de mis manos, ya todo resuelto y listo, un gran navío esperándome para llevarme al sitio donde me placiera más.
    En el primer momento me sentí incapaz de articular una sola palabra; y como el capitán me tenía abrazado, me aferré a él con fuerza porque de lo contrario hubiese caído al suelo.
    El advirtió mi emoción, y extrayendo una botella de su bolsillo me hizo beber un trago de cordial que había traído ex profeso. Me senté entonces en el suelo, y aunque el licor me devolvió la serenidad, pasó un rato antes de que pudiera decir algo a mi amigo.
    Mientras tanto, él estaba tan lleno de alborozo como yo, sólo que de distinta naturaleza. Me hablaba continuamente diciéndome mil cosas amables para ayudarme a recobrar la calma; pero tal era el ímpetu de la alegría que llenaba mi pecho que sólo servía para colmar mi espíritu de confusión. Por fin me eché a llorar, y poco después fui otra vez dueño de mis palabras.
    Me llegó entonces el turno de abrazar al capitán como a mi salvador, y los dos nos, regocijamos juntos. Le dije que lo consideraba como enviado por el cielo para librarme de mi cautiverio, y que todo lo ocurrido era para mí una cadena de maravillas; tales cosas, agregué, eran los mejores testimonios de cómo la secreta mano de la Providencia gobierna el mundo, y la evidencia de que los ojos de un Poder infinito alcanzan el más remoto rincón del mundo y envían ayuda al más miserable si a Dios le place hacerlo.
    No olvidé elevar mi corazón al cielo, lleno de gratitud. ¿Y qué corazón hubiera podido olvidar a quien no sólo le brindaba su socorro de tan maravillosa manera en la soledad, sino que era el dador de toda liberación en este mundo?
    Luego que hablamos un momento, el capitán me dijo que había hecho desembarcar para mí las pocas provisiones que quedaban en el buque después que los miserables habían dilapidado las existencias mientras fueron los amos. Llamó a los del bote y les ordenó que desembarcaran los presentes para el gobernador. Y por cierto que aquellos presentes parecían más propios para uno que tuviera que quedarse en la isla, en vez de embarcarse para no retornar ya nunca.
    Ante todo venía una caja con botellas conteniendo excelentes aguas cordiales, seis grandes botellas de vino de Madeira —cada una contenía dos pintas—, dos libras de excelente tabaco, doce pedazos de la carne que había a bordo, seis piezas de salazón de cerdo, un saco de guisantes y unas cien libras de galleta.
    También me trajo una caja de azúcar, otra de harina, un saco de limones, así como dos botellas de zumo de lima y abundancia de otras cosas; pero aparte de eso, lo que me resultó mil veces más útil y agradable fue que me obsequió seis camisas nuevas, seis corbatas, dos pares de guantes, uno de zapatos, un sombrero y un par de medias, así como un excelente traje elegido entre los suyos y apenas usado; en una palabra, me vistió de pies a cabeza.
    Fue ciertamente un útil y grato presente para quien estaba como yo en tales circunstancias; pero pocas cosas en el mundo pueden haber sido tan desagradables como lo fue para mí el ponerme por primera vez aquellas ropas que me parecían incómodas, inútiles y absurdas.
    Concluida esta ceremonia, y luego que aquellas excelentes cosas fueron trasladadas a mi castillo, empezamos a discutir qué haríamos con los prisioneros. Nos era necesario considerar seriamente la posibilidad de hacernos a la mar con ellos, especialmente con dos que sabíamos incorregibles y rebeldes en último grado. El capitán los consideraba temibles bandidos y no se sentía seguro en su proximidad; incluso, si los llevaba a bordo serían encadenados y con la decisión de entregarlos a la justicia en la primera colonia inglesa que alcanzáramos en nuestro viaje. Así y todo noté que parecía muy preocupado con esa idea.
    Al advertirlo, le dije que si lo deseaba yo me comprometía a lograr que aquellos dos hombres pidieran voluntariamente ser dejados en la isla.
    —Creedme que si eso fuera posible —replicó entonces el capitán— yo me alegraría de todo corazón.
    —Muy bien —dije—. Los haré venir y hablaré con ellos al respecto. Ordené entonces a Viernes y a los dos rehenes —que habían recobrado la libertad por cuanto sus compañeros habían cumplido bien su deber— que fueran a la caverna en busca de los cinco prisioneros, atados como estaban, y los condujesen a la enramada, adonde iría yo más tarde.
    Pasado cierto tiempo aparecí allí vestido con mi nueva indumentaria y acompañado del capitán, habiendo recobrado para ese entonces mi título de gobernador.
    Reunidos todos, y en presencia del capitán, ordené que comparecieran los presos y les manifesté que tenía una prolija descripción de su villano comportamiento para con el capitán, la forma en que se habían apoderado del barco y cómo se disponían a cometer nuevas fechorías, hasta que la Providencia los había hecho caer en sus propias redes, llevándolos a precipitarse en la fosa que para otros habían cavado.
    Les hice entender que el barco había sido recobrado bajo mi dirección, que estaba ahora en la rada y pronto verían cómo el capitán rebelde era recompensado por su villanía, ya que no tardarían en descubrirlo colgando del palo mayor; en cuanto a ellos, quise saber qué alegaban en su defensa antes de hacerlos ajusticiar como piratas, ya que poseía suficiente autoridad para disponer de inmediato tal castigo.
    Tomando la palabra en nombre de sus compañeros, uno de ellos contestó que nada tenían que decir salvo que en el momento de rendirse el capitán les había prometido la vida, por lo cual humildemente imploraban mi compasión. Repliqué a mi turno que no sabía qué clase de compasión podía brindarles, ya que por lo que a mí concernía acababa de decidirme a salir de la isla con todos mis hombres y a tal efecto había reservado pasajes en el barco del capitán. En cuanto a éste, no podría llevarlos a Inglaterra en otra condición que la de prisioneros, entregándolos a la justicia encadenados y bajo acusación de rebeldes; de más estaba decir que la consecuencia de aquello sería la horca, de manera que yo no alcanzaba a comprender qué ventaja podía tener eso para ellos, salvo la posibilidad de que se decidieran a quedarse en la isla y encarar allí su destino. Si así lo deseaban, agregué, estaba en condiciones de permitírselo; incluso me sentía inclinado a perdonarles la vida si les parecía posible arriesgarse a permanecer en aquella tierra.
    Al oír mi propuesta parecieron sentirse muy agradecidos, y me aseguraron que preferían en mucho quedarse allí antes que ser llevados a Inglaterra para perecer en la horca; de modo que los dejé en esa disposición de ánimo.
    El capitán fingió entonces poner algunas dificultades al proyecto, como si no quisiera dejar a los hombres en la isla.
    A mi vez fingí molestarme con él y le dije que se trataba de prisioneros míos y no suyos; ya que les había ofrecido esa posibilidad, quería hacer honor a mi palabra. Agregué que si no le parecía bien el arreglo, pondría a aquellos hombres en libertad tal como los había encontrado, y si él insistía en no aceptar el convenio, que se apoderara de ellos si podía encontrarlos en la isla.
    Todavía más agradecidos se mostraron aquellos hombres al oírme hablar así, y de inmediato mandé ponerlos en libertad ordenándoles que se retiraran por los bosques hasta el sitio de donde los habían traído; les dije que dejaría en sus manos algunas armas y municiones, dándoles también consejos necesarios para que pudieran vivir bien en la isla si seguían decididos a quedarse.
    Terminada la reunión me dispuse a embarcarme en el navío, pero pedí al capitán que me dejara esa noche para preparar mis cosas, rogándole que entretanto fuese a bordo y cuidara del orden, enviándome la chalupa al día siguiente; le recomendé también que apenas llegado al barco hiciera colgar del mástil el cuerpo del capitán rebelde para que los de la isla pudieran verlo.
    Apenas marchado el capitán, llamé a mi tienda a los rebeldes y me puse a hablar seriamente con ellos. Les dije que habían hecho a mi parecer una buena elección, ya que si el capitán los hubiera llevado consigo seguramente habrían terminado en el patíbulo. Les mostré la figura del rebelde balanceándose en la verga del mástil, y les dije que solamente podían esperar una cosa parecida.
    Cuando me repitieron su decisión de quedarse manifesté que les haría un relato detallado de mi vida en el lugar, mostrándoles al mismo tiempo los medios de procurarse una existencia confortable. Les narré punto por punto todo cuanto conocía de la isla, mi llegada a tierra, indicándoles cómo había levantado las fortificaciones, la forma en que logré tener pan, plantar el grano y secar las uvas; en fin, todo cuanto podían necesitar para que la existencia no les fuera penosa. También les conté la historia de los dieciséis españoles que llegarían a la isla según mis esperanzas, y les di una carta para ellos, haciéndoles prometer que los tratarían de igual a igual.
    Les dejé mis armas de fuego, es decir, cinco mosquetes, tres escopetas y además tres espadas. Quedaba todavía un barril y medio de pólvora, ya que después de los primeros años empleé muy poca evitando desperdiciarla. Les di completas instrucciones sobre el modo de domesticar las cabras, ordeñarlas y cebarlas, así como la manera de hacer manteca y queso.
    En una palabra, los interioricé de cada detalle de mi propia vida, agregando que intercedería ante el capitán para que les dejara otros dos barriles de pólvora, así como semillas de hortalizas, que tan útiles me hubieran sido. Les regalé el saco de guisantes que el capitán me había traído para comer, enseñándoles la forma de sembrarlos para tener mayor cantidad.
    Cumplido todo esto, me despedí de ellos a la siguiente mañana y embarqué de inmediato. Nos preparábamos para hacernos a la vela, pero no levamos anclas esa noche. A la mañana siguiente, dos de los cinco hombres llegaron nadando hasta el navío, y profiriendo toda clase de quejas contra los otros tres nos suplicaron en nombre de Dios que los recibiéramos a bordo, pues de lo contrario serían asesinados, y terminaron rogando al capitán que los admitiera aunque sólo fuese para ahorcarlos inmediatamente.
    Al oírlos, el capitán pretendió no tener autoridad para acceder a su pedido sin mi consentimiento; después de tenerlos así un rato, y luego que prometieron solemnemente corregirse, los hicimos trepar a bordo, donde luego de ser castigados con azotes se condujeron con toda prudencia y honradez.
    Al subir la marea la chalupa fue enviada a tierra con los efectos prometidos a aquellos hombres, a los cuales el capitán agregó por mi intercesión los arcones con sus ropas; se mostraron sumamente agradecidos al recibirlos, y yo les di coraje diciéndoles que si me era posible enviar algún buque para que los recogiera no dejaría de hacerlo.
    Al abandonar la isla llevé conmigo algunos recuerdos, como ser el gorro de piel de cabra que me había hecho, la sombrilla y mi papagayo; también cuidé de llevar el dinero ya mencionado, que durante tanto tiempo me había sido inútil; estaba enmohecido y oxidado, tanto que hasta no frotarlo bien nadie lo hubiese tomado por plata. Igualmente traje a bordo el dinero hallado en el naufragio del barco español.
    Así dejé mi isla, el 9 de diciembre, según el calendario del buque, y en el año 1686, luego de haber estado en ella por espacio de veintiocho años, dos meses y diecinueve días, y siendo librado de este segundo cautiverio en el mismo día en que antaño me fugara de los moros de Sallee, a bordo del barcolongo.
    Al fin de un largo viaje arribé a Inglaterra el 11 de junio de 1687, después de treinta y cinco años de ausencia.







    16. FORTUNA DE ROBINSON




    Al llegar a mi patria era yo en ella tan desconocido como si jamás hubiera pisado antes su suelo. Mi benefactora y depositaría, a quien dejara mi dinero, vivía aún, pero sufriendo grandes privaciones a causa de reveses de fortuna; había enviudado por segunda vez y llevaba una existencia sumamente modesta. Me apresuré a tranquilizarla sobre lo que me debía, asegurándole que no pensaba reclamarle nada; por el contrario, mi gratitud hacia su antigua fidelidad me llevó a ayudarla en cuanto mi pequeño peculio lo permitía. Cierto que en ese momento era bien poco lo que pude hacer por ella, pero le aseguré que jamás olvidaría sus bondades para conmigo, y como se verá más adelante cumplí mi promesa cuando estuve en condiciones de acudir en su ayuda.
    Viajé luego a Yorkshire, hallando que mis padres habían muerto y de la familia sólo quedaban dos hermanas, así como dos niños de uno de mis hermanos. Tanto tiempo había sido dado por muerto en mi hogar que no me habían reservado bienes, de manera que me encontré privado de auxilio y la pequeña cantidad de dinero que llevaba conmigo no era suficiente para establecerme de una manera apropiada en la sociedad.
    Recibí, sin embargo, una muestra de gratitud que no esperaba. El capitán del barco tan providencialmente salvado por mí junto con su navío y cargamento elevó un detallado informe de lo ocurrido a sus armadores, contándoles cómo había yo procedido. Recibí entonces una invitación para que acudiese a verlos, y los encontré en compañía de otros comerciantes, siendo objeto de afectuosas muestras de gratitud así como de un regalo de casi doscientas libras esterlinas.
    Después de reflexionar detenidamente sobre las circunstancias en que me encontraba y las escasas posibilidades de iniciar una empresa con los medios de que disponía, decidí ir a Lisboa y ver de lograr allí algún informe sobre el estado de mi plantación del Brasil, así como la suerte de mi socio, del que imaginaba naturalmente que me habría dado por muerto muchos años atrás.
    Saqué pasaje a Lisboa, a la que arribé en el mes de abril. Mi criado Viernes me acompañaba en todas aquellas andanzas, mostrándose en todo momento lleno de fidelidad hacia mí.
    Al llegar a Lisboa me puse a buscar y tuve al fin la satisfacción de ver a mi viejo amigo el capitán que me rescatara del mar, en la costa africana. Estaba muy anciano y se había retirado dejando a su hijo, ya hombre, a cargo del buque, que continuaba haciendo el tráfico con Brasil. El capitán no me reconoció, y a mí mismo me fue difícil reconocerlo a él, pero después de un momento recordé sus facciones, y lo mismo le ocurrió con las mías.
    Después de regocijarnos mutuamente con la reanudación de nuestra vieja amistad, le pregunté como es de imaginar por el estado de mi plantación y lo que había sido de mi socio. El anciano me dijo que no había viajado al Brasil en los últimos nueve años, pero que podía asegurarme que al abandonar aquellas tierras mi socio vivía aún; en cuanto a los apoderados, a quienes yo dejara junto con aquél al cuidado de mis bienes, ambos habían muerto.
    Con todo creía posible lograr un buen detalle del adelanto de mi plantación, pues luego de haberse difundido la creencia de que me había ahogado en un naufragio, mis apoderados se apresuraron a rendir cuentas de mi parte al procurador fiscal, quien decidió adjudicar aquellos bienes, en tanto no me presentase yo a reclamarlos, un tercio al fisco y dos tercios al monasterio de San Agustín, que los empleaba en beneficio de los pobres y la conversión de los indios al catolicismo.
    Naturalmente bastaría que yo me presentara, o enviase a alguien con suficiente poder para reclamar los bienes en mi nombre, para que todo me fuese entregado. Solamente no me serían devueltas las rentas anuales, que habían sido destinadas a usos de caridad. El capitán me aseguró que el administrador real de las rentas de tierras, así como el «provedidore» o ecónomo del monasterio, habían tenido gran cuidado de que mi socio rindiera anualmente cuenta de lo producido por la plantación, de la cual recibían la mitad.
    Le pregunté si estaba al tanto de las mejoras introducidas en la plantación, y si valía la pena que yo me embarcase rumbo al Brasil; también quise saber si a mi llegada no encontraría dificultades en la toma de posesión de mi parte. Me dijo que ignoraba con exactitud hasta qué punto había crecido la plantación, pero sí sabía que mi socio era ahora un hombre muy rico con sólo el producto de una mitad del total. También recordaba haber oído que el tercio de mi parte consagrado al fisco —que aparentemente era entregado a otro monasterio o fundación religiosa— sumaba más de doscientos moidores1 anuales.
    En cuanto a la toma de posesión de mis bienes, él no encontraba la menor dificultad, ya que mi socio vivía y podría testimoniar de mis derechos, fuera de que mi nombre estaba debidamente inscrito en el registro de propietarios.
    Agregó que los sucesores de mis dos apoderados eran excelentes y honestas personas, dueñas de gran riqueza, por lo cual yo tendría no solamente ayuda para recobrar mis posesiones sino que recibiría una gran suma de dinero, producto de lo rendido por la plantación antes de que pasara a manos del estado en la forma señalada, cosa ocurrida unos doce años atrás según creía recordar.
    Al escuchar sus palabras, me mostré sumamente preocupado e inquieto y quise saber cómo era posible que aquellos apoderados hubiesen dispuesto a su manera de mis efectos, siendo que yo había hecho testamento antes de embarcarme por el cual lo declaraba a él, el capitán portugués, mi legatario universal.
    Me dijo que eso era cierto, pero que no existiendo prueba de que yo hubiese muerto no podía él actuar como ejecutor testamentario hasta tanto se recibiera testimonio seguro de mi desaparición.
    Fuera de eso, no había querido intervenir en un asunto radicado en tierras tan remotas, aunque había registrado debidamente mi testamento a fin de que constasen sus derechos. De haber tenido prueba cierta de mi muerte, hubiese actuado por procuración recibiendo el «ingenio» —como llaman a las fábricas de azúcar— por intermedio de su hijo, que se encontraba actualmente en el Brasil.
    —Sin embargo —agregó el anciano— tengo que daros algunas otras noticias que acaso no os resulten tan agradables. Creyendo que habíais muerto, como lo creía todo el mundo, vuestro socio y los apoderados me rindieron cuentas y entregaron los beneficios en vuestro nombre durante los seis u ocho primeros años. Dichas sumas fueron aceptadas por mí, pero como en aquel entonces había grandes gastos en la plantación, tales como construir un ingenio y comprar esclavos, la suma no se elevó tanto como en años posteriores. Con todo —agregó el capitán— os rendiré el detalle de cuanto he recibido, y la forma en que dispuse del dinero.
    Días más tarde, prosiguiendo mis conversaciones con mi viejo amigo, me entregó la cuenta de lo producido por mi plantación en los primeros seis años, detalle que aparecía firmado por mi socio y los apoderados, y que había sido entregado en especies tales como tabaco en rama, cajas de azúcar, y también ron, melaza y otros productos derivados de la refinación del azúcar. Pude entonces observar que el total crecía de año en año, pero como el desembolso para los gastos mencionados había sido grande las sumas resultaban pequeñas.
    El capitán me hizo saber además que era mi deudor por la suma de cuatrocientos setenta moidores, aparte de sesenta cajas de azúcar y quince fardos dobles de tabaco, que se habían perdido en el naufragio de su barco, ocurrido al regresar a Lisboa unos once años después de mi desaparición.
    Entonces comenzó el anciano a quejarse de sus desgracias, y cómo se había visto obligado a hacer uso de mi dinero para recobrarse de sus pérdidas y adquirir una participación en un nuevo navío.
    —Pese a ello, mi viejo amigo —agregó—, no habrán de faltaros auxilios en vuestra presente necesidad; tan pronto vuelva mi hijo recibiréis todo lo que se os debe.
    Y sacando allí mismo un viejo saco me entregó 160 moidores portugueses así como los títulos de su participación en el buque, del cual su hijo y él tenían una cuarta parte respectivamente, y me los dio como garantía del resto.
    Mucho me emocionaron la honestidad y la gentileza de aquel hombre, tanto que apenas pude soportar aquella escena. Recordaba lo que el capitán había hecho por mí, cómo me libró del mar y con qué generosidad se había conducido en toda ocasión. Al darme cuenta de tan sincera amistad, apenas pude contener las lágrimas escuchando sus palabras, y lo primero que hice fue preguntarle si las circunstancias le permitían desprenderse de tal cantidad de dinero, y si ello no le ocasionaría apuros. Me respondió que sin duda ese pago significaba para él un trastorno, pero de todos modos se trataba de mi dinero y yo lo necesitaba más que él.
    Todo cuanto habló estaba impregnado de afecto, y a mí me costaba escucharlo sin prorrumpir en llanto. Por fin acepté cien moidores, y le pedí papel y pluma para extenderle un recibo por ellos. Entregándole luego el resto, le dije que si algún día entraba en posesión de mi ingenio le devolvería asimismo lo que ahora aceptaba, cosa que más adelante cumplí. En cuanto a los títulos del barco no quería recibirlos de ningún modo, seguro de que si algún día necesitaba yo dinero él era harto honrado para pagarme de inmediato, y si la suerte me permitía recobrar mi plantación jamás aceptaría un solo penique de sus manos.
    Decidido esto, el anciano capitán me ofreció su ayuda a fin de reclamar mis bienes. Le dije que estaba dispuesto a embarcarme en persona para el Brasil, a lo que me contestó que lo hiciera si me parecía bien, pero que había otros recursos para lograr el mismo fin y obtener una inmediata restitución de lo mío.
    Algunos barcos estaban alistándose en Lisboa para emprender viaje al Brasil, y el capitán hizo que mi nombre fuera inscripto de inmediato en un registro público, con una declaración jurada suya en la cual afirmaba que yo estaba vivo y que era la misma persona que había iniciado la plantación de cuya entrega se trataba.
    Legalmente consignada por un notario la declaración, y con un poder adjunto, el capitán me aconsejó enviarla con una carta suya a un comerciante amigo, proponiéndome luego que permaneciera con él en Lisboa hasta que los navios retornaran con noticias.
    Nunca hubo poder ejercido con más legalidad que el que yo diera a aquel comerciante; en menos de siete meses recibí un grueso paquete procedente de los herederos de mis apoderados, los plantadores a cuya cuenta me hice a la mar como he narrado, y dentro del cual encontré los siguientes documentos:
    Primero, una cuenta detallada de lo producido por mi plantación a partir del último año en que sus padres habían ajustado cuentas con el capitán portugués; el balance arrojaba un saldo de mil ciento setenta y cuatro moidores en mi favor.
    Segundo, la cuenta de otros cuatro años durante los cuales administraron los bienes, antes de que el gobierno reclamara la parte que la ley fija en caso de no tenerse noticias del dueño, cosa que ellos llaman muerte civil; el balance de dichos años, por haber aumentado entonces el producto de la plantación, arrojaba un total de treinta y ocho mil ochocientos noventa y dos cruzados, lo que hacían tres mil doscientos cuarenta y un moidores.
    Tercero, la cuenta rendida por el prior del convento agustino, que había recibido rentas por espacio de catorce años; descontando lo destinado a gastos de hospital, declaraba honestamente tener aún ochocientos setenta y dos moidores sin empleo, los que ponía a mi disposición. En lo que respecta a la porción del fisco, nada me fue devuelto.
    Venía además una letra de mi socio donde me expresaba su regocijo por saberme vivo, me hacía un prolijo relato de cómo había progresado la plantación, lo que producía anualmente, así como el número de acres que tenía en la actualidad; me indicaba la superficie sembrada, el número de esclavos que trabajaban allí, terminando por trazar veintidós cruces a manera de bendiciones, y diciéndome cuántas Ave Marías había rezado para agradecer a la Virgen Santísima mi salvación. Me invitaba con mucho calor a que fuese al Brasil para tomar posesión de mis bienes, y que entretanto le enviase órdenes para rendir cuentas a quien yo designase en mi ausencia. Por fin hacía protestas de su amistad, incluyendo a su familia, y me enviaba como regalo siete hermosas pieles de leopardo que había recibido de la costa africana adonde enviaba con frecuencia barcos que sin duda habrían tenido mejor viaje que el mío. Con las pieles venían cinco cajas de excelentes confituras y cien piezas de oro sin acuñar, no tan grandes como los moidores. Por el mismo barco mis apoderados me fletaron mil doscientas cajas de azúcar, ochocientos rollos de tabaco y el resto del producto en oro.
    Ciertamente podía decir yo ahora que el final de Job era mejor que el principio. Es imposible narrar los sentimientos de mi corazón al leer aquellas cartas y enterarme de la fortuna que poseía. Porque como los barcos del Brasil navegan siempre en convoy, junto con las cartas venían los bienes y éstos estaban ya desembarcados y en seguridad antes de que aquéllas llegaran a mis manos.
    En una palabra, palidecí y creí que iba a desmayarme, a no mediar el capitán, que corrió a hacerme beber un cordial. Pienso que la súbita sorpresa producida por la alegría hubiera superado la resistencia de la naturaleza y me hubiera fulminado allí mismo.
    Con todo estuve muy enfermo durante muchas horas, hasta que se envió por un médico, el cual averiguando en parte las razones de mi estado ordenó una sangría, que de inmediato me produjo alivio. Estoy convencido que de no haber recibido ese tratamiento hubiera muerto con seguridad.
    Era ahora dueño, súbitamente, de más de cinco mil libras esterlinas en dinero y una posesión en el Brasil que rendía más de mil libras anuales, tan segura como si hubiese estado en Inglaterra. En suma, me encontraba en una situación de la que apenas alcanzaba a darme clara cuenta, incapaz de serenarme lo bastante para gozar de ella.
    Lo primero que hice fue recompensar a mi antiguo benefactor, el anciano capitán que tan generoso había sido conmigo en mi desventura, lleno de bondad en mis comienzos y honrado al final. Le mostré lo que acababa de recibir, y le dije que aparte de la Providencia, que dispone de toda cosa, a nadie debía tanto como a él, y que afortunadamente estaba ahora en condiciones de recompensarlo, lo que quería hacer cumplidamente. En primer término le entregué los cien moidores que recibiera de él y luego, enviando por un notario, le hice redactar un documento librándolo del pago de los cuatrocientos setenta y dos moidores que había admitido deberme. A continuación mandé redactar un poder por el cual el capitán sería recaudador de las rentas anuales de mi plantación, indicando a mi socio que debería rendirle cuentas y enviarle los productos a mi nombre con las flotas anuales. Agregué una cláusula al final, disponiendo para él una renta vitalicia de cien moidores, así como otra de cincuenta moidores para su hijo. Y así pude pagar mi deuda de gratitud a ese anciano y buen amigo.
    Me quedaba ahora considerar qué camino seguiría y qué destino iba a dar a la fortuna que la Providencia acababa de poner en mis manos. Por cierto que pasaba más preocupaciones que en mi tranquila y sosegada vida en la isla, donde no deseaba más de lo que tenía ni tenía más de lo que deseaba. Ahora, en cambio, abrumado por el peso de mis bienes, reflexionaba sobre la manera de conservarlos en seguridad. Carecía de una caverna donde ir a enterrar mi oro, o un sitio donde dejarlo sin cerrojos ni llave hasta que se oxidara y enmoheciera sin que nadie lo tocase. Mi antiguo amigo el capitán era un hombre honesto, y por el momento el único refugio que tenía.
    En segundo lugar, mis intereses en el Brasil parecían reclamarme allá, pero no quise embarcarme rumbo a aquellas tierras hasta no haber ordenado mis asuntos y puesto mis bienes en manos seguras.
    Pensé primero en la anciana viuda, de cuya honestidad tenía muchas pruebas y que merecía toda mi confianza. Pero era ya muy anciana y pobre, y hasta donde yo podía imaginarlo estaría llena de deudas. No me quedaba más, en una palabra, que volverme personalmente a Inglaterra y llevar conmigo mis bienes.
    Pasaron empero algunos meses antes de resolverme al viaje. Entretanto, del mismo modo que había recompensado generosamente al capitán por sus muchas bondades, así quise hacerlo con aquella pobre viuda cuyo esposo había sido mi primer benefactor y ella, mientras le fue posible, mi fiel depositaría y consejera. Me apresuré, pues, a buscar un comerciante de Lisboa para que escribiera a su corresponsal en Londres con orden de ir en su busca y llevarle en persona cien libras esterlinas de mi parte, así como consuelo y aliento en su pobreza, con la seguridad de que si la vida me lo permitía, en ningún momento iba a faltarle auxilio. Al mismo tiempo remití cien libras a cada una de mis hermanas, que, aunque no estaban en la miseria, vivían rodeadas de preocupaciones; una de ellas había quedado viuda, y la otra estaba casada con un hombre cuya conducta no había sido la deseable.
    Pero entre todos mis parientes y relaciones no podía sin embargo encontrar ninguno a quien confiar el grueso de mis bienes, a fin de embarcarme para el Brasil dejando todo seguro a mis espaldas. Esto, naturalmente, me llenaba de perplejidad.
    Alguna vez había tenido idea de establecerme definitivamente en el Brasil, país en el que me sentía como quien dice naturalizado. Pero en mi conciencia se despertaban algunos pequeños escrúpulos en materia religiosa, que poco a poco fueron disuadiéndome de aquella idea, como contaré luego. Sin embargo, por el momento no era la religión lo que me apartaba de aquel viaje; así como no había tenido escrúpulos en profesar  abiertamente el culto del país mientras viví allá, tampoco lo tendría ahora. Solamente meditando una y otra vez la cuestión, con más profundidad que en otros tiempos, me imaginé viviendo y muriendo en aquel país* y principié a lamentar haber profesado aquella religión, de la que no tenía seguridad que fuese la mejor para que me acompañara en el momento de mi muerte.
    Repito, con todo, que no era esta la razón principal que me detuviera en el proyectado viaje, sino que seguía sin encontrar la persona a quien confiar mis bienes. Me decidí por fin a marcharme a Inglaterra con ellos, seguro de que una vez allá podría hacerme de relaciones que me fueran fieles; de inmediato empecé a hacer preparativos para encaminarme a mi patria con toda mi fortuna.
    Dispuesto ya a ello, y estando la flota del Brasil lista para zarpar, quise ante todo responder en debida forma a quienes me habían remitido tan fiel y excelente rendición de cuentas. Escribí al prior de San Agustín dándole mil gracias por su recta administración y su oferta de los sobrantes ochocientos setenta y dos moidores, de los cuales le rogué que apartara quinientos para el monasterio y trescientos setenta y dos para los pobres, de acuerdo con lo que él dispusiera, y pidiendo al buen padre que rogara por mí, y otras cosas semejantes.
    Escribí luego a mis dos apoderados, dándoles testimonio de toda la gratitud que su justicia y honestidad despertaban en mí. No les envié presente por cuanto su riqueza los colocaba por encima de eso.
    Por fin escribí a mi socio, testimoniándole mi reconocimiento por la diligencia con que había hecho progresar la plantación y el empeño que había puesto en acrecentar su rendimiento; le envié instrucciones para el gobierno de mi tierra y de acuerdo con el poder que había dado a mi amigo el capitán le indiqué mi deseo de que a él le fuera remitido todo lo que me correspondía a la espera de que yo enviara instrucciones más concisas.
    Finalmente le hice saber que no sólo proyectaba ir pronto al Brasil, sino que mi intención era la de establecerme allí por el resto de mi vida. A la carta agregué un regalo consistente en sedas de Italia para su esposa y sus dos hijas.
    Habiendo así arreglado mis asuntos, vendido mi cargamento y convertido todos mis efectos en buenas letras de cambio encaré la siguiente dificultad, que era la de elegir el camino para volver a Inglaterra. Harto habituado estaba yo al mar, y, sin embargo, sentía extraña aversión a la idea de embarcarme rumbo a mi patria. No hubiera podido explicar las causas, pero como no conseguía dominarla llegué incluso a abandonar el viaje cuando ya tenía mis maletas hechas; y no una sino dos o tres veces me ocurrió lo mismo.
    Es verdad que en el mar yo había sido muy desgraciado, y ésta puede ser una de las causas; pero que nadie desoiga nunca los irresistibles impulsos de su espíritu en casos como el mío. Dos de los barcos que había escogido para realizar el viaje (y a tal punto escogido que en uno de ellos llegué a hacer subir a bordo mi equipaje y en otro dispuse análogos arreglos con el capitán) sufrieron grandes desgracias. Uno fue apresado por los argelinos, mientras el otro naufragó cerca de Torbay y todo el pasaje se ahogó con excepción de tres hombres, lo que prueba que en cualquiera de aquellos barcos mi destino hubiera sido funesto.
    Después de atormentarme así en mis pensamientos, acabé por confiar mis aprensiones al anciano capitán, quien se apresuró a pedirme que no viajara por mar sino que hiciera el viaje a La Coruña por tierra, cruzando allí el golfo de Vizcaya hasta la Rochela, desde donde había un cómodo y seguro viaje por tierra a París, luego a Calais y Dover. El otro camino consistía en llegar a Madrid y de ahí por tierra a París.
    En resumen, tan inquieto me sentía ante la idea de navegar que, salvo el obligado tramo de Calais a Dover, me decidí a hacer la travesía enteramente por tierra, lo que además podía resultar mucho más placentero desde que no tenía ningún apuro en llegar a destino. Para mayor seguridad, el viejo capitán me presentó a un caballero inglés, hijo de un comerciante en Lisboa, quien se manifestó dispuesto a viajar conmigo; poco después se agregaron otros dos comerciantes ingleses y dos jóvenes caballeros portugueses, uno de los cuales iba sólo hasta París.
    Éramos en total seis viajeros con cinco sirvientes; los dos mercaderes axial como los dos portugueses se arreglaban con un sirviente entre ambos para evitar mayores gastos, y en cuanto a mí había elegido a un marinero inglés para que me sirviera en el viaje, ya que mi criado Viernes desconocía demasiado las costumbres para serme de utilidad en un trayecto semejante.
    Así salimos de Lisboa, y como el grupo estaba muy bien montado y armado, hacíamos un pequeño ejército en el cual tuve el honor de ser considerado capitán, tanto por ser el mayor de ellos como por llevar dos sirvientes, y también porque había sido el organizador de aquella travesía.
    Del mismo modo que no he querido fatigaros con el relato de mis viajes por mar, tampoco quiero hacerlo ahora con uno por tierra; sin embargo, algunas aventuras que nos acontecieron en tan tediosa y difícil marcha no deben ser omitidas.
    Cuando llegamos a Madrid, como éramos todos extranjeros en España, no quisimos seguir la marcha sin quedarnos un tiempo para visitar la corte y ver lo que merecía ser conocido. Sin embargo, concluía ya el verano, y nos apresuramos a reanudar el viaje abandonando Madrid a mediados de octubre. Apenas habíamos llegado a la frontera de Navarra cuando empezamos a recibir alarmantes noticias de los distintos pueblos que cruzábamos, según las cuales había nevado tanto del lado francés de las montañas que muchos viajeros se habían visto obligados a retornar a Pamplona, después de intentar a todo riesgo cruzar los Pirineos.
    Al llegar a Pamplona tuvimos la confirmación de las noticias. Para mí, adaptado a los rigores de un clima cálido, en países donde difícilmente se tolera alguna ropa, el frío me resultaba insoportable. Aún más penoso me parecía por el hecho de que apenas diez días antes habíamos salido de Castilla la Vieja, donde el clima no sólo es templado sino hasta muy cálido, y casi de inmediato recibíamos el viento tan crudo, tan glacial de los Pirineos, que resultaba intolerable y peligroso por la forma en que se nos helaban las manos y los pies.
    El pobre Viernes tenía un susto terrible al descubrir las montañas cubiertas de nieve y sentir los rigores del clima, cosas que no había visto ni experimentado jamás anteriormente.
    Para abreviar, diré que en Pamplona siguió nevando de tal modo y con tal persistencia que las gentes aseguraban que aquel invierno se presentaba adelantado; los caminos, hasta entonces difíciles de transitar, eran ahora impracticables. En algunas partes la nieve alcanzaba una altura que se oponía a todo intento de franquearla, y no endureciéndose como en los países septentrionales ofrecía el peligro de sepultar vivo al que osara dar allí un paso. Nos quedamos no menos de veinte días en Pamplona, hasta que observando que el invierno avanzaba y no había posibilidades de que el tiempo mejorara, ya que resultaba ser la estación más rigurosa de que se hubiera tenido memoria en Europa, acabé por proponer a mis compañeros irnos a Fuenterrabia y desde allí embarcarnos para Burdeos, lo que significaba solamente un pequeño viaje por mar.
    Mientras considerábamos esta posibilidad, llegaron cuatro caballeros franceses que, detenidos del lado francés de los pasos por la misma razón que lo estábamos nosotros en el lado español, habían encontrado un guía que los llevó a través del país cerca del extremo del Languedoc, haciéndoles pasar las montañas por caminos tales que la nieve no los incomodó mayormente; la que encontraron, según supimos de sus labios, estaba tan endurecida que soportaba fácilmente el peso de caballos y jinetes.
    Buscamos entonces al guía, quien se comprometió a llevarnos por los mismos pasos libres de nieve siempre que nos armáramos de manera adecuada para protegernos de los animales salvajes; era muy frecuente, según nos dijo, que en tiempos de grandes nevadas aparecieran lobos al pie de las montañas y el hambre que la desolación reinante les producía los tornaba altamente peligrosos.
    Le dijimos que veníamos bien preparados para recibir a tales fieras, pero que deseábamos su garantía de que no seríamos atacados por otra especie de lobos que marchan sobre dos pies y que, según se nos había dicho, abundaban mucho, especialmente del lado francés de las montañas.
    Nos aseguró que en los pasos por donde nos llevaría tal peligro era inexistente, de manera que nos decidimos a seguirlo y así lo hicieron también otros doce caballeros con sus sirvientes: algunos eran franceses, otros españoles, y entre ellos se contaban los que habiendo tratado de cruzar las montañas habían tenido que retroceder.
    El quince de noviembre salimos de Pamplona conducidos por nuestro guía, y mi sorpresa fue grande cuando en vez de llevarnos hacia el norte nos hizo desandar el mismo camino por el cual habíamos venido de Madrid. Lo seguimos unas veinte millas, cruzando dos ríos, y entramos en una zona de clima templado donde las tierras tenían un aspecto muy agradable y no había señales de nieve, hasta que de pronto, tornando a la izquierda, nos llevó hacia las montañas por otro camino.
    En verdad que los cerros y los precipicios eran espantosos, pero el guía nos hizo dar tantas vueltas y revueltas, nos llevó por crestas y cornisas tan vertiginosas que terminamos por trasponer las alturas mayores sin haber sido excesivamente molestados por la nieve. Ya entonces nos mostró nuestro guía las hermosas y fértiles provincias de Languedoc y Gascuña que se extendían abajo y a una gran distancia, verdes y florecientes, y a las cuales llegaríamos después de vencer otro trecho de áspero camino.
    Nos sentimos algo inquietos cuando se puso a nevar todo un día y una noche con tanta violencia que tuvimos que detenernos; pero el guía nos tranquilizó asegurándonos que pronto saldríamos del trance. En efecto, advertimos que estábamos ya en el descenso y que nos encaminábamos cada vez más hacia el norte, de manera que proseguimos confiados el viaje.
    Unas dos horas antes de que cayera la noche, cuando nuestro guía cabalgaba un poco adelantado y fuera de nuestra vista, tres monstruosos lobos y un oso salieron de un hueco que daba acceso a un espeso bosque. Dos de los lobos se precipitaron sobre el guía, y si hubiera estado una media milla más adelante de nosotros lo hubiesen devorado antes de poder acudir en su auxilio. Uno de los lobos atacó al caballo, mientras el otro saltaba sobre el jinete con tal violencia que no le dio tiempo a sacar la pistola sino que, perdiendo la cabeza, sólo atinó a gritar con todas sus fuerzas en demanda de socorro. Como mi criado Viernes marchaba a mi lado, le ordené que fuese al galope a ver lo que ocurría. Así que Viernes descubrió la escena gritó tan fuerte como el otro:
    — ¡Amo, amo!
    Pero al mismo tiempo, con extraordinaria valentía, galopó directamente hacia el atacado y sacando su pistola atravesó de un tiro la cabeza de la fiera.
    Fue una suerte para el guía que Viernes acudiera a ayudarlo, pues como estaba habituado a lidiar con esa clase de animales en su país no les temía y se acercaba casi hasta tocarlos antes de disparar sobre ellos; de haber sido alguno de nosotros habría tirado desde más lejos, tal vez errando el disparo o hiriendo al jinete.
    Lo que siguió hubiera bastado para aterrar a un hombre más valiente que yo, y por cierto hizo temblar a todos los que viajábamos cuando, al expandirse el ruido del disparo de Viernes, a ambos lados del camino oímos levantarse un horroroso aullar de lobos; aquellos aullidos, multiplicados por el eco de la montaña, nos daban la impresión de que había prodigiosa cantidad de fieras al acecho; y por cierto que la manada que nos causaba tanto miedo no debía ser de las más pequeñas.
    Apenas mató Viernes al lobo, el que se había encarnizado con el caballo lo abandonó para huir a toda carrera. Por fortuna había mordido al caballo en la cabeza, donde la copa del freno le atascó las mandíbulas, impidiéndole hacer mucho daño. El guía en cambio estaba mal herido, pues el furioso animal alcanzó a desgarrarle el brazo y el muslo. En el momento de llegar Viernes estaba a punto de caer de su encabritado caballo.
    Es de imaginar que al sonido del disparo lanzamos todos nuestros corceles al galope, y que aunque el camino era muy áspero pudimos llegar casi en seguida al sitio de la escena.
    Tan pronto dejamos atrás los árboles que nos impedían ver más adelante, advertimos lo que había ocurrido y cómo Viernes acababa de salvar al pobre guía, aunque en el primer instante tardamos en darnos cuenta de la especie de enemigo que lo había atacado.
    Pero nunca hubo lucha más temeraria, y librada de manera más original, que la que siguió entre Viernes y el oso, tanto que para nosotros, asustados en el primer instante, fue de inmediato la más grande de las diversiones. Así como el oso es un animal pesado y torpe, que no puede correr con la velocidad del lobo, tiene en cambio dos cualidades particulares que por lo común regulan sus acciones. Ante todo, los hombres no constituyen su presa, bien que en las circunstancias en que nos veíamos no es posible asegurar si las nevadas habrían tornado hambriento a aquel animal; pero habitualmente el oso no ataca nunca al hombre si éste no lo provoca antes. Cuando se encuentra un oso en los bosques, él no intentará molestaros si pasáis a su lado sin ocasionarle a vuestro turno molestias; eso sí, tenéis que ser exquisitamente educados con él y cederle el paso, porque es un caballero muy sensible. Por cierto que no se apartaría un milímetro de su camino aunque fuera para dejar pasar a un príncipe; de manera que si tenéis miedo, lo mejor es mirar hacia otro lado y continuar caminando, pues a veces basta detenerse y mirarlo fijamente para que considere esto una ofensa. Lo mismo si le tiráis algo que le acierte, aunque sólo sea un palillo más delgado que el dedo, lo considerará un insulto y abandonará todas sus ocupaciones por la sola de vengarse; en materia de honor, el oso exige siempre cumplida satisfacción. Tal es su primera cualidad, y la segunda consiste en que una vez ofendido jamás abandonará vuestra persecución, de noche o de día, hasta conseguir alcanzaros y obtener a toda costa su ansiada venganza.
    Mi criado Viernes había salvado al guía, y cuando llegamos a su lado lo ayudaba a desmontar, pues el hombre estaba a la vez herido y asustado, tal vez más lo segundo que lo primero; de pronto vimos surgir al oso del bosque, y por cierto que era monstruosamente grande, el mayor que yo haya visto. Nos quedamos sorprendidos de su aparición, pero cuando Viernes lo descubrió vimos que su rostro expresaba alegría y contento.
    — ¡Oh, oh, oh! —exclamó, señalando tres veces hacia el oso—. ¡Oh, amo, darme permiso para estrechar mano del oso! ¡Yo haceros reír mucho! Mi  sorpresa fue  grande  al  ver tan complacido  al muchacho.
    — ¡Estás loco! —atiné a decirle—, ¡Te comerá!
    — ¡Comerme! ¡Comerme! —dijo Viernes—. ¡Yo comerlo a él! ¡Yo haceros reír, vosotros quedar aquí y yo haceros reír mucho!
    Se sentó en el suelo, cambiándose en un instante sus botas por un par de zapatos que llevaba en la faltriquera, entregó su caballo a mi otro sirviente y llevando la escopeta en la mano echó a correr rápido como el viento.
    El oso marchaba lentamente, sin intención aparente de mezclarse con nadie, hasta que Viernes se le acercó llamándolo como si el animal pudiese entender sus palabras. — ¡Oye, oye, yo hablar contigo! —le decía. A cierta distancia seguíamos nosotros la escena, encontrándonos ya en el lado gascón de las montañas y en un vasto bosque, cuyo suelo era llano y bastante abierto, con árboles diseminados aquí y allá.
    Viernes, que como he dicho estaba casi pisándole los talones al oso, se acercó todavía más y levantando de pronto una piedra se la tiró a la cabeza, donde no le hizo más daño que si la hubiese arrojado contra una pared. Aquello sin embargo produjo el efecto que Viernes esperaba, ya que el muchacho se mostraba tan temerario que su intención evidente era que el oso lo persiguiera para «nosotros reír mucho», según su lenguaje.
    Al punto que el oso sintió la pedrada, y vio a su agresor, se volvió rápidamente y se lanzó tras él, dando largas zancadas y moviéndose de una manera tan extraña y rápida que hubiera obligado a trotar a un caballo para alcanzarlo. Viernes huía velozmente, y de pronto se dirigió en nuestra dirección como si quisiera buscar socorro, de modo que resolvimos hacer una descarga contra el oso y salvar al muchacho.
    Yo sentía una gran cólera contra él al verlo lanzar al oso sobre nosotros, en especial cuando la fiera en nada había pretendido atacarnos, de manera que empecé a gritarle lo que merecía.
    — ¡Gran imbécil! —exclamé—. ¿Es ésta tu manera de hacernos reír? ¡Ven aquí y toma tu caballo mientras nosotros matamos al oso!
    Al oírme, respondió a gritos: — ¡No tirar, no tirar! ¡Quedaros ahí, vos reír mucho! Y como el ágil muchacho corría dos metros por cada uno que franqueaba el oso, giró de improviso y viendo a un lado un magnífico roble que parecía apropiado a sus planes, nos hizo señas de que lo siguiéramos y redoblando su velocidad saltó al árbol, no sin antes dejar la escopeta en tierra, a unas cinco o seis yardas del tronco.
    Pronto llegó el oso al árbol, y lo contemplamos a alguna distancia. Lo primero que hizo fue detenerse junto a la escopeta y olfatearla, pero la abandonó en seguida y precipitándose al árbol empezó a trepar con la agilidad de un gato a pesar de su enorme corpulencia.
    Al ver esto me espanté de lo que consideraba una locura de mi criado y en nada vi motivo para reírme; todos nosotros nos apresuramos en cambio a acercarnos al árbol. Cuando llegamos casi junto a él vimos a Viernes que estaba trepado en el extremo de una larga rama del roble, y al oso que se encontraba a mitad de camino en la misma rama. Tan pronto como la fiera llegó a la porción donde era más delgada y flexible, oímos que Viernes nos gritaba: — ¡Ah! ¡Verme ahora enseñar a bailar al oso! Y se puso a saltar y a agitar violentamente la rama, con lo cual el animal empezó a bambolearse, pero hizo lo posible por sostenerse firme, aunque miraba hacia atrás para descubrir la manera de retroceder. Esto, como es de imaginar, nos hizo reír mucho. Pero Viernes no había concluido todavía con él. Al verlo indeciso, comenzó a hablarle como si aquel animal hubiese podido responderle en inglés.
    — ¡Cómo! ¿No venir más cerca? ¡Yo rogarte venir más cerca!
    Dejó entonces de sacudir la rama y el oso, como si hubiese comprendido la invitación, avanzó otro poco; pero nuevamente se puso Viernes a saltar en la rama y el oso se detuvo de inmediato.
    Pensamos que ese era buen momento para acertarle en la cabeza, y grité a Viernes que no se moviera a fin de tirar sobre la fiera, pero él nos detuvo con sus súplicas.
    — ¡Oh, ruego no tirar, no tirar! ¡Yo tirar después y entonces!
    Quería decir que tiraría en el debido momento. En fin, y para abreviar este relato, Viernes danzó tanto en la rama y el oso adoptó unas posturas tan grotescas que nos desternillamos de risa aunque no podíamos comprender cómo se las arreglaría finalmente mi criado. Al principio creímos que intentaba derribar al animal, pero éste era demasiado astuto para eso; no sólo evitaba avanzar más sino que hundía las garras en la madera con tal fuerza que no comprendíamos cómo sería posible terminar la aventura en esa situación.
    Pronto nos sacó Viernes de dudas, por lo que dijo al oso cuando comprendió que no podía desprenderlo de la rama ni persuadirlo de que avanzara otro poco.
    Bien, bien —exclamó—, tú no querer venir, yo ir, yo ir. Tú no venir a mí, yo ir a ti.
    Y con estas palabras deslizándose hasta la extremidad de la rama que se iba inclinando bajo su peso, se dejó resbalar suavemente sosteniéndose de la punta hasta que sus pies casi tocaron tierra. Soltó entonces la rama y fue a tomar su escopeta, quedándose allí a la espera.
    —Bueno —dije yo—. ¿Qué vas a hacer ahora, Viernes? ¿Por qué no le tiras?
    —No tirar —repuso él—; si yo tirar ahora no matar. Yo daros todavía mucha risa.
    Y  así fue, como podrá verse; porque cuando el oso advirtió que se le había escapado el enemigo, empezó a retroceder por la rama, haciéndolo con extremadas precauciones, midiendo cada paso que daba y andando hacia atrás hasta que alcanzó el tronco del roble; allí, con el mismo cuidado y marchando siempre hacia atrás, descendió por el tronco, clavando profundamente las garras y moviendo despacio cada pata.
    En este instante antes de que hubiera logrado apoyarse en tierra firme, Viernes se le acercó y metiéndole el caño de la escopeta en una oreja lo tendió sin vida a sus pies.
    El muy pícaro se volvió luego a nosotros para ver si efectivamente habíamos reído, y cuando advirtió el regocijo de nuestros rostros se echó a reír a carcajadas.
    —Así nosotros matar osos en nuestro país —explicó.
    — ¿Los matáis así? —repliqué—. ¡Pero si no tenéis escopetas!
    —No, no escopeta —dijo—. Tirarles muchas flechas largas.
    Todo aquello nos divirtió mucho, pero estábamos todavía en un sitio desolado, con el guía mal herido y sin saber exactamente qué hacer. El aullar de los lobos me preocupaba, ya que a excepción de los gritos que escuchara en la costa africana, en un episodio que ya he narrado, creo que nada podía haberme llenado más de espanto.
    Todo eso, sumado a la cercanía de la noche, nos disuadió de desollar al oso como nos lo pedía Viernes; de lo contrario hubiéramos llevado con nosotros la piel de aquel enorme animal, que por cierto merecía conservarse; pero aún nos quedaban tres leguas por recorrer y nuestro guía nos urgía a proseguir el camino.
    La tierra estaba allí cubierta de nieve, aunque no con el espesor peligroso de las montañas. Las manadas de lobos hambrientos, como supimos más tarde, habían descendido azuzadas por el hambre, a los bosques y las llanuras, y causaban graves daños en las aldeas, donde sorprendieron a las indefensas gentes, mataron gran cantidad de cabezas de ganado y también a algunas personas.
    Nos quedaba un peligroso paso por atravesar, del cual nos dijo el guía que si había aún lobos en la región los encontraríamos allí; se trataba de una pequeña planicie, rodeada por todas partes de bosques y con un angosto y profundo desfiladero por el cual era necesario pasar a fin de vernos al abrigo del pueblo donde pernoctaríamos. Apenas habíamos cargado nuestras escopetas y alistado para cualquier evento, oímos terribles aullidos en el bosque de la izquierda, un poco hacia adelante y justamente en la dirección por la cual teníamos que marchar.
    La noche caía y la luz era ya débil, lo que empeoraba nuestra situación; al crecer el confuso sonido percibimos distintamente que eran aullidos de aquellas diabólicas fieras. De improviso descubrimos dos o tres manadas de lobos, una a la izquierda, otra detrás y la tercera avanzando de frente, de manera que nos vimos casi rodeados por ellas. Sin embargo, como no se precipitaban sobre nosotros, seguimos avanzando con toda la rapidez de nuestros caballos que, dado lo áspero del camino, apenas podían andar a trote largo. Llegamos así a la entrada de un bosque situado al final de la planicie, bosque que debíamos atravesar por un desfiladero. Fue allí cuando tuvimos la sorpresa de ver, justamente a la entrada del paso, una gran cantidad de lobos detenidos y a la espera.
    En ese instante oímos un tiro en otro lado del bosque, y mirando hacia allí vimos pasar como una exhalación un caballo ensillado, que corría como el viento perseguido por dieciséis o diecisiete lobos. Los feroces animales estaban ya casi sobre el pobre caballo, y seguros de que no podría sostener mucho tiempo la velocidad de su carrera descontábamos que al final lo alcanzarían, como sin duda ocurrió.
    Pero una escena aún más horrible nos esperaba, pues al encaminarnos hacia la entrada por donde habíamos visto salir al caballo encontramos los restos de otro corcel y de dos hombres devorados por aquellas salvajes bestias; uno de los infelices era seguramente el que había disparado el tiro que escuchamos, pues una escopeta descargada yacía a su lado. Los lobos habían devorado la cabeza y parte superior de su cuerpo.
    Aquello nos llenó de horror y no supimos qué hacer, hasta que los mismos lobos se encargaron de señalarnos el camino cuando empezaron a reunirse en enormes cantidades al acecho de las nuevas presas; pienso que había no menos de trescientos de ellos. Afortunadamente para nosotros, a poca distancia de la entrada del bosque vimos los troncos de algunos grandes árboles que habían sido hachados en el verano anterior y dejados allí probablemente para transportarlos más tarde.
    Formé mi pequeña tropa en medio de aquellos árboles, ordenándole tender una línea detrás de un gran tronco; les indiqué que desmontaran y se parapetasen en dicho tronco, disponiéndose en triángulo para encarar tres frentes, dejando los caballos a salvo en el centro.
    Así lo hicimos, y justamente a tiempo; porque nunca se vio una carga más furiosa que la que nos dieron aquellos lobos allí mismo. Se abalanzaron sobre nosotros con furiosos gruñidos, saltando sobre el tronco que formaba nuestro parapeto como si la misma madera fuese su presa. Pensamos que su furia era debida a que alcanzaban a ver nuestros caballos, que constituían su principal objetivo. Ordené a mis hombres que tirasen alternativamente, y con tanta precisión lo hicieron que en la primera descarga mataron una gran cantidad de lobos; pero resultó necesario sostener una constante fusilería porque aquellas fieras volvían a la carga como demonios, los de atrás empujando a los que venían en primera fila.
    Cuando hubimos disparado la segunda andanada observamos que vacilaban algo, y creímos que tal vez retrocederían; pero aquello duró solo un instante porque otros se abalanzaron al asalto, de modo que hicimos dos descargas de pistola; pienso que en esas cuatro descargas alcanzamos a matar diecisiete o dieciocho lobos, hiriendo a doble número de ellos, y sin embargo volvían furiosamente al ataque.
    No quería yo gastar tan pronto nuestras últimas balas, de manera que llamé a mi sirviente (no a Viernes, que estaba ocupado en renovar con prodigiosa habilidad las cargas de mi escopeta y la suya) y dándole un frasco de pólvora le ordené que formara un ancho reguero a lo largo del tronco que nos servía de parapeto. Así lo hizo, y apenas había tenido tiempo de ponerse a salvo cuando los lobos volvieron al asalto y algunos treparon sobre el tronco en el preciso momento en que yo aplicaba a la pólvora la llave de una pistola descargada y tiraba del gatillo. La pólvora se inflamó instantáneamente, y aquellos que estaban sobre el tronco se quemaron mientras seis o siete, por huir del fuego, caían o más bien saltaban sobre nosotros. Los matamos de inmediato, y el resto se mostró tan aterrado con el resplandor, aún más vivo en la oscuridad de la noche, que retrocedieron paso a paso. Ordené entonces descargar una última andanada, y después de eso prorrumpimos en grandes gritos. Los lobos, ya aterrados, nos dieron la espalda y huyeron, aprovechando nosotros para caer sobre los que quedaban heridos en el suelo y rematarlos a golpes de espada. Aquello salió tal como lo esperábamos, porque los aullidos y quejidos de los animales que matábamos fueron claramente escuchados por sus compañeros que se apresuraron a escapar a toda carrera.
    En total habíamos dado cuenta de unos sesenta lobos, y de haber sido de día hubiésemos matado aún más. Ya despejado el campo de batalla nos apresuramos a reanudar la marcha, porque aún nos quedaba una legua larga que recorrer. Oímos a las salvajes bestias aullar en los bosques repetidas veces, y en alguna oportunidad creímos ver algunas, pero como la nieve nos cegaba no tuvimos la seguridad de que fuesen lobos.
    Una hora después arribamos al pueblo donde pernoctaríamos, y allí encontramos un gran pánico y a todo el mundo en armas; la noche anterior los lobos y algunos osos habían asaltado el villorrio provocando un espanto general, y los pobladores se veían obligados a mantener constante vigilancia, en especial durante la noche, para proteger al ganado y como es natural a las gentes.
    Tan enfermo amaneció al día siguiente nuestro guía, con los miembros inflamados a causa de las mordeduras, que nos vimos obligados a dejarlo y contratar un nuevo guía, que nos condujo a Tolosa. Allí encontramos un clima templado, una comarca fértil y placentera, sin nieve, lobos o nada parecido. Cuando narramos nuestra aventura en Tolosa nos dijeron que lo ocurrido era muy frecuente en los grandes bosques al pie de las montañas, especialmente cuando la nieve cubre el suelo; nos preguntaron con sorpresa quién era el guía que se había atrevido a traernos por ese camino en una época tan rigurosa, asegurándonos que habíamos tenido harta suerte de no ser devorados. Cuando les explicamos cómo nos habíamos defendido de los lobos poniendo a los caballos en el centro de nuestras líneas nos lo censuraron mucho, diciéndonos que había cincuenta probabilidades contra una de ser destrozados por los lobos. Parece que es la vista de los caballos los que los torna más furiosos, ya que ellos constituyen su presa preferida. En otras oportunidades temen el simple ruido de un disparo, pero el hambre que los devora sumado a la rabia que esto les produce y la visión de los caballos que ansian devorar, los tornan insensibles al peligro. Nos dijeron que si no hubiese sido por el continuo fuego y la estratagema final de encender un reguero de pólvora lo más probable era que hubiésemos terminado hechos pedazos. Quizá hubiese sido preferible permanecer montados, disparando desde allí, pues los lobos al ver los jinetes en sus corceles no hubieran considerado a estos últimos presa tan fácil; por fin nos aseguraron aquellos hombres que lo mejor hubiese sido quedarnos todos juntos y abandonar los caballos a los lobos, quienes los hubieran devorado permitiéndonos salir sin peligro del bosque, en especial siendo tantos y tan bien armados.
    Por lo que a mí respecta, nunca me sentí tan expuesto al peligro como en aquella ocasión. Al ver más de trescientos lobos precipitándose rugiendo y con las fauces abiertas sobre nosotros, y apenas contando con un débil parapeto para defendernos, me había considerado ya muerto; de lo que estoy seguro es de que jamás volveré a cruzar aquellas montañas, y preferiría hacer mil leguas por mar aunque tuviese la seguridad de ser sorprendido por una tormenta cada semana.
    Mi viaje por Francia no ofreció nada de extraordinario, sino esas incidencias que otros viajeros han narrado mucho mejor de lo que yo podría hacerlo. Fui de Tolosa a París, y luego de breve plazo me trasladé a Calais, donde felizmente hice la travesía hasta Dover, llegando a destino el 14 de enero, después de haber sufrido los rigores de una muy fría estación.
    Me encontraba ahora al fin de mis viajes, y en poco tiempo había logrado reunir mi nueva fortuna, ya que las letras de cambio que traje conmigo me fueron pagadas inmediatamente.
    Mi principal y mejor consejero era la anciana viuda que, llena de agradecimiento por el dinero que le había enviado, no reparaba en fatigas ni preocupaciones por serme útil. Tanta confianza depositaba yo en ella, que me sentía absolutamente tranquilo por la seguridad de mis bienes, ya que la intachable integridad de aquella excelente mujer se conservó invariable desde el principio hasta el fin.
    Pensé, pues, en dejar mi fortuna al cuidado de la anciana y volverme a Lisboa, de donde podría embarcarme rumbo al Brasil. Un escrúpulo religioso se presentó sin embargo en mis pensamientos; había dudado alguna vez sobre la religión romana mientras estuve fuera de mi patria, y especialmente en la soledad de la isla, pero sabía bien que no existía posibilidad de llegar al Brasil y mucho menos de establecerme en él si no me resolvía antes a abrazar sin reserva alguna la religión católica, salvo que, dispuesto a sobrellevarlo todo por mis principios, me convirtiera en un mártir religioso y muriera en la Inquisición. Me resolví por lo tanto a quedarme en mi tierra y, de serme posible llevarlo a cabo ventajosamente, vender mi propiedad.
    A tal fin escribí a mi viejo amigo de Lisboa, que me contestó diciéndome que le sería fácil realizar la venta, pero que le parecía conveniente pedir mi venia para ofrecer la plantación en mi nombre a los dos comerciantes, herederos de mis antiguos apoderados, que vivían en el Brasil y eran naturalmente buenos conocedores del valor de esas tierras; por otra parte, aquellos dos hombres eran riquísimos, de manera que él confiaba que les placería adquirir la propiedad, por la cual pensaba que podría yo obtener unas cuatro mil o cinco mil piezas de a ocho.
    Le contesté concediéndole la autorización para hacer la oferta, y unos ocho meses más tarde, cuando volvió el navío, recibí una carta informándome que la venta había sido aceptada y que los comerciantes remitían treinta y tres mil piezas de a ocho a un corresponsal de Lisboa para que me pagara el valor de la plantación.
    Firmé entonces el documento de venta que me enviaban de Lisboa, y lo envié al capitán, que me devolvió letras de cambio por treinta y dos mil ochocientas piezas de a ocho, reservando una renta de cien moidores anuales para él mientras viviera, y de cincuenta para su hijo, tal como yo se lo prometiera y que le serían entregadas del producto de la plantación según se estipuló.
    Y así he narrado la primera parte de una vida aventurera, una vida señalada por la Providencia y de una diversidad tan extraordinaria como pocas podría mostrar el mundo; principiando alocadamente para terminar con una felicidad a la que ninguno de los acontecimientos anteriores me daba derecho a esperar.
    Cualquiera pensaría que encontrándome de tal modo favorecido por la fortuna estaba muy lejos de correr nuevos azares, y en realidad así hubiera sido a no mediar ciertas circunstancias. En primer término estaba yo habituado a una existencia errante, no tenía familia ni muchas relaciones, y aunque rico no me sentía mayormente vinculado. Cierto que había vendido mi plantación del Brasil, pero no me era posible olvidar ese país y a cada instante sentía el deseo de lanzarme otra vez a viajar. Especialmente me costaba resistir a la tentación de ver de nuevo mi isla y saber silos pobres españoles habían logrado llegar a ella y cómo los trataban los tres picaros que dejé en tierra.
    Mi excelente amiga la viuda me disuadió con todas sus fuerzas de la empresa, y tanto calor puso en sus argumentos que logró impedir durante siete años que me embarcara, tiempo en el cual tomé a mi cargo a mis dos sobrinos, hijos de mi difunto hermano. Al mayor, que poseía algunos bienes, lo eduqué como a un caballero y agregué una buena cantidad a sus rentas para que recibiera esa fortuna después de mi muerte. Al segundo lo puse al cuidado de un capitán de navío, y cuando cinco años más tarde vi que era un sensato, valiente y emprendedor muchacho, le confié un barco y lo envié al mar. Este mismo muchacho fue el que más tarde me envolvió, viejo como yo estaba, en nuevas aventuras.
    Entretanto me radiqué allí, principiando por contraer matrimonio muy ventajosamente; de esa unión nacieron tres hijos, dos varones y una niña, pero mi esposa falleció más tarde, y cuando mi sobrino llegó a casa después de un afortunado viaje a España, mi inclinación aventurera sumada a sus requerimientos pudieron más que la prudencia y me llevaron a emprender viaje a bordo de su barco, en carácter de comerciante particular con destino a las Indias Orientales. Esto sucedía en el año 1694.
    En el transcurso de ese viaje visité mi nueva colonia en la isla, vi a mis sucesores, los españoles, oyendo de sus labios todo el relato de sus vidas, así como de los villanos que allí dejara; cómo al comienzo insultaron a los pobres españoles, y se pusieron luego de acuerdo para separarse y volver a unirse, y así hasta que al fin los españoles se vieron precisados a emplear la violencia con ellos; cómo quedaron sometidos y con cuánta justicia los trataron los españoles. Un relato, en suma, que de entrar en detalles resultaría tan maravilloso como el mío, en especial en lo que se refiere a sus batallas con los caribes, que desembarcaron repetidas veces en la isla, sin contar los adelantos que aquéllos hicieron en esas tierras; asimismo sería interesante referir cómo un grupo intentó llegar al continente del que volvió trayendo once hombres y cinco mujeres prisioneros, a causa de lo cual encontré a mi llegada cerca de veinte chiquillos en la isla.
    Allí estuve unos veinte días, dejándoles toda clase de provisiones necesarias, especialmente armas, pólvora, balas, ropas y herramientas, así como dos trabajadores que había llevado conmigo de Inglaterra: un carpintero y un herrero.
    Aparte de eso dividí la isla en parcelas que les confié, reservándome la propiedad total y entregando a cada uno la porción acorde a su persona y conveniencia; por fin, luego de dejar todo arreglado y comprometerlos a que no abandonaran la isla, me embarqué nuevamente.
    De allí fui al Brasil, desde donde envié un barco comprado por mí con más habitantes para la isla; entre ellos, y aparte de diversas cosas necesarias, iban siete mujeres que traté de elegir aptas para ocuparse de las faenas de la isla, y con las que podrían casarse quienes lo quisieran. En cuanto a los ingleses, les prometí enviarles algunas mujeres de Inglaterra junto con un cargamento de provisiones, siempre que se dedicaran a ser plantadores, como así lo hicieron más tarde. Por cierto que una vez dominados aquellos hombres demostraron ser honrados y trabajadores, y poseían sus propiedades aparte. Les hice llegar desde el Brasil cinco vacas, tres de ellas con terneros, algunas ovejas y también cerdos, todos los cuales estaban considerablemente multiplicados cuando volví a mi posesión.
    A todo esto habría que agregar la historia de cómo trescientos caribes invadieron la isla, arruinando las plantaciones y librando dos veces grandes batallas, en las cuales los colonos fueron al principio derrotados, perdiendo tres hombres, hasta que una tormenta destruyó las canoas enemigas, y el hambre y las luchas acabaron con la mayor parte de los caribes, permitiendo por fin la reconquista de las plantaciones, que fueron renovadas, y junto a las cuales todavía vivían los colonos... Todo eso, repito, con los sorprendentes episodios de otras nuevas aventuras mías durante diez años, podrán tal vez constituir más adelante otra narración.
\end{document}

