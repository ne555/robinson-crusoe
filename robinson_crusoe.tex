\documentclass{novela}
\title{Robinson Crusoe}
\author{Daniel Defoe}
\AnotherAuthor{Traducción: Julio Cortázar}
\date{}

\begin{document}
	\maketitle
	\tableofcontents
	%\part{}
	\chapter{Primeras aventuras de Robinson}

    Nací en el año 1632 en la ciudad de York, de buena familia aunque no del país, pues mi padre, oriundo de Bremen, se había dedicado al comercio en Hull, donde logró una buena posición. Desde entonces, y luego de abandonar su trabajo, se radicó en York, donde casó con mí madre; ésta pertenecía a los Robinson, una distinguida familia de la región, y de ahí que yo fuera llamado Robinson Kreutznaer, aunque por la habitual corrupción de voces en Inglaterra se nos llama Crusoe, nombre que nosotros mismos nos damos y escribimos y con el cual me han conocido siempre mis compañeros.

    Siendo el tercero de los hijos, y no preparado para ninguna carrera, mi cabeza empezó a llenarse temprano de desordenados pensamientos. Mi anciano padre me había dado la mejor educación que el hogar y una escuela común pueden proveer, y me destinaba a la abogacía; pero yo no ansiaba otra cosa que navegar y mi inclinación a los viajes me hizo resistir tan fuertemente la voluntad y las órdenes de mi padre, así como las persuasiones de mi madre y mis amigos, que se hubiera dicho que existía algo de fatal en esa tendencia que me arrastraba directamente hacia un destino miserable.

    Mi padre, hombre prudente y serio, trató con sus excelentes consejos de hacerme abandonar el intento que había adivinado en mí. Una mañana me llamó a su habitación, donde lo retenía la gota, para hacerme cordiales advertencias sobre mis proyectos. Con su tono más afectuoso me rogó que no cometiera una chiquillada y me precipitara a desdichas que la naturaleza y mi posición en la vida parecían propicias a evitarme; no tenía yo necesidad de ganarme el pan puesto que él me ayudaría con su impulso a obtener la situación acomodada que me había destinado; en fin, si no lograba una posición en el mundo sería sólo por culpa mía o del destino, sin que tuviera él que rendir cuentas de ello, ya que cumplía con su deber al prevenirme contra actitudes que sólo redundarían en mi desgracia; en una palabra, me aseguró que haría mucho por mí si me quedaba en casa, pero que no quería tener participación alguna en mis desventuras alentándome a partir. Para terminar me señaló el ejemplo de mi hermano mayor, con el cual había empleado el mismo género de persuasiones a fin de evitar que fuera a las guerras de Flandes, no pudiendo sin embargo impedir que sus juveniles impulsos lo llevaran a la lucha donde encontró la muerte. Me aseguró que no dejaría de rogar por mí, pero que se aventuraba a decirme que si me dejaba arrastrar por mi impulso Dios no me acompañaría, quedándome sobrado tiempo para lamentar haber desoído los consejos paternales y ello cuando ya nadie pudiera acompañarme en mi arrepentimiento.

    Sus palabras me afectaron profundamente, como es natural, y resolví abandonar toda idea de viajes estableciéndome en casa de acuerdo con la voluntad paterna. Mas, ¡ay!, muy pocos días disiparon los buenos propósitos, y unas semanas después me decidí a evitar lo que consideraba importunidades de mi padre yéndome de su lado. Sin embargo, no permití que el calor de mi resolución me arrastrara. Y acudiendo a mi madre un día en que la creí de mejor humor que otras veces le confié que mis deseos de conocer el mundo eran tan irresistibles que jamás podría dedicarme a cosa alguna que me lo impidiera, y agregué que mi padre haría mejor en darme su consentimiento que obligarme a partir sin él. Ya tenía yo dieciocho años, edad demasiado avanzada para entrar de aprendiz en cualquier comercio o como pasante en un bufete, y si me forzaban a ello estaba seguro de escapar de mi amo a toda costa y lanzarme al mar. Por fin le aseguré que si convencía a mi padre de que me dejara partir y a mi regreso encontraba yo que el viaje no me había gustado, le prometía no volver a intentarlo jamás y rescatar, con todo celo y diligencia, el tiempo perdido.

    Todo esto sólo sirvió para encolerizar a mi madre. Me dijo que era vano hablar a mi padre del asunto, que lo sabía demasiado seguro de cuál era el camino provechoso para dar un consentimiento que sólo sería mi desgracia, y se maravilló de que pudiera insistir después de la conversación que había tenido con él y las tiernas y bondadosas frases que había empleado conmigo; en fin, si yo estaba dispuesto a perderme no había manera de impedirlo, pero jamás mi intención lograría el consentimiento de ambos; por su parte no estaba dispuesta a colaborar en mi ruina y nunca podría decirse de ella que había obrado contra la voluntad de su esposo.

    Aunque se cuidó de decir todo esto a mi padre, vine a saber más tarde que le contó lo ocurrido y que el anciano, tras de mostrar gran preocupación, dijo suspirando:

    —El muchacho sería dichoso si se quedara en casa, pero si se lanza a viajar será el hombre más infeliz que haya pisado la tierra. No puedo darle mi consentimiento.

    Sólo un año después de todo esto dejé mi casa, aunque entretanto me mantuve sordo a toda proposición que se me hizo de dedicarme al comercio, y discutía frecuentemente con mis padres sobre lo que yo consideraba su empecinamiento contra mis más ardientes inclinaciones. Pero un día, hallándome casualmente en Hull y sin la menor intención de escaparme en esa oportunidad, encontré un amigo que se embarcaba para Londres en el barco de su padre y que me instó a que lo acompañara, valiéndose del cebo habitualmente empleado por los marinos, esto es, que el pasaje no me costaría nada. Sin consultar a mis padres ni comunicarles mi partida, dejándolos que se enteraran como pudiesen; sin pedir la bendición de Dios ni la de mi padre y sin cuidado alguno de las circunstancias y las consecuencias de mi acción, en un día aciago como Dios sabe, el primero de septiembre de 1651 me embarqué en aquel navío rumbo a Londres. No creo que las desgracias de ningún muchacho aventurero hayan comenzado tan pronto y durado tanto. Apenas habíamos salido del Humber cuando se desató el viento y las olas empezaron a encresparse horriblemente; yo, que jamás había estado en el mar, sufrí a la vez el padecimiento del cuerpo y el terror del alma. Me puse a pensar seriamente en lo que había hecho, y con qué justicia me castigaba el cielo por mi perversa conducta al abandonar la casa de mi padre y mi deber.

    Entretanto la tormenta crecía y el mar, aún desconocido para mí, parecía levantarse, aunque nunca en la forma en que lo vi más adelante; no, nunca como lo vi unos días después. Pero entonces bastaba para impresionar a un joven marino que no tenía noción alguna al respecto. Me parecía que cada ola iba a tragarnos, y que cada vez que el barco se hundía, en lo que a mí me daba la impresión de ser el fondo del mar, jamás volvería a surgir a la superficie. En tal estado de terror hice solemnes promesas y adopté la resolución de que si Dios llevaba su bondad a perdonarme la vida y me permitía desembarcar a salvo, iría directamente a la casa de mis padres para no volver a pisar la cubierta de una nave en lo que me quedara de vida. Prometí también que seguiría el consejo paterno sin precipitarme nunca más en tan miserables andanzas; veía claramente ahora la justeza de sus palabras acerca de una cómoda medianía en la vida, cuan fácil y confortable había transcurrido para él la existencia, lejos de toda tempestad en el mar y conflicto en la tierra; y decidí volver, como el hijo pródigo, a casa de mis padres.

    Mis prudentes y sosegados pensamientos duraron lo que la tormenta y hasta un poco más; pero al día siguiente el viento había amainado, el mar estaba menos revuelto y yo comencé a habituarme a ambos. No obstante me mantuve serio todo el día, a lo que hay que sumar un resto de mareo, pero hacia la tarde el tiempo aclaró completamente, el viento cesó en absoluto y tuvimos un hermoso crepúsculo. Con igual claridad que al ponerse se levantó el sol a la siguiente mañana; soplaba apenas una brisa, el mar estaba terso y el sol, brillando sobre las aguas, componía el más hermoso de los espectáculos que me fuera dado ver.

    Habiendo dormido profundamente me sentía ya libre del mareo, y lleno de ánimo miraba maravillado el mar tan terrible el día anterior y capaz de mostrarse tan sereno y agradable muy poco después. Entonces, como para impedir que continuaran mis buenas resoluciones, el camarada que me había impulsado a embarcarme se me acercó y me dijo, palmeándome el hombro:

    —Y bien, Bob... ¿cómo lo has pasado? Apuesto a que te diste un buen susto anoche, y eso que no sopló más que una ráfaga.

    — ¿Le llamas ráfaga? —exclamé—. ¡Pero si fue una terrible tormenta!

    — ¡Tormenta! —dijo mi amigo—. ¿Le llamas tormenta a eso, gran tonto? ¡Pero si no fue nada! Con un buen barco y mar abierto no nos preocupamos por un viento como ése. Es que tú eres marino de agua dulce, Bob. Ven, apuremos un jarro de ponche y nos olvidaremos de todo. ¿No ves qué hermoso tiempo hace ahora?

    Para abreviar esta lamentable parte de mi relato, diré que seguimos el camino de todos los marinos; el ponche fue servido, yo me embriagué con él y en el desorden de aquella noche abandoné todo arrepentimiento, mis reflexiones sobre el pasado y mis resoluciones acerca del futuro. En algunos momentos de meditación, empero, aquellos pensamientos parecían esforzarse por retornar a mí, pero me apresuraba a rechazarlos y me salía de ellos como de una enfermedad. Así, dedicándome a beber y a alternar con los camaradas, pronto dominé aquellos ataques —como yo los llamaba— y en cinco o seis días logré la más completa victoria sobre la conciencia que pudiera desear un muchacho resuelto a no escucharla. Pero otra prueba me esperaba, y la Providencia, tal como lo hace en casos así, resolvió dejarme esta vez sin la menor excusa en mi futura conducta; porque si el primer episodio podía no parecerme una advertencia, el siguiente fue tal que el peor y más empedernido miserable entre nosotros hubiera admitido a la vez el peligro y la gracia.

    Al sexto día de navegación entramos en la rada de Yarmouth; con viento contrario y tiempo sereno, habíamos avanzado muy poco desde la tormenta. Nos vimos obligados a anclar en la rada y quedarnos allí, mientras el viento soplaba continuamente del sudoeste, por espacio de siete u ocho días, durante los cuales muchos barcos provenientes de Newcastle entraron en la rada, puerto común donde los navíos podían aguardar viento favorable para remontar el río.

    Sin embargo, no hubiéramos permanecido tanto tiempo allí sin remontar el río de no levantarse un viento que, entre el cuarto y quinto día, empezó a soplar con furia. Con todo, aquellas radas eran consideradas tan seguras como un puerto y estábamos muy bien y sólidamente anclados, por lo cual nuestros hombres no se preocupaban, en un todo ajenos al peligro, y pasaban el tiempo en diversiones y descanso como todo marino. Pero en la mañana del octavo día el viento arreció, y fue necesario que toda la tripulación se lanzara a calar los masteleros y aligerar lo bastante para que el buque se mantuviera fondeado lo mejor posible. A mediodía creció el mar, y el castillo de proa se hundía mientras las olas barrían la cubierta, al extremo de que llegamos a creer que el ancla se había cortado y el capitán mandó echar el ancla de esperanza, con lo cual el barco se mantuvo con dos anclas y los cables tendidos hasta las bitas.

    Esta vez era verdaderamente un terrible temporal, y yo comencé a ver señales de espanto hasta en el rostro de los marinos. El capitán atendía las maniobras para preservar el barco, pero mientras entraba y salía de su cabina y pasaba cerca de mí le oí decir varias veces:

    — ¡Dios se apiade de nosotros, nos ahogaremos todos, estamos perdidos!

    Durante los primeros momentos, yo permanecí en mi camarote de proa como petrificado, y no podría describir lo que pasaba por mí. Me dolía recordar mi primer arrepentimiento, del que aparentemente me había sido tan fácil librarme y contra el cual me había endurecido; pensaba que no había peligro de muerte y que el temporal amainaría como el otro. Pero cuando el capitán pasó cerca de mí y le oí decir que estábamos todos perdidos me espanté horriblemente y levantándome de mi cucheta me asomé fuera. Jamás había visto un espectáculo tan espantoso; el mar se hinchaba como si fueran montañas y nos barría a cada instante; cuanto veían mis ojos en torno era desolación. En dos barcos anclados cerca de nosotros habían cortado los mástiles por exceso de arboladura, y nuestros marineros gritaban que un navío fondeado a una milla del nuestro acababa de naufragar. Otros dos barcos que habían perdido sus anclas eran arrebatados de la rada hacia el mar, librados a su suerte. Los barcos livianos resistían mejor el embate, pero dos o tres de ellos pasaron desmantelados frente a nosotros, huyendo con sólo la botavara al viento.

    Hacia la tarde, el piloto y el contramaestre pidieron al capitán que les dejara cortar el palo de trinquete. Aunque se negó al principio, las protestas del contramaestre que aseguraba que el buque se hundiría en caso contrario lo llevaron a consentir; pero cuando cayó el mástil se vio que el palo mayor quedaba suelto y sacudía de tal manera el barco que fue necesario cortarlo a su vez y dejar la cubierta arrasada.

    Cualquiera puede inferir en qué estado de ánimo estaría yo a todo esto, siendo un novato en el mar y habiendo pasado poco antes tanto miedo por una simple ráfaga. Pero —sí me es posible describir ahora los pensamientos que me asaltaban entonces— recuerdo que sentía diez veces más miedo por haber abominado de mis anteriores resoluciones y recaído en los malos designios que por la idea de la muerte. Eso, agregado al espanto de la tormenta, me ocasionó un estado de ánimo que jamás podría narrar. Y sin embargo lo peor no había sobrevenido aún; el temporal continuaba con tal furia que los mismos marineros aseguraban no haber visto jamás uno semejante. Teníamos un buen barco, pero excesivamente cargado y calaba tanto que los marineros esperaban verlo irse a pique a cada momento. El único alivio que se me brindó entonces fue ignorar el sentido de la expresión «irse a pique», hasta que lo supe más tarde. Pude entonces ver en medio de la furia de la tormenta algo que no es frecuente: al capitán, al contramaestre y algunos otros más cuerdos que el resto, elevando sus ruegos mientras el navío parecía zozobrar a cada instante. A mitad de la noche, y para colmo de nuestras desventuras, uno de los marineros que descendiera de intento para observar la cala volvió gritando que el barco hacía agua; otro hombre aseguró que ya había cuatro pies en la bodega. De inmediato se llamó a todos a las bombas, y cuando oí esa palabra el corazón pareció dejar de latirme en el pecho y caí de espaldas sobre la cucheta donde había estado sentado. Pronto, sin embargo, los marineros vinieron a decirme que si hasta entonces no había sido capaz de ayudar en nada, bien podía hacerlo en una bomba como cualquier otro. Me levanté y obedecí poniendo todas mis fuerzas en el trabajo. Entretanto el capitán había divisado algunos barcos carboneros que, incapaces de resistir anclados la tormenta, se veían obligados a salir de la rada y lanzarse al mar; como habían de pasar cerca de nosotros, ordenó el capitán disparar un cañonazo en demanda de socorro. Yo no sabía lo que eso significaba y me sorprendí tanto que me pareció que el barco se había partido en dos o que acababa de ocurrir alguna otra cosa tremenda. Para decirlo en una palabra, me desmayé. En aquella hora cada uno tenía su propia vida que cuidar, y naturalmente nadie se preocupó por lo que pudiera haberme ocurrido; otro marinero que vino a la bomba me hizo a un costado con el pie, creyendo seguramente que había muerto, y pasó un largo rato antes de que recobrara el sentido.

    Trabajábamos más y más, pero el agua crecía en la bodega y era evidente que terminaríamos por hundirnos; aunque la tormenta había decrecido un poco no parecía probable que pudiéramos sostenernos a flote hasta entrar en puerto, por lo cual el capitán siguió disparando cañonazos. Un barco pequeño que estaba anclado justamente delante de nosotros osó enviar un bote en nuestro auxilio. Fue harto afortunado que el bote pudiera acercarse, pero nos resultaba imposible transbordar a él así como al bote mantenerse al costado, hasta que los remeros, con un supremo esfuerzo en el que exponían sus vidas para salvar las nuestras, consiguieron alcanzar el cable que por la popa les tiramos con una boya al extremo, y después de infinitas dificultades los remolcamos hasta nuestra popa y pudimos así transbordar. No era su propósito volver al navío de donde partieran, de modo que estuvimos de acuerdo en dejarnos llevar por el viento y solamente encaminar en lo posible el bote hacia tierra firme; nuestro capitán, por su parte, aseguró que si la embarcación se averiaba al tocar la costa, él indemnizaría a su dueño y con eso, remando algunos y otros dirigiendo el rumbo, fuimos hacia el norte sesgando la costa casi a la altura de Winterton Ness.

    Mientras los hombres se inclinaban sobre los remos tratando de acercar el bote a tierra, y en los momentos en que éste, al montar sobre una ola, nos permitía la visión de la costa, podíamos distinguir una gran cantidad de gentes corriendo por ella con intención de ayudarnos. Pero avanzábamos con gran lentitud y no pudimos alcanzar la costa hasta más allá del faro de Winterton, donde hace una entrada hacia el oeste en dirección a Cromer y, por tanto, la misma tierra protege al mar contra la violencia del viento. Allí desembarcamos no sin bastantes dificultades, y fuimos a pie hacia Yarmouth donde nuestra desgracia fue aliviada por la generosidad de todos, desde los magistrados de la ciudad que nos dieron buen alojamiento hasta los comerciantes y propietarios de barcos, que nos facilitaron suficiente dinero para ir a Londres o retornar a Hull, según nuestra voluntad.

    Si hubiera tenido entonces bastante sensatez para volver a Hull y a mi hogar, habría encontrado allí la felicidad, y mi padre, como un emblema de la parábola de Nuestro Señor, habría matado para mí el ternero cebado; en verdad, al enterarse de la desgracia ocurrida en la rada de Yarmouth al barco en el cual yo había huido, pasó largo tiempo inquieto hasta asegurarse de que no me había ahogado.

    Pero mi mala estrella seguía impulsándome con una fuerza que nada podía resistir, y aunque muchas veces me sentí agobiado por el pensamiento y la voluntad de volver a casa, no encontré fuerza suficiente para hacerlo. Ignoro qué nombre debo dar a esto, ni pretendo que se trate de una secreta predestinación que nos lleva a ser instrumentos de nuestra propia ruina, aun cuando la estemos viendo y corramos hacia ella con los ojos abiertos. Por cierto que sólo una desdicha inevitablemente destinada a mí, y de la cual me era imposible escapar, podía haberme arrastrado contra todo sensato razonamiento y las persuasiones de mi propia meditación, máxime teniendo en cuenta las dos evidentes advertencias que acababa de recibir en mi primera tentativa.

    El camarada que me había empujado en mi decisión, y que era el hijo del capitán, parecía ahora mucho menos animoso que yo. La primera vez que me habló en Yarmouth, es decir, dos o tres días más tarde, porque nos alojábamos en lugares distintos, me dio la impresión de que estaba cambiado, y luego de preguntarme con aire melancólico y moviendo la cabeza cómo estaba mi salud, se volvió hacia su padre y le dijo quién era yo y cómo había intentado ese viaje a manera de prueba para más distantes expediciones. Su padre se volvió a mí con un aire a la vez grave y afectuoso, para decirme:

    —Joven, no os embarquéis nunca más. Lo que ha ocurrido debe bastaros como indudable signo de que no estáis destinado a ser marino. Estad seguro de que si no volvéis al hogar, en cualquier sitio adonde vayáis encontraréis desastres y decepciones, hasta que las palabras de vuestro padre se hayan cumplido en vos.

    Nos separamos al rato, sin que yo le hubiera contestado gran cosa, y no sé qué fue más tarde de él. Por lo que a mí respecta, dueño de algún dinero, me fui por tierra a Londres y allí, lo mismo que en el curso del viaje, sostuve duras luchas conmigo mismo para decidir cuál debería ser mi camino, si volvería a casa o al mar. De ir a casa me detenía la vergüenza, opuesta a mis mejores impulsos; se me ocurría que todos iban a reírse de mí, que no sólo me humillaría presentarme ante mis padres sino a los vecinos y amigos; y puedo decir que desde entonces he observado cuan absurdo e irracional es el carácter de los hombres, en especial en los jóvenes, que los lleva a no avergonzarse de sus faltas y sí de su arrepentimiento, que no se reprochan los actos por los cuales merecen el nombre de insensatos mientras que los humilla el retorno a la verdad que les valdría en cambio la reputación de hombres prudentes.

    Tuve suerte al hallarme a poco de mi llegada a Londres en muy buena compañía, cosa no muy frecuente en jóvenes tan libres y mal encaminados como lo era yo entonces, ya que el diablo no tarda en prepararles sus trampas. En primer lugar conocí al capitán de un barco que venía de Guinea y que, habiendo tenido allá muy buena fortuna, estaba resuelto a volver. Mi conversación, que en aquel entonces no era del todo torpe, le agradó mucho y oyéndome decir que ansiaba conocer el mundo me propuso hacer el viaje con él sin que me costara nada; sería su compañero de mesa y su camarada, sin contar que, llevando alguna cosa conmigo para comerciar, tendría todas las ventajas del intercambio y tal vez eso acrecentara mi decisión.

    Acepté la propuesta y habiéndome hecho muy amigo del capitán, que era hombre simple y honesto, emprendí viaje con él llevando conmigo una modesta pacotilla que, gracias a la desinteresada probidad de mi compañero aumentó considerablemente. Había comprado por valor de cuarenta libras las baratijas y chucherías que el capitán me aconsejaba llevar, y ese dinero fue el producto de la ayuda de algunos parientes con los cuales me mantenía en contacto, de donde infiero que mi padre, o por lo menos mi madre, contribuyeron con ello a mi primera aventura.

    Aquél fue el único viaje que puedo llamar excelente entre todas mis andanzas, y lo debo a la honesta integridad de mi amigo el capitán junto al cual adquirí además un discreto conocimiento de las matemáticas y las reglas de navegación, aprendí a llevar un diario de ruta, calcular la longitud y latitud para determinar la posición del buque y, en resumen, comprender aquellas cosas que deben ser conocidas r por un marino. Es verdad que así como él tenía placer en enseñarme yo lo tenía en aprender; y en realidad aquel viaje hizo de mí a la vez un comerciante y un marino. Traje de regreso cinco libras y nueve onzas de oro en polvo a cambio de mi pacotilla, y ello me reportó en Londres no menos de trescientas libras, terminando de llenarme de ambiciosos proyectos que desde entonces me han traído a la ruina.

	Y con todo, aun en aquel viaje tuve inconvenientes, por ejemplo, una continua enfermedad, producto de la elevada temperatura del clima que me producía calenturas; comerciábamos en la costa, desde los 15{\grado} hasta el mismo ecuador.

    Podía considerarme ya un comerciante de Guinea, y cuando para desdicha mía a poco de desembarcar falleció mi amigo, me resolví a emprender nuevamente el viaje y embarqué en el mismo barco capitaneado ahora por el que había sido piloto en la anterior travesía. Nadie hizo nunca un viaje menos afortunado, pues aunque sólo llevé conmigo cien libras de mi nueva fortuna, dejando las doscientas restantes en manos de la viuda de mi amigo, que las guardó celosamente, las desgracias llovieron sobre mí. La primera ocurrió cuando nuestro barco navegaba hacia las islas Canarias o, mejor, entre aquéllas y la costa africana, pues fuimos sorprendidos una mañana por un corsario turco de Sallee que empezó a perseguirnos con todas las velas desplegadas. De inmediato soltamos cuanto trapo eran capaces de soportar los mástiles, pero nuestra esperanza de ganar distancia se vio pronto desmentida por el avance de los piratas, por lo cual nos dispusimos a la lucha contando con doce cañones contra los dieciocho que tenía el buque pirata. A las tres de la tarde se puso a tiro, pero en vez de soltarnos su andanada por la popa como parecía dispuesto vino sesgando para alcanzarnos más de lleno, permitiéndonos asestarle ocho cañones de ese lado y enviarle una andanada que lo obligó a alejarse, no sin antes responder a nuestro fuego agregando a los cañones una nutrida fusilería de los doscientos hombres que tenía a bordo. Por suerte no habían herido a nadie y nuestros hombres se mantenían a cubierto. Vimos que se preparaba a atacar nuevamente, pero esta vez se aproximó por la otra borda lanzándose al abordaje contra el castillo de proa, donde unos sesenta piratas que consiguieron saltar se precipitaron con hachas y cuchillos a cortar los mástiles y aparejos. Los recibimos con fusilería, atacándolos también con bayonetas y granadas de mano, hasta conseguir despejar por dos veces la cubierta. Pero resumiendo esta triste parte de mi relato, después que nuestro barco quedó desmantelado, con tres marineros muertos y ocho heridos, no tuvimos otro remedio que rendirnos y los piratas nos condujeron prisioneros a Sallee, puerto que pertenecía a los moros.






	\chapter{Cautiverio y evasión}





    El trato que me dieron en Sallee no resultó tan duro como yo había esperado; ni siquiera me llevaron al interior del país con destino a la corte del emperador como les ocurrió a mis compañeros, sino que el capitán pirata me conservó como su parte en el botín, considerándome un esclavo joven y listo y por lo tanto apropiado para esa clase de andanzas.

    Mi nuevo amo me había conducido a su casa, donde yo vivía en la esperanza de que me llevara consigo cuando volviera a embarcarse, confiando que el destino lo hiciera caer tarde o temprano prisionero de algún marino español o portugués y eso me valiera la libertad. Pronto, sin embargo, tuve que abandonar mi esperanza, porque cuando el pirata se embarcó me puso al cuidado del jardín y a cargo del resto de las tareas que son propias de los esclavos; y cuando volvió de su viaje me hizo subir a bordo para que me quedara vigilando el barco. Yo no hacía más que pensar en mi fuga y la manera de llevarla a cabo, pero no se me presentaba la más mínima ocasión y para mayor desgracia no tenía a nadie a quien participar mis intenciones y convencer de que se embarcara conmigo. Así pasaron dos años, en los que mi imaginación no descansó un momento, pero en los cuales jamás tuve oportunidad de utilizar mis ideas.

    Pasados los dos años se presentó una ocasión bastante curiosa que volvió a animar en mí la esperanza de escaparme. Hacía mucho tiempo que mi amo permanecía en su casa sin alistar el barco para hacerse a la mar, según oí, por falta de dinero; dos veces a la semana, cuando el tiempo estaba bueno, acostumbraba salir de pesca en la pinaza del barco. En aquellas ocasiones me llevaba consigo, así como a un joven morisco, para que remáramos; ambos le placíamos mucho, en especial yo por mi habilidad en la pesca, tanto que terminó por enviarme algunas veces con un moro pariente suyo y el joven morisco a fin de que pescáramos para su mesa.

    Aconteció que estando en la pinaza una mañana de mucha calma, se levantó tan espesa niebla que a media legua de la costa no podíamos verla, y remábamos sin saber en qué dirección; así pasamos todo el día y toda la noche hasta que al despuntar la mañana encontramos que habíamos salido al mar en vez de volver a tierra, de la que nos separaban por lo menos dos leguas. Con gran trabajo pudimos retornar, ya que el viento arreciaba y estuvimos en peligro, pero lo que más molestaba era el hambre.

    Nuestro amo, advertido por la aventura, resolvió ser más precavido en el futuro, y disponiendo de la chalupa del buque inglés que había apresado se decidió a no salir de pesca sin llevar una brújula y algunas provisiones, ordenando al carpintero del barco —que era también un esclavo inglés— que le construyera una pequeña cabina en el centro de la chalupa, como las que tienen las falúas, con bastante espacio atrás para dirigir el timón y halar la vela mayor, y delante para que un marinero o dos pudiesen maniobrar el velamen.

    Con esta chalupa salíamos frecuentemente y mi amo no me dejaba nunca en tierra porque apreciaba mi destreza en la pesca. Ocurrió que habiendo invitado a bordo, con intenciones de paseo o de pesca, a dos o tres moros distinguidos, hizo llevar provisiones en cantidad extraordinaria, ordenando que por la noche se cargara la chalupa con todo lo necesario y mandándome que alistara las tres escopetas que había a bordo con las correspondientes balas y pólvora, ya que les agradaba tanto cazar como pescar.

    Hice todo lo que me había indicado y a la mañana siguiente esperaba con la chalupa perfectamente limpia, su bandera y gallardetes enarbolados y todo lo necesario para recibir a los huéspedes, cuando vino mi amo a decirme que sus amigos habían renunciado al paseo a causa de imprevistos negocios, por lo cual me mandaba que saliera con el moro y el muchacho que eran mis acompañantes habituales a pescar para la cena, ya que aquellos amigos comerían en su casa; agregó que tan pronto hubiera pescado lo bastante me apresurara a llevarlo a la casa, todo lo cual me dispuse a ejecutar.

    Fue entonces cuando mis contenidas ansias de libertad me asaltaron con renovada violencia al darme cuenta de que tendría a mi disposición un pequeño barco, y cuando mi amo se alejó me apresuré a proveerme, no para una partida de pesca sino para un viaje; cierto que no sabía hacia dónde iba a encaminar mi rumbo, pero ni siquiera me detuve a pensarlo; cualquier camino que me llevara lejos de allí era mi camino.

    Mi primera medida fue convencer al moro de que necesitábamos embarcar con nosotros algunas provisiones para no sentir hambre durante la pesca, y aduje que no correspondía que tocáramos los alimentos que el amo había almacenado en la chalupa. A él le pareció bien y pronto vino trayendo un gran canasto de galleta o bizcochos y tres tinajas de agua. Yo sabía dónde guardaba mi amo sus licores, encerrados en una caja que, por el aspecto, era indudablemente de fabricación inglesa, sin duda botín de algún navío apresado; mientras el moro estaba en tierra llevé la caja a bordo como para hacer creer que el amo lo había ordenado así anteriormente. Llevé también un gran pedazo de cera que pesaba más de cincuenta libras, un rollo de bramante, una hachuela, una sierra y un martillo, todo lo cual nos sería muy útil más adelante, especialmente la cera para hacer velas. Equipados con todo lo necesario salimos del puerto a pescar, y los guardianes del castillo que defiende el puerto nos conocían tan bien que no nos molestaron, por lo que seguimos más de una milla fuera hasta encontrar sitio donde arriar las velas y principiar la tarea. El viento soplaba del N-NE, y por tanto no me convenía, mientras que viniendo del sur me hubiera llevado con seguridad a la costa española y a la bahía de Cádiz. Pero mi resolución estaba tomada; soplara de donde soplase yo me fugaría de aquel horrible lugar dejando el resta en manos del destino.

    Estuvimos largo rato sin pescar nada, pues cuando yo sentía picar no alzaba el anzuelo, hasta que dije al moro:

    —Este lugar es malo y si nos quedamos en él nuestro amo no será servido como se merece; tenemos que alejarnos más.

    Sin sospechar nada, el moro asintió y se puso a tender las velas mientras yo piloteaba la chalupa hasta una legua más allá donde nos detuvimos como para pescar; entonces, dejando el timón al muchacho, me fui hasta donde estaba el moro y fingiendo inclinarme para levantar algo a sus espaldas lo tomé de las piernas y lo precipité por la borda al mar. Salió inmediatamente a la superficie porque nadaba como un pez y me suplicó lo dejara subir a bordo asegurándome que iría conmigo a cualquier parte. Nadaba tan rápidamente detrás de la chalupa que pronto la hubiera alcanzado, ya que apenas había viento, de modo que corrí a la cabina y tomando una de las escopetas le apunté diciéndole que no le deseaba ningún mal y que si desistía de subir a bordo no tiraría sobre él.

    —Sabes nadar lo bastante como para llegar a tierra —agregué— y el mar está tranquilo, de modo que vuélvete ahora mismo; si insistes en subir a la chalupa te tiraré a la cabeza, porque estoy dispuesto a recuperar mi libertad.

    Oyendo estas palabras giró en el agua y lo vimos volverse hacia la costa, adonde no dudo habrá llegado fácilmente, pues ya he dicho lo bien que nadaba.

    Hubiera preferido tener al moro a mi lado y tirar por la borda al muchacho, pero no me fiaba de aquél. Cuando se hubo alejado me volví hacia mi compañero, que se llamaba Xury y le dije:

    —Xury, si me eres fiel tendrás una gran recompensa; pero si no te golpeas la cara (es decir, si no juraba por Mahoma y la barba de su padre) tendré que tirarte también al agua.

    El muchacho, sonriendo con inocencia, dijo tales palabras y me hizo tales juramentos de que iría conmigo hasta el fin del mundo, que no me quedó ninguna desconfianza.

    Mientras estuvimos al alcance de la mirada del moro, que seguía nadando, mantuve la chalupa al pairo inclinándola más bien a barlovento para que me creyera encaminado hacia la boca del estrecho. Pero tan pronto como oscureció cambié el rumbo y puse proa al sudeste, ligeramente hacia el este para no perder de vista la costa; con buen viento y el mar en calma navegamos tanto que a las tres de la tarde del día siguiente, cuando calculé la posición, deduje que habíamos recorrido no menos de ciento cincuenta millas al sur de Sallee, mucho más allá de los dominios del emperador de Marruecos y probablemente de todo otro imperio, ya que en la costa no se veía a nadie.

    Pero era tal el miedo que me inspiraban los moros y desconfiaba tanto de caer en sus manos que no quise detenerme para bajar a tierra, ni siquiera anclar, sino que aprovechando el buen viento seguimos navegando por espacio de cinco días; entonces el viento cambió al cuadrante sur y como yo sabía que aquello perjudicaba igualmente a todo buque perseguidor, me aventuré a acercarme a la costa y anclamos en la desembocadura de un riacho tan desconocido como la latitud, el país y los habitantes. Por cierto que prefería no ver a nadie, siendo única razón del desembarco la necesidad de agua dulce. Llegamos por la tarde al riacho, decidiendo nadar de noche hasta la costa y explorar los alrededores, pero así que oscureció empezamos a oír tan horribles rugidos, ladridos y aullidos de los animales salvajes que el pobre Xury se moría de miedo y me rogó que no bajase a tierra hasta que viniera el día.

    Yo estaba tan asustado como el pobre muchacho, pero nuestro espanto creció cuando oímos a uno de aquellos enormes animales que venía nadando hacia la chalupa. No alcanzábamos a verlo, pero comprendíamos por sus resoplidos que debía ser un animal enorme y furioso. Xury sostenía que se trataba de un león —lo que acaso era cierto— y me rogaba que levantáramos anclas y huyéramos.

    —No, Xury —le dije—. Podemos soltar el cable con la boya y dejarnos llevar hacia el mar; los animales no osarán nadar tanta distancia.

    Apenas había dicho esto cuando vi al monstruo (fuera lo que fuese) a dos remos de distancia de la chalupa. Venciendo mi sorpresa tomé una de las escopetas de la cabina y tiré sobre él, viéndolo girar de inmediato en el agua y volverse hacia la costa.

    Seria imposible describir los horribles sonidos, el aullar y rugir que se elevó en la costa y desde muy adentro del país como un eco a mi disparo, ruido que probablemente aquellas bestias oían por vez primera. Aquello me convenció de que sería insensato desembarcar de noche, pero también durante el día. Caer en manos de salvajes era tan desastroso como caer en las garras de tigres y leones; ambas cosas nos parecían igualmente funestas.

    Sea lo que fuese, necesitábamos obtener agua de alguna manera, puesto que no teníamos ni una pinta. Pero ¿cómo? Fue entonces que Xury me rogó que lo dejara desembarcar con una de las tinajas para buscar y traerme agua. Le pregunté por qué quería ir él en vez de quedarse esperándome en la chalupa. La respuesta del muchacho me hizo quererlo profundamente desde ese momento.

    —Si hombres salvajes venir —dijo— ellos comerme a mí, vos salvaros.

    —Muy bien, Xury —le contesté—, entonces iremos los dos y si vienen los salvajes los mataremos para que no nos coman.

    Le di un pedazo de galleta y un trago del licor que saqué de la caja ya mencionada, y tras de acercar la chalupa todo lo posible a la costa desembarcamos sin otra defensa que nuestros brazos y dos tinajas para el agua.

    No me atrevía a perder de vista la chalupa por miedo a que los salvajes salieran del río en canoas y la abordaran; entretanto el muchacho había visto un terreno bajo a una milla aproximadamente y corrido hacia él, hasta que de improviso lo vi volver a toda carrera. Pensé que algún salvaje lo perseguía o que había tenido miedo de las fieras, por lo que fui en su ayuda, pero cuando estuvo más cerca vi que traía algo colgando del hombro, un animal que acababa de cazar parecido a una liebre, pero de patas más largas y distinto color. Nos alegramos mucho y su carne nos pareció excelente, aunque la mayor alegría de Xury fue hacerme saber que había encontrado agua potable y ningún salvaje en los alrededores.

    Yo había navegado por aquellas costas y sabía que las islas Canarias así como las de Cabo Verde no podían estar muy distantes. Me faltaban sin embargo instrumentos para calcular la latitud; no recordaba con precisión la de las islas, de manera que no sabía si continuar en una u otra dirección para encontrarlas; salvo esto, hubiera sido muy simple tocar tierra en ellas. Mi esperanza estaba en seguir la línea de la costa hasta las regiones donde comercian los ingleses, y dar con alguno de sus barcos mercantes que nos rescatara de nuestras desdichas.

    Una o dos veces me pareció ver el Pico de Tenerife, la cresta culminante de las montañas de Tenerife en las Canarias, y por dos veces intenté llegar a las islas, pero los vientos contrarios me lo impidieron, así como un mar demasiado agitado para nuestro barquichuelo; entonces me resigné a proseguir el viaje sin perder de vista la costa.

    Muchas veces nos vimos obligados a desembarcar en procura de agua dulce, y recuerdo una ocasión en que anclamos muy temprano al pie de un promontorio bastante alto, esperando que la marea nos llevara aún más adentro. Xury, que tenía mejor vista que yo, me llamó de pronto para decirme que haríamos mejor en levar anclas cuanto antes.

    —Mirad allá —agregó— ese horrible monstruo que duerme en la ladera de la colina.

    Seguí la dirección que me apuntaba y vi ciertamente al monstruo: un enorme león tendido sobre la playa y protegiéndose del sol por una proyección rocosa de la colina.

    —Xury —dije al muchacho—, irás a la tierra y lo matarás.

    Me miró aterrado.

    — ¿Yo matarlo? ¡El comerme de un boca! —exclamó, queriendo significar un bocado.

    No le dije más nada, pero indicándole que se quedara quieto tomé la escopeta más grande, cuyo calibre era casi el de un mosquete, y la cargué con suficiente pólvora y dos pedazos de plomo; metiendo dos balas en otra escopeta, puse en la tercera cinco plomos pequeños. Apunté lo mejor posible con la primera arma, buscando darle en la cabeza, pero como dormía con una pata tapándole parcialmente la nariz los plomos le alcanzaron la rodilla y le rompieron el hueso. Se levantó gruñendo, pero al sentir la pata rota volvió a caer para enderezarse luego sobre tres patas y exhalar el más horroroso rugido que haya escuchado en mi vida. Me sorprendía no haberle acertado en la cabeza, por lo cual le apunté con la segunda escopeta y, aunque se movía de un lado a otro, tuve el placer de verlo desplomarse ya sin rugir, pero todavía luchando en su agonía. Xury, que había recobrado los ánimos, me pidió que lo dejara desembarcar y cuando se lo consentí saltó al agua, con una escopeta en la mano y nadando con la otra hasta llegar junto al león, y apoyándole el caño en la oreja le disparó el tiro de gracia.

    Todo ello nos había divertido un buen rato, pero sin darnos alimentos, tanto que empecé a lamentar haber desperdiciado aquella pólvora y balas en un animal que de nada nos servia. Xury quería conservar algo de él y cuando vino a bordo me pidió permiso para llevar el hacha a tierra.

    — ¿Para qué la quieres, Xury? —pregunté.

    —Yo cortarle cabeza —me contestó. Pero aunque hizo lo posible no pudo cortársela y se conformó con una pata que trajo a bordo y que era monstruosamente grande. Entonces se me ocurrió que la piel del león podía sernos de alguna utilidad y resolvimos desollarlo. Xury fue mucho más hábil que yo en esta tarea que me resultaba muy difícil. Trabajamos el día entero, pero al fin le sacamos la piel y la pusimos sobre el techo de la cabina, donde el sol la secó en un par de días, tras de lo cual me sirvió para dormir sobre ella.

    Nuevamente embarcados, seguimos hacia el sur sin interrupción durante diez o doce días, tratando de ahorrar las provisiones que disminuían rápidamente y bajando a tierra sólo cuando la sed nos obligaba. Mi intención era llegar hasta el río Gambia o Senegal —es decir, a la altura de Cabo Verde— donde confiaba encontrar algún barco europeo; de no tener esa suerte ignoraba qué iba a ser de mí, ya fuera buscando las islas o pereciendo a mano de los negros. Sabia que todos los barcos que navegan de Europa a la costa de Guinea, Brasil o las Indias Orientales, tocan en el Cabo o en aquellas islas; en una palabra, depositaba mi entera suerte en el hecho de encontrar un barco y de lo contrario sólo podía esperar la muerte.

    Mientras trataba de poner en práctica esa intención, y en el transcurso de aquellos diez días, empecé a notar que la tierra estaba habitada; en dos o tres lugares vimos en las playas gentes que nos miraban pasar, advertimos que eran negros y que estaban completamente desnudos. Me inclinaba yo a trabar relación con ellos, pero Xury era mi mejor consejero y repetía:

    —No, no ir, no ir.

    Acerqué sin embargo la chalupa a distancia suficiente para hablar, pero los negros echaron en seguida a correr por la playa. Noté que no llevaban armas, salvo uno que tenía una especie de largo bastón que Xury dijo ser una lanza que aquellos salvajes arrojan con gran puntería y a larga distancia. Me mantuve, pues, alejado, pero traté de entenderme con ellos por signos haciendo aquellas señales que se refieren al acto de comer. Me contestaron a su modo que anclara la chalupa y que me darían alimentos, y mientras yo arriaba la vela y quedaba a la espera, dos de ellos fueron tierra adentro, de donde regresaron a la media hora trayendo consigo dos grandes pedazos de carne seca y grano como el que produce su país.

    Aunque no teníamos idea de lo que podían ser tales alimentos los aceptamos de inmediato, pero el problema estaba en cómo recibirlos, pues ni yo me animaba a desembarcar ni ellos a llegarse hasta la chalupa; pronto vi, sin embargo, que habían encontrado un procedimiento satisfactorio para ambos, ya que dejaron la carne y los granos en la playa, se alejaron a gran distancia y me dieron tiempo de ir a buscarlos, tras lo cual volvieron a acercarse.

    Teníamos, pues, provisiones y agua, y separándonos de aquellos cordiales negros seguimos navegando otros once días aproximadamente sin volver a arrimar a la costa, hasta que un día vi una tierra que penetraba profundamente en el mar a una distancia de cuatro o cinco leguas de donde estábamos; como el día era sereno, dimos una gran bordada para llegar a ella, y por fin, cuando doblamos la punta a unas dos leguas de la costa, distinguimos con toda claridad tierras al otro lado, mirando hacia el mar. Supuse que la tierra más próxima era Cabo Verde y la otra las islas que llevan su mismo nombre. Desgraciadamente estaban a una enorme distancia y no me decidía a lanzarme en su dirección por miedo a que una borrasca me sorprendiera a mitad de camino y sin poder llegar a una ni otra.

    En este dilema me fui a la cabina a pensarlo mejor, dejando a Xury en el timón, cuando repentinamente le oí gritar:

    — ¡Señor, señor, un barco con vela!

    El pobre muchacho estaba mortalmente asustado, temiendo que se tratara de algún navío enviado por el moro para perseguirnos y sin pensar que ya estábamos demasiado lejos de su alcance. Salté de la cabina y conocí de inmediato que el barco era portugués y que se dirigía sin duda a Guinea en procura de negros. Con todo, observando la ruta que seguía, me convencí de que el barco iba a otra parte y no mostraba intenciones de acercarse a tierra, por lo que saqué la chalupa mar afuera, resuelto a hablar con aquellos marinos si estaba a mi alcance.

    Soltando todo el trapo que teníamos, vine a descubrir que no sólo era imposible acercarnos al navío sino que éste se alejaría antes de que me fuera posible hacerle señal alguna; pero mientras yo, después de haber intentado todo lo imaginable, empezaba a desesperar, parece que ellos alcanzaron a ver la chalupa con ayuda de su anteojo descubriendo que se trataba de un bote europeo, por lo cual imaginaron que un barco había naufragado y se apresuraron a arriar velamen para que yo pudiera ganar terreno. Esto me llenó de alegría, y como conservaba la bandera de mi antiguo amo la enarbolé en señal de socorro y disparé un tiro de escopeta, cosas ambas que observaron desde el barco, pues más tarde me dijeron que habían visto el humo aunque no les llegó el ruido del disparo. Tales señales los determinaron a detener el barco y esperarme; tres horas después subía yo a bordo.

    Me hicieron muchas preguntas que no entendí, hablándome en portugués, español y francés, hasta que un marinero natural de Escocia se dirigió a mí y pude explicarle que era inglés y cómo me había fugado de los moros en Sallee, siendo de inmediato muy bien recibido a bordo con todos mis efectos.

    Es fácil de comprender la inmensa alegría que tuve al considerarme librado de tan desdichada situación; de inmediato ofrecí cuanto tenía al capitán como compensación por mi rescate, pero él no quiso aceptar nada y me dijo generosamente que todo lo mío me sería devuelto cuando llegásemos al Brasil.

    —Al salvar vuestra vida —me aseguró— he procedido tal como quisiera ser tratado yo mismo si alguna vez me encontrara en las mismas circunstancias. Además si os llevara a un lugar tan lejano de vuestra patria y os privara de lo que es vuestro, seria como condenaros a perecer de hambre y quitaros así la misma vida que acabo de salvar. No, no, señor inglés, os llevaré allá sin recibir nada, y lo que poseéis os servirá para vivir en el Brasil y pagar el pasaje de retorno.

    Pronto comprendí que sus actos se ajustaban celosamente a sus promesas; ordenó a los marineros que nadie tocara lo mío, lo puso bajo su propia responsabilidad y mandó hacer un inventario que me entregó, donde se incluían hasta las tres tinajas de barro.

    Cuando vio mi chalupa, que era excelente, quiso comprármela para incorporarla a su barco y me preguntó en cuánto estimaba yo su valor. Le contesté que había sido tan generoso conmigo que no me correspondía fijar el precio sino que lo dejaba en sus manos. Me propuso entonces librarme una letra pagadera en el Brasil por valor de ochenta piezas de a ocho, y que si al llegar allí alguien ofrecía más por la chalupa él compensaría la diferencia. Me ofreció también sesenta piezas de a ocho por Xury, pero me desagradaba recibirlas, no porque me preocupara la suerte del muchacho junto al capitán sino porque me dolía vender la libertad de quien tan fielmente me ayudara a lograr la mía. Cuando dije esto al capitán me contestó que era muy justo, pero que para tranquilizarme se comprometía a firmar una obligación por la cual Xury sería libre al cabo de diez años siempre que se hiciera cristiano. Satisfecho con esto, y más cuando el mismo Xury me manifestó su conformidad, se lo cedí.

    Tuvimos buen viaje al Brasil y a los veintidós días llegamos a la bahía de Todos los Santos. Nuevamente me había salvado de la más miserable situación en que puede verse un hombre, y otra vez debía enfrentar el problema de mi futuro destino.






	\chapter{La plantación. El naufragio}





    Nunca estaré bastante agradecido al generoso comportamiento del capitán. Sin aceptar nada por mi pasaje, me dio veinte ducados por una piel de leopardo y cuarenta por la de león, ordenando que todo cuanto tenía yo a bordo me fuera entregado al detalle; me compró aquellas cosas que yo quería vender, como la caja de licores, dos escopetas y lo que quedaba del pedazo de cera con el cual había fabricado muchas velas. En resumen, me encontré en posesión de unas doscientas veinte piezas de a ocho y con esta suma desembarqué en el Brasil.

    No llevaba mucho tiempo allí cuando fui recomendado por el capitán a un hombre de su misma honestidad que poseía un «ingenio», como llaman ellos a una plantación y fábrica de azúcar. Allí viví cierto tiempo, en cuyo transcurso aprendí a plantar y obtener el azúcar, y reparando en la agradable vida que llevaban los colonos y con cuánta facilidad se enriquecían resolví que si obtenía permiso para radicarme entre ellos me dedicaría a las plantaciones, tratando entretanto de recobrar los fondos que había dejado en Londres. Con este fin solicité y obtuve una especie de carta de naturalización y gasté el dinero que poseía en comprar tierra inculta, trazando los planes para una plantación y establecimiento de acuerdo con la cantidad que esperaba recibir de Inglaterra.

    Era mi vecino un portugués de Lisboa, hijo de padres ingleses y de apellido Wells. Como se encontraba en condiciones semejantes a las mías y su plantación era lindera, yo le llamaba vecino y llegamos a ser buenos amigos. Ambos teníamos poco capital y plantábamos para comer, más que para otra cosa; pero poco a poco empezamos a progresar, y nuestras tierras a rendir provecho. El tercer año plantamos tabaco, y a la vez despejamos un gran pedazo de tierra para plantar caña de azúcar al año siguiente. Nos faltaban brazos que nos ayudaran, y fue entonces cuando advertí el error cometido al separarme de Xury.

    Me encontraba ya avanzado en la tarea de mejorar la plantación cuando mi salvador y buen amigo el capitán decidió hacerse a la vela, pues su barco había permanecido tres meses completando el cargamento y alistándose. Cuando le conté lo del pequeño capital que tenía en Londres, me dio este amistoso y sincero consejo:

    —Señor inglés —porque siempre me llamaba así—, si me libráis cartas y poder en debida forma, con orden a la persona que tiene vuestro dinero en Londres para que lo transfiera a quien yo designe, en Lisboa, os lo traeré si Dios quiere a mi regreso en diversos artículos que tengan fácil salida en este país. Como todo lo humano está sujeto a desastres y cambios, os aconsejo que sólo libréis órdenes por cien libras esterlinas, que según me decís es la mitad de vuestro capital; si las cosas resultan bien podréis rescatar el resto en la misma forma, y de lo contrario os quedará siempre esa reserva.

    El consejo era tan sano y amistoso que comprendí que debía seguirlo, de manera que inmediatamente escribí cartas a la dama depositaría de mis fondos y entregué un poder al capitán. Conté a la viuda del capitán inglés todas mis aventuras, la esclavitud, mi fuga y cómo había conocido al capitán portugués; le narré su generoso comportamiento y en qué circunstancias me encontraba en ese momento, agregando las instrucciones necesarias para la transferencia de los fondos. Cuando el capitán llegó a Lisboa hizo que alguno de los comerciantes ingleses allí establecidos enviaran a Londres la orden y además el entero relato de lo que me había ocurrido, de tal modo que la viuda no solamente entregó sin vacilar el dinero sino que de su propio bolsillo envió un presente al capitán portugués, como homenaje a su generoso y humano proceder.

    El corresponsal en Londres invirtió mis cien libras esterlinas en artículos ingleses tal como el capitán se lo había mandado, y los remitió a Lisboa, de donde mi amigo los trajo felizmente al Brasil. Entre aquellas mercancías, y sin que yo las hubiera pedido, pues era aún demasiado inexperto en la plantación para pensar en ello, venían, por encargo del capitán, herramientas, instrumentos y utensilios necesarios para el trabajo, que me fueron de gran utilidad.

    Cuando llegó el cargamento creí que mi fortuna estaba hecha, tanto me maravilló aquello. Mi servicial amigo el capitán había empleado las cinco libras que le regalara la viuda en contratar por seis años un criado que él mismo me trajo, y no quiso aceptar la menor retribución salvo una pequeña cantidad de tabaco que, por ser de mi plantación, logré al fin que aceptara.

    No todo concluyó allí: las mercancías inglesas tales como paños, tejidos y bayetas eran sumamente solicitadas en el país, de modo que pronto las vendí con tal ganancia que puede asegurarse que cuadripliqué el valor de mi primer cargamento, dejando pronto atrás a mi pobre vecino en el progreso de la plantación; lo primero que hice fue comprar un esclavo negro y obtener los servicios de otro criado europeo, fuera del que el capitán me había traído de Lisboa.

    ¡Cuántas veces la excesiva prosperidad es el más seguro medio de precipitarnos en la mayor desgracia! Así ocurrió conmigo. Al año siguiente la plantación me dio gran cosecha y recogí cincuenta fardos de tabaco fuera de la cantidad destinada a cambiar a los vecinos por otros productos. Cada rollo pesaba más de cien libras, y luego de prepararlos convenientemente los dejé en depósito hasta que volviera el convoy de Lisboa. Entretanto, próspero en negocios y riqueza, empecé a dejarme llevar por proyectos y ambiciones superiores a mis medios, fantasías que terminan por ser la ruina de los comerciantes más expertos.

    Podéis imaginar que llevando casi cuatro años en el Brasil y dirigiendo una floreciente plantación, no sólo había aprendido el idioma sino que sostenía relaciones con los demás plantadores y los comerciantes de San Salvador, que era nuestro puerto. En diversas ocasiones les había narrado mis dos viajes a la costa de Guinea, la forma de comerciar con los negros y qué fácil es conseguir no solamente oro en polvo, granos, colmillos de elefante, sino también negros para el servicio de las plantaciones, a cambio de insignificancias como cuentas de vidrio, cuchillos, tijeras, hachuelas, pedazos de cristal y otras chucherías.

    Escuchaban mis narraciones con gran atención, que se acrecentaba más cuando mencionaba yo la forma de comprar negros, ya que en aquel entonces la trata de esclavos no sólo estaba muy restringida sino que existía un monopolio a cargo de los «asientos» o permisos de los reyes de España y Portugal, lo cual hacía que los negros fueran escasos y exageradamente caros.

    Ocurrió que después de haber estado en compañía de algunos comerciantes y plantadores de mi relación hablando de aquellas cosas con mucho detenimiento, vinieron a verme a la mañana siguiente tres de ellos para decirme que habían reflexionado mucho sobre lo que yo les contara y qué tenían una proposición que hacerme, siempre que les guardara el secreto. Me confiaron que estaban dispuestos a fletar un buque a Guinea, ya que teniendo plantaciones al igual que yo, nada les preocupaba tanto como la falta de brazos, pero que como no podían procurárselos, ya que estaba prohibida la venta pública de esclavos negros, intentaban realizar un viaje secreto en busca de ellos, traerlos a tierra sin despertar sospechas y repartirlos entre sus plantaciones. En una palabra, se trataba de saber si aceptaría ir como encomendero en el barco para dirigir la compra de negros, a cambio de lo cual me ofrecían igual participación que la de ellos en el reparto de los esclavos sin contribuir en nada a los gastos del flete.

    Preciso es confesar que aquélla hubiera sido una excelente proposición para cualquiera que no estuviese ya radicado con una plantación próspera a cuidar, crecientes ganancias y un buen capital. Para mí, ya establecido y sin otra tarea que continuar tres o cuatro años más lo que había iniciado, agregando a ellos las cien libras que debían enviarme de Londres; para mí, que en ese momento y con aquella adición poseía no menos de tres o cuatro mil libras esterlinas, en camino a aumentar todavía, el solo pensar en aquel viaje representaba las más descabellada idea que un hombre en tales circunstancias pudiera concebir.

    Pero yo había nacido para ser causa de mi propia desgracia y no pude resistir aquella oferta, como no había logrado impedir mis primeros planes aventureros a pesar de los consejos de mi padre. Les dije que partiría sin vacilar siempre que se encargasen de velar por mi plantación mientras durara mi ausencia y cumplieran mi voluntad en todo si me ocurría una desgracia. Se comprometieron formalmente y lo rubricaron por escrito, tras lo cual hice testamento declarando mi legatario universal al capitán portugués que me había salvado la vida, y dejándole la mitad de mis bienes con la condición expresa de que enviaría la otra mitad a Inglaterra.

    En resumen, tomé todas las medidas para salvaguardar mis propiedades y la plantación. Si hubiera sido capaz de emplear sólo la mitad de esa prudencia en velar por mis verdaderos intereses y meditar serenamente lo que debía o no debía hacer, jamás habría renunciado a una situación tan próspera dejando todo al azar de las circunstancias y lanzándome a un viaje por mar con lo mucho que tiene de azaroso, para no mencionar las razones que yo tenía para prever especiales catástrofes. Pero cediendo a mis impulsos obedecí ciegamente los dictados del capricho y no los de la razón. Cuando estuvo alistado el barco, el cargamento a bordo y todo perfectamente dispuesto por mis socios, me embarqué en un día aciago, el primero de septiembre de 1659, justamente al cumplirse el octavo aniversario de mi abandono del hogar de mi padre en Hull, cuando me rebelé a su autoridad para hacer el tonto a mis expensas.

    Nuestro barco era de unas ciento veinte toneladas, tenía seis cañones y catorce hombres fuera del capitán, su asistente y yo. No llevábamos gran cargamento fuera de las baratijas necesarias para el intercambio con los negros, tales como cuentas de vidrio, trozos de cristal, conchas y diversas chucherías, en especial pequeños espejos, cuchillos, tijeras y hachuelas.

    Nos hicimos a la vela el mismo día en que embarqué, costeando hacia el norte para luego rumbear al África cuando estuviéramos a los diez o doce grados de latitud norte, que era el camino seguido en aquellos tiempos. A los doce días cruzamos la línea y nos encontrábamos, según la última observación que alcanzamos a hacer, a unos siete grados veintidós minutos norte cuando un violento tornado o huracán nos privó completamente de referencias. Empezó a soplar del sudeste, luego del noroeste, hasta fijarse en el cuadrante noreste, de donde nos azotó con tal furia que por espacio de doce días no pudimos hacer otra cosa que dejarnos llevar a la deriva y, arrastrados por su violencia, ser impulsados hacia donde el destino y la fuerza del viento lo quisieran. Sería ocioso decir que en aquellos momentos cada uno de nosotros esperaba ser devorado por el mar, y que nadie guardaba la menor esperanza de salvar su vida.

    Fuera de la furia de la borrasca, tuvimos la desgracia de que uno de los hombres muriera de calenturas y que otro, juntamente con el muchacho asistente, fuera arrebatado por el mar. Hacia el duodécimo día el tiempo mejoró un poco y el capitán pudo hacer una precaria observación, según la cual nos encontrábamos sobre la costa de Guinea o bien sobre la del norte de Brasil, más allá de las bocas del Amazonas y cerca del Orinoco, llamado también Río Grande. Consultó conmigo qué camino deberíamos tomar, puesto que el buque estaba averiado y navegaba difícilmente, por lo cual creía conveniente ganar lo antes posible la costa del Brasil.

    Me negué de plano a esta sugestión, y mirando juntos los mapas de la costa americana descubrimos que no existía región habitada donde pudiéramos hallar socorro hasta entrar en el círculo de las islas Caribes, y por lo tanto pusimos proa hacia las Barbados para alcanzarlas desde alta mar y evitarnos así la entrada de la bahía o Golfo de México; confiábamos en llegar a ellas en unos quince días, ya que de ninguna manera podíamos proseguir viaje a la costa africana sin las reparaciones que el barco necesitaba.

    Decidido esto cambiamos el rumbo y tomamos el de O-NO, tratando de alcanzar alguna de las islas inglesas donde nos auxiliarían; pero nuestro viaje estaba predestinado a ser distinto, pues una segunda tormenta cayó sobre nosotros arrastrándonos hacia el oeste y tan lejos de toda ruta comercial que aun logrando salvarnos de la furia del océano estábamos más próximos a ser devorados por salvajes que volver alguna vez a nuestro país.

    Mientras padecíamos angustiados la furia de los vientos, oímos de mañana gritar « ¡Tierra!» a uno de los marineros. No habíamos acabado de salir de las cabinas para tratar de distinguir a qué regiones habíamos arribado cuando el barco encalló en las arenas y de inmediato el oleaje empezó a azotarlo con tal furia que tuvimos la impresión de que pereceríamos allí mismo y nos refugiamos en los camarotes para guarecernos del agua y las espumas.

    No es fácil para uno que jamás se ha visto en tal situación concebir la angustia que sentíamos en esas circunstancias. Ignorábamos dónde habíamos encallado, si era el continente o una isla, si habitada o desierta; y como el viento seguía azotando, bien que con menos fuerza que al comienzo, no nos cabía duda de que el barco iba a destrozarse en contados minutos a menos que un milagro calmara la tempestad. Nos mirábamos unos a otros esperando la muerte a cada instante, y tratábamos de prepararnos para la otra vida, ya que comprendíamos que poco nos quedaba por hacer en ésta. Algo nos consolaba que el navío hubiera resistido hasta ese instante, y el capitán sostenía que el viento estaba amainando un poco; pero aunque fuera así, el buque encallaba profundamente en las arenas y parecía demasiado hundido para pensar en sacarlo de su posición, de manera que seguíamos en terrible peligro y sólo nos quedaba tratar de salvar la vida de cualquier manera. Había un bote en la popa antes de que estallara la borrasca, pero se destrozó al chocar incesantemente contra el timón y luego de partirse fue arrebatado por las olas, de manera que no contábamos con él; quedaba otro bote a bordo, ¿pero podríamos echarlo al agua? Sin embargo, no había nada que discutir, pues estábamos seguros de que el barco iba a partirse en pedazos de un momento a otro, y ya algunos aseguraban que estaba destrozado.

    En esta confusión, el piloto se decidió a asegurar el bote y con ayuda de la tripulación consiguió hacerlo pasar sobre la borda; inmediatamente embarcamos, once en total, y nos confiamos a la merced de Dios en aquel mar embravecido, que, aunque había amainado el viento, seguía encrespándose horrorosamente.

    Al punto comprendimos que estábamos perdidos; el oleaje era tan alto que el bote no podía resistirlo y no pasaría mucho antes de ahogarnos. Otra vez confiamos nuestras almas a la Providencia, y como el viento nos arrastraba hacia la costa apresuramos nuestra destrucción remando con toda la fuerza posible hacia tierra.

    ¿Cómo era la costa? ¿Rocosa o arenosa, abrupta o de suave pendiente? No lo sabíamos; nuestra única sombra de esperanza era la de ir a parar a un golfo o bahía, quizá las bocas de un río donde nuestro bote, a cubierto por el sotavento de la tierra, encontrara aguas tranquilas. Pero nada de esto parecía probable y mientras nos acercábamos a la costa la encontrábamos aún más espantosa que el mismo mar.

    Después de remar, o mejor, de dejarnos llevar, aproximadamente una legua y media, una gigantesca ola, como rugiente montaña líquida, se precipitó súbitamente sobre nosotros, dándonos la impresión de que era el «coup de gráce». Nos cayó con tal violencia que el bote se dio vuelta en un instante, y separándonos de él como de nosotros mismos, sin darnos tiempo a decir: « ¡Mi Dios!»,'nos engulló a todos.

    No podría describir el estado de ánimo que tenía cuando me sentí hundir en las aguas, porque aunque sabía nadar muy bien no conseguía librarme de la fuerza de las olas y ascender a respirar, hasta que después de arrastrarme interminablemente en dirección a la playa, la ola rompió allí y al retroceder me dejó en tierra firme, medio muerto por el agua que había tragado. Me quedaban suficiente aliento y presencia de ánimo como para advertir que estaba más cerca de la playa de lo que había supuesto, y enderezándome traté de correr hacia ella con toda la velocidad posible antes de que otra ola me arrebatara. Pero de inmediato supe que aquello era imposible porque vi crecer el mar a mis espaldas como una montaña y con la furia de un enemigo que me superaba infinitamente en fuerzas. Mi salvación estaba en retener el aliento y sostenerme a flote todo lo posible, tratando en esa forma de nadar hacia la playa; pero me aterraba pensar que acaso el oleaje, después de sumirme profundamente en el mar, no me devolvería a la costa en su retorno.

    La ola que me cayó encima me hundió veinte o treinta pies en su seno, y otra vez me sentí arrastrado con una salvaje violencia y velocidad hacia la tierra, pero contuve la respiración y traté de nadar hacia adelante con todas mis fuerzas. Me parecía que iba a estallar por falta de aire, cuando me sentí levantado y de pronto tuve la cabeza y las manos fuera del agua; aunque esto solamente duró un segundo, me permitió recobrar el aliento y nuevo valor. Otra vez me tapó el agua, pero no tanto como para hacerme perder las energías, y cuando advertí que estaba en la playa y que la ola iba a volver, luché por sostenerme hacia adelante y toqué tierra con los pies. Me estuve quieto un momento para recobrar la respiración y mientras el agua se retiraba eché a correr con toda la velocidad posible hacia la costa. Pero ni esto me libró de la furia del mar y por dos veces consecutivas volví a ser arrebatado y devuelto otra vez a la playa, que era sumamente suave.

    La segunda vez estuvo a punto de serme fatal porque el oleaje, después de llevarme mar adentro, me proyectó con violencia contra una roca y tal fue la fuerza del golpe que me privó de los sentidos, dejándome indefenso contra su furia. El golpe me había magullado el pecho y el costado, privándome por completo de la respiración; estoy seguro de que si el mar hubiera vuelto inmediatamente habría perecido ahogado. Pero recuperé los sentidos un momento antes del retorno de la ola, y viendo que otra vez iba a ser arrastrado por ella me aferré con todas mis fuerzas a la roca, luchando por contener el aliento hasta que el agua retrocediera. Las olas ya no eran tan altas como antes, por la proximidad de la costa, y pude por lo tanto resistir el embate hasta que cesó, y entonces eché a correr hacia tierra con tal fortuna que la siguiente ola, aunque me alcanzó, ya no pudo arrancarme de donde estaba y en una segunda carrera me libré totalmente de su rabia, encaramándome sobre los acantilados hasta desplomarme sobre la hierba, libre de todo peligro y a salvo del mar.

    Cuando comprendí con claridad el riesgo del que acababa de salvarme, elevé mis ojos a Dios y le agradecí que hubiera perdonado una vida que segundos antes no conservaba la menor esperanza. Me paseaba por la playa alzando no sólo las manos sino todo mi ser en acción de gracias por mi rescate, haciendo mil ademanes que no podría describir y reflexionando sobre mis camaradas que se habían ahogado, siendo yo el único que había conseguido pisar tierra; nunca volví a verlos, ni siquiera encontré señales de ellos, salvo tres sombreros, una gorra y dos zapatos de distinto par.

    Fijé los ojos en el barco encallado, al que la distancia y la furia del mar apenas me permitían divisar, y me maravillé.

    — ¡Oh, Señor! —prorrumpí—. ¿Cómo he podido llegar a tierra?

    Después de alegrar mi espíritu con el lado feliz de mi aventura, empecé a reconocer el lugar en torno mío para averiguar qué clase de sitio era y cuáles medidas debía tomar. Mas pronto cesó mi contento al comprender que de nada me servía la salvación. Estaba empapado, sin ropa que cambiarme y nada para comer y beber; la perspectiva más probable era la de morir de hambre o ser devorado por animales feroces. Lo que más me afligía era no tener armas con que matar un animal para alimentarme o como defensa contra cualquier bestia que quisiera hacerlo a costa mía. En una palabra, sólo tenía un cuchillo, una pipa y un poco de tabaco en una cajita. Al comprender la miseria en que me encontraba sentí crecer en mí tal desesperación que eché a correr como un loco. La noche se acercaba y en mi angustia me pregunté si en aquel país habría bestias salvajes, sabiendo de sobra que aquellas eligen las tinieblas para acechar sus presas. Todo lo que se me ocurrió fue treparme a un frondoso árbol, especie de abeto pero con espinas, y allí me propuse estarme la noche entera y decidir, a la mañana siguiente, cuál sería mi muerte; porque ya no veía esperanza alguna de seguir viviendo.

    Anduve primero en busca de agua dulce, que con gran alegría encontré a un octavo de milla aproximadamente; después de beber y mascar un poco de tabaco para adormecer el hambre, trepé a mi árbol, tratando de hallar una posición de la cual no me cayera si el sueño me vencía. Había cortado un sólido garrote para defenderme, y era tal mi extenuación que pronto quedé dormido con un sueño profundo y tranquilo como no creo que nadie haya podido disfrutar en semejantes circunstancias.






	\chapter{La isla desierta}





    Era pleno día cuando desperté; el tiempo estaba despejado y sin huellas del temporal, por lo que el mar aparecía muy tranquilo. Lo que más me sorprendió fue advertir que la marea había zafado el barco de las arenas donde encallara y traído hasta junto a la roca donde por poco me matan las olas al golpearme contra ella. Apenas una milla me separaba del barco, y notando que éste se mantenía a flote se me ocurrió ir a bordo en procura de aquellas cosas que me fueran necesarias.

    Bajando del árbol, dirigí la vista en torno y no tardé en descubrir el bote que el viento y las olas habían arrojado a las arenas dos millas a mi derecha. Fui hacia él para asegurarlo, pero encontré un brazo de mar ancho de media milla entre el bote y yo, y volviéndome por el mismo camino busqué acercarme al barco, donde esperaba encontrar alimentos.

    Poco después de mediodía el mar se puso como un espejo y la marea bajó tanto que pude acercarme a un cuarto de milla del barco; ya entonces sentía renovarse mi desesperación al comprender que si nos hubiéramos quedado a bordo todos estaríamos a salvo y en tierra, sin verme yo reducido a una absoluta soledad, huérfano de socorro y alivio. Derramé nuevamente lágrimas, pero como de nada me servían resolví si era posible llegar al barco. Hacía mucho calor, por lo cual me quité parte de la ropa antes de tirarme al agua, y nadando hasta el buque empecé a buscar un modo de trepar a cubierta. La dificultad estaba en que el buque se mantenía derecho, sin punto alguno de apoyo para intentar escalarlo. Nadé dos veces en torno a él, y a la segunda advertí un cabo de cuerda que colgaba de los portaobenques de mesana. Asombrado de no haber reparado antes en ella, así su extremo después de muchos esfuerzos y me encaramé al castillo de proa. El barco tenía una vía de agua y estaba parcialmente inundado; encallado en un banco de arena muy dura —o más bien de tierra—, la popa se levantaba sobre aquél mientras la proa casi tocaba el agua. Era de alegrarse que toda la popa estuviera sobre el nivel del banco, ya que cuanto contenía se encontraba intacto, cosa que de inmediato me apresuré a verificar. Las provisiones de a bordo no habían sufrido absolutamente nada, y de inmediato pude satisfacer mi gran apetito llenándome los bolsillos de galleta y comiendo a la vez que revisaba el resto del barco para no perder tiempo. Hallé un poco de ron en la cabina del capitán, y bebí un buen trago para fortalecerme ante la tarea que me esperaba. Ahora solamente me hacía falta un bote para llenarlo con todo aquello que presentía iba a serme de gran necesidad.

    Era inútil sentarse a esperar lo imposible, y la dificultad aguzó mi ingenio. Había a bordo muchas verjas sueltas, dos o tres perchas o berlingas y uno o dos masteleros de juanete. Me resolví a emplearlos y levantándolos por la borda los arrojé al agua no sin antes atarlos con sogas para que el mar no los llevase lejos. Hecho esto me descolgué por el costado del buque y atrayendo los palos cerca de mí empecé a atar juntamente cuatro de ellos, sujetándolos por ambos extremos para formar una especie de balsa; cruzando los palos menores para reforzarla comprobé que me sostenía muy bien sobre el agua pero que no sería capaz de soportar un gran peso por la fragilidad de la madera. Subiendo a bordo corté con la sierra del carpintero un mastelero de juanete en tres partes, que incorporé a mi balsa no sin gran esfuerzo y fatiga; pero la esperanza de proveerme de aquello que tanto iba a necesitar me movió a hacer más de lo que me hubiera creído capaz en otro momento.

    Ahora mi balsa era lo bastante resistente para llevar una carga razonable. Se presentaba el problema de elegir lo indispensable y al mismo tiempo preservarlo de los golpes del mar. Ante todo puse en la balsa todas las planchas y tablas que pude reunir y después de pensar bien lo que me hacía falta busqué tres arcones de marinero y vaciándolos los puse en la balsa. Al primero lo llené de provisiones, como ser arroz, pan, tres quesos de Holanda, cinco trozos de carne seca de cabra —que había sido nuestro alimento habitual a bordo— y un pequeño sobrante de granos que fuera embarcado para alimentar las aves que llevábamos y que ya habíamos comido. Recordé la existencia de alguna cantidad de cebada y de trigo candeal, pero con gran disgusto mío las ratas lo habían devorado. Hallé muchas cajas de botellas de licor, pertenecientes al capitán, y además unos cinco o seis galones de la bebida llamada arak. Llevé las cajas a la balsa, no habiendo necesidad de meterlas en los arcones donde, por otra parte, no cabían.

    Mientras me ocupaba en esto advertí que la marea empezaba a subir aunque muy lentamente, y tuve la mortificación de ver mi saco, camisa y chaleco que dejara en la playa, arrastrados por el agua; había nadado hasta el barco con los calzones, que eran de lienzo y abiertos hasta la rodilla, y los calcetines. Lo ocurrido me hizo pensar en la necesidad de ropas, y aunque había mucha a bordo sólo tomé las indispensables por el momento, puesto que otras cosas reclamaban mi interés con mayor fuerza; sobre todo herramientas para trabajar en tierra. Después de mucho buscarlo di con el arcón del carpintero, que me parecía más valioso que todo un cargamento de oro. Lo llevé tal como estaba a la barca, sin perder tiempo en abrirlo, puesto que tenía una idea aproximada de su contenido.

    Mi inmediata tarea fue procurarme armas y municiones. Había dos magníficas escopetas de caza en la cabina del capitán, y dos pistolas; las cogí, así como algunos frascos de pólvora, un saquito de balas y dos viejas espadas enmohecidas. Recordaba que a bordo había tres barriles de pólvora, pero no el lugar donde los tenía el artillero. Tras mucho buscar di con ellos, y aunque uno se había mojado los restantes parecían secos y me los llevé todos a la balsa. Mi cargamento me llenaba de satisfacción, pero el problema estaba en llegar con él a la playa no teniendo vela, remo ni timón; el más pequeño golpe de viento hubiera acabado con mis esperanzas. Tenía, sin embargo, tres razones para sentirme confiado. En primer lugar la tranquilidad del océano, luego la marea alta que se movía hacia la costa y por fin el leve viento que soplaba en dirección de tierra. Encontré dos o tres remos rotos que habían sido del bote, y tras de hallar en cubierta algunas otras herramientas tales como dos sierras, un hacha y un martillo, bajé todo a la balsa y con tal cargamento me hice a la mar. Por espacio de una milla aproximadamente mi balsa navegó muy bien, sólo desviándose un poco del sitio donde tocara primeramente tierra, lo que me hizo suponer alguna corriente marina; acaso, pensé, hallaría cerca algún arroyo o ensenada que pudiera servirme de puerto para desembarcar mi cargamento.

    Ocurrió como lo imaginaba; pronto vi a la distancia una entrada en la tierra hacia donde la fuerte corriente de la marea se precipitaba, e hice por mantener mi balsa en el centro de la corriente. Fue entonces cuando estuve a punto de sufrir un segundo naufragio que, estoy seguro de ello, hubiera concluido conmigo. Ignorante por completo de la costa no pude impedir que mi balsa chocara contra un bajío, y no tocando tierra por el otro extremo faltó poco para que el cargamento resbalara hacia ese lado y cayera al agua. Apoyé la espalda contra los arcones para mantenerlos en su lugar, pero me fue imposible desencallar la balsa a pesar de mis esfuerzos, y por otra parte no me animaba casi a moverme y así, sosteniendo los arcones con todas mis fuerzas, permanecí cerca de media hora hasta que el ascenso de la marea niveló un poco la balsa. Un rato después se zafó sola, y aprovechando un remo la hice entrar en el canal y avanzando un poco me hallé en la boca de un riachuelo, con tierra a ambos lados y la fuerte corriente de la marea remontando la balsa. Miré hacia las orillas en procura de un buen sitio para desembarcar, ya que no quería internarme demasiado sino establecerme junto a la costa para esperar el paso de algún buque.

    Luego de elegir una pequeña caleta en la orilla derecha de la ensenada conduje con gran trabajo la balsa hasta allí y me puse tan cerca que clavando el remo pude hacer que tocara la tierra. Pero por segunda vez estuvo mi cargamento a punto de caer al agua, porque la orilla era muy abrupta y la balsa sólo tocaba en ella con uno de sus extremos, de modo que el otro quedaba a un nivel inferior y ponía en peligro mis cosas. Sólo me quedó esperar que la marea creciera aún más, empleando el remo como ancla para mantener la balsa junto a un sitio llano donde confiaba que alcanzaría la marea. Así fue, y tan pronto vi que había fondo suficiente para que la balsa pudiera moverse la llevé hasta esa plataforma llana sujetándola por medio de los remos clavados en el suelo, uno a cada extremo; y ahí quedamos hasta que la marea empezó a bajar y nos dejó en tierra firme.

    Mi inmediata tarea era la de reconocer el lugar en busca de un sitio adecuado para instalarme y almacenar mis efectos con toda seguridad. Ignoraba por completo dónde me encontraba. ¿Era el continente o una isla, estaba o no habitado, habría bestias salvajes en los alrededores? A una milla de donde me hallaba vi una colina alta y escarpada, que parecía sobrepasar a otras que continuaban la cordillera hacia el norte. Tomando una escopeta, una pistola y suficiente pólvora, me encaminé hacia la cumbre de la colina adonde llegué después de duras dificultades. Apenas dirigí la mirada en torno cuando tuve conciencia de mi triste destino: estaba en una isla, enteramente rodeada por el mar y sin tierras próximas, excepción hecha de algunos lejanos escollos y dos pequeñas islas, menores que ésta, a unas tres leguas hacia el oeste.

    Evidentemente la tierra era inculta y, como podía suponerse, solamente habitada por animales salvajes, de los que sin embargo no vi ninguno. Había una diversidad de aves cuyas especies eran desconocidas para mí; me pregunté si su carne sería o no comestible. Mientras regresaba, maté un gran pájaro que posaba en un árbol junto a un bosque. Pienso que aquél fue el primer disparo hecho en aquella tierra desde la creación del mundo, pues como respuesta al estampido se levantó una bandada inmensa de aves de toda clase produciendo una algarabía confusa en la que distinguí los gritos de cada especie, pero ninguno me resultó familiar. El pájaro muerto era parecido a un halcón, sobre todo en el pico y el color, pero no tenía las garras que son propias de este pájaro; en cuanto a su carne resultó imposible de comer.

    Satisfecho con lo que había investigado me volví a la balsa y empecé a trasladar el cargamento, lo que me ocupó el resto del día. Cuando vino la noche me pregunté dónde la pasaría, pues desconfiaba quedarme en el suelo por temor a alguna bestia salvaje, temor que como descubrí más adelante era infundado. Hice una barricada con los arcones y las tablas, en forma de tosca cabaña, y en ella me parapeté para pasar la noche. Aún ignoraba de qué manera iba a alimentarme, puesto que solamente había visto dos o tres animales parecidos a liebres en el bosque donde matara al pájaro.

    Se me ocurrió que aún podría sacar muchas cosas útiles del barco, en especial aparejos, velas y todo lo que pudiera ser transportado a tierra, y me decidí a hacer otro viaje a bordo. No ignoraba que la próxima tormenta acabaría con el barco, de modo que me pareció mejor dejar de lado toda tarea hasta concluir con aquélla. Consulté conmigo mismo si llevaría la balsa, pero me pareció poco apropiado y preferí irme otra vez nadando en cuanto bajara la marea. Así lo hice, abandonando la choza sin más que una camisa, calzoncillos y zapatos livianos.

    Cuando estuve a bordo construí una segunda balsa, pero aprovechando la experiencia de la primera no la hice tan pesada ni la cargué tanto. Muy pronto hallé cantidad de cosas útiles; primeramente, en el cuarto del carpintero, dos o tres cajas de clavos y tornillos, un gran barreno, una o dos docenas de hachuelas, y lo más precioso de todo, una piedra de afilar. Reuní todo juntamente con varios objetos pertenecientes al artillero, como ser algunas palancas de hierro, dos barriles de balas de mosquete, siete mosquetes y otra escopeta, con alguna pequeña cantidad de pólvora; hallé también una gran caja llena de perdigones y un pedazo de plomo, pero este último pesaba tanto que no pude hacerlo pasar por la borda. Fuera de esto reuní todas las ropas que pude encontrar, una vela sobrante de la cofa de trinquete, una hamaca, colchones y ropa de cama. Cargué mi segunda balsa con todo aquello y conseguí llevarla sin dificultad a tierra.

    Durante el tiempo que pasé a bordo había sentido el temor de que en mi ausencia mis provisiones fueran devoradas, pero cuando desembarqué pude ver que no había señales de vida en torno, salvo la presencia de un animal semejante a un gato montes que se había subido a uno de los arcones y que al verme se alejó un poco. Noté que no tenía miedo, y que me miraba fijamente como mostrando su intención de trabar relaciones amistosas. Le apunté con la escopeta, pero como no comprendió la razón del gesto se quedó inmóvil y sin mostrar deseos de alejarse, por lo que le ofrecí un pedazo de galleta que, dicho sea de paso, no me sobraba como para andar regalándola. Sin embargo, le di aquel trozo, que el gato comió después de olfatearlo; pareció quedar muy satisfecho y. desear aún más, pero ya entonces no repetí el obsequio y al rato lo vi marcharse.

    Trasladando mi segundo cargamento a tierra, me vi en la obligación de abrir los barriles de pólvora y dividir el contenido, porque el excesivo peso no me dejaba moverlos; me puse en seguida a construir una pequeña tienda con la vela y algunas estacas que corté; puse en la tienda todo aquello que podría estropearse con la lluvia o el sol, y por fuera hice una barricada con los arcones y los barriles vacíos, a manera de fortificación contra cualquier ataque de hombre o animal.

    Terminado esto bloqueé la puerta por dentro con tablones y por fuera con un arcón parado; tendiendo uno de los colchones, con las dos pistolas cerca de la mano y la escopeta a mi alcance, me metí en cama por primera vez y dormí plácidamente toda la noche porque estaba rendido hasta la extenuación, habiendo dormido muy poco la noche anterior y trabajado el día entero en lo que ya he descrito.

    Era dueño del más completo y variado surtido de efectos que jamás fuese reunido por un solo hombre, pero no me sentía aún satisfecho, ya que estando el barco a mi alcance me pareció necesario extraer de él todo lo posible. Por lo tanto iba diariamente a bordo aprovechando la marea baja y sacaba una y otra cosa del navío; en especial la tercera vez que fui traje todos los aparejos que pude reunir, las cuerdas y jarcias, con un pedazo de lona que servía para remendar las velas, y hasta el barril de pólvora que se había mojado. Por fin saqué del barco todas las velas, aunque me vi obligado a cortarlas en pedazos para llevarlas juntas; pero no me importaba, ya que en adelante sólo servirían como lonas.

    Lo que más me alegró en aquellos viajes fue que después de estar a bordo cinco o seis veces, y cuando ya no esperaba encontrar nada que valiera la pena de mover de su sitio, descubrí un gran barril de galleta, tres pipas de ron o aguardiente, una caja de azúcar y un barril de harina flor; me quedé admirado, porque ya no creí hallar provisión alguna, salvo las que estaban estropeadas por el agua. Vacié el barril de galleta, haciendo paquetes pequeños con los pedazos de velas, y pronto arribé felizmente a tierra con todo.

    Al otro día hice un nuevo viaje. El barco estaba ya despojado de todo lo que podía moverse y transportarse fácilmente, de modo que la emprendí con los cables cortando el más grueso en trozos que pudieran llevarse, y así preparé dos cables y una guindaleza, junto con todo el herraje que pude juntar y que puse en una gran balsa hecha con los trozos de vergas de cebadera y de mesana. Pero mi buena suerte empezó a abandonarme, porque la balsa era tan pesada y tenía tanta carga que apenas habíamos llegado a la pequeña caleta que me servía de desembarcadero cuando por una falsa maniobra se hundió arrojándome al agua con todos aquellos efectos. Yo no corría peligro puesto que estaba junto a la costa, pero el cargamento se perdió en gran parte, especialmente el hierro, que tan útil me hubiera sido. Cuando bajó la marea pude salvar buena parte de los pedazos de cable y algo de los herrajes, aunque con gran esfuerzo, porque tenía que zambullirme para buscarlo y pronto me agotó la tarea. Pese a todo seguí yendo al barco y trayendo lo que encontraba aprovechable.

    Llevaba ya trece días en la playa y había hecho once viajes al barco, en cuyo transcurso retiré todas aquellas cosas que un par de manos pueden mover, tanto que de haber continuado el buen tiempo estoy seguro que hubiera terminado por traerme el barco pieza por pieza a la costa. Cuando me disponía a mi duodécimo viaje empezó a soplar viento, pero aproveché la marea baja para intentar otra expedición. Me parecía haber saqueado completamente la cabina del capitán, y sin embargo hallé todavía un armario con cajones, en uno de los cuales había dos o tres navajas, un par de tijeras largas, y casi una docena de excelentes cuchillos y tenedores; en otro cajón hallé un valor de treinta y seis libras esterlinas en monedas europeas, brasileñas y algunas piezas de a ocho de oro y plata. Sonreí a la vista de aquel dinero. — ¡Ah, metal inútil! —exclamé—. ¿Para qué me sirves? No mereces que me moleste en recogerte; cualquiera de esos cuchillos vale más que tú. ¡En nada podría emplearte y mejor es que te quedes donde estás y te hundas como un ser cuya vida no vale la pena salvar!

    Pero luego lo pensé mejor y tomé el dinero, envolviendo todo con una pieza de lona y pensando ya en construir otra balsa; mas cuando salí a cubierta el cielo se había encapotado, el viento crecía y al cuarto de hora soplaba fuerte desde la costa. Comprendí que era inútil hacer una nueva balsa soplando viento de tierra, y que me convenía alejarme de allí antes de que comenzara el reflujo impidiéndome alcanzar la orilla. Me arrojé inmediatamente al agua y nadé hacia el canal con gran dificultad, en parte por el peso del bulto que llevaba y en parte por el fuerte oleaje que el viento levantaba cada vez con más violencia hasta convertirse en tempestad.

    Pude llegar con fortuna a mi pequeña tienda, donde me refugié con todas mis riquezas bien aseguradas. La tormenta arreció aquella noche, y a la mañana siguiente encontré que el barco había desaparecido. Me afligió un poco, pero mi consuelo fue reflexionar que no había perdido tiempo ni escatimado esfuerzos para retirar de él todo lo que pudiera serme útil, siendo bien poco lo que podía haber quedado a bordo.

    Desentendiéndome, pues, del barco y su recuerdo, sólo me ocupé de aquellos pedazos que la tormenta había arrastrado a la playa, pero pronto supe que serían de muy poca utilidad. Mis pensamientos estaban consagrados ahora a encontrar los medios de asegurarme contra los salvajes o las bestias que pudiera haber en la isla; vacilé mucho acerca de las medidas que debía tomar, si me convenía construir una choza o cavar un abrigo en la profundidad de la tierra. Por fin, luego de meditarlo bien, me resolví por ambas cosas. Y se me ocurre que puede ser interesante la descripción de cómo las llevé a cabo.

    Había advertido que el lugar en que estaba no era conveniente para establecerme, en especial porque se hallaba sobre terrenos pantanosos e insalubres próximos al mar, y cerca de allí no había agua dulce. Me resolví, por tanto, a buscar un sitio más saludable y apropiado para construir mi vivienda.

    Calculé aquello que necesitaba de manera indispensable: en primer lugar agua dulce y aire saludable, como ya he dicho; luego abrigo de los ardores solares y seguridad contra posibles atacantes, fueran hombres o animales. Finalmente quería tener frente a mí el horizonte marino, para que, si Dios me enviaba algún barco por las cercanías, no perdiera yo esa oportunidad de salvarme, ya que tal esperanza no había perecido todavía en mí.

    En busca del lugar que reuniera tales condiciones, hallé una pequeña explanada al costado de una colina cuya ladera era tan escarpada como un muro y me evitaba, por tanto, todo peligro de ese lado. En un lugar de la roca había un hueco, semejante a la entrada de una caverna, pero en realidad no se trataba de ninguna cueva ni entrada. Decidí instalar mi choza en la explanada, justamente delante de ese hueco; noté que la parte llana tenía unas cien yardas de ancho y el doble de largo, y que se extendía como un parque delante de mi puerta, descendiendo luego irregularmente hacia las tierras bajas del lado del mar. Estaba hacia el N-NO de la colina, de modo que me protegía de los calores diurnos hasta que el sol descendiera al O cuarto SO, que en aquellas latitudes ocurre casi al crepúsculo.

    Antes de principiar mi tienda tracé un semicírculo delante de la parte hueca, cuyo diámetro a partir de la roca era de unas diez yardas, y veinte en el diámetro total desde uno a otro extremo. En este semicírculo clavé dos hileras de fuertes estacas, hundiéndolas en tierra hasta que quedaron absolutamente firmes, sobresaliendo de la tierra hasta unos cinco pies y medio, y las agucé en la punta. Las dos hileras no estaban separadas más de seis pulgadas entre sí. Tomando entonces los pedazos de cable que me había procurado en el barco, los apilé en el interior del círculo apretándolos hasta que cubrieron el espacio entre las estacas, y sostuve mi empalizada con otras estacas de unos dos pies y medio que coloqué inclinadas por el lado de adentro, a manera de puntales. Tan fuerte quedó el vallado que ningún animal o ser humano hubiera podido derribarlo, ni siquiera pasar por encima. Tuve mucho que trabajar en él, especialmente cortando la madera de los bosques, llevándola al lugar y clavándola en tierra.

    Decidí que la entrada no sería una puerta sino una corta escalera para trepar a la empalizada, puesta de tal modo que una vez dentro fuera fácil retirarla, con lo cual me encontraba perfectamente amurallado y defendido contra todo el mundo y podía dormir sin temor a enemigos, aunque más tarde vine a saber que mis precauciones no eran necesarias.

    Con infinito trabajo reuní todos mis afectos en la fortaleza, provisiones, armas y demás cuya lista es ya conocida, y armé una gran tienda; para que me preservara de las lluvias, que en cierta estación caen allí con violencia, hice una tienda doble, es decir, una pequeña y otra mayor tendida por encima, cubriendo esta última con una tela embreada que traje del barco juntamente con las velas. Ya no dormía en el colchón sino que instalé la hamaca que había pertenecido al piloto del barco y era excelente.

    Puse en la tienda todas las provisiones y aquello que pudiera estropearse con las lluvias, y habiendo comprobado que mis bienes estaban a salvo cerré la entrada que hasta ese momento dejara abierta en la empalizada y desde entonces utilicé la escalera para entrar y salir.

    A partir de ese día principié a excavar la roca, de la que arranqué gran cantidad de piedras y tierra que fui apilando al pie de mi empalizada a manera de terraplén de pie y medio de alto. Pronto tuve, pues, una nueva cueva justamente detrás de mi tienda, que me servía de bodega y despensa.

    Todo aquello me llevó mucho tiempo y grandes fatigas, y en ese transcurso ocurrieron varias cosas que me preocuparon. En los días en que trazaba los planes para armar mi tienda y excavar la roca, ocurrió que en medio de una violenta tormenta que acababa de desatarse vi caer un rayo, seguido inmediatamente de un terrible trueno. No me asustó tanto el rayo como el pensamiento que de inmediato cruzó por mi mente:

    — ¡La pólvora!

    Creí que mi corazón cesaba de latir al pensar que en un segundo mi pólvora podía arder, privándome no sólo de defensa sino del alimento que contaba lograr con ella. Ni siquiera sentí miedo por mí mismo, porque sabía bien que si la pólvora estallaba no me daría tiempo a pensar de dónde procedía la catástrofe.

    Tal impresión me causó lo sucedido que después de la tormenta dejé de lado mis tareas —la tienda, la fortificación— y me apliqué a fabricar cajas y bolsas donde separar la pólvora para impedir que ardiera toda, y al mismo tiempo distanciarla lo bastante entre sí para que el incendio de una parcela no determinara el de las restantes. El trabajo llevó una quincena, pero por fin la pólvora, que alcanzaba a unas doscientas cuarenta libras, quedó dividida en no menos de cien paquetes. Por lo que respecta al barril que se había mojado no me inspiraba temor, de modo que lo puse en mi caverna, a la que yo llamaba «la cocina»; el resto lo distribuí en agujeros entre las rocas, cuidando que ninguna humedad llegara a los paquetes, y marcando exactamente el sitio donde los dejaba.

    Mientras me ocupaba en todo esto no dejé de salir por lo menos una vez al día con mi escopeta, en parte para distraerme y en parte para ver si cazaba algo comestible, a la vez que exploraba las posibilidades de la isla. El primer día que salí tuve gran satisfacción al encontrar que había cabras en los alrededores, pero pronto me desalentó lo tímidas, astutas y ágiles que se mostraban, al extremo de que era casi imposible acercarse a ellas. Sin descorazonarme me dije que la ocasión se presentaría de alcanzar alguna con mis disparos, como efectivamente ocurrió una vez que hube localizado los lugares que frecuentaban. Noté que si me acercaba viniendo por el valle, las cabras huían aterradas, aunque estuviesen al abrigo de las altas rocas, pero que si triscaban en el valle y yo venía por las alturas ni siquiera reparaban en mi presencia, de lo cual deduje que la posición de sus ojos era tal que no veían sino aquello que estaba a su nivel o por debajo. Adopté de inmediato la costumbre de encaramarme a las rocas más altas, y desde allí me fue bastante simple abatir alguna. El primer disparo que hice mató una cabra cuyo cabrito todavía se amamantaba, lo cual me produjo mucha pena, ya que al caer la madre vi que no se movía de su lado, incluso cuando me acerqué a él. Mientras llevaba la cabra sobre mis hombros, el cabrito me siguió hasta la empalizada, y allí lo tomé en los brazos y lo hice pasar al interior con la esperanza de domesticarlo. Pero se negó a comer y al fin me vi precisado a matarlo para comerlo yo. Aproveché aquella carne durante bastante tiempo, pues me alimentaba con mucha prudencia y trataba de economizar las provisiones, especialmente la galleta.






	\chapter{El diario de Robinson}





    Ahora que me toca iniciar la melancólica narración de una vida solitaria, tal como acaso nunca fuera imaginada en el mundo, quiero hacerlo desde su comienzo y proseguir ordenadamente. Según mis cálculos, había arribado en la forma narrada a tan hórrida isla un 30 de setiembre, cuando el sol en su equinoccio otoñal estaba casi sobre mi cabeza, de donde calculé que me hallaba a una latitud de nueve grados veintidós minutos norte.

    Después de vivir allí diez o doce días se me ocurrió que por falta de calendarios, así como de papel y tinta, perdería la cuenta del tiempo y no sería capaz de distinguir los días de fiesta de los de trabajo. Para evitarlo hice un poste en forma de cruz, que clavé en el sitio donde por primera vez había tocado tierra, y grabé en él con mi cuchillo y en letras mayúsculas:



\begin{quote}

    LLEGUE A ESTA PLAYA EL

    30 DE SETIEMBRE DE 1659

\end{quote}



    Sobre los lados del poste practicaba diariamente un corte, y cada siete una marca algo mayor; el primer día del mes hacía una señal aún más grande, y en esa forma llevé mi calendario de semanas, meses, años.

    Entre lo mucho que había traído del barco encallado en los viajes arriba mencionados se encontraban diversas cosas muy útiles para mí, aunque menos que las otras, por lo cual no las describí antes. En particular plumas* tinta y papel, y objetos pertenecientes al capitán, piloto, artillero y carpintero, tales como tres o cuatro compases, instrumentos matemáticos, cuadrantes, anteojos de larga vista, mapas y libros de navegación, etc., todo lo cual traje a tierra sin saber si me serviría o no. Encontré también tres excelentes Biblias que vinieran de Inglaterra con mi cargamento y que yo había cuidado de llevar conmigo; algunos libros portugueses, entre ellos dos o tres libros católicos de oraciones y varios otros que conservé cuidadosamente. No debo olvidarme de señalar que teníamos a bordo un perro y dos gatos, de cuya importante historia habré de ocuparme en su justo lugar. Había traído conmigo los dos gatos, y en cuanto al perro se arrojó él mismo al agua y vino nadando a mi lado el día siguiente a mi primer viaje al barco; desde entonces estuvo conmigo y fue un fiel compañero por muchos años. No me interesaba lo que pudiera apresar para mí, ni la compañía que me hacía; hubiera solamente deseado oírle hablar, y por desgracia eso era lo imposible.

    Como antes he dicho encontré plumas, tinta y papel, e hice lo indecible por economizarlos; mientras duró la tinta pude llevar una crónica muy exacta, pero cuando se terminó me hallé imposibilitado de continuarla, ya que no pude hacer tinta a pesar de todo lo que probé. Esto vino a demostrarme que necesitaba muchas cosas fuera de las que había acumulado; así como tinta, debo citar la falta que me hacían una azada, pico y pala para roturar la tierra, y también agujas, alfileres e hilo; en cuanto al lienzo, pronto me pasé fácilmente sin él.

    Tal falta de utensilios tornaba fatigosa toda tarea que emprendía, y transcurrió casi un año antes de que hubiera terminado mi empalizada y las demás obras. Las estacas, que eran tan pesadas como podía encontrar, llevaba mucho tiempo cortarlas y aguzarlas en el bosque y otro tanto moverlas hasta la explanada. A veces pasaba dos días entre cortar y trasladar uno de aquellos postes y un tercer día en hundirlo firmemente en el suelo, para lo cual me valía de una pesada maza de madera hasta que se me ocurrió emplear una de las palancas de hierro; asimismo me daba mucho trabajo asegurar aquellos postes.

    Pero ¿por qué había de preocuparme el mucho tiempo que insumían estas cosas? Bien claro estaba que me sobraba tiempo, y si mis trabajos hubieran terminado antes me habría quedado sin saber qué hacer, salvo explorar la isla en busca de alimento, cosa que llevaba a cabo casi diariamente.

    Empecé así a meditar seriamente sobre la condición en que me hallaba y las circunstancias a que me veía reducido, y redacté por escrito mis pensamientos, no tanto por dejarlos a mis herederos, que por lo visto serían pocos, sino para aliviar a mi espíritu de llevarlos constantemente consigo hasta la aflicción. Mi razón empezaba a dominar mis desfallecimientos, veía de consolarme lo mejor posible y a oponer el bien al mal para que mi situación no me pareciera tan desesperada en comparación a otras mucho peores. Todo eso fue escrito imparcialmente, a manera de un debe y haber, señalando los consuelos que me habían sido dados a cambio de las desgracias que sufría, en la siguiente forma:






LO MALO


He sido arrojado a una isla desierta sin la menor esperanza de rescate.

He sido excluido del resto del mundo, a solas con mi miseria.









LO BUENO


Pero vivo, sin haberme ahogado como mis compañeros.

Pero también he sido excluido de la muerte, al contrario de toda la tripulación del barco; y El, que me salvó milagrosamente de tal muerte, puede salvarme igualmente de esta condición en que me hallo.


LO MALO


Vivo separado de la humanidad, solitario y desterrado de toda sociedad. No tengo ropas para cubrirme.

Carezco de toda defensa contra los animales y los hombres.

No tengo a nadie con quien hablar, a nadie que me consuele.














LO BUENO


Pero no he muerto de hambre en un lugar desierto, privado de toda subsistencia.

Pero estoy en un clima cálido donde las ropas me servirían de poco.

Pero me encuentro en una isla donde no he visto animales feroces que me amenacen, como los viera en la costa de África. ¿Y si hubiera naufragado allá? Pero Dios envió milagrosamente el barco cerca de la costa para que pudiera sacar de él multitud de cosas necesarias que suplen mis necesidades o me permitirán hacerlo mientras viva.



    Habiendo conseguido acostumbrar un poco mi espíritu a su actual condición y abandonando la costumbre de mirar el mar por si divisaba algún navío, me apliqué desde entonces a organizar mi vida y a hacerla lo más confortable posible.

    He descrito ya mi vivienda, que era una tienda junto a la ladera rocosa, rodeada de un fuerte vallado de estacas y cables al que puedo llamar ahora muro porque del lado exterior le puse una base de tierra con césped que alcanzaba a dos pies de alto; más tarde —pienso que un año y medio después— agregué unas vigas y cabrias que iban de la empalizada hasta las rocas, e hice un techo con ramas de árbol y todo aquello que pudiera protegerme mejor de las lluvias, que en ciertas épocas del año caían con gran violencia.

    Ya he dicho que había puesto todos mis efectos dentro de la empalizada y en la caverna. Al principio estaban tan revueltos, apilados sin orden ni cuidado, que ocupaban casi todo mi sitio, no dejándome lugar libre. Me puse entonces a agrandar la caverna, siéndome fácil porque se trataba de una roca arenosa que cedía fácilmente. Ya en aquel entonces estaba seguro de que no había fieras en la isla, y ahondando la cueva hacia la derecha hice un túnel que formaba una salida más allá de la empalizada, lo cual me permitiría salir y entrar de lo que llamaríamos la parte trasera de mi casa y a la vez depósito de efectos.

    Pude luego dedicarme a fabricar aquellas cosas que más falta me hacían, como por ejemplo una mesa y una silla, sin las cuales no podría gozar de las pocas comodidades que tenía en el mundo, ya que era difícil escribir o comer agradablemente sin una mesa. Nunca había manejado una herramienta en mi vida, pero con tiempo, aplicación y perseverancia descubrí que si hubiera tenido los elementos necesarios habría podido fabricar cuanto me faltaba. Así y todo hice muchas cosas sin herramienta alguna, y otras con la sola ayuda de una azuela y un hacha, aunque con infinitas dificultades. Si, por ejemplo, necesitaba un tablón, no me quedaba otro remedio que derribar un árbol, ponerlo en un caballete y hacharlo por ambos lados hasta darle el espesor de un tablón, y lo pulía luego convenientemente con la azuela. Con este método sólo sacaba un tablón por árbol, pero como no encontraba otra manera de lograrlo me armaba de paciencia ante la enormidad de tiempo que me llevaba la sola obtención de una tabla. Cierto que mi tiempo y mi trabajo nada valían allí, y tanto me daba emplearlos de un modo que de otro.

    Así fabriqué en primer lugar una mesa y una silla, aprovechando los pedazos de tabla que trajera del barco. Después, cuando obtuve algunos tablones de la manera ya descrita, hice estantes de pie y medio de ancho, uno sobre otro, a lo largo de las paredes de mi cueva, que servían para poner mis herramientas, clavos y herrajes teniendo todo clasificado y puede decirse que al alcance de la mano. Clavé soportes en las paredes para colgar mis escopetas y lo que en esa forma quedara cómodo, tanto que si alguien hubiera podido ver mi cueva le hubiera parecido un depósito general de objetos necesarios. Tenía todo tan al alcance de la mano que me encantaba ver cada cosa en orden y, más que nada, descubrir que mi provisión era tan abundante.

    Fue entonces cuando empecé a llevar un diario de mis tareas cotidianas. En un principio había estado demasiado ocupado, no solamente con mi trabajo sino con los confusos pensamientos que pasaban por mi mente, y mi diario hubiera aparecido lleno de cosas torpes y melancólicas. Pero habiendo superado en alguna medida ese estado de ánimo y sintiéndome seguro en mi casa, dueño de una mesa y silla y con todo lo que me rodeaba aceptablemente bueno, empecé a llevar mi diario, del cual he de dar aquí una copia —aunque a veces resulte repetición de lo ya dicho— hasta el punto en que, por falta de tinta, hube de interrumpirlo.



    FRAGMENTOS DEL DIARIO



    4 de noviembre — Empecé esta mañana a reglar mis horas de trabajo, de salidas, sueño y esparcimiento. Todas las mañanas partía con mi escopeta por espacio de dos o tres horas, siempre que no lloviera; luego trabajaba hasta las once más o menos, comía, y me echaba a dormir de doce a dos por ser intolerable el calor a tales horas. A la tarde volvía a trabajar. La tarea de este día y del siguiente fue dedicada enteramente a la construcción de la mesa, sintiéndome todavía muy torpe como carpintero, aunque con el tiempo llegué a ser tan diestro como cualquier otro.



    5  de noviembre — Salí con la escopeta y mi perro, matando un gato montes. Piel muy suave, pero carne imposible de comer. Desollaba los animales que había cazado para aprovechar después las pieles. Volviendo por la playa vi toda clase de aves marinas desconocidas para mí; me sorprendieron y casi asustaron a dos o tres focas, y mientras trataba de verificar qué clase de animales eran las vi huir en el mar.



    6  de noviembre — Después del paseo matinal seguí trabajando en la mesa y la terminé, aunque no a mi gusto; muy pronto me di cuenta de cómo podía mejorarla.



    7  de noviembre — El buen tiempo parece mantenerse firme. Pasé desde el 7 hasta parte del 12 (salvo el domingo 11) trabajando en la construcción de una silla, y logré por fin darle una forma aceptable aunque no de mi gusto; varias veces la deshice a mitad de trabajo.



    NOTA: Pronto descuidé la observancia del domingo, porque olvidándome de señalarlos en el poste con una marca mayor perdí la noción de los mismos.



    17 de noviembre — Empecé a excavar la roca detrás de mi tienda para tener más lugar de almacenamiento.



    NOTA: Tres cosas me hacían gran falta en esta tarea: un azadón, pala y una carretilla o espuerta, de manera que desistí de mi trabajo y busqué la manera de procurarme aquellas herramientas necesarias o sus equivalentes. A manera de azadón utilicé una de las barras de hierro, que aunque muy pesadas daban buen resultado, pero subsistía la cuestión de la pala. Esto me era tan necesario que sin ella no podía seguir la excavación, aunque ignoraba cómo podría fabricar una.



    18 de noviembre — Explorando los bosques encontré un árbol de esa madera que en Brasil llaman palo de hierro por su gran dureza; si no era el mismo se le parecía mucho, tanto que con enorme trabajo y estropeando la hoja de mi hacha conseguí cortar un pedazo y traerlo a casa no sin esfuerzos porque pesaba enormemente.

    La extraordinaria dureza de esta madera me obligó a perder mucho tiempo mientras poco a poco le iba dando la forma de una pala, con un mango igual al que usamos en Inglaterra; desgraciadamente, como no tenía hierro para guarnecer la extremidad más ancha, bien poco habría de durarme su filo.

    Todavía me faltaba la espuerta. No sabía cómo arreglármelas para hacer una canasta careciendo de varillas de mimbre lo bastante flexibles, y no habiendo descubierto todavía su existencia en la isla. Para fabricar una carretilla encontraba la dificultad en la rueda, ya que no veía modo de construirla, sin contar que tampoco podría arreglármelas para forjar los soportes y el eje que debería sostener la rueda. Renunciando a esa idea me conformé con transportar lo que sacaba de la caverna en una especie de artesa como las que los albañiles emplean para llevar el mortero.



    23 de noviembre — Volví a mi excavación, dueño ya de herramientas suficientes para ello, y trabajé dieciocho días consecutivos, tanto como me lo permitían mis fuerzas y el tiempo disponible, en ensanchar y profundizar la caverna para que mis efectos cupieran adecuadamente.



    NOTA: Durante todo este tiempo ensanché la caverna con intención de que me sirviera al mismo tiempo de almacén, cocina, comedor y bodega. Preferí seguir durmiendo en la tienda, salvo cuando en la estación de las lluvias los chaparrones eran tan fuertes que terminaban por mojarme, lo que me llevó más adelante a techar el espacio dentro de la empalizada con largas pértigas que apoyaban contra la roca, y que fui cubriendo con espadañas y grandes hojas de árboles, como un techo de paja.



    10 de diciembre — Empezaba a considerar concluida mi caverna cuando debido acaso a la excesiva anchura se produjo un hundimiento lateral, cayendo tanta piedra que llegó a atemorizarme y no sin motivo, pues si hubiera estado debajo no habría necesitado sepulturero. Este desastre me obligó a reanudar el duro trabajo, sacando fuera lo que se había desplomado y asegurando el techo para que no volara.



    11 de diciembre — De acuerdo con lo decidido coloqué dos puntales contra el techo con dos tablas cruzadas en su extremidad. Terminé la tarea al día siguiente, y agregando luego otros postes con tablones que sostuvieron el techo pude asegurarlo firmemente una semana más tarde. Como los postes habían sido dispuestos en hileras, me sirvieron para establecer distintas habitaciones en mi nueva casa.



    17 de noviembre — Desde la fecha hasta el veinte estuve fijando estanterías y poniendo clavos en los postes para colgar diversas cosas; ya empiezo a encontrar ordenada mi casa.



    20 de diciembre — Llevé al interior de la cueva todas mis cosas, y comencé a amueblarla poniendo algunos tablones a modo de aparador para las vituallas. Empiezan a faltarme tablas, pero aún alcanzaron para hacer otra mesa.



    24 de diciembre — Llovió todo el día y toda la noche, sin que pudiera salir.



    25 de diciembre — Llovió todo el día.



    26 de diciembre — Cesó la lluvia, refrescando la tierra de un modo muy agradable.



    27 de diciembre — Maté una cabra y herí a otra, alcanzando a apresarla y llevarla a casa sujeta con una cuerda. Allí le entablillé y vendé la pata rota.



    NOTA: Tanto la cuidé que se mejoró, quedándole la pata igual que antes. A causa de mis cuidados se domesticó, comía del césped en torno a mi casa y no se alejaba mucho. Por primera vez pensé en la posibilidad de criar animales domésticos para que no me faltaran alimentos el día en que se concluyera la pólvora.

    28, 29 y 30 de diciembre — Fuertes calores y ninguna brisa, de modo que apenas salía al atardecer en busca de alimentos. Pasé este tiempo ordenando mis cosas.



    1.o de enero — Todavía muy caluroso, por lo que salía temprano y al anochecer con la escopeta, descansando a mitad del día. Al entrar esta tarde en los valles que conducen al centro de la isla hallé gran cantidad de cabras, aunque tan asustadizas que era difícil acercarse. Se me ocurrió  L que acaso mi perro fuera capaz de echarlas hacia mi lado.



    2  de enero — Llevé al perro y lo solté a las cabras, pero contra lo que esperaba le hicieron frente, y él advirtió el peligro sin animarse a avanzar.



    3  de enero — Empecé el muro o empalizada, y en previsión de algún posible ataque traté de darle una extraordinaria solidez y tamaño.



    NOTA: Como esto ha sido ya narrado, omito todo lo que a su respecto contiene el diario. Basta observar que la tarea me llevó desde el 3 de enero hasta el 14 de abril, y durante este tiempo construí, terminé y mejoré aquel vallado que sólo tenía sin embargo veinticuatro yardas de largo y formaba un semicírculo desde un punto de la pared rocosa hasta otro situado a ocho yardas más allá, con la entrada de la caverna en el justo medio.



    En todo este tiempo trabajé intensamente a pesar de estorbármelo la lluvia muchos días y a veces semanas enteras; me perseguía la idea de que no iba a estar bien seguro hasta que concluyera la empalizada. Es difícil creer lo que me costó cada cosa, en especial cortar madera del bosque y clavarla en tierra, ya que había hecho estacas más grandes de lo que hubiera sido necesario.

    Concluido el vallado, su parte exterior doblemente protegida por un terraplén de tierra con césped de bastante altura, me persuadí de que si alguien desembarcaba en la isla no se daría cuenta de que era una habitación humana; y tal cosa me fue harto útil, como lo comprobé más adelante.

    Diariamente iba al bosque de caza, salvo cuando llovía, y con frecuencia realizaba algún descubrimiento ventajoso. Una vez hallé una especie de palomas silvestres que no anidaban en los árboles como la torcaz sino que formaban especie de palomares en los agujeros de las rocas. Traté de domesticar algunos pichones y lo conseguí, pero cuando fueron mayores no pude impedir que se volaran, probablemente por falta de alimento, que yo no tenía para darles; con todo, iba frecuentemente a sus nidos y me apoderaba de los pichones, cuya carne era excelente.

    A medida que atendía mis cosas fui descubriendo todo lo que me faltaba y que a primera vista me parecía imposible de hacer o procurarme. Por ejemplo, comprendí que no podría construir un tonel con aros; tenía uno o dos barriles pequeños, como ya he dicho, pero jamás pude aprovecharlos de modelo para uno mayor, aunque pasé semanas probando. Era imposible colocar los fondos y unir las duelas con suficiente justeza para que no dejaran escapar el agua, de manera que por fin abandoné la tentativa.

    En segundo término carecía de velas; la falta de luz me obligaba a acostarme apenas oscurecía, lo que allí ocurre a eso de las siete. Me acordaba del pedazo de cera con el cual hice velas durante mi aventura en Africa, pero ahora el único remedio a mi alcance era aprovechar la grasa de las cabras que mataba; fabriqué un platillo de arcilla que puse a cocer al sol, y agregándole un pabilo de estopa conseguí hacer una lámpara que daba una luz mucho más débil y vacilante que la de una vela.

    Mientras me ocupaba en todo esto, encontré al registrar entre mis cosas un pequeño saco que, como ya lo he dicho antes, había contenido granos para el alimento de las aves que teníamos a bordo, pero que acaso había sido llenado en el viaje anterior, cuando el navío vino de Lisboa. Lo poco que quedaba en el saco aparecía devorado por las ratas, y sólo encontré polvo y cáscaras, de manera que precisando el saco para otro uso —creo recordar que para poner pólvora en él cuando me asustó el episodio del rayo— fui a sacudir las cáscaras a un lado de la empalizada, junto a las rocas. Esto sucedía un poco antes de las grandes lluvias ya citadas, y muy pronto olvidé que había vaciado allí los restos del saco, cuando aproximadamente un mes más tarde vi surgir de la tierra unos tallos verdes que me parecieron de una planta desconocida; pero mi asombro fue inmenso al notar poco después que las plantas echaban diez o doce espigas que reconocí ser de cebada, el mismo tipo de cebada que se cultiva en Europa, sobre todo en Inglaterra.

    Podéis imaginar cómo habré cuidado aquellas espigas, que recogí a su debido tiempo, es decir a fines de junio. Me resolví a sembrar todo el grano, confiando que con el tiempo tendría bastante para hacer pan, pero recién al cuarto año pude permitirme separar algo de la cosecha para alimentarme, y esto con mucha prudencia, como relataré luego, pues perdí casi todo lo que sembrara la primera vez, no habiendo calculado bien la época adecuada; lo hice antes de la estación de sequía, por lo cual se malogró todo o casi todo, como contaré en su debido tiempo.

    Además de la cebada habían crecido allí veinte o treinta tallos de arroz que cuidé con la misma atención, pensando que de su grano podría hacer pan u otro alimento, y descubrí el modo de cocerlo sin necesidad de horno, aunque más adelante lo tuve. Pero volvamos a mi diario.

    Trabajé hasta la extenuación durante esos tres o cuatro meses para terminar la empalizada, y el 14 de abril quedó cerrada, y podía entrar y salir de ella por una escalera que no dejaba huellas exteriores de que allí hubiera una habitación humana.



    16 de abril — Terminé la escalera con la cual trepaba a la empalizada, retirándola luego y dejándola del lado de adentro. Esto me aislaba totalmente y nadie podía llegar hasta mí a menos que escalara la pared.



    Al día siguiente de concluir el trabajo estuve a punto de que todo se malograra y hasta me vi en peligro de muerte. Ocurrió que trabajando detrás de mi tienda, justo delante de la entrada de la cueva, me espantó de improviso algo horroroso: el material que formaba el techo de mi caverna empezó a desplomarse mientras tierra y piedras de la ladera de la colina caían sobre mí; dos de los postes que pusiera como puntales se quebraron con un ruido terrible. Me asusté mucho, pero en ese instante no tuve la visión de lo que verdaderamente sucedía, y me pareció tan sólo que el techo de la caverna se desplomaba, como ya había ocurrido parcialmente antes; temeroso de ser alcanzado corrí entonces a la escalera y pasé por encima de la empalizada, temiendo a cada instante que las rocas de la colina cayeran sobre mí aplastándome. Tan pronto pisé suelo firme me di cuenta de que se trataba de un violento terremoto; tres veces tembló la tierra con intervalo de ocho minutos, y sus sacudidas eran tales que hubiera derribado el más sólido edificio de la tierra. Vi que un trozo de roca, a una media milla de donde me hallaba, caía hacia el mar con el ruido más espantoso que haya oído en mi vida. El mar estaba también revuelto por el cataclismo, y me parece que las sacudidas eran aún más fuertes allí que en la isla.



    Después de la tercera conmoción hubo un rato de calma y empecé a cobrar valor: sin embargo, no me animaba a trasponer la empalizada por miedo a ser enterrado vivo, y me senté en el suelo profundamente abatido y desconsolado, sin saber qué hacer. En ningún momento tuve el menor pensamiento religioso, salvo la común imploración: « ¡Apiádate de mí, Señor!», y cuando cesó el terremoto también dejé de pronunciarla.



    Mientras permanecí allí reparé en que el cielo se encapotaba como si fuera a llover. Pronto comenzaron ráfagas cada vez más violentas, y media hora más tarde se desencadenaba un terrible huracán. El océano estaba cubierto de espumas, rompía con violencia en la playa, eran arrancados los árboles de raíz y aquella horrorosa tormenta duró casi tres horas antes de calmar, y sobrevino una profunda tranquilidad tras la cual principió a llover copiosamente.

    Me vi obligado a volver a la cueva, aunque lleno de temor porque me parecía que iba a desplomarse sobre mí. La lluvia era tan violenta que para evitar que la acumulación del agua dentro de mi fortificación concluyera por inundar la cueva, tuve que hacer un agujero en la muralla como vía de escape. Me quedé allí cobrando más coraje a medida que pasaba el tiempo y los temblores no se repetían. Buscando reanimar mis ánimos, que por cierto lo necesitaban mucho, fui a mi pequeño almacén y bebí un poco de ron, cosa que hacía siempre con mucha prudencia, sabedor de que no podría reemplazarlo cuando se concluyera.

    Llovió toda esa noche y gran parte del día siguiente, de modo que no pude salir, pero sintiéndome ya repuesto medité lo que me convendría hacer, llegando a la conclusión de que si la isla estaba sujeta a tales terremotos no me convenía vivir en una caverna; era mejor levantar mi choza en un sitio abierto que circundaría con una empalizada como lo hiciera aquí, para asegurarme contra bestias o seres humanos; porque si osaba quedarme en la cueva terminaría por morir enterrado vivo.

    Me resolví, pues, a mover mi tienda del sitio en que estaba, justamente debajo de la escarpada ladera de la colina, ya que indudablemente sería sepultado al producirse un nuevo terremoto. Pasé los días siguientes —19 y 20 de abril— en estudiar dónde y cómo mudaría mi habitación.

    El miedo de ser aplastado por un alud no me dejaba dormir tranquilo, pero menos aún quería hacerlo en sitio descubierto y sin la protección de la empalizada. Cuando miraba en torno y veía cuan ordenadas estaban mis cosas, lo bien ocultas y a salvo que se encontraban, me dolía mucho la idea de abandonar el sitio.

    Se me ocurrió entonces que me llevaría mucho tiempo la nueva instalación, y que mientras tanto era mejor correr el riesgo de seguir viviendo allí hasta que hubiera encontrado un lugar apropiado y puesto en condiciones de defensa para mudarme a él. Ya resuelto, decidí que empezaría con toda la rapidez posible a levantar una empalizada circular en el sitio elegido, haciéndola con estacas y cables como la primera, y que una vez concluida pondría dentro mi tienda; pero entretanto decidí arriesgarme a permanecer en mi primera morada. Esto sucedía el veintiuno.



    22 de abril — A la mañana siguiente me dispuse a poner en práctica mis decisiones, pero el gran problema lo constituían las herramientas. Tenía tres grandes hachas, abundancia de hachuelas (que habíamos llevado en cantidad para el intercambio con los negros), pero de tanto cortar madera dura estaban llenas de muescas y sin filo. Tenía una piedra de afilar, pero no era posible hacerla dar vueltas al mismo tiempo que aplicaba las hojas; este problema me ocupó tanto tiempo como a un hombre de estado resolver una difícil situación política o a un juez la vida o muerte de un hombre. Por fin armé la rueda con un cable que la pusiera en movimiento con el impulso del pie, dejándome ambas manos libres.



    NOTA: Jamás había visto mecanismo igual en Inglaterra, o por lo menos no había observado su funcionamiento, aunque más tarde vine a saber que allí era muy común. Aparte de eso, el gran tamaño y peso de la piedra dificultaba mi tarea, de modo que perfeccionar la máquina me llevó una semana de trabajo.



    28 y 29 de abril — Pasé estos días afilando mis herramientas y tuve la alegría de que la máquina funcionara muy bien.






	\chapter{El diario de Robinson (\Romano{2})}





    Primero de mayo — Mirando hacia la playa de mañana a la hora del reflujo, vi un objeto bastante grande y semejante a un barril. Me acerqué y hallé un pequeño tonel y dos o tres pedazos del barco que el reciente huracán había tirado a la costa. Mirando hacia el casco mismo, me pareció que emergía del agua más que en días anteriores. Examiné el barril y vi que contenía pólvora, pero tan mojada que estaba dura como piedra. Lo hice rodar para alejarlo de las olas y me acerqué cuanto pude por la playa a fin de examinar el casco más de cerca.

    Cuando llegué a su lado noté que había cambiado extrañamente de posición. El castillo de proa, antes enterrado en la arena, estaba ahora a seis pies de elevación; la popa, que se había partido y separado del resto por la violencia del mar —poco después que yo cesara de explorarla—, estaba tumbada de lado y la arena se acumulaba de tal manera en aquella parte, hasta la popa, que pude llegar caminando a ella cuando antes debía nadar cerca de un cuarto de milla. Al principio me maravillé, pero pronto deduje que el cambio se debía al terremoto. Y como a causa de esto el barco estaba más destrozado que antes, diariamente llegaban objetos a la playa que el viento y el oleaje sacaban del navío y depositaban en tierra.

    Esta novedad apartó mis pensamientos del proyecto de mudanza, y empecé a buscar la manera de introducirme en el barco; pero mi desilusión fue grande al comprobar que el casco estaba lleno de arena. Decidí, sin embargo, sacar todos los pedazos que pudiera, ya que sin duda me serían de utilidad.



    3 al 17 de mayo — Fui diariamente al casco, extraje gran cantidad de madera, planchas y tablones, así como unas trescientas libras de hierro.



    24 de mayo — Trabajé hasta hoy en el casco del barco, aflojando con la palanca diversas partes que flotaron en cuanto se levantó viento, pero como por desgracia soplaba de la costa nada llegó a tierra salvo algunas maderas y un barril que contenía salazón de cerdo del Brasil, tan estropeado por el agua que no era de ningún provecho.

    Seguí trabajando en el casco hasta el 15 de junio, salvo los momentos dedicados a cazar, que elegía a las horas de marea alta para tener tiempo libre durante el reflujo. Ya entonces había obtenido suficiente madera y herraje como para construir un buen bote si hubiera sabido cómo. También saqué poco a poco y en muchos pedazos casi cien libras de plomo.



    16 de junio — Yendo hacia la playa encontré una enorme tortuga. Era la primera que veía, más por mala suerte que por otra cosa, ya que si hubiera ido al otro lado de la isla habría encontrado cientos de ellas, como lo descubrí más tarde; pero acaso me hubiera salido aquello demasiado caro.



    17 de junio — Pasé el día cocinando la tortuga, dentro de la cual había sesenta huevos. Su carne me pareció en esa ocasión la más deliciosa que hubiera probado en mi vida, ya que desde mi arribo a tan triste lugar mi único alimento habían sido las cabras y las aves.



    18  de junio — Llovió el día entero y me quedé dentro. Esta vez encontré que el agua era muy fría y sentí escalofríos, lo que me pareció muy raro en estas latitudes.



    19 de junio — Muy enfermo y temblando como si hiciese mucho frío.



    20 de junio — No dormí en toda la noche; terrible dolor de cabeza, fiebre.



    21  de junio — Muy enfermo, mortalmente asustado con la idea de sentirme tan mal y no tener ayuda alguna. Rogué a Dios .por primera vez desde la tempestad en Hull, pero apenas recuerdo lo que dije y por qué lo dije. Mis pensamientos eran confusos.



    22  de junio — Algo mejor, pero lleno de aprensiones por mi enfermedad.



    23  de junio — Otra vez muy mal; tiritando de frío y luego con una fuerte jaqueca.



    24 de junio — Mucho mejor.



    25 de junio — Violenta calentura. La crisis duró siete horas, con alternancias de calor y frío y luego una copiosa transpiración.



    26 de junio — Mejor. No teniendo qué comer salí con la escopeta, sintiéndome muy débil. Con todo maté una cabra, la traje penosamente a casa y luego de cocer un pedazo lo comí. Hubiera preferido hervirlo y hacer un poco de caldo, pero no tenía olla.



    27  de junio — Tan violenta calentura que estuve el día entero en cama sin comer ni beber. Me parecía que iba a morir de sed, sintiéndome demasiado débil para levantarme en busca de agua. Rogué otra vez a Dios, pero en mi delirio e ignorando lo que debía decir sólo atinaba a implorar: « ¡Señor, apiádate! ¡Señor, protégeme! ¡Ten compasión de mí, Señor!» Estuve así continuamente por dos o tres horas hasta que la calentura cedió y quedé dormido; me desperté ya entrada la noche. Me sentía mejor, pero muy débil y con una sed continua. No tenía agua en mi habitación de modo que hube de esperar hasta la mañana, durmiendo entretanto. Mientras dormía tuve un sueño terrible.

    Soñé que estaba sentado en el suelo, más allá de la empalizada, donde permanecí mientras la tormenta arreciaba después del terremoto, y que veía un hombre que se acercaba en una oscura nube, envuelto en un halo de fuego que iluminaba el terreno; brillaba de tal manera que apenas podía soportar su presencia. Su aspecto era tan imponente que no hay palabras para describirlo. Cuando posé los pies en tierra creí que el suelo temblaba con un nuevo terremoto, y el aire entero pareció llenarse, para mi mayor espanto, de ígneas lenguas. Apenas había descendido cuando se adelantó hacia mí, con una lanza en la mano para matarme; de pie en una eminencia, oí que me hablaba con voz tan terrible que es imposible tratar de describir el espanto que me produjo. Todo lo que alcancé a entender fue esto: «puesto que lo que has visto no te ha movido a arrepentirte, ahora morirás». Y levantó la lanza para atravesarme con ella.

    Nadie que lea este relato esperará que yo sea capaz de describir el espanto que pasó mi alma ante tan terrible visión. Aunque solamente se tratara de un sueño, soñé también el espanto que me produjo. Y menos aún podría dar una idea de la impresión que quedó en mí una vez que hube despertado y comprendido que se trataba de un sueño.

    No tenía, ¡ay!, instrucción religiosa; de todo lo que la bondad de mi padre me había inculcado apenas quedaba nada tras ocho años de errantes extravíos y continuo contacto con aquellos que, como yo, eran malos y profanos en máximo grado. No recordaba haber tenido en todo aquel tiempo un solo pensamiento que tendiera a la contemplación de Dios o a un examen severo de mi propia conducta.

    Es verdad que cuando me salvé del naufragio y tuve la certeza de que todos habían muerto salvo yo, pasé por un momento de éxtasis y por tales transportes que, de haberme asistido la gracia de Dios, me hubieran llevado a una verdadera gratitud; pero todo terminó donde había principiado —un simple arranque de alegría por sentirme aún vivo— sin que eso me moviera a reflexionar sobre la bondad de la mano que, preservándome, había guardado mi vida mientras perecían todos los demás. No se me ocurrió pensar por qué la Providencia había sido generosa conmigo; tuve sólo la vulgar alegría que todo marino salvado de un naufragio se apresura a ahogar en un vaso de ponche para olvidarla de inmediato; y toda mi vida había sido así.

    Aun el terremoto, bien que nada podía ser más terrible que sus manifestaciones o más revelador del invisible poder que rige tales fuerzas, no me había impresionado más que mientras duró, y lo olvidé casi en seguida. Tenía tan poca noción de Dios y su justicia, olvidaba a tal punto que mi miserable condición podía ser obra de Su mano, que se hubiera creído que estaba viviendo en la prosperidad. Pero cuando enfermé y los temores de la muerte se presentaron a mis ojos; cuando mis ánimos cedieron ante la fuerza de tan grave mal y mi resistencia se agotó por la fiebre, la conciencia tanto tiempo dormida empezó a despertarse y a hacerme reproches sobre mi pasada vida, por la cual había provocado a la justicia de Dios para que me abatiera con tan duros golpes, siendo mi empecinada maldad la causa de su severo castigo.

    «Ahora —dije en voz alta— van a cumplirse las palabras de mi querido padre. La justicia de Dios me ha fulminado y no tengo nadie que me ayude o me escuche. Rechacé la gracia de la Provincia que generosa me había colocado en una condición de vida donde habría tenido comodidad y calma; pero no fui capaz de verlo, ni siquiera a través de lo que me decían mis padres. Rehusé su ayuda y asistencia que me hubieran hecho adelantar en la vida, dándome todo lo que podía necesitar; y ahora me veo precisado a luchar contra fuerzas que la misma naturaleza no podría vencer, sin compañía, sin socorro, sin consuelo, sin consejo...» Y grité con todas mis fuerzas: « ¡Señor, ayúdame en mi aflicción!»

    Esta fue la primera plegaria, si así puedo llamarla, que elevaba al Cielo en muchos años. Pero vuelvo a mi diario.



    28 de junio — Algo aliviado por el profundo sueño, y encontrando que había pasado el acceso, conseguí levantarme todavía aterrado por el recuerdo de lo que había soñado; alcancé, sin embargo, a pensar que la calentura volvería al siguiente día y que ahora era momento de procurarme agua y alimentos que necesitaría después. Llené de agua una gran damajuana y la puse sobre la mesa, al alcance de mi lecho; para quitarle lo que pudiera causarme más fiebre mezclé en ella un cuarto de pinta de ron. Luego asé un pedazo de carne de cabra, pudiendo comer unos pocos bocados. Estaba tan débil que apenas pude moverme; me agobiaban la tristeza y el temor de que la calentura volviera al día siguiente. Por la noche cené tres huevos de tortuga que cocí enteros en las cenizas; y hasta donde alcanzo a recordar ésa fue la primera comida para la cual solicité la bendición de Dios.

    Quise caminar un poco después de la cena, pero apenas podía sostener la escopeta, que jamás abandonaba al salir; a poco de andar me senté en tierra, mirando hacia el mar que se extendía sereno a lo lejos. Y allí se me ocurrieron estos pensamientos: que todo cuanto me ocurría era por la voluntad de Dios; que había sido llevado a tan miserable situación por Su decisión, puesto que El tenía poder no sólo sobre mí sino sobre todo cuanto ocurría en el universo. De inmediato me pregunté: « ¿Por qué Dios ha hecho esto conmigo? ¿Cuál ha sido mi culpa para ser tratado así?»

    Mi conciencia me impidió seguir más adelante en tales interrogaciones, como si fueran blasfemias, y me pareció que hablaba dentro de mí una voz: « ¡Miserable!», decía. « ¿Preguntas lo que has hecho? ¿Por qué no miras tu vida malgastada y te preguntas más bien qué es lo que no has hecho? ¿Por qué no preguntas la razón de no haber perecido mucho antes, por qué no te ahogaste en la rada de Yarmouth o te mataron en la pelea cuando el pirata de Sallee apresó tu barco? ¿Por qué las bestias salvajes no te devoraron en la costa africana, o te ahogaste aquí donde pereció toda la tripulación? ¿Y te atreves todavía a preguntar qué has hecho?»

    Quedé tan abatido por semejantes reproches que, sin encontrar una sola palabra que responder, me levanté triste y pensativo y volví a mi tienda como dispuesto a dormir; pero me sentía demasiado confundido para que el sueño me venciera, de modo que me dejé caer en la silla y encendí la lámpara, pues ya era casi de noche. El temor de que volviera la fiebre me asaltaba, y entonces recordé que los brasileños no toman otra medicina que su propio tabaco para cualquier clase de enfermedades; yo guardaba un pedazo en uno de los arcones, ya curado, y otros que todavía estaban verdes.

    Fui al arcón, sin duda guiado por el Cielo, ya que allí encontré a la vez remedio para el cuerpo y para el alma. Al abrirlo en busca del tabaco hallé los pocos libros que salvara del naufragio, y entre ellos una de las Biblias que antes mencionara y que hasta ese momento no había mirado por falta de tiempo y de inclinación. Tomándola, la traje juntamente con el tabaco a mi mesa.

    Ignoraba la manera de emplear el tabaco para curarme y ni siquiera estaba seguro de que me hiciera bien; pero con la idea de acertar en alguna forma me propuse tomarlo de distintos modos. Ante todo corté un pedazo para mascar, lo que me produjo gran embotamiento, ya que el tabaco era fuerte y yo no tenía el hábito. Puse otra porción en una cantidad, de ron para beberlo al acostarme; y finalmente, quemando algunas hojas sobre el fuego, me incliné sobre él y aspiré profundamente el humo, resistiendo lo más posible el calor y la sofocación.

    En los intervalos de este tratamiento había abierto mi Biblia y empezado a leer, pero tenía la cabeza demasiado mareada por el tabaco para seguir con atención la lectura; al abrir el libro al azar, las primeras palabras que vieron mis ojos fueron: «Invócame en los días de aflicción, y yo te libraré, y tú me alabarás.»

    Como era tarde y los efectos del tabaco se sentían con fuerza, noté que el sueño me vencía, de manera que dejando encendida la lámpara en la cueva por si necesitaba algo durante la noche me fui a la cama. Pero antes hice lo que no había hecho nunca; me arrodillé para rogar a Dios que cumpliera su promesa de ayudarme si yo lo invocaba en los días de aflicción. Cuando hube terminado mi torpe y simple plegaria, bebí el ron donde había puesto tabaco; la bebida era tan fuerte y su gusto tan desagradable que apenas pude tragarla y caí en el lecho. De inmediato noté que la poción me mareaba, y dormí tan profundo sueño que no desperté hasta las tres de la tarde del siguiente día. Incluso llegué a pensar más adelante que en realidad había dormido todo ese día y la noche, hasta la tarde del tercero, porque de otro modo no me explico cómo pude saltar un día en la cuenta que llevaba, error que descubrí años "después. Sin duda, de haber hecho mal las líneas o puesto una sobre otra habría perdido más de un día. En fin, me faltó uno en la cuenta y nunca supe cómo.

    Al despertarme me sentía otro hombre, aliviado y con el espíritu animoso. Noté al levantarme que estaba mucho más fuerte que el día anterior, tenía apetito y la fiebre no mostraba señales de volver. En suma, no tuve acceso al día siguiente y continuó la mejoría; esto era el 29 del mes.

    El 30 estaba ya restablecido y me animé a salir con la escopeta aunque sin alejarme mucho. Maté una o dos aves marinas parecidas a los ánades y las traje a casa aunque sin mucha disposición para comerlas, de manera que me sustenté con algunos huevos de tortuga, que eran excelentes. Por la noche renové la medicina que imaginaba me había hecho tanto bien, o sea la infusión de tabaco con ron, pero bebí una dosis más pequeña, sin masticar aparte tabaco ni aspirar el humo. Al día siguiente ya no me sentía tan bien como esperaba, porque tuve algunos escalofríos de fiebre, aunque en pequeña escala.

    4 de julio — Por la mañana abrí la Biblia empezando a leer el Nuevo Testamento, dispuesto a hacer lo mismo todas las mañanas y las noches, sin proponerme un determinado número de capítulos sino llegar cada vez hasta donde mis pensamientos fueran claros.

    Empecé a interpretar el pasaje ya mencionado —«Invócame y te libraré»— en un sentido distinto del que antes le diera; porque hasta ese momento mi concepto de la liberación se refería únicamente al cautiverio en que me hallaba. Cierto que vivía libre en una isla, pero para mí era una cárcel en el más duro sentido de la palabra. Ahora principié a imaginar otro modo de libertad, al contemplar con horror mi pasada vida, y mis pecados surgieron tan terriblemente ante mí que mi alma sólo ansiaba de Dios liberación de ese insoportable peso de culpas que la privaba de toda alegría.

    Pronto mi espíritu se sintió más aligerado, aunque las condiciones de mi vida fueran las mismas de antes; llevados mis pensamientos por la lectura de la Biblia y la oración a regiones que jamás habían alcanzado en su vuelo, un profundo alivio fue surgiendo en mí como jamás lo conociera anteriormente. Y pues al mismo tiempo recobraba la salud y mis fuerzas volvían, me consagré a procurarme todas las cosas necesarias, haciendo a la vez una vida tan regular como fuera posible.

    Del 4 al 14 de julio di pequeños paseos con mi escopeta sin alejarme mucho, como convenía a un hombre que recobra lentamente las energías después de una enfermedad; difícil es imaginar lo débil que me sentía al comienzo. El tratamiento que he descrito más arriba era ciertamente original y acaso nunca habría curado antes una calentura; es por eso que no recomiendo a nadie que lo intente a su turno. Evidentemente me libró de la fiebre, pero acaso contribuyó a debilitarme tanto, pues incluso sufrí cierto tiempo de convulsiones nerviosas y musculares.

    Llevaba ya más de diez meses en tan triste isla, y toda posibilidad de rescate parecía imposible; yo estaba firmemente convencido de que jamás un pie humano había pisado antes ese sitio. Fue entonces cuando, después de terminar mi vivienda del modo que me pareció más adecuado, se me ocurrió hacer una exploración completa de la isla para descubrir aquellos productos naturales que me resultaran útiles.

    Desde el 15 de julio principié a recorrer la isla con tal fin, yendo ante todo a la ensenada donde, como he contado, mis balsas fondearon con su cargamento. Descubrí que a dos millas corriente arriba la marea ya no alcanzaba a penetrar y sólo había un arroyuelo de aguas límpidas y frescas; pero como estábamos en la estación seca apenas traja aguas para formar una corriente.

    A la orilla de este riacho encontré hermosas sabanas, vastas llanuras cubiertas de verdes pastos; en las partes más elevadas, ya cerca de las mesetas donde se hubiera supuesto que jamás alcanzaba el agua, hallé una gran cantidad de tabaco que crecía vigorosamente, así como otras diversas plantas desconocidas para mí, que acaso fueran de gran utilidad, aunque no podía aprovecharlas por mi ignorancia.

    Busqué entre ellas alguna raíz de cazabe o yuca, con la cual los indios de aquellas latitudes fabrican su pan, pero no vi ninguna. Había grandes plantas de áloe, cuya utilidad desconocía entonces, y mucha caña de azúcar en estado silvestre y, por lo tanto, poco aprovechable. Me contenté ese día con tales descubrimientos y volví meditando cómo podría arreglármelas para conocer las virtudes de las plantas y frutos que iba descubriendo. Desgraciadamente no arribé a ninguna conclusión, porque tan poco observador había sido mientras viví en el Brasil que no sabía nada de sus productos naturales, o tan poco que apenas podía ayudarme en mi presente desgracia.

    Al día siguiente," 6 de julio, seguí el mismo camino y avanzando más allá encontré que el arroyo y las sabanas se iban perdiendo y que la región era más boscosa. Hallé diferentes frutos, en especial melones en abundancia y uvas entre los árboles. Las viñas habían crecido entremezcladas en los árboles, y magníficos racimos ya maduros pendían de las ramas. Tan extraordinario descubrimiento me llenó de alegría, pero tuve cuidado de no excederme en la cantidad de uvas que comía porque recordaba lo sucedido en Berbería, donde muchos ingleses esclavos perecieron a causa de las fiebres y disentería que les produjo comer demasiada cantidad de esta fruta. Se me ocurrió que la mejor manera de aprovecharlas era ponerlas a secar al sol para conservar las pasas que tan grato me resultaría comer en las épocas en que ya no hubiera uvas maduras en las viñas. Pasé allí la noche sin regresar a mi casa, cosa que ocurría por primera vez desde que estaba en la isla. Recordando mi anterior precaución, trepé a un árbol apenas oscureció y dormí perfectamente; por la mañana mi exploración me llevó cuatro millas más allá, según juzgué por el largo del valle que se extendía hacia el norte, con una cresta de colinas al sur y al norte de donde yo me hallaba.

    Al fin de este recorrido vine a dar a un espacio abierto, y allí el terreno parecía descender hacia el oeste; un arroyuelo que nacía en las laderas de la colina bajaba en dirección contraria, es decir, al este; la región eran tan fresca, tan fértil y florida, que al ver ese derroche de vegetación se la hubiera tomado por un jardín en primavera.

    Exploré un lado de aquel delicioso valle, observándolo todo con secreto placer en el cual se mezclaba sin embargo la aflicción, y pensando que aquello era mío. Podía considerarme dueño y señor de esas tierras, con derechos incontestables, incluso el de legarlas si me parecía bien, al igual que cualquier lord de Inglaterra. Noté la abundancia de cocoteros, naranjos, limoneros y cidras, todos ellos silvestres y con muy pocos frutos. Las limas que recogí no sólo eran agradables de comer sino de buen tamaño, y desde entonces mezclé su jugo con agua y bebí con agrado ese refresco sano y tonificante.

    Tenía ya bastante que llevar a mi morada y proyectaba acumular suficientes uvas, limas y limones para que no faltaran en la próxima estación de las lluvias. Junté, pues, una cantidad considerable de uvas, otra más pequeña al lado y un montón de limas y limones en otro sitio; tomando luego una poca cantidad de cada clase me volvía a mi casa, resuelto a regresar con un saco o cosa parecida para llevarme el resto.

    El camino de vuelta insumió tres días, de modo que cuando llegué a casa (como debo llamar a mi tienda y mi cueva) las uvas se habían echado a perder, debido a su completa madurez y al peso de los racimos. Tuve, pues, que tirarlas sin probar ni una; las limas eran excelentes, pero por desgracia había traído muy pocas.

    Al día siguiente, 19 de julio, volví a ponerme en camino con dos pequeños sacos destinados a acarrear mi cosecha. Allá me esperaba la desagradable sorpresa de encontrar mis racimos de uva, tan perfectos y hermosos cuando los cortara, completamente estropeados, esparcidos por el suelo y en la mayor parte devorados. Comprendí que había animales salvajes en los alrededores, pero cuáles y cómo eran no pude imaginármelo.

    En la alternativa de juntar racimos para que se secaran al sol o llevármelos de inmediato en los sacos, y seguro de que en el primer caso los animales los devorarían y en el otro iban a aplastarse por su propio peso, tuve la siguiente idea: cortando gran cantidad de racimos los colgué en las ramas exteriores de los árboles para que se secaran al sol. Luego llené mis sacos con todas las limas y limones que eran capaces de contener y los traje conmigo.

    Cuando estuve de regreso de mi expedición, recordaba continuamente y con agrado la fertilidad de aquel valle y el excelente lugar en que estaba situado, a cubierto de tempestades, con agua dulce y bosques. No me faltó más para concluir que de los rincones de la isla, aquel donde yo tenía mi morada era desde todo punto de vista el peor. ¿Por qué no —pensé— cambiar de lugar, irme a un sitio tan seguro como el que ahora tenía, pero situado en aquella fértil y hermosa planicie de la isla?

    La idea me tentó durante mucho tiempo y no podía olvidar la belleza del valle, pero cuando lo pensé con más calma comprendí que iba a cometer un error. Me encontraba junto a la orilla del mar, donde había por lo menos una posibilidad de recibir algún socorro; la misma desgracia que me arrojara a la costa podía traer a otros infelices de la misma manera; y aunque tal cosa era improbable, si yo me alejaba e iba a encerrarme entre las colinas y los bosques del centro de la isla, me condenaba definitivamente a mi destino, tornando imposible lo que sólo habría sido improbable. Comprendí, pues, que era preciso quedarme donde vivía.

    Con todo, tan encantado estaba de aquellos lugares, que pasé en ellos buena parte de mi tiempo hasta final de julio, y aunque seguía firmemente decidido a no mudarme, levanté para mi comodidad una especie de enramada protegida a distancia por un fuerte vallado hecho de una doble hilera de altas estacas bien clavadas y con maleza en medio. Allí, sintiéndome seguro, solía quedarme dos o tres noches, empleando el mismo sistema de escalera para entrar y salir. Me complacía pensar que era dueño de una casa de campo, así como de otra junto al mar; y en aquellos trabajos pasé hasta principios de agosto.

    Había terminado mi empalizada y principiaba a cosechar los agradables frutos de ese trabajo cuando vinieron las lluvias obligándome a volver a mi primera morada, porque aunque en el valle tenía una tienda semejante a ésta, fabricada con una vela y bien tendida, me faltaba el abrigo de la colina para protegerme de tempestades, y la cueva para refugiarme cuando las lluvias fueran demasiado copiosas. He dicho que hacia principios de agosto terminé la enramada y principié a utilizarla. El 3 encontré que los racimos se habían secado perfectamente y eran ya muy buenas pasas, apresurándome a descolgarlos con el justo tiempo para impedir que las lluvias los estropearan privándome de lo mejor de mi alimentación invernal. Era dueño de más de doscientos magníficos racimos. Tuve el tiempo de llevarlos a mi cueva cuando se precipitaron las lluvias, y desde ese día, 14 de agosto, hasta mediados de octubre, llovió diariamente con más o menos fuerza y a veces con violencia tan extraordinaria que me obligaba a estarme en la cueva por espacio de muchos días.

    Durante aquel tiempo tuve la sorpresa de ver aumentar mi familia. Había lamentado mucho la pérdida de uno de mis gatos, al que suponía muerto o perdido, y no tuve la menor noticia de él hasta que para mi gran asombro apareció hacia fin de agosto acompañado de tres gatitos. Desgraciadamente con el tiempo se multiplicaron tanto que eran una calamidad, y me vi forzado a matarlos sin lástima o arrojarlos lejos de mi casa.

    Desde el 14 hasta el 26 de agosto llovió incesantemente, por lo que me cuidé de salir, temeroso de mojarme. Comenzó a faltar alimento en mi prisión, pero aventurándome a salir dos veces, la primera maté una cabra y la segunda, justamente el 26, encontré una enorme tortuga que fue un regalo para mí. Regulaba mis comidas en esta forma: un racimo de pasas de desayuno, como almuerzo un pedazo de carne de cabra o tortuga, asado, pues por desgracia no tenía olla para hervirlo o guisarlo, y dos o tres huevos de tortuga a modo de cena.

    Mientras estuve confinado por la lluvia trabajé varias horas diarias ensanchando mi cueva, cavando un túnel que desviaba poco a poco hacia un lado, hasta que vine a salir a la ladera de la colina y tuve una puerta que daba fuera de mi empalizada y me resultaba muy cómoda. Pero no me sentía con la tranquilidad de antes, porque hasta entonces mi morada había sido un recinto completamente cerrado, mientras que ahora cualquiera podía entrar por aquella puerta. No comprendía aún que el mío era un temor infundado, ya que el animal de mayor tamaño en la isla era la cabra.



    30 de setiembre — Llegó finalmente el triste aniversario de mi naufragio. Conté las marcas en el poste y vi que llevaba en la isla trescientos sesenta y cinco días. Consideré ese día como de ayuno, y lo dediqué a meditaciones religiosas.

    Hasta entonces no había observado nunca el domingo, ya que olvidándome de señalarlos con una línea más grande, perdí la cuenta de cuáles eran los días.

    Al sumarlos y comprobar que llevaba un año en tierra, dividí ese tiempo en semanas y consideré domingo el séptimo día de cada una, bien que hacia el final de mis cálculos di con un error de uno o dos días.

    Poco después de esto noté que la tinta escaseaba, de modo que preferí emplearla con cuidado y sólo escribir los acontecimientos dignos de mención, sin llevar una prolija crónica de cada cosa.

    La estación de las lluvias y la de sequía se alternaban regularmente y me habitué a dividirlas para tomar las necesarias precauciones.

    Esto lo aprendí a costa de una de las experiencias más desalentadoras que puedan imaginarse. Ya he dicho que había recogido las pocas espigas de cebada y arroz que tan maravillosamente crecieran sin el menor cuidado y al azar; había unas treinta espigas de arroz y veinte de cebada, que consideré conveniente sembrar una vez pasadas las lluvias, cuando el sol se aleja del cenit declinando hacia el sur.

    Cavé lo mejor que pude con mi azadón de madera un buen trozo de tierra, y dividiéndolo en dos partes planté el grano; mientras lo estaba haciendo se me ocurrió que no convenía sembrar todo porque acaso no era el momento propicio, de manera que planté sólo dos terceras partes de cada clase, conservando cuidadosamente el puñado restante.

    Pronto tuve razones para alegrarme de mi prudencia porque no germinó ni un solo grano de los sembrados, pues sucediendo el período de sequía al de las lluvias aquella tierra no recibía suficiente humedad; sin embargo, cuando cambió la estación, vi nacer los cereales como si acabara de plantarlos.

    Al advertir que la semilla no germinaba por causa de la sequía, busqué un terreno más húmedo para intentar otra siembra, y hallándolo cerca de mi nueva enramada lo cavé y en el mes de febrero, poco antes del equinoccio de primavera, le confié el resto de mi semilla. Con las lluvias de marzo y abril germinó muy bien, dándome excelente cosecha, pero como por temor a que también esos granos se perdieran había puesto aparte otra pequeña reserva, el producto fue muy escaso, apenas medio celemín de cada clase. Me bastaba, sin embargo, para saber exactamente cuándo debía sembrar el grano, y la posibilidad de obtener fácilmente dos cosechas anuales.

    Mientras crecían mis sembrados hice un pequeño descubrimiento que después me fue harto útil. Tan pronto cesaron las lluvias y el tiempo se estabilizó —cosa que ocurría en noviembre— fui a visitar mi enramada, de la que faltaba desde hacía varios meses, encontrando todo en orden y tal como lo dejara al abandonarla. El círculo o doble empalizada no sólo estaba tan firme como antes sino que las estacas que cortara de algunos árboles de las inmediaciones habían echado raíces y ramas, iguales a las que produce el sauce al primer año de haber sido podado. Ignoraba el nombre de aquellos árboles que me habían provisto de tales estacas y me sorprendí grandemente, pero a la vez me alegró ver crecer los arbolillos, y traté de que todos se desarrollaran igualmente, podándolos lo mejor posible. Es difícil describir lo hermosos que se pusieron al cabo de tres años, tanto que a pesar de tener el círculo un diámetro de casi veinticuatro yardas, las copas se encontraban en su centro y daban una espesa sombra, bastante para abrigarme durante toda la estación de sequía.

    Esto me decidió a cortar más estacas de la misma clase y formar con ellas un semicírculo arbolado en torno a mi empalizada, me refiero a la de mi primera residencia, cosa que hice en una doble hilera situada a ocho yardas de la antigua empalizada. Pronto crecieron tanto y cubrieron tan bien mi morada que hasta me sirvieron más adelante de defensa, como se verá.






	\chapter{Viajes y trabajos}





    Había yo observado que las estaciones del año no se dividían como en Europa en invierno y verano, sino en estación seca y lluviosa. Luego de experimentar en carne propia los inconvenientes de las lluvias, tuve buen cuidado de proveerme por adelantado de lo más necesario a fin de no tener que salir para nada, y durante los meses de lluvia hacía todo lo posible por quedarme a cubierto.

    No estaba sin embargo ocioso mientras duraba mi encierro, efectuando toda clase de trabajos aplicables a esa circunstancia, tales como diversos objetos necesarios que sólo con gran paciencia y dedicación podían ser fabricados. Intenté muchas veces tejer un canasto, pero los mimbres que a tal efecto ensayaba eran tan quebradizos que de nada servían. Fue entonces que me resultó de gran utilidad el haber observado siendo joven a un cestero de mi pueblo natal, siguiendo con atención su modo de tejer el mimbre; como todo muchacho dispuesto a ayudar y lleno de curiosidad por la forma en que se fabricaban aquellos cestos, y a veces participando en la tarea, llegué a conocer bastante bien los procedimientos usuales, faltándome ahora sólo el material suficiente. Se me ocurrió que acaso los tallos de aquel árbol del que había sacado las estacas que prendían fueran tan resistentes como los del sauce o mimbre, y me propuse averiguarlo.

    Al día siguiente fui a mi casa de campo, como me agradaba llamarla, y cortando algunos de los tallos más tiernos descubrí que se adaptaban admirablemente a mi propósito; volví, pues, la vez siguiente armado de una hachuela para cortar gran cantidad, lo que era fácil por la abundancia de árboles. Los puse a secar dentro del vallado, y cuando estuvieron listos los traje a la cueva; allí, durante la estación de las lluvias me entretuve en fabricar toda clase de canastos tanto para acarrear tierra como para poner en ellos distintas cosas. Cierto que no estaban muy bien terminados, pero servían pasablemente para lo que yo los destinaba. Desde entonces me preocupé de que no faltaran, y a medida que los veía estropearse con el uso los iba reemplazando con otros mejores, en especial unos grandes cestos que hice para depositar el grano de la cosecha en vez de meterlos en sacos.

    Superada aquella dificultad y puesto mucho tiempo en lograrlo, empecé a buscar el modo de suplir dos grandes necesidades. No tenía vasijas para líquidos a excepción de dos barrilitos llenos de ron y algunas botellas, ya de tamaño común o bien las cuadradas que se emplean en guardar licores y bebidas. Carecía de ollas para guisar o hervir alimentos, salvo una enorme marmita que salvé del naufragio y que era demasiado grande para hacer en ella caldo o guisar un trozo de carne. Y la segunda cosa que deseaba intensamente era una pipa. Pero no hallaba la manera de fabricarme una hasta que al fin pude dar con el procedimiento.

    Me ocupé en plantar la segunda hilera de estacas y tejer cestas durante toda la estación seca, cuando una nueva tarea se presentó para demandarme mucho más tiempo del que imaginaba dedicarle. He dicho que sentía el deseo de explorar íntegramente la isla, y que llegando hasta el arroyuelo y siguiendo aguas arriba había desembocado en el valle donde ahora estaba mi enramada, y desde el cual podía verse un paso que conducía a la costa opuesta y al mar. Resolví franquear esa distancia que aún me faltaba conocer, y llevando una hachuela, la escopeta y mi perro, así como suficiente cantidad de pólvora y balas, provisto de dos galletas y un gran racimo de pasas para comer en camino, empecé la jornada. Luego de atravesar el valle donde estaba mi casa, llegué por el oeste a la vista del mar, y como era un día excepcionalmente diáfano pude ver tierra a lo lejos, aunque sin distinguir si se trataba de un continente o una isla. Era una tierra muy alta, extendiéndose del oeste al O-SO a una enorme distancia que, según mis cálculos, no podía ser menos de quince o veinte leguas.

    Vi abundancia de papagayos, y me hubiera gustado apresar uno vivo para domesticarlo y enseñarle a que me dirigiera la palabra. Con gran trabajo alcancé a darle con el bastón a uno muy joven, aturdiéndolo; pero una vez en casa pasaron varios años antes de que aprendiera a hablar. Por fin supo llamarme con mucha familiaridad por mi nombre, y el episodio a que esto dio lugar, aunque sea una insignificancia, divertirá cuando sea narrado en su debido momento.

    Lo pasé muy bien en aquel viaje. En las tierras bajas había animales parecidos a liebres y zorros, pero tan distintos de las otras clases que ya conocía en la isla que no me animé a probar su carne aunque había matado unos cuantos. ¿Para qué arriesgarme si tenía suficiente comida y de la mejor, tal como la carne de cabra, pichones de paloma y tortugas? Sumando mis pasas, el mismo mercado de Leadenhall no hubiera podido abastecer tan bien una mesa en proporción al número de comensales. Aunque mi situación era deplorable, tenía razones para estar agradecido y yo no padecía necesidades sino que hasta me sobraban las cosas. Nunca hice más de dos millas en línea recta mientras cumplía este viaje, pero eran tantas mis vueltas y revueltas en procura de nuevos descubrimientos que llegaba rendido hasta el sitio que elegía por campamento nocturno; allí trepaba a un árbol o me protegía rodeándome con un círculo de estacas que clavaba en el suelo —y a veces tendía de árbol a árbol— para estar seguro de que ningún animal salvaje se acercaría sin despertarme antes. Tan pronto arribé a la orilla del océano tuve la sorpresa de comprobar que había elegido el peor lado de la isla para vivir, ya que aquí la playa estaba cubierta por innumerables tortugas, mientras que en la costa opuesta sólo había visto tres en año y medio. También descubrí infinidad de aves de toda clase, muchas que ya había encontrado y otras desconocidas. La mayor parte tenía una carne exquisita, pero ignoraba su nombre, salvo el de los pájaros llamados pingüinos.

    Era fácil matar gran cantidad de estos animales, mas me interesaba economizar todo lo posible la pólvora y el plomo, por lo cual preferí dedicarlos a las cabras que me daban alimento más duradero; vi aquí todavía más cabras que del lado donde vivía, pero la dificultad para cazarlas era mayor por la regularidad del terreno que les permitía divisarme desde muy lejos.

    Confieso que este lado de la isla me pareció harto más bueno que el otro, pero no me sentí movido a cambiar de vivienda. Ya me había habituado a mi casa, me parecía algo natural y propio, tanto que ahora en todo momento sentía la impresión de hallarme de viaje, y proveniente de mi casa. Seguí la costa marina hacia el este, unas doce millas según presumo, y después de clavar un palo a modo de señal me decidí a regresar a casa, quedando resuelto que en la próxima exploración saldría de ella en dirección al otro lado de la isla costeándola hasta dar con el palo que a propósito dejaba.

    Para volver elegí otro camino, pensando que me sería fácil recordar la topografía de la isla y que era más agradable regresar viendo cosas nuevas. Pero pronto me encontré perdido, erré de un sitio a otro y por fin tuve que volver a la costa buscando la señal, y enderezar hacia el camino ya andado. Regresé haciendo etapas muy cortas porque el calor era excesivo y cuanto yo llevaba —escopeta, municiones, hacha— resultaba muy pesado.

    En esos días sorprendió mi perro a un cabrito y, saltándole encima, me dio tiempo a que llegara corriendo y lo salvara de sus colmillos. Tuve otra vez la idea de llevarlo a casa y me pregunté si no sería posible domesticar uno o dos cabritos a fin de irme procurando un hato que supliera mi falta de alimentos cuando no tuviese más pólvoras ni balas.

    Tejí un collar para el animalito y atándolo con una cuerda que siempre llevaba conmigo lo arrastré, aunque no sin dificultades, hasta mi enramada, donde lo dejé encerrado porque me sentía impaciente por arribar a mi casa, de la que faltaba desde hacía casi un mes.

    No puedo expresar con cuánta satisfacción penetré en la vieja tienda y me dejé caer en la hamaca. Aquella exploración, sin lugar fijo de residencia, me había resultado tan poco grata que mi propia casa —como me gustaba llamarla— era una morada perfecta comparada a lo anterior. Tan confortable me parecía tener mis cosas a mi alrededor que prometí no alejarme nunca más tan lejos mientras mi suerte me tuviera encadenado a aquella isla.

    Descansé una semana de las fatigas del viaje, entreteniéndome en la importante tarea de fabricar una jaula para mi papagayo, que se domesticaba rápidamente. Me acordé luego del pobre cabrito que dejara encerrado en la enramada, y me apresuré a ir en su busca o por lo menos a llevarle alimentos. Estaba donde lo había dejado, ya que le era imposible escaparse, pero medio muerto de hambre. Cortando follaje de árbol y ramas de arbustos tiernos se los di a comer y después que se hubo satisfecho lo até para llevarlo a casa, pero se había amansado tanto con el hambre que no era necesaria esta precaución, pues me seguía como un perro. Como continuara alimentándolo, el cabrito se volvió tan dócil y tan cariñoso que desde entonces permaneció conmigo y formó parte de mi familia sin abandonarme jamás.

    Venía la estación de las lluvias del equinoccio otoñal, y celebré el 30 de setiembre de la misma solemne manera. Se cumplía el segundo aniversario de mi arribo a aquellas tierras y en todo ese tiempo no había tenido la menor posibilidad de ser rescatado. Pasé el día en humilde y reconocido agradecimiento de los muchos y admirables beneficios que aliviaran mi desgracia, y sin los cuales hubiera sido infinitamente miserable.

    Fue entonces cuando empecé a sentir claramente cuánto más feliz era esta vida, con todos sus rigores, que la perversa, maldita y abominable existencia en que había dejado deslizarse mis años pasados. Tanto mis alegrías como tristezas eran muy distintas de las antiguas; mis deseos cambiaron, así como mis afectos, y la alegría que ahora era capaz de experimentar tenía razones totalmente opuestas a las que sentía a mi llegada a la isla o en los dos años que acababan de cumplirse.

    En tal disposición de espíritu principié mi tercer año de soledad, y aunque no fatigaré al lector con la detallada crónica de mis trabajos en este período debo decirle que muy pocas veces estuve ocioso ya que dividí regularmente mi tiempo de acuerdo con las tareas que debía efectuar cotidianamente. Estas eran, ante todo, mis deberes para con Dios y la lectura de la Biblia, a la que dedicaba un rato tres veces al día; luego salía de caza, lo que me llevaba unas tres horas por la mañana salvo que lloviera; tercero, me ocupaba en preparar y cocer la carne así obtenida. En esta forma se iba buena parte del día, sin contar que hacia las doce, cuando el sol estaba en el cenit, el calor era tan intenso que no se podía salir, de manera que recién reanudaba el trabajo a eso de las cuatro; a veces, como único cambio en este orden de vida, alteraba las horas de caza y de labor, haciendo ésta de mañana y cazando al atardecer.

    Al tiempo empleado en el trabajo es preciso agregar la extraordinaria dificultad de cada tarea y las muchas horas que, por falta de herramientas, ayuda y habilidad, me llevaba cada cosa que hacía. Por ejemplo, pasé cuarenta y dos días para hacer un tablón que me sirviera de estante en la cueva, mientras que dos serradores con herramientas apropiadas hubieran cortado seis tablones del mismo árbol en medio día.

    El trabajo era el siguiente: buscaba ante todo un árbol lo bastante grande para sacar un tablón ancho. Tres días se iban hachando el árbol, más dos para quitarle el follaje y reducirlo a una sola pieza de madera. A fuerza de hachazos y tajos lo iba rebajando de ambos lados hasta que el menor peso me permitía moverlo; lo daba vuelta y trataba de alisar un lado para que quedase como el tablón que quería, y volviéndolo a cambiar de posición alisaba el lado opuesto hasta conseguir la plancha deseada, de unas tres pulgadas de espesor y bien nivelada. Cualquiera puede imaginar lo que significa una labor semejante, pero la paciencia y el trabajo me permitían al fin lograr mi propósito. Si he traído este ejemplo ha sido para mostrar cómo una sola tarea podía insumir tal cantidad de tiempo, y que algo tan simple de hacer con herramientas adecuadas y alguna ayuda se convertía en una empresa harto compleja y reclamaba largo tiempo a un hombre solo y sin más instrumentos que sus manos. Pese a todo, con paciencia y perseverancia conseguí triunfar en muchas empresas y en todo cuanto me era necesario en tales circunstancias, como se verá en el siguiente relato.

    Venían noviembre y diciembre, y esperaba yo mi cosecha de cebada y arroz. El terreno donde los sembrara no era muy grande, pues ya he dicho que sólo tenía medio celemín de cada semilla, habiendo perdido el resto por sembrarlo en la estación seca. Mi cosecha prometía ser excelente cuando de improviso me encontré en peligro de volver a perderla por causa de algunos enemigos difíciles de combatir. En primer término las cabras y aquellos animales que yo llamaba liebres, los cuales habiendo advertido que los tallos eran tiernos venían noche y día a comerlos, estropeándolos de tal modo que temí que no echaran espiga.

    Para esto no hallé más remedio que fabricar un vallado en torno a mi terreno, lo que me dio un gran trabajo, ya que además de su extensión era necesario apresurarse. Por suerte mi tierra arada no era mucha, y en tres semanas de continua labor la cerqué completamente; matando algunos de esos animales durante el día y atando el perro a una estaca por la noche, para que ladrara cuando alguno se acercaba, conseguí que los enemigos abandonaran el sitio. El grano creció fuerte y sano y lo vi madurar admirablemente.

    Pero así como los animales mencionados estuvieron a punto de estropear mi cosecha cuando aún no levantaba del suelo, ahora fueron los pájaros quienes acudieron para hacerlo cuando surgieron las espigas. Yendo a visitar mi tierra la encontré rodeada de aves que no parecían esperar otra cosa sino que me fuera de allí. De inmediato les disparé un tiro, ya que jamás me separaba de mi escopeta, y apenas sonó el disparo vi levantarse una nube de pájaros que hasta entonces no había sospechado y que estaban posados entre el grano. Esto me preocupó seriamente, porque preví que en pocos días aquellas aves devorarían mis espigas y yo estaría condenado a perderlo todo y pasar hambre.

    Decidí hacer cuanto pudiera para impedir el daño, aunque tuviese que estar noche y día de centinela. Ante todo entré en el sembrado para verificar los daños causados, y en verdad que los pájaros habían destruido buena parte de las espigas; pero como había muchas aún demasiado verdes para su gusto, podía confiar en que el resto de la cosecha me compensaría.

    Me detuve a cargar la escopeta, y alejándome un poco observé que los ladrones estaban posados en los árboles circundantes como a la espera de que me marchara. Y así fue, pues apenas hube andado un poco más fingiendo no tener intención de volver, los vi precipitarse uno a uno sobre el sembrado. Tanto me indignó esto que no tuve paciencia para esperar a que los otros hicieran lo mismo, ya que me parecía que cada grano que devoraban equivalía un pan entero en el futuro. Volviendo, pues, hacia el sembrado, disparé sobre ellos consiguiendo matar tres. Esto era lo que deseaba, para emplear las víctimas tal como lo hacemos en Inglaterra con los bandoleros a quienes se deja colgados del cadalso a fin de que sirvan de escarmiento a los demás.

    Es casi imposible imaginar que mi procedimiento tuviera buen éxito, pero no sólo los pájaros cesaron por completo de volar sobre mi sembrado sino que huyeron de esa parte de la isla y jamás volví a verlos por ese lado mientras los espantapájaros colgaron allí.

    Esto me alegró sobremanera, y hacia la última mitad de diciembre, época de la segunda recolección anual, coseché el grano.

    Mi problema era fabricar una hoz o guadaña a tal efecto, y lo resolví bien que mal con ayuda de uno de los sables que trajera del barco. Como la primera cosecha era harto menguada no me costó mucho su recolección y la hice a mi modo, cortando solamente las espigas, que iba poniendo en un gran cesto y desgranaba luego entre mis manos. Cuando hube terminado encontré que aquellos medios celemines se habían convertido en casi dos fanegas de arroz y dos y media de cebada, según calculé aproximadamente, ya que me faltaba con qué medirlos.

    Esto me llenó de entusiasmo y llegué a imaginar que con el tiempo Dios me concedería tener pan. Algo sin embargo me preocupaba, porque no sabía cómo moler el grano para hacer harina, ni siquiera limpiarlo y cernirlo. Luego, aunque obtuviera la harina, ¿cómo arreglármelas para hacer pan si no tenía horno? Agregándose tales cosas al deseo que experimentaba de guardar suficiente cantidad de grano en depósito y asegurarme la subsistencia en lo venidero, resolví reservar aquella cosecha para sembrarla entera en la siguiente estación, y dedicar entretanto todo mi ingenio y mis horas de trabajo a la fabricación de lo indispensable para que mi sueño de tener pan se cumpliera al fin.

    Mientras llovía y yo estaba encerrado en la cueva, hallé oportunidad de dedicarme a esas tareas y a la vez me entretenía mucho hablando a mi papagayo para enseñarle a hacer lo mismo. Pronto aprendió el nombre que le pusiera y por fin dijo en alta voz: «Poli». Aquélla fue la primera palabra pronunciada en la isla por otra boca que la mía. Mi tarea no consistía sin embargo en eso, que era sólo diversión; tenía por delante un complicado trabajo que venía meditando desde tiempo atrás. Se trataba de ver si era posible fabricar vasijas de barro, tan necesarias para mí. Pensé que dada la elevada temperatura de aquel clima bastaría encontrar una arcilla conveniente y moldear con ella los recipientes, los que pronto se secarían al sol con dureza bastante para resistir el manejo y contener aquellas cosas que requerían un sitio seco y seguro; igualmente pensé que iban a ser útiles para guardar el grano y la harina que yo proyectaba acumular en mi casa. De modo que busqué la arcilla y me puse a fabricar recipientes a manera de grandes tinajas que almacenaran mis productos.

    El lector se apiadaría de mí, o acaso le produjeran risa los raros procedimientos que puse en práctica para dar forma a aquella pasta, las grotescas, deformes y feísimas vasijas que hice. ¡Cuántas se aplastaban o se abrían porque la arcilla no era bastante espesa para resistir su propio peso! ¡Cuántas estallaban a causa del violento calor del sol, demasiado pronto expuestas a sus rayos! ¡Cuántas se rompían de sólo tocarlas, antes o después de secas! En fin, luego de trabajar duramente para descubrir buena arcilla, extraerla en cantidad, mezclarla antes de llevarla a casa y moldear los recipientes, al cabo de dos meses de trabajo sólo conseguí fabricar dos grandes y desgarbadas cosas que no me atrevo a llamar tinajas.

    Habiéndolas secado muy bien al sol las levanté con todo cuidado y las puse dentro de unos cestos que a tal propósito había tejido, temeroso de que se rompieran en otra forma. Como entre los cestos y las vasijas quedaba un espacio hueco, lo rellené con la paja de la cebada y el arroz y notando que estaban perfectamente secas creí que serían depósito adecuado para el grano y, tal vez, la harina, una vez molido aquél.

    Aunque fracasé casi completamente en mi intención de fabricar grandes recipientes, con los más pequeños no me fue tan mal; hice potecitos, platos playos, cántaros y todo lo que se me ocurría; el calor del sol los coció bastante bien. Sin embargo nada de esto me consolaba de no poder tener una olla que soportara el fuego para cocer los alimentos, pues aquellos cacharros no me servían. Tiempo después, habiendo encendido un gran fuego para asar mi comida encontré al apagarlo un pedazo de cacharro que inadvertidamente había quedado entre las llamas, perfectamente cocido, duro como una piedra y del color de una teja. Me llenó de alegría descubrir tal cosa, y me dije que si los cacharros se cocían estando rotos igualmente lo harían enteros. El problema estaba en cómo disponer el fuego para tal fin. No tenía idea alguna de lo que era un horno de alfarero, ni del barniz de plomo que se pone a los cacharros, aunque disponía del metal suficiente. Puse tres grandes pucheros y dos o tres potes uno sobre otro, y encendí fuego a su alrededor cuidando de que gran número de brasas estuvieran colocadas directamente debajo de la pila. Renovaba continuamente el combustible, tratando de mantener avivado el fuego hasta que los cacharros empezaron a ponerse al rojo vivo sin que ninguno se quebrara. Cuando los vi así, sostuve el fuego por cinco o seis horas más, hasta que de pronto uno de los cacharros empezó a fundirse debido al excesivo calor que derretía la arena que yo mezclara con la arcilla. De haber continuado habría visto convertirse en vidrio aquella pasta, pero disminuí gradualmente el fuego hasta que los cacharros perdieron su rojo vivo, y quedándome despierto toda la noche para que el fuego no disminuyera demasiado bruscamente, me encontré de mañana dueño de tres excelentes —si no bellos— pucheros, así como de dos potes, todos ellos cocidos como pudiera desearse, y uno de ellos bonitamente barnizado por la arena derretida.

    Después de tal experimento está de más decir que tuve todos los cacharros necesarios. Su único defecto era la forma irregular y tosca, ya que carecía de medios para hacerlos mejor y trabajaba como los niños cuando hacen pasteles de barro o una cocinera que sin saber amasar quisiera hacer una tarta. Pocas veces hubo alegría tan desproporcionada a la insignificancia de su objeto como la que yo sentí al descubrir el modo de fabricar una olla que resistiera el calor del fuego. Tuve que contener mi impaciencia mientras se enfriaba, y apenas estuvo lista volví a ponerla al fuego con agua, en al que herví carne, viendo con júbilo que la olla resistía perfectamente la prueba. Con un trozo de carne de cabrito hice caldo, aunque me faltaba harina de avena y demás ingredientes necesarios que le dieran el debido sabor.

    Mi inmediata tarea fue fabricar una especie de mortero para moler el grano, ya que construir un molino era tarea inaccesible para un simple par de manos. Anduve varios días buscando una piedra lo bastante grande para excavarla en el centro y darle forma de mortero, mas no encontré ninguna salvo las rocas, cuya dureza impedía todo intento. Las piedras sueltas de la isla eran de una sustancia arenosa que se disgregaba fácilmente, y no hubieran resistido el golpe de otra piedra o llenado de arena el grano molido. Después de buscar inútilmente durante mucho tiempo resolví abandonar la tarea y elegir en cambio un trozo de madera suficientemente dura, cosa que me fue muy fácil. Llevando a casa el pedazo más grande que pude mover, lo redondeé exteriormente con ayuda del hacha, y luego por medio del fuego —aunque con infinito trabajo— pude vaciarlo interiormente a la manera como los indios del Brasil fabrican sus canoas. De un pedazo de palo de hierro hice la mano del mortero, y así equipado me dispuse a esperar la próxima cosecha cuyo grano había decidido moler —o más bien machacar— para hacer pan.

    Otra dificultad era la de procurarme un cedazo o tamiz para cerner la harina y separarla del salvado, sin lo cual me parecía imposible obtener el pan. Esto resultó lo más difícil de todo ya que carecía de lo indispensable para construirlo, es decir, una tela de trama bastante abierta para tamizar la harina. Tal cosa me detuvo durante muchos meses, y al final estaba enteramente desorientado; tenía algo de género de hilo, pero reducido a andrajos; también guardaba pelo de cabra, pero ¿cómo hilarlo y tejerlo si ignoraba el procedimiento y aun habiéndolo sabido carecía de todo instrumento adecuado? Por fin encontré una solución transitoria al recordar que entre las ropas de marinero que salvara del barco había algunas corbatas de zaraza o muselina, y con sus pedazos pude hacer tres pequeños tamices que me prestaron excelente servicio durante muchos años. Más adelante habré de narrar cómo los reemplacé.

    El problema de la cocción venía en seguida. ¿Podría hacer pan una vez que tuviera harina? Ante todo me faltaba levadura, y aunque esto último no me preocupa grandemente me afligía la carencia de horno adecuado. Por fin inventé un procedimiento que consistió ante todo en fabricar vasijas de arcilla, muy anchas pero no profundas, es decir, de unos dos pies de diámetro y apenas nueve pulgadas de hondo; las cocí en el fuego al igual que las anteriores, y las puse aparte. Luego, cuando deseaba hornear, encendía una gran hoguera sobre mi fogón, que estaba recubierto de tejas cuadradas —si puedo darles ese nombre— que yo mismo fabricara. Una vez que el fuego se había reducido a brasas, las disponía de modo que cubrieran enteramente el fogón y las dejaba hasta que lo hubiera recalentado; sacando luego las brasas ponía mis panes sobre las tejas y cubriéndolos con las vasijas mencionadas los rodeaba por fuera con brasas que mantuvieran el calor. Así, como en el mejor horno del mundo, mi pan de cebada cocía maravillosamente y pronto fui un excelente pastelero, ya que me animé a hornear en la forma descrita varias tortas de arroz y budines; no pude hacer pasteles porque no tenía nada con que rellenarlos, salvo carne de aves o de cabra.

    No habrá de causar asombro el que estas tareas se llevaran la mayor parte de mi tercer año en la isla, ya que además de ellas tenía en los intervalos que ocuparme en mi nueva cosecha y la labranza. Hice a su debido tiempo la recolección del grano y lo llevé a casa como pude, guardando las espigas en las grandes tinajas hasta tener tiempo para desgranarlas a mano, pues carecía de lugar para trillarlas así como de los necesarios instrumentos.

    Como mi provisión de cereales iba en aumento, empecé a ver la necesidad de construir mayores graneros. Quería un sitio donde tenerlos bien guardados, porque la cosecha había sido tan buena que me dio cerca de veinte fanegas de cebada y otro tanto de arroz, de modo que me resolví a emplearlos sin hacer economía. Mi provisión de pan se había agotado y debía renovarla; además quise calcular qué cantidad de semilla iba a bastarme para todo un año, a fin de sembrar anualmente una sola vez.

    Llegué a calcular que las cuarenta fanegas de arroz y cebada excedían en mucho a lo que podía gastar en un año, y por tanto me propuse sembrar cada vez una cantidad igual a la de mi última plantación, confiando en que, de esa manera, tendría bastante para hacer mi pan y otras comidas.






	\chapter{Expedición temeraria}





    Mientras atendía a todas esas cosas, podéis imaginar que muchas veces mis pensamientos tornaban hacia aquella tierra que divisara desde el lado opuesto de la isla, y que sentía nacer en mí la secreta esperanza de alcanzar alguna vez sus playas, pensando que acaso fuese tierra continental a donde pudiera trasladarme más adelante y hallar, por fin, el camino de la liberación.

    En ningún momento dejé de ver claramente los peligros que aquello suponía; lo peor era caer en manos de salvajes, sobre todo de aquellos que tenía razones para suponer peores que los leones y tigres del África. Sabía que de ser apresado lo más probable era que me asesinaran e incluso comieran; había oído decir que los caribes eran antropófagos e imaginaba, por la latitud, que no podía hallarme a mucha distancia de aquellas costas. Aun suponiendo que no fueran caníbales, lo mismo me matarían como a tantos europeos víctimas de su salvajismo; recordé que hasta grupos de veinte o treinta hombres habían sucumbido a los ataques de los salvajes y poca esperanza de defenderme podía abrigar con tal comparación. Como se ve, puse todo en la balanza; pero aquellas consideraciones que más tarde influyeron sobre mí no podían impedir ahora que mi imaginación volviera una y otra vez a su fantasía de llegar a aquellas tierras.

    Hubiese querido tener conmigo a Xury y la chalupa con la vela triangular que me había llevado más de mil millas por la costa de África. Pensé luego en utilizar el bote de nuestro barco, que como ya he narrado fuera arrojado sobre la costa. Lo encontré casi en el mismo sitio pero tumbado por la violencia constante del oleaje y el viento, cubierto casi por la arena gruesa de la playa y completamente en seco.

    De haber tenido ayuda para reflotar el bote y reparar sus averías, no dudo que me hubiera prestado buen servicio y acaso llevado hasta el Brasil. Pero me bastó estudiar su posición para darme cuenta de que tan difícil sería darlo vuelta como mover la isla entera. Hice sin embargo todo lo posible; corté troncos que sirvieran de rodillos y palancas, cobrando ánimos con la esperanza de enderezar el bote, lo que me permitiría repararlo y ponerlo en condiciones de navegar con seguridad.

    Tres o cuatro semanas empleé vanamente en este trabajo sin recompensa. Por fin, cuando estuve seguro de que mis pobres fuerzas no bastaban para dar vuelta el casco, imaginé excavar la arena que tenía debajo para que su peso hiciera lo que yo no había podido; poniendo adecuadamente los rodillos y palancas, imaginaba que podía guiar su caída. Cuando conseguí eso me resultó imposible levantar el bote del hueco en que estaba sumido, y mucho menos moverlo en dirección al mar, de modo que por fin abandoné la tarea. Pero mientras mis esperanzas se disipaban en ese sentido, las de llegar a las lejanas tierras se acrecentaban como si aquel fracaso las estimulara.

    Fue entonces cuando se me ocurrió la idea de intentar la construcción de una canoa o piragua, como la llaman los nativos, que son capaces de construirlas sin herramientas y hasta podría decirse que sin manos, empleando grandes troncos de árbol. Pensé que aquello no solamente era posible sino fácil, y cobré ánimos con la idea de que llevarlo a cabo me sería más simple que a los indios, pues disponía de mayores recursos que ellos. Olvidaba, sin embargo, cosas que me faltaban y que los naturales tienen en abundancia, por ejemplo, brazos para trasladar las canoas al mar cuando están concluidas; esta sola dificultad era para mí mucho mayor que la falta de herramientas para ellos. ¿De qué iba a servirme elegir un buen árbol en el bosque, derribarlo con gran trabajo, afanarme con mis herramientas en darle por fuera el perfil y las dimensiones de una canoa y ahuecarlo por dentro con mis instrumentos o el fuego, si al final tendría que dejarlo donde estaba por no tener fuerzas suficientes para llevarlo hasta el mar?

    Es como para creer que no había pesado la más insignificante de estas reflexiones mientras construía una canoa, ya que de inmediato hubiera advertido el problema de botarla al agua. Pero mis pensamientos estaban a tal punto absorbidos con la idea de intentar el viaje que en ningún momento consideré seriamente el problema. No fui capaz de advertir que me resultaría más fácil hacer navegar mi canoa cuarenta y cinco millas marinas que moverla las cuarenta y cinco toesas que la separaban del agua.

    Al dedicarme a su construcción hice la más grande locura que pueda cometer un hombre cuerdo. Me engañé a mí mismo con el proyecto, sin pararme a medir si era posible cumplirlo. No es que a veces no me preocupara la dificultad de botar la canoa al mar, pero de inmediato atajaba el hilo de mis pensamientos con una insensata respuesta que yo mismo me daba: «Hagámosla primero: de seguro encontraré después un medio u otro para ponerla a flote.»

    Era el sistema más absurdo que pueda concebirse, pero la intensidad de mi capricho prevaleció y me puse a la tarea. Haché un cedro tan hermoso que me pregunto si Salomón tuvo alguno tan grande para la construcción del Templo de Jerusalén. Medía cinco pies diez pulgadas de diámetro inferior, y cuatro pies once pulgadas en el superior, a una distancia de veintidós pies de altura, a partir de la cual adelgazaba un poco y se dividía luego en ramas. Con infinito trabajo derribé este coloso, tardando veinte días en hacharlo por la base y catorce en cortar las ramas y troncos menores hasta separar de él su vasta copa, en medio de fatigas indescriptibles. El mes siguiente lo pasé dando a la parte exterior del tronco la forma y las proporciones aproximadas de un bote que navegara pasablemente. Luego me llevó tres meses ahuecar el interior, hasta que tuvo la exacta apariencia de un bote. No empleé para ello el fuego sino escoplo y martillo, con agotador trabajo, hasta conseguir que el todo se pareciera bastante a una piragua y fuese capaz de llevar a bordo veintiséis hombres, cosa equivalente a mi persona y todo mi cargamento.

    ¡Qué alegría sentí cuando hube terminado el trabajo! Él bote era en verdad mucho mayor que todas las canoas o piraguas hechas de troncos que yo viera en mi vida. Muchos hachazos me había costado por cierto, y ahora sólo faltaba botarlo al agua; de haberlo conseguido hubiera yo emprendido a su bordo el más alocado e imposible viaje de que se tenga memoria alguna.

    Todas las tentativas de llevar el bote al agua fracasaron una tras otra, aunque cada una me costaba enorme trabajo. No había más de cien yardas hasta el agua, pero el primer inconveniente fue que el terreno se iba elevando hacia el lado del arroyo. Proyecté entonces excavar el suelo para formar un declive. Lo llevé a cabo aunque con prodigiosas dificultades, pero, ¿quién repara en eso cuando tiene a la vista su liberación? Y con todo, terminada la pendiente resultó imposible mover la canoa y fracasó como antes con el bote de la playa.

    Resuelto a todo, medí la distancia y resolví abrir una especie de dique o pequeño canal para traer el agua hasta la canoa, ya que no podía llevar la canoa al agua. Empecé el trabajo. Cuando había avanzado algo y me di clara cuenta de la profundidad y anchura que había que cavar, así como la arena que sería preciso extraer, hallé que teniendo por toda ayuda mis solas manos pasarían diez o doce años antes de que el canal estuviera terminado, porque la playa iba subiendo y en su parte superior el corte debía tener por lo menos veinte pies de profundidad... Con infinito disgusto, después de vacilar mucho, tuve que abandonar aquella postrera tentativa.

    Tal cosa me deprimió hondamente. Aunque demasiado tarde advertía la locura de empezar un trabajo sin calcular antes su costo y si mis fuerzas eran capaces de sobrellevarlo.

    A mitad de esta tarea se cumplió mi cuarto aniversario en la isla y celebré el día con la misma devoción y aún más recogimiento que los anteriores; el constante estudio y mi aplicación a interpretar la palabra de Dios, así como el socorro a Su gracia, me habían dado un conocimiento más hondo que el de otros tiempos. Ahora veía las cosas de muy distinta manera. El mundo se me aparecía como algo remoto, que en nada me concernía y del que nada debía esperar o desear. En una palabra, me hallaba del todo aislado de él y como si ello hubiera de durar siempre; me habitué a considerarlo en la forma en que acaso lo hacemos cuando ya no estamos en él; un lugar en el cual se ha vivido pero al que ya no se pertenece. Y bien podía decir como el patriarca Abraham al hombre rico: «Entre tú y yo hay un abismo.»

    Vivía ahora de un modo mucho más confortable que al comienzo, y con una tranquilidad harto mayor tanto para el alma como para el cuerpo. Me sentaba a comer sintiéndome lleno de gratitud, y admiraba la providencia de Dios que así tendía mi mesa en la soledad. Aprendí a estar reconocido a la parte buena de mi situación y a olvidar en lo posible la mala; prefería tener más en cuenta lo que me daba placer que las privaciones; y esto me hacía experimentar a veces tan secreto júbilo que no podría expresarlo, y si lo menciono aquí es solamente para llamar la atención de aquellos que no saben gozar alegremente lo que Dios les ha dado porque sólo ven y envidian lo que El no ha querido concederles. Toda nuestra aflicción por lo que no tenemos nace de la falta de gratitud hacia lo que nos ha sido dado.

    Pasaba muchas horas y hasta días en tratar de representarme con todo detalle lo que habría sido de mí si no hubiera podido sacar nada del barco. Me hubiera faltado alimento en primer lugar, salvo tal vez peces y tortugas; pero sí no hubiera podido encontrarlos pronto habría perecido de hambre. Aun así, ¿no hubiera sido mi existencia la de un salvaje? Suponiendo que la suerte me hubiera ayudado a capturar un pájaro o una cabra, no habría tenido cómo desollar o abrir esos animales, separar la carne de los huesos y las entrañas, sino que a modo de las bestias habría desgarrado y comido con mis uñas y dientes.

    Tales reflexiones me llenaban de gratitud por la bondad de la Providencia y a pesar de todos mis temores y dificultades estaba reconocido por lo que me había sido dado. Si mi vida era lamentable en un sentido, en otro mostraba claramente las señales de la bondad divina; llegué a desear, no una existencia confortable, sino que mi reconocimiento por la gracia de Dios, su protección incesante en tan duro trance, fueran mi diario consuelo. Y cuando hube alcanzado tal serenidad de espíritu ya no volví a sentirme nunca triste.

    Llevaba tanto tiempo en la isla, que diversas cosas que había traído del barco estaban agotadas o poco menos. La tinta, como he dicho ya, se consumió casi enteramente salvo una porción que fui salvando con pequeñas dosis de agua que le agregaba hasta que se puso tan débil que apenas se veían los trazos sobre el papel. Mientras duró, la empleé en llevar la crónica de las cosas más importantes que me ocurrían mensualmente. Fue entonces cuando, releyendo los acontecimientos del pasado, vi que existía una extraña coincidencia de fechas en los sucesos que me habían ocurrido y que, de ser supersticioso al extremo de considerar días afortunados y días nefastos, hubiera tenido harto motivo para sentirme extrañado.

    En primer lugar noté que el mismo día en que abandonando a mis padres y amigos salí de Hull, fue tiempo más tarde el de mi captura y esclavitud a manos del pirata de Sallee.

    El mismo día que tuve la fortuna de salvarme del naufragio en la rada de Yarmouth, huí del cautiverio de Sallee en la chalupa.

    El mismo día de mi nacimiento, es decir, el 30 de setiembre, salvé milagrosamente mi vida veintisiete años más tarde al quedar como único sobreviviente en la isla: de manera que mi vida de errores y mi vida solitaria principiaron el mismo día.

    Excepto la tinta, lo que me faltó después de un tiempo fue la galleta que había conseguido sacar del barco. Aunque la había economizado en extremo, limitándome a una por día, me encontré privado de ella casi un año antes de que lograra hacer pan con mi grano; y de esto último tenía sobrada razón para mostrarme agradecido: ya he contado la manera casi milagrosa en que pude obtener las primeras plantas.

    Luego fueron mis vestidos los que se deterioraron. Casi no tenía ropa blanca salvo algunas camisas a cuadros que encontré en los arcones de los marineros y que tuve cuidado de guardar, ya que solamente en algunas épocas toleraba la camisa sobre el cuerpo, siendo una gran suerte para mí tener casi tres docenas de ellas. Había también varios capotes de marino, pero me resultaban demasiado abrigados para usarlos. Aunque el clima era cálido y allí la ropa no era demasiado necesaria en modo alguno, en realidad no podía vivir completamente desnudo aun cuando me hubiera agradado estar así, cosa que no ocurría pese a encontrarme solo.

    La razón por la cual no podía vivir allí desnudo era que la violencia del sol se amortiguaba un poco llevando algo de ropa; de lo contrario me hubiera lacerado la piel, mientras que con una camisa liviana, el aire que circundaba y se movía entre ella y mi cuerpo era siempre más fresco que el ambiente. De la misma manera hubiese sido imposible andar bajo el sol sin un sombrero o gorro. Son tan fuertes los rayos del sol en esa latitud que producen de inmediato violentas jaquecas si no se los contrarresta con un sombrero, y yo había notado que bastaba ponérmelo para que el dolor cesara en seguida.

    Empecé a preocuparme por salvar los pocos harapos que me quedaban y que yo llamaba ropas. Había gastado todas las chaquetas que tenía y pensaba ahora en la posibilidad de hacer algunas con los pesados capotes marineros y cualquier otro material a mi alcance. Comencé, pues, a hacer el sastre, o mejor el remendón porque sólo conseguí resultados deplorables. Me las arreglé, con todo, para cortar dos o tres chaquetas que esperaba me serian muy útiles: en cuanto a calzoncillos y pantalones los resultados fueron aún peores hasta que pude hallar una mejor solución.

    He mencionado que guardaba las pieles de todos los animales que cazaba; luego de desollarlos tendía las pieles al sol entre estacas, por lo cual algunas se pusieron tan secas y duras que no me sirvieron, pero otras en cambio parecían útiles. Lo primero que obtuve de esas pieles fue un gorro grande, con el pelo hacia afuera para preservarme de la lluvia; y tan bien me resultó que me animé a cortarme un traje hecho enteramente de dichas pieles, es decir, una chaqueta y calzones abiertos en las rodillas, todo muy holgado, ya que se trataba de preservarme más del calor que del frío. No debo omitir que estaban atrozmente mal hechos, pues si era mal carpintero resulté aún peor sastre. Pero me sirvieron admirablemente, y cuando la lluvia me sorprendía en un viaje, el pelo hacia afuera de mi chaqueta y gorra me mantenía perfectamente seco.

    Pasé también bastante tiempo y no poco trabajo fabricándome una sombrilla. Me era muy necesaria y ansiaba tener una, pues recordaba la que había visto en el Brasil, donde son de gran utilidad contra los calores, siendo aquí en verdad imprescindible, ya que la isla se hallaba aún más cerca del ecuador. Aparte de eso, como tenía que alejarme con frecuencia de mi casa, la sombrilla podría ser igualmente útil contra los calores y las lluvias. Me dio bastante trabajo y pasaron muchos días antes de que pudiera fabricar algo parecido a lo que quería; incluso estropeé dos o tres veces mi obra antes de estar seguro de haber conseguido lo que deseaba, pero por fin obtuve un quitasol adecuado a mis necesidades. La mayor dificultad consistiría en cerrarlo, ya que abrirlo era fácil, pero si luego no podía plegarlo resultaba demasiado incómodo de llevar, salvo sobre mi cabeza, donde a veces no era necesario. Por fin di con la manera de manejarlo fácilmente y lo cubrí de pieles, con el pelo hacia arriba para que me protegiera de las lluvias como un tejadillo y a la vez atajara los rayos del sol; con él podía andar sin preocupaciones en el tiempo más caluroso como en el más frío, y cuando no me hacía falta lo plegaba para llevarlo bajo el brazo.

    De esa manera aumenté mis comodidades de vida, mientras mi espíritu se resignaba cada vez más a la voluntad de Dios y se entregaba enteramente a lo que su Providencia dispusiera.

    No puedo decir que a partir de entonces y durante cinco años me haya ocurrido nada de extraordinario. Vivía en la forma ya narrada, en el mismo sitio y tal como antes. Mis ocupaciones dominantes eran la siembra anual de la cebada y el arroz, la preparación de uvas en cantidad suficiente para tener, entre granos y pasas, lo bastante para alimentarme un año entero. Al margen de esta labor anual, y la cotidiana de salir de caza con la escopeta, tenía otra, que era la de hacerme una canoa; por fin la terminé, y cavando un canal de unos seis pies de ancho y cuatro de profundidad, la llevé hasta la ensenada, distante casi media milla. Se recordará que mi primera canoa, excesivamente pesada por haber sido construida sin la conveniente reflexión acerca de cómo la botaría al mar, quedó en el sitio; incapaz de llevarla hasta el agua o conducir el agua hasta ella, me vi obligado a abandonarla en el lugar donde la construyera como un claro ejemplo para ser más sensato otra vez. Por lo tanto, en esta segunda tentativa, aunque no pude conseguir un árbol tan bueno ni un lugar donde el agua estuviera a menos de la media milla ya mencionada, comprendí que la tarea era practicable y me entregué a ella de lleno. Estuve ocupado durante dos años, pero jamás renuncié un solo día a mi labor con la esperanza de tener al fin un bote que me permitiera hacerme a la mar.

    Sin embargo, cuando mi pequeña piragua estuvo concluida, supe que su tamaño no se prestaba para la intención que yo había tenido en cuenta al construir la primera, es decir, lanzarme hacia «tierra firme» cuarenta millas más allá. La fragilidad de esta canoa me obligaba a desistir de esa esperanza y no pensar más en ella; pero ya que tenía un barquichuelo, proyecté dar con él la vuelta a la isla. El viaje que hiciera a la costa del lado opuesto, atravesando por tierra y haciendo tantos interesantes descubrimientos, me impulsaba a explorar otras regiones de la costa. Era dueño de un bote; ¿por qué no navegar en torno a la isla?

    A tal fin equipé discretamente mi canoa, comenzando por fijar en ella un pequeño mástil cuya vela hice aprovechando pedazos que había traído del barco y que había conservado cuidadosamente en depósito.

    Emplazados el mástil y la vela, descubrí después de algunas pruebas que el bote navegaba muy bien. Le adapté entonces varias cajas donde poner provisiones, pólvora y balas, así como otros efectos que debían conservarse secos tanto de la lluvia como de la espuma del mar; hice luego en el interior de la canoa una larga y profunda ranura donde cabía mi escopeta, protegida por una lona contra toda humedad.

    Puse mi sombrilla en la popa, como un segundo mástil cuya vela se extendiera sobre mi cabeza protegiéndome del sol a manera de toldo. Así equipado emprendí una y otra vez pequeños recorridos por el mar, aunque sin alejarme mucho de la costa conocida y de la caleta. Por fin, cada vez más deseoso de hacer la circunnavegación de mi pequeño reino, me resolví a él. Puse a bordo suficientes vituallas para el viaje; dos docenas de panes de cebada (a los que debería llamar más bien galletas), un puchero de barro con arroz tostado, alimento del que hacía gran consumo, una botellita de ron, la mitad de una cabra, sin contar pólvora y balas en cantidad suficiente para cazar y dos de los abrigos de marinero que ya he mencionado como provenientes de los arcones que salvé del naufragio; contaba con ellos para que me sirvieran de colchón y de frazada.

    El dieciséis de noviembre, en el sexto año de mi reino o mi cautiverio, como se quiera llamarle, inicié el viaje, que me resultó mucho más extenso de lo que había imaginado; en verdad la isla no era muy grande, pero cuando llegué a su extremo oriental vi una gran cadena de rocas que penetraban más de dos leguas en el mar, algunas emergiendo del agua y otras submarinas, y más allá un banco de arena, casi en seco, otra media legua hacia adentro. Era preciso contornear ese cabo para seguir viaje.

    Cuando descubrí el accidente estuve a un paso de dar por finalizada la empresa y volverme, no sabiendo cuánto tendría que internarme en el océano y, sobre todo, preguntándome si me sería posible volver; en la duda me decidí a anclar allí mismo, cosa simple, pues me había fabricado una especie de ancla con un pedazo de cloque roto que traje del barco.

    Ya fondeado el bote, tomé la escopeta y trepé a una colina que me parecía adecuada para tener una visión panorámica del lugar; desde allí vi el largo total del cabo rocoso, y decidí aventurarme.

    Observando el mar desde la colina noté una fuerte y violenta corriente que corría hacia el este y pasaba casi rozando el cabo; la estudié detenidamente porque advertía el peligro de que si mi bote era envuelto por ella podría ser arrastrado mar afuera y sin posibilidad de regresar a la isla. Hasta pienso que si no hubiera tenido el cuidado de subir antes a la colina, algo de eso me habría ocurrido porque la misma corriente se desplazaba en el lado opuesto de la isla, sólo que se dirigía hacia afuera a mayor distancia. Entre ambas advertí la presencia de un fuerte remolino junto a la costa, y deduje que si me las arreglaba para zafarme del impulso de la primera corriente encontraría ventaja en la protección de aquel remolino.

    Con todo me quedé allí dos días, porque el viento soplaba fuerte del E-SE y siendo contrario a la citada corriente producía un violento oleaje contra el cabo rocoso; no era posible navegar cerca de las rocas por temor a ese oleaje, ni más lejos porque allí podía arrastrarme la corriente.

    En la mañana del tercer día observé que el viento había cedido durante la noche y que el mar estaba sereno, por lo cual emprendí viaje. Que lo que sigue sea un ejemplo para todos los pilotos ignorantes o temerarios: apenas había llegado al extremo del cabo, sin apartarme de él más que la longitud de mi canoa, cuando me hallé en un agua tan profunda y una corriente tan violenta como la compuerta de un molino. Fue tal la fuerza con que me arrastró que no pude mantenerme cerca de la orilla y pronto fui llevado más y más lejos del remolino que quedaba a mi mano izquierda. No soplaba viento alguno que pudiera ayudarme, y la fuerza de los remos no servía de nada. Entonces me creí perdido, ya que la corriente marina bordeaba la isla por ambos lados y venía a unirse algunas leguas más allá, hacia donde sería arrastrado con mayor fuerza. Carecía de medios para evitarlo; ante mí se extendía solamente la visión de la muerte, no por naufragio, ya que reinaba absoluta calma, sino por el hambre. Verdad que había encontrado una tortuga en la playa, tan pesada que apenas pude levantarla y meterla en el bote, y que también llevaba un cacharro grande lleno de agua dulce. ¿Pero de qué iba a servirme aquello cuando me encontraba perdido en el vasto océano donde sin duda no había tierra firme, ni siquiera una isla, por lo menos en mil leguas a la redonda?

    Puede entonces advertir cuan fácil le es a la divina Providencia tornar aún peor la más miserable condición humana. Miraba hacia mi desolada isla como si se tratara del más bello lugar de la tierra, y cuanta felicidad podía desear mi corazón era volver allá.

    Hice lo que estaba a mi alcance, hasta casi caer extenuado luchando por mantener el bote rumbo al norte, es decir, hacia el lado de la corriente donde había divisado el remolino. A eso de mediodía, cuando el sol alcanzaba el cénit, me pareció sentir en la cara un aleteo de brisa soplando del S-SE. Aquello me animó no poco, y más aún cuando media hora después la brisa se transformó en viento. Estaba terriblemente alejado de la isla, y la menor nube o niebla que hubiese surgido bastaba para mi perdición; no tenía brújula, y perder de vista la tierra un solo instante hubiera bastado para no dar ya nunca con el rumbo.

    El tiempo sin embargo seguía despejado, y aprovechando mi mástil y vela traté de mantenerme hacia el norte para salir de la corriente que desviaba al noroeste. Una hora más tarde había conseguido llegar a una milla de la costa y allí, aprovechando el agua tranquila, pronto toqué tierra.

    Apenas me sentí a salvo caí de rodillas para dar gracias a Dios por su bondad, y me propuse abandonar toda idea de salir de la isla en el bote. Luego me alimenté con lo que tenía a bordo, y dejando el bote anclado en una pequeña caleta que había descubierto bajo la protección de los árboles, dormí profundamente, agotado por las fatigas de aquel viaje.

    Me preocupaba ahora la idea de cómo volver a casa con el bote. Había corrido demasiado peligro y conocía bien las dificultades de regresar por el mismo lado que viniera; en cuanto a lo que pudiese haber en el lado opuesto —es decir, el oeste— lo ignoraba y no me sentía con deseos de arriesgarme más. Resolví, pues, a la mañana siguiente, ir por el oeste siguiendo la costa, y buscar algún arroyo o caleta donde dejar la canoa bien asegurada por si me hacía falta más adelante. A unas tres millas, costeando la playa, vine a dar con una pequeña bahía o ensenada, a una milla más allá, que se iba estrechando hasta la desembocadura de un arroyuelo, donde descubrí el fondeadero más indicado para mi bote, y en el que podría quedar como si fuera un puerto hecho a su medida. Allí dejé la canoa después de amarrarla sólidamente, y volví a la costa para explorar el sitio en que me hallaba.

    No me costó mucho advertir que apenas había pasado el lugar adonde llegara en mi anterior viaje a pie por la costa. Sacando del bote la escopeta y la sombrilla, porque hacía mucho calor, empecé a recorrer el camino conocido, que me resultó bastante agradable después de un viaje como el que acababa de hacer; por la tarde llegué al sitio donde se levantaba mi enramada, encontrando todo en orden, lo que me complació, ya que aquella era mi casa de campo y me gustaba que se conservara en buen estado. Escalé la empalizada y tendiéndome a descansar de tan fatigosa jornada pronto me venció el sueño. Pero juzguen los que leen esta historia cuál habrá sido mi sorpresa cuando entre sueños oí una voz que me llamaba repetidas veces por mi nombre:

    — ¡Robin, Robin, Robin Crusoe! ¡Pobre Robin Crusoe! ¿Dónde estás, Robin Crusoe? ¿Dónde estás? ¿Dónde has estado?

    Al principio yacía tan profundamente dormido, con la fatiga de remar toda la mañana y más tarde hacer a pie el resto del viaje, que no me desperté de una vez, sino que entre sueños me pareció que aquella voz llamándome era producto de mi fantasía. Pero como continuaba repitiendo « ¡Robin Crusoe, Robin Crusoe!», por fin me arrancó del sueño haciéndome pasar a un mortal terror. Mas apenas había abierto los ojos divisé a Poli, mi papagayo, posado en el borde de la empalizada, y comprendí que era él quien me había estado hablando en ese lenguaje lastimero que yo le había enseñado ex profeso. Tan bien lo había aprendido que era capaz de estarse largo rato posado en mi dedo y con su pico contra mi cara, diciéndome:

    — ¡Pobre Robin Crusoe! ¿Dónde estás? ¿Dónde has estado? ¿Cómo has venido aquí?

    Aun después que me hube convencido de que se trataba del papagayo y de nadie más, pasó un rato antes que recobrase la serenidad. Me asombraba que el animal hubiera llegado hasta allí, y que en vez de irse a otro lugar estuviera como esperándome en la enramada. Pero cuando estuve bien convencido de que era mi buen Poli, deseché mis preocupaciones, y tendiéndole la mano lo llamé por su nombre. El cariñoso pájaro voló a mi lado y luego de posarse como le gustaba hacerlo en mi pulgar, prosiguió hablándome como antes:

    — ¡Pobre Robin Crusoe! —me decía, y me preguntaba cómo había llegado allí y dónde había estado, igual que si hubiera sentido una inmensa alegría al verme otra vez. Entonces lo llevé conmigo a casa.

    Bastante tenía ahora de correrías por el mar, y suficiente tema para quedarme meditando muchos días acerca del peligro pasado. No me gustaba tener el bote del otro lado de la isla y tan alejado de mí, pero tampoco hallaba manera segura de traerlo. Por el lado oriental no quería ni pensar en la posibilidad de aventurarme a bordearlo por segunda vez; a la sola idea sentía paralizárseme el corazón y helarse mi sangre. En cuanto al lado opuesto ignoraba sus características, pero suponiendo que la corriente tuviera en aquella costa la misma fuerza que ya había yo experimentado en la parte opuesta, intentar el viaje equivalía a correr los mismos riesgos de ser arrastrado mar afuera. Terminé por resignarme a no tener el bote conmigo, aunque tantos meses de duro trabajo me había costado entre hacerlo y lanzarlo al mar.






	\chapter{La pisada en la arena}





    En tal disposición de ánimo viví cerca de un año, haciendo una vida retirada y tranquila como puede imaginarse; mis pensamientos estaban tan adaptados a mi presente condición, y había llegado a resignarme tanto a los designios de la Providencia, que hasta me consideré un hombre feliz en todos los aspectos salvo el de la compañía.

    Mi ingenio seguía aplicándose a las labores mecánicas que debía realizar para suplir tantas cosas necesarias, y pienso que llegué a ser un excelente carpintero, sobre todo si se tiene en cuenta la escasez de herramientas en que me encontraba. Aparte de esto, mi experiencia como alfarero se acrecentó también y pude por fin moldear la arcilla con una rueda, lo que permitía obtener más fácilmente cacharros de buena forma, mientras que los antiguos apenas podían ser mirados. Pero nada creo que me haya ocasionado mayor satisfacción, haciéndome sentir tan orgulloso de mi habilidad, como el día en que llegué a construirme una pipa. Cierto que era muy tosca y fea, cocida al fuego como los otros objetos de arcilla, pero resultó fuerte y el humo tiraba perfectamente. Mucho me alegré porque me gustaba en extremo fumar; a bordo había pipas, pero al principio no las busqué, ya que ignoraba la existencia de tabaco en la isla, y cuando volví más tarde al casco del barco no pude encontrarlas.

    También hice grandes progresos en cestería, tejiendo muchos canastos según mi gusto; aunque de no muy buena apariencia, resultaban extremadamente útiles para guardar efectos o acarrear diversas cosas a casa. Por ejemplo, si mataba lejos una cabra, podía colgarla allí mismo de un árbol y luego de haberla desollado y cortado en trozos los traía en uno de los canastos. Lo mismo si atrapaba una tortuga; allí mismo extraía los huevos y algunos pedazos de carne que me bastaban trayéndolos en mi cesto y dejando el resto en la playa. Los canastos más grandes y profundos eran mi depósito de granos, pues me apresuraba a desgranar las espigas apenas estaban secas y guardaba la semilla en la forma indicada.

    Pronto me di cuenta de que la pólvora disminuía considerablemente, y como de ninguna manera sería posible reemplazarla con los medios a mi alcance, me di a pensar qué haría para procurarme carne de cabra cuando ya no tuviese medios de cazarlas. Se recordará que durante mi tercer año en la isla apresé un cabrito que se crió muy manso, tanto que jamás pude decidirme a matarlo y lo dejé que viviera hasta que murió de viejo. Ahora, al cumplirse el undécimo año de mi residencia en la isla, y advirtiendo que las municiones disminuían, me puse a pensar algún medio de tender trampas a las cabras para atraparlas vivas.

    A tal fin tejí algunas redes en las que estoy seguro que cayeron varias cabras, pero como las cuerdas no eran solidad y yo no tenía alambre, las encontraba siempre rotas y el cebo comido. Por fin probé una trampa distinta; luego de hacer varios pozos profundos en aquellos sitios que frecuentaban las cabras, los disimulé con haces entretejidos que yo mismo había fabricado, sobre los cuales puse un gran peso; esparciendo espigas de cebada y arroz aunque sin alistar las trampas, observé que los animales acudían a esos lugares, como me lo probaron las huellas de sus patas. Una noche apresté tres trampas, y al acudir por la mañana vi que los haces estaban removidos y que faltaba el grano; pero las cabras habían evitado la celada. Esto me descorazonó bastante y me puse a rehacer las trampas; por fin, y para abreviar, yendo una mañana z revisarlas encontré en una un viejo macho cabrío y en la otra tres cabritos.

    Con respecto al macho cabrío no encontraba qué hacer con él, porque era tan fiero que no me atrevía a bajar al pozo para sacarlo vivo, lo que me hubiera agradado mucho. Por fin lo dejé escapar, y huyó a tal velocidad que parecía haberse vuelto loco de espanto. Yo había olvidado lo que una vez aprendiera, y es que el hambre amansa al mismo león; si hubiera dejado al macho tres o cuatro días en la trampa sin darle de comer, y le hubiera llevado después un poco de agua y algo de grano, se hubiera domesticado lo mismo que los cabritos, ya que son animales sagaces y tratables cuando se los cría convenientemente.

    Ignorando todo eso, lo dejé escapar; después, sacando uno a uno los cabritos del pozo, los até con sogas y no sin trabajo pude llevarlos a casa.

    Pasó bastante tiempo antes de que aceptaran lo que les daba de comer, pero terminé por tentarlos con granos maduros y pronto vi que se amansaban. Ya para ese entonces había decidido que si quería contar con carne de cabra el día en que se concluyera mi pólvora, criar un rebaño al lado de mi casa era la única solución posible.

    Meditando en esto, advertí la conveniencia de mantener separados los ya mansos de los salvajes en libertad, pues si los dejaba juntarse no tardarían aquéllos en hacerse tan salvajes como éstos. No veía otro remedio que elegir un buen pedazo de tierra y rodearlo de una empalizada, a fin de que la separación fuera absoluta y para siempre.

    Un par de manos era harto poco para semejante tarea, pero como advertía su urgente necesidad me apresuré a elegir terreno adecuado donde hubiese suficiente hierba para pastar, agua dulce y protección contra los calores solares.

    Para empezar resolví construir la empalizada en torno a un área de unas ciento cincuenta yardas de largo por cien de ancho; como no me faltaban tierras aptas en torno, podría más adelante ensanchar el vallado si mi rebaño aumentaba mucho. La tarea no me pareció excesiva, y la comencé con decisión. Durante tres meses estuve cercando el corral, y en este plazo tuve a las cabritas en la mejor parte, cuidando de alimentarlas lo más cerca posible de mí para que se amansaran bien. Con frecuencia les llevaba algunas espigas de cebada o un puñado de arroz y se los ofrecía en mi mano, por lo cual después que el vallado rodeó el terreno y pude soltarlas dentro, corrían detrás de mí balando por un poco de grano.

    Todo resultó como lo había deseado, y un año y medio más tarde era dueño de un rebaño de unas doce cabras, incluyendo los cabritos; dos años después ascendía a cuarenta y tres, fuera de las muchas que había matado para alimentarme. Aparte cerqué cinco corrales menores para que pastaran, con portillos que comunicaban a mi gusto unos con otros, y especie de pequeñas jaulas donde las hacía entrar para apresarlas fácilmente.

    No fue esto todo, porque además de la carne necesaria para comer disponía asimismo de leche, cosa que no se me había ocurrido pensar al principio, pero que me llenó de agradable sorpresa cuando comprendí lo simple que era. De inmediato monté una lechería, y diariamente ordeñaba un galón o dos de leche. La naturaleza, que da los medios de alimentarse a toda criatura, parece enseñarle al mismo tiempo cómo debe aprovechar ese alimento; yo que jamás había ordeñado una vaca y mucho menos una cabra, ni había visto preparar manteca o queso, llegué a hacer todo eso de la manera más natural y simple, aunque no sin muchos ensayos y fracasos. Desde entonces tuve tanta manteca y queso como podía desearlo.

    Hasta un estoico se hubiera reído al verme comer rodeado de mi pequeña familia. Yo era allí la majestad y el poder, príncipe y señor de la isla entera; la vida de mis súbditos estaba librada a mi arbitrio; podía ahorcar, descuartizar, conceder libertad y privar de ella. No había rebeldes entre mis súbditos.

    Solo como un rey, comía atendido por mis sirvientes.

    Poli, a manera de un favorito, era el único con derecho a dirigirme la palabra. Mi perro, ya muy viejo y chocho, se tendía a mi derecha mientras dos gatos, uno a cada lado de la mesa, esperaban que les cediera uno que otro bocado, como una prueba de especial favor.

    Rodeado de tal corte, y con tanta liberalidad, transcurría mi vida. Nada podía desear, como no fuera la compañía de mis semejantes; y por cierto que poco tiempo después la logré en exceso.

    Ya he dicho que muchas veces me volvía la idea de tener el bote conmigo, aunque no me impulsara el deseo de correr nuevos peligros a su bordo. En algunas ocasiones me ponía a pensar el modo de traerlo de este lado de la isla, pero otras veces me conformaba fácilmente con su ausencia. Poco a poco, sin embargo, predominaron aquellos deseos, y sobre todo el de llegar al punto de la isla donde, como ya he narrado, trepé a una colina para ver desde allí la línea de la costa y la dirección de las corrientes marinas. El deseo aumentó diariamente hasta que decidí irme a pie, recorriendo la costa. Así lo hice y si algún inglés hubiera podido en aquel entonces tropezar conmigo se hubiera asustado mucho o por el contrario reído a morirse. Yo mismo, cuando a veces me contemplaba, no podía menos de sonreír a la idea de atravesar Yorkshire con semejantes ropas y el correspondiente equipo. Que el lector juzgue por el siguiente esbozo:

    Llevaba un gran gorro sin forma alguna, hecho de piel de cabra, con una pieza de piel colgando detrás para que me protegiera de los rayos del sol y a la vez impidiera a la lluvia entrarme por el cuello, porque pocas cosas dañan tanto en aquellos climas como la lluvia entre los vestidos y la piel.

    Usaba una corta chaqueta también de piel de cabra, cuyos faldones me llegaban a la mitad de los muslos, y un par de calzones cortos del mismo material. Estos calzones habían sido cortados de la piel de un viejo macho cabrío y el pelo era tan largo que colgaba, a manera de pantalón, hasta la mitad de la pantorrilla. Me faltaban medias y zapatos, pero me había ingeniado para fabricarme unos borceguíes, si es que puedo darles algún nombre, altos de pierna y que se anudaban a los lados como las polainas; es de imaginar la forma que tendrían, al igual que el resto de mi atavío.

    Como cinturón usaba una larga tira de piel de cabra que se ajustaba con dos tiras más pequeñas en lugar de hebillas; a cambio de la espada o el puñal que se lleva en el cinturón, tenía un hacha y una pequeña sierra. Poseía además un segundo cinturón, especie de tahalí que me cruzaba el hombro, y en su extremo, bajo el brazo izquierdo, había colgado dos sacos de piel de cabra; en uno estaba la pólvora y en el otro las balas. Con una cesta en la espalda y la escopeta al hombro, sostenía sobre la cabeza una fea y pesada sombrilla también hecha de piel, que después de la escopeta era el objeto más necesario para mí. En cuanto a mi rostro, no lo tenía tan atezado como se hubiera podido suponer de un hombre que en modo alguno lo cuidaba y que vivía dentro de los diecinueve grados de latitud. Al principio toleré el crecimiento de mi barba hasta que tuvo casi un cuarto de yarda, pero como tenía tijeras y navajas, la recorté, salvo el bigote, que me complacía en retorcer a la manera de las patillas mahometanas (como había visto que lo usaban los turcos que conociera en Sallee, ya que los moros lo cortan de diferente modo).

    De mis bigotes o patillas no diré que fuesen lo bastante largos para colgar en ellos el sombrero, pero tenían suficiente longitud y espesor como para resultar espantosos en Inglaterra.

    Todo esto carece de importancia: tan poco me ocupaba de mi aspecto que no le concedía la más insignificante atención, de modo que nada más diré al respecto. Con tal traza empecé mi viaje, que duró cinco o seis días. En primer término seguí la costa hasta el lugar donde había anclado el bote para encaramarme a la colina. No teniendo ahora canoa de la cual preocuparme, busqué la vía más corta para subir a la misma altura que la vez anterior, y cuando estuve en la cumbre miré el cabo rocoso que penetraba en el océano y que en aquel terrible día intenté bordear a bordo de la canoa. ¡Cuál no sería mi asombro al descubrir que el mar estaba allí profundamente tranquilo, sin oleaje, ni movimiento, ni corriente!

    No podía comprender cómo había cambiado de esa manera; resolví por lo tanto quedarme algún tiempo observándolo, para estudiar lo que ocurría con las distintas mareas.

    El detallado estudio me demostró pronto que la única precaución a tomar consistía en tener presente el flujo y reflujo de la marea, y que no había dificultad alguna en llevar el bote al otro lado de la isla. Pero cuando pensé en llevar esto a la práctica, me invadió un terror tan grande con el recuerdo del peligro que había pasado la otra vez, que ni siquiera fui capaz de imaginar esa posibilidad. Preferí adoptar una segunda resolución, más segura aunque mucho más trabajosa: construir otra canoa o piragua, a fin de poseer una en cada lado de la isla.

    Es preciso tener en cuenta que para entonces disponía yo de dos fundos —si puedo llamarlos así— en la isla; el primero era la tienda con su fortificación de empalizada y la cueva a sus espaldas, que había profundizado y dividido en varios departamentos que comunicaban entre sí. Uno de estos depósitos, el mayor y menos húmedo, con una salida que daba más allá de la empalizada, estaba ocupado con las tinajas más grandes de que ya he hablado y además catorce o quince canastos capaces cada uno de contener cinco o seis fanegas. Allí acumulaba mis reservas de alimentos, especialmente el grano, del que una parte estaba aún en espiga y el resto había sido desgranado a mano.

    En cuanto a la empalizada, hecha con los troncos que ya he descrito, se había convertido en una muralla de árboles tan grandes y extendidos que no dejaban sospechar en modo alguno la existencia de una habitación humana.

    Cerca de mi morada, pero hacia el interior de la isla y sobre tierras más bajas, estaban mis dos plantaciones que cuidaba y araba para cosechar anualmente el grano en su punto; si hubiera deseado más semilla, disponía de abundante tierra a continuación de aquélla.

    En segundo término era dueño de mi residencia de campo, que constituía por cierto un fundo bastante pasable. Ante todo la enramada, que tenía buen cuidado de arreglar podando el cerco circundante para mantenerlo siempre a la misma altura, con la escalera del lado de adentro. En cuanto a los árboles, que al principio no eran más que estacas pero crecían ahora con gran lozanía, los podé como para que su copa se desarrollara espesa y amplia, dándome la sombra más agradable que pueda imaginarse. En el centro estaba la tienda, hecha con un gran trozo de vela sostenido por pértigas; era tan firme y sólida que nunca necesitaba reparación alguna. Bajo ella había armado una especie de cama con pieles de los animales que cazaba y otras cosas blandas, colocadas sobre un colchón salvado del naufragio, y si era necesario me cubría con un gran capote de marinero. Toda vez que me ausentaba de mi morada principal tenia, pues, este refugio en pleno campo.

    A esto hay que agregar los corrales del ganado, es decir, las cabras. Tanto como me costara rodear el terreno con un vallado, me costaba ahora cuidar que no se rompiera y escaparan por allí los animales; no abandoné mis esfuerzos hasta rodear el exterior del cerco con gran cantidad de pequeñas estacas, tan juntas que era difícil pasar por entre ellas una mano. Cuando aquellas estacas echaron raíces, cosa que sucedió en la estación lluviosa, el cerco se puso más fuerte que cualquier pared.

    Esto dará testimonio de que no pasaba mi tiempo sin hacer nada y que no escatimaba energías en lo que consideraba necesario para mi comodidad; estaba seguro de que criar aquellos animales al alcance de mi mano equivalía a tener un almacén de carne, leche, manteca y queso para toda mi vida, aunque durase cuarenta años más; pero, por otra parte, criar las cabras cerca de mí exigía perfeccionar de tal modo los cercos que de ninguna manera pudieran escaparse; y como he dicho obtuve tan buen éxito con el procedimiento de las pequeñas estacas que cuando crecieron vine a descubrir que eran demasiadas y tuve que entresacar algunas.

    Mis viñedos crecían también en la enramada, y contaba principalmente con ellos para tener pasas durante el invierno. Cuidé por tanto de conservarlos bien, ya que me parecían los más agradables entre mis alimentos y porque reunían virtudes medicinales que las tornaban muy refrescantes y saludables.

    Como la enramada venía a estar a mitad de camino entre mi casa y el sitio donde dejé fondeado el bote, habitualmente pernoctaba en ella en mi viaje hacia allá. Me gustaba mucho visitar la caleta y ver si el bote se mantenía en buenas condiciones. Algunas veces salí con él para distraerme, pero sin intentar nunca un verdadero viaje; navegaba a uno o dos tiros de piedra de la costa, de miedo a ser otra vez arrastrado por el viento o la violencia de las corrientes.

    Y llego ahora a una nueva etapa de mi vida. Cierta mañana, a eso del mediodía, yendo a visitar mi bote, me sentí grandemente sorprendido al descubrir en la costa la huella de un pie descalzo que se marcaba con toda claridad en la arena.

    Me quedé como fulminado por el rayo, o como en presencia de una aparición. Escuché recorriendo con la mirada en torno mío; nada oí, nada se dejaba ver. Trepé a tierras más altas para mirar desde allí; anduve por la playa, inspeccionando cada sitio, pero nada encontré como no fuera esa única huella. Empecinado, me puse a buscar otra vez preguntándome si no me estaría dejando llevar por una fantasía. Pero pronto hube de desechar esa idea: la huella era exactamente la de un pie humano, con su talón, dedos y forma característica. No podía imaginarme la procedencia de aquel pie, y después de debatir en mí mismo innumerables y confusos pensamientos, regresé a mi fortificación sin sentir, como suele decirse, el suelo que pisaba; tanto era el terror que me había invadido. A cada paso me daba vuelta a mirar en torno, confundía los arbustos y árboles y creía ver un hombre en cada tronco. Imposible es describir las distintas formas en que la imaginación sobreexcitada me hacía ver las cosas, las extrañas ideas que cruzaban por mi mente y hasta qué punto me dejé arrebatar por sus enfermizas fantasías mientras hice el camino de regreso.

    Al llegar a mi castillo —como creo que le llamé a partir de entonces— entré en él como un perseguido. Si lo hice mediante la escalera en la forma ya descrita, o entré por la abertura de la cueva, es cosa que no recuerdo. ¡Nunca una liebre corrió a su cueva ni un zorro a la suya con mayor espanto que el mío al entrar en mi morada!

    No dormí en toda la noche. Cuanto más tiempo transcurría desde el descubrimiento mayores eran mis aprensiones, al contrario de lo que parecería natural en tal circunstancia, sobre todo teniendo en cuenta la habitual reacción de los hombres ante el miedo. Tan aplastado quedé por el peso de mis fantasías en torno a lo que había descubierto, que a cada instante éstas iban en aumento aunque ya era tiempo de serenarme. De pronto se me ocurría que la huella era del diablo, y hasta encontraba apoyo razonable a tal suposición, porque ¿cómo podía haber llegado otra criatura con forma humana a la isla? ¿Dónde estaba el barco que la trajo? ¿Por qué no había otras señales de su paso? ¿De qué manera había podido un hombre llegar allí? Pero casi de inmediato me ponía a pensar lo contrario. ¿Por qué iba Satanás a adoptar forma humana en aquella playa donde nada había que pudiera interesarle? ¿Y por qué dejar su única huella en un sitio donde no había seguridad ninguna de que yo alcanzara a verla? Nada de eso tenía consistencia. Me dije que el diablo conocía infinidad de maneras más efectivas para aterrorizarme —si se lo hubiera propuesto— que dejar una señal en la playa; por otra parte, habitando yo en el extremo opuesto de la isla, ¿no hubiera sido más lógico que estampara allí la huella y no en un sitio donde había diez mil probabilidades contra una de que no la viera? ¿Y por qué en la arena, donde el primer embate del mar la borraría sin dejar rastro? Todo esto parecía incoherente ante el hecho mismo y la idea que habitualmente nos formamos de la sutileza del demonio.

    Estos argumentos me ayudaron a desterrar la idea de que fuera el diablo, y por ellos llegué a la conclusión de que se trataba de algo peor, es decir, algunos de los salvajes del continente próximo que, navegando en sus canoas, hubieran sido arrastrados por las corrientes o vientos contrarios hasta la costa, donde después de recorrerla habían vuelto a embarcarse quizá, tan poco deseosos de quedar en la desolada isla como yo de que lo hicieran.

    Mientras tales reflexiones ocupaban mi mente, me sentí profundamente reconocido por la fortuna que había tenido de no estar justamente en aquella parte de la isla, y que los salvajes no hubieran visto mi bote por el cual habrían descubierto la presencia de habitantes y acaso intentado su búsqueda. De ahí pasé a imaginarme con mortal terror que acaso habían dado con el bote, y que adivinando que la isla estaba poblada volverían en gran número para devorarme; aun suponiendo que lograra esconderme, lo mismo descubrirían mi vivienda, destruirían mis plantaciones, llevándose todas las cabras y dejándome morir al fin de inanición.

    Mis esperanzas en lo divino parecían disiparse bajo la fuerza del miedo. Toda mi confianza en Dios, fundada en las prodigiosas pruebas que había tenido de Su bondad, se desvanecieron. ¡Como si El, que hasta entonces me había alimentado milagrosamente, no tuviera poder suficiente para preservar los bienes que su bondad me había concedido!

    Me reproché no haber sembrado más semilla que la necesaria para sustentarme hasta la siguiente estación, como si nada pudiera suceder que me impidiera cosechar cada vez el grano. Tan fundado me pareció este reproche que decidí para el futuro acumular semilla suficiente para dos o tres años, a fin de no morir de hambre viniera lo que viniese.

    Reflexionando luego que Dios no sólo era justo sino todopoderoso, deduje que así como había dispuesto castigarme y afligirme, lo mismo podía salvarme si lo quería; y que si no era esa Su voluntad, mi deber estaba en someterme absoluta y enteramente a esa voluntad, al mismo tiempo que poner en ella toda mi esperanza, rogar al Señor y someterme a los dictados y decretos de Su providencia.

    Estos pensamientos me absorbieron durante horas y días, y hasta puedo decir semanas y meses. No debo omitir uno de ellos en particular; cierta mañana, mientras meditaba en mi lecho sobre los peligros que me acechaban a causa de los salvajes, me sentí hondamente afligido; pero en ese momento surgió en mi mente la palabra de la Escritura: Invócame en los días de aflicción, y yo te libraré, y tú me alabarás.

    En medio de estas meditaciones, terrores y conjeturas, se me ocurrió un día que acaso era víctima de las quimeras de mi imaginación. ¿No habría marcado yo mismo la huella en la arena el día en que desembarqué del bote en aquella playa? Esto me animó un poco y empecé a persuadirme de que sufría una ilusión y que aquel pie en la arena era el mío. ¿Acaso no podía haber andado por ese camino al salir de la piragua, cuando para volver a ella tomaba por ahí? Me dije que de ninguna manera podía recordar con exactitud el lugar por donde caminara aquella vez, y que si al final resultaba que la huella era mía, estaba haciendo lo que esos tontos que cuentan historias de fantasmas y apariciones y terminan por ser los primeros en asustarse de ellas.

    Esto me devolvió algo de coraje, y me puse a hacer pequeñas excursiones por los alrededores; llevaba tres días con sus noches sin salir del castillo y me faltaban alimentos porque no tenía a mano más que algunas galletas de cebada y un poco de agua. Recordé que debía ordeñar mis cabras, lo que antes era mi entretenimiento vespertino. Las pobres bestias habían padecido mucho por falta de cuidado, y a algunas se les había secado la leche.

    Alentándome con la creencia de que la huella provenía de mi propio pie, y que en realidad me había asustado de mi sombra, volví a salir y fui a mi casa de campo para ordeñar las cabras. ¡Pero con qué miedo avanzaba, cuan a menudo me daba vuelta para mirar a mis espaldas y cómo me aprontaba a arrojar la canasta a la primera alarma y correr para salvar la vida! Cualquiera que hubiese podido verme habría pensado que el remordimiento me perseguía, o que acababa de pasar por un miedo espantoso, lo que en verdad era así.

    Con todo, después que hube hecho el viaje dos o tres veces sin ver nada de inquietante, empecé a sentirme más animoso y a persuadirme de que todo aquello era simple producto de la imaginación. Nada de esto bastaba sin embargo para calmarme enteramente; era necesario volver a la playa, buscar la huella y comparándola con mi pie adquirir el convencimiento de que coincidía con mi pisada y era por lo tanto mía. Pero me bastó llegar allí para darme cuenta, en primer término, que al desembarcar del bote no había podido alejarme en dirección hacia donde estaba la huella; y luego, al compararla con mi pisada, descubrí que la misteriosa señal era mucho más grande. Ambas revelaciones volvieron a hundirme en el fantaseo más desatinado, y tal fue su violencia que me estremecía con escalofríos como si tuviese calentura. Volví a casa plenamente convencido de que un hombre, o muchos, habían desembarcado en aquella costa, salvo que en realidad la isla estuviera habitada, lo cual me exponía a ser atacado antes de volver de mi sorpresa. Tenía que defenderme a toda costa. Pero ¿cómo?

    Tal confusión de pensamientos me tuvo despierto la noche entera, aunque de mañana conseguí dormirme; la agitación de mi mente así como la angustia de mi espíritu me habían fatigado de tal manera que dormí profundamente y al despertar me sentí mucho mejor y más animado que antes. Principié a pensar serenamente, y luego de profundas reflexiones llegué a la conclusión de que esta isla tan hermosa, tan fértil y relativamente cercana al continente, no estaba abandonada como yo había supuesto hasta entonces. Cierto que no vivían en ella residentes fijos, pero era probable que con cierta frecuencia arribaran canoas a su costa, acaso deliberadamente o tal vez arrastradas por vientos contrarios. Llevaba yo allí quince años y jamás había visto la sombra de un ser humano, por lo que podía inferir que a poco de llegar a tierra volvían a embarcarse, sin haber mostrado hasta ahora la menor intención de permanecer en la isla. El peligro que podía amenazarme radicaba, pues, en algún desembarco occidental de esos errantes pueblos de mar, desembarco que ocurriría ciertamente contra su voluntad, lo que era fácil de advertir en su prisa por volverse al océano, permaneciendo sólo una noche en la costa hasta que la marea y la luz del día los ayudaban a reanudar el viaje. En vista de todo eso no me quedaba más que buscar algún sitio seguro donde refugiarme si los salvajes tocaban tierra.

    Me arrepentí inmediatamente de haber hecho la cueva tan profunda que la salida daba más allá de la empalizada que constituía mi fortificación. Medité el modo de evitar este peligro y resolví levantar una segunda línea de defensa, también en semicírculo, justamente donde doce años atrás plantara una doble hilera de árboles. Tan juntos los había puesto que me bastó intercalar unas pocas estacas entre ellos para dar al conjunto una extraordinaria solidez.

    Tenía, pues, una doble muralla de defensa; la exterior estaba reforzada con tablones, cables viejos y todo lo que sirviera para darle más resistencia y en ella había practicado siete orificios grandes como para pasar el brazo. Del lado interior acumulé tierra que extraía de la cueva, apisonándola fuertemente hasta lograr en la base un espesor de diez pies; luego puse en los mencionados orificios siete mosquetes que, como ya he narrado, había podido sacar del barco. Estaban sostenidos por horcones que hacían de cureñas como en los cañones, de manera que resultaba posible disparar toda la artillería en unos dos minutos. Me llevó muchos meses terminar aquella empalizada, pero no me sentí seguro hasta que la vi concluida.

    Hecho esto planté más allá de la muralla y en una gran extensión multitud de estacas de un árbol parecido al sauce mimbrero, que crece con gran prontitud y es muy sólido. Creo que puse cerca de veinte mil estacas, cuidando de dejar un claro entre ellas y la muralla para tener visibilidad del enemigo y evitar al mismo tiempo que se protegiera entre los árboles para asaltar la empalizada.

    A los dos años tenía formado un tupido seto, y cinco o seis años más tarde se había convertido en un verdadero bosque delante de mi morada, tan espeso y compacto que resultaba absolutamente intransitable. Ningún ser humano, sea quien fuere, podría haber imaginado que detrás de aquella selva había una vivienda. En cuanto a la manera de entrar y salir, cuidé de no dejar señal ni paso alguno. Colocaba una escalera hasta la parte baja de la roca donde había lugar para apoyar una segunda, de manera que cuando había retirado las dos escaleras nadie hubiese podido llegar hasta mí sin destrozarse; y aun llegando, se habría encontrado fuera de mi muralla exterior.

    Había, pues, adoptado todas las precauciones que la prudencia humana podía aconsejar para mi propia seguridad, y pronto se verá que no estaban del todo injustificadas, bien que en aquel entonces sólo preveía vagamente lo que mi miedo me insinuaba.






	\chapter{Los caníbales}





    Mientras me ocupaba en las cosas ya descritas, no descuidé sin embargo el resto de mis trabajos. Lo que más me preocupaba era la cuestión del pequeño rebaño de cabras. Para ese entonces, no solamente me daban carne en abundancia y proveían a mis necesidades sin tener que gastar pólvora y balas, sino que me eximían de la difícil caza de las cabras salvajes. Es por eso que sentía profunda inquietud ante la idea de perder aquellas ventajas y verme obligado a principiar nuevamente la domesticación.

    Pensándolo mucho, no vi más que dos caminos en ese sentido. Uno de ellos era encontrar sitio adecuado para excavar una caverna bajo tierra a fin de recoger allí las cabras por la noche; el otro consistía en cercar dos trozos de terreno, lejos uno del otro y lo más ocultos posible, donde pudiera yo criar una media docena de cabras. En esa forma, si alguna desgracia le ocurría al rebaño mayor, podría renovarlo pronto sin mucha fatiga. Cierto que esto último requería gran trabajo, pero me pareció la mejor de las dos soluciones.

    Pasé, pues, un tiempo en buscar sitio adecuado en los lugares más remotos de la isla y por fin di con uno que reunía todas las condiciones que podía desear. Era una porción de tierra húmeda, en medio del profundo y espeso bosque donde, como ya he contado, me perdí una vez cuando trataba de volver del lado oriental de la isla. Eran unos tres acres, tan rodeados de bosque que parecía provisto de cerco por la misma naturaleza. Gracias a eso la tarea de hacer el vallado no sería tan fatigosa como en los otros lugares elegidos anteriormente por mí.

    Inmediatamente me puse a trabajar, y en menos de un mes lo había cercado de tal manera que mis cabras, que eran mucho menos salvajes de lo que podría imaginarse, estuvieron en lugar seguro. Llevé ahí diez cabras jóvenes y dos machos cabríos, sin querer perder más tiempo, y cuando los tuve allí me dediqué a perfeccionar el vallado hasta que quedó tan seguro como el otro, el cual había sido levantado con menos prisa y empleando mucho más tiempo.

    ¡Pensar que todas estas fatigas tenían por única causa la huella de un pie humano sobre la arena! Hasta ese momento no había encontrado otra señal de presencia extraña en la isla. Dos años llevaba viviendo bajo esa preocupación constante que, como es de imaginar, tornó mi vida mucho menos apacible de lo que era antes; cualquiera que haya vivido obsesionado por el terror al hombre puede concebirlo. Aunque me duela decirlo, la confusión de mi espíritu era tanta que hasta se reflejaba sobre el lado religioso de mis pensamientos; el horror de caer en manos de salvajes y caníbales era tal que raramente me sentía en disposición de elevarme hacia Dios, por lo menos con aquella calma y resignación de espíritu necesarias a tal fin.

    Pero prosigamos. Luego de haber asegurado la existencia de mi pequeño rebaño, empecé a explorar nuevamente para descubrir otro sitio análogo. Me hallaba en una ocasión en el lado occidental de la isla cuando, al mirar hacia el océano, me pareció distinguir una embarcación a gran distancia. Tenía uno o dos anteojos que había encontrado en los arcones de marinero salvados del naufragio, pero no llevaba ninguno conmigo, y el barco, si lo era, estaba a una distancia que no me permitía distinguirlo bien, aunque miré con tal fijeza que mis ojos se fatigaron. Ignoro si se trataba o no de un barco, pero como al descender de la colina ya no lo divisé más, no quise seguir pensando en ello; con todo me propuse no volver a salir sin uno de los anteojos.

    Descendiendo la colina hacia la extremidad de la isla —donde jamás había estado anteriormente— me convencí de que la huella de un pie humano en la costa no era una cosa tan extraña como me había parecido al principio. Si la providencia no me hubiera hecho la merced de depositarme en la parte de la costa donde jamás desembarcaban los salvajes, hubiera advertido en seguida que nada era más frecuente para aquellas canoas arrastradas mar afuera que tocar tierra en este lado y procurarse refugio. Asimismo, como los tripulantes de las piraguas frecuentemente se abordaban y combatían entre sí, los vencedores traían a sus prisioneros a la costa donde, de acuerdo con sus horrorosas costumbres de antropófagos, los mataban y comían como se verá a continuación.

    Apenas había descendido de la colina a la playa, en la parte SO de la isla, cuando me sentí presa del espanto. ¿Cómo traducir la confusión y el terror de mi mente al ver la costa sembrada de cráneos, manos, pies y otros huesos humanos? A un lado se veían señales de que habían hecho fuego, y en su torno una especie de círculo como el corral de las luchas de gallos, en el cual sin duda se habían sentado aquellos salvajes para efectuar sus inhumanos festines con la carne de sus semejantes.

    Tan aterrado permanecía mirando aquellas cosas que ni siquiera pensé que pudiera encontrarme en peligro. Todas mis aprensiones desaparecieron a la vista de semejante colmo de monstruosa, infernal brutalidad, ante el horror de la degeneración humana llegada a tal punto. Muchas veces había oído hablar de los caníbales, pero nunca me había sido dado ver una cosa semejante. Por fin aparté el rostro de tan atroz espectáculo, y trepando rápidamente la colina me volví de inmediato a casa.

    Cuando me hube alejado algo de esa parte de la isla, me detuve como paralizado; entonces, recobrando mis sentidos, miré hacia el cielo con profundo reconocimiento y dejé que corrieran mis lágrimas mientras daba gracias a Dios por haberme hecho nacer en un lugar del mundo tan diferente del de aquellos espantosos seres.

    Lleno de gratitud volví a mi castillo y empecé a sentirme mucho más seguro bajo tales circunstancias que unos años antes. Comprendía que aquellos salvajes jamás arribaban a la isla en procura de algo; probablemente no esperaban encontrar gran cosa en ella, y si habían explorado como era muy natural la parte boscosa de la misma, debían sentirse desilusionados al no hallar nada que les conviniera. Me animaba la idea de que llevaba allí casi dieciocho años sin haber visto jamás la menor presencia humana, y que por lo tanto podría vivir otros dieciocho años tan oculto como hasta ahora, salvo que me dejara descubrir o sorprender por los salvajes; mi ocupación primordial debía consistir por lo tanto en mantenerme oculto, salvo que la suerte trajera a aquella tierra otras gentes mejores que los caníbales.

    Pese a estas ideas conservé una repugnancia tan grande hacia los salvajes, y me causaba tal horror su costumbre de devorarse unos a otros, que seguí pensativo y melancólico, casi sin salir de mis fundos por espacio de dos años. Me refiero a mi castillo, la casa de campo o enramada, y el corral oculto en los bosques. A este último sólo iba para cuidar de las cabras, ya que la aversión que sentía hacia aquellos diabólicos salvajes era tal que tenía miedo de encontrarme con ellos como con el demonio.

    El tiempo y la seguridad de que no sería descubierto lograron quitarme poco a poco aquella ansiedad, y llegué a vivir de la misma manera que antes, con la única diferencia que me mostraba más precavido y nunca salía sin tomar medidas para no ser sorprendido por algún salvaje. Cuidaba de modo especial no disparar inútilmente la escopeta, por temor a que oyeran el tiro si acertaban a hallarse en la isla. Me alegraba profundamente haber tenido la precaución de domesticar un rebaño dé cabras, cosa que tornaba innecesaria toda caza en los bosques. Si capturaba algunas a veces, era mediante las trampas que me habían permitido iniciar mi rebaño, y creo que por espacio de dos años a partir de lo narrado no disparé una sola vez la escopeta aunque la llevaba siempre conmigo. A mi armamento agregué las tres pistolas que salvara del barco, o por lo menos dos, que llevaba sujetas a mi cinturón de piel de cabra. También me colgué al cinto, con ayuda de un tahalí, uno de los grandes machetes que encontrara a bordo, de manera que mi aspecto debía ser formidable cuando emprendía cualquier viaje si a la descripción ya hecha de mi indumentaria y equipo se agregan ahora las dos pistolas y el gran sable colgando sin vaina a mi costado.

    A medida que pasaba el tiempo, y aparte de las precauciones mencionadas, volvía yo a mi antigua vida apacible y sosegada. Todo ello servía para mostrarme, más que nunca, qué lejos estaba mi condición de ser desesperada en comparación a la de otros, y cómo Dios, de haberlo querido, me hubiera reducido a una miseria infinitamente peor. Reflexioné entonces cuan pocas protestas habría entre los hombres de cualquier condición si tuvieran la prudencia de comparar sus vidas con otras más desdichadas, y sentirse agradecidos en vez de mirar a aquellos que se hallan por encima y creerse así con derecho a murmurar y quejarse.

    En mi actual situación no carecía de nada que me fuera indispensable, pero era tal el miedo y la inquietud que me produjeran los salvajes, como la necesidad de ocuparme de mi seguridad, que llegué a pensar que mi ingenio para procurarme nuevas cosas se había agotado. Abandoné un proyecto que anteriormente me preocupara mucho: intentar la transformación en malta de una parte de mi cebada, a fin de obtener cerveza.

    Mi ingenio, sin embargo, se explayaba en otro sentido; no dejé de pensar un momento en el modo de destruir a algunos de esos monstruos cuando estuvieran entregados a su sangriento festín, y si fuera posible salvar a la víctima que iban a inmolar. Llenaría un volumen mucho mayor que el presente el relatar todas las ideas que se me ocurrieron, y que rumiaba incesantemente, para destruir a aquellos salvajes o al menos aterrarlos de tal modo que jamás volvieran a aproximarse a la isla. Pero ninguna me parecía aceptable. Además, ¿qué podía hacer un hombre contra tantos, si acaso desembarcaban veinte o treinta armados de sus dardos, o arco y flechas, con los cuales podían tirar tan eficazmente como yo con mi escopeta?

    Una vez se me ocurrió hacer una excavación debajo del sitio donde encendían la hoguera y poner allí cinco o seis libras de pólvora, con lo cual apenas se dispusieran a comer volarían todos en pedazos. Pero, en primer lugar, me disgustaba la idea de gastar en ellos tanta pólvora, ya que apenas me quedaba un barril, y luego no estaba seguro de que la explosión se produciría en el momento debido para sorprenderlos; acaso alcanzara a aturdirlos y aterrarlos, pero sin fuerza suficiente como para que abandonaran el lugar.

    Deseché, pues, el proyecto y me propuse en cambio emboscarme en algún sitio conveniente con las tres escopetas y doble carga en cada una, esperando que estuvieran congregados para su sangriento festín; entonces podría disparar sobre ellos con la certeza de que cada tiro mataría o dejaría mal heridos a dos o tres, lanzándome finalmente al asalto con las pistolas y el machete. Tenía la seguridad de que en esa forma era posible dar cuenta hasta de veinte salvajes, y esta fantasía me complació tanto que la abrigué durante semanas; me absorbía a tal punto que hasta soñaba con ella, y frecuentemente me parecía que ya iba a lanzarme sobre la horda de caníbales.

    Tan lejos llevé el deseo de poner en práctica mi idea que anduve buscando los lugares indicados para emboscarme y espiar sus movimientos; volví muchas veces a aquel sitio, que ya me iba resultando familiar; y especialmente cuando mi cerebro estaba inflamado con ideas de venganza que me movían a exterminar sin piedad a veinte o treinta de ellos, el horror que me inspiraba ese sitio, con todos los restos de aquellos espantosos festines, apenas si atemperaba mi cólera.

    Por fin encontré un apostadero a un lado de la colina donde me pareció posible esperar a cubierto que alguna canoa se aproximase a la costa; desde allí, y antes de que los salvajes hubieran tenido tiempo de desembarcar, podía deslizarme sin ver visto entre los árboles hasta una concavidad que me cubría completamente; era un excelente puesto para tomar posición, observar en detalle sus sangrientos preparativos y hacer puntería sobre sus cabezas cuando estuvieran congregados, con tal precisión que no dudaba alcanzaría a dos o tres con cada disparo.

    Resolví, pues, fijar allí mi escondite, y de acuerdo con el plan preparé convenientemente dos mosquetes y mi escopeta de caza. Cargué los mosquetes con un puñado de pedazos de plomo y cuatro o cinco balas de pistola; a la escopeta le puse abundantes balines de grueso calibre, y finalmente cargué las pistolas con cuatro balas. Así artillado, y teniendo abundante munición para una segunda y tercera carga, completé los preparativos para el ataque.

    Luego de haber planeado los detalles y hasta haberlos puesto en práctica en mi imaginación, diariamente me iba a la cresta de la colina que quedaba a unas tres millas de mi castillo, para otear el océano y descubrir si había alguna embarcación que se aproximara a la isla. A los dos o tres meses de este cansador ejercicio empecé a fatigarme de él, ya que regresaba sin haber descubierto nada, no solamente en la isla sino en la vasta extensión del mar hacia el cual se dirigían mis ojos y mi catalejo.

    Mientras practiqué diariamente el viaje de reconocimiento a la colina, mantuve vivo el deseo de poner mi plan en práctica; me parecía absolutamente natural matar veinte o treinta salvajes desnudos por un crimen que no había entrado a discutir, dejándome llevar por el horror que me producían las monstruosas costumbres de aquellos pueblos.

    Pero cuando lo medité con más serenidad, necesariamente tenía que llegar a la conclusión de que estaba equivocado. Aquellos salvajes no eran más asesinos, en el sentido que me llevara antes a condenarlos mentalmente, que aquellos cristianos que frecuentemente sentencian a muerte prisioneros apresados en la batalla; o aquellos otros que, en tantas ocasiones, pasan a cuchillo batallones enteros sin querer darles cuartel a pesar de haber rendido las armas.

    En segundo término se me ocurrió que, aunque se devoraban unos a otros, nada de eso debía importarme. ¿Qué injurias me habían hecho aquellas gentes? Si atentaban contra mí, si yo veía que para preservarme de su ataque era conveniente caer sobre ellos, entonces se justificaría mi acción; pero hasta ahora me hallaba a salvo y ni siquiera mi existencia les era conocida, por lo cual no era justo precipitarme como lo proyectaba.

    Estas consideraciones me hicieron vacilar al principio y después me detuvieron completamente en mis planes; poco a poco los abandoné convenciéndome a la larga que había estado equivocado al resolverme a exterminar a los salvajes. No me correspondía mezclarme en sus asuntos si no me atacaban primero, y mi deber era solamente tratar de impedir esto; si de todos modos el ataque se producía, entonces quedaba en libertad de acción para repelerlo.

    Por otra parte llegué a darme cuenta de que mi proyecto no era precisamente un modo de asegurarme la tranquilidad, sino, por el contrario, acarrearme la peor de las catástrofes a menos que tuviese la seguridad de matar, no solamente a los que estuviesen en tierra en ese instante, sino a los que pudieran venir más tarde; porque estaba claro que si uno solo conseguía escapar se apresuraría a ir con la noticia a su pueblo, y pronto invadirían por millares la isla a fin de tomarse venganza por la muerte de sus semejantes. Comprendí que era atraerme la destrucción, mientras que hasta el presente nada tenía que temer de aquellos caníbales.

    En fin, por un doble motivo, moral y práctico, vi la conveniencia de mantenerme al margen de sus vidas. Mi tarea consistía en ocultarme a su vista por todos los medios, no dejando la menor señal que les permitiese sospechar en la isla la existencia de un ser humano.

    Unida aquí la religión a la prudencia, pronto adquirí la convicción de que había estado en un perfecto error cuando tramaba mis sangrientas venganzas contra aquellos seres inocentes (inocentes en lo que a mí respecta). Con sus culpas y crímenes personales nada tenía yo que ver; eran cuestiones concernientes a sus hábitos nacionales, y yo debía librarlos a la justicia de Dios, que es el Gobernador de las naciones y sabe cómo, con castigos adecuados, penar a quienes ofenden Su ley y juzgar públicamente y de acuerdo con Sus designios a quienes también públicamente han cometido las ofensas.

    Aclarados mis pensamientos al respecto, viví durante otro año con tan pocos deseos de estorbar a aquellos miserables que en todo ese tiempo no fui ni una sola vez a la cresta de la colina para observar si habían desembarcado o si estaban a la vista; temía no poder resistir la tentación de renovar mi cólera o sentirme arrastrado por las circunstancias a caer sobre ellos. Me ocupé en cambio de llevar a otra parte mi canoa, y sacándola de su caleta la conduje hasta el extremo oriental de la isla, donde la dejé a cubierto en una pequeña ensenada al abrigo de las rocas, seguro de que los salvajes, por temor a las corrientes, jamás sé atreverían a acercarse a un sitio semejante.

    Con el bote me llevé todo lo que había dejado cerca de él y que le pertenecía, tal como el mástil y la vela especialmente construida para impulsarlo, y una especie de ancla que no sé si merecía llamarse así o solamente rezón. Todo eso fue ocultado de manera que no quedase ni sombra que guiara a descubrirlo, así como la menor apariencia de bote o de habitación humana en la isla entera.

    Aparte de eso continué haciendo una vida todavía más retirada que antes; salía solamente para mis tareas cotidianas, es decir, ordeñar las cabras y cuidar del pequeño rebaño que tenía en los bosques y que, hallándose en el otro extremo de la isla, se encontraba perfectamente a salvo. Estaba seguro de que los salvajes, pese a acercarse a veces a la isla, no lo hacían con la esperanza de hallar nada en ella y por tanto cuidaban de no alejarse de la costa; tampoco me cabía duda de que habían vuelto varias veces a tierra después que mi descubrimiento me tornara tan cauteloso. A veces pensaba con espanto en lo que hubiera sido de mí al darme inesperadamente de boca con ellos, en la época en que sin más defensa que la escopeta —y ésta apenas con algunos balines— me paseaba sin cuidado por mis dominios. ¿Qué hubiera podido hacer si en vez de descubrir la huella de un pie humano me hubiese encontrado de pronto frente a quince o veinte salvajes que, a la velocidad que son capaces de correr, me hubieran apresado inmediatamente?

    Confío en que el lector de esta narración no hallará extraño que le confiese hasta qué punto aquellas ansiedades, ese constante peligro en que vivía ahora y las muchas preocupaciones que se cernían sobre mí, agotaron mi capacidad inventiva para las tantas cosas que antaño proyectara en busca de mayor comodidad. Necesitaba ahora mis manos más para procurarme seguridad que alimentos; no me atrevía a clavar un clavo o a cortar un pedazo de madera por miedo a que el ruido fuera escuchado. Mucho menos me atrevía a disparar la escopeta y, por sobre todo ello, buscaba no encender fuego por temor a que el humo, visible de día a gran distancia, me traicionara. Trasladé, pues, aquellas tareas que requerían el empleo del fuego, tal como la cocción de cacharros y tinajas, al abrigo de los bosques, donde después de estar cierto tiempo hallé con indescriptible alegría una enorme caverna natural en la entraña de la tierra, que parecía extenderse profundamente y donde me atrevería a decir que ningún salvaje se hubiera aventurado nunca a penetrar; incluso era capaz de aterrar a cualquiera, salvo a mí, que tanto la necesitaba como escondite.

    La boca de la caverna daba al pie de un gran peñasco donde se hubiera dicho que por casualidad (si no tuviera yo bastante motivo para considerar tales cosas como obra de la Providencia) me encontraba un día cortando algunas ramas gruesas para hacer carbón de leña. Quiero, antes de proseguir, explicar por qué hacía carbón y la razón es simple: evitar a toda costa que el humo me denunciara. Como no me era posible vivir sin hornear el pan, cocer mis alimentos y demás, me ingenié entonces en quemar leña bajo tierra como lo había visto hacer en Inglaterra, hasta que se carbonizara; luego, apagando el fuego, retiraba el carbón y lo llevaba a casa, donde podía utilizarlo sin peligro de humo.

    Pero dejemos esto. Cortando leña un día, observé que detrás de una espesa ramazón de arbustos bajos había como un hundimiento en el peñasco. La curiosidad me movió a acercarme, y cuando tras no poca dificultad llegué delante de aquella boca vi que era muy honda y lo bastante alta para estar de pie en el interior un hombre de mi estatura o aún más alto. Debo confesar que salí de allí con más apuro del que había entrado al divisar en la absoluta oscuridad del interior unos ojos brillantes clavados en mí, ojos que no sabía si eran del diablo o de un ser humano y que brillaban como dos estrellas, al reflejar la luz de la abertura.

    Reuniendo todo mi valor y tratando de darme ánimo con la idea de que el poder y la presencia de Dios están en todas partes y me protegerían, avancé unos pasos alumbrándome con una tea que sostenía por encima de mi cabeza; en el suelo yacía un enorme y espantoso macho cabrío, respirando anhelante y haciendo ya, como suele decirse, su testamento, pues estaba en las últimas a fuerza de viejo.

    Lo hostigué para ver si conseguía echarlo de la cueva, pero aunque hizo esfuerzos por levantarse no lo consiguió; no quise entonces molestarlo pensando que si tanto me había asustado aterraría aún más a cualquier salvaje que osara acercarse a la boca de la cueva mientras el animal se conservara con vida.

    Ya curado de mi temor empecé a reconocer la caverna, que era muy pequeña; tendría unos doce pies de diámetro, pero no es posible hablar de su forma, ya que no era ni cuadrada ni circular, siendo en un todo la obra de la Naturaleza. Reparé en que hacia el lado más profundo aparecía una segunda abertura, pero para pasar por allí hubiese sido necesario arrastrarme sobre pies y manos y yo ignoraba hacia dónde me llevaría. Renunciando por el momento a reconocer el segundo compartimiento, me propuse retornar al día siguiente con algunas velas y un yesquero que había sacado de la llave de un mosquete, pensando emplear el mixto de la cazoleta para encenderlo.

    Volví, pues, al otro día provisto de seis grandes velas hechas con cebo de cabra y que alumbraban muy bien; penetrando por la segunda abertura, tuve que arrastrarme por espacio de unas diez yardas, cosa que dicho sea de paso era harto aventurada, ya que no sabía hacía dónde me llevaba el pasadizo ni lo que encontraría al final. Por fin noté que el techo se elevaba hasta cerca de veinte pies, y me vi frente al espectáculo más hermoso que jamás contemplara en la isla. Iluminadas por la luz de dos velas, las paredes de la caverna, así como el techo, devolvían la luz en mil reflejos maravillosos. ¿Qué había en la roca? ¿Diamantes, piedras preciosas, acaso oro como me parecía sospechar? No podía decirlo a ciencia cierta.

    El lugar en que me encontraba era una admirable cavidad o gruta, aunque absolutamente oscura. El suelo, seco y llano, aparecía cubierto de una ligera capa de arena suelta, sin que en parte alguna se vieran animales venenosos; mirando hacia las paredes tampoco noté en ellas la menor huella de humedad. La única dificultad era la entrada, pero meditando que aquella caverna podía ser el sitio indicado para estar a salvo de los salvajes, me pareció que resultaba una ventaja. Profundamente regocijado con mi descubrimiento me resolví sin perder tiempo a trasladar a la gruta las cosas cuya seguridad me interesaba de modo especial; en primer término mis reservas de pólvora y todas las armas que no empleaba, es decir, dos escopetas y tres de los ocho mosquetes. En el castillo dejé cinco montados en las ya descritas cureñas, listos para tirar desde la empalizada; también podían servirme en cualquier expedición que emprendiera.

    En oportunidad de llevar mis municiones a la caverna, se me ocurrió abrir el barril que había salvado del mar y cuya pólvora estaba mojada. Al hacerlo comprobé que el agua había penetrado tres o cuatro pulgadas en la masa de pólvora y que la porción mojada, endureciéndose como una costra, había preservado del agua el resto como si fuera el corazón de un fruto. Tenía, pues, a mi disposición cerca de sesenta libras de excelente pólvora que extraje del centro del casco. Muy agradable sorpresa fue para mí en las circunstancias en que me encontraba, y llevándome todo a la gruta dejé en el castillo apenas dos o tres libras de pólvora para evitar cualquier sorpresa. Igualmente puse a salvo el plomo que me quedaba para hacer balas.

    Me complacía ahora imaginarme como uno de aquellos gigantes legendarios que moraban en cavernas y grutas a las cuales nadie podía llegar; estaba persuadido de que aunque quinientos salvajes anduvieran tras de mí, jamás descubrirían mi paradero y en el peor de los casos no se atreverían a atacarme en mi refugio.

    El viejo macho cabrío agonizante murió a la mañana siguiente de mi descubrimiento. Me pareció más simple excavar una sepultura en la misma cueva y cubrirlo bien de tierra, que arrojarlo al exterior.

    Se cumplían ya los veintitrés años de mi residencia en la isla; tan habituado me sentía a ella y a mi manera de vivir, que de haber tenido la certidumbre de que los salvajes no vendrían a estorbarme hubiera aceptado pasar en ella el resto de mi existencia, aunque al fin tuviese que tenderme en el suelo y esperar la muerte como el viejo macho cabrío de la caverna. Hasta había llegado a imaginar algunas diversiones y entretenimientos que me ayudaban a pasar el tiempo de modo mucho más agradable que en otras épocas. En primer término ya he contado que enseñé a hablar a Poli, y llegó a hacerlo tan bien, me hablaba tan familiarmente y con tanta claridad, que resultaba encantador; estuvo a mi lado nada menos que veintiséis años, e ignoro si vivió todavía más. En el Brasil afirman que estos animales alcanzan una existencia de un siglo, y tal vez mi Poli sigue aún viviendo en la isla, llamando al pobre Robin Crusoe. No deseo a ningún inglés la mala suerte de andar por ahí y escucharlo hablar, porque con seguridad creerá hallarse en presencia del mismo demonio.

    Mi perro fue un excelente y cariñoso compañero por espacio de dieciséis años, hasta que murió de viejo. En cuanto a los gatos, ya he dicho que se habían multiplicado tanto que tuve que matar a muchos para impedir que devoraran cuanto tenía; después, cuando murieron los dos más viejos que traje del barco, hostigué tanto a los otros sin darles el menor alimento que terminaron por huir al bosque y hacerse salvajes, excepto dos o tres favoritos que conservé a mi lado y cuyas crías me apresuraba a ahogar apenas nacidas. Fuera de estos animales tenía siempre conmigo dos o tres cabritos mansos a los que había enseñado que comieran de mi mano. Tenía también otros dos papagayos a quienes enseñé a decir mi nombre, pero ninguno podía compararse a Poli; cierto que no me tomé con ellos el trabajo que había dedicado a mi primer papagayo. En mi casa había varios pájaros marinos domesticados, cuyos nombres ignoro y que había capturado en la costa, cortándoles las alas. Las pequeñas estacas que plantara delante del castillo se habían convertido en un espeso seto, y allí vivían mis pájaros anidando entre los árboles más bajos, lo cual me agradaba mucho. Como puede apreciarse, con todo aquello había llegado a considerar mi vida como muy pasable, si sólo hubiera logrado desechar el temor a los salvajes.

    Pero mi suerte disponía otra cosa, y acaso no sea inútil para los que lean esta historia la observación que sigue. ¡Cuántas veces, en el curso de nuestra vida, el mal que con más empeño tratamos de evitar y que nos parece, cuando se precipita sobre nosotros, la más horrible cosa, resulta al fin la verdadera áncora de nuestra salvación, la única puerta por la cual podemos salir de la aflicción que nos embargaba! Muchos ejemplos de esto podría dar en el transcurso de mi extraña vida, pero donde más se manifestó fue en las circunstancias que rodearon mis últimos años de residencia en la isla.






	\chapter{Un naufragio y un sueño}





    Transcurría el mes de diciembre de mi vigésimo tercer año de soledad, y era la época del solsticio austral (porque no puedo darle el nombre de invierno) en la que me ocupaba yo de la recolección del grano, viéndome obligado a permanecer gran parte de mi tiempo en las plantaciones. Una mañana, cuando aún no era día claro y empezaba mi tarea, me sorprendió ver la luz de un fuego en la costa, a unas dos millas hacia el extremo donde primeramente advirtiera la huella de los salvajes, y al mirar con atención comprobé que no se trataba del lado opuesto de la isla, sino de la parte donde yo residía.

    Fue tal el azoramiento que se apoderó de mí que no me atrevía a salir de la enramada por miedo a que me sorprendieran, pero tampoco podía quedarme allí por temor a que los salvajes, errando por los alrededores, encontraran mis sembrados, las parvas de grano o cualquiera de mis otros trabajos, lo que les demostraría de inmediato la existencia de habitantes en el lugar. No dudaba que inmediatamente se pondrían a buscarme sin descanso, de manera que armándome de valor volví al castillo, levanté la escalera una vez que hube pasado, y traté de que todo tuviera el aspecto más salvaje y natural posible.

    Inmediatamente me apresté a la defensa. Cargando lo que yo llamaba mis cañones, es decir, los mosquetes montados sobre horcones, y alistando las pistolas, me resolví a defenderme hasta el último aliento, sin olvidar encomendarme con fervor a la protección divina y rogar ardientemente a Dios que me salvara de las manos de aquellos bárbaros. Así me quedé por espacio de unas dos horas, lleno de impaciencia por saber lo que ocurría más allá y careciendo de exploradores o espías que fuesen a buscar novedades.

    Después de estarme quieto, pensando qué debía hacer en la emergencia, no pude resistir por más tiempo la inactividad, de manera que coloqué la escalera haciéndola llegar como ya he descrito hasta el sitio donde la roca formaba una especie de plataforma; levantando luego la escala y volviéndola a colocar en dicho apoyo, me encaramé a la cresta de la colina. Me había tirado de boca contra el suelo, y con ayuda del anteojo que trajera ex profeso empecé a buscar el sitio donde ardía el fuego. Pronto descubrí que había nueve salvajes desnudos que rodeaban una hoguera, no para calentarse, ya que ninguna falta les hacía el calor en ese clima ardiente, sino probablemente para entregarse a alguno de sus horribles banquetes de carne humana que habrían traído consigo, aunque no alcanzaba a distinguir a los posibles prisioneros.

    Vi dos canoas que habían arrastrado fuera del agua; y como la marea estaba baja, parecían a la espera del flujo para embarcarse nuevamente. No es fácil describir mi estado de ánimo contemplando aquella escena, sobre todo al darme cuenta de que ocurría de este lado de la isla y tan cerca de mí. Pero al comprender que probablemente los desembarcos acontecían en el momento del reflujo, me tranquilicé un poco pensando que me sería posible salir con toda tranquilidad siempre que al empezar la marea no hubiese visto antes aproximarse las canoas. Esto me permitió proseguir con más calma las tareas de la cosecha.

    Ocurrió tal como lo esperaba. Tan pronto creció la marea vi a los salvajes embarcarse y remar (o más bien palear) hacia fuera. Olvidaba decir que durante la hora y media que precedió a su marcha estuvieron bailando en la playa, y que con ayuda de los anteojos pude ver perfectamente sus movimientos y ademanes.

    Tan pronto se alejaron me eché dos escopetas a la espalda, y con dos pistolas al cinto y la gran espada sin vaina al costado, corrí con toda la rapidez posible a la colina donde por primera vez había tenido noticia de los salvajes. Cuando llegué allá, después de dos horas de fatigosa marcha, cargado como estaba con tantas armas, descubrí que en ese lugar habían atracado otras tres piraguas; mirando hacia el mar alcancé a verlas todavía mientras se internaban en el océano.

    Aquello era espantoso de ver, pero algo peor me esperaba cuando descendí a la playa y encontré los restos que después del atroz festín habían quedado diseminados; sangre, huesos, trozos de carne humana que aquellos monstruos habían devorado en medio de danzas y júbilo. Tan lleno de indignación me sentí a la vista del horrendo espectáculo que empecé inmediatamente a premeditar la destrucción de los que desembarcasen una próxima vez en la isla, sin importarme su número.

    Transcurrieron con todo un año y tres meses antes de que volviera a ver a los salvajes, como contaré en su lugar. Es probable sin embargo que vinieran una o dos veces, pero se quedaron muy poco tiempo o yo no tuve noticia de su presencia. En el mes de mayo, según creo recordar, y en el año vigésimo cuarto de mi residencia, tuve un extraño encuentro con ellos que narraré en su debido momento.

    Durante ese intervalo de quince o dieciséis meses, la perturbación de mi espíritu fue grande. Dormía mal, despertándome en medio de terribles pesadillas y sobresaltado. Como de día no abrigaba más que esa constante preocupación, tal inquietud se reflejaba en mis sueños, donde me veía matando salvajes o preguntándome cuál era el motivo para hacerlo. Pero, dejando esto por el momento, diré que a mediados de mayo, creo que el dieciséis según los inseguros datos de mi calendario de madera que yo trataba de mantener al día; el dieciséis, digo, se levantó una gran tormenta de viento, con relámpagos y truenos, y la noche que siguió fue tempestuosa. No recuerdo exactamente las circunstancias, pero sí que me encontraba leyendo la Biblia y meditando seriamente en mi presente condición cuando escuché, viniendo del mar, un sonido semejante al de un cañonazo.

    Sentí una sorpresa muy distinta de las que había experimentado hasta entonces, porque las ideas que aquel cañonazo despertaron en mí eran de naturaleza harto diferente. Me lancé como un rayo fuera de mi tienda, y en un santiamén puse la escalera contra la roca, la retiré, volví a colocarla en el segundo apoyo y me encaramé a la cumbre de la colina en el preciso instante en que un destello me anunciaba el segundo cañonazo, cosa que efectivamente escuché medio minuto más tarde; y por el sonido deduje que venía del lado del mar hacia donde una vez la corriente me arrastrara con el bote.

    De inmediato comprendí que se trataba de un navío en peligro, que tal vez disparaba los cañonazos en demanda de socorro a otro navío que navegaba cerca. Tuve presencia de ánimo para pensar que aunque yo nada podía hacer por ellos, acaso ellos pudiesen hacer mucho por mí, de manera que juntando toda la leña seca que había a mi alcance y encendiéndola, iluminé con una gran hoguera la cumbre de la colina. Aunque el viento era muy fuerte, la madera seca ardió de inmediato, dándome la certeza de que si en verdad un barco navegaba en las cercanías tendría que enterarse de mi presencia. Y no dudo que así fue, porque apenas había alzado la hoguera cuando resonó otro cañonazo y después varios seguidos provenientes del mismo lugar. Mantuve encendido el fuego toda la noche; cuando fue día claro y despejado alcancé a divisar algo a una gran distancia en el mar, hacia el lado este de la isla, aunque no podía decir si era un casco o una vela. Ni siquiera con ayuda del anteojo pude reconocerlo a esa distancia, ya que aún persistía una cierta niebla.

    Miré todo el día en aquella dirección, y no tardé en darme cuenta de que no se movía; evidentemente era un barco fondeado. Ansioso por saciar mi curiosidad, tomé la escopeta y corrí hacia el sur de la isla buscando aquellas rocas donde la corriente me había arrebatado con la canoa. El tiempo estaba muy claro, y trepando a la altura pude ver con toda nitidez y profunda aflicción que el barco había naufragado durante la noche en aquellas rocas ocultas que prolongaban el cabo y que yo había visto desde mi bote; las mismas rocas que, oponiéndose a la violencia de la corriente y haciendo una especie de contracorriente o remolino, me salvaran de la más desesperada situación en que jamás me viera antes.

    Lo que salva a un hombre puede perder a otro. Estaba claro que aquellos marinos, ignorantes de la costa y de los arrecifes, habían sido arrastrados hacia ellos por el fuerte viento que toda la noche soplara del este y E-NE. De haber visto la isla —cosa al parecer muy improbable— lo más lógico era que hubiesen intentado llegar a tierra embarcándose en la chalupa; pero aquellos cañonazos en demanda de auxilio, especialmente después de haber visto, según yo suponía, mi hoguera, me llenaban de ideas contradictorias. Pensé primero que tras de divisar mi fuego se habrían embarcado en el bote del barco y puesto rumbo a la costa, pero que estando el mar embravecido los habría arrastrado lejos. Luego imaginaba que habrían perdido la chalupa antes de encallar, como tantas veces ocurre, en especial cuando el oleaje barre la cubierta y obliga a los marineros a soltar el bote o romperlo para precipitarlo sobre la borda. Después pensé que otro navío, escuchando aquellas llamadas, se habría acercado y recogido a los náufragos. Por fin imaginé a la tripulación mar afuera en la chalupa, arrastrada por la gran corriente marina que la llevaría hacia la desolada extensión del océano donde sólo reina la muerte. Acaso en este instante empezaban a sentir hambre, y pronto estarían en estado de comerse los unos a los otros.

    Todas aquellas eran conjeturas, pero en la situación en que me encontraba yo, ¿qué otra cosa podía hacer sino meditar sobre la desgracia de aquellos hombres y apiadarme de ellos? Una vez más pude comparar por su suerte lo que debía agradecer a Dios, que tanto y tan bien me había asistido en mi desdicha. De dos enteras tripulaciones ahora perdidas en esta región del mundo, ninguna vida se había salvado más que la mía. Aprendí nuevamente que es muy raro que la Providencia de Dios nos abandone a una vida tan baja y miserable como para no tener oportunidades de mostrarnos agradecidos, aunque sólo sea viendo a otros en peores condiciones que nosotros.

    No puedo expresar con ningún lenguaje la ansiedad que se apoderó de mí, la violencia de mis deseos al contemplar el triste espectáculo que me obligó a prorrumpir en exclamaciones:

    — ¡Oh, que por lo menos se hayan salvado uno o dos, aunque solamente sea uno! ¡Que pueda yo tener un compañero, un semejante con el cual hablar, con el cual vivir!

    En todos aquellos años de vida solitaria nunca había sentido una necesidad tan grande de tener compañía; y nunca su falta se tradujo en una melancolía más honda.

    Así estaba dispuesto. Su destino o el mío, acaso ambos, lo prohibían; hasta el último año de mi permanencia en la isla ignoré si alguno se había salvado de la catástrofe. Tuve con todo el dolor de encontrar en la playa, algunos días más tarde, el cadáver de un grumete ahogado. Yacía en la parte próxima al sitio del naufragio y por ropas tenía una chaqueta de marino, un par de calzones abiertos y una camisa de tela azul; no llevaba nada que me permitiera conocer su nacionalidad. Encontré en sus bolsillos dos piezas de a ocho y una pipa, que para mí valía diez veces más que el dinero.

    Había vuelto la calma, y sentí deseos de aventurarme en mi canoa hasta el casco encallado, con la seguridad de encontrar a bordo cosas que me fueran útiles. Lo que más me impulsaba a hacerlo era la esperanza de que en la nave pudiese haber quedado alguien con vida y no sólo me alentaba el deseo de salvar esa vida sino que imaginaba lo que para mí significaría adquirir en esa forma un compañero. Tanto me torturó la idea que no encontraba un instante de paz, ni de día ni de noche, y me repetía que era necesario arriesgarme y llegar hasta el casco. Tan fuerte era mi ansiedad que terminé por encomendarme a la Providencia Divina y pensar que aquel impulso provenía de lo alto, que me equivocaba al resistirlo y que cometería una falta si dejaba transcurrir más tiempo.

    Dominado por una fuerza superior a mí, me apresuré a regresar al castillo y hacer los preparativos del viaje, reuniendo buena cantidad de pan, una tinaja de agua dulce, brújula, una botella de ron del que me quedaba buena cantidad y un canasto de pasas. Cargado con todo aquello fui al sitio donde fondeaba mi bote, achiqué el agua que contenía y después de depositar el cargamento volví en procura de más. Este consistió en un saco grande de arroz, la sombrilla para fijar en la popa, otra tinaja de agua y dos docenas de panecillos de cebada, a lo que agregué también una botella de leche de cabra y un queso. Con gran trabajo pude llevar todo hasta el bote, y rogando a Dios que dirigiera mi rumbo me embarqué de inmediato. Ayudado por los remos y sin apartarme de la costa, llegué por fin al punto extremo de la isla, es decir, al noroeste. Ahora se trataba de penetrar en el océano, de aventurarse o no en la empresa. Miré las rápidas corrientes que corrían a ambos lados de la isla y que tanto terror me producían al recordar el peligro en que estuviera; sentí que mi corazón me abandonaba, porque estaba seguro de que llevado por cualquiera de ellas me internaría de tal modo en el mar que la isla quedaría fuera de mi vista y de mi alcance. Sólo con que se levantara una simple brisa, mi pequeño bote naufragaría irremisiblemente.

    Tanto me angustiaron estos pensamientos que pensé en abandonar la empresa. Llevando el bote hasta una pequeña caleta en la playa, desembarqué y sentado en una eminencia me puse a pensar, abatido y ansioso a la vez, luchando entre el miedo y el deseo. En esta perplejidad advertí que cambiaba la marea y que empezaba el flujo, de manera que mi posible viaje se tornaba impracticable durante muchas horas.

    Decidí entonces trepar al terreno más alto de las inmediaciones para tratar de ver en qué dirección y cómo se movían las corrientes de la marea, a fin de saber a ciencia cierta si, en caso de que mi bote fuese arrastrado mar afuera, la misma marea no podría traerme otra vez a la costa con igual rapidez y fuerza. Apenas había pensado en esta posibilidad cuando ya me encaramaba a una pequeña colina lo bastante elevada para tener visión completa del mar y sus movimientos, buscando calcular qué rumbo debería seguir a mi retorno del casco. Descubrí que así como la corriente de reflujo pasaba rozando el extremo sur de la isla, la motivada por el flujo lo hacía contra la costa del norte, de manera que cuidando de llevar el bote hacia allá podría volverme a tierra sin peligro.

    Animado por mi descubrimiento decidí embarcarme con la marea matinal, y después de haber pernoctado en la canoa al abrigo del capote de marino que ya he mencionado, zarpé temprano. Al comienzo puse rumbo al norte hasta que empecé a sentir la fuerza de la corriente que me arrastró un buen trecho hacia el este, aunque no con la terrible violencia que lo hiciera la corriente austral en la anterior ocasión que me privó de todo gobierno de la canoa. Con ayuda de los remos pude encaminar el bote hacia el sitio del naufragio, y en menos de dos horas me encontraba junto al casco encallado.

    ¡Lamentable espectáculo para mis ojos! El barco, que por sus líneas parecía español, estaba como encajado entre dos rocas, la popa y buena parte de su casco destrozadas por el oleaje; el castillo de proa, incrustado en las rocas, había recibido tal golpe que el palo mayor y el trinquete se quebraron en la base. Sin embargo el bauprés estaba entero y el esperón parecía firme.

    Al acercarme, vi a un perro en la borda que al divisarme aulló y ladró. Apenas lo hube llamado cuando se arrojó al mar y pronto estuvo a bordo casi muerto de hambre y de sed. Le di una galleta y la devoró como un lobo salvaje que llevara dos semanas en la nieve sin comer. Le ofrecí después agua dulce, y bebió tanta que de haberlo dejado hacer su gusto hubiera reventado.

    Subí a bordo; lo primero que alcanzaron a ver mis ojos fueron dos hombres ahogados en la cocina, sobre el castillo de proa; estaban estrechamente abrazados, y comprendí por su actitud que al encallar el buque en medio de la tempestad, tan alto había sido el oleaje y de tal modo barría la cubierta que aquellos infelices no habían podido resistirlo, ahogándose a bordo lo mismo que si hubieran estado bajo el agua. Fuera del perro, nada quedaba con vida en aquel navío; y por lo que alcancé a ver el cargamento estaba averiado. Descubrí algunos cascos de licor, ignoro si vino o aguardiente, que se apilaban en la sentina y eran visibles con la marea baja; pero mis fuerzas no bastaban para moverlos de su lugar. Había también numerosos arcones, pertenecientes sin duda a los tripulantes; eché dos de ellos en mi bote, sin perder tiempo en examinar el contenido.

    Si al encallar el barco se hubiera destrozado la proa en vez de la popa, estoy seguro de que mi viaje habría resultado fructífero, ya que de acuerdo con lo que encontré en los dos arcones el navío tenía muchas riquezas a bordo. Calculando por el rumbo que llevaba en el momento de naufragar, supuse que había sido fletado desde Buenos Aires, o el Río de la Plata, en la parte austral de América más allá del Brasil, y que su destino era La Habana, en el Golfo de México, o tal vez España. Llevaba un gran tesoro a bordo que de nada serviría ya, y el destino de su tripulación era entonces para mí un misterio.

    Aparte de los arcones encontré un pequeño barril de licor de unos veinte galones, que con no poco trabajo puse en el bote. Había muchos mosquetes en una cabina y un frasco de pólvora conteniendo no menos de cuatro libras. Los mosquetes no me eran necesarios, pero sí la pólvora, por lo cual la tomé, así como una pala y tenazas, que me hacían muchísima falta. Di con un par de ollítas de cobre, una chocolatera y unas parrillas, y con ese cargamento, además del perro, emprendí el regreso aprovechando la marea que empezaba a subir. Esa misma tarde, ya entrada la noche, alcancé la isla, donde desembarqué melancólico y fatigado hasta la extenuación.

    Pasé la noche en el bote, y por la mañana decidí guardar mis nuevos efectos en la gran caverna en vez de conducirlos al castillo. Después de alimentarme puse el cargamento en tierra y empecé a examinarlo con detalle. El casco de licor contenía una especie de ron, pero no como el que se bebe en el Brasil, que es harto más bueno. Sin embargo los arcones me consolaron porque contenían diversas cosas de gran utilidad. Por ejemplo encontré una caja de extraordinaria forma, llena de botellitas conteniendo cordiales de excelente calidad y delicado sabor; cada botella tenía unas tres pintas de licor y estaba cerrada con tapón de plata. Había también dos frascos de frutas en almíbar o confitadas, tan bien cerrados que el agua de mar no los había dañado; otros dos, en cambio, estaban averiados. Encontré algunas excelentes camisas que fueron un verdadero regalo, y una docena y media de pañuelos de hilo blanco, así como corbatas de color. Los pañuelos me llenaron de contento, ya que me serían muy útiles para enjugarme el rostro en los días calurosos. Luego, al mirar en el fondo del cofre, vi tres grandes sacos conteniendo piezas de a ocho, lo que daba unas mil piezas en total; en uno de los sacos y envueltos en papel hallé seis doblones y algunas barritas de oro. En conjunto creo que pesaban cerca de una libra.

    El otro arcón contenía también algunas ropas, pero de poco valor; supuse que el cofre era el perteneciente al oficial de artillería, pero aunque busqué pólvora sólo pude dar con tres frascos pequeños conteniendo una pólvora muy fina y brillante, que sin duda se reservaba para cargar las escopetas de caza. En realidad mi expedición al barco me fue de poco provecho; por lo que respecta al dinero no tenia oportunidad de usarlo, y me importaba tanto como la tierra que pisaban mis pies. Lo hubiera dado íntegramente a cambio de tres o cuatro pares de zapatos ingleses y de medias, que mucha falta me hacían desde varios años atrás. Cierto que era dueño de dos pares de zapatos que quité a los marineros ahogados que viera en el puente, y en uno de los cofres encontré otros dos pares; pero no eran como nuestros zapatos ingleses, ni por su solidez ni por su comodidad, mereciendo más el nombre de escarpines. En uno de los cofres hallé unas cincuenta piezas de a ocho en reales, pero no oro. Presumo que el arcón pertenecía a uno de los marineros, mientras el otro debió ser de un oficial.

    De todos modos me llevé el dinero a la caverna, donde lo puse junto al que extrajera de mi propio barco. Era una verdadera lástima que la parte más importante del buque no hubiera estado a mi alcance, ya que tengo la seguridad de que habría podido llenar de oro mi canoa varias veces, y acumulado riquezas suficientes en la gruta para llevármelas conmigo si alguna vez conseguía escapar de la isla.

    Asegurado el cargamento sólo me quedaba volver a mi bote y remar en él hasta dejarlo fondeado en su vieja ensenada; de allí, luego de asegurarlo bien, volví a mi morada donde todo estaba en orden y sin novedad. Tras de haber descansado lo bastante reanudé mi existencia habitual cuidando de mis intereses domésticos; por un buen espacio de tiempo viví sin inquietudes, sólo que ponía más cuidado en mis movimientos y no me alejaba tan a menudo de mi casa. Si algún paseo emprendía era hacia el lado oriental de la isla, donde contaba con la seguridad de que los salvajes no desembarcarían nunca. Eso me evitaba adoptar tantas precauciones y llevar conmigo un enorme peso en armas y municiones, absolutamente necesario cuando me encaminaba en dirección opuesta.

    Dos años más transcurrieron en tales condiciones; pero mi malhadada imaginación, siempre dispuesta a recordarme que yo había nacido para hacer de mí un desdichado, estuvo todo ese tiempo fraguando proyectos y planes para escapar de la isla; a veces me sugería la conveniencia de hacer otro viaje hasta el casco encallado, aunque la razón me decía claramente que nada quedaba allí que me sirviera; otras veces me insinuaba navegar hacia un lado o hacia otro. En fin, estoy convencido de que si hubiera tenido a mi disposición la chalupa con la cual huí de Sallee me habría aventurado a cruzar el mar, con rumbo desconocido y destino incierto.

    En todas las circunstancias de mi vida yo he sido una especie de aviso para aquellos que también sufren la más grande plaga de la humanidad, plaga de la cual proviene por lo menos la mitad de sus desdichas; me refiero a los que no se sienten satisfechos con aquello que Dios y Naturaleza les han concedido.

    Una lluviosa noche de marzo, en el vigésimo cuarto año de mi existencia solitaria, reposaba en mi lecho o hamaca, despierto aunque sin sentir la menor molestia; mi salud era excelente, no tenía dolores ni la preocupación de mi mente era mayor que otras veces, y sin embargo no conseguía de ningún modo cerrar los ojos; me fue imposible dormir un solo instante en toda la noche.

    Sería tan difícil como inútil tratar de describir la innumerable multitud de pensamientos que se precipitaban a través de ese vasto camino del cerebro que es la memoria. Volvía a ver la entera historia de mi vida, aunque en miniatura o compendiada, hasta mi arribo a la isla; y también la siguiente etapa solitaria de mi existencia.

    Mi mente se detuvo un cierto tiempo a considerar las costumbres de aquellos miserables salvajes, y me pregunté cómo podía ocurrir en este mundo que el sabio Rector de todas las cosas hubiera podido dejar caer alguna de sus criaturas hasta semejante grado de inhumanidad, algo todavía por debajo de la brutalidad, como lo es devorar a sus semejantes. Pero terminando aquellas ideas en inútiles consideraciones, se me ocurrió de pronto preguntarme en qué parte del mundo vivían aquellos monstruos. ¿Estaba muy lejos la costa desde donde venían? ¿Por qué se aventuraban a apartarse tanto de su tierra? ¿Qué clase de canoas tenían? Y por primera vez encaré la posibilidad de lanzarme a un viaje que me llevase hasta el país de los salvajes, así como ellos eran capaces de llegar al mío.

    No sentí en ese momento la menor preocupación por lo que me esperaría al arribar allá. Ignoraba qué iba a ser de mí si era apresado por los salvajes, o cómo me las arreglaría para impedirlo. Tampoco se me ocurrió la manera de llegar hasta sus playas sin que me alcanzaran antes con sus piraguas, cosa de la que me sería imposible defenderme. Y luego, aun si me salvaba de sus manos, ¿cómo evitar morirme de hambre, cuál debería ser mi rumbo en tierra firme? Nada de todo eso, lo repito, cruzó entonces por mi cerebro; demasiado absorbido estaba con la esperanza de llegar al continente. Me limitaba a considerar mi actual situación como la más miserable que pudiera imaginarse, y creía que nada, salvo la muerte, podría parecerme peor que ella. Traté de animarme con la idea de que, ya en tierra firme, encontraría pronto algún socorro o bien podría ir costeando el continente como ya una vez lo hiciera en África, hasta dar con un país habitado donde me auxiliaran. Tal vez en mi camino encontrara algún buque cristiano que me recibiera a bordo, y si venía lo peor, sólo tenía la muerte por delante, lo cual era una manera de terminar de una vez con todas aquellas desdichas.

    Estos pensamientos se agitaron en mí por espacio de dos horas o más, con tal violencia que mi sangre parecía arder y mi pulso latía como si estuviera bajo la acción de la fiebre. ¡Tal era la fuerza de mi imaginación y su poder! Pero la Naturaleza, como si quisiera rescatarme de tan gran fatiga, terminó por sumirme en un profundo sueño. Se podría pensar que mis sueños siguieron el curso de aquellas ideas de la vigilia, pero nada de ello ocurrió, sino algo muy distinto.

    Soñé que, al salir como todas las mañanas del castillo, veía dos canoas en la costa y once salvajes que desembarcaban arrastrando a otro que sin duda se disponían a asesinar y comer; repentinamente, el salvaje prisionero se desasió de un salto y confió su vida a la velocidad de la carrera. Me parecía en mi sueño que se acercaba hasta ocultarse entre el espeso seto delante de mis fortificaciones; entonces, viendo que estaba solo y que sus enemigos no lo buscaban de ese lado, me mostré a él sonriéndole bondadosamente para darle ánimo. El salvaje cayó de rodillas ante mí, pareciéndome que me rogaba auxilio. Le mostré la escalera, haciéndolo entrar por ella en el castillo, y escondiéndolo en la cueva pronto se transformó en mi criado. Tan pronto como en sueños me sentía dueño de ese hombre, me decía: «Ahora puedo aventurarme sin temor al continente; este salvaje me servirá de piloto, me indicará qué debo hacer, si me conviene o no desembarcar en procura de provisiones, en fin, me irá evitando todo peligro de ser apresado y comido.»

    Me desperté bajo esa impresión y tal había sido el rapto de mi alegría ante la posibilidad de escapar de la isla, que el desencanto subsiguiente lo igualó en intensidad, sumiéndome en una profunda melancolía.

    Aquel sueño, sin embargo, me llevó a la conclusión de que mi única probabilidad de escapar de la isla estaba en apoderarme de algún salvaje, en lo posible algún prisionero traído a la isla para ser muerto y devorado. La dificultad del proyecto consistía en que no iba a ser fácil llegar a tal fin sin atacar antes a toda la pandilla de caníbales y matarlos. La tentativa podía muy bien fracasar, y a la vez se renovaban en mí los escrúpulos acerca de mi derecho a hacer una cosa semejante; mi corazón se estremecía a la idea de derramar tanta sangre aunque fuera para mi salvación. No necesito repetir todos los argumentos que acerca de esto se me ocurrieron, ya que son los mismos expuestos antes. Hasta había llegado a acumular nuevas excusas, como la de que aquellos salvajes eran un peligro para mi vida, pues si me echaban mano me devorarían; que mi proceder contra ellos equivalía a una defensa propia en su más extremo grado, ya que con él obtendría la liberación de esta existencia peor que la muerte; que si me adelantaba a atacarlos procedía con el mismo derecho que si ellos hubieran abierto el asalto, y otras cosas parecidas. Pero aunque todo aquello argüía en defensa de mis planes, la idea de verter sangre humana como precio de mi libertad se me antojaba terrible y durante mucho tiempo no pude conciliar ambas cosas en mi conciencia.

    Por fin, después de muchas y renovadas disputas conmigo mismo en las que pasaba por extraordinarias perplejidades, ya que los argumentos luchaban y se debatían en mi cerebro, las incontenibles ansias de libertad dominaron toda reserva y me decidí, costara lo que costase, a tratar de apoderarme de alguno de los salvajes.

    De inmediato se planteó el problema de llevar esto a la práctica, y no creo que haya tenido otro más arduo. No hallando por el momento solución plausible, me dediqué a hacer de centinela a la espera de que llegaran a tierra, dejando el resto confiado a los acontecimientos que por sí mismos me dictarían el caminó a seguir.

    Adoptadas estas resoluciones, principié a vigilar la costa casi de continuo y con tal intensidad que llegué a hartarme de ello, pues transcurrió más de un año y medio en espera, durante el cual casi diariamente iba yo hasta el extremo oeste o al ángulo sudoeste de la isla en busca de posibles canoas que jamás arribaban. La inacción era descorazonante, y empezó a torturarme con violencia, porque contrariamente a la vez anterior, en que el tiempo calmó mi irritación contra los salvajes, ahora parecía como si su ausencia exacerbara mi ansiedad por descubrirlos. Así como años atrás me mostraba deseoso de no tener contacto con aquellas gentes y evitaba hasta espiarlos, ahora me desvivía por las ganas de verlos desembarcar.

    Había pensado que quizá pudiera apoderarme no sólo de uno, sino de dos o tres de ellos, y confiaba en convertirlos en esclavos que no solamente me obedecieran en todo sino que resultaran incapaces de hacerme el menor daño. Imaginaba constantemente el modo de lograrlo, pero entretanto la isla continuaba desierta. Todos mis proyectos empezaron a sucumbir y pasó mucho tiempo sin que los salvajes se aproximaran a tierra.






	\chapter{Viernes}





    Llevaba así un año y medio, y después de haber abrigado tantos planes los veía desvanecerse en el aire por falta de ocasión para ejecutarlos. Una mañana, sin embargo, me sorprendió la presencia de unas cinco canoas en la costa, cuyas tripulaciones habían desembarcado y estaban fuera de mi vista. Todas mis previsiones se derrumbaron al pensar en el número de aquellos salvajes, porque viendo tantas canoas y seguro de que en cada una venían cuatro o cinco tripulantes, no se me ocurría la manera de atacar a veinte o treinta hombres con mis solas fuerzas. Perplejo y desilusionado permanecí en el castillo, pero adopté, al igual que la vez anterior, las necesarias precauciones en caso de ataque, y pronto estuve listo para repelerlo. Aguardé un rato tratando de oír si hacían algún ruido, mas como mi impaciencia crecía por momentos puse las escopetas al pie de la escalera y me encaramé a la cumbre de la colina con el procedimiento ya descrito, teniendo sin embargo buen cuidado de que mi cabeza no sobrepasara el nivel de la roca y quedara completamente oculta a las miradas de aquellos hombres. Con ayuda del anteojo vi que no eran menos de treinta, que acababan de encender una hoguera y aderezaban allí sus alimentos. No pude distinguir qué clase de carne era aquélla y de qué modo la cocían; pero los vi bailar como locos en torno al fuego, haciendo toda clase de contorsiones y bárbaros ademanes.

    Mientras los observaba, mi anteojo me mostró de pronto a dos miserables prisioneros que eran arrastrados desde las canoas y conducidos al sacrificio. Vi a uno de ellos caer inmediatamente, y supongo que lo golpearon con una maza o cachiporra, como es su costumbre habitual. Inmediatamente dos o tres salvajes se precipitaron sobre el caído y empezaron a descuartizarlo, mientras el otro desgraciado permanecía inmóvil y a la espera de que le llegara el turno. Pero en ese mismo instante, como el infeliz había sido descuidado por sus captores y el instinto le inspirara una esperanza de vida, echó a correr con velocidad increíble a lo largo de la playa, justamente en dirección al lugar donde se hallaba mi morada.

    Confesaré que el espanto se apoderó de mí al verle tomar esa dirección, y sobre todo cuando la pandilla entera se lanzó en su persecución. Pensé que mi sueño iba a cumplirse y que el salvaje se ocultaría en el bosquecillo; pero no contaba con que el resto del sueño se cumpliera igualmente, es decir, que los salvajes renunciaran a seguirlo por esos lados. Permanecí inmóvil y a la espera, y pronto recobré algo de ánimo al advertir que solamente tres hombres perseguían al prisionero, y más aún comprobando que su rapidez en la carrera era muy superior a la de aquéllos, con lo cual si conseguía mantenerla por una media hora jamás se pondría de nuevo a su alcance.

    Entre el lugar hasta donde habían llegado y mi castillo se encontraba la ensenada que he citado en la primera parte de mi relato, cuando desembarqué los efectos del buque. El perseguido debía necesariamente nadar a través de ella, o lo apresarían en la orilla. Lo vi llegar a toda carrera y, sin preocuparse de que la marea estaba alta, zambullirse y lanzarse a la otra orilla sin perder un segundo; en unas veinte brazadas alcanzó el lado opuesto y allí siguió corriendo aún con más rapidez y energía que antes. Los tres perseguidos llegaron de inmediato a la ensenada, pero solamente dos sabían nadar; el otro, luego de mirar al fugitivo, no se animó a tirarse al agua y poco después se volvió lentamente hacia atrás, lo que fue para su propio bien.

    Desde mi apostadero pude observar que los dos perseguidores emplearon el doble de tiempo que el perseguido en cruzar la ensenada. Entonces me invadió el impulso irresistible de procurarme allí mismo el criado, o tal vez el compañero y ayudante que necesitaba, y pensé que la Providencia me había designado para salvar la vida de aquel infeliz. Descendí a toda velocidad por la escalera, tomé las armas que, como he dicho, había dejado al pie de ésta, y volví a subir a la cresta de la colina.

    Marchando en dirección al mar, como había un camino de atajo que descendía bruscamente de la colina a la playa, pronto me hallé entre el perseguido y los perseguidores, llamando a aquél en alta voz. Cuando, al mirar hacia atrás, me vio distintamente, tuvo más miedo de mí que de los otros, pero le hice señas con la mano de que se acercara; entretanto avancé sigiloso hacia los dos salvajes, y saltando bruscamente sobre el que venía adelante lo .derribé de un culatazo. No me atrevía a disparar el arma por temor a que el resto oyera el ruido, aunque a tan gran distancia no era fácil, máxime que tampoco podrían ver el humo y orientarse por él. Ya en el suelo el salvaje, el otro que iba más atrás se detuvo como aterrado; me acerqué lentamente, pero entonces vi que tenía un arco y flechas, que estaba armando para atravesarme, por lo cual no quedó otro remedio que disparar sobre él y lo derribé muerto al primer tiro.

    El pobre salvaje fugitivo, que ante mi actitud permanecía inmóvil a cierta distancia, vio a sus enemigos caídos y muertos, pero tuvo un terror tan grande al oír el estampido de la escopeta que se quedó como piedra, incapaz de avanzar o retroceder y, sin embargo, con más ganas de seguir huyendo que de venir hacia mi. Lo llamé otra vez, haciéndole signos de que se aproximara, lo que entendió fácilmente. Dio unos pasos, se detuvo, luego caminó otro trecho y volvió a pararse; advertí que temblaba a la idea de sufrir el mismo destino que sus perseguidores. Insistí en hacerle señas de que se acercara, tratando de demostrarle en toda forma que no le haría nada, para animarlo; fue aproximándose lentamente, pero cada diez o doce pasos se arrodillaba en señal de reconocimiento por haberle salvado la vida. Le sonreí de la manera más cariñosa, haciéndole seña de que se adelantara aún más, y por fin llegó a mi lado.

    Entonces, dejándose caer de rodillas, besó el suelo y apoyó en él su cabeza, y tomando mi pie lo puso sobre ella, lo que sin duda significaba su voluntad de hacerse mi esclavo por toda la vida.

    Lo levanté, acariciándolo y tratando de devolverle el coraje en todo lo posible. Sin embargo aún había tarea que realizar, porque de pronto advertí que el salvaje que golpeara con la culata no estaba muerto sino solamente desmayado y daba señales de recobrar los sentidos. Le apunté con la escopeta mientras hacía señas a mi salvaje para que reparara en su enemigo; comprendiendo, me habló algunas palabras que, aunque carentes para mí de sentido, fueron muy dulces de oír, ya que era el primer sonido humano que escuchaba yo en aquella isla después de veinticinco años.

    Pero no había tiempo ahora para reflexiones: el salvaje se recobraba poco a poco de su desmayo, lo vi que se sentaba en el suelo y advertí que mi compañero principiaba a asustarse otra vez, por lo cual le ofrecí la otra escopeta por si quería emplearla. Pero él me señaló por el contrario la espada que yo llevaba desnuda en la cintura, y se la alcancé. Apenas lo había hecho cuando lo vi precipitarse sobre su enemigo y cortarle la cabeza de un solo tajo con tal destreza que el mejor verdugo de Alemania no lo hubiese hecho más pronto ni mejor.

    Aquello me asombró en un hombre que, según imaginaba yo, jamás había visto antes una espada, salvo las de madera que usan esos pueblos. Más tarde, sin embargo, vine a saber que fabrican sus espadas con una madera tan dura como pesada, y que el filo es tan agudo que con ellas pueden decapitar de un golpe, e incluso tajar un brazo entero.

    En cuanto a él, después de matar a su enemigo, vino hacia mi riendo en señal de triunfo, y con abundancia de ademanes que no entendí depositó a mis pies la espada ensangrentada y la cabeza del salvaje.

    Lo que más lo pasmaba era la forma en que yo había matado al otro indio, y señalándolo parecía pedirme permiso para ir a examinarlo, lo que le concedí lo mejor que pude. Cuando llegó junto al cadáver se quedó como helado, mirándolo por todas partes; lo dio vuelta a un lado, después al otro, observó la herida de la bala, que había alcanzado a darle en el pecho, haciendo un orificio del cual manaba muy poca sangre, ya que la muerte se había producido por hemorragia interna. Por fin tomó el arco y las flechas y vino junto a mí, que me disponía a regresar. Le hice señas de que me siguiera, tratando de atemorizarlo a la vez con la idea de que otros salvajes podían presentarse de improviso.

    Como entendiera muy bien, me hizo señas de que lo dejara enterrar los cuerpos en la arena para que el resto de la pandilla no los encontrara. Cuando asentí se puso a cavar un hoyo con las manos, y pronto fue lo bastante grande para enterrar a uno de los muertos; repitió el procedimiento y un cuarto de hora más tarde los dos estaban sepultados. Llamándolo entonces lo llevé conmigo, no al castillo, sino a la gruta que quedaba en el otro extremo de la isla; de modo que no dejé cumplirse el sueño en aquella parte, según la cual el salvaje había buscado refugio en mi soto.

    Le di pan y un racimo de pasas, así como agua, de la que estaba muy necesitado después de aquella carrera.

    Luego que se hubo refrescado le hice signos de que se acostara a dormir, señalándole un sitio donde había un colchón de paja de arroz cubierto con una manta, que yo empleaba a veces para descansar allí. El pobre obedeció y pronto estuvo dormido.

    Era un individuo bien parecido, muy bien formado y fuerte, no demasiado alto pero de gran esbeltez, que contaría según calculé unos veintiséis años. Tenía un rostro agradable, sin ninguna fiereza ni ferocidad, aunque advertí que sus facciones eran muy varoniles; cuando sonreía, encontraba yo en su rostro toda la suavidad y la dulzura de los europeos. Su largo y negro cabello no se encrespaba como lana; la frente era ancha y despejada, y había vivacidad e inteligencia en su mirada. La piel no era negra sino atezada, pero sin ese desagradable matiz amarillento de los naturales del Brasil, Virginia y otros lugares americanos, sino más bien un aceitunado oscuro que resultaba muy agradable de ver aunque no sea fácil describirlo. La cara era redonda y llena, con una nariz pequeña y no aplastada como la de los negros, una boca firme de labios pequeños y dientes tan perfectos y blancos como marfil.

    Luego que hubo dormitado, más que dormido, una media hora, se levantó y saliendo de la gruta fue hacia donde estaba yo terminando de ordeñar las cabras que guardaba en ese sitio. Cuando me divisó vino corriendo a arrodillarse otra vez a mis plantas, con fervientes demostraciones de reconocimiento y humildad, haciendo mil gestos para que yo comprendiera. Por fin apoyó la cabeza contra el suelo junto a mi pie, y volvió a levantar mi otro pie y colocárselo encima, tras lo cual hizo todos los ademanes posibles de sumisión y servidumbre para darme a entender que sería mi esclavo por siempre. Comprendí bastante todo esto, y traté de demostrarle que me sentía muy contento con él. Poco después empecé a hablarle, a fin de que aprendiera a contestarme poco a poco. Ante todo le hice saber que su nombre sería Viernes, ya que en este día lo salvé de la muerte y me pareció adecuado nombrarlo así. A continuación le enseñé a que me llamara amo y a que contestara SÍ' o no, precisándole la significación de ambas cosas. Llené de leche un cacharro que puse en sus manos, mostrándole primero cómo se bebía aquello y mojando mi pan en la leche; de inmediato hizo lo mismo, dando señales visibles de que le gustaba mucho.

    Lo tuve conmigo aquella noche, y a la mañana siguiente le indiqué que me siguiera, haciéndole comprender que le daría algunas ropas para que se vistiera, ya que estaba completamente desnudo. Cuando cruzamos el lugar donde había enterrado a los dos salvajes me señaló con precisión el sitio, mostrándome las marcas que había hecho para encontrarlos otra vez, y comprendí por sus signos que me invitaba a desenterrarlos y comerlos. A esto me mostré encolerizado, dándole a entender la repugnancia que me producía la sola idea, e hice como si su intención me causara náuseas, ordenándole que se alejara de allí al punto, cosa que hizo con gran sumisión. Lo llevé conmigo hasta la cumbre de la colina, para observar si sus enemigos habían vuelto a embarcarse; con ayuda del anteojo recorrí la costa y aunque encontré el lugar donde se habían congregado no descubrí la menor señal de su presencia, lo que indicaba evidentemente que se habían marchado sin inquietarse en lo más mínimo por la suerte de sus dos compañeros.

    No contento con este descubrimiento, y como el mayor coraje aumentaba en igual grado mi curiosidad, confié a Viernes mi espada así como el arco y flechas que llevaba a la espalda y que sabía usar diestramente; le di también una escopeta para mí, y llevando yo otras dos, nos encaminamos hacia la costa donde habían pernoctado los salvajes. Cuando estuvimos allí la sangre se me heló en las venas y me pareció que mi corazón se detenía; ¡tan atroz era el espectáculo! Me quedé inmóvil de espanto, aunque Viernes no parecía conmovido en lo más mínimo. El lugar estaba cubierto de huesos humanos, el suelo tinto en sangre; grandes trozos de carne aparecían diseminados aquí y allá, devorados a medias y carbonizados; en fin, eran los testimonios del banquete triunfal con que aquellos salvajes habían celebrado la victoria sobre sus enemigos. Encontré tres cráneos, cinco manos y los huesos de tres o cuatro piernas y pies, así como abundancia de otras porciones de carne humana.

    Por medio de signos, Viernes me dio a entender que habían traído cuatro prisioneros para devorar, que aquellos restos pertenecían a tres y que él —se apuntaba con la mano— era el cuarto. Me explicó del mismo modo que había habido una gran batalla entre aquellos salvajes y los súbditos de un rey vecino, del cual parecía ser vasallo, y que habiendo resultado vencedores los otros, habían tomado gran número de prisioneros que fueron conducidos a distintos lugares para servir de pasto en el bárbaro festín de la victoria; un grupo de aquellos miserables era el que había desembarcado en mi isla.

    Ordené a Viernes que reuniera los cráneos, huesos y demás restos e hiciera con ellos una pirámide y le pegara fuego hasta que se calcinaran. Observé que se mostraba harto dispuesto a comerse parte de aquella carne, y que seguía siendo caníbal en su naturaleza; pero di tantas señales de repugnancia a la sola idea de semejante cosa, que no se atrevió a manifestar sus verdaderos instintos, ante todo porque yo le había dado a entender que si cedía a ellos no vacilaría en matarlo.

    Terminada la tarea volvimos al castillo, donde empecé a trabajar para mi criado Viernes. Ante todo le di unos calzoncillos de lienzo que encontrara en el arcón del pobre artillero y que rescaté del naufragio; con pequeñas modificaciones, le sentaron muy bien. Luego hice una chaqueta de piel de cabra, lo mejor que me fue posible, ya que era un discreto sastre; le di una gorra de piel de liebre, muy cómoda y pasablemente elegante, con lo cual quedó bastante presentable y me pareció satisfecho de verse igual que su amo. Cierto que al comienzo se sentía incómodo con aquellas ropas; los calzoncillos le estorbaban enormemente, y las mangas de la chaqueta le lastimaban los hombros y la piel de los brazos. Pero cuando se quejó de ello le hice los retoques convenientes y pronto se habituó sin la menor dificultad.

    Al siguiente día de tenerlo conmigo empecé a considerar dónde alojaría a mi criado. Se necesitaba un sitio que fuera cómodo para él y conveniente para mí, de modo que terminé levantando una pequeña tienda en el espacio libre que quedaba entre las dos fortificaciones, es decir, en el interior de la segunda y el exterior de la primera. Como justamente allí estaba la abertura que permitía entrar en la cueva, construí una verdadera puerta, clavando tablas sólidas en un marco del tamaño conveniente y fijándola en el interior del pasaje a la cueva. La puerta se abría hacia adentro, y de noche la aseguraba sólidamente teniendo también la precaución de retirar las escaleras, con lo cual nunca hubiera podido Viernes llegar hasta mí sin hacer mucho ruido que me hubiera despertado de inmediato.

    Es de recordar que mi primera empalizada tenía ahora un verdadero techo, formado por largas pértigas que cubrían enteramente la tienda y se apoyaban en la roca; sobre ellas había colocado troncos finos en lugar de vigas, y todo estaba cubierto espesamente con paja de arroz, tan sólida como si fuese caña. En el agujero que dejé para salir por la escalera había instalado una especie de trampa, que, al intentar abrirla desde afuera, hubiese caído, con gran estrépito. En cuanto al armamento, lo guardaba todas las noches conmigo.

    Sin embargo ninguna de estas precauciones resultó necesaria, porque nunca hombre alguno tuvo un sirviente tan fiel, amante y sincero como lo fue Viernes conmigo. Sin violencias, enojos o mala intención, se mostraba profundamente adicto y dispuesto; su afecto por mí parecía más bien el de un hijo por su padre, y me atrevo a decir que hubiera sacrificado voluntariamente su vida para salvar la mía en cualquier ocasión. Muchos testimonios me dio de ello, y pronto me convencí de que era inútil emplear con él aquellas excesivas precauciones

    Esto me dio oportunidad de pensar frecuentemente y con no poca maravilla que si Dios, en su Providencia y en el gobierno de su Creación, había decidido privar a tantas de sus criaturas del mejor empleo de sus facultades y sentimientos, sin embargo los había dotado de las mismas disposiciones, la misma razón, iguales afectos y sentimientos de humildad y devoción, así como de las mismas pasiones y resentimientos ante las ofensas, sentido de gratitud, sinceridad, fidelidad y todo el poder de hacer el bien y recibirlo que diera a sus demás criaturas. Y que cuando a Dios le placía ofrecerles oportunidades de ejercitar aquellas cualidades, estaban tan dispuestos y hasta parecían más capaces que nosotros en emplearlas para el bien.

    Pero volvamos a mi nuevo compañero. Me sentía muy contento con él, y traté de enseñarle en seguida aquellas cosas que lo tornarían útil, capaz y diestro. Mi mayor deseo era enseñarle a hablar, y que entendiera lo que yo le decía. Nunca se encontró mejor alumno que él; se mostraba tan contento, tan aplicado y daba muestras de tal alegría cuando alcanzaba a comprenderme o lograba que yo lo entendiera a él, que resultaba un placer hablarle. Mi vida se tornó tan placentera que con frecuencia me decía que de no mediar el peligro de los salvajes no me hubiera afligido tener que quedarme allí para siempre.

    Después de llevar dos o tres días en el castillo pensé que para alejar a Viernes de su horrible costumbre de comer carne humana y hacerle perder el hábito adquirido lo mejor sería darle a probar otras carnes. Una mañana, pues, lo llevé conmigo a los bosques dispuesto a matar para él una de las cabras que tenía en cautiverio y traer su carne al castillo. Pero en el camino di de pronto con una cabra que descansaba rodeada por dos cabritos. Detuve inmediatamente a Viernes.

    — ¡Quieto! —le ordené, haciéndole seña de que no se moviera.

    Inmediatamente apunté y el tiro alcanzó a uno de los cabritos. Viernes, que me había visto matar a distancia a uno de sus perseguidores, pero seguía sin comprender cómo era posible tal cosa, se puso a temblar y me miraba con un aire tan horrorizado que me pareció que iba a caer desvanecido. Ni siquiera se dio cuenta de que yo había tirado contra un cabrito y que allí yacía muerto, sino que empezó a tantearse la chaqueta como si quisiera descubrir alguna herida. Debió pensar, como me di cuenta en seguida, que intentaba matarlo, porque precipitándose a mis pies me abrazó las piernas, mientras decía gran cantidad de cosas que no entendí, aunque evidentemente me rogaba que no le arrebatase la vida.

    No me costó mucho convencerlo de que no le haría daño alguno, y tomándolo de la mano lo obligué a levantarse y le señalé el cabrito muerto, ordenándole que fuera a buscarlo, lo que hizo enseguida. Mientras él observaba maravillado la forma en que fuera herido el animalito, cargué de nuevo la escopeta y pronto vi un gran pájaro semejante a un halcón posado en un gran árbol cercano. Para darle a entender a Viernes cuál era mi intención le mostré el pájaro —que era en realidad un papagayo y no un halcón— y después señalé la escopeta y el sitio debajo del árbol donde estaba el animal, para que comprendiese que no dejase de mirar el papagayo, e inmediatamente lo vio caer. Se quedó de nuevo como petrificado, a pesar de mis explicaciones, y advertí que como no me había visto cargar otra vez el arma pensaba sin duda que había en ella un inagotable caudal de muerte y destrucción para todo hombre, animal o pájaro que se pusiera a distancia de tiro.

    El pasmo que esto le causaba era tal que transcurrió un tiempo antes de que se recuperara y creo que de haberlo dejado me hubiese adorado tanto a mí como a la escopeta. Durante muchos días no se atrevió a tocar el arma, pero a veces, cuando estaba solo con ella, le hablaba y parecía esperar una respuesta. Más tarde me confesó que le había suplicado con insistencia que no lo matara.

    Luego que se le pasó el primer susto al ver cómo mataba yo al pájaro, obedeció mi orden y fue a recogerlo, pero como el papagayo no estaba más que herido revoloteó para alejarse y Viernes tuvo que correr tras él hasta que al fin pudo alcanzarlo; entretanto yo aproveché su ausencia para cargar otra vez la escopeta, pues había advertido cómo lo asombraba este detalle del arma y la guardaba lista para un nuevo disparo si se presentaba la ocasión. No la hubo, sin embargo, y volvimos trayendo el cabrito, que desollé aquella misma tarde. En una de las ollas herví y guisé una cantidad de carne, obteniendo un excelente caldo. Luego de haberlo probado le di una porción a mi criado, que pareció gustar muchísimo de él. Lo que sin embargo lo maravillaba era verme comer la carne con sal, e hizo señas de que la sal no era sabrosa, y poniéndose un poco en la boca pareció sentir una viva repugnancia, escupiéndola en seguida y yendo a enjuagarse la boca con agua fresca. A mi vez me llevé a la boca un trozo de carne sin sal haciendo toda clase de demostraciones de repugnancia, para convencerlo de que así no debía comerse; pero no obtuve resultado alguno, y Viernes siguió comiendo su carne y bebiendo el caldo sin sal; más tarde empezó a salar su comida, pero apenas.

    Habiéndolo alimentado con el caldo y la carne hervida, me propuse ofrecerle al día siguiente un cuarto de cabrito asado. A tal fin colgué el trozo de una cuerda, tal como había visto hacerlo a mucha gente en Inglaterra; fijando dos estacas a ambos lados del fuego y un palo atravesado sobre ellas, sujeté la cuerda en este último cuidando de hacer girar continuamente el trozo de carne. Viernes admiró mucho estos preparativos, pero aún más maravillado se mostró al probar la carne, empleando tales gestos y ademanes para indicarme cuánto le había gustado que hubiese sido imposible no advertirlo. Por fin me dio a entender que jamás volvería a comer carne humana, lo que me produjo una gran alegría.

    Al otro día lo puse a trillar grano, así como a cernirlo de la manera que ya he contado; pronto aprendió a hacerlo tan bien como yo, especialmente cuando hubo advertido cuál era el objeto de ese trabajo, es decir, la obtención de pan. Le mostré cómo se preparaba y se cocía el pan, y en poco tiempo Viernes fue tan hábil en efectuar aquellos trabajos como pudiera haberlo sido yo mismo.

    Había que tener ahora en cuenta que éramos dos bocas para alimentar en vez de una, de modo que urgía preparar más tierras y sembrar mayor cantidad de grano que hasta entonces. Luego de elegir una superficie conveniente me di a la tarea de hacer un vallado igual al anterior, y Viernes no solamente me ayudó con habilidad y tesón sino que parecía mostrar verdadero entusiasmo. Le di a entender para qué trabajábamos, que ahora era necesaria una mayor cosecha a fin de disponer de más pan, ya que ambos teníamos que alimentarnos.

    Comprendió con suma inteligencia mi razonamiento, y me significó que él se daba clara cuenta de que mis tareas aumentaban mucho con su presencia, pero que estaba dispuesto a trabajar con todas sus fuerzas si yo le enseñaba el modo de hacerlo.

    Aquél fue el más agradable año de todos los que viví en la isla. Viernes empezaba a hablar bastante bien, entendía los nombres de casi todas las cosas que yo podía pedirle y de los lugares adonde lo enviaba. Como hablábamos mucho, volví a tener ocasión de emplear el idioma que durante tanto tiempo me había sido inútil, por lo menos para conversar. Fuera del gusto que me daban estas charlas, me sentía cada vez más atraído hacia el muchacho; su sencilla y franca manera de ser se me revelaba cada día con más claridad, y llegué a quererlo profundamente. Pienso también que él sentía por mí un cariño que jamás había experimentado en su vida.

    Una vez se me ocurrió comprobar si Viernes guardaba alguna nostalgia de su país. Como le había enseñado bastante inglés para que pudiese contestar casi todas mis preguntas, le interrogué sobre si su nación era capaz de triunfar en las guerras. A esto se sonrió y me dijo:

    —Sí, sí, nosotros siempre en pelea mejores.

    Quería significar que combatían mejor que los otros pueblos. Entonces mantuvimos el siguiente diálogo:

    — ¿Así que vosotros peleáis mejor? —dije yo—. ¿Y cómo es que te tomaron prisionero, Viernes?

    VIERNES: Mi nación vencer muchos por eso.

    AMO: ¿Cómo vencer? Si tu nación venció, ¿cómo es que fuisteis apresados?

    VIERNES: Ellos más que mi nación en sitio donde yo estar; mi nación apresar uno, dos, muchos mil.

    AMO: ¿Y por qué, entonces, los de tu nación no acudieron a rescataros de las manos del enemigo?

    VIERNES: Ellos meter uno, dos, tres y yo en canoa; mi nación no tener entonces canoa.

    AMO: Dime, Viernes, ¿qué hacen los de tu nación con los enemigos que toman prisioneros? ¿Se los llevan para comerlos, como tus enemigos?

    VIERNES: Sí, mi nación también comer hombres; comerlos todos.

    AMO: ¿Adonde los llevan?

    VIERNES: Otros sitios, donde gustarles.

    AMO: ¿Vienen a esta isla?

    VIERNES: Sí, sí, venir aquí; venir muchas partes.

    AMO: ¿Viniste aquí con ellos alguna vez?

    VIERNES: Sí, yo estar allí. (Y señalaba el lado noroeste de la isla que, al parecer, era su sitio preferido).

    A través de este diálogo descubrí que mi criado había formado anteriormente parte de las partidas de salvajes que desembarcaban en el extremo de la isla, haciendo con otros lo mismo que ahora habían pretendido hacer con él. Más adelante, cuando tuve ánimo suficiente para llevarlo conmigo a aquel lugar que ya he descrito, conoció inmediatamente el sitio y me dijo que allí mismo habían devorado en una ocasión veinte hombres, dos mujeres y un niño. No sabía decir «veinte» en inglés, pero se puso a alinear piedras y me suplicó que las contase.

    He narrado este episodio porque explicará lo que sigue, y es que luego de oír a Viernes hablar de su nación, le pregunté a qué distancia quedaba nuestra isla de aquellas costas y si las canoas no se perdían con frecuencia. Me respondió que no había peligro y que jamás se perdían las canoas, ya que apenas salidas mar afuera encontraban siempre una misma corriente y un mismo viento, en una dirección por la mañana y en otra por la tarde.

    Comprendí que se refería simplemente a las mareas alternas, pero más tarde vine a descubrir que se trataba de los grandes movimientos y el reflujo del enorme río Oroonoko,1 ya que como terminé por saber nuestra isla se hallaba en el gran golfo de su desembocadura. La tierra que se alcanzaba a ver hacia el O y el NO era la gran isla Trinidad, en la parte norte de las bocas del río. Hice mil preguntas a Viernes sobre el país, los habitantes, el mar, la costa, y cómo se llamaban las naciones próximas. Con toda buena voluntad me dijo cuanto sabía. Le pregunté los nombres de las distintas tribus de su raza, pero sólo supo responder: «Caríb». Me fue fácil deducir que se trataba de los caribes, que nuestros mapas colocan en la parte de América que va desde las bocas del Oroonoko hasta la Guayana y Santa Marta. También me dijo que mucho más allá de la luna —queriendo significar el poniente de la luna, hacia el oeste de su nación— vivían hombres de barba blanca como yo (y señalaba mis bigotes y patillas ya mencionados). Agregó que los blancos habían matado «mucho hombre», según sus palabras, por lo cual comprendí que se refería a los españoles, cuyas crueldades en América se han difundido en el mundo entero al punto de ser recordadas y transmitidas de padres a hijos en cada nación.

    Pregunté a Viernes si podía decirme el modo de salir de la isla y llegar al país de los hombres blancos.

    —Sí, sí —replicó—. Poder ir en dos canoas.

    No supe qué quería significar ni conseguí que me describiera su pensamiento, hasta que al fin y con gran dificultad vine a darme cuenta de que al decir «dos canoas» quería indicar un bote que tuviese un tamaño equivalente al doble de una piragua.

    Estas afirmaciones de Viernes me agradaron mucho, y desde entonces volví a abrigar la esperanza de que alguna vez hallaría oportunidad de fugarme de la isla, y que aquel pobre salvaje sería para mí una valiosa ayuda.

    En el ya largo tiempo que Viernes llevaba a mi lado, cuando fue capaz de hablar y comprender lo suficiente, no descuidé de sembrar en su alma los fundamentos del conocimiento religioso. En una ocasión le pregunté quién lo había creado, pero el pobre muchacho no fue capaz de comprender el sentido de mi pregunta, de modo que busqué otra manera y le pregunté quién había creado el mar, la tierra sobre la cual andábamos, las colinas y los bosques. Me dijo que el creador era el anciano Benamuki, que vivía más allá de todo. Era incapaz de decirme nada acerca de él, sino que Benamuki era viejo, mucho más viejo que el mar y la tierra, que la luna y las estrellas. Le pregunté por qué si aquel anciano era el creador de todas las cosas, no era adorado por ellas. Me miró gravemente y luego, con una absoluta inocencia, dijo:

    —Todas las cosas dicen « ¡Oh!» a Benamuki.

    Lo interrogué sobre si los hombres que morían en su nación iban a alguna parte.

    —Sí —me contestó—. Ellos ir a Benamuki.

    — ¿También los que son devorados?

    —Sí —dijo Viernes.

    Partiendo de esas conversaciones principié a instruirlo en el conocimiento del verdadero Dios. Le dije que el Hacedor de todas las cosas vivía en lo alto, y le señalé el cielo; que gobierna el mundo con el mismo poder y providencia de que se valió para crearlo; que era omnipotente, pudiendo hacer todo por nosotros, darnos o quitarnos todo; y así gradualmente fui iluminando su inteligencia. Escuchaba con gran atención, y aceptó con placer la idea de que Jesucristo había sido enviado a la Tierra para redimirnos, así como la forma de rogar a Dios y la seguridad de que El escuchaba las plegarias desde el cielo. Un día me dijo Viernes que si nuestro Dios podía escucharnos desde más allá del sol era necesariamente un dios más grande que Benamuki, que apenas vivía algo más lejos de la tierra y solamente escuchaba a los hombres cuando subían a lo alto de las montañas donde moraba para invocarlo. Le pregunté si alguna vez había subido a hablarle; me dijo que no, que los jóvenes jamás lo hacían sino que era privilegio de los ancianos a quienes llamaban Uwokaki, queriendo significar, según me explicó, los sacerdotes o ascetas; aquellos eran los que subían a decir « ¡Oh!» —evidentemente, a elevar sus plegarias— y a su regreso manifestaban la voluntad de Benamuki. De ahí deduje que aun entre los más ciegos, ignorantes y paganos habitantes del mundo existe la superchería, y que la astucia de crear una religión secreta a fin de mantener la veneración popular se practica acaso en todas las religiones del mundo, incluso las de los más embrutecidos y bárbaros salvajes.

    Hice lo posible por explicarle a Viernes ese fraude, y le dije que la artimaña de los ancianos al subir a las montañas para decir « ¡Oh!» al dios Benamuki era un engaño, y mucho más su pretensión de ser los portadores de mensajes divinos. Que si alguna palabra recibían en lo alto era proveniente del espíritu del mal, y de ahí nos internamos en una larga conversación sobre el diablo, su origen, su rebelión contra Dios, el odio que le profesa y las causas del mismo, su residencia en los lugares más sombríos de la tierra para que allí se lo adore como si fuese Dios, y las muchas estratagemas de que es capaz para precipitar en la ruina a la humanidad.

    Le mostré cómo el diablo tiene secreto acceso a nuestras pasiones y nuestros afectos, y la astucia con que tiende sus trampas aprovechando nuestras inclinaciones a fin de que nosotros mismos nos tentemos y nos hundamos voluntariamente en la destrucción.

    Le decía yo cómo el diablo es el enemigo de Dios en el corazón de los hombres y emplea allí toda su malicia y su destreza para impedir los buenos designios de la Providencia, a fin de ocasionar la ruina del reino de Cristo, cuando Viernes me interrumpió.

    —Bueno —me dijo—. ¿Vos decir Dios tan grande, tan fuerte, mucho más que diablo?

    —Sí, sí —afirmé yo—. Dios es más fuerte que el diablo, Viernes. Dios está por encima del diablo, por eso rogamos a Dios que nos permita pisotear al diablo, resistir sus tentaciones y apagar el fuego de sus dardos.

    —Pero —declaró él— si Dios más fuerte, si Dios más poderoso que diablo, ¿por qué no matar Dios al diablo, y éste así no hacer más daño?

    Aquella pregunta me sorprendió grandemente, pues aunque en aquel entonces era yo hombre maduro, mi capacidad teológica no excedía a la de un novicio y de ninguna manera podía dármelas de casuista para solucionar tales dificultades.

    Me encontré sin saber qué contestar, y fingiendo que no le había entendido le pedí que me repitiera su pregunta. Demasiado inteligente era para haber olvidado su duda, y me la repitió con las mismas pintorescas palabras. Ya entonces había recobrado un poco la serenidad, y le contesté:

    —Dios lo castigará severamente al fin; ha sido reservado para el juicio final, y será precipitado en los abismos sin fondo, donde lo consumirá un fuego eterno.

    Esto no satisfizo a Viernes, sino que volvió a la carga empleando mis propias palabras:

    — ¡Reservado al fin! Yo no entender. ¿Por qué no matar ya diablo? ¿Por qué no desde antes?

    —Lo mismo podrías preguntarme —le dije— por qué Dios no nos mata a nosotros cuando cometemos pecados que lo ofenden. El nos reserva la oportunidad de arrepentirnos y ser perdonados.

    Meditó un rato esta observación, y entonces me dijo de pronto y muy emocionado:

    —Bien, bien, eso muy bien; entonces vos, yo, diablo, todos malos, todos ser reservados, arrepentirse, Dios perdonar a todos.

    Interrumpí entonces el diálogo, levantándome bruscamente como si me llamara alguna tarea urgente; y luego de haber enviado lejos a Viernes elevé mis plegarias a Dios para que me hiciera capaz de instruir convenientemente a aquel pobre salvaje, y que mi enseñanza de la Palabra de Dios fuera tal que su conciencia se abriera a ella, sus ojos vieran la luz y se salvara su alma. Cuando volvió Viernes, le di una extensa explicación acerca de cómo habían sido redimidos los hombres por el Salvador del mundo, y le enseñé la doctrina del Evangelio dictada por el mismo Cielo, insistiendo en la noción del arrepentimiento y en la fe hacia nuestro bendito señor Jesucristo. Le expliqué después lo mejor posible por qué el santo Redentor no había adoptado la naturaleza y forma de los ángeles sino la estirpe de Abraham; y cómo, por eso, los ángeles caídos no participaban de la redención; en fin, le narré que El había venido solamente a salvar la oveja descarriada de la casa de Israel y así las restantes nociones religiosas.

    Continuando del mismo modo en todo momento libre, las conversaciones que mantuvimos Viernes y yo fueron tales que aquéllos tres años que vivimos juntos en la isla me parecieron absolutamente felices y venturosos, como si en verdad fuera posible la dicha total en algún sitio sublunar. Ya entonces el salvaje era un excelente cristiano, mejor por cierto que yo; tengo razón para creer y esperar, Dios sea bendito por ello, que ambos estábamos arrepentidos y que el consuelo divino nos había alcanzado ya. Con nosotros estaba la Palabra de Dios que podíamos leer, y no nos sentíamos más lejos de la ayuda de su gracia que si hubiésemos vivido en Inglaterra.

    Todas las disputas, riñas, debates y cuestiones que la religión ha suscitado en el mundo, ya por discrepancias sutiles de doctrina o cismas en el gobierno de la Iglesia, nos eran totalmente ajenos, así como a mi entender lo han sido para el resto del mundo. Poseíamos la más segura guía del Cielo, es decir, la Palabra de Dios, claras nociones del Espíritu Divino que Su Palabra nos enseñaba, conduciéndonos seguramente hacia la verdad e inculcándonos la voluntad y obediencia a sus dictados. No alcanzo a entender la utilidad que hubiese podido darnos el más profundo conocimiento de los puntos discutibles de la religión, por los cuales tantas confusiones acontecen en la Tierra. Pero debo ya proseguir con el relato de nuestra existencia y ordenar sus distintos episodios.






	\chapter{Batalla con los caníbales}





    Después que Viernes y yo hubimos intimado, y que él fue capaz de entender casi todo lo que le decía así como hablarme en un inglés chapurreado, empecé a hacerle saber mi historia, por lo menos la parte referente a mi existencia en la isla. Le conté cómo y cuánto había podido vivir allí, lo introduje en los misterios —pues tales eran para él— de la pólvora y las balas, y hasta le enseñé a tirar. Le regalé un cuchillo, lo que le causó una inmensa alegría, y le hice un cinturón con una presilla como los que empleamos en Inglaterra para colgar machetes, dándole una hachuela para que la llevase allí, ya que no sólo era un arma excelente sino que servía muy bien para diversos usos.

    Hice a Viernes una descripción de Europa, y en especial de mi patria, Inglaterra; cómo vivimos, adoramos a Dios, nos conducimos en nuestra vida social y comerciamos en todos los mares del mundo. Le describí el naufragio por cuya causa arribara a la isla, y traté de indicarle con precisión el sitio donde estaba el casco; pero tanto se había roto la estructura del barco que nada quedaba a la vista.

    Entonces señalé a Viernes los restos del bote que había naufragado mientras estábamos a su bordo y que en vano había tratado yo de mover del sitio en que encallara; ahora aparecía destruido y deshecho. Al verlo, Viernes se quedó silencioso y permaneció largo rato pensativo. Le pregunté en qué estaba meditando, y por fin me dijo:

    —Yo ver bote igual ese llegar a mi nación.

    Al principio no le entendí, pero después de interrogarlo mucho supe que un bote semejante al mío había arribado a las costas caribes; de acuerdo con sus referencias, la violencia de un temporal lo precipitó a tierra. Imaginé de inmediato que algún barco europeo se habría estrellado cerca y que el bote, soltándose, había llegado solo y vacío a la costa; pero tan ocupado estaba en estos pensamientos que no cruzó por mi mente la idea de que algunos tripulantes podían haberse salvado de la catástrofe, y menos aún su procedencia, de manera que sólo pedí a Viernes detalles sobre el bote.

    Lo describió lo mejor posible, pero pronto me llamó a la realidad al decirme con bastante entusiasmo:

    —Nosotros salvar hombres blancos de ahogarse.

    — ¿Entonces había hombres blancos en el bote? —me apresuré a preguntar.

    —Sí, bote lleno hombres blancos.

    Le pregunté cuántos, y contó hasta diecisiete con sus dedos. ¿Qué había sido de ellos?

    —Vivir —contestó—. Vivir en mi nación.

    Esto me sumió en nuevos pensamientos, a la sola idea de que aquellos hombres pudieran provenir del navío que había encallado a poca distancia de mi isla. Tal vez, después de comprobar la destrucción del navío en los arrecifes y comprendiendo que estaban perdidos si no se alejaban del lugar del naufragio, habían embarcado en el bote yendo a dar a aquellas salvajes tierras.

    Con mayor detalle interrogué a Viernes sobre el destino de aquellos hombres. Me repitió que vivían allí, llevando cuatro años de residencia, que los salvajes los dejaban solos y les daban vituallas. Le pregunté qué razón había para no haberlos matado y comido.

    —No —repuso Viernes—. Ellos hacer hermanos.

    Supuse que significaba alguna tregua o alianza, ya que agregó:

    —Solamente comer hombres cuando luchar en guerra.

    Comprendí entonces que la costumbre caribe era la de devorar solamente a los prisioneros de las batallas.

    Mucho tiempo después de esto nos hallábamos un día en la cumbre de la colina situada en el lado este de la isla, justamente donde, como ya he dicho, había descubierto en un día claro el perfil del continente americano. El tiempo estaba muy despejado y Viernes se había puesto a mirar intensamente en aquella dirección, cuando de pronto empezó a saltar y bailar como en un rapto de entusiasmo, llamándome a gritos y haciendo que acudiera a preguntarle qué le pasaba.

    — ¡Oh, alegría! —gritaba Viernes—. ¡Contento! ¡Allí ver mi país, ver mi nación!

    Una expresión de intenso júbilo se pintaba en su fisonomía; le chispeaban los ojos y en su aspecto se traslucía la vehemente ansiedad de retornar alguna vez a su país. Tan extraña exaltación me llenó de preocupaciones, haciéndome perder la confianza que hasta ahora había sentido por Viernes; me pareció que si conseguía volverse a su pueblo no sólo olvidaría cuanto le había enseñado de religión sino también los deberes que tenía para conmigo. Probablemente hiciera a sus compatriotas una relación de mi persona y tal vez volviese con cien o doscientos de ellos para devorarme, de lo cual se sentiría tan satisfecho como cuando mataban a los enemigos apresados en la batalla.

    Al pensar así cometía una injusticia con aquel pobre muchacho, y más adelante me arrepentí de ello; pero como este estado de ánimo me dominó durante algunas semanas, me mostré más circunspecto hacia Viernes, sin manifestarme tan familiar y amable hacia él como antes. Mi error fue lamentable, porque aquel agradecido y sencillo salvaje no tenía ya entonces otros pensamientos que aquellos nacidos de los excelentes principios cristianos y de su gratitud amistosa, como más tarde pude comprobarlo a mi entera satisfacción.

    Mientras duró mi desconfianza es de imaginar que constantemente lo sondeaba para tratar de descubrir alguno de los sentimientos que imaginaba habían cobrado cuerpo en él; pero como lo que decía o manifestaba era tan honesto y candoroso, tales sospechas cedieron por falta de razones, y a pesar de mi secreta duda terminé por confiarme otra vez por completo a Viernes; en cuanto a él, en ningún momento se dio cuenta de mi estado de ánimo y por lo tanto no podía sospecharlo de falsedad.

    Paseando un día por la misma colina, pero con tiempo tan nublado que no se divisaba el continente, pregunté a Viernes:

    —Dime, ¿no te gustaría volver a tu país, a tu nación?

    —Sí —repuso él—. Yo muy contento, ¡oh mucho!, volver mi nación.

    — ¿Que harías allí? —insistí—. ¿Te volverías salvaje de nuevo, comerías carne humana lo mismo que cuando te encontré?

    Viernes me miró con aire grave y movió negativamente la cabeza.

    —No, no. Viernes decirles a ellos vivir bien; decirles rogar a Dios, decirles comer pan, carne de cabra, leche, no comer más hombre.

    —Pero entonces te matarán a ti.

    Se puso aún más grave.

    —No —dijo luego—, ellos no matarme, ellos aprender gustando.

    Quería decir que aprenderían gustosos. Agregó en seguida que ya habían aprendido muchas cosas de los hombres llegados en el bote. Le pregunté si le gustaría regresar allá y lo vi sonreírse al responder que no era capaz de nadar hasta tan lejos. Cuando le propuse construirle una canoa para que se volviera me contestó que lo haría si yo lo acompañaba.

    — ¿Yo? —dije—. ¡Me devorarán apenas llegue a tu país!

    —No, no —insistió Viernes—. Yo hacer ellos no coman vos, yo hacer ellos amaros mucho.

    Evidentemente se proponía narrarles cómo había matado a sus enemigos para salvarle la vida, y contaba ganar con eso el afecto de su pueblo. Inmediatamente se puso a explicarme lo bondadosos que eran con aquellos diecisiete hombres blancos (u hombres barbudos, como él les llamaba) que habían llegado indefensos a la costa.

    Desde ese momento confieso que sentí el impulso de aventurarme en el mar y ver si era posible dar con esos hombres que, a mi juicio, debían ser españoles o portugueses. No dudaba de que en su compañía sería posible intentar una fuga de aquellas regiones continentales, cosa más simple que salir sin ayuda y completamente solo de una isla situada por lo menos a cuarenta millas de tierra firme. Días más tarde volví a sondear el ánimo de Viernes y le dije que le daría un bote para que pudiese volver a su nación. Llevándolo hasta el otro lado de la isla, donde fondeaba mi canoa, la saqué del agua, ya que habitualmente la tenía sumergida, y luego de achicarla se la mostré y nos embarcamos en ella.

    Vi en seguida que era muy diestro en la maniobra, capaz de pilotear el bote con la misma rapidez y habilidad que yo. Mientras estaba a bordo le dije:

    —Bueno, Viernes, ¿nos vamos a tu nación?

    Noté que la pregunta le causaba un efecto desagradable, probablemente porque el bote le parecía demasiado pequeño. Le dije entonces que tenía otro más grande, y al día siguiente lo llevé al lugar donde construyera la chalupa y fracasara luego de mi tentativa de botarla. Viernes afirmó que era suficientemente grande, pero yo vi que el abandono en que la chalupa había quedado por espacio de veintidós o veintitrés años la había resquebrajado y destruido mucho, tanto que apenas parecía aprovechable. Viernes insistía en que un bote de ese tamaño era el adecuado para el viaje, y que llevaría «muchos bastantes víveres, bebida, pan», según su pintoresco lenguaje.

    Tan ardiente era ya entonces mi deseo de navegar con él hasta el continente, que le propuse construirle una chalupa tan grande como aquélla y darle libertad para que se volviera a su tierra. No contestó una palabra, pero se puso muy pensativo y triste. Le pregunté qué le ocurría, y a esto me contestó con otra pregunta:

    — ¿Por qué amo enojado con Viernes? ¿Qué haber hecho?

    Quise saber qué significaba aquello, y le aseguré que no estaba en modo alguno enfadado con él.

    — ¡No enfadado, no enfadado! —exclamó, repitiendo varias veces la palabra—. ¿Por qué mandar Viernes entonces a su nación?

    —Pero, Viernes, ¿no decías que estabas ansioso de volver con los tuyos?

    —Sí, sí —dijo él—. Los dos allá, no desear Viernes allá, amo acá.

    En suma, que no quería pensar en irse sin mí.

    — ¿Ir yo allá, Viernes? —le dije—. ¿Y qué podría hacer allá, dime?

    Me respondió vivamente:

    —Vos hacer bien mucho, enseñar hombres salvajes ser buenos, amigos; enseñarles conocer Dios, rogar Dios, vivir nueva vida.

    — ¡Ah, Viernes! —exclamé yo—. No sabes lo que dices. Yo soy un pobre ignorante.

    —Sí, sí, vos enseñarme bien, vos enseñar bien ellos.

    —Oh, no, Viernes —repetí—. Vete solo a tu pueblo; déjame aquí viviendo como antes.

    Cuando le dije eso pareció quedarse confuso y aturdido. Luego, corriendo a tomar una de las hachuelas que habitualmente llevaba, la trajo y me la presentó.

    — ¿Qué quieres que haga con ella? —me pregunté.

    —Amo matar a Viernes.

    — ¿Por qué habría de matarte? •

    — ¿Por qué enviar Viernes lejos? —le replicó rápidamente—. Matar Viernes, no enviar lejos.

    Estaba tan profundamente emocionado que le vi lágrimas en los ojos, y entonces tuve la prueba del profundo cariño que me tenía. Tan resuelto estaba que me apresuré a decirle muchas veces que jamás lo alejaría de mi lado si su voluntad era acompañarme en la isla.

    Aquel episodio no solamente sirvió para demostrarme el profundo afecto de Viernes y su voluntad de no separarse de mi lado, sino que la verdadera razón de sus deseos de volver a su pueblo se fundaba en el cariño que le tenía y su esperanza de que yo pudiera hacerle mucho bien. En cuanto a esto, seguro de mis pocas fuerzas, no me sentía inclinado en lo más mínimo a intentarlo; pero mis ansias de libertad se veían reforzadas por la existencia en el continente de aquellos diecisiete hombres blancos. Sin querer perder más tiempo empecé a buscar con ayuda de Viernes un árbol lo bastante grande para construir una piragua o canoa que soportara la travesía. En la isla había árboles en cantidad como para construir una pequeña flota, no ya de piraguas sino de barcos mayores. Con todo, necesitaba dar con un árbol que estuviera lo bastante cerca del agua para que luego, al tratar de botar la canoa, no ocurriera lo mismo que la primera vez.

    Por fin Viernes señaló el indicado; yo había advertido que era más capaz que yo de reconocer las maderas apropiadas, aunque me sería imposible decir aquí cuál era el árbol que derribamos, excepto que se asemejaba bastante al que llamamos fustete, y también al palo de Nicaragua, al que se parecía por el olor y el tono. Viernes pretendía quemar el centro del tronco para darle la forma de canoa, pero le enseñé a hacerlo con ayuda de herramientas, y cuando hubo aprendido trabajó con habilidad extremada. Un mes más tarde terminamos la canoa, que quedó muy bien; usando nuestras hachas, Viernes y yo habíamos cortado la parte exterior dándole la exacta forma de un bote. Todavía nos quedaba tarea, y pasó una quincena hasta que pudimos llevarla al agua, arrastrándola pulgada a pulgada sobre gruesos rodillos; pero cuando estuvo a flote vimos que era capaz de contener a bordo veinte hombres con toda facilidad.

    Ya en el agua, y aunque se trataba de un bote grande, me maravilló observar con qué destreza y rapidez lo gobernaba Viernes, haciéndole variar el rumbo y empleando los remos. Le pregunté entonces si se animaba a que intentáramos el viaje en la canoa.

    —Sí —dijo—. Navegar muy bien, aunque viento fuerte sople.

    Tenía yo un proyecto del que Viernes no estaba enterado, y era construir un mástil y una vela, así como proveer de ancla a nuestra embarcación. Era fácil encontrar un tronco para mástil, pues había muchos hermosos cedros en la isla y elegí uno joven y absolutamente recto que crecía cerca del mar. Ordené a Viernes que lo cortara, y le di todas las instrucciones necesarias. En cuento a la vela, su construcción me preocupaba seriamente; cierto que tenía viejas velas, o más bien cantidad de pedazos; pero aquellos trozos llevaban veintiséis años conmigo y como no los había cuidado mayormente por no imaginar jamás que llegaría a utilizarlos en esta forma, no dudaba que se habrían estropeado.

    En efecto, me bastó examinarlos para comprobar que la mayoría no estaba en condiciones de uso. Con todo hallé dos pedazos que se conservaban fuertes, y me puse con ellos al trabajo. No sin grandes fatigas y tediosas costuras —para las cuales carecía de verdaderas agujas— conseguí por fin hacer una tosca vela triangular como las que en Inglaterra llamamos «espalda de carnero», con un botalón en la base y una corta botavara en lo alto, tal como la llevan habitualmente las chalupas de nuestros navíos. Esa clase de vela me era familiar y sabía cómo manejarla, ya que una igual tenía la chalupa a cuyo bordo escapé de Berbería, como ha sido contado en la primera parte de esta historia.

    Casi dos meses me ocupó la tarea, es decir, fijar el mástil y montar la vela; además, para completar la arboladura, agregué un pequeño estay al mástil, y a él fijé una vela menor, especie de trinquete que ayudaría a tomar el viento. Finalmente, y esto fue lo más importante, puse un timón en la popa de la canoa; aunque pésimo carpintero naval, como me había dado cuenta de la utilidad y hasta la necesidad de dicho gobernalle, hice cuanto pude para que resultara bien y al fin lo conseguí; pero teniendo en cuenta las diversas tentativas que fracasaron sucesivamente, estoy seguro de que sólo el timón me costó más que la embarcación entera.

    Ya todo listo, faltaba adiestrar a Viernes en las maniobras del pilotaje, porque aunque era muy hábil en dirigir una canoa de remo ignoraba completamente lo referente a las velas y el timón. Se asombraba enormemente al verme dirigir el bote con una u otra dirección con el gobernalle, variar la posición de las velas para modificar el rumbo; su admiración no tenía entonces límites. Pronto, sin embargo, aquello se tornó familiar para él y al poco tiempo era un buen marinero, salvo para la brújula, que no conseguí hacerle entender sino imperfectamente. Cierto que en aquellas latitudes había muy pocos días nublados o con niebla, de manera que poca aplicación tenía la brújula cuando las estrellas servían de guía por la noche y la línea de la costa durante el día. En la estación lluviosa, por otra parte, nadie pensaba en navegar y ni siquiera hacer viajes por tierra firme.

    Se iniciaba el vigésimo séptimo año de mi cautiverio en la isla, aunque pienso que los tres últimos, estando en compañía de Viernes, deberían ser puestos fuera de la cuenta, ya que durante ellos mi vida tuvo un carácter muy diferente de la anterior. Celebré el aniversario de mi arribo con igual reconocimiento que en otras ocasiones por las bondades de Dios. En verdad que si entonces no me faltaban motivos para mostrarme agradecido, ahora debía estarlo aún más con las nuevas pruebas que tenía de la bondad de la Providencia y las esperanzas que en mí renacían de verme pronto liberado de aquella soledad. Día a día se acentuaba en mí el pensamiento de que mi libertad no tardaría en llegar y que ni siquiera alcanzaría a estar otro año en la isla. Cuidé sin embargo de proseguir mis tareas domésticas tales como plantar, cercar y ocuparme de la casa al igual que antes. Coseché y sequé mis uvas, y atendía como siempre a cada cosa necesaria.

    Vino la estación lluviosa, obligándome a permanecer a cubierto buena parte del tiempo. Fue preciso entonces cuidar de la chalupa, y la llevamos a la ensenada donde, como he contado, llegara con las balsas trayendo el cargamento del barco. Después de vararla en la costa aprovechando la marea alta, hice que Viernes cavara una pequeña rada lo bastante grande para contenerla y que aún flotara en ella; luego, al descender la marea, levantamos un fuerte dique en el extremo de la rada a fin de impedir que el agua volviera hasta allí, y la canoa quedó en seco, libre del mar. Para preservarla de las lluvias la cubrimos con tal cantidad de ramas de árbol que quedó como techada, y dejándola a la espera de noviembre y diciembre, tiempo en el que sería posible intentar la aventura.

    Cuando comenzó a manifestarse el buen tiempo, y como si el deseo de ejecutar mis planes creciera con él, diariamente hacía yo preparativos de viaje; lo primero fue almacenar cantidad suficiente de provisiones, calculando que nos alcanzaran para la travesía. Una semana o quince días más tarde esperaba derribar el dique y poner a flote la embarcación.

    Una mañana me ocupaba en estas tareas, cuando se me ocurrió llamar a Viernes y mandarlo a que fuera a la costa en busca de una tortuga, cosa que hacíamos generalmente una vez por semana para comer su carne y los huevos. No llevaba Viernes mucho tiempo ausente cuando lo vi volver corriendo y saltar el vallado como uno que no toca el suelo con los pies. Antes que hubiera podido hablarle, gritó: — ¡Oh amo, amo! ¡Desgracia! ¡Pena! — ¿Qué te ocurre, Viernes? — ¡Allá, allá! —exclamó—. ¡Una, dos, tres canoas! ¡Una, dos, tres!

    Por su manera de expresarse deduje que eran seis canoas, pero al interrogarlo vi que sólo eran tres.

    —Bueno, Viernes —le dije—, no te asustes.

    Traté de animarlo lo mejor posible, pero me di cuenta de que el pobre muchacho estaba mortalmente aterrado. Parecía convencido de que los salvajes venían exclusivamente en su busca, dispuestos a descuartizarlo y a comérselo; temblaba de tal manera que no sabía qué hacer con él. Traté de conformarlo y le dije que también yo estaba en peligro, ya que si nos capturaban sería igualmente devorado.

    —Por eso, Viernes —agregué—, tenemos que resolvernos a pelear. ¿Sabes tú pelear?

    —Yo tirar —dijo él—, pero ellos venir gran número.

    —Eso no importa, Viernes; nuestras escopetas asustarán a los que no hieran.

    Le pregunté entonces si estaba dispuesto a defenderme come yo a él, y si permanecería a mi lado obedeciendo las órdenes que le diera.

    —Yo morir cuando vos mandar —dijo.

    Busqué entonces ron y le di a beber un buen trago; por fortuna había cuidado tanto el licor que me quedaba todavía mucho. Luego que hubo bebido, le di las dos escopetas que llevábamos siempre, cargadas con munición muy gruesa, casi como balines de pistola. Tomé cuatro mosquetes, cargándolos con dos plomos y cinco balines cada uno. A las dos pistolas les puse un puñado de balines y dando a Viernes su hachuela me colgué a la cintura mi sable desnudo.

    Así pertrechados, tomé el anteojo y ascendí a la cumbre de la colina para observar a los enemigos. Me bastó fijar sobre ellos el anteojo para descubrir que había veintiún salvajes, tres prisioneros y tres canoas, y que su intención allí no era otra que proceder a un banquete triunfal con los cuerpos de sus víctimas. Fiesta bárbara, ciertamente, pero sin nada que la distinguiera de las que se llevaban a cabo habitualmente entre ellos.

    Noté también que no habían desembarcado en el sitio que lo hicieran cuando Viernes pudo escaparse, sino más cerca de mi ensenada, donde la costa era más baja y el espeso bosque llegaba casi hasta el mar. Esto, más el horror que la inhumana costumbre de aquellos monstruos me producía, me llenó de tal indignación que descendí a buscar a Viernes y le anuncié que estaba dispuesto a caer sobre los salvajes y matarlos, por lo cual quería saber si contaba con él. Ya se le había pasado el susto y el ron había estimulado sus ánimos, de modo que parecía bien dispuesto y me repitió que moriría si yo se lo mandaba.

    Sin poder contener la furia repartí entre los dos las armas que ya había cargado. Di a Viernes una pistola para llevar en el cinturón, y tres escopetas que se colgó al hombro; tomé la otra pistola y las escopetas restantes y así nos pusimos en marcha. Llevaba yo una botellita de ron en el bolsillo y di a Viernes un saco con bastante pólvora y balas; le ordené que se mantuviera constantemente a mi lado y que no se moviera o tirara hasta recibir una indicación mía; entretanto era preciso no pronunciar una sola palabra. Tomando por la derecha, describimos un círculo de cerca de una milla a fin de llegar a la ensenada por la parte alta cubierta de bosque, y hallarnos a tiro antes de que pudieran descubrirnos; de acuerdo con lo que me había mostrado mi observación con el anteojo, esto era bastante fácil de llevar a cabo.

    Mientras nos acercábamos sigilosos, mis pensamientos empezaron a perder su primitivo ardor. No es que tuviera miedo al número de salvajes, puesto que sabiéndolos desnudos y casi desarmados me sentía superior a ellos, incluso estando solo. Pero volvía a preguntarme qué razón, qué motivo y, lo que es más importante, qué necesidad podía impulsarme a correr hacia esas gentes y bañar mis manos en su sangre, atacándolos sin que me hubiesen hecho daño alguno o tuvieran intención maligna hacia mí. Me dije que si a Dios le parecía justo, El mismo tomaría la venganza en Sus manos y castigaría en conjunto a aquellas gentes, como a una nación, por sus crímenes nefandos; pero en el Ínterin nada de eso me concernía. Viernes, por su parte, tenía una justificación al atacar, puesto que era enemigo declarado de aquellos salvajes, su pueblo estaba en guerra con el de ellos y era legal que los atacara si podía; pero yo no estaba en las mismas circunstancias.

    Tanto me oprimieron aquellas meditaciones mientras nos acercábamos por el bosque, que por fin resolví apostarme solamente en las cercanías de la playa y observar su bárbaro festín, actuando entonces según creyera que Dios me lo ordenaba; hasta entonces, y mientras no recibiera un impulso que me sirviera de suficiente justificativo, estaba dispuesto a no intervenir en lo que ocurría.

    Así resuelto penetramos en el bosque, y andando con toda la cautela y silencio posibles, Viernes pegado a mis espaldas, llegamos hasta el borde arbolado en la parte más próxima al sitio donde estaban reunidos, y del que sólo nos separaba una franja de bosque. Llamando en voz baja a Viernes, le mostré un gran árbol que formaba justamente la saliente del bosque y le dije que fuese hasta allá a observar lo que estaban naciendo los salvajes. Volvió un momento después diciéndome que desde allí se los veía muy bien, que estaban en torno a la hoguera comiendo la carne de uno de los prisioneros, y que otro yacía en la arena, un poco más lejos, esperando su turno. Pero lo que me encendió el alma de coraje fue enterarme de que aquel prisionero no era un caribe sino uno de los hombres barbudos que, según Viernes me contara, habían llegado a la costa en un bote. Sentí que el horror me dominaba a la sola mención de un hombre blanco en tal estado y yendo hasta el árbol pude divisar, con ayuda del anteojo, que efectivamente se trataba de un semejante mío, tirado en la arena con las manos y los pies atados con cuerdas o juncos, y que indudablemente se trataba de un europeo por las ropas que tenía puestas.

    Vi otro árbol, con un matorral adyacente, a unas cincuenta yardas más cerca de los salvajes que el lugar en que ahora estábamos, y al que era fácil llegar con un pequeño rodeo: allí nos pondríamos a medio tiro de escopeta solamente. Reprimiendo, pues, mi furor, aunque estaba encolerizado hasta el límite, retrocedí unos veinte pasos y luego me deslicé por entre los arbustos, que me ocultaron hasta poder apostarme en aquel árbol. Llegué así a una pequeña eminencia del suelo, desde donde tenía una vista total de la escena a menos de ochenta yardas.

    No había un solo momento que perder, pues diecinueve de aquellos horribles monstruos permanecían unos contra otros rodeando el fuego mientras los dos restantes acababan de levantarse con intención de matar al infeliz cristiano y conducirlo, probablemente ya descuartizado, al fuego. Vi que se inclinaban a desatarle las cuerdas de los pies, y me volví a Viernes.

    —Haz lo que te mande —dije, y cuando él asintió agregué—: Pues bien, imítame en todo lo que me veas hacer, y no vaciles ante nada.

    Puse en tierra uno de los mosquetes y la escopeta, y Viernes repitió mis actitudes; tomando luego el otro mosquete, apunté a los salvajes indicándole que me imitara. Luego, al preguntarle si estaba listo y contestarme él que sí, ordené:

    — ¡Fuego, entonces!

    Viernes había apuntado mucho mejor que yo, pues del lado de su tiro vi caer dos salvajes muertos y tres heridos, mientras que yo alcancé a matar a uno y herir a dos. Es de imaginarse la confusión que reinaba entre ellos. Los que no habían recibido heridas saltaron precipitadamente, pero no sabían hacia dónde huir o qué hacer, ya que ignoraban de dónde les llegaba la muerte. Viernes tenía los ojos puestos en los míos para imitar todos mis movimientos, como se lo ordenara. Tan pronto como hubimos disparado, dejé caer el mosquete y tomé la escopeta, cosa que él repitió al punto. Al mismo tiempo amartillamos y apuntamos las armas.

    — ¿Estas listo, Viernes? —pregunté.

    —Sí —repuso.

    — ¡Fuego, entonces, en nombre de Dios!

    Y por segunda vez descargamos las armas sobre los aterrados salvajes. En esta ocasión, como las escopetas tenían por carga balines pequeños de pistola, solamente cayeron dos enemigos, pero tantos resultaron heridos que los vimos correr enloquecidos, aullando y cubiertos de sangre, la mayoría con múltiples heridas; otros tres fueron cayendo luego, aunque no muertos.

    —Ahora, Viernes —mandé dejando en tierra la pieza y levantando el otro mosquete cargado—, ¡sígueme!

    Con gran valor se levantó para obedecerme, y nos precipitamos fuera del bosque exponiéndonos a la vista de los salvajes. Tan pronto como advertí que me habían descubierto lancé un terrible alarido, mientras Viernes hacía lo mismo, y avanzamos a la carrera —no demasiado rápida por el peso de las armas que llevábamos— en dirección donde yacía la pobre víctima, tendida como he dicho en la arena entre la hoguera y el mar. Los dos carniceros que se disponían a descuartizar al prisionero acababan de abandonarlo con el terror de los disparos, huyendo a toda carrera hacia el mar, donde saltaron a una canoa, seguidos por otros tres. Mandé a Viernes que disparara sobre ellos, y comprendiendo en seguida corrió hasta situarse a unas cuarenta yardas y desde allí descargó el arma sobre los que huían. Pensé que los había matado a todos porque cayeron en montón dentro de la piragua, pero dos de ellos se enderezaron al instante. Con todo había muerto a dos y herido a un tercero, que yacía en el fondo de la canoa como fulminado.

    Mientras Viernes se entendía con ellos, extraje el cuchillo y corté los lazos que ataban a la pobre víctima. Lo ayudé a incorporarse, mientras le preguntaba en portugués quién era. Me contestó en latín: «Christianus», pero estaba tan débil que apenas podía hablar o moverse. Le di a beber un trago de ron que había traído en una botella haciéndole señales que bebiera para reanimarse, y también saqué del bolsillo un trozo de pan, que comió. Al preguntarle a qué nación pertenecía, me contestó.

    —Español.

    Ya un poco recobrado de su postración, me dejó entender con toda suerte de signos y ademanes lo reconocido que me estaba por haberlo salvado.

    —Señor —le dije en el mejor español que recordaba—, luego hablaremos, pero ahora es preciso pelear. Si os quedan fuerzas tened esta pistola y esta espada y ved de emplearlas.

    Las recibió con gratitud y apenas las hubo empuñado cuando pareció que con ellas recobraba todo su vigor, pues se lanzó como una furia sobre los asesinos y en un instante mató a dos a estocadas. La verdad es que aquellos infelices estaban tan espantados con la sorpresa que les habíamos dado y el estampido de las armas que el miedo los tenía como atontados y carecían de inteligencia para escapar o combatir en defensa de la vida. Eso era justamente lo sucedido en la canoa sobre la cual Viernes había disparado; aunque sólo tres de los cinco cayeron por efecto de las heridas, los otros dos lo habían hecho a causa del espanto sufrido.

    Mantuve el mosquete listo, sin dispararlo, queriendo reservar la carga porque había dado mi pistola y el sable al español. Llamé a Viernes, ordenándole que corriera hasta el árbol y trajera aquellas armas que habían quedado allí descargadas. Lo hizo a gran velocidad, y mientras él me escudaba con el mosquete me puse a cargar las armas, gritando a mis compañeros que acudiesen a buscarlas a medida que las necesitaran. Mientras me ocupaba en esto observé que se desarrollaba una terrible lucha entre el español y uno de los salvajes, que lo atacaba con una pesada espada de madera, justamente la misma arma que habrían empleado para descuartizarlo si yo no lo hubiera impedido. El español, que era tan osado y valiente como pueda imaginarse, había luchado sin ceder terreno a pesar de su extrema debilidad, y ya había herido dos veces al salvaje en la cabeza; pero aprovechando su falta de fuerzas el astuto y robusto enemigo acabó por acortar distancias y luego, derribando al español, parecía a punto de arrebatarle mi espada de la mano. Fue entonces cuando el español tuvo la inteligencia de abandonar la espada mientras sacaba de la cintura la pistola que le diera, y disparándole un tiro a quemarropa dejó muerto al salvaje antes de que yo pudiera llegar en su ayuda.

    Viernes, librado a su criterio, se había puesto a perseguir a los restantes sin más arma que su hachuela. Con ella acabó de matar a los tres que primeramente habían caído heridos, luego a todos los que pudo alcanzar. El español vino a mí en busca de un arma y le entregué una escopeta, con la cual logró herir a dos salvajes, pero como no tenía fuerzas para correr en su persecución se refugiaron en el bosque donde fue a buscarlos Viernes y mató a uno. El otro era sin embargo demasiado ágil para él, y, aunque herido, logró zambullirse en el mar y reunirse, nadando rápidamente, a los dos sobrevivientes de la canoa. Esos tres salvajes, más uno herido, que ignoramos si murió o no, fueron los únicos que se salvaron sobre veintiuno. La suerte de los restantes fue la siguiente:



	\begin{tabular}{lr}
Muertos por nuestro primer disparo desde el árbol   &   3\\
Muertos por el segundo disparo   &   2\\
Muertos por Viernes en la canoa  &   2\\
Heridos primero y muertos después por él mismo   &   2\\
Muerto por él mismo en el bosque   &   1\\
Muertos por el español   &   3\\
Muertos a causa de las heridas, o rematados por Viernes  &  4\\
Escapados en la canoa, de los cuales uno herido o muerto  &  4\\
TOTAL   &   21
	\end{tabular}


    Los que se salvaron en la canoa huyeron a toda velocidad para escapar a nuestras balas, y aunque Viernes les hizo dos o tres disparos no creo que alcanzara a ninguno. El muchacho quería que tomásemos una de las canoas y los persiguiéramos, lo que me pareció bien, ya que me inquietaba mucho su fuga por temor a que consiguieran llegar a sus playas y avisaran de lo ocurrido a sus compañeros, quienes podían volver en gran número y terminar por apoderarse de nosotros y devorarnos. Corriendo, pues, a una de las canoas, salté en ella y Viernes hizo lo mismo. Pero grande fue mi sorpresa al encontrar en el fondo de la piragua otra víctima atada de pies y manos, destinada al sacrificio lo mismo que el español y casi muerta de miedo por no darse cuenta clara de lo que ocurría. Tan fuertemente estaba atado el pobre hombre que le había sido imposible enderezarse para mirar por la borda de la canoa, y me dio la impresión de que en realidad le quedaba poca vida.

    Inmediatamente corté los lazos o juncos que le sujetaban los miembros y traté de ayudarlo a incorporarse, pero él no podía ni sostenerse ni hablar, y solamente se quejaba con voz lastimera, creyendo probablemente que lo desataba para asesinarlo y comerlo.

    Cuando Viernes llegó a mi lado le ordené que hablase al salvaje y le dijera que estaba libre. Sacando la botella le hice beber un trago, lo cual, junto a la noticia de que se había salvado, lo reanimó bastante y dióle fuerzas para sentarse en la canoa. Pero cuando Viernes se le acercó para hablarle y le vio la cara, de improviso empezó a abrazarlo, a besarlo, estrechándolo en sus brazos con fuerza y haciendo tales demostraciones de alegría que nadie hubiera podido contener las lágrimas al presenciar tal escena. Reía, lanzaba exclamaciones de entusiasmo, saltaba y bailaba, luego se puso a cantar, llorando de emoción, retorció sus manos, se golpeó la cara y la cabeza, en fin, hizo tales demostraciones y dio tales saltos y gritos que parecía haber perdido enteramente el juicio. Pasó un largo tiempo antes de que consiguiera hacerlo hablar con claridad, pero cuando logré al fin que se calmara un poco, me dijo que aquel salvaje era su padre.

    No es fácil de expresar la emoción que me produjo aquel rapto de júbilo y de afecto filial que acababa de presenciar en el pobre salvaje a la vista de su padre librado de la muerte; ni siquiera puedo describir las formas extravagantes que adoptaba su entusiasmo, porque tan pronto saltaba a la canoa como fuera de ella. Por fin se sentó junto a su padre, y abriéndose la chaqueta apoyó la cabeza del anciano contra su pecho, teniéndolo así más de media hora para ayudarlo a recobrar las fuerzas; luego empezó a frotarle las muñecas y los tobillos, que estaban entumecidos por la fuerza de las ligaduras, y como advertí su intención le alcancé la botella para que frotara con ron los miembros agarrotados, lo que hizo gran bien al salvaje.

    Todo esto nos distrajo de la persecución de la otra canoa, que estaba ya casi pérdida en la distancia. Y fue fortuna para nosotros no embarcarnos tras ella, porque unas dos horas después, cuando la piragua no podía haber navegado más que un cuarto de la travesía, empezó a soplar viento fuerte que continuó toda la noche del noroeste, es decir, contra ellos, de manera que resulta difícil creer que se salvaran de un naufragio o que pudiesen volver a sus tierras.

    Mas retornemos a Viernes. Estaba tan ocupado con su padre que no tuve valor para alejarlo de allí, pero cuando pensé que podía abandonarlo por su instante lo llamé y vino saltando y riendo, con todas las demostraciones de la felicidad más completa. Le pregunté si había dado algo de pan a su padre.

    —No —repuso moviendo la cabeza—. Yo perro ruin comerme todo.

    Le di un pan de los que llevaba conmigo en un saquito, y le ofrecí un trago de ron, pero no quiso probarlo y se lo llevó a su padre. Como tenía en mis bolsillos algunos racimos de pasas, le di un puñado, y lo vi correr con todo eso hasta donde estaba el anciano, y de pronto alejarse de él y escapar como si estuviera alucinado. Jamás, por otra parte, he visto a nadie que tuviera velocidad comparable en la carrera. Con tal rapidez se alejó que en un momento estuvo fuera de mi vista, y aunque lo llamé a gritos fue como si no lo hiciese, pues siguió su marcha. Un cuarto de hora más tarde lo vi regresar, aunque ya no con la misma velocidad, pues parecía cuidar algo que tenía en la mano.

    Cuando pasó a mi lado vi que había ido hasta casa en busca de un jarro de agua dulce, y que traía además otros dos panes. Me entregó el pan y fue a dar el agua a su padre, pero antes bebí yo un trago porque me sentía sediento. El agua hizo más bien al anciano que el ron que yo le diera antes a tomar, porque la sed lo devoraba.

    Luego que hubo bebido llamé a Viernes para saber si quedaba un poco de agua. Como me contestara afirmativamente le ordené que la llevase al pobre español, que también la necesitaba mucho, así como uno de los panes. El español se sentía muy débil y reposaba a la sombra de un gran árbol, tendido en el césped; sus miembros estaban aún paralizados y tenían claras señales de la fuerza con que habían sido atados. Cuando vi que aceptaba el agua que Viernes le ofrecía, así como el pan que comió inmediatamente, me acerqué a él y le di un puñado de pasas. Alzó su mirada hacia mí con la más profunda expresión de reconocimiento que pueda pintarse en un rostro humano, pero estaba tan débil —pese a haber tenido aún fuerzas para combatir— que no le fue posible sostenerse en pie. Trató de hacerlo dos o tres veces, pero sus piernas no le sostenían y los tobillos le dolían mucho. Le dije entonces que se estuviera quieto, mientras Viernes le friccionaba los miembros con ron, tal como lo hiciera con su padre.

    Noté que el pobre y afectuoso muchacho miraba cada dos minutos o menos hacia el sitio donde había dejado a su padre, para saber si seguía allí en la misma actitud en que él lo colocara. De pronto, al no verlo, se enderezó y fue hacia allí con tal velocidad que sus pies apenas tocaban el suelo. Pero como el anciano solamente se había tendido en el suelo para reposar, Viernes volvió a nosotros; dije entonces al español que Viernes lo ayudaría a caminar hasta el bote, a fin de trasladarlo luego a la morada en que podríamos atenderlo convenientemente. Mi criado, que era muy fuerte, levantó al español y cargándolo sobre sus espaldas lo llevó hasta la canoa, donde lo dejó en el borde con mucho cuidado, los pies vueltos hacia el interior; levantándolo luego otra vez, lo hizo entrar del todo y lo puso al lado de su padre. Saltando de la canoa, la empujó para botarla al agua, y aunque el viento soplaba con fuerza la llevó a remo más pronto de lo que yo podía ir por la costa, y corrió por la playa en busca de la otra piragua. Cuando pasó junto a mí le pregunté adonde iba, y me respondió:

    —Buscar más canoa.

    Con la velocidad del viento lo vi alejarse, y por cierto que nunca caballo u hombre corrieron como él. Llegó tripulando la otra canoa casi al mismo tiempo que yo arribaba a la ensenada, y después de pasarme en ella al otro lado se puso a ayudar a nuestros nuevos huéspedes para que salieran de la piragua. Pero cuando estuvieron en tierra, como no les era posible dar un paso, el pobre Viernes no sabía qué hacer.

    Principié a pensar el modo de llevarlos a casa, y luego de ordenar a Viernes que los dejara cómodamente sentados en la playa, entre los dos construimos una especie de angarillas de tal modo que cupieran ambos, y así iniciamos la marcha. El problema se presentó al llegar a la fortificación exterior del castillo, ya que de ningún modo aquellos hombres tenían fuerzas para montar sobre el vallado ni yo estaba dispuesto a romperlo por su causa. Volví, pues, a ponerme a trabajar, y en un par de horas hicimos, Viernes y yo, una confortable tienda, cubierta con pedazos de velas y por encima ramas de árbol; la instalamos en el espacio abierto que había entre la empalizada exterior y el bosque que yo plantara. Pusimos finalmente allí dos camas hechas con el mismo material que teníamos a mano, es decir, paja de arroz, cubiertas con una manta a modo de colchón y otra por encima para abrigo.


    .



	\chapter{Un español y algunos ingleses}





    Mi isla estaba ahora poblada y, de pronto, me encontré rodeado de muchos súbditos; frecuentemente afirmaba yo en broma que de veras parecía un rey. Ante todo, la tierra era de mi absoluta propiedad, lo cual me aseguraba un indiscutible derecho de dominio. Segundo, mi pueblo estaba formado por sumisos vasallos, de los cuales era señor y juez; todos me debían la vida, y estaban dispuestos a entregarla por mí si la ocasión se presentaba. Lo más digno de notarse era que mis tres súbditos pertenecían a religiones distintas. Mi criado Viernes era protestante; su padre, pagano y caníbal, y el español, católico. Dicho sea de paso, yo había asegurado la libertad de conciencia en todos mis dominios.

    Tan pronto como los rescatados prisioneros estuvieron bajo techo y con suficiente abrigo, me puse a pensar en la forma de alimentarlos. Lo primero que hice fue ordenar a Viernes que eligiera del rebaño un cabrito como de un año y lo matara; saqué de él un cuarto trasero que corté en pequeños trozos, que Viernes hirvió y guisó; en esta forma obtuvimos un buen plato de carne y caldo, al que agregamos algo de cebada y arroz. Como cocinaba al exterior porque no quería encender fuego dentro de la empalizada, llevé todo a la nueva tienda y poniendo allí una mesa nos, sentamos a comer en compañía. Traté de animar a los huéspedes y darles coraje. Viernes era mi intérprete ante su padre, e incluso ante el español, ya que éste hablaba muy bien el idioma de los salvajes.

    Luego de haber comido o más bien cenado, ordené a Viernes que fuera en una de las canoas a recoger los mosquetes y demás armas que por falta de tiempo dejáramos tirados en la costa; también le mandé al día siguiente enterrar los cadáveres de los salvajes que yacían bajo el sol y pronto ofenderían el olfato. Le dije que hiciera lo mismo con los numerosos restos del bárbaro festín, ya que jamás habría podido emprender yo una tarea semejante; ni siquiera me animaba a mirar aquello cuando pasaba cerca. Viernes cumplió puntualmente mis indicaciones, y borró de tal modo la huella de la presencia de los caníbales que cuando volví al sitio apenas reconocí el escenario de la lucha, salvo por el extremo del bosque que apuntaba hacia allí.

    Ya entonces empezaba a sostener alguna conversación con mis dos nuevos súbditos. Pedí a Viernes que preguntara a su padre lo que pensaba sobre la fuga de los salvajes sobrevivientes, y si era de esperar que volvieran pronto en número tan grande que tornasen imposible toda defensa. Su primera opinión fue que los salvajes no eran capaces de resistir la violencia del temporal de aquella noche; si no se habían ahogado habrían derivado hacia las costas del sur, donde sin duda otras tribus caníbales los devorarían. En cuanto a lo que pudieran hacer si llegaban sanos y salvos a su tierra, no lo sabía, pero su opinión fue que sin duda estaban tan aterrados por aquel ataque inexplicable, las detonaciones y el fuego, que sin duda dirían a sus compatriotas que no era una mano humana sino el mismo rayo quien había exterminado a los restantes. A sus ojos, Viernes y yo debimos aparecer ante ellos como la encarnación misma de los dioses infernales presentándose para destruirlos sin armas humanas. Agregó que estaba seguro de ello porque tales cosas los oyó gritarse unos a otros en su lenguaje, que entendía bien; ninguno parecía concebir la posibilidad de que un hombre pudiera arrojar fuego, hablar tronando y matar a tal distancia sin siquiera levantar la mano, como había ocurrido. Pronto vimos que el viejo salvaje estaba en lo cierto, porque como llegué a enterarme más tarde por otros testimonios, jamás un caníbal se atrevió a pisar de nuevo aquella isla. Tan aterrados quedaron con el relato hecho por los cuatro sobrevivientes (que, por lo visto, se salvaron de la tempestad) que desde entonces vivieron convencidos de que cualquier osado que desembarcara en la isla sería destruido por el fuego de los dioses.

    Naturalmente yo ignoraba entonces todo aquello, y durante mucho tiempo viví bajo una continua aprensión y sin descuidar las precauciones que nuestro pequeño ejército adoptaba. Ahora éramos cuatro y, sin temor, nos hubiéramos enfrentado en cualquier momento con un centenar de salvajes en campo abierto.

    Como pasara el tiempo y las canoas enemigas no volvían, empecé a perder el miedo y otra vez invadieron mi mente los proyectos de escapar de la isla. Tenía ahora como un nuevo incentivo la garantía formalmente dada por el padre de Viernes asegurándome que si yo iba con ellos hasta su nación sería muy bien tratado bajo su responsabilidad.

    Con todo, mis proyectos se enfriaron algo después de sostener una conversación con el español, por el cual vine a saber que dieciséis de sus connacionales, así como portugueses vivían en paz con los salvajes después de haberse salvado de un naufragio que los arrojó a aquellas costas; pero su existencia era muy penosa, y pasaban inmensas privaciones en las cuales hasta su vida estaba amenazada. Le pedí datos sobre su viaje y supe que los náufragos pertenecían a un barco fletado desde el Río de la Plata con destino a La Habana, donde debían desembarcar su cargamento, compuesto principalmente de pieles y plata, y retornar trayendo manufacturas europeas que pudieran encontrarse en aquel puerto. Me contó que llevaban a bordo cinco marinos portugueses a quienes recogieron en el mar, y que, cuando naufragaron más tarde, otros cinco españoles perecieron ahogados, mientras los que consiguieron salvarse habían llegado después de infinitas aventuras y peligros, casi extenuados de hambre a la costa de los caníbales, donde esperaban de un momento a otro ser devorados.

    Me dijo que tenían algunas armas con ellos, pero que de nada les servían, pues el agua de mar les había arrebatado o estropeado completamente la pólvora, de la que sólo salvaron una pequeña porción que fue empleada para procurarse algún alimento cuando desembarcaron.

    Pregunté al español qué sería de ellos allí, y si no tenían algún proyecto de fuga. Me dijo que muchas veces lo habían pensado, pero como carecían de embarcación y de toda herramienta para construirla, sus conciliábulos terminaban siempre en lágrimas y desesperación.

    Quise saber de qué manera recibirían una propuesta mía que pudiera ayudarlos a escapar, y si él veía alguna posibilidad de fuga desde mi isla una vez que todos consiguiéramos reunimos en ella. Le advertí con toda franqueza que mi mayor temor era el de que me traicionaran una vez que hubiese puesto mi vida en sus manos; sabía de sobra que la gratitud no es una cualidad inherente a la naturaleza humana, y que con frecuencia obran los hombres más de acuerdo a las ventajas que esperan obtener que por los favores que hayan recibido. Le manifesté que me resultaría harto cruel ser instrumento de su liberación para que luego me llevaran prisionero a Nueva España, donde todo inglés está seguro de ser sacrificado cualesquiera sean los motivos que lo hayan conducido allí. Por cierto que prefería mucho más caer en manos de los salvajes y ser devorado vivo por ellos, que en las garras despiadadas de los frailes y la Inquisición. Le participé mi seguridad de que si tantos hombres conseguíamos reunimos sería posible construir una embarcación capaz de llevarnos hacia el sur, es decir, al Brasil, o bien hasta las islas españolas del norte, pero que si después de haberlos ayudado y puesto armas en sus manos me arrastraban por fuerza a su tierra, de nada me habría valido mi generosidad para con ellos y mi situación sería mucho peor que la presente.

    Con una gran sinceridad y franqueza me contestó el español que la condición de sus compañeros era tan triste y de tal modo conocían sus miserias, que de ninguna manera le parecía admisible que llegaran a traicionar a quien les ofrecía la libertad. Agregó que si yo lo autorizaba él iría a ellos en compañía del anciano salvaje y les plantearía la proposición para volver de inmediato con la respuesta. Sólo iba a tratar con ellos previo solemne juramento de que se pondrían incondicionalmente a mis órdenes en mi carácter de jefe y capitán; tal juramento sería realizado sobre los Santos Sacramentos y el Evangelio, comprometiéndose a serme fieles, dirigirse al país cristiano que a mí me pareciera bien y, en una palabra, obedecer total y absolutamente mis órdenes hasta que hubiésemos arribado con felicidad al país que yo designara. Finalmente me prometió que todo aquello sería redactado por escrito para mayor seguridad, que él iba a ser el primero en pronunciar el juramento y que jamás se apartaría de mi lado mientras yo no dispusiera otra cosa. Hasta la última gota de sangre estaba dispuesto a verter por mí si advirtiera la menor señal de mala fe entre sus camaradas.

    De acuerdo con lo que el español me dijo, los náufragos eran hombres honrados y buenos que se encontraban en la peor de las miserias, privados de armas y ropas, casi sin comer y enteramente a merced de los salvajes; no conservaban ninguna esperanza de regresar algún día a su país, de modo que si yo emprendía su liberación se sentirían tan agradecidos como para consagrarme su vida e incluso morir por mí.

    Asegurado de tales cosas, me resolví a tentar la aventura de librarlos si era posible, y enviar ante todo al viejo salvaje y al español con mis propuestas. Todo estaba ya listo para el viaje cuando el español me señaló una objeción, hecha con tal prudencia por un lado y tanta sinceridad por otro que la encontré atinadísima, llevándome a postergar la liberación de sus camaradas por otros seis meses como mínimo.

    He aquí por qué: durante el tiempo que llevaba el español a mi lado, alrededor de un mes, le había mostrado yo los recursos que poseía para alimentarme con ayuda de la Providencia. Había podido observar mis depósitos de cebada y arroz, harto abundantes para mí, pero que debimos economizar ahora que nuestra familia se elevaba a cuatro miembros. Como es natural mucho menos podrían bastarnos cuando los sobrevivientes del naufragio que sumaban dieciséis hombres en ese momento, llegaran a la isla. Todavía menos podían alcanzarnos esas reservas para avituallar el navío que pensábamos construir y que debía llevarnos hasta alguna de las colonias cristianas de América. El español me manifestó que le parecía preferible cultivar y sembrar nuevas tierras, y que yo destinara a ello todo el grano de que pudiera desprenderme; luego esperaríamos otra cosecha, a fin de tener cantidad suficiente para cuando sus compatriotas desembarcaran. La falta de alimentos podía, en caso contrario, ser causa de disgustos o de que aquellos hombres no se sintieran libertados en modo alguno sino simplemente movidos de una dificultad a otra no menos grave.

    —Bien sabéis —agregó— que aunque los hijos de Israel se regocijaron al principio por haber sido salvados del cautiverio en Egipto, acabaron por rebelarse contra el mismo Dios, su salvador, cuando les faltó pan en el desierto.

    Su prevención era tan razonable, su consejo tan excelente, que no podía yo sino sentirme satisfecho de haberlo recibido, así como me satisfacía su fidelidad hacia mí. Nos pusimos pues los cuatro a cavar la tierra, todo lo bien que nuestras herramientas de madera nos lo permitían, y un mes después, en la época propicia, habíamos dispuesto suficiente extensión para sembrar veintidós fanegas de cebada y dieciséis tinajas de arroz, lo que constituía toda la cantidad de que disponíamos para una siembra. Nos quedó apenas grano para alimentarnos durante los seis meses en que debíamos esperar la nueva cosecha, contando a partir del día en que apartamos el grano para sembrarlo, puesto que en aquellas latitudes la recolección se hace antes de ese plazo.

    Como entonces nos sentíamos acompañados mutuamente y nuestro número era bastante para alejar todo temor de los salvajes —salvo que la isla fuera invadida por un enorme ejército —andábamos libremente en cuanto se nos presentaba ocasión. La esperanza de alcanzar alguna vez la libertad nos asaltaba continuamente, por lo menos a mí. A tal efecto busqué varios árboles que me parecieron adecuados, y dije a Viernes y a su padre que los cortaran. En cuanto al español, que estaba al tanto de mis proyectos, se encargó de dirigir la tarea. Les mostré con qué indescriptibles fatigas había logrado yo hacer tablones de un enorme árbol, y los puse a la misma tarea hasta que hubieron obtenido una docena de grandes tablones de buen roble, de dos pies de ancho por treinta y cinco de largo y dos o cuatro pulgadas de espesor. Es de imaginarse el prodigioso trabajo que costó obtenerlos.

    Al mismo tiempo trataba de aumentar en todo lo posible mi rebaño de cabras. Un día enviaba al español con Viernes a cazar, y al siguiente íbamos Viernes y yo, alternándonos en la tarea; pronto capturamos así más de veinte cabritos que pusimos con las demás cabras en los corrales. Cuando encontrábamos cabras salvajes, matábamos una madre y nos apoderábamos inmediatamente de sus pequeños. Pero lo notable fue la cantidad de pasas que obtuvimos poniendo uvas a secar al sol cuando vino la época en que maduraron los racimos; creo que si hubiésemos estado en Alicante, donde se preparan las pasas, hubiéramos podido llenar sesenta u ochenta barriles. Junto con el pan, aquellas pasas eran lo principal de nuestro alimento, y por cierto muy agradable a la vez que de un gran valor alimenticio.

    Llegó el tiempo de la siega y nuestro grano espigó muy bien. No era la cosecha más productiva que viera yo en la isla, pero evidentemente alcanzaba para nuestros fines; las veintidós fanegas de cebada se convirtieron, después de la siega y trilla, en más de doscientas veinte fanegas, y la misma proporción se obtuvo de arroz, todo lo cual nos daba suficiente provisión hasta la siguiente cosecha, contando con nosotros a los dieciséis españoles. Incluso si hubiéramos estado ya listos para viajar, aquel grano hubiese sido bastante para avituallar el navío y alimentarnos en la travesía hasta cualquier punto de la costa americana.

    Concluida la tarea concerniente a la cosecha y almacenamiento del grano, nos pusimos a trabajar en cestería, haciendo grandes canastos donde pudiéramos guardar con seguridad la semilla. El español era muy diestro y habilidoso en dicho trabajo, y con frecuencia me reprendía por no haber empleado esta clase de tejido para mi defensa; pero yo no lo veía necesario.

    Ya dueño de suficiente cantidad de alimentos para todos los huéspedes que esperábamos di mi consentimiento al español para que cruzara el mar y fuese en busca de sus compañeros. Le hice estrictas recomendaciones, mandándole que no trajera consigo ningún hombre que previamente no jurara, en presencia suya y del anciano salvaje, no ofender en modo alguno, luchar o atacar a la persona que encontrara en la isla y que era quien enviaba en busca de ellos; que prometieran con el mismo juramento apoyar y defenderme contra toda tentativa hostil, así como someterse entera y absolutamente a mis órdenes. Todo aquello debía ser consignado por escrito y firmado de puño y letra. Cómo se haría eso, cuando era de imaginarse que carecían en absoluto de pluma y tinta, es cosa que a ninguno se nos ocurrió.

    Provistos de dichas instrucciones, el español y el anciano padre de Viernes partieron en una de las canoas con las cuales arribaran o, mejor dicho, fueron traídos prisioneros por los salvajes.

    A cada uno le entregué un mosquete y ocho cargas de pólvora y balas, encargándoles que las economizaran en lo posible y que sólo hicieran uso de ellas en ocasiones urgentes.

    Todo esto me llenaba de contento, pues eran las primeras medidas tomadas por mí en procura de mi libertad después de veintisiete años y algunos días de mi permanencia en la isla. Di a los viajeros suficiente cantidad de pan y pasas para una travesía de varios días, así como para alimentar a sus compatriotas una semana. Luego de desearles buen viaje los dejé partir, conviniendo antes una señal por la cual pudiera reconocerlos a su vuelta y desde muy lejos.

    Partieron con excelente viento, un día de plenilunio que correspondía, según mi cuenta, al mes de octubre. Mi calendario no era exacto, ya que después de haber perdido un día al comienzo nunca pude encontrar el error; ni siquiera estaba ya seguro del número de años, aunque cuando más tarde pude cotejar mis cálculos con la realidad vi que los años coincidían exactamente.

    No habían pasado ocho días de la partida de la canoa cuando se produjo un extraño e imprevisto acontecimiento que acaso no tenga equivalente en la historia. Dormía yo una mañana en mi tienda cuando Viernes llegó corriendo y gritando:

    — ¡Amo, amo, ellos venir, ellos venir!

    Sin mirar el peligro salté del lecho y tan pronto me hube vestido, atravesé el soto que, dicho sea de paso, era ya entonces un espeso bosque. Sin atender al peligro, repito, corrí desprovisto de armas, lo que nunca hacía, pero mi sorpresa fue grande cuando, al mirar el océano, vi un bote a legua y media de distancia que navegaba en procura de la costa, con una vela de las llamadas «espalda de carnero» y el viento a favor. Observé de inmediato que el bote no venía del lado más abierto de la costa sino de la parte sur de la isla.

    Llamé entonces a Viernes ordenándole que se mantuviera a cubierto, pues aquellas no eran las gentes que esperábamos y aún no podíamos saber si se trataba de amigos o enemigos. Apenas había alcanzado a poner mi pie en la cumbre cuando mis ojos vieron con toda claridad un navío fondeado a unas dos leguas y media de mi apostadero, hacia el S-SE, pero no a más de legua y media de la costa. Me bastó verlo para descubrir que era un barco inglés y el bote una lancha de igual procedencia.

    Es imposible expresar la confusión que experimenté. La alegría que me invadió al ver un barco que por todas las señales era tripulado por compatriotas, es decir, por amigos, no es de las que pueden ser descritas. Con todo, algunas dudas me asaltaban sin que me fuera posible comprender cuál era su motivo, y me mantuve oculto y en guardia. Ante todo se me ocurrió preguntarme qué clase de comercio podía traer a un barco inglés por estas regiones del mundo, ya que la isla estaba alejada de todo lugar donde la marina británica tuviese intercambio alguno. Por otra parte, no había venido arrastrado por una tormenta, y si verdaderamente se trataba de ingleses acaso no arribaran con buenas intenciones a esta tierra, de manera que era preferible continuar como hasta ahora y no caer en manos de probables ladrones y asesinos.

    Llevaba poco tiempo en la colina cuando vi acercarse la chalupa a tierra, costeándola en busca de alguna ensenada que facilitara el desembarco. Como no se alejaron mucho no les fue posible descubrir la pequeña caleta donde yo había llevado mis primeras balsas, sino que arrastraron la chalupa sobre la arena, a una media milla de mi puesto. Esto me alegró, porque de haber seguido la costa habrían terminado por arribar prácticamente a la puerta de mi casa, como podría decirse. Sin duda hubiesen descubierto el castillo y acaso saqueado todos mis bienes.

    Cuando desembarcaron tuve la satisfacción de reconocer en ellos a ingleses, por lo menos la mayor parte, aunque uno o dos me parecieron holandeses, en lo que estaba equivocado. Eran once en total, de los cuales tres se veían desarmados y al parecer atados, porque cuando los cinco o seis primeros pisaron tierra los hicieron salir como si se tratara de prisioneros. De los tres, uno parecía verdaderamente desesperado y mostraba tales señales de aflicción y de angustia que llegaba por momentos a la extravagancia; los otros dos, aunque a veces alzaban las manos al cielo como si estuvieran profundamente doloridos, se mostraban mucho más tranquilos que el primero.

    Aquel espectáculo aumentó mi confusión, sin que me fuera posible adivinar lo que iba a ocurrir. Entonces vino Viernes a hablarme en su chapurreado lenguaje.

    — ¡Oh, amo! ¡Ved hombres ingleses comer prisioneros igual salvajes comer hombres!

    — ¿Es que piensas, Viernes —repliqué yo—, que van a devorarlos?

    —Sí —insistió él—, ellos comer hombres.

    —Te equivocas —le dije—. Temo mucho que los asesinen, pero ya verás que no los comen.

    Entretanto seguía en la más grande ignorancia sobre lo que verdaderamente acontecía en la costa, y temblaba de horror a la idea de que aquellos tres prisioneros fueran asesinados de un momento a otro. Incluso vi a uno de los malvados alzar un gran machete o una espada para matar a los desdichados; me pareció que alguno iba a caer a cada instante, y mi sangre se heló en las venas.

    ¡Cuánto hubiera dado por tener ahora conmigo al español y al padre de Viernes! Ansiaba encontrar un medio de ponerme a tiro de aquellos individuos sin ser descubierto, a fin de rescatar a los tres infelices; había observado que no traían armas de fuego, pero pronto vi que las cosas cambiaban. Después de la insolente conducta del marinero hacia sus víctimas, todos ellos se dispersaron por la costa, como queriendo reconocer la tierra. Los tres prisioneros quedaron igualmente en libertad, pero se dejaron caer en tierra pensativos y con todo el aspecto de la desesperación más profunda.

    Esto me recordó el momento de mi llegada a la isla, cuando principié a mirar en torno mío sintiéndome perdido y sin esperanzas; me acordaba de la aprensión y el miedo con que había reconocido las inmediaciones, y cómo pasé la primera noche en un árbol por miedo a que me devoraran animales salvajes.

    Así como entonces yo ignoraba el socorro que la Providencia iba a enviarme al arrastrar el buque cerca de la costa y concederme extraer de él todo lo que me permitió alimentarme y vivir en adelante, así también aquellos pobres hombres angustiados ignoraban lo cerca de ellos que estaba la salvación y cómo en realidad podían escapar al peligro en el mismo instante en que se imaginaban abandonados a la muerte.

    Los tripulantes habían arribado a la costa justamente en el momento en que culminaba la pleamar, y en el tiempo que emplearon con los prisioneros y en recorrer más tarde las inmediaciones descuidaron tanto la chalupa que al producirse el reflujo quedó varada en tierra.

    Había, sin embargo, dos hombres a bordo, pero como sin duda habían bebido demasiado aguardiente estaban dormidos y ajenos a lo que pasaba. Uno de ellos despertó de improviso, y viendo que sus fuerzas no bastaban para empujar la chalupa al agua llamó a los demás que andaban sin rumbo fijo, y pronto se reunieron para moverla. No obstante, la tarea era superior a sus fuerzas, pues se trataba de una embarcación grande y la playa, en esa parte, tenía una arena suave y cenagosa; siendo casi arena movediza.

    Viéndose en tal situación, y como verdaderos marinos —que tal vez de todos los hombres sean los menos previsores— abandonaron la tarea y se pusieron otra vez a vagabundear. Oí que uno de ellos gritaba a otro que todavía permanecía junto a la chalupa:

    — ¡Eh, Jack, déjala quieta! ¡Ya flotará con la marea! Me bastó escuchar eso para comprobar cuál era su nacionalidad.

    Durante todo este tiempo me había mantenido muy oculto, sin animarme a salir del castillo más allá de mi puesto de observación en la cumbre de la colina, y sintiéndome harto satisfecho por las sólidas fortificaciones.

    Me dispuse entretanto para una posible batalla, aunque con mayor cuidado, pues tenía que habérmelas con otra clase de enemigos. Viernes era ya entonces un excelente tirador, y le ordené que se equipara convenientemente. Tomé dos escopetas para mí y puse tres mosquetes en sus manos. Por cierto que mi aspecto debía ser impresionante: tenía mi formidable chaqueta de piel de cabra y el gran gorro ya mencionado, una espada desnuda en la cintura, dos pistolas en el cinturón y una escopeta sobre cada hombro.

    Mi intención era no hacer ningún movimiento hasta que oscureciera; pero a eso de las dos de la tarde, a la hora de mayor calor, descubrí que todos se habían internado en los bosques y probablemente dormían tirados en el suelo. Los tres infelices prisioneros, demasiado ansiosos y apenados para encontrar descanso alguno, habían buscado la sombra de un gran árbol que se alzaba a un cuarto de milla de mi apostadero y, según me pareció, lejos de las miradas de los otros marinos.

    Resolví entonces mostrarme a ellos y averiguar qué les pasaba. Con la apariencia ya descrita me encaminé hacia el árbol seguido a cierta distancia por Viernes, a quien sus armas también hacían formidable, pero que no tenía el aspecto fantasmagórico de mi persona. Acercándome todo lo posible sin ser descubierto, les dirigí de pronto la palabra en español:

    — ¿Quiénes sois, caballeros?

    Se levantaron al oírme, pero su espanto creció diez veces al verme. No solamente no me contestaron sino que advertí su intención de escapar a la carrera.

    —Caballeros —les dije entonces en inglés—. No os espantéis de mí; acaso sea para vosotros el amigo que sin duda no esperabais.

    —Ese amigo —me contestó entonces uno de ellos gravemente, a tiempo que se quitaba el sombrero— debe haber sido enviado por el Cielo, porque en verdad nuestra situación está por encima del auxilio humano.

    —Todo auxilio viene del Cielo, señor —repliqué—. Os ruego que expliquéis a un extraño lo que os ocurre a fin de que pueda intentar ayudaros, pues me dais la impresión de hallaros en un gran apuro. Os he visto desembarcar, y mientras observaba que dirigíais súplicas a aquellos malvados que os han traído, vi a uno de ellos alzar su espada como para mataros.

    El desdichado, que me escuchaba con lágrimas en los ojos, se puso a temblar como alguien que no vuelve de su sorpresa.

    — ¿Estoy hablando con Dios o con un hombre? —dijo—. ¿Sois un ser humano o un ángel?

    —Desechad todo cuidado, caballero —le contesté—. Si Dios os hubiera enviado un ángel para salvaros, sin duda estaría mejor vestido que yo y con un armamento muy distinto. Dejad vuestros temores, soy inglés como vosotros y dispuesto a ayudaros según veis. Sólo dispongo de un criado, pero tengo armas y municiones. Decidme francamente: ¿puedo seros útil? ¿Qué os pasa?

    —Lo que nos pasa, caballero —me replicó entonces—, es demasiado largo de contar ahora que nuestros asesinos andan cerca: pero en resumen os diré que soy el capitán de aquel navío, mis hombres se han amotinado y apenas han podido contener sus deseos de asesinarme. Por fin han resuelto abandonarme en esta isla desolada junto con estos dos compañeros, uno mi piloto y el otro un pasajero. Esperábamos morir de hambre, creyendo que el lugar estaba totalmente deshabitado e incapaces de abrigar la menor esperanza al respecto.

    — ¿Dónde están esos miserables enemigos vuestros? —pregunté—. ¿Sabéis hacia dónde han ido?

    —Duermen allí, señor —contestó señalando un bosquecilio cercano—. Mi corazón tiembla de miedo al pensar que acaso nos han visto y os han oído hablar, porque en ese caso seguramente nos asesinarán a todos. — ¿Tienen armas de fuego?

    Me contestó que había dos piezas, una de las cuales había quedado en la chalupa.

    —Muy bien —repuse entonces—. Dejad el resto en mis manos. Ya veo que esos hombres duermen, y sería cosa fácil matarlos, salvo que prefiráis tomarlos prisioneros.

    El capitán me dijo entonces que entre ellos había dos desalmados a los cuales era imposible tratar con piedad, pero que eliminados ellos el resto volvería tal vez a su deber. A esa distancia le era imposible describirme su aspecto, pero se manifestó dispuesto a obedecer mis órdenes en todo lo que le mandase.

    —Entonces —dije— alejémonos en primer lugar de las cercanías para evitar ser vistos u oídos, y luego deliberaremos.

    Me siguieron de inmediato, hasta que los bosques nos ocultaron.

    —Ahora bien, caballero —dije al capitán—. Si os ayudo a recobrar vuestra libertad, ¿estáis dispuesto a cumplir dos condiciones que os fijaré?

    Se anticipó a mi propuesta diciéndome que tanto él como su barco, si era recobrado, quedarían totalmente a mis órdenes a partir de entonces, y que en caso de que el buque se perdiese, lo mismo permanecería a mi lado en cualquier sitio del mundo donde yo lo dispusiera; los otros dos hombres afirmaron lo mismo.

    —Muy bien —dije—. Mis condiciones son las siguientes. En primer lugar, que mientras estéis conmigo en esta isla no pretendáis aquí la menor autoridad; si pongo armas en vuestras manos, ellas me serán devueltas cuando así lo disponga y nada se cometerá en mi territorio que resulte en perjuicio mío. Segundo, que si el barco es recobrado, me llevaréis a mí y a mi criado a Inglaterra sin gastos de pasaje.

    De inmediato me dio todas las seguridades que la buena fe y el ingenio hayan podido inventar, asegurándome que me debería la vida y que esa deuda sería reconocida eternamente por él en cuanta ocasión se presentara.

    —Entonces —agregué— aquí hay tres mosquetes para vosotros, con pólvora y balas; decidme ahora qué consideráis conveniente hacer.

    El capitán siguió manifestándome calurosamente su gratitud, pero en cuanto a la conducta a seguir quiso que yo los guiara en un todo.

    Manifesté entonces que la tentativa me parecía peligrosa, pero a mi parecer lo más sensato era hacer fuego de inmediato sobre aquellos hombres en el sitio en que se encontraban; si alguno se salvaba y ofrecía rendirse, le perdonaríamos la vida, encomendándonos a la Providencia para que nuestros disparos fuesen certeros.

    Contestóme con mucha moderación que le repugnaba matar a aquellos hombres y que hubiera preferido evitarlo, pero que los dos incorregibles villanos habían sido los promotores del motín y que si llegaban a escaparse estaríamos perdidos, ya que eran capaces de regresar del barco en compañía de los demás tripulantes y no cejar hasta encontrarnos.

    —Muy bien, entonces —dije yo—; ya veis que la necesidad sanciona mi consejo, y que no hay otro modo de salvar nuestras vidas.

    Pese a todo, y viendo cuánta repugnancia le causaba derramar sangre humana, le di permiso para que procediera junto con sus compañeros del modo que creyese mejor.

    —Bueno —dije—, dejadlos entonces escapar; la Providencia parece haberlos despertado a tiempo para salvarse. Ahora, si el resto huye, la culpa será vuestra.

    Animado con mis palabras tomó el mosquete que le había dado, se puso una pistola en el cinto y sus camaradas lo imitaron, cada uno con un arma en la mano. Al avanzar, estos últimos hicieron algún ruido y uno de los marineros ya despiertos se dio vuelta y al verlos en tal actitud gritó un alerta a los demás. Pero ya era tarde porque en el mismo instante los dos hombres dispararon sobre ellos, mientras el capitán reservaba prudentemente su carga. Tan bien habían apuntado que uno de los marineros cayó instantáneamente muerto y el otro, muy mal herido, apenas podía incorporarse en el suelo pidiendo auxilio a los demás. El capitán se le acercó de inmediato, diciéndole que ya era tarde para pedir auxilio, y que pidiera perdón a Dios por su villanía; dicho eso le descargó en la cabeza la culata del mosquete, dejándolo exánime. De los restantes, solamente uno estaba ligeramente herido, pero como yo llegué en ese momento y comprendieron que no estaban en condiciones de resistir, pidieron perdón al punto. El capitán expresó que les salvaría la vida si le daban absoluta seguridad de su arrepentimiento por la abominable traición cometida y si juraban serle fieles, ayudarlo a recobrar el barco y tripularlo hasta Jamaica, que era su procedencia. Todos ellos hicieron abundantes protestas de sinceridad y él parecía dispuesto a creerles y salvar así sus vidas, cosa a la que yo no me opuse aunque le exigí que tuviera a esos hombres atados de pies y manos mientras permanecieran en la isla.

    En tanto que esto ocurría, mandé a la costa a Viernes con el pilo para que aseguraran la chalupa, ordenándoles que sacaran los remos y la vela; mientras se ocupaban en ello, tres marineros que habían andado vagabundeando por la isla, separados para suerte suya de los otros tripulantes, se acercaron a nosotros atraídos por los disparos. Pero viendo al capitán, un rato antes su prisionero y ahora otra vez el amo, se sometieron de inmediato y nuestra victoria fue completa.






	\chapter{Robinson logra su libertad}





    Aún nos quedaba al capitán y a mí contarnos mutuamente nuestras aventuras. Principié narrándole toda mi historia, que escuchó con una atención vecina al asombro, especialmente las circunstancias maravillosas por las cuales había llegado a proveerme de armas y vituallas. En verdad que siendo mi vida una serie de episodios extraordinarios lo impresionó profundamente. Luego, cuando reflexionó sobre sí mismo y cómo parecía que yo hubiese sido preservado en aquella isla para salvarle más tarde la vida, lágrimas brotaron de sus ojos y no pudo pronunciar una sola palabra.

    Luego de concluir mi narración llevé a los tres hombres a mi morada, haciéndolos entrar por el mismo sitio que empleaba para salir, es decir, la plataforma en lo alto; allí les di de comer los alimentos que había podido llevar mostrando todas las invenciones que había podido llevar a cabo en mi larguísima permanencia en la isla.

    Todo cuanto les mostré, todo cuanto les dije, los dejaba pasmados; pero el capitán admiró por sobre todo mi fortificación, en especial la forma en que había ocultado mi castillo con un soto que, plantado veinte años atrás y formado por árboles que crecen aquí mucho más pronto que en Inglaterra, era ahora un bosquecillo tan espeso que no había manera de atravesarlo salvo por el angosto pasaje trazado por mí. Dije al capitán que aquél era mi castillo y mi residencia, pero que al igual que muchos príncipes poseía una finca en el campo donde me gustaba pasar temporadas y que le mostraría en otra ocasión, pues de momento nuestro problema era considerar el modo de hacernos dueños del navío.

    Convino en ello, pero agregó que no tenía la menor idea de cómo proceder, ya que a bordo quedaban todavía veintiséis hombres que, entregados a tan perversa conspiración y sabiendo que la ley la penaría en sus vidas, procederían arrastrados por la desesperación y dejándose llevar a cualquier extremo, seguros de que si eran reducidos su destino sería la horca apenas llegaran a Inglaterra o a cualquier colonia inglesa. Era por lo tanto imposible pretender atacarlos siendo nosotros tan pocos.

    Medité largo tiempo lo que me dijo, encontrándolo muy atinado; urgía sin embargo encontrar algún camino ya fuera tendiendo una celada a los de a bordo o impidiéndoles desembarcar a toda costa para evitar ser masacrados.

    Entonces pensé que aquellos hombres, asombrados por el retraso de sus camaradas de la chalupa, vendrían a tierra tripulando la otra chalupa del barco, sin duda armados y en gran superioridad de número. El capitán encontró esto muy probable.

    Opiné que nuestra primera medida debía consistir en inutilizar la chalupa que quedara varada, sacando de ella todos los implementos y tornándola inútil para navegar. Nos apresuramos a bajar a la playa y retiramos las armas que habían quedado en la chalupa así como otras cosas que hallamos en ella tales como una botella de aguardiente, otra de ron, algunas galletas, un frasco de pólvora y un gran pedazo de azúcar envuelto en tela, que debía pesar cinco o seis libras.

    Todo esto fue muy bien recibido por mí, especialmente el aguardiente y el azúcar, de los cuales carecía desde muchos años atrás.

    Cuando hubimos llevado todo esto a tierra (ya he dicho nup anteriormente habían sido retirados el mástil, vela y timón de la chalupa) practicamos un gran agujero en el fondo a fin de que si los marineros venían en número suficiente para dominarnos no pudieran sin embargo llevarse la embarcación.

    En verdad apenas pasó por mi pensamiento que pudiéramos apoderarnos del barco, pero mi proyecto era que si aquellos individuos se marchaban sin llevarse la chalupa la pondríamos nuevamente en condiciones para navegar hasta las islas de sotavento, donde podríamos recoger de paso a nuestros amigos españoles, ya que en ningún momento los había olvidado.

    Preparábamos entretanto los planes. Con todas nuestras fuerzas movimos la chalupa para alejarla lo más posible del mar a fin de que la marea alta no pudiera ponerla a flote; luego practicamos un agujero en el fondo del casco, lo bastante grande para que no fuese fácil repararlo, y nos habíamos sentado cerca meditando qué debíamos hacer a continuación cuando oímos un cañonazo disparado de a bordo, mientras con la bandera hacían señales de que la chalupa debía retornar al barco.

    Como ninguna respuesta llegó de la isla, repitieron varias veces las señales y los cañonazos.

    Por fin, cuando se convencieron de que ni una cosa ni otra daba resultado alguno, los vimos con ayuda de mi catalejo arriar otra chalupa y embarcarse rumbo a tierra; a medida que se aproximaban pudimos distinguir que había a bordo no menos de diez hombres, y que venían provistos de armas de fuego.

    Como el barco estaba fondeado a unas dos leguas de la costa, veíamos muy bien la chalupa mientras se acercaba, incluso el rostro y aspecto de sus tripulantes. La marea los arrastraba un poco hacia el este del sitio donde varara la otra chalupa, de modo que remaban tratando de remontar la cosa hasta dar con el fondeadero de la primera embarcación.

    Gracias a eso, repito, tuvimos oportunidad de distinguirlos muy bien, y el capitán por su parte conocía al dedillo el aspecto y carácter de cada uno de los tripulantes de la chalupa. Me dijo que había entre ellos tres hombres honestos que indudablemente habían sido arrastrados al motín por la fuerza o el miedo. En cuanto al contramaestre, que parecía ser jefe absoluto de la conspiración, y los demás marineros, eran tan malvados como el resto de la tripulación y sin duda harían desesperados esfuerzos para dominarnos, por lo cual el capitán tenía fundados temores de que fuesen más fuertes que nosotros.

    Sonreí al escucharlo, y le dije que hombres en nuestras circunstancias debían sentirse más allá del miedo; cualquier situación imaginable sería mejor que aquella en la que estábamos colocados en ese momento, y sus posibles consecuencias, fuesen la vida o la muerte, equivalían de todas maneras a una liberación. Le pedí que considerara mi propia vida, y si una posibilidad de rescate no merecía correr el riesgo.

    — ¿Y qué se ha hecho, señor —agregué—, esa creencia que teníais de que mi vida había sido preservada deliberadamente para salvar alguna vez la vuestra? ¿No estabais tan animado hace un rato? Por lo que a mí respecta, sólo veo un inconveniente en la forma en que se presentan los sucesos.

    — ¿Cuál es él? —preguntó el capitán. —Pues vuestra afirmación de que en esa chalupa hay tres o cuatro hombres honestos cuya vida debería respetarse; si toda esa tripulación hubiera estado constituida por malvados, yo habría supuesto que la Providencia Divina la había escogido para ponerla en vuestras manos. Porque estad seguro de que cada hombre que desembarque en esta costa nos pertenece, y vivirá o morirá según se comporte.

    Como le dije estas cosas con voz animosa y rostro decidido, le devolví grandemente el coraje y proseguimos nuestra tarea.

    Por cierto que apenas habíamos advertido la partida de la segunda chalupa rumbo a la costa convinimos en separar nuestros prisioneros y asegurarlos convenientemente.

    Dos de ellos, a quienes el capitán tenía menos confianza que al resto, fueron llevados por Viernes y uno de los compañeros del capitán a la caverna que estaba bastante lejos y oculta para ser descubierta, y tan pérdida entre los bosques que de nada les habría valido a aquellos individuos escaparse de su prisión.

    Allí los dejaron atados, pero con provisiones suficientes y la promesa de que si se quedaban quietos recobrarían su libertad en un día o dos, pero que si pretendían escaparse serían muertos sin lástima. Aseguraron reiteradamente que soportarían con paciencia su prisión y se mostraron muy agradecidos de que les dejáramos provisiones y luz, ya que Viernes les dio algunas de las velas hechas por nosotros a fin de que lo pasaran mejor. Además, el marinero se quedó de centinela a la entrada de la cueva, para mayor seguridad.

    Los restantes prisioneros recibieron mejor trato. Dos de ellos, sin embargo, siguieron atados, porque el capitán no sentía plena confianza a su respecto, pero los otros fueron puestos a mis órdenes por recomendación de su amo y luego de haber jurado solemnemente vivir y morir a nuestro lado. Con ellos, más los tres rescatados por mí, éramos siete hombres bien armados y no me cabía duda de que podríamos luchar con los diez sublevados que venían en la chalupa, máxime que entre ellos el capitán había reconocido a tres o cuatro hombres honestos.

    Tan pronto como arribaron a la costa se apresuraron a varar la chalupa en la playa y desembarcaron para remolcarla fuera del agua, lo que me alegró mucho porque había temido que la dejaran anclada a cierta distancia de la costa, a cargo de algunos hombres, cosa que nos hubiera impedido apoderarnos de la embarcación.

    Ya en tierra, lo primero que hicieron fue correr a la otra chalupa y es de imaginarse la profunda sorpresa que tuvieron al encontrarla desmantelada y con un enorme agujero en el fondo.

    Luego de discutir un rato en torno a la chalupa se pusieron a dar grandes gritos, repitiéndolos con todas sus fuerzas para llamar la atención de sus perdidos compañeros. Como no obtuvieran respuesta alguna se reunieron en círculo y dispararon al aire sus armas, cosa que oímos perfectamente y que los bosques repitieron como un eco. Pero tampoco les sirvió la descarga, ya que los prisioneros en la caverna no podían escucharla, y aquellos en nuestro poder, aunque la oyeron, no se hubieran atrevido a contestarla.

    Los de la chalupa se quedaron tan asombrados ante su fracaso que, como nos lo dijeron más tarde, resolvieron embarcarse inmediatamente y volver lo antes posible al navío, seguros de que así como la chalupa estaba averiada sus tripulantes habían sido asesinados al desembarcar. De acuerdo con eso, botaron su lancha al agua y se embarcaron al punto.

    El capitán se mostró entonces sorprendido y luego desconcertado, seguro de que apenas llegaran a bordo se harían a la mar dejando abandonados a sus camaradas, con lo cual el navío se perdería para él, que tanto había confiado en rescatarlo.

    Pero pronto tuvo un nuevo motivo para aterrarse. Apenas habían botado aquellos hombres la chalupa cuando lo vimos que cambiaban de idea y retornaban a la costa, con la diferencia de que ahora dejaron tres hombres al cuidado de la chalupa mientras los restantes se internaban en procura de sus compañeros.

    Aquello nos causó una gran decepción, porque ignorábamos la mejor conducta a seguir; apoderarnos de los siete marineros en tierra no nos daba ninguna ventaja si dejábamos escapar la chalupa, ya que sus tripulantes irían de inmediato a bordo, induciendo a los otros a hacerse a la vela, y el rescate del buque se tornaría así imposible. Advertimos de inmediato que lo más prudente era quedar a la espera, a fin de que el curso de los acontecimientos nos dictara el camino a seguir. Los siete marinos bajaron a tierra, mientras los tres restantes mantenían la chalupa a buena distancia de la costa, andándola para quedarse esperando.

    Vimos entonces que tratar de apoderarnos de la chalupa era una empresa imposible.

    Los que habían desembarcado tuvieron buen cuidado de mantenerse unidos, encaminándose hacia las alturas de la pequeña colina bajo cuya ladera estaba mi morada; aunque no podían divisarnos, advertíamos claramente sus movimientos. Nos hubiera agradado verlos acercarse de modo de poder disparar sobre ellos, o bien que se alejaran lo bastante para poder salir de nuestro escondite.

    Cuando llegaron a lo alto de la colina desde donde tenían un amplio panorama de los valles y los bosques que se extendían hacia el noreste, donde la isla era más baja, se pusieron a gritar y hacer señales hasta que estuvieron exhaustos. No pareciendo dispuestos a alejarse mucho más de la costa, así como a separarse entre ellos, terminaron por reunirse bajo un árbol para discutir la situación. De haber decidido dormir allí, como lo hiciera antes el otro grupo, la tarea hubiese sido muy simple para nosotros, pero estaban demasiado inquietos y llenos de aprensiones para aventurarse a dormir, pese a que parecían incapaces de determinar qué clase de peligro los acechaba.

    El capitán me hizo entonces una juiciosa proposición mientras los marineros continuaban deliberando; suponía que iban a resolverse a hacer una nueva descarga cerrada para llamar la atención de sus camaradas, oportunidad que podíamos aprovechar precipitándonos sobre ellos en el preciso momento en que sus armas estuviesen descargadas, con lo cual los obligaríamos a rendirse sin necesidad de ningún derramamiento de sangre.

    Me pareció un excelente y atinado plan, ya que estábamos bastante cerca de ellos para sorprenderlos antes de que hubieran tenido el tiempo de cargar otra vez sus armas.

    Sin embargo la descarga no se produjo, y nos quedamos largo tiempo allí, sin resolvernos a emprender otra cosa. Por fin opiné que nada podría hacerse hasta llegada la noche, y que si entonces los hombres no habían vuelto aún a la chalupa tal vez encontraríamos un medio de situarnos entre ellos y la costa, así como una estratagema para convencer a los de la chalupa que no se acercaran a tierra.

    Impacientes en extremo, permanecimos sin embargo a la espera, pero nos invadió el temor al ver que, luego de largas consultas y deliberaciones, los hombres se levantaban y se ponían en marcha hacia la costa. Como si la aprensión del lugar fuese demasiado para su valor, habían resuelto al parecer retornar lo antes posible al navío, dar a sus compañeros por perdidos y reanudar de inmediato el viaje.

    Tan pronto los vi encaminarse de nuevo hacia la playa comprendí que cesaban la búsqueda, y el capitán, cuando le comuniqué mis temores, pareció desmayarse de angustia.

    Pero entonces se me ocurrió una estratagema para obligarlos a retornar, que me pareció perfectamente realizable. Ordené en consecuencia a Viernes y al piloto que se encaminaran hacia la pequeña ensenada del oeste, en la zona donde los caníbales habían desembarcado cuando Viernes huyó de ellos, y tan pronto llegaran a un altozano, a una media milla de distancia, se pusieran a dar grandes gritos para llamar la atención de los marineros. Les dije que apenas oyesen una respuesta volvieran a gritar para conseguir que fuesen hacia allí, y entonces, internándose cada vez más en la isla y si era posible entre los bosques, fueran describiendo un círculo que los trajera hasta nosotros por un camino que les señalé.

    Los marineros estaban ya embarcándose cuando Viernes y el piloto dejaron oír sus gritos. Tan pronto oyeron el llamamiento lo contestaron a coro y echaron a correr hacia el oeste en dirección de donde venían las voces. A mitad de camino tropezaron con la ensenada que, estando la marea alta, no podían vadear, por lo cual hicieron señales a los de la chalupa que vinieran a pasarlos, que era lo que yo estaba esperando.

    Apenas hubieron pasado al otro lado, advertí que la chalupa se había internado bastante en la ensenada que formaba una especie de seguro puerto en el interior de la isla, por lo cual los marineros se llevaron consigo a uno de los tres que cuidaban la embarcación, quedando los otros a bordo, luego de asegurar la chalupa al tronco de un árbol que crecía en la misma orilla.

    Esto era lo que yo deseaba, y dejando a Viernes y al piloto que prosiguieran su labor avancé con mis compañeros cruzando la ensenada a cubierto de los dos marineros desprevenidos.

    Antes de que pudieran reaccionar caímos sobre ellos; uno yacía descansando en tierra y el otro permanecía en la chalupa. El primero, que estaba dormitando, trató de incorporarse, pero el capitán, que iba delante, lo alcanzó de un culatazo derribándolo, y luego ordenó al otro que se rindiera o era hombre muerto.

    Pocos argumentos fueron necesarios para decidir a un individuo solo contra cinco bien armados que ya habían dado cuenta de su compañero; además, era uno de aquellos que no habían participado voluntariamente en el motín, y por lo tanto no sólo se rindió de inmediato sino que estuvo luego de nuestra parte con toda buena fe.

    Entretanto, tan bien habían cumplido Viernes y el piloto el papel que debían desempeñar, que con sus gritos y respuestas fueron llevando al resto de los amotinados de colina en colina y de bosque en bosque, hasta que quedaron tan agotados que de ninguna manera hubiesen podido volver a la chalupa antes de que anocheciera.

    Viernes y el piloto, ya de regreso entre nosotros, se mostraban también fatigadísimos.

    Sólo nos quedaba esperar ahora su regreso en la oscuridad y caer sobre ellos con la seguridad de dominarlos.

    Varias horas después de la vuelta de Viernes el grupo de los amotinados pudo alcanzar el sitio donde dejara la chalupa. Mucho antes de que llegaran a la ensenada oímos a los que venían delante dar gritos a los retrasados para que se apresuraran, y escuchábamos las voces de los otros quejándose de lo cansados y rendidos que estaban y de que no podían andar más ligero; todo lo cual nos llenó de alegría. Por fin llegaron a la chalupa, pero es imposible describir la confusión que experimentaron al encontrar la embarcación profundamente internada en la caleta, en seco por el reflujo y sus dos tripulantes desaparecidos. Oímos que se llamaban los unos a los otros con acento desgarrador, diciéndose que habían desembarcado en una isla encantada. Estaban seguros de que si había habitantes en ella iban a presentarse de improviso para asesinarlos a todos, y si solamente había espíritus y demonios en torno, los arrebatarían para devorarlos.

    Volvieron a gritar a coro, llamando muchas veces por sus nombres a los dos camaradas desaparecidos, pero no recibieron respuesta. A la débil luz alcanzamos a verlos, corriendo como enloquecidos y retorciéndose las manos en su desesperación; algunos entraban a descansar en la chalupa, volvían luego a salir como si no pudieran hallar reposo, y esto se repetía constantemente.

    Mis hombres hubieran querido que los dejara caer en la oscuridad sobre sus enemigos, pero yo prefería emplear algún otro recurso que evitara tener que matar a tantos hombres; en especial quería proteger la vida de mis compañeros, sabiendo de sobra que los otros estaban muy bien armados. Me resolví a esperar, confiando en que terminarían por separarse; y entretanto, para estar más seguro de ellos, decidí aproximar aún más nuestras líneas, por lo que mandé al capitán y a Viernes que se arrastraran sobre pies y manos hasta ponerse lo más cerca posible sin ser descubiertos, antes de abrir el fuego.

    No llevaban mucho tiempo en su nueva posición cuando el contramaestre, que había sido el principal promotor del motín y que ahora se mostraba el más cobarde y desesperado de todos, vino en dirección a ellos seguido de otros dos de los suyos.

    El capitán estaba tan excitado al comprender que tenía a aquel miserable villano en su poder, que apenas conservó paciencia para esperar que se acercara lo bastante, ya que hasta entonces sólo habían oído su voz. Pero cuando estuvo casi junto a ellos, el capitán y Viernes se enderezaron a un tiempo e hicieron fuego.

    El contramaestre quedó muerto instantáneamente, y de los otros dos uno recibió un balazo en el cuerpo y cayó junto a su jefe, aunque sólo murió una o dos horas más tarde; el otro huyó a toda carrera.

    Al oír los disparos avancé de inmediato con todo mi ejército, del cual era generalísimo, y que contaba con ocho hombres: Viernes —mi teniente general—, el capitán con sus dos compañeros y los tres amotinados que se plegaron a nosotros y a quienes habíamos dado armas.

    En las tinieblas avanzamos sobre ellos, de manera que no podían saber a cuántos ascendíamos. Ordené al marinero que capturáramos en la chalupa, y que ahora era de los nuestros, llamar a los amotinados por su nombre a fin de parlamentar con ellos y tratar de reducirlos sin trabarnos en batalla. Era de imaginarse que, en la situación en que se encontraban, no tardarían en rendirse voluntariamente.

    Con todas sus fuerzas, el marinero llamó a uno de sus compañeros:

    — ¡Tom Smith, Tom Smith!

    — ¿Quién me llama? —repuso Smith—. ¿Eres tú, Robinson?

    El otro, cuya voz había sido así reconocida, replicó de inmediato:

    —Sí, soy yo. En nombre de Dios, Tom Smith, arrojad vuestras armas y rendios o sois hombres muertos en un minuto.

    — ¿A quién tenemos que rendirnos? ¿Dónde están? —gritó Smith.

    —Aquí están. Es nuestro capitán y cincuenta hombres que os han estado acechando todo el tiempo. El contramaestre ha muerto, Will Frye está herido y yo soy prisionero. Rendios o estáis perdidos todos.

    —Si nos rendimos —preguntó Smith—, ¿nos darán cuartel?—Iré a preguntarlo, si prometéis entregaros —repuso Robinson.

    El capitán, que había escuchado el diálogo, se adelantó entonces.

    —Ya conoces mi voz, Smith —gritó—. Si entregáis de inmediato las armas y os sometéis, os garantizo la vida a todos menos a Will Atkins.

    — ¡Por Dios, capitán, dadme cuartel! —gritó entonces la voz de Will Atkins—. ¿Qué he hecho yo? ¿No han sido los otros tan culpables como yo?

    Nada de eso era cierto, porque este Will Atkins había sido el que primero se apoderó del capitán cuando se amotinaron, tratándolo bárbaramente, amarrándole las manos y vociferando toda suerte de injurias. Con todo, el capitán le gritó que debía rendirse a discreción y confiarse a la clemencia del gobernador, con lo cual se refería a mí, ya que todos me daban ese título.

    Momentos después los villanos habían rendido las armas y pedían por sus vidas. Envié entonces al hombre que había parlamentado con ellos y a otros dos para que los ataran sólidamente. Y luego, mi gran ejército de cincuenta hombres, que sumando los tres mencionados se elevaba a ocho soldados, avanzó hacia el enemigo y se apoderó de él y de su chalupa, quedándome yo con otro compañero fuera de su vista por razones de estado.

    Nuestra tarea inmediata consistía en reparar la chalupa y tratar de apoderarnos del navío. En cuanto al capitán, ahora que estaba en libertad para hablar con sus hombres, los apostrofó severamente sobre la villanía que habían cometido, haciéndoles ver la maldad de su designio y cómo sólo les hubiera servido para arrastrarlos finalmente a la peor de las miserias, cuando no a la horca.

    Todos ellos se mostraron muy arrepentidos y suplicaron se les perdonara la vida. A esto les replicó que no eran prisioneros suyos sino del comandante de la isla; y que aunque habían pensado al principio que esa tierra estaba deshabitada, Dios había dirigido sus pasos a un lugar poblado cuyo gobernador era inglés. Agregó que si le parecía bien podía colgarlos a todos sin vacilar, pero que como les había concedido cuartel suponía que el gobernador iba a enviarlos prisioneros a Inglaterra, donde serían juzgados por los tribunales, salvo Atkins, a quien el gobernador enviaba a decir por su intermedio que se preparase a morir, pues sería ahorcado por la mañana.

    Aunque todo esto era una invención, tuvo el efecto que el capitán deseaba. Atkins cayó de rodillas, suplicando al capitán que intercediera ante el gobernador para salvarle la vida; y todo el resto se unió a sus lamentaciones rogando encarecidamente que no los enviaran a Inglaterra.

    Se me ocurrió entonces que la hora de nuestra libertad había llegado y que sería cosa fácil lograr que aquellos individuos colaboraran con nosotros en apoderarnos del barco. Oculto como estaba a sus miradas, a fin de que no descubrieran qué clase de gobernador tenía la isla, llamé en alta voz al capitán. Como lo hacía desde buena distancia, uno de mis hombres tenía la orden de repetir el llamado y decir:

    —Capitán, el gobernador quiere veros.

    —Decid a Su Excelencia que voy de inmediato —replicó entonces el capitán.

    La escena resultó muy bien, y los prisioneros quedaron convencidos de que el gobernador andaba con sus cincuenta hombres por las inmediaciones.

    Cuando llegó a mi lado, expuse al capitán mi proyecto para apoderarnos del barco, el que le pareció excelente, y dispusimos llevarlo a ejecución a la siguiente mañana. En orden a que todo resultara sin tropiezos y con la mayor seguridad, dije a mi compañero que debíamos dividir a los prisioneros de modo que Atkins y otros dos entre los peores fueran enviados sólidamente sujetos a la caverna donde ya estaban los otros. Viernes y los dos compañeros del capitán se ocuparon de cumplir ese cometido.

    Fueron, pues, conducidos a la cueva que hacía de prisión, y que, por cierto, era un lugar espantoso para individuos en el estado en que se encontraban aquéllos. A los otros los conduje a mi enramada, de la que he hecho ya una descripción completa; como había allí empalizada y los hombres seguían atados, el sitio resultaba bastante seguro, máxime cuando la suerte de aquellos dependía de su conducta.

    Por la mañana envié al capitán a conferenciar con los prisioneros de la enramada, a fin de indagarlos y hacerme saber si le parecían dignos de confianza para acompañarnos en la expedición contra el buque. El capitán volvió a hablarles de la injuria que le habían hecho, de la situación en que se encontraban, y les dejó entrever que aunque el gobernador les concedía cuartel por el momento, apenas fueran llevados prisioneros a Inglaterra serían ahorcados con toda seguridad; pero agregó que si estaban dispuestos a unirse a nosotros para tratar de reconquistar el buque, acaso fuera posible lograr formalmente el perdón del gobernador.

    Es de imaginar con cuánta rapidez habrá sido aceptada semejante proposición por hombres que se encontraban en semejante alternativa. Cayeron de rodillas ante el capitán y le prometieron, con las demostraciones más sinceras, que le serían fieles hasta último momento; puesto que iban a deberle la vida, estaban dispuestos a acompañarlo a todas partes, y mientras vivieran lo considerarían como su padre. —Entonces —dijo el capitán— iré a decir al gobernador lo que acabo de oír y trataré por todos los medios de que acceda.

    Vino a mí con el relato de lo hablado y me participó su impresión de que aquellos individuos le serían fieles. Con todo, y para asegurarnos bien le dije que volviera a ellos y apartara a cinco hombres, que serían sus asistentes en la empresa, diciéndoles que el gobernador conservaría en su poder a los otros dos, así como a los tres prisioneros en la caverna, en calidad de rehenes para garantizar la fidelidad de aquellos cinco; y que si alguno mostraba la menor señal de traición, los rehenes serían ahorcados inmediatamente en la playa.

    Todo esto los impresionó, dándoles la seguridad de que el gobernador procedía con severidad. No les quedaba sin embargo otro recurso que aceptar aquellas condiciones, y a partir de ese instante los cinco rehenes, además del capitán, se empeñaron en persuadir a los otros para que cumplieran su deber al pie de la letra.

    Nuestras fuerzas estaban ahora listas para la expedición según el siguiente detalle:

	\begin{enumerate}

		\item el capitán, su segundo y el pasajero;

		\item los dos prisioneros de la primera partida a los que, de acuerdo con la recomendación del capitán, habíamos dado libertad y confiado armas;

		\item los otros dos, que hasta entonces habíamos tenido en la enramada, atados, pero que ahora pusimos en libertad por pedido del capitán;

		\item los cinco recién libertados.

	\end{enumerate}

 Eran, pues, doce en total, aparte de los cinco que mantuvimos prisioneros en la caverna en calidad de rehenes.


    Pregunté al capitán si estaba dispuesto a aventurarse con aquel número en la empresa de abordar el navío; por lo que a mí y a Viernes respecta, pensé que no nos convenía movernos por el momento, teniendo siete hombres que cuidar en tierra, ya que bastante tarea suponía vigilarlos y darles suficiente alimento. Con respecto a los cinco de la caverna decidí mantenerlos atados, pero Viernes iba dos veces por día a llevarles vituallas; los otros fueron empleados en acarrear provisiones hasta cierta distancia de la cueva, donde iba Viernes a tomarlas.

    Me presenté entonces a los dos rehenes en compañía del capitán, quien les dijo que yo era la persona designada por el gobernador para vigilarlos, y que sus órdenes eran que no se apartaran un solo momento del sitio donde yo me encontrara; si lo hacían serían llevados de inmediato al castillo y encadenados. Naturalmente aquellos hombres, como no habían visto nunca al gobernador, me tomaron por su representante, y yo hablaba a cada momento de Su Excelencia, de la guarnición, el castillo y demás cosas parecidas. El capitán no tenía ahora otra dificultad que la de aparejar las dos chalupas, reparar la avería en una de ellas y tripularlas con sus hombres. Hizo a su pasajero capitán de una de las embarcaciones, y puso cuatro hombres a sus órdenes; en persona, y acompañado de su segundo y cinco hombres, se embargó en la otra; aprovecharon la mejor hora y navegaron en dirección al navío a eso de medianoche. Tan pronto estuvieron al alcance de la voz, el capitán ordenó al llamado Robinson que gritara a los del navío, diciéndoles que traían los hombres y la chalupa, pero que les había llevado muchísimo tiempo dar con ellos; con esas y otras   parecidas  charlas  debía  entretener   su  atención mientras los demás se acercaban al navío.

    Entonces, abordando el buque, el capitán y su segundo sorprendieron y derribaron a culatazos al segundo y al carpintero, siendo fielmente ayudados por los hombres que iban con ellos. Asegurando rápidamente a los restantes marineros que estaban en el puente y la popa, trancaron las escotillas para aislar a los que quedaban abajo. Entretanto la otra chalupa, desembarcando a sus tripulantes en los portaobenques del trinquete, aseguró la posesión del castillo de proa y la escotilla que daba a la cocina, donde fueron apresados otros tres hombres.

    Cumplido esto, y dueño del puente, el capitán ordenó al piloto que tomara por asalto la toldilla donde se encontraba el capitán sublevado; éste, despierto y alerta, se encontraba en compañía de dos hombres y un grumete armados con fusiles. Cuando el piloto forzó la puerta con una palanca, el nuevo capitán y sus hombres dispararon a quemarropa sobre los. atacantes, hiriendo al piloto de un tiro de mosquete que le atravesó el brazo, así como a dos de sus hombres, pero sin matar a ninguno.

    Mientras pedía auxilio a gritos entró el piloto, herido como estaba, en la toldilla y con su pistola atravesó la cabeza del capitán rebelde; la bala entró por la boca y salió detrás de una oreja haciéndolo caer sin tiempo de pronunciar una palabra.

    Al ver esto, los otros se rindieron de inmediato y el buque fue tomado sin que resultara necesario sacrificar más vidas.

    Tan pronto estuvo el barco en su poder el capitán ordenó que se dispararan siete cañonazos, señal convenida conmigo para hacerme saber el buen resultado de la empresa; es de imaginar la alegría con que los escuché, habiendo esperado novedades en la playa hasta las dos de la madrugada.

    Escuchada la señal, me dejé caer en la arena y rendido por las muchas fatigas de aquel día dormí profundamente hasta que el sonido de otro cañonazo me despertó sobresaltado. Mientras me incorporaba, oí a alguien gritando:

    — ¡Gobernador, gobernador!

    Reconocí inmediatamente la voz del capitán, y subiendo a la cumbre de la colina lo encontré; al verme señaló en dirección del navío, y viniendo a mí me estrechó en sus brazos mientras exclamaba:

    — ¡Amigo mío, mi salvador! ¡Ahí tenéis vuestro barco que os pertenece, así como nosotros, y todo lo que a bordo existe os pertenece también!

    Miré en dirección al mar y vi el navío a una media milla de la costa. Supe entonces que apenas dueños de la situación se habían apresurado a levar anclas y acercarse, aprovechando la calma que reinaba, hasta la boca de la pequeña ensenada. Estando alta la marea, el capitán había venido con la pinaza hasta el mismo sitio donde yo fondeara mis primeras balsas, y puede decirse que acababa de llegar a la puerta de mi casa.

    La emoción me embargó al extremo de que estuve a punto de desplomarme. Veía ahora con toda claridad la liberación al alcance de mis manos, ya todo resuelto y listo, un gran navío esperándome para llevarme al sitio donde me placiera más.

    En el primer momento me sentí incapaz de articular una sola palabra; y como el capitán me tenía abrazado, me aferré a él con fuerza porque de lo contrario hubiese caído al suelo.

    El advirtió mi emoción, y extrayendo una botella de su bolsillo me hizo beber un trago de cordial que había traído ex profeso. Me senté entonces en el suelo, y aunque el licor me devolvió la serenidad, pasó un rato antes de que pudiera decir algo a mi amigo.

    Mientras tanto, él estaba tan lleno de alborozo como yo, sólo que de distinta naturaleza. Me hablaba continuamente diciéndome mil cosas amables para ayudarme a recobrar la calma; pero tal era el ímpetu de la alegría que llenaba mi pecho que sólo servía para colmar mi espíritu de confusión. Por fin me eché a llorar, y poco después fui otra vez dueño de mis palabras.

    Me llegó entonces el turno de abrazar al capitán como a mi salvador, y los dos nos, regocijamos juntos. Le dije que lo consideraba como enviado por el cielo para librarme de mi cautiverio, y que todo lo ocurrido era para mí una cadena de maravillas; tales cosas, agregué, eran los mejores testimonios de cómo la secreta mano de la Providencia gobierna el mundo, y la evidencia de que los ojos de un Poder infinito alcanzan el más remoto rincón del mundo y envían ayuda al más miserable si a Dios le place hacerlo.

    No olvidé elevar mi corazón al cielo, lleno de gratitud. ¿Y qué corazón hubiera podido olvidar a quien no sólo le brindaba su socorro de tan maravillosa manera en la soledad, sino que era el dador de toda liberación en este mundo?

    Luego que hablamos un momento, el capitán me dijo que había hecho desembarcar para mí las pocas provisiones que quedaban en el buque después que los miserables habían dilapidado las existencias mientras fueron los amos. Llamó a los del bote y les ordenó que desembarcaran los presentes para el gobernador. Y por cierto que aquellos presentes parecían más propios para uno que tuviera que quedarse en la isla, en vez de embarcarse para no retornar ya nunca.

    Ante todo venía una caja con botellas conteniendo excelentes aguas cordiales, seis grandes botellas de vino de Madeira —cada una contenía dos pintas—, dos libras de excelente tabaco, doce pedazos de la carne que había a bordo, seis piezas de salazón de cerdo, un saco de guisantes y unas cien libras de galleta.

    También me trajo una caja de azúcar, otra de harina, un saco de limones, así como dos botellas de zumo de lima y abundancia de otras cosas; pero aparte de eso, lo que me resultó mil veces más útil y agradable fue que me obsequió seis camisas nuevas, seis corbatas, dos pares de guantes, uno de zapatos, un sombrero y un par de medias, así como un excelente traje elegido entre los suyos y apenas usado; en una palabra, me vistió de pies a cabeza.

    Fue ciertamente un útil y grato presente para quien estaba como yo en tales circunstancias; pero pocas cosas en el mundo pueden haber sido tan desagradables como lo fue para mí el ponerme por primera vez aquellas ropas que me parecían incómodas, inútiles y absurdas.

    Concluida esta ceremonia, y luego que aquellas excelentes cosas fueron trasladadas a mi castillo, empezamos a discutir qué haríamos con los prisioneros. Nos era necesario considerar seriamente la posibilidad de hacernos a la mar con ellos, especialmente con dos que sabíamos incorregibles y rebeldes en último grado. El capitán los consideraba temibles bandidos y no se sentía seguro en su proximidad; incluso, si los llevaba a bordo serían encadenados y con la decisión de entregarlos a la justicia en la primera colonia inglesa que alcanzáramos en nuestro viaje. Así y todo noté que parecía muy preocupado con esa idea.

    Al advertirlo, le dije que si lo deseaba yo me comprometía a lograr que aquellos dos hombres pidieran voluntariamente ser dejados en la isla.

    —Creedme que si eso fuera posible —replicó entonces el capitán— yo me alegraría de todo corazón.

    —Muy bien —dije—. Los haré venir y hablaré con ellos al respecto. Ordené entonces a Viernes y a los dos rehenes —que habían recobrado la libertad por cuanto sus compañeros habían cumplido bien su deber— que fueran a la caverna en busca de los cinco prisioneros, atados como estaban, y los condujesen a la enramada, adonde iría yo más tarde.

    Pasado cierto tiempo aparecí allí vestido con mi nueva indumentaria y acompañado del capitán, habiendo recobrado para ese entonces mi título de gobernador.

    Reunidos todos, y en presencia del capitán, ordené que comparecieran los presos y les manifesté que tenía una prolija descripción de su villano comportamiento para con el capitán, la forma en que se habían apoderado del barco y cómo se disponían a cometer nuevas fechorías, hasta que la Providencia los había hecho caer en sus propias redes, llevándolos a precipitarse en la fosa que para otros habían cavado.

    Les hice entender que el barco había sido recobrado bajo mi dirección, que estaba ahora en la rada y pronto verían cómo el capitán rebelde era recompensado por su villanía, ya que no tardarían en descubrirlo colgando del palo mayor; en cuanto a ellos, quise saber qué alegaban en su defensa antes de hacerlos ajusticiar como piratas, ya que poseía suficiente autoridad para disponer de inmediato tal castigo.

    Tomando la palabra en nombre de sus compañeros, uno de ellos contestó que nada tenían que decir salvo que en el momento de rendirse el capitán les había prometido la vida, por lo cual humildemente imploraban mi compasión. Repliqué a mi turno que no sabía qué clase de compasión podía brindarles, ya que por lo que a mí concernía acababa de decidirme a salir de la isla con todos mis hombres y a tal efecto había reservado pasajes en el barco del capitán. En cuanto a éste, no podría llevarlos a Inglaterra en otra condición que la de prisioneros, entregándolos a la justicia encadenados y bajo acusación de rebeldes; de más estaba decir que la consecuencia de aquello sería la horca, de manera que yo no alcanzaba a comprender qué ventaja podía tener eso para ellos, salvo la posibilidad de que se decidieran a quedarse en la isla y encarar allí su destino. Si así lo deseaban, agregué, estaba en condiciones de permitírselo; incluso me sentía inclinado a perdonarles la vida si les parecía posible arriesgarse a permanecer en aquella tierra.

    Al oír mi propuesta parecieron sentirse muy agradecidos, y me aseguraron que preferían en mucho quedarse allí antes que ser llevados a Inglaterra para perecer en la horca; de modo que los dejé en esa disposición de ánimo.

    El capitán fingió entonces poner algunas dificultades al proyecto, como si no quisiera dejar a los hombres en la isla.

    A mi vez fingí molestarme con él y le dije que se trataba de prisioneros míos y no suyos; ya que les había ofrecido esa posibilidad, quería hacer honor a mi palabra. Agregué que si no le parecía bien el arreglo, pondría a aquellos hombres en libertad tal como los había encontrado, y si él insistía en no aceptar el convenio, que se apoderara de ellos si podía encontrarlos en la isla.

    Todavía más agradecidos se mostraron aquellos hombres al oírme hablar así, y de inmediato mandé ponerlos en libertad ordenándoles que se retiraran por los bosques hasta el sitio de donde los habían traído; les dije que dejaría en sus manos algunas armas y municiones, dándoles también consejos necesarios para que pudieran vivir bien en la isla si seguían decididos a quedarse.

    Terminada la reunión me dispuse a embarcarme en el navío, pero pedí al capitán que me dejara esa noche para preparar mis cosas, rogándole que entretanto fuese a bordo y cuidara del orden, enviándome la chalupa al día siguiente; le recomendé también que apenas llegado al barco hiciera colgar del mástil el cuerpo del capitán rebelde para que los de la isla pudieran verlo.

    Apenas marchado el capitán, llamé a mi tienda a los rebeldes y me puse a hablar seriamente con ellos. Les dije que habían hecho a mi parecer una buena elección, ya que si el capitán los hubiera llevado consigo seguramente habrían terminado en el patíbulo. Les mostré la figura del rebelde balanceándose en la verga del mástil, y les dije que solamente podían esperar una cosa parecida.

    Cuando me repitieron su decisión de quedarse manifesté que les haría un relato detallado de mi vida en el lugar, mostrándoles al mismo tiempo los medios de procurarse una existencia confortable. Les narré punto por punto todo cuanto conocía de la isla, mi llegada a tierra, indicándoles cómo había levantado las fortificaciones, la forma en que logré tener pan, plantar el grano y secar las uvas; en fin, todo cuanto podían necesitar para que la existencia no les fuera penosa. También les conté la historia de los dieciséis españoles que llegarían a la isla según mis esperanzas, y les di una carta para ellos, haciéndoles prometer que los tratarían de igual a igual.

    Les dejé mis armas de fuego, es decir, cinco mosquetes, tres escopetas y además tres espadas. Quedaba todavía un barril y medio de pólvora, ya que después de los primeros años empleé muy poca evitando desperdiciarla. Les di completas instrucciones sobre el modo de domesticar las cabras, ordeñarlas y cebarlas, así como la manera de hacer manteca y queso.

    En una palabra, los interioricé de cada detalle de mi propia vida, agregando que intercedería ante el capitán para que les dejara otros dos barriles de pólvora, así como semillas de hortalizas, que tan útiles me hubieran sido. Les regalé el saco de guisantes que el capitán me había traído para comer, enseñándoles la forma de sembrarlos para tener mayor cantidad.

    Cumplido todo esto, me despedí de ellos a la siguiente mañana y embarqué de inmediato. Nos preparábamos para hacernos a la vela, pero no levamos anclas esa noche. A la mañana siguiente, dos de los cinco hombres llegaron nadando hasta el navío, y profiriendo toda clase de quejas contra los otros tres nos suplicaron en nombre de Dios que los recibiéramos a bordo, pues de lo contrario serían asesinados, y terminaron rogando al capitán que los admitiera aunque sólo fuese para ahorcarlos inmediatamente.

    Al oírlos, el capitán pretendió no tener autoridad para acceder a su pedido sin mi consentimiento; después de tenerlos así un rato, y luego que prometieron solemnemente corregirse, los hicimos trepar a bordo, donde luego de ser castigados con azotes se condujeron con toda prudencia y honradez.

    Al subir la marea la chalupa fue enviada a tierra con los efectos prometidos a aquellos hombres, a los cuales el capitán agregó por mi intercesión los arcones con sus ropas; se mostraron sumamente agradecidos al recibirlos, y yo les di coraje diciéndoles que si me era posible enviar algún buque para que los recogiera no dejaría de hacerlo.

    Al abandonar la isla llevé conmigo algunos recuerdos, como ser el gorro de piel de cabra que me había hecho, la sombrilla y mi papagayo; también cuidé de llevar el dinero ya mencionado, que durante tanto tiempo me había sido inútil; estaba enmohecido y oxidado, tanto que hasta no frotarlo bien nadie lo hubiese tomado por plata. Igualmente traje a bordo el dinero hallado en el naufragio del barco español.

    Así dejé mi isla, el 9 de diciembre, según el calendario del buque, y en el año 1686, luego de haber estado en ella por espacio de veintiocho años, dos meses y diecinueve días, y siendo librado de este segundo cautiverio en el mismo día en que antaño me fugara de los moros de Sallee, a bordo del barcolongo.

    Al fin de un largo viaje arribé a Inglaterra el 11 de junio de 1687, después de treinta y cinco años de ausencia.








	\chapter{Fortuna de robinson}





    Al llegar a mi patria era yo en ella tan desconocido como si jamás hubiera pisado antes su suelo. Mi benefactora y depositaría, a quien dejara mi dinero, vivía aún, pero sufriendo grandes privaciones a causa de reveses de fortuna; había enviudado por segunda vez y llevaba una existencia sumamente modesta. Me apresuré a tranquilizarla sobre lo que me debía, asegurándole que no pensaba reclamarle nada; por el contrario, mi gratitud hacia su antigua fidelidad me llevó a ayudarla en cuanto mi pequeño peculio lo permitía. Cierto que en ese momento era bien poco lo que pude hacer por ella, pero le aseguré que jamás olvidaría sus bondades para conmigo, y como se verá más adelante cumplí mi promesa cuando estuve en condiciones de acudir en su ayuda.

    Viajé luego a Yorkshire, hallando que mis padres habían muerto y de la familia sólo quedaban dos hermanas, así como dos niños de uno de mis hermanos. Tanto tiempo había sido dado por muerto en mi hogar que no me habían reservado bienes, de manera que me encontré privado de auxilio y la pequeña cantidad de dinero que llevaba conmigo no era suficiente para establecerme de una manera apropiada en la sociedad.

    Recibí, sin embargo, una muestra de gratitud que no esperaba. El capitán del barco tan providencialmente salvado por mí junto con su navío y cargamento elevó un detallado informe de lo ocurrido a sus armadores, contándoles cómo había yo procedido. Recibí entonces una invitación para que acudiese a verlos, y los encontré en compañía de otros comerciantes, siendo objeto de afectuosas muestras de gratitud así como de un regalo de casi doscientas libras esterlinas.

    Después de reflexionar detenidamente sobre las circunstancias en que me encontraba y las escasas posibilidades de iniciar una empresa con los medios de que disponía, decidí ir a Lisboa y ver de lograr allí algún informe sobre el estado de mi plantación del Brasil, así como la suerte de mi socio, del que imaginaba naturalmente que me habría dado por muerto muchos años atrás.

    Saqué pasaje a Lisboa, a la que arribé en el mes de abril. Mi criado Viernes me acompañaba en todas aquellas andanzas, mostrándose en todo momento lleno de fidelidad hacia mí.

    Al llegar a Lisboa me puse a buscar y tuve al fin la satisfacción de ver a mi viejo amigo el capitán que me rescatara del mar, en la costa africana. Estaba muy anciano y se había retirado dejando a su hijo, ya hombre, a cargo del buque, que continuaba haciendo el tráfico con Brasil. El capitán no me reconoció, y a mí mismo me fue difícil reconocerlo a él, pero después de un momento recordé sus facciones, y lo mismo le ocurrió con las mías.

    Después de regocijarnos mutuamente con la reanudación de nuestra vieja amistad, le pregunté como es de imaginar por el estado de mi plantación y lo que había sido de mi socio. El anciano me dijo que no había viajado al Brasil en los últimos nueve años, pero que podía asegurarme que al abandonar aquellas tierras mi socio vivía aún; en cuanto a los apoderados, a quienes yo dejara junto con aquél al cuidado de mis bienes, ambos habían muerto.

    Con todo creía posible lograr un buen detalle del adelanto de mi plantación, pues luego de haberse difundido la creencia de que me había ahogado en un naufragio, mis apoderados se apresuraron a rendir cuentas de mi parte al procurador fiscal, quien decidió adjudicar aquellos bienes, en tanto no me presentase yo a reclamarlos, un tercio al fisco y dos tercios al monasterio de San Agustín, que los empleaba en beneficio de los pobres y la conversión de los indios al catolicismo.

    Naturalmente bastaría que yo me presentara, o enviase a alguien con suficiente poder para reclamar los bienes en mi nombre, para que todo me fuese entregado. Solamente no me serían devueltas las rentas anuales, que habían sido destinadas a usos de caridad. El capitán me aseguró que el administrador real de las rentas de tierras, así como el «provedidore» o ecónomo del monasterio, habían tenido gran cuidado de que mi socio rindiera anualmente cuenta de lo producido por la plantación, de la cual recibían la mitad.

    Le pregunté si estaba al tanto de las mejoras introducidas en la plantación, y si valía la pena que yo me embarcase rumbo al Brasil; también quise saber si a mi llegada no encontraría dificultades en la toma de posesión de mi parte. Me dijo que ignoraba con exactitud hasta qué punto había crecido la plantación, pero sí sabía que mi socio era ahora un hombre muy rico con sólo el producto de una mitad del total. También recordaba haber oído que el tercio de mi parte consagrado al fisco —que aparentemente era entregado a otro monasterio o fundación religiosa— sumaba más de doscientos moidores1 anuales.

    En cuanto a la toma de posesión de mis bienes, él no encontraba la menor dificultad, ya que mi socio vivía y podría testimoniar de mis derechos, fuera de que mi nombre estaba debidamente inscrito en el registro de propietarios.

    Agregó que los sucesores de mis dos apoderados eran excelentes y honestas personas, dueñas de gran riqueza, por lo cual yo tendría no solamente ayuda para recobrar mis posesiones sino que recibiría una gran suma de dinero, producto de lo rendido por la plantación antes de que pasara a manos del estado en la forma señalada, cosa ocurrida unos doce años atrás según creía recordar.

    Al escuchar sus palabras, me mostré sumamente preocupado e inquieto y quise saber cómo era posible que aquellos apoderados hubiesen dispuesto a su manera de mis efectos, siendo que yo había hecho testamento antes de embarcarme por el cual lo declaraba a él, el capitán portugués, mi legatario universal.

    Me dijo que eso era cierto, pero que no existiendo prueba de que yo hubiese muerto no podía él actuar como ejecutor testamentario hasta tanto se recibiera testimonio seguro de mi desaparición.

    Fuera de eso, no había querido intervenir en un asunto radicado en tierras tan remotas, aunque había registrado debidamente mi testamento a fin de que constasen sus derechos. De haber tenido prueba cierta de mi muerte, hubiese actuado por procuración recibiendo el «ingenio» —como llaman a las fábricas de azúcar— por intermedio de su hijo, que se encontraba actualmente en el Brasil.

    —Sin embargo —agregó el anciano— tengo que daros algunas otras noticias que acaso no os resulten tan agradables. Creyendo que habíais muerto, como lo creía todo el mundo, vuestro socio y los apoderados me rindieron cuentas y entregaron los beneficios en vuestro nombre durante los seis u ocho primeros años. Dichas sumas fueron aceptadas por mí, pero como en aquel entonces había grandes gastos en la plantación, tales como construir un ingenio y comprar esclavos, la suma no se elevó tanto como en años posteriores. Con todo —agregó el capitán— os rendiré el detalle de cuanto he recibido, y la forma en que dispuse del dinero.

    Días más tarde, prosiguiendo mis conversaciones con mi viejo amigo, me entregó la cuenta de lo producido por mi plantación en los primeros seis años, detalle que aparecía firmado por mi socio y los apoderados, y que había sido entregado en especies tales como tabaco en rama, cajas de azúcar, y también ron, melaza y otros productos derivados de la refinación del azúcar. Pude entonces observar que el total crecía de año en año, pero como el desembolso para los gastos mencionados había sido grande las sumas resultaban pequeñas.

    El capitán me hizo saber además que era mi deudor por la suma de cuatrocientos setenta moidores, aparte de sesenta cajas de azúcar y quince fardos dobles de tabaco, que se habían perdido en el naufragio de su barco, ocurrido al regresar a Lisboa unos once años después de mi desaparición.

    Entonces comenzó el anciano a quejarse de sus desgracias, y cómo se había visto obligado a hacer uso de mi dinero para recobrarse de sus pérdidas y adquirir una participación en un nuevo navío.

    —Pese a ello, mi viejo amigo —agregó—, no habrán de faltaros auxilios en vuestra presente necesidad; tan pronto vuelva mi hijo recibiréis todo lo que se os debe.

    Y sacando allí mismo un viejo saco me entregó 160 moidores portugueses así como los títulos de su participación en el buque, del cual su hijo y él tenían una cuarta parte respectivamente, y me los dio como garantía del resto.

    Mucho me emocionaron la honestidad y la gentileza de aquel hombre, tanto que apenas pude soportar aquella escena. Recordaba lo que el capitán había hecho por mí, cómo me libró del mar y con qué generosidad se había conducido en toda ocasión. Al darme cuenta de tan sincera amistad, apenas pude contener las lágrimas escuchando sus palabras, y lo primero que hice fue preguntarle si las circunstancias le permitían desprenderse de tal cantidad de dinero, y si ello no le ocasionaría apuros. Me respondió que sin duda ese pago significaba para él un trastorno, pero de todos modos se trataba de mi dinero y yo lo necesitaba más que él.

    Todo cuanto habló estaba impregnado de afecto, y a mí me costaba escucharlo sin prorrumpir en llanto. Por fin acepté cien moidores, y le pedí papel y pluma para extenderle un recibo por ellos. Entregándole luego el resto, le dije que si algún día entraba en posesión de mi ingenio le devolvería asimismo lo que ahora aceptaba, cosa que más adelante cumplí. En cuanto a los títulos del barco no quería recibirlos de ningún modo, seguro de que si algún día necesitaba yo dinero él era harto honrado para pagarme de inmediato, y si la suerte me permitía recobrar mi plantación jamás aceptaría un solo penique de sus manos.

    Decidido esto, el anciano capitán me ofreció su ayuda a fin de reclamar mis bienes. Le dije que estaba dispuesto a embarcarme en persona para el Brasil, a lo que me contestó que lo hiciera si me parecía bien, pero que había otros recursos para lograr el mismo fin y obtener una inmediata restitución de lo mío.

    Algunos barcos estaban alistándose en Lisboa para emprender viaje al Brasil, y el capitán hizo que mi nombre fuera inscripto de inmediato en un registro público, con una declaración jurada suya en la cual afirmaba que yo estaba vivo y que era la misma persona que había iniciado la plantación de cuya entrega se trataba.

    Legalmente consignada por un notario la declaración, y con un poder adjunto, el capitán me aconsejó enviarla con una carta suya a un comerciante amigo, proponiéndome luego que permaneciera con él en Lisboa hasta que los navios retornaran con noticias.

    Nunca hubo poder ejercido con más legalidad que el que yo diera a aquel comerciante; en menos de siete meses recibí un grueso paquete procedente de los herederos de mis apoderados, los plantadores a cuya cuenta me hice a la mar como he narrado, y dentro del cual encontré los siguientes documentos:

    Primero, una cuenta detallada de lo producido por mi plantación a partir del último año en que sus padres habían ajustado cuentas con el capitán portugués; el balance arrojaba un saldo de mil ciento setenta y cuatro moidores en mi favor.

    Segundo, la cuenta de otros cuatro años durante los cuales administraron los bienes, antes de que el gobierno reclamara la parte que la ley fija en caso de no tenerse noticias del dueño, cosa que ellos llaman muerte civil; el balance de dichos años, por haber aumentado entonces el producto de la plantación, arrojaba un total de treinta y ocho mil ochocientos noventa y dos cruzados, lo que hacían tres mil doscientos cuarenta y un moidores.

    Tercero, la cuenta rendida por el prior del convento agustino, que había recibido rentas por espacio de catorce años; descontando lo destinado a gastos de hospital, declaraba honestamente tener aún ochocientos setenta y dos moidores sin empleo, los que ponía a mi disposición. En lo que respecta a la porción del fisco, nada me fue devuelto.

    Venía además una letra de mi socio donde me expresaba su regocijo por saberme vivo, me hacía un prolijo relato de cómo había progresado la plantación, lo que producía anualmente, así como el número de acres que tenía en la actualidad; me indicaba la superficie sembrada, el número de esclavos que trabajaban allí, terminando por trazar veintidós cruces a manera de bendiciones, y diciéndome cuántas Ave Marías había rezado para agradecer a la Virgen Santísima mi salvación. Me invitaba con mucho calor a que fuese al Brasil para tomar posesión de mis bienes, y que entretanto le enviase órdenes para rendir cuentas a quien yo designase en mi ausencia. Por fin hacía protestas de su amistad, incluyendo a su familia, y me enviaba como regalo siete hermosas pieles de leopardo que había recibido de la costa africana adonde enviaba con frecuencia barcos que sin duda habrían tenido mejor viaje que el mío. Con las pieles venían cinco cajas de excelentes confituras y cien piezas de oro sin acuñar, no tan grandes como los moidores. Por el mismo barco mis apoderados me fletaron mil doscientas cajas de azúcar, ochocientos rollos de tabaco y el resto del producto en oro.

    Ciertamente podía decir yo ahora que el final de Job era mejor que el principio. Es imposible narrar los sentimientos de mi corazón al leer aquellas cartas y enterarme de la fortuna que poseía. Porque como los barcos del Brasil navegan siempre en convoy, junto con las cartas venían los bienes y éstos estaban ya desembarcados y en seguridad antes de que aquéllas llegaran a mis manos.

    En una palabra, palidecí y creí que iba a desmayarme, a no mediar el capitán, que corrió a hacerme beber un cordial. Pienso que la súbita sorpresa producida por la alegría hubiera superado la resistencia de la naturaleza y me hubiera fulminado allí mismo.

    Con todo estuve muy enfermo durante muchas horas, hasta que se envió por un médico, el cual averiguando en parte las razones de mi estado ordenó una sangría, que de inmediato me produjo alivio. Estoy convencido que de no haber recibido ese tratamiento hubiera muerto con seguridad.

    Era ahora dueño, súbitamente, de más de cinco mil libras esterlinas en dinero y una posesión en el Brasil que rendía más de mil libras anuales, tan segura como si hubiese estado en Inglaterra. En suma, me encontraba en una situación de la que apenas alcanzaba a darme clara cuenta, incapaz de serenarme lo bastante para gozar de ella.

    Lo primero que hice fue recompensar a mi antiguo benefactor, el anciano capitán que tan generoso había sido conmigo en mi desventura, lleno de bondad en mis comienzos y honrado al final. Le mostré lo que acababa de recibir, y le dije que aparte de la Providencia, que dispone de toda cosa, a nadie debía tanto como a él, y que afortunadamente estaba ahora en condiciones de recompensarlo, lo que quería hacer cumplidamente. En primer término le entregué los cien moidores que recibiera de él y luego, enviando por un notario, le hice redactar un documento librándolo del pago de los cuatrocientos setenta y dos moidores que había admitido deberme. A continuación mandé redactar un poder por el cual el capitán sería recaudador de las rentas anuales de mi plantación, indicando a mi socio que debería rendirle cuentas y enviarle los productos a mi nombre con las flotas anuales. Agregué una cláusula al final, disponiendo para él una renta vitalicia de cien moidores, así como otra de cincuenta moidores para su hijo. Y así pude pagar mi deuda de gratitud a ese anciano y buen amigo.

    Me quedaba ahora considerar qué camino seguiría y qué destino iba a dar a la fortuna que la Providencia acababa de poner en mis manos. Por cierto que pasaba más preocupaciones que en mi tranquila y sosegada vida en la isla, donde no deseaba más de lo que tenía ni tenía más de lo que deseaba. Ahora, en cambio, abrumado por el peso de mis bienes, reflexionaba sobre la manera de conservarlos en seguridad. Carecía de una caverna donde ir a enterrar mi oro, o un sitio donde dejarlo sin cerrojos ni llave hasta que se oxidara y enmoheciera sin que nadie lo tocase. Mi antiguo amigo el capitán era un hombre honesto, y por el momento el único refugio que tenía.

    En segundo lugar, mis intereses en el Brasil parecían reclamarme allá, pero no quise embarcarme rumbo a aquellas tierras hasta no haber ordenado mis asuntos y puesto mis bienes en manos seguras.

    Pensé primero en la anciana viuda, de cuya honestidad tenía muchas pruebas y que merecía toda mi confianza. Pero era ya muy anciana y pobre, y hasta donde yo podía imaginarlo estaría llena de deudas. No me quedaba más, en una palabra, que volverme personalmente a Inglaterra y llevar conmigo mis bienes.

    Pasaron empero algunos meses antes de resolverme al viaje. Entretanto, del mismo modo que había recompensado generosamente al capitán por sus muchas bondades, así quise hacerlo con aquella pobre viuda cuyo esposo había sido mi primer benefactor y ella, mientras le fue posible, mi fiel depositaría y consejera. Me apresuré, pues, a buscar un comerciante de Lisboa para que escribiera a su corresponsal en Londres con orden de ir en su busca y llevarle en persona cien libras esterlinas de mi parte, así como consuelo y aliento en su pobreza, con la seguridad de que si la vida me lo permitía, en ningún momento iba a faltarle auxilio. Al mismo tiempo remití cien libras a cada una de mis hermanas, que, aunque no estaban en la miseria, vivían rodeadas de preocupaciones; una de ellas había quedado viuda, y la otra estaba casada con un hombre cuya conducta no había sido la deseable.

    Pero entre todos mis parientes y relaciones no podía sin embargo encontrar ninguno a quien confiar el grueso de mis bienes, a fin de embarcarme para el Brasil dejando todo seguro a mis espaldas. Esto, naturalmente, me llenaba de perplejidad.

    Alguna vez había tenido idea de establecerme definitivamente en el Brasil, país en el que me sentía como quien dice naturalizado. Pero en mi conciencia se despertaban algunos pequeños escrúpulos en materia religiosa, que poco a poco fueron disuadiéndome de aquella idea, como contaré luego. Sin embargo, por el momento no era la religión lo que me apartaba de aquel viaje; así como no había tenido escrúpulos en profesar  abiertamente el culto del país mientras viví allá, tampoco lo tendría ahora. Solamente meditando una y otra vez la cuestión, con más profundidad que en otros tiempos, me imaginé viviendo y muriendo en aquel país* y principié a lamentar haber profesado aquella religión, de la que no tenía seguridad que fuese la mejor para que me acompañara en el momento de mi muerte.

    Repito, con todo, que no era esta la razón principal que me detuviera en el proyectado viaje, sino que seguía sin encontrar la persona a quien confiar mis bienes. Me decidí por fin a marcharme a Inglaterra con ellos, seguro de que una vez allá podría hacerme de relaciones que me fueran fieles; de inmediato empecé a hacer preparativos para encaminarme a mi patria con toda mi fortuna.

    Dispuesto ya a ello, y estando la flota del Brasil lista para zarpar, quise ante todo responder en debida forma a quienes me habían remitido tan fiel y excelente rendición de cuentas. Escribí al prior de San Agustín dándole mil gracias por su recta administración y su oferta de los sobrantes ochocientos setenta y dos moidores, de los cuales le rogué que apartara quinientos para el monasterio y trescientos setenta y dos para los pobres, de acuerdo con lo que él dispusiera, y pidiendo al buen padre que rogara por mí, y otras cosas semejantes.

    Escribí luego a mis dos apoderados, dándoles testimonio de toda la gratitud que su justicia y honestidad despertaban en mí. No les envié presente por cuanto su riqueza los colocaba por encima de eso.

    Por fin escribí a mi socio, testimoniándole mi reconocimiento por la diligencia con que había hecho progresar la plantación y el empeño que había puesto en acrecentar su rendimiento; le envié instrucciones para el gobierno de mi tierra y de acuerdo con el poder que había dado a mi amigo el capitán le indiqué mi deseo de que a él le fuera remitido todo lo que me correspondía a la espera de que yo enviara instrucciones más concisas.

    Finalmente le hice saber que no sólo proyectaba ir pronto al Brasil, sino que mi intención era la de establecerme allí por el resto de mi vida. A la carta agregué un regalo consistente en sedas de Italia para su esposa y sus dos hijas.

    Habiendo así arreglado mis asuntos, vendido mi cargamento y convertido todos mis efectos en buenas letras de cambio encaré la siguiente dificultad, que era la de elegir el camino para volver a Inglaterra. Harto habituado estaba yo al mar, y, sin embargo, sentía extraña aversión a la idea de embarcarme rumbo a mi patria. No hubiera podido explicar las causas, pero como no conseguía dominarla llegué incluso a abandonar el viaje cuando ya tenía mis maletas hechas; y no una sino dos o tres veces me ocurrió lo mismo.

    Es verdad que en el mar yo había sido muy desgraciado, y ésta puede ser una de las causas; pero que nadie desoiga nunca los irresistibles impulsos de su espíritu en casos como el mío. Dos de los barcos que había escogido para realizar el viaje (y a tal punto escogido que en uno de ellos llegué a hacer subir a bordo mi equipaje y en otro dispuse análogos arreglos con el capitán) sufrieron grandes desgracias. Uno fue apresado por los argelinos, mientras el otro naufragó cerca de Torbay y todo el pasaje se ahogó con excepción de tres hombres, lo que prueba que en cualquiera de aquellos barcos mi destino hubiera sido funesto.

    Después de atormentarme así en mis pensamientos, acabé por confiar mis aprensiones al anciano capitán, quien se apresuró a pedirme que no viajara por mar sino que hiciera el viaje a La Coruña por tierra, cruzando allí el golfo de Vizcaya hasta la Rochela, desde donde había un cómodo y seguro viaje por tierra a París, luego a Calais y Dover. El otro camino consistía en llegar a Madrid y de ahí por tierra a París.

    En resumen, tan inquieto me sentía ante la idea de navegar que, salvo el obligado tramo de Calais a Dover, me decidí a hacer la travesía enteramente por tierra, lo que además podía resultar mucho más placentero desde que no tenía ningún apuro en llegar a destino. Para mayor seguridad, el viejo capitán me presentó a un caballero inglés, hijo de un comerciante en Lisboa, quien se manifestó dispuesto a viajar conmigo; poco después se agregaron otros dos comerciantes ingleses y dos jóvenes caballeros portugueses, uno de los cuales iba sólo hasta París.

    Éramos en total seis viajeros con cinco sirvientes; los dos mercaderes axial como los dos portugueses se arreglaban con un sirviente entre ambos para evitar mayores gastos, y en cuanto a mí había elegido a un marinero inglés para que me sirviera en el viaje, ya que mi criado Viernes desconocía demasiado las costumbres para serme de utilidad en un trayecto semejante.

    Así salimos de Lisboa, y como el grupo estaba muy bien montado y armado, hacíamos un pequeño ejército en el cual tuve el honor de ser considerado capitán, tanto por ser el mayor de ellos como por llevar dos sirvientes, y también porque había sido el organizador de aquella travesía.

    Del mismo modo que no he querido fatigaros con el relato de mis viajes por mar, tampoco quiero hacerlo ahora con uno por tierra; sin embargo, algunas aventuras que nos acontecieron en tan tediosa y difícil marcha no deben ser omitidas.

    Cuando llegamos a Madrid, como éramos todos extranjeros en España, no quisimos seguir la marcha sin quedarnos un tiempo para visitar la corte y ver lo que merecía ser conocido. Sin embargo, concluía ya el verano, y nos apresuramos a reanudar el viaje abandonando Madrid a mediados de octubre. Apenas habíamos llegado a la frontera de Navarra cuando empezamos a recibir alarmantes noticias de los distintos pueblos que cruzábamos, según las cuales había nevado tanto del lado francés de las montañas que muchos viajeros se habían visto obligados a retornar a Pamplona, después de intentar a todo riesgo cruzar los Pirineos.

    Al llegar a Pamplona tuvimos la confirmación de las noticias. Para mí, adaptado a los rigores de un clima cálido, en países donde difícilmente se tolera alguna ropa, el frío me resultaba insoportable. Aún más penoso me parecía por el hecho de que apenas diez días antes habíamos salido de Castilla la Vieja, donde el clima no sólo es templado sino hasta muy cálido, y casi de inmediato recibíamos el viento tan crudo, tan glacial de los Pirineos, que resultaba intolerable y peligroso por la forma en que se nos helaban las manos y los pies.

    El pobre Viernes tenía un susto terrible al descubrir las montañas cubiertas de nieve y sentir los rigores del clima, cosas que no había visto ni experimentado jamás anteriormente.

    Para abreviar, diré que en Pamplona siguió nevando de tal modo y con tal persistencia que las gentes aseguraban que aquel invierno se presentaba adelantado; los caminos, hasta entonces difíciles de transitar, eran ahora impracticables. En algunas partes la nieve alcanzaba una altura que se oponía a todo intento de franquearla, y no endureciéndose como en los países septentrionales ofrecía el peligro de sepultar vivo al que osara dar allí un paso. Nos quedamos no menos de veinte días en Pamplona, hasta que observando que el invierno avanzaba y no había posibilidades de que el tiempo mejorara, ya que resultaba ser la estación más rigurosa de que se hubiera tenido memoria en Europa, acabé por proponer a mis compañeros irnos a Fuenterrabia y desde allí embarcarnos para Burdeos, lo que significaba solamente un pequeño viaje por mar.

    Mientras considerábamos esta posibilidad, llegaron cuatro caballeros franceses que, detenidos del lado francés de los pasos por la misma razón que lo estábamos nosotros en el lado español, habían encontrado un guía que los llevó a través del país cerca del extremo del Languedoc, haciéndoles pasar las montañas por caminos tales que la nieve no los incomodó mayormente; la que encontraron, según supimos de sus labios, estaba tan endurecida que soportaba fácilmente el peso de caballos y jinetes.

    Buscamos entonces al guía, quien se comprometió a llevarnos por los mismos pasos libres de nieve siempre que nos armáramos de manera adecuada para protegernos de los animales salvajes; era muy frecuente, según nos dijo, que en tiempos de grandes nevadas aparecieran lobos al pie de las montañas y el hambre que la desolación reinante les producía los tornaba altamente peligrosos.

    Le dijimos que veníamos bien preparados para recibir a tales fieras, pero que deseábamos su garantía de que no seríamos atacados por otra especie de lobos que marchan sobre dos pies y que, según se nos había dicho, abundaban mucho, especialmente del lado francés de las montañas.

    Nos aseguró que en los pasos por donde nos llevaría tal peligro era inexistente, de manera que nos decidimos a seguirlo y así lo hicieron también otros doce caballeros con sus sirvientes: algunos eran franceses, otros españoles, y entre ellos se contaban los que habiendo tratado de cruzar las montañas habían tenido que retroceder.

    El quince de noviembre salimos de Pamplona conducidos por nuestro guía, y mi sorpresa fue grande cuando en vez de llevarnos hacia el norte nos hizo desandar el mismo camino por el cual habíamos venido de Madrid. Lo seguimos unas veinte millas, cruzando dos ríos, y entramos en una zona de clima templado donde las tierras tenían un aspecto muy agradable y no había señales de nieve, hasta que de pronto, tornando a la izquierda, nos llevó hacia las montañas por otro camino.

    En verdad que los cerros y los precipicios eran espantosos, pero el guía nos hizo dar tantas vueltas y revueltas, nos llevó por crestas y cornisas tan vertiginosas que terminamos por trasponer las alturas mayores sin haber sido excesivamente molestados por la nieve. Ya entonces nos mostró nuestro guía las hermosas y fértiles provincias de Languedoc y Gascuña que se extendían abajo y a una gran distancia, verdes y florecientes, y a las cuales llegaríamos después de vencer otro trecho de áspero camino.

    Nos sentimos algo inquietos cuando se puso a nevar todo un día y una noche con tanta violencia que tuvimos que detenernos; pero el guía nos tranquilizó asegurándonos que pronto saldríamos del trance. En efecto, advertimos que estábamos ya en el descenso y que nos encaminábamos cada vez más hacia el norte, de manera que proseguimos confiados el viaje.

    Unas dos horas antes de que cayera la noche, cuando nuestro guía cabalgaba un poco adelantado y fuera de nuestra vista, tres monstruosos lobos y un oso salieron de un hueco que daba acceso a un espeso bosque. Dos de los lobos se precipitaron sobre el guía, y si hubiera estado una media milla más adelante de nosotros lo hubiesen devorado antes de poder acudir en su auxilio. Uno de los lobos atacó al caballo, mientras el otro saltaba sobre el jinete con tal violencia que no le dio tiempo a sacar la pistola sino que, perdiendo la cabeza, sólo atinó a gritar con todas sus fuerzas en demanda de socorro. Como mi criado Viernes marchaba a mi lado, le ordené que fuese al galope a ver lo que ocurría. Así que Viernes descubrió la escena gritó tan fuerte como el otro:

    — ¡Amo, amo!

    Pero al mismo tiempo, con extraordinaria valentía, galopó directamente hacia el atacado y sacando su pistola atravesó de un tiro la cabeza de la fiera.

    Fue una suerte para el guía que Viernes acudiera a ayudarlo, pues como estaba habituado a lidiar con esa clase de animales en su país no les temía y se acercaba casi hasta tocarlos antes de disparar sobre ellos; de haber sido alguno de nosotros habría tirado desde más lejos, tal vez errando el disparo o hiriendo al jinete.

    Lo que siguió hubiera bastado para aterrar a un hombre más valiente que yo, y por cierto hizo temblar a todos los que viajábamos cuando, al expandirse el ruido del disparo de Viernes, a ambos lados del camino oímos levantarse un horroroso aullar de lobos; aquellos aullidos, multiplicados por el eco de la montaña, nos daban la impresión de que había prodigiosa cantidad de fieras al acecho; y por cierto que la manada que nos causaba tanto miedo no debía ser de las más pequeñas.

    Apenas mató Viernes al lobo, el que se había encarnizado con el caballo lo abandonó para huir a toda carrera. Por fortuna había mordido al caballo en la cabeza, donde la copa del freno le atascó las mandíbulas, impidiéndole hacer mucho daño. El guía en cambio estaba mal herido, pues el furioso animal alcanzó a desgarrarle el brazo y el muslo. En el momento de llegar Viernes estaba a punto de caer de su encabritado caballo.

    Es de imaginar que al sonido del disparo lanzamos todos nuestros corceles al galope, y que aunque el camino era muy áspero pudimos llegar casi en seguida al sitio de la escena.

    Tan pronto dejamos atrás los árboles que nos impedían ver más adelante, advertimos lo que había ocurrido y cómo Viernes acababa de salvar al pobre guía, aunque en el primer instante tardamos en darnos cuenta de la especie de enemigo que lo había atacado.

    Pero nunca hubo lucha más temeraria, y librada de manera más original, que la que siguió entre Viernes y el oso, tanto que para nosotros, asustados en el primer instante, fue de inmediato la más grande de las diversiones. Así como el oso es un animal pesado y torpe, que no puede correr con la velocidad del lobo, tiene en cambio dos cualidades particulares que por lo común regulan sus acciones. Ante todo, los hombres no constituyen su presa, bien que en las circunstancias en que nos veíamos no es posible asegurar si las nevadas habrían tornado hambriento a aquel animal; pero habitualmente el oso no ataca nunca al hombre si éste no lo provoca antes. Cuando se encuentra un oso en los bosques, él no intentará molestaros si pasáis a su lado sin ocasionarle a vuestro turno molestias; eso sí, tenéis que ser exquisitamente educados con él y cederle el paso, porque es un caballero muy sensible. Por cierto que no se apartaría un milímetro de su camino aunque fuera para dejar pasar a un príncipe; de manera que si tenéis miedo, lo mejor es mirar hacia otro lado y continuar caminando, pues a veces basta detenerse y mirarlo fijamente para que considere esto una ofensa. Lo mismo si le tiráis algo que le acierte, aunque sólo sea un palillo más delgado que el dedo, lo considerará un insulto y abandonará todas sus ocupaciones por la sola de vengarse; en materia de honor, el oso exige siempre cumplida satisfacción. Tal es su primera cualidad, y la segunda consiste en que una vez ofendido jamás abandonará vuestra persecución, de noche o de día, hasta conseguir alcanzaros y obtener a toda costa su ansiada venganza.

    Mi criado Viernes había salvado al guía, y cuando llegamos a su lado lo ayudaba a desmontar, pues el hombre estaba a la vez herido y asustado, tal vez más lo segundo que lo primero; de pronto vimos surgir al oso del bosque, y por cierto que era monstruosamente grande, el mayor que yo haya visto. Nos quedamos sorprendidos de su aparición, pero cuando Viernes lo descubrió vimos que su rostro expresaba alegría y contento.

    — ¡Oh, oh, oh! —exclamó, señalando tres veces hacia el oso—. ¡Oh, amo, darme permiso para estrechar mano del oso! ¡Yo haceros reír mucho! Mi  sorpresa fue  grande  al  ver tan complacido  al muchacho.

    — ¡Estás loco! —atiné a decirle—, ¡Te comerá!

    — ¡Comerme! ¡Comerme! —dijo Viernes—. ¡Yo comerlo a él! ¡Yo haceros reír, vosotros quedar aquí y yo haceros reír mucho!

    Se sentó en el suelo, cambiándose en un instante sus botas por un par de zapatos que llevaba en la faltriquera, entregó su caballo a mi otro sirviente y llevando la escopeta en la mano echó a correr rápido como el viento.

    El oso marchaba lentamente, sin intención aparente de mezclarse con nadie, hasta que Viernes se le acercó llamándolo como si el animal pudiese entender sus palabras. — ¡Oye, oye, yo hablar contigo! —le decía. A cierta distancia seguíamos nosotros la escena, encontrándonos ya en el lado gascón de las montañas y en un vasto bosque, cuyo suelo era llano y bastante abierto, con árboles diseminados aquí y allá.

    Viernes, que como he dicho estaba casi pisándole los talones al oso, se acercó todavía más y levantando de pronto una piedra se la tiró a la cabeza, donde no le hizo más daño que si la hubiese arrojado contra una pared. Aquello sin embargo produjo el efecto que Viernes esperaba, ya que el muchacho se mostraba tan temerario que su intención evidente era que el oso lo persiguiera para «nosotros reír mucho», según su lenguaje.

    Al punto que el oso sintió la pedrada, y vio a su agresor, se volvió rápidamente y se lanzó tras él, dando largas zancadas y moviéndose de una manera tan extraña y rápida que hubiera obligado a trotar a un caballo para alcanzarlo. Viernes huía velozmente, y de pronto se dirigió en nuestra dirección como si quisiera buscar socorro, de modo que resolvimos hacer una descarga contra el oso y salvar al muchacho.

    Yo sentía una gran cólera contra él al verlo lanzar al oso sobre nosotros, en especial cuando la fiera en nada había pretendido atacarnos, de manera que empecé a gritarle lo que merecía.

    — ¡Gran imbécil! —exclamé—. ¿Es ésta tu manera de hacernos reír? ¡Ven aquí y toma tu caballo mientras nosotros matamos al oso!

    Al oírme, respondió a gritos: — ¡No tirar, no tirar! ¡Quedaros ahí, vos reír mucho! Y como el ágil muchacho corría dos metros por cada uno que franqueaba el oso, giró de improviso y viendo a un lado un magnífico roble que parecía apropiado a sus planes, nos hizo señas de que lo siguiéramos y redoblando su velocidad saltó al árbol, no sin antes dejar la escopeta en tierra, a unas cinco o seis yardas del tronco.

    Pronto llegó el oso al árbol, y lo contemplamos a alguna distancia. Lo primero que hizo fue detenerse junto a la escopeta y olfatearla, pero la abandonó en seguida y precipitándose al árbol empezó a trepar con la agilidad de un gato a pesar de su enorme corpulencia.

    Al ver esto me espanté de lo que consideraba una locura de mi criado y en nada vi motivo para reírme; todos nosotros nos apresuramos en cambio a acercarnos al árbol. Cuando llegamos casi junto a él vimos a Viernes que estaba trepado en el extremo de una larga rama del roble, y al oso que se encontraba a mitad de camino en la misma rama. Tan pronto como la fiera llegó a la porción donde era más delgada y flexible, oímos que Viernes nos gritaba: — ¡Ah! ¡Verme ahora enseñar a bailar al oso! Y se puso a saltar y a agitar violentamente la rama, con lo cual el animal empezó a bambolearse, pero hizo lo posible por sostenerse firme, aunque miraba hacia atrás para descubrir la manera de retroceder. Esto, como es de imaginar, nos hizo reír mucho. Pero Viernes no había concluido todavía con él. Al verlo indeciso, comenzó a hablarle como si aquel animal hubiese podido responderle en inglés.

    — ¡Cómo! ¿No venir más cerca? ¡Yo rogarte venir más cerca!

    Dejó entonces de sacudir la rama y el oso, como si hubiese comprendido la invitación, avanzó otro poco; pero nuevamente se puso Viernes a saltar en la rama y el oso se detuvo de inmediato.

    Pensamos que ese era buen momento para acertarle en la cabeza, y grité a Viernes que no se moviera a fin de tirar sobre la fiera, pero él nos detuvo con sus súplicas.

    — ¡Oh, ruego no tirar, no tirar! ¡Yo tirar después y entonces!

    Quería decir que tiraría en el debido momento. En fin, y para abreviar este relato, Viernes danzó tanto en la rama y el oso adoptó unas posturas tan grotescas que nos desternillamos de risa aunque no podíamos comprender cómo se las arreglaría finalmente mi criado. Al principio creímos que intentaba derribar al animal, pero éste era demasiado astuto para eso; no sólo evitaba avanzar más sino que hundía las garras en la madera con tal fuerza que no comprendíamos cómo sería posible terminar la aventura en esa situación.

    Pronto nos sacó Viernes de dudas, por lo que dijo al oso cuando comprendió que no podía desprenderlo de la rama ni persuadirlo de que avanzara otro poco.

    Bien, bien —exclamó—, tú no querer venir, yo ir, yo ir. Tú no venir a mí, yo ir a ti.

    Y con estas palabras deslizándose hasta la extremidad de la rama que se iba inclinando bajo su peso, se dejó resbalar suavemente sosteniéndose de la punta hasta que sus pies casi tocaron tierra. Soltó entonces la rama y fue a tomar su escopeta, quedándose allí a la espera.

    —Bueno —dije yo—. ¿Qué vas a hacer ahora, Viernes? ¿Por qué no le tiras?

    —No tirar —repuso él—; si yo tirar ahora no matar. Yo daros todavía mucha risa.

    Y  así fue, como podrá verse; porque cuando el oso advirtió que se le había escapado el enemigo, empezó a retroceder por la rama, haciéndolo con extremadas precauciones, midiendo cada paso que daba y andando hacia atrás hasta que alcanzó el tronco del roble; allí, con el mismo cuidado y marchando siempre hacia atrás, descendió por el tronco, clavando profundamente las garras y moviendo despacio cada pata.

    En este instante antes de que hubiera logrado apoyarse en tierra firme, Viernes se le acercó y metiéndole el caño de la escopeta en una oreja lo tendió sin vida a sus pies.

    El muy pícaro se volvió luego a nosotros para ver si efectivamente habíamos reído, y cuando advirtió el regocijo de nuestros rostros se echó a reír a carcajadas.

    —Así nosotros matar osos en nuestro país —explicó.

    — ¿Los matáis así? —repliqué—. ¡Pero si no tenéis escopetas!

    —No, no escopeta —dijo—. Tirarles muchas flechas largas.

    Todo aquello nos divirtió mucho, pero estábamos todavía en un sitio desolado, con el guía mal herido y sin saber exactamente qué hacer. El aullar de los lobos me preocupaba, ya que a excepción de los gritos que escuchara en la costa africana, en un episodio que ya he narrado, creo que nada podía haberme llenado más de espanto.

    Todo eso, sumado a la cercanía de la noche, nos disuadió de desollar al oso como nos lo pedía Viernes; de lo contrario hubiéramos llevado con nosotros la piel de aquel enorme animal, que por cierto merecía conservarse; pero aún nos quedaban tres leguas por recorrer y nuestro guía nos urgía a proseguir el camino.

    La tierra estaba allí cubierta de nieve, aunque no con el espesor peligroso de las montañas. Las manadas de lobos hambrientos, como supimos más tarde, habían descendido azuzadas por el hambre, a los bosques y las llanuras, y causaban graves daños en las aldeas, donde sorprendieron a las indefensas gentes, mataron gran cantidad de cabezas de ganado y también a algunas personas.

    Nos quedaba un peligroso paso por atravesar, del cual nos dijo el guía que si había aún lobos en la región los encontraríamos allí; se trataba de una pequeña planicie, rodeada por todas partes de bosques y con un angosto y profundo desfiladero por el cual era necesario pasar a fin de vernos al abrigo del pueblo donde pernoctaríamos. Apenas habíamos cargado nuestras escopetas y alistado para cualquier evento, oímos terribles aullidos en el bosque de la izquierda, un poco hacia adelante y justamente en la dirección por la cual teníamos que marchar.

    La noche caía y la luz era ya débil, lo que empeoraba nuestra situación; al crecer el confuso sonido percibimos distintamente que eran aullidos de aquellas diabólicas fieras. De improviso descubrimos dos o tres manadas de lobos, una a la izquierda, otra detrás y la tercera avanzando de frente, de manera que nos vimos casi rodeados por ellas. Sin embargo, como no se precipitaban sobre nosotros, seguimos avanzando con toda la rapidez de nuestros caballos que, dado lo áspero del camino, apenas podían andar a trote largo. Llegamos así a la entrada de un bosque situado al final de la planicie, bosque que debíamos atravesar por un desfiladero. Fue allí cuando tuvimos la sorpresa de ver, justamente a la entrada del paso, una gran cantidad de lobos detenidos y a la espera.

    En ese instante oímos un tiro en otro lado del bosque, y mirando hacia allí vimos pasar como una exhalación un caballo ensillado, que corría como el viento perseguido por dieciséis o diecisiete lobos. Los feroces animales estaban ya casi sobre el pobre caballo, y seguros de que no podría sostener mucho tiempo la velocidad de su carrera descontábamos que al final lo alcanzarían, como sin duda ocurrió.

    Pero una escena aún más horrible nos esperaba, pues al encaminarnos hacia la entrada por donde habíamos visto salir al caballo encontramos los restos de otro corcel y de dos hombres devorados por aquellas salvajes bestias; uno de los infelices era seguramente el que había disparado el tiro que escuchamos, pues una escopeta descargada yacía a su lado. Los lobos habían devorado la cabeza y parte superior de su cuerpo.

    Aquello nos llenó de horror y no supimos qué hacer, hasta que los mismos lobos se encargaron de señalarnos el camino cuando empezaron a reunirse en enormes cantidades al acecho de las nuevas presas; pienso que había no menos de trescientos de ellos. Afortunadamente para nosotros, a poca distancia de la entrada del bosque vimos los troncos de algunos grandes árboles que habían sido hachados en el verano anterior y dejados allí probablemente para transportarlos más tarde.

    Formé mi pequeña tropa en medio de aquellos árboles, ordenándole tender una línea detrás de un gran tronco; les indiqué que desmontaran y se parapetasen en dicho tronco, disponiéndose en triángulo para encarar tres frentes, dejando los caballos a salvo en el centro.

    Así lo hicimos, y justamente a tiempo; porque nunca se vio una carga más furiosa que la que nos dieron aquellos lobos allí mismo. Se abalanzaron sobre nosotros con furiosos gruñidos, saltando sobre el tronco que formaba nuestro parapeto como si la misma madera fuese su presa. Pensamos que su furia era debida a que alcanzaban a ver nuestros caballos, que constituían su principal objetivo. Ordené a mis hombres que tirasen alternativamente, y con tanta precisión lo hicieron que en la primera descarga mataron una gran cantidad de lobos; pero resultó necesario sostener una constante fusilería porque aquellas fieras volvían a la carga como demonios, los de atrás empujando a los que venían en primera fila.

    Cuando hubimos disparado la segunda andanada observamos que vacilaban algo, y creímos que tal vez retrocederían; pero aquello duró solo un instante porque otros se abalanzaron al asalto, de modo que hicimos dos descargas de pistola; pienso que en esas cuatro descargas alcanzamos a matar diecisiete o dieciocho lobos, hiriendo a doble número de ellos, y sin embargo volvían furiosamente al ataque.

    No quería yo gastar tan pronto nuestras últimas balas, de manera que llamé a mi sirviente (no a Viernes, que estaba ocupado en renovar con prodigiosa habilidad las cargas de mi escopeta y la suya) y dándole un frasco de pólvora le ordené que formara un ancho reguero a lo largo del tronco que nos servía de parapeto. Así lo hizo, y apenas había tenido tiempo de ponerse a salvo cuando los lobos volvieron al asalto y algunos treparon sobre el tronco en el preciso momento en que yo aplicaba a la pólvora la llave de una pistola descargada y tiraba del gatillo. La pólvora se inflamó instantáneamente, y aquellos que estaban sobre el tronco se quemaron mientras seis o siete, por huir del fuego, caían o más bien saltaban sobre nosotros. Los matamos de inmediato, y el resto se mostró tan aterrado con el resplandor, aún más vivo en la oscuridad de la noche, que retrocedieron paso a paso. Ordené entonces descargar una última andanada, y después de eso prorrumpimos en grandes gritos. Los lobos, ya aterrados, nos dieron la espalda y huyeron, aprovechando nosotros para caer sobre los que quedaban heridos en el suelo y rematarlos a golpes de espada. Aquello salió tal como lo esperábamos, porque los aullidos y quejidos de los animales que matábamos fueron claramente escuchados por sus compañeros que se apresuraron a escapar a toda carrera.

    En total habíamos dado cuenta de unos sesenta lobos, y de haber sido de día hubiésemos matado aún más. Ya despejado el campo de batalla nos apresuramos a reanudar la marcha, porque aún nos quedaba una legua larga que recorrer. Oímos a las salvajes bestias aullar en los bosques repetidas veces, y en alguna oportunidad creímos ver algunas, pero como la nieve nos cegaba no tuvimos la seguridad de que fuesen lobos.

    Una hora después arribamos al pueblo donde pernoctaríamos, y allí encontramos un gran pánico y a todo el mundo en armas; la noche anterior los lobos y algunos osos habían asaltado el villorrio provocando un espanto general, y los pobladores se veían obligados a mantener constante vigilancia, en especial durante la noche, para proteger al ganado y como es natural a las gentes.

    Tan enfermo amaneció al día siguiente nuestro guía, con los miembros inflamados a causa de las mordeduras, que nos vimos obligados a dejarlo y contratar un nuevo guía, que nos condujo a Tolosa. Allí encontramos un clima templado, una comarca fértil y placentera, sin nieve, lobos o nada parecido. Cuando narramos nuestra aventura en Tolosa nos dijeron que lo ocurrido era muy frecuente en los grandes bosques al pie de las montañas, especialmente cuando la nieve cubre el suelo; nos preguntaron con sorpresa quién era el guía que se había atrevido a traernos por ese camino en una época tan rigurosa, asegurándonos que habíamos tenido harta suerte de no ser devorados. Cuando les explicamos cómo nos habíamos defendido de los lobos poniendo a los caballos en el centro de nuestras líneas nos lo censuraron mucho, diciéndonos que había cincuenta probabilidades contra una de ser destrozados por los lobos. Parece que es la vista de los caballos los que los torna más furiosos, ya que ellos constituyen su presa preferida. En otras oportunidades temen el simple ruido de un disparo, pero el hambre que los devora sumado a la rabia que esto les produce y la visión de los caballos que ansian devorar, los tornan insensibles al peligro. Nos dijeron que si no hubiese sido por el continuo fuego y la estratagema final de encender un reguero de pólvora lo más probable era que hubiésemos terminado hechos pedazos. Quizá hubiese sido preferible permanecer montados, disparando desde allí, pues los lobos al ver los jinetes en sus corceles no hubieran considerado a estos últimos presa tan fácil; por fin nos aseguraron aquellos hombres que lo mejor hubiese sido quedarnos todos juntos y abandonar los caballos a los lobos, quienes los hubieran devorado permitiéndonos salir sin peligro del bosque, en especial siendo tantos y tan bien armados.

    Por lo que a mí respecta, nunca me sentí tan expuesto al peligro como en aquella ocasión. Al ver más de trescientos lobos precipitándose rugiendo y con las fauces abiertas sobre nosotros, y apenas contando con un débil parapeto para defendernos, me había considerado ya muerto; de lo que estoy seguro es de que jamás volveré a cruzar aquellas montañas, y preferiría hacer mil leguas por mar aunque tuviese la seguridad de ser sorprendido por una tormenta cada semana.

    Mi viaje por Francia no ofreció nada de extraordinario, sino esas incidencias que otros viajeros han narrado mucho mejor de lo que yo podría hacerlo. Fui de Tolosa a París, y luego de breve plazo me trasladé a Calais, donde felizmente hice la travesía hasta Dover, llegando a destino el 14 de enero, después de haber sufrido los rigores de una muy fría estación.

    Me encontraba ahora al fin de mis viajes, y en poco tiempo había logrado reunir mi nueva fortuna, ya que las letras de cambio que traje conmigo me fueron pagadas inmediatamente.

    Mi principal y mejor consejero era la anciana viuda que, llena de agradecimiento por el dinero que le había enviado, no reparaba en fatigas ni preocupaciones por serme útil. Tanta confianza depositaba yo en ella, que me sentía absolutamente tranquilo por la seguridad de mis bienes, ya que la intachable integridad de aquella excelente mujer se conservó invariable desde el principio hasta el fin.

    Pensé, pues, en dejar mi fortuna al cuidado de la anciana y volverme a Lisboa, de donde podría embarcarme rumbo al Brasil. Un escrúpulo religioso se presentó sin embargo en mis pensamientos; había dudado alguna vez sobre la religión romana mientras estuve fuera de mi patria, y especialmente en la soledad de la isla, pero sabía bien que no existía posibilidad de llegar al Brasil y mucho menos de establecerme en él si no me resolvía antes a abrazar sin reserva alguna la religión católica, salvo que, dispuesto a sobrellevarlo todo por mis principios, me convirtiera en un mártir religioso y muriera en la Inquisición. Me resolví por lo tanto a quedarme en mi tierra y, de serme posible llevarlo a cabo ventajosamente, vender mi propiedad.

    A tal fin escribí a mi viejo amigo de Lisboa, que me contestó diciéndome que le sería fácil realizar la venta, pero que le parecía conveniente pedir mi venia para ofrecer la plantación en mi nombre a los dos comerciantes, herederos de mis antiguos apoderados, que vivían en el Brasil y eran naturalmente buenos conocedores del valor de esas tierras; por otra parte, aquellos dos hombres eran riquísimos, de manera que él confiaba que les placería adquirir la propiedad, por la cual pensaba que podría yo obtener unas cuatro mil o cinco mil piezas de a ocho.

    Le contesté concediéndole la autorización para hacer la oferta, y unos ocho meses más tarde, cuando volvió el navío, recibí una carta informándome que la venta había sido aceptada y que los comerciantes remitían treinta y tres mil piezas de a ocho a un corresponsal de Lisboa para que me pagara el valor de la plantación.

    Firmé entonces el documento de venta que me enviaban de Lisboa, y lo envié al capitán, que me devolvió letras de cambio por treinta y dos mil ochocientas piezas de a ocho, reservando una renta de cien moidores anuales para él mientras viviera, y de cincuenta para su hijo, tal como yo se lo prometiera y que le serían entregadas del producto de la plantación según se estipuló.

    Y así he narrado la primera parte de una vida aventurera, una vida señalada por la Providencia y de una diversidad tan extraordinaria como pocas podría mostrar el mundo; principiando alocadamente para terminar con una felicidad a la que ninguno de los acontecimientos anteriores me daba derecho a esperar.

    Cualquiera pensaría que encontrándome de tal modo favorecido por la fortuna estaba muy lejos de correr nuevos azares, y en realidad así hubiera sido a no mediar ciertas circunstancias. En primer término estaba yo habituado a una existencia errante, no tenía familia ni muchas relaciones, y aunque rico no me sentía mayormente vinculado. Cierto que había vendido mi plantación del Brasil, pero no me era posible olvidar ese país y a cada instante sentía el deseo de lanzarme otra vez a viajar. Especialmente me costaba resistir a la tentación de ver de nuevo mi isla y saber silos pobres españoles habían logrado llegar a ella y cómo los trataban los tres picaros que dejé en tierra.

    Mi excelente amiga la viuda me disuadió con todas sus fuerzas de la empresa, y tanto calor puso en sus argumentos que logró impedir durante siete años que me embarcara, tiempo en el cual tomé a mi cargo a mis dos sobrinos, hijos de mi difunto hermano. Al mayor, que poseía algunos bienes, lo eduqué como a un caballero y agregué una buena cantidad a sus rentas para que recibiera esa fortuna después de mi muerte. Al segundo lo puse al cuidado de un capitán de navío, y cuando cinco años más tarde vi que era un sensato, valiente y emprendedor muchacho, le confié un barco y lo envié al mar. Este mismo muchacho fue el que más tarde me envolvió, viejo como yo estaba, en nuevas aventuras.

    Entretanto me radiqué allí, principiando por contraer matrimonio muy ventajosamente; de esa unión nacieron tres hijos, dos varones y una niña, pero mi esposa falleció más tarde, y cuando mi sobrino llegó a casa después de un afortunado viaje a España, mi inclinación aventurera sumada a sus requerimientos pudieron más que la prudencia y me llevaron a emprender viaje a bordo de su barco, en carácter de comerciante particular con destino a las Indias Orientales. Esto sucedía en el año 1694.

    En el transcurso de ese viaje visité mi nueva colonia en la isla, vi a mis sucesores, los españoles, oyendo de sus labios todo el relato de sus vidas, así como de los villanos que allí dejara; cómo al comienzo insultaron a los pobres españoles, y se pusieron luego de acuerdo para separarse y volver a unirse, y así hasta que al fin los españoles se vieron precisados a emplear la violencia con ellos; cómo quedaron sometidos y con cuánta justicia los trataron los españoles. Un relato, en suma, que de entrar en detalles resultaría tan maravilloso como el mío, en especial en lo que se refiere a sus batallas con los caribes, que desembarcaron repetidas veces en la isla, sin contar los adelantos que aquéllos hicieron en esas tierras; asimismo sería interesante referir cómo un grupo intentó llegar al continente del que volvió trayendo once hombres y cinco mujeres prisioneros, a causa de lo cual encontré a mi llegada cerca de veinte chiquillos en la isla.

    Allí estuve unos veinte días, dejándoles toda clase de provisiones necesarias, especialmente armas, pólvora, balas, ropas y herramientas, así como dos trabajadores que había llevado conmigo de Inglaterra: un carpintero y un herrero.

    Aparte de eso dividí la isla en parcelas que les confié, reservándome la propiedad total y entregando a cada uno la porción acorde a su persona y conveniencia; por fin, luego de dejar todo arreglado y comprometerlos a que no abandonaran la isla, me embarqué nuevamente.

    De allí fui al Brasil, desde donde envié un barco comprado por mí con más habitantes para la isla; entre ellos, y aparte de diversas cosas necesarias, iban siete mujeres que traté de elegir aptas para ocuparse de las faenas de la isla, y con las que podrían casarse quienes lo quisieran. En cuanto a los ingleses, les prometí enviarles algunas mujeres de Inglaterra junto con un cargamento de provisiones, siempre que se dedicaran a ser plantadores, como así lo hicieron más tarde. Por cierto que una vez dominados aquellos hombres demostraron ser honrados y trabajadores, y poseían sus propiedades aparte. Les hice llegar desde el Brasil cinco vacas, tres de ellas con terneros, algunas ovejas y también cerdos, todos los cuales estaban considerablemente multiplicados cuando volví a mi posesión.

    A todo esto habría que agregar la historia de cómo trescientos caribes invadieron la isla, arruinando las plantaciones y librando dos veces grandes batallas, en las cuales los colonos fueron al principio derrotados, perdiendo tres hombres, hasta que una tormenta destruyó las canoas enemigas, y el hambre y las luchas acabaron con la mayor parte de los caribes, permitiendo por fin la reconquista de las plantaciones, que fueron renovadas, y junto a las cuales todavía vivían los colonos... Todo eso, repito, con los sorprendentes episodios de otras nuevas aventuras mías durante diez años, podrán tal vez constituir más adelante otra narración.


	%\part{}

	\chapter{Robinson vuelve al mar}





    Existe un proverbio frecuentemente empleado y que encuentra en la historia de mi vida su mejor verificación: «Genio y figura, hasta la sepultura.» Cualquiera podría pensar que después de treinta y cinco años de aflicciones y toda clase de desdichados sucesos que pocos hombres, según pienso, habrán tenido que soportar, y luego de casi siete años de tranquilidad y gozo rodeado de las cosas más apetecibles, ya viejo y con experiencia suficiente para discriminar sobre las distintas posibilidades de una vida atemperada y elegir entre ellas la más propia para hacer a un hombre enteramente feliz, cualquiera hubiese pensado, repito, que mi propensión natural a las aventuras, cuya intensidad he descrito al referir mis primeras andanzas por el mundo, habría ya cedido terreno y que a los sesenta y un años de edad me sentiría más inclinado a permanecer en mi hogar que a lanzarme fuera de él arriesgando otra vez la vida y la fortuna.

    A esto hay que agregar que la razón habitual de esta clase de riesgos ya no existía para mí, por cuanto era hombre rico y sin ninguna necesidad de buscar otros bienes. De ganar diez mil libras no hubiera sido más rico por ello, ya que tenía suficiente para mí y aquellos a quienes legaría mi fortuna, la que por otra parte iba en aumento; de manera que mi verdadera ocupación consistía en quedarme quieto y gozar plenamente de cuanto la suerte me otorgara, viendo a la vez cómo aumentaba día a día su caudal.

    Todas estas consideraciones no producían efecto en mí, por lo menos en medida suficiente como para combatir la fuerte tentación que me acometía de navegar una vez más, la que se presentaba con la regularidad de un mal crónico. Lo que más me movía era el deseo de ver mi nueva plantación en la isla, así como la colonia que allí dejara; esto bullía constantemente en mi cerebro. Soñaba noche a noche con la isla, y de día me la imaginaba, y la tenía a cada instante en mis pensamientos; tanto y tan ardientemente incubó mi fantasía esa idea que hasta en sueños hablaba yo de ella.

    Con frecuencia he oído decir a personas de buen sentido que toda la algazara que hacen las gentes a propósito de fantasmas y apariciones obedece simplemente a la fuerza de su imaginación y los excesos a que la fantasía puede llegar en sus mentes; agregan que no hay tales espíritus que se aparezcan, ni fantasmas, ni cosas parecidas.

    Por mi parte, hasta ahora no sé lo que existe de cierto en materia de apariciones, espectros, muertos que retornan, ni si cuanto se narra en relatos de esa clase es simplemente producto de alucinaciones, mentes enfermas o caprichos imaginarios. Pero sí puedo asegurar que mi imaginación obraba con tal fuerza, sumiéndome en arrobadores éxtasis —si puedo llamarles así—, que frecuentemente me parecía estar en la isla, en mi castillo detrás de los árboles, y ver a mi viejo español, al padre de Viernes y a los marineros rebeldes que quedaran allá; incluso creía hablar con ellos, y aunque estaba bien despierto los veía tan claramente como si los tuviera delante de mí. Esto llegó a un extremo que terminó por asustarme a mí mismo.

    En una oportunidad, mientras dormía, oí al español y al padre de Viernes relatarme con tal claridad la villanía de los marineros amotinados, que me dejaron pasmado. Dijeron que los ingleses habían tratado de asesinar a los españoles llegando a incendiar las provisiones que éstos habían acumulado con el propósito de matarlos por hambre. Jamás había sabido yo nada de tales cosas, y por cierto que en la realidad ninguna de ellas resultó cierta, pero mi imaginación me las mostraba con tanta claridad que en la misma hora en que las vi en sueños tuve la certeza de que eran exactas. En mi fantasía el español me presentaba sus quejas, las cuales me ocasionaron gran inquietud, por lo cual ordené se hiciera justicia, formando tribunal a los tres picaros y condenándolos allí mismo a ser ahorcados. Lo que hubiera de cierto en todo esto se verá más adelante, porque aunque estas imágenes vinieron a mí en sueños, traídas quién sabe por qué secreta comunicación de espíritus, mucho de verdadero había en ellas.

    Volvamos, sin embargo, a mi historia. En tal estado de ánimo viví algunos años sin poder gozar de la vida, sin horas gratas ni diversión placentera salvo cuando de algún modo se relacionaban con el objeto de mi preocupación. Mi esposa, que había observado la forma en que yo vivía, absorbido por esa idea fija, me dijo una noche que había llegado a convencerse de que tal vez un secreto y poderoso impulso de la Providencia pesaba sobre mí incitándome a navegar otra vez, y que pensándolo bien sólo encontraba un obstáculo para mi partida: las obligaciones que le debía a ella y a mis hijos. Agregó que, naturalmente, no podía pensar en partir conmigo, pero estaba segura de que a su muerte lo primero que haría yo sería marcharme, por lo cual viéndome a tal punto determinado en mi empresa no quería ser mi única obstrucción, de modo que si lo creía conveniente y me resolvía a partir... Al llegar aquí observó que yo consideraba muy atentamente sus palabras, mirándola a la vez con ansiedad, de manera que se turbó algo y no dijo nada más de lo que sin duda había proyectado decir. Observé, sin embargo, que estaba emocionada y que las lágrimas brillaban en sus ojos.

    —Habla, pues, querida mía —dije—. ¿En verdad quieres que me vaya?—No —repuso ella vivamente—. Estoy lejos de querer eso, pero puesto que estás decidido a viajar, y antes de constituir el único obstáculo que te detiene aquí, prefiero ir contigo. Aunque me parece una idea descabellada para un hombre de tu edad y tu situación, si estás decidido a hacerlo —continuó mientras sollozaba— yo no te dejaré un instante. Si es una orden del Cielo hay que cumplirla; de ningún modo debes oponerte a ella; y si el Cielo dispone que tu deber sea marchar, no me impedirá que el mío sea acompañarte, o de lo contrario proveerá para que no sea yo un obstáculo.

    Tan afectuoso comportamiento de mi esposa disipó un poco los vapores de mi mente, y me puse a considerar mi proceder. Hice lo que estaba a mi alcance por dominar mi fantasía y con toda la calma posible empecé a discutir conmigo mismo qué razón podía guiarme, después de una vida de sesenta años llena de desastres y sufrimientos, llegada, no obstante, a un final tan feliz y próspero, para lanzarme otra vez a nuevos azares y verme envuelto en aventuras solamente apropiadas para la juventud y la pobreza.

    Junto con esos pensamientos consideré mi situación: tenía una esposa y tres niños; poseía cuanto el mundo podía darme sin necesidad de correr riesgos para conseguirlo; empezaba a declinar en mi vejez, edad más apropiada para pensar en disponer de mis bienes que en acrecentarlos. En cuanto a lo que mi esposa dijera acerca de un impulso proveniente del Cielo y mi deber de obedecerlo, no tenía ninguna noción clara acerca del mismo. De manera que luego de muchas meditaciones parecidas empecé a luchar contra los poderes de mi imaginación teniendo a la cordura por arma contra ella, como pienso que debe hacer todo individuo en análogas circunstancias. Buscando el método más efectivo y seguro me decidí a ocuparme en otras cosas, dedicarme a negocios y tareas que me alejaran definitivamente de tales fantasías, habiendo advertido que cuando estaba sin hacer nada aquellas ideas volvían con más fuerza aprovechando mi inacción.

    A tal propósito adquirí una pequeña granja en el condado de Bedford, donde decidí afincarme. Tenía una casa no muy grande pero cómoda, y las tierras que la rodeaban me parecieron aptas para hacer en ellas grandes mejoras; todo esto se adaptaba mucho a mis inclinaciones, ya que me agradaba cultivar las restantes faenas rurales. Finalmente, como se trataba de una zona interior del país, me hallaba a salvo de encontrar marineros y barcos que me recordaran de inmediato los lugares más remotos de la tierra.

    En suma, que nos instalamos en la granja; compré arados, rastrillos, una carreta, un carro, caballos, vacas y ovejas, y poniéndome al trabajo con toda dedicación llegué a convertirme seis meses más tarde en un simple hacendado rural. Mis pensamientos estaban absorbidos por la tarea de dirigir a mi servidumbre, cultivar los terrenos, cercar, plantar y demás; llegué a gozar de la más agradable vida que la naturaleza haya podido darnos, y que parecía señalada para un hombre acostumbrado a toda clase de infortunios.

    Cultivaba mi propia tierra, no tenía arrendamiento que pagar y ninguna obligación me afligía. Lo que sembraba era para mí, y las mejoras serían para mi familia; abandonado ya todo pensamiento de aventuras, no sentía pesar sobre mí la menor preocupación. Llegué a decirme que por fin estaba gozando de esa medianía que tan encarecidamente me recomendara mi padre, y que llevaba una existencia casi celestial, algo como lo que describe el poeta a propósito de la vida de campo:



    A salvo de los vicios, a salvo de los daños.

    Ancianidad sin males, juventud sin engaños.



    Pero en medio de toda esta dicha un imprevisto y duro golpe de la Providencia volvió a desquiciar m< vida bruscamente, no sólo reabriendo en mí una llaga incurable, sino arrastrándome por sus consecuencias aúna profunda recaída en mi temperamento errante, tan arraigada estaba en mi sangre esta tendencia que no tardó en dominarme con una fuerza tan irresistible que nada hubiera podido oponerse a ella. Ese golpe fue la pérdida de mi esposa.

    Era ella el apoyo, el puntal de todas mis actividades, el centro de mis empresas, la fuerza que, por su prudencia, había podido reducirme a las felices dimensiones de mi actual vida, alejándome de extravagantes o insensatos proyectos que bullían en mi mente como ya he contado, y haciendo más por guiar mi errante disposición que todas las lágrimas de una madre, los avisos de un padre, los consejos de un amigo o mi propia capacidad de reflexión. Yo me sentía feliz al ceder ante sus lágrimas y sentirme conmovido por sus instancias, de manera que su muerte me dejó desolado y confundido en lo más hondo del ser.

    Apenas hubo ella partido de este mundo, todo me pareció incongruente en torno mío. No sabía qué hacer ni qué dejar de hacer. Mis pensamientos volvían en torbellino a la vieja idea; mi mente se trastornaba con los caprichos de remotas aventuras; todos los gratos, inocentes placeres de mi granja y mi jardín, el ganado y la familia, que me absorbían antes por entero, dejaron de tener significado a mis ojos y perdieron su sabor; eran como música para un sordo o alimento para quien carece de paladar. Me resolví finalmente a abandonar la granja, volver a Londres, y pocos meses más tarde había hecho ambas cosas.

    Cuando estuve en Londres no me sentí más tranquilo. No hallaba gusto en la ciudad, ni nada interesante que hacer en ella, salvo vagar como un desocupado del cual pudiera decirse que resultaba perfectamente inútil en la Creación y de cuya vida o muerte nada importaba a la sociedad. De todas las maneras de vivir, para mí, que había estado siempre en actividad incesante, ésta era la más odiosa, y con frecuencia me repetía: «La holgazanería es la escoria de la vida.» ¡Cuánto más aprovechado me parecía mi tiempo en la época en que necesitaba veintiséis días para construir una mesa de pino!

    Principiaba el año 1693 cuando mi sobrino, del que ya he dicho que se había educado en la marina y hecho capitán de un barco, volvió a casa después de un corto viaje a Bilbao, el primero que hacía. Apenas llegado me comunicó que algunos comerciantes de su relación le habían propuesto un viaje por cuenta suya a las Indias Orientales y a la China, en carácter de comercio privado.

    —Ahora bien, tío —agregó—, si queréis haceros a la mar conmigo, me comprometo a llevaros a vuestra antigua morada en la isla, ya que tenemos que hacer escala en Brasil.

    Ninguna demostración de que existe una vida futura y un mundo invisible puede ser más profunda que la concurrencia de causas secundarias con las ideas que nos formamos y mantenemos en el secreto de nuestra mente, sin hacerlas saber a nadie en el mundo.

    Mi sobrino ignoraba por completo hasta qué punto la enfermedad de las aventuras había vuelto a apoderarse de mí, y a la vez yo ignoraba lo que él proyectaba proponerme; esa misma mañana, rato antes de que viniese a verme, acababa de debatir una vez más la cuestión y de resolver, luego de pesar todas las circunstancias, que me iría a Lisboa para consultar al viejo capitán portugués. Si de esta consulta surgía algo razonable y posible, embarcaría rumbo a mi isla para averiguar qué había sido de las gentes que en ella dejara. Mucho me había complacido la idea de poblar aquellas tierras llevando colonos, obteniendo un derecho de posesión, y muchas otras cosas, cuando a mitad de mis proyectos apareció mi sobrino, como he contado, con la intención de llevarme en su travesía a las Indias Orientales.

    Apenas hubo hablado lo miré fijamente por un momento.

    — ¿Qué poder diabólico —le pregunté— te ha enviado con tan maligno mensaje? Se sobresaltó mucho, pero dándose cuenta de inmediato que yo no estaba disgustado con su propuesta, recobróse al punto.

    —Confío en que no sea un maligno proyecto, tío —respondió—. Me atrevo a decir que os complacerá volver a visitar vuestra colonia, donde una vez reinasteis con más acierto que muchos de vuestros colegas, los monarcas de este mundo.

    En suma, la proposición coincidía tan exactamente con mis deseos y los impulsos a los cuales vivía sometido, que pocos momentos después le dije que si concertaba el viaje con sus amigos comerciantes yo lo acompañaría, pero que no podía prometer ir más allá de la isla.

    — ¡Cómo, tío! —exclamó—. ¿Es que acaso queréis quedaros otra vez allá?

    — ¿Por qué? —repuse—. ¿Acaso no puedes recogerme a tu vuelta?

    Me explicó que sin duda los comerciantes que fletaban el buque no le permitirían que volviese por aquella ruta con un navío cargado con ricas mercancías, ya que el viaje demoraría por lo menos un mes más, y tal vez tres o cuatro.

    —Aparte de eso, tío —concluyó—, suponiendo que yo naufragara y no pudiese llegar en vuestra busca, quedaríais reducido a la condición de antaño.

    Todo esto era razonable, pero pronto encontramos un remedio que consistía en llevar a bordo un balandro desarmado cuyas piezas, una vez desembarcadas en la isla y con ayuda de carpinteros que contrataríamos ex profeso, permitirían armarlo y disponerlo para navegar en pocos días.

    No tardé mucho en decidirme porque la insistencia de mi sobrino se sumaba irresistiblemente a mi propia inclinación, de manera que nada podría haberme detenido ya. Fallecida mi esposa, no tenía a nadie tan próximo a mí que tuviera autoridad para aconsejarme esto o aquello, salvo mi anciana amiga la viuda, que hizo cuanto pudo para convencerme de que mi edad y mi situación debían apartarme de los innecesarios riesgos de un largo viaje; pero en especial, trató de mostrarme mis obligaciones para con mis pequeños hijos. Pero de nada sirvió todo esto, ya que le dije francamente que había algo tan extraño en los impulsos que sentía de viajar otra vez, que resistirme a ellos y quedarme en mi hogar sería casi atentar contra la Providencia. Después de esto cesó en sus tentativas y se puso de mi parte, no solamente en la tarea de preparar el viaje, sino prometiéndome ocuparse de los asuntos de mi familia durante mi ausencia, así como de la educación de mis hijos.

    Hice entonces testamento, disponiendo de tal manera la entrega de mis bienes a mis hijos y confiando su administración en tales manos que me sentí absolutamente seguro de que nada malo podría ocurrirles sucediera lo que sucediese en mi viaje. En cuanto a su educación la confié a la viuda, con una renta suficiente para atender a sus necesidades, lo cual ella merecía sobradamente, pues nunca madre alguna se preocupó más de la crianza de sus hijos o supo encaminarlos mejor; y como alcanzó a vivir hasta mi vuelta al hogar, tuve también vida suficiente para darle las gracias por todo.

    Mi sobrino estaba listo para iniciar el viaje a principios de enero de 1695. Viernes y yo embarcamos en los Downs el día ocho, llevando con nosotros —aparte del balandro mencionado— un considerable cargamento de cosas necesarias en mi colonia, que pensaba desembarcar allí en caso de encontrarla en condiciones desventajosas.

    Ante todo llevé conmigo varios sirvientes que me proponía establecer como colonos o, por lo menos, hacer trabajar por cuenta mía mientras durara mi permanencia en la isla, dejándolos luego en tierra o no, según su voluntad. En especial contraté dos carpinteros, un herrero y un diestro e ingenioso muchacho que era tonelero de profesión, pero que entendía mucho de mecánica, que era capaz de hacer ruedas, molinos de mano para el trigo, y muy hábil en alfarería, así como buen tornero. Era hombre diestro para fabricar cualquier cosa con barro o con madera, de manera que le llamábamos Juan Sabelotodo.

    Junto con ellos llevé a un sastre, que al comienzo proyectaba ir como pasajero hasta las Indias Orientales, pero más tarde consintió en quedarse en nuestra nueva plantación y fue un hombre excelente y utilísimo en toda clase de cosas aparte de su profesión. En fin, como ya he dicho antes, necesitábanse hombres diestros en todas las tareas.

    Mi cargamento, hasta donde lo recuerdo con exactitud, ya que no he conservado detalle en particular, consistía en suficiente cantidad de géneros y algunas piezas de telas inglesas livianas, para que los españoles que yo esperaba hallar allá pudieran hacerse ropas. Embarqué cantidad aproximada para que les durase unos siete años, y aparte de eso creo recordar que llevaba prendas de vestir tales como guantes, sombreros, zapatos, medias y otras cosas, que en total sumaban un valor de doscientas libras, incluyendo algunas camas con sus colchones y menaje doméstico, en especial vajilla de cocina, cacharros, ollas, peltre, calderos de cobre y cerca de cien libras más en ferretería, clavos, toda clase de herramientas, goznes, anzuelos, cerrojos y cuanto pensé que pudiera ser necesario.

    Embarqué asimismo un centenar de armas como mosquetes y fusiles, aparte de pistolas, considerable cantidad de balas de todos los tamaños y dos cañones de bronce; por lo mismo que ignoraba cuánto tiempo tendría que pasar allá y a qué extremos podía verme reducido, llevé conmigo cien barriles de pólvora, espadas, machetes y hierros de picas y alabardas, de modo que en resumen teníamos a bordo un gran depósito conteniendo toda clase de cosas. Hice que mi sobrino instalara, aparte de los necesarios al buque, dos pequeños cañones de alcázar a fin de dejarlos en la isla si se presentara motivo; con todo ello estábamos en condiciones de montar un fuerte y defendernos contra toda clase de enemigos. Desde un principio pensé que necesitaríamos aquello y aún más si esperábamos mantenernos en posesión de la isla, como se verá a lo largo de este relato.

    Vientos contrarios nos llevaron primero hacia el norte, y nos vimos precisados a refugiarnos en Galway, Irlanda, donde nos quedamos veintidós días. Como compensación por esta contrariedad encontramos que las provisiones eran allí extremadamente baratas y que abundaban muchísimo, de modo que mientras permanecimos en la rada no tocamos para nada las vituallas de a bordo y hasta las acrecentamos; compré allí varios cerdos, así como dos vacas y terneros, lo que confiaba en llevar vivos a mi isla si teníamos buen viaje; sin embargo, las circunstancias nos obligaron a disponer de ellos en otra forma.

    El 5 dé febrero salimos de Irlanda y durante varios días tuvimos buen viento. En la noche del 20 de febrero, según creo recordar, el segundo que estaba de guardia entró en la toldilla y nos dijo que acababa de ver un resplandor como de fuego, oyendo también el disparo de un cañón. Mientras nos narraba esto un grumete entró a decirnos que el contramaestre había oído otro disparo. Salimos corriendo al alcázar donde por un rato no oímos nada, pero minutos más tarde vimos una gran luz y comprendimos que había un terrible incendio a la distancia. A la media hora de navegar en aquella dirección, y como el viento nos impulsaba con fuerza aunque no excesivamente, descubrimos claramente que un gran navío se había incendiado en medio del mar.

    La vista de este desastre me conmovió profundamente, pese a que desconocía a los tripulantes de aquel barco. Recordé sin embargo mis antiguas aventuras, en qué triste condición fuera recogido del mar por el capitán portugués, y cuan peores podrían ser aún las desdichas de aquellas pobres gentes del navío a no encontrarse tan cerca el nuestro para auxiliarlas. Ordené de inmediato que se dispararan cinco cañonazos, uno después de otro, para tratar de avisarles que acudíamos en su auxilio y a fin de que se decidieran a embarcarse en la chalupa del navío; es preciso advertir que, en la oscuridad de la noche, aunque veíamos muy bien el navío incendiado, nosotros permanecíamos invisibles a su vista.

    Estuvimos un tiempo a la espera, manteniéndonos a igual distancia del barco y esperando la luz del día, cuando de improviso y para nuestro espanto —pese a que era de imaginarse que ello ocurriría— el buque voló en pedazos y pocos minutos más tarde el fuego se había extinguido. Triste espectáculo fue aquél, y en especial afligente por la suerte de los desdichados tripulantes que, según imaginamos, debían haber perecido todos o encontrarse en la peor de las situaciones, en medio del océano y a bordo de la chalupa; tan oscuro estaba que no alcanzábamos a ver nada. Traté sin embargo, de encaminarlos hacia nosotros si era posible, mandando encender luces en distintas partes del buque, y permanecimos toda la noche con linternas y disparando cañonazos a fin de que supieran que había socorro cercano para ellos.

    Con ayuda de los anteojos descubrimos a eso de las ocho de la mañana las chalupas del buque hundido; había dos de ellas, repletas de gente y casi zozobrando por el excesivo peso. El viento estaba en contra, pero remaban incesantemente en nuestra dirección, haciendo todo lo posible para que los viéramos.

    De inmediato enarbolamos el pabellón para tranquilizarlos, y pusimos una bandera a modo de señal para que se acercaran a nuestro barco, a la vez que desplegábamos más velas a fin de ir a su encuentro. Media hora más tarde estábamos a su lado y pronto tuvimos a todos a bordo, no menos de sesenta y cuatro hombres, mujeres y niños, pues entre ellos se contaba buen número de pasajeros.

    A través de su relato supimos que se trataba de un navío mercante francés de trescientas toneladas, que volvía a su patria procedente de Quebec, en el río del

    Canadá (1). El capitán nos hizo un detallado relato de la catástrofe de su navío, explicando cómo el incendio se inició en la antecámara por una negligencia del piloto; al comienzo habían creído dominar el fuego, después de los primeros gritos de socorro, cuando descubrieron que algunas chispas habían pegado fuego a otras partes del navío donde no era fácil llegar para sofocar los incendios; desde allí, deslizándose por entre el maderamen y tomando incremento en la parte inferior del puente y luego en la bodega, terminó por imponerse a todos los desesperados esfuerzos que se intentaron para dominarlo.

    Sólo les quedaba refugiarse en los botes, los que afortunadamente eran bastante grandes; tenían una lancha, una chalupa y un pequeño esquife que les sirvió solamente para embarcar en él algo de agua dulce y provisiones, una vez que se hubieron puesto a salvo del fuego. Pocas esperanzas tenían sin embargo de salvar la vida, abandonados en aquellos botes a tan enorme distancia de tierra; se consolaban pensando que por lo menos habían escapado de las llamas y que acaso algún navío anduviera por esas aguas y alcanzara a verlos. Tenían velas, remos y una brújula, y se disponían a poner rumbo nuevamente a Terranova, ya que el viento soplaba favorablemente.

    En medio de sus deliberaciones, cuando todos se sentían desesperados y prontos a la peor angustia, el capitán nos refirió con lágrimas en los ojos su maravillosa sorpresa al escuchar el sonido de un cañonazo, y luego otros cuatro, tal como yo había ordenado tirar durante la noche. Esto animó sus corazones, llevándoles la seguridad que justamente había deseado yo que recibieran, es decir, que un navío estaba en las inmediaciones listo para acudir en su socorro.

    Me sería imposible pintar aquí la diversidad de ademanes y gestos, los raptos de júbilo y la variedad de expresiones por las cuales aquellas pobres gentes así salvadas trataban de demostrarnos su alegría. Es fácil describir la pena y el miedo: suspiros, lágrimas, quejidos y unos pocos movimientos de las manos y la cabeza constituyen toda su variedad; pero un exceso de alegría, un arrebato de júbilo contiene en sí la posibilidad de mil extravagancias distintas. Algunos lloraban, otros se retorcían y desgarraban a sí mismos como si se encontraran en la peor agonía o aflicción; unos parecían enfurecidos o abiertamente lunáticos mientras otros corrían por el barco pisando con todas sus fuerzas o se retorcían las manos; los había que bailaban o cantaban, algunos riendo y muchos más llorando; no faltaban los que parecían alelados, incapaces de articular una palabra; vi a varios enfermos, vomitando o desmayándose; y por fin, unos pocos hacían el signo de la cruz y daban gracias al Señor.

    No quisiera ser injusto con ninguno de ellos. Sin duda hubo muchos otros que poco más tarde se sintieron también reconocidos por su salvación, pero en ese primer momento sus pasiones eran demasiado fuertes y no podían dominarlas suficientemente; parecían más bien lanzados al éxtasis o a una especie de delirio, y muy pocos mostraban una alegría más serena y compuesta.

    Tal vez contribuía a tan excesivas manifestaciones la nacionalidad de aquellos pasajeros; ya se sabe que los franceses poseen un carácter más voluble, apasionado y vivo, y que su ánimo es más ligero que el de otros pueblos. No sé bastante filosofía para determinar la causa de esta diferencia, pero nada de cuanto había visto hasta ahora podía compararse a aquello. Recordaba los arrebatos del pobre Viernes, mi fiel salvaje, cuando al llegar a la canoa halló en su interior a su padre, así como los transportes del capitán y sus dos compañeros cuando los libré de los villanos que iban a abandonarlos en la isla; sin duda sus arrebatos se habían asemejado a éstos, pero no resistían comparación posible, y nunca más vi transportes semejantes, ni en Viernes ni en nadie bajo ninguna circunstancia.

    Reparé asimismo que todas aquellas extravagancias no se mostraban del modo ya descrito en distintas personas, sino que la variedad entera se manifestaba sucesivamente y en pocos minutos en cualquiera de ellos. Un hombre que veíamos de improviso como atontado, dando una impresión de absoluta inconsciencia y atonía, se lanzaba al minuto siguiente a bailar y gritar como un bufón, para pasar momentos más tarde a desgarrar sus ropas o arrancarse los cabellos, pisoteando los jirones como un endemoniado; luego de eso rompía a llorar, luego palidecía de improviso y se desmayaba; por cierto que de no haberlo ayudado en tal trance pocos minutos más tarde hubiera muerto. Y esto no sucedía solamente con uno o dos, ni siquiera diez o veinte, sino con la gran mayoría de ellos, y si no estoy muy equivocado nuestro cirujano tuvo que sangrar a más de treinta.

    Entre los rescatados se contaban dos sacerdotes, uno de ellos ya anciano y el otro muy joven; lo extraordinario es que de los dos fue el anciano el que se condujo peor. El más joven demostró un gran dominio de sus pasiones, dando ejemplo de una mente controlada con firmeza. Apenas subido a bordo se dejó caer de rodillas, prosternándose en señal de gratitud por su liberación, y fue entonces que cometí la imprudencia desdichada de turbarlo en su éxtasis creyendo que estaba a punto de desvanecerse. Al verme acudir a él me habló serenamente, dándome las gracias y diciéndome que iba a agradecer al Señor por haberlos librado; me pidió que lo dejase solo por un momento, agregando que una vez que hubiera cumplido con su Hacedor acudiría a mí para darme también las gracias.

    Lamenté profundamente haberlo interrumpido de ese modo, y no sólo lo dejé aparte sino que mandé a los demás que no se acercaran. Durante tres minutos o más continuó en la misma postura, y luego se levantó y vino a mí tal como lo había prometido. Con la misma calma de antes, pero profundamente conmovido y llenos los ojos de lágrimas, me agradeció que hubiera sido el instrumento divino que salvara de la muerte a él y a esas desdichadas criaturas.

    Tras eso, el joven sacerdote se dedicó a calmar a sus compatriotas, esforzándose por volverlos a la realidad; trató de persuadirlos, de convencerlos por medio de razonamientos y súplicas, haciendo todo cuanto estaba a su alcance para que recobraran el uso de la razón; logró buen éxito con muchos, aunque otros estuvieron durante bastante tiempo más allá de todo dominio de sí mismos.

    No he podido menos de hacer estas descripciones que acaso sean útiles a aquellos a cuyas manos vayan, enseñándoles a dominarse en todos los excesos de sus pasiones; pues si un exceso de júbilo puede arrastrar a los hombres hasta tal punto más allá de la razón, ¿cuáles no serán las extravagancias del odio, la cólera y la irritación? En aquella oportunidad vi motivo suficiente para mantener una constante vigilancia sobre todos nuestros impulsos, tanto los derivados de la alegría y la satisfacción como los causados por la angustia o el resentimiento.

    Durante el primer día nos desconcertaron un poco los arrebatos de nuestros nuevos huéspedes, pero cuando se hubieran retirado a los alojamientos que les preparamos con toda la comodidad que el buque permitía, y hubieran dormido profundamente, encontramos al otro día que eran ya personas muy distintas.

    Ninguna de las amabilidades y finezas por las cuales pudieran expresarnos su reconocimiento faltó entonces. Ya es sabido que los franceses son naturalmente aptos para destacarse en materia de cortesía. El capitán y uno de los sacerdotes vinieron a hablar conmigo y mi sobrino el capitán, queriendo consultarnos sobre lo que decidiríamos acerca de ellos. Principiaron por manifestarnos que les habíamos salvado la vida y que todo cuanto tenían era insignificante para retribuir nuestras bondades. El capitán manifestó que habían conseguido salvar algún dinero y objetos de valor en las chalupas, arrebatándolos a último momento de las llamas, y que si aceptábamos esos bienes estaban autorizados a ofrecérnoslos en nombre de todos. Su único deseo era ser llevados a tierra en algún punto de nuestro itinerario desde donde, a ser posible, pudieran lograr medios para retornar a Francia.

    Mi sobrino fue de opinión de aceptar en primera instancia el dinero y decidir luego el destino de aquellas gentes, pero yo lo disuadí en esta parte, pues sabía bien lo que significaba ser dejado en tierra y en país extraño. Si el capitán portugués que me recogió en el mar hubiera hecho eso conmigo, aceptando todo cuanto yo tenía por pago de mi salvación hubiese muerto más tarde de hambre o vivido en el Brasil como un esclavo, al igual que antes en Berbería, con la única diferencia de no pertenecer a un mahometano; por cierto que un portugués no es mejor amo que un turco, y a veces mucho peor.

    Dije por lo tanto al capitán francés que si los habíamos librado de su afligente situación, habíamos cumplido solamente con nuestro deber de semejantes, haciendo lo que hubiésemos esperado a nuestro turno de encontrarnos en una situación como la suya; no dudábamos por otra parte que de haberse cambiado los papeles ellos hubiesen obrado del mismo modo con nosotros, y que nuestra intención había sido la de salvarlos y no someterlos a un saqueo; por lo tanto no estaba dispuesto a permitir que la menor cosa les fuera quitada a bordo. En cuanto a dejarlos en tierra, admití que para nosotros constituía una gran dificultad ya que nuestro navío estaba destinado a las Indias Orientales; mi sobrino el capitán no podía obrar de modo distinto del dispuesto por los que fletaban el navío, con los cuales estaba comprometido por contrato a seguir viaje a Brasil. Todo cuanto podíamos hacer era mantenernos en el rumbo de aquellos navíos que retornaran a la patria viniendo de las Indias Occidentales y obtener para ellos pasaje con destino a Inglaterra o Francia.

    La primera parte de esta propuesta era tan amable y generosa que mis interlocutores se manifestaron profundamente agradecidos, pero cuando oyeron el resto cayeron en una gran consternación, en especial los pasajeros que advertían que los llevaríamos a las Indias Orientales; me suplicaron entonces que, en vista de que ya habíamos sido llevados hacia el oeste antes de nuestro encuentro, por lo menos mantuviéramos el rumbo hasta llegar a los bancos de Terranova, donde era posible dar con algún navío o balandro que se prestara a embarcarlos de regreso a Canadá, de donde procedían. Pensé que la súplica era harto razonable, y por lo tanto accedí a ella con toda buena voluntad, reparando en que llevar semejante número de personas a las Indias Orientales sería no sólo intolerable desgracia para los desdichados, sino que además nuestro viaje resultaría una ruina por el consumo obligado de todas las provisiones. Preferí entonces considerar lo ocurrido no como una violación del contrato sino un accidente imprevisto que nos ponía en situación de proceder como se ha dicho, de lo cual nadie era en lo más mínimo culpable. Las leyes de Dios y de la Naturaleza hubieran prohibido que negásemos acceso a bordo a chalupas llenas de gente en tan horrible situación, y la naturaleza de lo sucedido, tanto desde el punto de vista de los náufragos como de nuestra conveniencia, nos obligaba a desembarcarlos en un sitio u otro para su total liberación. Consentí, pues,  que hiciéramos rumbo a Terranova, si el tiempo y el viento lo permitían; de lo contrario los llevaría a la Martinica,  en las  Indias Occidentales.

    Una semana más tarde alcanzamos los bancos de Terranova; para abreviar mi relato diré que transbordamos a todos nuestros franceses a un navío que ellos contrataron a fin de que los llevase primero a tierra y más tarde a Francia, si conseguían suficientes provisiones para avituallarse en el viaje. Cuando digo que todos los franceses desembarcaron debo señalar que el joven sacerdote de quien ya he hablado, sabedor de que íbamos rumbo a las Indias Orientales, sintió deseos de hacer con nosotros el viaje y desembarcar en la costa de Coromandel, cosa a la que accedí de inmediato porque aquel hombre me parecía admirable; y no me equivocaba, según pude comprobarlo más adelante. Otros cuatro franceses, marineros, entraron al servicio de nuestro navío y fueron excelentes y muy útiles.

    Desde allí hicimos rumbo directamente hacia las Indias Occidentales, siguiendo una dirección S. y S.E. durante veinte días, en los que encontramos con frecuencia calma chicha, hasta que al fin dimos con otra ocasión para que nuestro sentido humanitario pudiera ejercitarse, en circunstancias casi tan deplorables como las que termino de narrar.

    Se trataba de un barco de Bristol que debía volver a la patria desde las Barbadas, pero que había sido arrebatado por un furioso huracán de la rada de las islas cuando le faltaban todavía unos días para estar en condiciones de hacerse a la mar; al producirse la catástrofe, el capitán y el segundo se encontraban en tierra, de manera que los tripulantes, además de estar atemorizados, carecían de hombres capaces de dirigir la maniobra de retorno al puerto. Llevaban ya nueve semanas en alta mar, y apenas concluido el primer huracán fueron arrebatados por una nueva y furiosa borrasca que los arrastró hacia el oeste sin que pudieran tener idea alguna del rumbo que llevaban, y en el transcurso de la cual perdieron sus mástiles. Lo peor de todo era que estaban ya medio muertos de hambre por falta de vituallas, aparte de las constantes fatigas sufridas. No tenían galleta ni carne y llevaban once días sin probar ni una onza de esos alimentos. Su único alivio era que todavía les quedaba agua, así como medio barril de harina; también tenían azúcar, pero las frutas en jarabe y los dulces se les habían terminado hacía mucho; disponían aún de siete barriles de ron.

    Viajaban a bordo un jovencito, con su madre y una criada, que habían tomado pasaje en el navío y creyendo infortunadamente que estaba pronto para zarpar acudieron a embarcarse justamente la noche antes de que se desatara el huracán. Careciendo de provisiones propias, rápidamente consumidas, aquellos tres infelices se encontraban en una situación aún peor que la del resto, pues los marineros, reducidos a tan extrema necesidad, no habían tenido compasión alguna con los pasajeros, y apenas puedo describir el grado de inanición en que se hallaban.

    Quizá no hubiera llegado a saberlo, con todo, si la curiosidad no me hubiera impulsado a trasladarme a bordo aprovechando el buen tiempo y la calma del mar. El segundo piloto, que comandaba el barco en la emergencia y había pasado a bordo del nuestro, me dijo que quedaban tres pasajeros en la cámara de popa, los cuales se encontraban en una deplorable condición.

    —Hasta creo que deben haber muerto —agregó— porque hace más de dos días que no los oímos, y yo no me atrevía a llamarlos, pues en nada podía aliviar su desgracia.

    De inmediato tratamos de brindarles todo el socorro de que disponíamos; convencí a mi sobrino de que debíamos darles todas las provisiones que pudieran necesitar, aunque más tarde tuviéramos que poner rumbo a Virginia o a otro punto de la costa americana para avituallarnos a nuestro turno; pero no hubo necesidad de llegar a eso.

    Aquellos hombres enfrentaban ahora un nuevo peligro, y era el de comer demasiado de una vez, pese a las pequeñas porciones que les dábamos. El segundo, o capitán del barco, había acudido a nuestro navío con seis de sus hombres, pero aquellos desdichados parecían más bien esqueletos y estaban tan débiles que apenas podían mover los remos. También el segundo se sentía enfermo y medio muerto de hambre, y supimos que no había reservado raciones aparte de las de sus hombres, compartiendo lo que había de igual a igual.

    Le aconsejé que comiera poco, luego de hacerle traer alimentos, pero apenas había tomado dos o tres bocados cuando se sintió terriblemente mal; dejó entonces de comer, mientras nuestro cirujano mezclaba algo en una ración de caldo, asegurando que ello le serviría de alimento y purgante al mismo tiempo; apenas lo hubo bebido se sintió mejor. No olvidaba yo entretanto a los demás hombres; ordené que se les diera de comer y los desdichados devoraron más que comieron, ya que el hambre era tal que habían perdido el control de sí mismos y parecían como rabiosos; por cierto que dos se excedieron tanto que estuvieron a punto de morir a la mañana siguiente.

    Todo el tiempo que el segundo empleó en relatarnos los tristes acontecimientos ocurridos en su barco, no dejé de pensar en lo que me había dicho sobre aquellos infelices pasajeros encerrados en el camarote de popa, es decir, una madre con su hijo y una sirvienta; el segundo afirmaba no haberlos oído desde hacía dos o tres días, y en cierto modo confesó que habían olvidado por completo a sus pasajeros, absorbidos en las propias tribulaciones. De ahí deduje que en realidad no les habían dado ningún alimento, y que probablemente habrían muerto ya de hambre, por lo cual solamente encontraría sus cadáveres en la cabina.

    Mientras tuvimos al segundo —a quien llamábamos capitán— a bordo con sus hombres para que se alimentaran, no olvidamos al resto de la desfalleciente tripulación que había quedado en el otro barco, tanto que ordené enviar mi propia lancha tripulada por mi segundo y doce hombres que llevarían un saco de galleta, a más de cuatro o cinco trozos de carne para hervir. Nuestro cirujano encargó especialmente que la carne fuese hervida sin darla antes a nadie, y que se montara guardia en la cocina para impedir que aquellos infelices se apoderaran de ella para comerla cruda o antes de que estuviese en su punto; dijo asimismo que a cada hombre debía dársele cada vez una porción muy pequeña, y en esa forma consiguió salvar a los marineros, que de otra manera hubieran perecido a causa de lo mismo que se les ofrecía para recobrar las fuerzas.

    Ordené al segundo que fuera a la cabina de popa para averiguar en qué estado se encontraban los pasajeros, por si quedaba todavía la posibilidad de prestarles auxilio y ofrecerles ayuda de todo género; el cirujano le entregó una olla del mismo caldo que había hecho tomar al segundo que estaba en nuestro barco, y que según su parecer debía restablecerlos gradualmente. Con todo no me sentí satisfecho, y sintiendo, como he dicho más arriba, un gran deseo de ver por mí mismo el escenario de aquel drama, y seguro de que en persona tendría una impresión mucho más nítida que la que pudiera lograr por relatos de mis hombres, llamé al capitán del otro barco y después de un rato nos fuimos en su bote.

    Encontré a los infelices marineros frenéticos en su ansia por comer la carne antes de que estuviera bien hervida, pero mi segundo había cumplido las órdenes y encontré una sólida guardia a la entrada de la cocina. El hombre que allí estaba después de emplear la persuasión hasta el cansancio, había tenido que emplear la fuerza para alejar a la tripulación. Sin embargo se le ocurrió remojar algunos bizcochos en el caldo de la carne, haciendo unas sopas que repartió a los hombres para que confortaran poco a poco su estómago, mientras les decía que les daba pequeñas porciones para su propio bien.

    Pero el infortunio de los pasajeros del camarote era de distinta naturaleza y harto peor que el de los marineros. Desde un comienzo la tripulación había tenido tan pocas provisiones que la ración dada a los pasajeros fue muy pequeña y cesó totalmente pocos días más tarde, tanto que en los últimos seis o siete días nada habían tenido que comer y en los días anteriores apenas una mínima cantidad. La desdichada madre, que según decían los hombres, era una mujer de excelente cuna y llena de sensatez, trató de reservar las raciones para su hijo hasta que al fin había cedido al desfallecimiento natural; cuando mi segundo entró en el camarote estaba sentada en el suelo, entre dos sillas que allí aparecían atadas, con la espalda apoyada en el tabique y la cabeza hundida entre los hombros, con todo el aspecto de un cadáver, aunque no había muerto todavía. Mi segundo hizo todo lo posible por reanimarla y darle fuerzas, tratando de que bebiera algo de caldo con ayuda de una cuchara. Abrió los labios y levantó una mano, intentando expresar algo, lo que no consiguió, mas al entender lo que el segundo le decía hizo débiles señales queriendo significar que ya era demasiado tarde para ella, pero señalaba a la vez en dirección a su hijo como recomendándoles que se preocuparan solamente por él.

    El segundo, profundamente emocionado ante esa prueba de cariño, insistió en hacerle beber algo de caldo, y según creía alcanzó a tragar dos o tres cucharadas, aunque dudo si tenía la seguridad de ello; de todas maneras el auxilio resultó tardío, y la madre murió aquella misma noche.

    El muchacho, salvado a costa de la vida de su amante madre, no parecía tan desfalleciente, pero sin embargo estaba tendido en el lecho del camarote con todo el aire de un moribundo. Tenía en la mano un pedazo de guante, cuyo resto había devorado. Tan joven, y con más vigor que su madre, bastó que el segundo le hiciera tragar algo de líquido para que de inmediato principiara a revivir; sin embargo, cuando momentos después le dieron a beber otras dos o tres cucharadas de caldo, se sintió muy mal y las devolvió sin tolerarlas.

    La infeliz doncella llamó entonces la atención del segundo. Yacía tendida en el suelo casi al lado de su cama y daba la impresión de alguien que ha sufrido un ataque de apoplejía y lucha por conservar su vida. Tenía las piernas contraídas, y con una mano aferraba fuertemente el marco de una silla, de tal modo que no fue fácil desprenderla de allí. El otro brazo formaba un arco sobre su cabeza y tenía los pies apretados contra una mesa. En una palabra, yacía como alguien que ha sufrido la agonía postrera; y sin embargo aún estaba viva.

    La desdichada criatura no solamente se hallaba reducida a la peor inanición y llena de terror a la idea de la muerte, sino que supimos más tarde por los marineros que su corazón quedó desgarrado a la vista de los sufrimientos de su ama, a la que había visto moribunda durante los últimos dos o tres días, y a la que amaba tiernamente.

    No sabíamos qué hacer con aquella pobre muchacha, ya que cuando nuestro cirujano, hombre de gran conocimiento y experiencia, la hubo salvado poco a poco de la muerte, encontró que su razón había cedido paso a un estado vecino a la locura, el que se prolongó por un tiempo considerable, como se verá más tarde.

    El que lea estas memorias habrá de tener en consideración que las visitas en alta mar de un buque a otro no se parecen en nada a las que pueden hacerse en tierra firme, donde los visitantes suelen quedarse a veces por una semana o quince días en un mismo sitio. Nuestra tarea consistía en socorrer a los tripulantes de aquel navío pero no quedarnos a su lado, y aunque ellos se mostraron deseosos de seguir nuestro rumbo durante algunos días, no podíamos retardar nuestro viaje esperando a un barco que carecía de mástiles. El capitán nos rogó, sin embargo, que los ayudáramos a levantar un mastelero en un improvisado palo de trinquete. Ocupados en esta tarea permanecimos tres o cuatro días, entregando a aquellos hombres cinco barriles de carne salada, uno de tocino, dos sacos de galleta y una cantidad adecuada de guisantes, harina y otras cosas que podíamos cederles. Aceptamos en cambio tres barriles de azúcar, algo de ron y algunas piezas de a ocho, y nos separamos de ellos llevando a bordo al jovencito, la sirvienta y todos sus efectos.

    El muchacho, que contaba unos diecisiete años, era bien parecido, de una excelente educación, modesto y juicioso. Estaba abrumado por la pérdida de su madre, y según supimos había perdido a su padre unos meses antes en las Barbadas. Pidió al cirujano que lo atendía que intercediese ante mí para recibirlo en mi barco, porque según decía aquellos crueles marinos habían asesinado a su madre. Tenía razón, ya que con su egoísmo habían ocasionado la muerte de la viuda, pues reservándole solamente una pequeña ración hubiesen conseguido mantenerla viva hasta la llegada de socorro. Desgraciadamente el hambre no conoce amigos ni parientes, ignora la justicia y el derecho, y es tan incapaz de remordimientos como de compasión.

    El cirujano le advirtió lo prolongado de nuestro viaje, y cómo la travesía iba a alejarlo de sus amigos y tal vez sumirlo en desgracias tan grandes como aquella de la cual acabábamos de rescatarlo, es decir, morirse de hambre en algún lugar del mundo. Pero respondió que lo tenía sin cuidado el sitio adonde lo lleváramos con tal de sentirse apartado de esa odiosa tripulación en cuya compañía había tenido que vivir. Agregó que el capitán (se refería a mí, pues nada sabía de mi sobrino) le había salvado la vida, y él estaba seguro que en nada iba a perjudicarlo; en cuanto a la doncella, suponiendo que volviera a recobrar la razón, se sentiría harto satisfecha de que la llevásemos a bordo de nuestro navío.






	\chapter{Una colonia turbulenta}





    No fatigaré al lector con los menudos incidentes ocurridos en el resto del viaje y que se refieren al tiempo, vientos y corrientes; abreviando mi relato en homenaje a lo que va a seguir, diré que arribamos a mi antigua residencia, la isla, el 10 de abril de 1695. No fue sin dificultad que alcancé a reconocer aquella tierra, pues como antes había llegado y salido de ella por el lado austral y oriental, proveniente del Brasil, arribando ahora por el lado que da al océano y sin mapa que señalara en modo alguno su situación, apenas la reconocí al verla y hasta dudaba si aquella tierra era o no mi isla.

    Anduvimos errando un buen tiempo por las inmediaciones y desembarcamos en varias islas que se encuentran en las bocas del gran río Orinoco sin dar con la mía; aquello me sirvió sin embargo, para advertir el error en que había estado al creer que desde mi isla alcanzaba a divisar el continente, cuando en realidad sólo apenas percibía una gran isla o mejor una cadena de islas que forman como un abanico en las bocas del gran río. En cuanto a los salvajes que desembarcaban en la isla, no eran precisamente los caribes, sino isleños y otros bárbaros de la misma clase, que vivían algo más próximos a nuestro lado que el resto de las tribus.

    En resumen, visité varias de aquellas islas sin resultado alguno; vi que muchas estaban deshabitadas y otras no; por fin, costeando de una a otra, a veces con el barco y en algunas ocasiones con la chalupa francesa (que nos había parecido una excelente embarcación y habíamos conservado con el permiso del capitán), por fin alcanzamos el lado sur de mi isla y de inmediato reconocí la fisonomía de la costa, de manera que pude dirigir el navío y hacerlo anclar con toda seguridad en las proximidades de la pequeña ensenada donde estaba mi antigua vivienda.

    Tan pronto como reconocí el lugar llamé a Viernes y le pregunté si sabía dónde estábamos. Miró en torno, y luego golpeó las manos.

    — ¡Oh, sí! —exclamó vivamente—. ¡Oh, sí, oh, allí!

    Señalaba el emplazamiento de nuestra casa, y se puso a bailar y a saltar como un enloquecido; me costó bastante trabajo impedirle que se tirara de un salto al mar y fuera nadando hasta la costa.

    —Bueno, Viernes —le dije—. ¿Crees o no que encontraremos a alguien allí? ¿Te parece que veremos a tu padre?

    El muchacho se quedó de pronto tieso como un tronco, y después de oír el nombre de su padre pareció llenarse de aflicción; vi que las lágrimas rodaban por las mejillas de mi pobre y cariñoso compañero.

    — ¿Qué te ocurre, Viernes? —pregunté—. ¿Es que lamentas ver otra vez a tu padre?

    —No, no —murmuró sacudiendo la cabeza—. Yo no verlo más, no verlo nunca más.

    — ¿Por qué, Viernes? ¿Cómo sabes que no lo verás más?

    — ¡Oh, no! ¡Oh, no! —insistió él—. El morir hace mucho, el morir ya, hombre muy anciano.

    —Vamos, vamos —le dije—. Tú no puedes saberlo, Viernes. Dime, ¿crees que no veremos a nadie allí?

    El muchacho tenía por lo visto mejores ojos que yo, pues señalando la colina justamente encima de donde se hallaba mi morada, aunque estábamos a más de media legua de distancia, exclamó:

    — ¡Ver, ver! ¡Sí, ver muchos hombres allí, y allí, y allí!

    Aunque traté de descubrirlos con ayuda de un anteojo, no pude divisar a nadie, probablemente porque no alcanzaba a precisar el sitio exacto. Viernes, sin embargo, estaba en lo cierto, como lo averigüé al día siguiente, pues cinco o seis hombres se habían encaramado a aquel sitio para observar nuestro buque, sin saber todavía qué pensar de nosotros.

    Tan pronto Viernes me aseguró que había visto gente ordené que se desplegara la bandera y se disparara una salva de tres cañonazos, para indicar que éramos amigos. Un cuarto de hora más tarde percibimos una columna de humo que se alzaba en la región de la ensenada. Ordené de inmediato arriar un bote en el cual me embarqué con Viernes, y alzando bandera blanca en señal de parlamento me encaminé directamente a tierra, llevando conmigo al joven sacerdote, del cual ya he hablado y a quien había descrito las incidencias de toda mi vida en la isla, así como diversos detalles acerca de mí y de las gentes que dejara en esas tierras, por lo cual se manifestaba deseoso de acompañarme en mi desembarco. Llevábamos con nosotros a dieciséis hombres bien armados por si encontrábamos en la isla huéspedes inesperados; pero no tuvimos necesidad alguna de armas.

    Como llegamos a la costa cuando estaba en ascenso la marea, nos internamos directamente en la ensenada. El primer hombre que reconocí fue el español cuya vida había salvado y cuyo rostro recordaba perfectamente; en cuanto a su vestimenta, la describiré más adelante. Ordené que nadie se moviera en el bote, pues quería desembarcar solo. Sin embargo, no hubo manera de tener quieto a Viernes, pues el excelente muchacho había descubierto a su padre a cierta distancia de donde estaban los españoles, sin que yo me percatara de su presencia; y estoy seguro de que si no lo hubiesen dejado saltar del bote se hubiera precipitado de un brinco al mar. Tan pronto estuvo en tierra voló hacia su padre como una flecha recién disparada del arco; hubiera arrancado lágrimas al hombre más empedernido contemplar los primeros transportes de' alegría del pobre muchacho cuando llegó junto a su padre; lo abrazó, besándolo y haciéndole caricias en el rostro, lo levantó en sus brazos y lo hizo sentar en un tronco, y él se puso a su lado. Luego, levantándose, se dedicó a contemplarlo como alguien que mira un cuadro extraordinario, quedándose extático por un cuarto de hora; luego se dejó caer al suelo, abrazando las piernas del anciano y besándolas, y se levantó en seguida para seguir contemplándolo como si estuviese repentinamente embrujado. Y esto no es nada en comparación a los extremos a que sus sentimientos lo llevaron al día siguiente, y que hubieran hecho sonreír a cualquiera. Toda la mañana anduvo Viernes paseando por la playa con su padre, a quien llevaba tomado de la mano como si hubiese sido una dama; a cada momento corría hasta el bote para buscar algo que pudiese agradarle, tal como un terrón de azúcar, un trago de licor o una galleta. Por la tarde su extravagancia se mostró en otra forma, pues dejando sentado al anciano en el suelo empezó a bailar en torno suyo haciendo mil gestos raros y adoptando las más fantásticas posturas; todo el tiempo que duró esto seguía hablando a su padre, contándole para divertirlo incidentes de sus viajes, y todo cuanto le había sucedido en países lejanos. En suma, si de nuestro lado del mundo se encontrase entre los cristianos la misma afección filial, uno se sentiría tentado a declarar inútil el quinto mandamiento divino.

    Pero volvamos, después de esta disgresión, a nuestro desembarco. Sería imposible describir todas las ceremonias y amabilidades con que los españoles me recibieron. Ya he dicho que el primer español que reconocí, por recordar muy bien su rostro, era aquel cuya vida había salvado. Vino hacia el bote acompañado por otro y trayendo también una bandera de parlamento. Pero no sólo no me reconoció al principio, sino que estaba totalmente ajeno a la idea de que pudiera ser yo quien venía a su lado para hablarle.

    —Señor —le dije entonces en portugués—, ¿no me reconocéis?

    Al oír esto no pronunció una sola palabra, pero entregando su mosquete al hombre que lo acompañaba, tendió los brazos abiertos y murmurando una frase en español que no alcancé a comprender corrió a mí y me abrazó con fuerza, reprochándose amargamente no haber reconocido un rostro que antaño se le apareciera como el de un ángel venido para librarlo. Pronunció infinidad de bellas frases, como los españoles bien educados saben hacerlo siempre; y luego volviéndose a quien lo acompañaba le ordenó que fuese a llamar a los restantes camaradas. Me preguntó si deseaba volver a mi vieja morada, de la cual se apresuraba a devolverme la posesión y donde podría observar que sólo se habían hecho pocas mejoras. Eché pues a andar a su lado, pero grande fue mi asombro al no poder encontrar el sitio, como si jamás hubiese estado allí; los españoles habían plantado tantos árboles, y colocado tan cerca uno del otro, que en diez años aquello se había convertido en un espeso bosque que tornaba el lugar inaccesible, salvo por estrechos pasajes que solamente conocían los habitantes.

    Le pregunté entonces qué razón habían tenido para aumentar de tal manera las fortificaciones, y me contestó que ya vería yo la necesidad de aquellas obras cuando me hubiera narrado detalladamente lo que les había sucedido desde su llegada a la isla, en especial cuando se encontraron con la desgracia de saberme ausente del lugar. Me dijo que había sentido alegría al enterarse de mi buena fortuna al conseguir embarcar en un navío a mi entera satisfacción, pero que muchas veces tuvo el presentimiento de que alguna vez volvería a verme; me confesó que nada de cuanto le ocurriera en toda su vida había sido capaz de sumirlo en tan grande aflicción y angustia que arribar a la isla para encontrarse con que yo no estaba ya en ella.

    En cuanto a los tres bárbaros (como él les llamaba) que habíamos dejado en la isla, y de los cuales me aseguró que tenía un largo relato que hacerme, los españoles hubieran preferido seguir viviendo entre los salvajes que con ellos, salvo que su número era mucho menor. —Por cierto —agregó— que si hubiesen sido más, hace rato que estaríamos en el purgatorio.

    Y se persignó al decir esto.

    —De modo, señor —prosiguió—, espero que no os desagradará escuchar el relato que os haré contándoos cómo nos fue necesario desarmar, en defensa de nuestra vidas, a aquellos individuos para someterlos a nuestra ley, ya que de lo contrario no solamente hubieran sido nuestros amos, sino nuestros asesinos.

    Contesté que había temido mucho que eso ocurriera, y que si algo había lamentado al dejar la isla era precisamente que ellos no estuviesen ya de regreso para ponerlos primero en posesión de todos mis efectos y dejar a los tres amotinados en una situación de servidumbre como la que merecían. En fin, si las cosas habían terminado por tomar ese cariz, me alegraba de saberlo y estaba muy lejos de hacerle el menor reproche, ya que sabía bien que aquellos tres individuos eran díscolos e ingobernables picaros, dispuestos siempre a las peores artimañas.

    Mientras hablábamos volvió el hombre que el español enviara con su orden acompañado de otros once españoles. En el estado en que se encontraban hubiese sido difícil averiguar a qué nación pertenecían, pero mi amigo se encargó de aclarar las cosas tanto para ellos como para mí. Señalándolos con la mano, me dijo:

    —Estos, señor, son algunos de los caballeros que os deben la vida.

    Luego, volviéndose a sus camaradas, les habló contándoles quién era yo, y entonces fueron avanzando de uno en uno hasta donde me encontraba no como simples marineros o personas ordinarias, sino con todo el aspecto de embajadores o nobles que se dirigen a un monarca o a un gran conquistador. Su comportamiento era cortés y fino en sumo grado, pero esa conducta estaba empapada de una masculina gravedad que sentaba muy bien a sus personas; en resumen, me superaban tanto en sus maneras que no sabía yo cómo recibir sus atenciones y mucho menos el modo de retribuirlas.

    La historia de su llegada e instalación en la isla después de mi partida es tan interesante y está tan llena de incidentes que la primera parte de mi relato ayudará a comprender mejor —ya que muchos de sus episodios se refieren a los sucesos allí relatados— que no puedo dejar de transmitirla con verdadero gusto a aquellos que la leerán alguna vez.

    Lo primero que yo pregunté —y que sirve para continuar el relato donde yo lo dejara— fueron los detalles de la travesía, y pedí al español que me hiciera una descripción precisa de su viaje en la canoa hasta el sitio donde vivían sus compatriotas. Me explicó que la travesía había tenido poco de interesante, pues el tiempo se mantuvo en calma y el mar muy sereno; en cuanto a sus compañeros, está de más decir la alegría que experimentaron al verlo regresar sano y salvo. (Parece que en ese entonces era el principal entre ellos, pues el capitán del buque en el cual naufragaran había muerto tiempo atrás.) Me relató la gran sorpresa que tuvieron al verlo volver, ya que, enterados de que había caído en manos de los salvajes descontaban que habría sido devorado al igual que los restantes prisioneros. Cuando les contó la manera en que había sido salvado de la muerte, y la esperanza de liberación que les traía por mi encargo, les pareció que estaban soñando; no volvieron a la realidad hasta que les hizo ver las armas, la pólvora y las balas, así como las provisiones que había traído para la travesía, y compartiendo entonces el júbilo de la liberación se apresuraron a disponerse para el viaje.

    El primer problema fue procurarse canoas; aquí no pudieron los españoles detenerse demasiado en consideraciones de honestidad, sino que traicionando la amistad de los salvajes que con ellos se mostraban pacíficos, les pidieron dos grandes canoas con el pretexto de ir de pesca o de paseo.

    A la mañana siguiente zarparon en ellas. Parece que no malgastaron tiempo en alistarse mayormente ya que no poseían equipaje, ni ropas, ni provisiones; desposeídos de todo, llevaban sólo lo que tenían puesto y unas raíces de las cuales se alimentaban a manera de pan. Tardaron en total tres semanas hasta arribar a la isla, y para desgracia de ellos, yo había tenido en ese plazo la oportunidad de recobrar mi libertad y salir de allí, como he mencionado en la primera parte de mi relato; en tierra sólo habían quedado tres de los más desaforados, empedernidos, díscolos y perversos villanos que jamás pudiera encontrar hombre alguno en su camino, de modo que es de imaginarse el desencanto y la angustia de los españoles al encontrarlos en mi lugar.

    Lo único decente que aquellos malvados hicieron al ver a los españoles fue entregarles mi carta y cumplir mi orden de darles algunas provisiones y utensilios, así como un papel conteniendo instrucciones que había dejado para que pudiesen continuar la misma vida que yo llevara hasta entonces en la isla; allí les explicaba cómo cocía mi pan, les enseñaba a domesticar cabras y mantener la plantación; también les decía el modo de secar las uvas, hacer alfarería y, en una palabra, todo lo que yo sabía. Entregaron dicho mensaje a los españoles, dos de los cuales comprendían bastante el inglés, y no se rehusaron a aceptar la compañía de los recién llegados, con los cuales convivieron en buenos términos por algún tiempo. Les concedieron igual derecho que los suyos en lo que se refiere a la casa y la caverna, y principiaron a vivir en excelente sociedad. El español que hacía de jefe, y que juntamente con el padre de Viernes me había visto trabajar y dirigir mis posesiones, fue considerado el comandante entre ellos, y en lo que a los ingleses se refiere no hacían otra cosa que vagabundear por la isla cazando papagayos y tortugas, volviendo a la casa al anochecer donde los españoles les tenían ya lista la cena.

    Por cierto que éstos se habrían conformado con esa situación si los otros los hubiesen dejado en paz, pero aquellos malvados eran como el perro del hortelano, que no quería comer ni dejar a los otros que comiesen. Al principio las discrepancias fueron triviales y no merecen mencionarse, pero crecieron poco a poco hasta convertirse en una guerra manifiesta, principiada por los ingleses con toda la insolencia y el desenfado imaginables, sin razón ni provocación, contraria a las leyes de la naturaleza y del sentido común. En verdad, aunque el primer relato de lo ocurrido me vino de los labios de los españoles mismos —a quienes podría llamar los acusadores—, más tarde no pude lograr que los otros desmintiesen una sola palabra de las que acababa de escuchar.

    Pero antes de que entre en detalles sobre este sucedido, debo llenar una laguna existente en la primera parte de mi historia. Había olvidado decir que cuando nos disponíamos a levar anclas después que fui rescatado de mi soledad, ocurrió a bordo un pequeño incidente que me hizo temer la posibilidad de un segundo motín. La pelea no fue apaciguada hasta que el capitán, empleando todo su coraje y cuanta ayuda disponía, separó por fuerza a los que contendían y apresando a dos de los marineros más díscolos los hizo encadenar inmediatamente. Esos dos hombres habían participado en el primer motín, y como acababan ahora de pronunciar algunas palabras sediciosas, el capitán los amenazó con llevarlos a Inglaterra encadenados como estaban y hacerlos colgar allí por rebeldes y por haberse apoderado en una oportunidad del barco.

    Parece que el capitán no pensaba poner en ejecución sus amenazas, pero el resto de la tripulación se asustó sobremanera al oírlas; algunos habían insinuado a los restantes que el capitán fingía hacer la paz con ellos para llevarlos así engañados a Inglaterra, donde se apresuraría a denunciarlos ante los tribunales y hacerlos juzgar por el motín anterior.

    El segundo se enteró de todo esto y nos lo hizo saber, y se resolvió entonces que yo, considerado a bordo como un personaje de gran importancia, fuera en compañía del segundo a hablar con los hombres y les diera plena garantía de que si se portaban bien en el resto del viaje todo cuanto pudieran haber cometido anteriormente les sería perdonado. Así lo hice, y después de comprometer en ese sentido mi palabra de honor conseguí que se calmaran completamente, en especial cuando ordené que los dos hombres encadenados fuesen puestos al punto en libertad.

    Todo este episodio nos había mantenido durante la noche en el mismo sitio y sin levar anclas, estando además el viento muy débil. A la mañana siguiente descubrimos que los dos hombres que fueran antes encadenados habían robado un par de mosquetes y otras armas, y embarcándose en la pinaza del buque, todavía no subida a bordo, habían escapado para reunirse en tierra con sus compañeros de fechorías.

    Tan pronto comprendimos lo ocurrido ordené que la lancha fuese a tierra tripulada por doce hombres y el segundo, pero aunque buscaron prolijamente a aquellos bandidos no pudieron dar ni con ellos ni con los otros tres; es evidente que al ver acercarse la lancha habían buscado refugio en los bosques. Como castigo de tanta villanía, el segundo parecía dispuesto a destruir la plantación y quemar la casa así como todas las instalaciones, para dejar a aquellos miserables sin auxilio alguno; pero no tenía órdenes para ello, y volvió a embarcarse sin tocar nada.

    El total de aquellos individuos era, pues, de cinco; pero ocurrió que los tres primeros eran todavía más malvados que los otros dos, y después de vivir juntos un par de días, terminaron por arrojar de su lado a los recién venidos y dejarlos librados a sus propios recursos, no queriendo tener con ellos nada en común; ni siquiera, durante bastante tiempo, se dignaron darles algún alimento. Todo esto ocurría cuando los españoles estaban todavía lejos de la isla.

    Al llegar éstos a tierra, las cosas principiaron a complicarse aún más. Los españoles trataron de persuadir a los tres miserables ingleses que aceptaran la compañía de los otros dos, a fin de constituir entre todos una sola familia, pero aquéllos ni siquiera quisieron escucharlos. Los dos pobres diablos tuvieron entonces que vivir abandonados, y comprendiendo que sólo el ingenio y el trabajo les permitirían sobrellevar una existencia semejante, levantaron sus tiendas en la costa norte de la isla, mirando un poco hacia el oeste para hallarse a salvo del peligro de los caníbales, que siempre elegían la parte oriental para sus desembarcos.

    Construyeron allí dos chozas, una para vivir y la otra destinada a atesorar sus provisiones; como los españoles les habían dado algo de grano para sembrar, especialmente los guisantes que yo les había dejado, cultivaron tierras y sembraron las semillas de acuerdo con mis instrucciones, pudiendo por fin vivir bastante bien. Su primera cosecha fue buena, y aunque solamente habían cultivado un pequeño espacio de tierra por falta de tiempo, obtuvieron lo bastante para hacer pan y otros alimentos. Uno de los dos había sido ayudante del cocinero a bordo, de manera que era diestro en preparar sopas, pasteles y otras comidas a base de arroz, leche y la poca cantidad de carne que podían obtener.

    Estaban, pues, viviendo en tan modesta situación cuando los otros tres miserables, sus compatriotas en maldad y origen, por el solo espíritu de burla y de maldad se acercaron a sus viviendas para insultarlos y provocarlos, diciéndoles que la isla les pertenecía y que el gobernador —se referían a mí— les había dado posesión de ella, por lo cual nadie más tenía derecho alguno sobre esas tierras. Agregaron, con grandes maldiciones, que no permitirían construir ninguna casa en la isla salvo que se les pagara la debida renta por ella.

    Los otros dos creyeron al comienzo que se trataba de una broma; los invitaron a entrar y a sentarse para que vieran las excelentes habitaciones que habían levantado y discutir después la cuestión de la renta; uno de ellos les dijo festivamente que si eran terratenientes, él esperaba que de acuerdo con la costumbre tradicional concederían un prolongado arriendo a aquellos que no sólo levantaban chozas en sus tierras, sino que introducían mejoras en ellas; y terminó su discurso rogándoles que fuesen a buscar a un escribano para redactar la escritura. Uno de los tres, enfurecido al oír esto, gritó que pronto verían que no se trataba de una broma, y corriendo hasta un sitio algo alejado donde los otros pobres hombres habían encendido un fuego para aderezar sus alimentos, tomó un tizón y lo aplicó a la parte exterior de la choza, que inmediatamente empezó a incendiarse. Habría quedado destruida en unos pocos minutos si uno de los dos no hubiese corrido hacia el miserable alejándolo de allí y ahogado el fuego con sus pies no sin gran dificultad.

    El villano estaba tan enfurecido al ver que su víctima le había empujado lejos del incendio, que se precipitó de inmediato sobre él armado de una estaca, y si éste no hubiese evitado a tiempo el golpe refugiándose en la choza es seguro que sus días hubiesen terminado allí. Su compañero, viendo el peligro que corrían ambos, se lanzó en su ayuda y un momento después asomaban armados de sus mosquetes. Lanzándose sobre el que había pretendido matarle con la estaca, el primero lo derribó de un culatazo de su mosquete antes de que los restantes pudieran acudir en su auxilio; y cuando los vieron venir, les apuntaron con las armas ordenándoles que se rindieran.

    Los otros tenían también armas de fuego, pero uno de los pobres atacados, más temerario que su compañero y desesperado por el inminente peligro en que se veían, gritó que si movían solamente una mano eran hombres muertos, ordenándoles que arrojaran las armas. Los otros no le obedecieron, pero viéndolo tan resuelto decidieron parlamentar y terminaron por consentir en marcharse llevándose al herido, que estaba bastante lastimado con el culatazo. Fue una grave equivocación permitirles marcharse en esa forma, ya que teniendo en ese momento ventaja sobre ellos hubieran hecho mejor en desarmarlos y acudir de inmediato a los españoles para narrarles la conducta de aquellos tres bandidos. Naturalmente que éstos, al verse en libertad, no pensaron en otra cosa que en vengarse y todos los días daban alguna señal de que estaban dispuestos a llevar a efecto su desquite.

    Sería excesivo llenar esta parte del relato con detalles de todas las miserables villanías que los tres bandidos cometieron contra aquellos dos desdichados; les estropeaban su plantación, mataron tres cabritos y una cabra que los colonos habían conseguido al fin domesticar para tener alimento; en una palabra, torturándolos noche y día en esa forma, los redujeron a un estado de desesperación tan extremado que los dos hombres se decidieron a luchar a muerte contra los otros tres en cuanto se presentara una oportunidad favorable. Después de meditarlo, se decidieron a ir al castillo —como le seguían llamando— donde los tres miserables y los españoles vivían juntos, a fin de desafiarlos a luchar teniendo como árbitros imparciales a los españoles. Una mañana temprano se encaminaron a mi antigua morada y al llegar llamaron por sus nombres a los ingleses, diciendo al español que les contestó que deseaban hablar con aquéllos.

    Aconteció que el día antes, paseando dos españoles por los bosques, dieron con uno de los ingleses a quienes llamaré honestos para distinguirlos de los tres restantes, quien les hizo un triste relato de la bárbara conducta de sus compatriotas para con ellos, describiéndoles cómo habían arruinado la plantación, destruyendo todo el grano por el cual tanto habían tenido que trabajar y sacrificarse; les explicó que habían matado la cabra que les proveía de leche así como sus tres cabritos, que era todo cuanto tenían entonces para su subsistencia. Por fin declaró que si ellos y los demás españoles no volvían a socorrerlos, morirían seguramente de hambre.

    Cuando los españoles hubieron vuelto a la casa por la noche y estaban reunidos cenando, uno de los que había escuchado el relato se tomó la libertad de dirigir un reproche a los tres ingleses, haciéndolo sin embargo con las palabras y términos más corteses, preguntándoles cómo podían ser tan crueles con aquellos pobres, desamparados e inofensivos colonos que sólo se ocupaban en trabajar esforzadamente para proveer su propia subsistencia, y que realizaban tan extraordinarios esfuerzos para mantener la mísera situación que habían alcanzado.

    Uno de los ingleses respondió vivamente:

    — ¿Qué tienen ésos que hacer aquí? Vinieron a tierra sin autorización, y no poseen derechos para construir ni plantar en la isla; la tierra no es de ellos.

    —Pero entonces, señor inglés —repuso gravemente el español—, morirán de hambre.

    El otro replicó con toda la rudeza de que era capaz:

    — ¡Que se mueran y que el diablo se los lleve! ¡No plantarán ni sembrarán aquí!

    — ¿Pero qué harán entonces, señor? —insistió el español.

    Otro de los miserables lanzó una grosera maldición.

    — ¿Qué harán? ¡Pues serán nuestros sirvientes y trabajarán para nosotros!

    — ¿Cómo podéis pretender tal cosa de ellos? —observó entonces el español—. No los habéis comprado que yo sepa, y por lo tanto carecéis de derecho para convertirlos en vuestros sirvientes.

    A esto le replicaron que la isla les pertenecía, por cuanto el gobernador se la había entregado, y que ningún hombre fuera de ellos tenía derecho alguno. Jurando en nombre de Dios, amenazaron con ir a incendiar las chozas de los dos colonos y no permitir que ninguna construcción se levantara en la isla.

    —De acuerdo con tales principios —dijo entonces el español— ¿Nos consideráis también sirvientes vuestros?

    —Por cierto que sí —replicó insolentemente uno de los miserables— y lo seréis bien pronto, estad seguros.

    A esto el español se contentó con sonreír sin replicar cosa alguna, pero como la discusión los había acalorado, uno de los bribones terminó por levantarse —creo que era el llamado Will Atkins— y llamó a otro.

    —Ven, Jack —dijo—, vamos a escarmentarlos. Les aplastaremos el castillo, puedes tener mi palabra; no habrá más colonias en nuestros dominios.

    Con esto se marcharon los tres, llevándose cada uno una escopeta, una pistola y una espada, mientras murmuraban multitud de palabras injuriosas y amenazas sobre lo que harían a los españoles apenas se presentase una oportunidad propicia. Parece que los españoles no entendieron bien lo que decían como para darse clara cuenta de sus intenciones; pero era evidente que su cólera se debía a que ellos acababan de tomar partido por los dos ingleses.

    La dirección en que marcharon y lo que hicieron aquella noche, los españoles lo ignoraban: probablemente vagabundearon un rato por los alrededores y luego se encaminaron al sitio que yo llamaba mi enramada, donde, sintiéndose cansados, se tendieron a dormir. Su intención era quedarse hasta medianoche y luego, aprovechando que los dos infelices estaban durmiendo, tomarlos por sorpresa; como lo confesaron más adelante proyectaban incendiar las chozas durante su sueño y asesinarlos en esa forma o cuando se lanzaran fuera. Por lo común los malvados no tienen el sueño profundo, y es muy extraño que en esa oportunidad los tres hombres despertaran más tarde de lo previsto para ejecutar su plan.

    Por su parte, los dos habitantes de las cabañas también tenían un proyecto que ya he mencionado, sin duda mucho menos innoble que el de quemar y asesinar de manera que para fortuna suya no estaban en las chozas cuando sus desalmados enemigos llegaron finalmente a. ellas.

    Al advertir las chozas vacías, Atkins, que iba delante, llamó a sus compañeros con una maldición.

    — ¡Eh, Jack, que el infierno se los trague, los pájaros han volado!

    Se quedaron reflexionando cuál podría ser la causa que hubiese alejado a los ingleses tan temprano de sus chozas, y de pronto se les ocurrió que los españoles debían haberles prevenido; plenamente seguros de ello, se estrecharon las manos jurando que se vengarían de los españoles por lo que consideraban una traición. Tan pronto se hubieron comprometido a su sangriento designio, principiaron con las chozas de los desdichados; en vez de quemarlas las derribaron, destruyéndolas con tal encarnizamiento que no dejaron la más pequeña estaca en pie, ni la menor señal de donde antes se alzaban. Hicieron pedazos todos los utensilios y los diseminaron de tal modo que los colonos encontraron más tarde algunas piezas a una milla de distancia. Terminado esto cortaron los arbolillos que los pobres habían plantado, rompieron el vallado que aseguraba el ganado y el grano; en fin, saquearon y rompieron todo con una perfección que ni las hordas de los tártaros los hubiesen igualado.

    Mientras esto sucedía, los dos colonos iban justamente en su búsqueda, decididos a desafiarlos dondequiera los encontraran, a pesar de hallarse en proporción de dos contra tres; por cierto que si los hubieran sorprendido en esa faena habría habido derramamiento de sangre entre ellos, porque en rigor de justicia todos eran hombres resueltos y temerarios.

    Pero la Providencia tomó más cuidado en mantenerlos separados que ellos en encontrarse, y como si deliberadamente hubiesen querido evitarse, cuando unos estaban en el castillo los otros habitaban sus chozas, y justamente cuando los dos colonos acudían en busca de sus enemigos, los otros tres se encaminaban a cumplir su siniestro plan.

    Veremos lo que ocurrió entonces. Los tres volvieron al castillo, furiosos y encendidos por la cólera que su perversa tarea les había ocasionado, y fueron directamente a los españoles para jactarse de lo que acababan de hacer y provocarlos. Uno de ellos, acercándose al que estaba más adelantado, y lo mismo que si fuera un muchacho jugando con otro, le quitó el sombrero que el español tenía puesto y dándole unas vueltas burlonamente le dijo en la cara:

    —A vos, señor Juan Español, os va a suceder lo mismo si no cambiáis de modales.

    El español, que aunque exquisitamente educado era tan valiente como el mejor, además de corpulento y robusto, miró con calma a su adversario durante un buen rato, y después dando un paso hacia él lo derribó de un solo puñetazo; y el hombre cayó como un buey bajo la maza del matarife. Inmediatamente otro de los bandidos, tan insolente como el primero, disparó su pistola sobre el español. No dio en el blanco, pero las balas pasaron junto a su cabeza y una de ellas alcanzó a herirlo en el lóbulo de la oreja, que principió a sangrar abundantemente. Al ver la sangre, el español creyó sin duda que estaba malherido, y esto lo privó de la extraordinaria calma que hasta entonces demostrara. Lanzándose sobre el hombre que había derribado, tomó su mosquete y estaba ya a punto de tirar sobre el que le había disparado el pistoletazo cuando los otros españoles, que permanecían en la caverna, salieron de improviso y le gritaron que no tirase, a tiempo que se apoderaban de los ingleses y los privaban de sus armas. Cuando se encontraron indefensos, y con el agravante de que los españoles eran ahora sus enemigos al igual que sus dos compatriotas, los ánimos se les enfriaron. Con mejores maneras pidieron a los españoles que les devolviesen sus armas, pero éstos, que conocían el conflicto existente entre ellos y los otros dos colonos y sabían que negarles las armas era el mejor modo de evitar que acabaran matándose entre todos, les aseguraron que no les harían daño alguno si estaban dispuestos a vivir pacíficamente en adelante, y que los ayudarían igual que antes; pero de ningún modo se hallaban dispuestos a devolverles sus armas mientras siguieran tan resueltos a perjudicar a sus compatriotas e incluso reducirlos a la servidumbre.

    Los malvados no eran más capaces de escuchar buenas razones que de ponerlas en práctica, y viendo que se les rehusaban las armas se marcharon rabiosos, enloquecidos de cólera y amenazando con toda suerte de venganzas a pesar de que carecían de medios para cumplirlas. Despreciando aquellas bravatas, los españoles les dijeron que se cuidaran de perjudicar en lo más mínimo los plantíos y el ganado, ya que si lo hacían serían cazados a tiros como bestias feroces o, si caían vivos en sus manos, ahorcados sin la menor lástima.

    Como es natural esto no era lo indicado para sosegarlos, y se marcharon todavía más rabiosos, vociferando y jurando como furias del infierno. Tan pronto se habían alejado cuando aparecieron los dos colonos, también enfurecidos y exaltados aunque por razones muy diferentes; al llegar a sus chozas y plantaciones las habían encontrado totalmente demolidas y asoladas, lo cual les produjo la consternación que es de suponer. Apenas tuvieron tiempo de narrar lo ocurrido cuando los españoles les contaron a su vez lo que acababa de pasar; por cierto que resultaba harto extraño que tres hombres pudieran llevar su osadía al extremo de desafiar abiertamente a diecinueve sin recibir el merecido castigo.

    Los españoles, como es natural, los despreciaban y ahora que los otros carecían de armas no se afligían por sus amenazas; pero ambos ingleses estaban resueltos a vengarse de ellos, costara lo que costase encontrarlos en la isla.

    Fue entonces cuando los españoles se interpusieron, diciéndoles que desde el momento que los tres miserables estaban desarmados no podían tolerar que los otros dos los persiguieran bien armados y concluyeran por asesinarlos.

    —Con todo —aseguró el español que ellos consideraban gobernador—, nos comprometemos a haceros justicia si la dejáis en nuestras manos. No hay duda de que esos individuos volverán apenas su rabia se haya calmado, ya que no tienen medio alguno de subsistencia y necesitan nuestro auxilio; os prometemos que no habrá paz con ellos hasta que os hayan dado toda la satisfacción debida. Bajo tales condiciones, esperamos vuestra promesa de que no emplearéis la violencia contra esos malvados, salvo que se tratara de vuestra defensa.

    Los dos ingleses aceptaron esto con poco agrado y reluctancia, pero los españoles insistieron en que sólo deseaban evitar el derramamiento de sangre y que al fin la querella se apaciguaría.

    —Porque —agregaron— no somos tantos en la isla y hay espacio de sobra para todos. ¿Por qué no habíamos de ser buenos amigos?

    Los ingleses terminaron por consentir y se quedaron a la espera de lo que pudiera suceder, viviendo algunos días con los españoles ya que sus chozas habían sido destruidas.

    Unos cinco días más tarde, los tres villanos, cansados de vagabundear y casi muertos de hambre, pues habían tenido que alimentarse solamente de huevos de tortuga, volvieron al castillo y encontrando al español que hacía de gobernador y a dos compañeros que paseaban por las orillas de la ensenada, se acercaron con actitudes humildes y de sometimiento, a la vez que rogaban se les permitiera volver a vivir en común.

    El español les respondió con toda cortesía, pero advirtiéndoles que habían procedido de un modo tan inhumano con sus compatriotas y tan grosero para con ellos los españoles, que no podía responderles nada ante de consultar con los dos colonos y el resto de la población; de todas maneras iría a participarles el pedido para volver a la media hora con una respuesta. Es de imaginar el grado de necesidad a que los tres malvados habían llegado cuando al enterarse de las condiciones, suplicaron que se les enviara algo de pan para comer mientras esperaban el resultado. Así lo hizo el gobernador, enviándoles pan, un gran trozo de carne de cabra y un papagayo asado, que devoraron.

    Transcurrida la media hora fueron llamados y se inició un prolongado debate. Los dos compatriotas los acusaron de haber destruido el fruto de su labor e intentado asesinarlos; como ellos mismos lo habían reconocido anteriormente, no les fue ahora posible negarlo. Acerca de este punto los españoles mediaron entre los contendientes, y así como habían obligado antes a los dos ingleses a que no agredieran a los que andaban desarmados e indefensos, así ahora obligaron a éstos a que de inmediato levantaran las dos chozas para sus compañeros, una del tamaño de las anteriores y la segunda más grande; fueron también forzados a construir cercos en el mismo sitio donde antes los destruyeran, plantar árboles donde habían desarraigado los existentes, y también a que sembraran los plantíos asolados; en una palabra, se los obligó a que rehicieran todo hasta dejarlo en el mismo estado en que lo habían encontrado, o lo más parecido posible, ya que la estación de la cosecha había pasado y los árboles tardarían en crecer hasta el nivel de los anteriores. Se sometieron a todo eso, y como durante el tiempo que duró su trabajo recibieron las provisiones necesarias, se tornaron muy dóciles y la colonia entera empezó a vivir agradable y placenteramente en compañía. La única dificultad estaba en convencer a los tres ingleses de que trabajaran en su propio provecho, pues sólo hacían alguna cosa cuando su capricho se lo sugería. Los españoles les dijeron entonces llanamente que si estaban dispuestos a conservar la buena armonía y la amistad, así como colaborar en el bien común, les ahorrarían las tareas dejándoles tiempo sobrado para que holgazanearan y fuesen a pasear por la isla. En esa forma vivieron bien por un mes o dos, al punto que los españoles les devolvieron sus armas y les dieron libertad para que anduviesen con ellas.






	\chapter{Expedición a las islas}





    No hacía una semana que eran otra vez dueños de sus armas, cuando aquellos desagradecidos empezaron a manifestarse con la misma insolencia y provocación que antes; un suceso inesperado, sin embargo, al poner en peligro la seguridad de todos, los obligó a deponer sus resentimientos privados y ocuparse sólo en salvar sus vidas. Una noche, el gobernador español (como llamo al hombre a quien salvara la vida), que era el capitán y dirigente entre ellos, se sintió intranquilo y no pudo conciliar el sueño por más que se esforzó. Según me dijo más tarde se sentía perfectamente bien de salud, y sólo notaba que sus pensamientos se sucedían tumultuosamente, haciéndole ver su imaginación hombres que luchaban y se mataban los unos a los otros; sin embargo, seguía bien despierto y no lograba dormirse de ninguna manera. Quedó así largo rato, mas como su inquietud fuera en aumento decidió por fin levantarse.

    Miró entonces hacia fuera, aunque se veía muy poco en plena noche, máxime que los árboles plantados por mí en la forma descrita en el anterior relato, y que eran muy altos y espesos, interceptaban su visión. Sólo pudo ver que era una clara noche estrellada y, como todo parecía tranquilo, volvió a acostarse.

    El malestar se repitió, le era imposible dormir y tampoco quedarse en la actitud del que descansa, pues sus pensamientos se tornaban angustiosos y no podía él descubrir la causa que los motivaba.

    Como al levantarse y andar había hecho algún ruido, uno de sus compañeros se despertó y preguntó quién era el que se movía. El gobernador le explicó entonces lo que le pasaba.

    — ¿Verdaderamente sentís eso? —dijo el otro español—. Tales señales no deben ser despreciadas, y por cierto que alguna cosa mala se está preparando en contra de nosotros. ¿Dónde están los ingleses?

    —En sus chozas —repuso el gobernador— y bien tranquilos.

    Parece que desde el último motín, los españoles habían decidido quedarse con la habitación principal, designando un sitio donde los tres ingleses habían tenido que instalarse lejos de los otros.

    —De todas maneras —dijo el español— algo hay en esto que me inquieta, os lo digo por experiencia. Venid, salgamos a reconocer los alrededores, y si nada encontramos que justifique nuestras aprensiones os contaré un relato que os demostrará la razón de mi inquietud.

    Salieron entonces para subir a lo alto de la colina, pero siendo dos y sintiéndose fuertes no tomaron las precauciones que utilizaba yo en mi soledad, colocando la escalera y retirándola luego para apoyarla en el tramo superior; por el contrario, dieron la vuelta al soto despreocupados e imprudentes, cuando los sorprendió descubrir la luz de un fuego a poca distancia del punto que habían alcanzado, así como las voces de un gran número de hombres.

    Cuantas veces había yo alcanzado a ver salvajes desembarcando en la isla, mi inmediata preocupación había sido impedirles que advirtieran la más pequeña señal de que la tierra estaba habitada. En verdad que cuando alcanzaron a descubrirlo fue a costa de una experiencia de la cual pocos se salvaron para ir con el relato a los demás, ya que nos apresuramos a desaparecer lo antes posible, y no creo que ninguno de los que alcanzaron a verme pudiera ir con la noticia, salvo aquellos tres salvajes que en nuestra última batalla consiguieron saltar a una piragua y de los cuales temía yo que fuesen con el relato y trajeran refuerzos.

    No es preciso decir que apenas el gobernador y su acompañante advirtieron la presencia de aquellos hombres, retrocedieron precipitadamente dando la voz de alarma; explicaron a los compañeros lo que acababan de ver y el peligro que los acechaba, por lo cual habían de defenderse al instante; les fue sin embargo, imposible convencerlos de que se quedaran dentro de las fortificaciones, pues todos querían salir para enterarse en persona de lo que ocurría.

    Mientras duró la noche, esto fue relativamente fácil y no les faltó oportunidad de observar durante varias horas a los salvajes iluminados por tres hogueras que habían encendido a cierta distancia una de otra. Ignoraban lo que podían estar haciendo allí, y también lo que a ellos les convenía resolver; ante todo, los enemigos eran extraordinariamente numerosos, y en segundo lugar no estaban juntos, sino separados en distintos grupos y ocupando diversas partes de la costa.

    Cuando lo advirtieron, los españoles cayeron presa de la consternación, sobre todo al notar que los salvajes andaban errando por la costa, ya que tarde o temprano alguno de ellos daría con la casa o cualquier otro lugar donde hubiera huellas de habitantes. Sentían gran temor por la suerte de su rebaño de cabras, cuya destrucción hubiera significado para ellos poco menos que la muerte por hambre. Lo primero que entonces se les ocurrió fue enviar, antes de que llegase el día, a dos españoles y un inglés para que condujesen el rebaño hacia el gran valle donde estaba la caverna, a fin de que si las cosas se tornaban peores pudiese ser escondido en su interior.

    Luego de considerar largamente las medidas a adoptarse, y de agotar su ingenio calculando las posibilidades, se resolvieron, aprovechando la oscuridad, a pedir al anciano padre de Viernes que espiara a los salvajes y tratase de averiguar alguna cosa más, tal como el propósito que los había guiado en su desembarco y cuáles eran sus intenciones. El anciano aceptó de inmediato y después de desnudarse completamente, como era el uso entre los salvajes, partió para volver una o dos horas más tarde con el anuncio de que se había mezclado entre ellos sin ser descubierto. Traía la noticia de que había dos grupos pertenecientes a distintas naciones en guerra y que acababan de sostener una gran batalla en su país. Ambos grupos se habían adueñado de numerosos prisioneros, y la sola casualidad había dispuesto que desembarcasen casi en el mismo sitio para devorarlos y realizar los festejos del triunfo. Ahora, el saberse vecinos ahogaba toda su alegría, reemplazándola por una terrible cólera que al parecer del anciano los llevaría a lanzarse a una nueva batalla apenas viniera la luz del día, ya que estaban muy cerca los unos de los otros. Agregó que en ningún momento había advertido que los salvajes creyeran habitada la isla, y apenas acababa de pronunciar estas palabras cuando los que lo escuchaban oyeron el ruido característico de los dos pequeños  ejércitos  al  trabarse  en  sangrienta  lucha.

    El padre de Viernes empleó todos sus argumentos para persuadir a los nuestros de que no se mezclaran en la batalla y permanecieran ocultos. Les dijo que ésa era su única salvación, ya que los salvajes se limitarían a combatir entre ellos, terminando los sobrevivientes por embarcarse, como efectivamente ocurrió. Pero era imposible contener la curiosidad de algunos, en especial de los ingleses, cuyo deseo de presenciar la batalla les hacía olvidar toda prudencia. Cierto que emplearon algunas precauciones, tal como no ir en línea recta desde su vivienda, sino dando un rodeo por los bosques, a fin de situarse en posición ventajosa para contemplar el combate sin ser vistos por los salvajes, según ellos creían; la verdad es que los salvajes los divisaron, como se contará más adelante.

    La batalla fue encarnizada, y de creer a los ingleses, uno de ellos afirmó que muchos de los salvajes eran hombres de extraordinaria valentía, indomable firmeza y mucha habilidad en el mando. La batalla duró dos horas, según sus cálculos, antes de que pudiera distinguirse un vencedor y un vencido, pero por fin el grupo que se encontraba del lado más cercano a la morada de los colonos dio señales de desconcierto, y al cabo de un rato algunos emprendieron la fuga. Esto produjo gran consternación entre los españoles, porque temían que alguno de los fugitivos se internara en el soto buscando protección y descubriera involuntariamente el lugar de su residencia, e igual cosa ocurrió con los que venían en su seguimiento. Resolvieron entonces apostarse armados detrás de la empalizada y quedar a la espera, listos para hacer una salida apenas un salvaje apareciera en el soto y matarlo inmediatamente, tratando de que ninguno pudiese volver con la noticia. Decidieron que sólo emplearían las espadas o bien la culata de los mosquetes, a fin de que las detonaciones no atrajeran a los demás.

    Tal como preveían ocurrieron las cosas. Tres hombres del bando derrotado, huyeron para salvar la vida, cruzaron la ensenada directamente hacia las fortificaciones, sin tener la menor idea del lugar al cual se encaminaban, pero eligiendo el espeso bosque como un conveniente refugio. El centinela avanzado que habían puesto en observación dio la noticia a los de adentro, con el agradable agregado de que los vencedores no se preocupaban por perseguir a aquellos salvajes ni parecían molestarse en averiguar la dirección que tomaran. El gobernador español, que era hombre humanitario, no quiso entonces que se matara a los tres fugitivos, sino que envió a tres de los suyos por lo alto de la colina para que, dando un rodeo, los sorprendieran por la espalda y tomaran prisioneros a los tres salvajes, lo cual se efectuó sin inconvenientes. El resto de los vencidos huía entretanto en sus canoas, y los vencedores, después de mostrar muy pocas intenciones de perseguirlos, se reunieron y lanzaron dos veces un penetrante alarido, lo que sin duda valía como su grito de guerra. La batalla había, pues, terminado, y ese mismo día a las tres de la tarde el resto de los salvajes se embarcó en sus piraguas. Los españoles eran otra vez dueños de su isla, con la libertad perdieron todo temor y no volvieron a ver salvajes por espacio de varios años.

    Luego que los combatientes se hubieron marchado, los colonos salieron de su refugio para reconocer el campo de batalla, hallando en él treinta y dos muertos. Algunos habían sido alcanzados por largas flechas y varias aparecían clavadas en los cadáveres, pero la mayoría había encontrado la muerte bajo los golpes de grandes espadas de madera, de las cuales había dieciséis o diecisiete tiradas en el campo de batalla, así como buen número de arcos y flechas. Las espadas eran sólidas y pesadas, de difícil manejo, por lo cual se puede deducir la fortaleza de los guerreros que las empleaban. La mayoría de los salvajes muertos con esas armas tenían la cabeza deshecha o, como decimos en Inglaterra, los sesos saltados, mientras muchos otros mostraban brazos y piernas rotos. Puede deducirse por ello la indescriptible furia con que combaten aquellos individuos. No hallamos uno solo que no estuviese bien muerto, pues es costumbre de los salvajes permanecer junto a su enemigo golpeándolo hasta acabar con él, o bien llevarse a todos los heridos que aún conservan un soplo de vida.

    Este episodio amansó mucho a los tres malvados ingleses, y por un buen espacio de tiempo se mostraron muy tratables, manifestándose dispuestos a trabajar al igual que el resto de la comunidad; plantaban, sembraban, recogían los frutos, dando la impresión de haberse adaptado ya a su nueva existencia. Sin embargo, poco tiempo después cometieron actos que volvieron a precipitarlos en los peores conflictos.

    Habían apresado, como ya se ha dicho, a tres salvajes, hombres jóvenes y sobremanera robustos, que fueron obligados a servir como criados y se les enseñó a trabajar para la comunidad. En su condición de esclavos, aquellos tres hombres cumplían bastante bien su tarea, pero sus amos no procedieron con ellos en la forma en que yo lo había hecho antaño con Viernes, es decir, principiando por hacerles advertir con claridad que les habían salvado la vida, y luego instruirlos paulatinamente en los principios racionales de la existencia; tampoco se preocuparon de inculcarles nociones religiosas e irlos elevando a la civilización por medio de un trato amable y argumentos convincentes. Por el contrario, se limitaban a darles una ración de alimentos diaria a cambio de un trabajo equivalente que por su intensidad los embrutecía aún más. Los resultados de este trato es que jamás pudieron lograr que aquellos hombres combatieran por ellos y los acompañaran en el peligro, como lo había hecho Viernes, que me era tan fiel como la carne a los huesos.

    Pero volvamos a la colonia. Ahora que todos eran buenos amigos —ya que el peligro, como he señalado más arriba, los había reconciliado— empezaron a reflexionar sobre las circunstancias en que se hallaban. Lo primero que pusieron en consideración fue el hecho de que los salvajes preferían acercarse al lado de la isla donde estaba su morada; había en cambio otros sitios más remotos y ocultos, pero que se prestaban igualmente bien para instalar la pequeña comunidad a salvo de aquellas acechanzas. ¿Por qué no mudarse, entonces, yéndose a algún sitio donde todos ellos, así como el grano y los ganados, estuvieran a salvo?

    Se produjo un largo debate, al final del cual decidieron que no les convenía abandonar su presente morada, puesto que sin duda alguna vez recibirían noticias del gobernador —como me llamaban— y si en vez de venir en persona enviaba yo a algún otro, estaba claro que le daría instrucciones de buscarlos en mi antigua vivienda. El enviado, encontrando abandonado el sitio, pensaría que los colonos habrían perecido a manos de los salvajes y se marcharía,  dejándolos sin auxilio alguno.

    En lo que se refiere al ganado y al grano, decidieron sin embargo, trasladarlos al valle donde estaba mi caverna; la tierra se prestaba para ambas cosas y era muy extensa. Lo pensaron mucho, y por fin introdujeron una modificación en el plan, conviniendo en llevar parte del ganado y hacer plantíos en el valle de tal modo que si los salvajes destruían las existencias acumuladas en el castillo, el resto podría salvarse. Mostraron tanta prudencia y tanto tino que no se confiaron a los tres salvajes que tenían prisioneros, manteniéndolos en la ignorancia de la nueva plantación así como del ganado que allí criaban; mucho menos aún les hablaron de la caverna que en caso de apuro sería un segurísimo refugio, y al cual llevaron los dos barriles de pólvora que yo les dejara antes de embarcarme.

    En suma, que se decidieron a no cambiar de habitación, pero advirtiendo que yo la había disimulado cuidadosamente con una empalizada y luego con un bosque, pues toda la seguridad y defensa del lugar estaba en que pudiera mantenerse secreto, decidieron aumentar las obras de fortificación y ocultar aún más el sitio donde vivían. A tal efecto, de la misma manera que yo había plantado árboles (o, mejor, estacas que con el tiempo llegaron a ser tales) desde la entrada de mi residencia hasta una cierta distancia, así prolongaron ese soto llenando de árboles el espacio que aún restaba libre hasta la misma orilla de la ensenada donde por primera vez había traído mis balsas; pusieron árboles incluso en la parte anegadiza donde alcanzaba la marea, para que no quedara ningún espacio adecuado a un desembarco ni el menor signo de que había habido un sitio abordable en los alrededores. Eligieron estacas de una especie que se desarrolla rápidamente y de la cual he hablado antes, cuidando que fueran mucho más gruesas y altas que las puestas por mí; como las habían plantado muy juntas y crecían de inmediato, apenas transcurrieron tres o cuatro años cuando ya ninguna mirada humana hubiese podido penetrar en la espesura del bosque. En lo que respecta a la porción puesta por mí, los árboles tenían troncos como el muslo de un hombre, y en los espacios libres plantaron otros más pequeños y en tal cantidad que aquello terminó siendo una verdadera empalizada cuyo espesor llegaba al cuarto de milla, y de todo punto impenetrable a no ser que lo hiciera un pequeño ejército dispuesto a derribar árbol por árbol; un perro hubiera tenido dificultad para pasar entre aquellos troncos, tan cerca estaban unos y otros.

    No fue esto todo, porque los colonos completaron las defensas en todo el lado que miraba a la derecha y a la izquierda, llegando incluso a lo alto de la colina, no dejando otra entrada —ni siquiera para ellos— que la escalera puesta en la roca y de la que he dicho que se alzaba para colocarla en un segundo apoyo que permitía subir la cumbre; una vez que retiraban la escalera, nadie que no tuviese alas o poderes mágicos hubiera conseguido franquear la distancia que de ellos lo separaba.

    Todo esto había sido muy bien ideado, y más adelante tuvieron sus autores ocasión de comprobarlo, lo que me convence aún más de que así como la prudencia está justificada por la autoridad de la Providencia, así también está dirigida por ella cuando se aplica prácticamente, si fuéramos capaces de escuchar su voz, estoy persuadido de que evitaríamos muchos de los desastres a que nuestra vida se ve expuesta a causa de nuestra negligencia. Que esto sea dicho de paso.

    Y vuelvo a mi relato. Los colonos vivieron otros dos años perfectamente seguros, sin recibir nuevas visitas de los salvajes. Cierta mañana, sin embargo, tuvieron una alarma que los llenó de consternación, cuando algunos españoles que habían ido muy temprano á la costa oeste, mejor dicho, al fin de la isla —sitio al que yo por mi parte no me había acercado jamás por miedo a ser descubierto—, se sorprendieron al descubrir de improviso no menos de veinte canoas de indios que se aprestaban a desembarcar. Corrieron con toda la rapidez posible a la casa, y luego de dar la alarma a sus camaradas permanecieron al acecho todo* ese día y el siguiente, saliendo sólo de noche para hacer reconocimientos. Pero por fortuna se habían equivocado al creer que los salvajes iban a desembarcar, ya que aquéllos sin duda tenían otros planes y se marcharon con rumbo desconocido.

    Fue por ese entonces cuando una nueva querella se suscitó con los tres ingleses; uno de estos, individuo extremadamente díscolo, se enfureció contra uno de los esclavos porque el infeliz no había cumplido a su gusto algo que le ordenara y no parecía bien dispuesto a escuchar sus indicaciones. Sacando una hachuela que llevaba en la cintura, aquel desalmado se precipitó sobre el desgraciado salvaje, no con la intención de corregirle, sino de asesinarlo. Uno de los españoles que presenciaba la escena vio que el hacha se descargaba con bárbara fuerza sobre la víctima, y aunque el golpe, dirigido a la cabeza, sólo lo alcanzó en un hombro, su fuerza era tal que casi le arrancó el brazo; corriendo entonces, el español se interpuso entre ambos para evitar el homicidio.

    Tan furioso estaba el malvado al ver esto, que se precipitó con el hacha sobre el español jurando que haría con él lo mismo que con el salvaje, y le tiró un golpe que el otro, prevenido, alcanzó a parar a la vez que, devolviéndolo con la pala que tenía en la mano (ya que estaban todos entregados a las faenas del plantío), derribó sin sentido a su adversario. Otro de los ingleses vino inmediatamente en ayuda de su compatriota derribando al español, en cuyo auxilio acudieron dos de sus compañeros, que fueron atacados entonces por un tercer inglés. Ninguno llevaba fusiles, y sus armas consistían en hachuelas y otras herramientas, pero el último de los ingleses tenía consigo uno de mis viejos y enmohecidos machetes con el cual alcanzó a herir a sus dos adversarios. Esta querella exaltó los ánimos de toda la comunidad, y los españoles terminaron por reunirse y tomar prisioneros a los tres ingleses. De inmediato se planteó la cuestión de qué debía hacerse con ellos. Se habían amotinado tantas veces, se mostraban tan furiosos, rebeldes y holgazanes, que no sabían ya de qué manera castigarlos por tantos crímenes, temerosos de su carácter vengativo y de lo poco que se contenían al agredir a los demás; evidentemente la vida no estaba asegurada si permanecían a su lado.

    El español que hacía de gobernador les manifestó que de ser compatriotas suyos los habría hecho colgar, puesto que las leyes y los gobiernos estaban destinados a preservar la sociedad, y los tres eran lo bastante peligrosos para ser expulsados de ella. Con todo, teniendo en cuenta que se trataba de ingleses y que a la generosidad de un inglés debían ellos su libertad y su vida, trataría de ser misericordioso y los entregaría a la decisión de los otros dos ingleses que eran sus compatriotas.

    Al oír esto, uno de los dos honestos ingleses se puso de pie y suplicó que el destino de los prisioneros no les fuera confiado.

    —Porque —agregó— en conciencia deberíamos sentenciarlos a la horca.

    Luego de estas palabras contó cómo Will Atkins, uno de los tres, les había propuesto reunirse y asesinar a todos los españoles mientras estuvieran entregados al sueño.

    Cuando el gobernador español hubo oído esto, se dirigió a William Atkins.

    — ¡Cómo, señor Atkins! —exclamó—. ¿Es que pensabais matarnos? ¿Qué tenéis que decir a esta acusación?

    El miserable villano, lejos de negar el hecho, dijo que era cierto y que trataría de llevarlo a cabo antes de que ellos lo matasen.

    —Muy bien, señor Atkins —dijo entonces el español—, ¿pero qué os hemos hecho nosotros para que intentéis matarnos así? ¿Y qué ganaréis cometiendo esa atrocidad? Contestadme también: ¿qué debemos hacer para impedir que nos matéis? ¿Tendremos que adelantarnos para no ser muertos? ¿Por qué nos lleváis a tales extremos, señor Atkins? —concluyó el español muy sereno y sonriendo al hablar.

    Will Atkins estaba tan rabioso al oír que el gobernador se burlaba de su proyecto, que de no haber estado sujeto por tres hombres es de creer que hubiese intentado, pese a no tener armas, asesinar al español delante de todos los otros. Tan estúpido comportamiento los obligó a que consideraran seriamente la conducta a seguir. Los dos ingleses y el español que librara al pobre salvaje eran de opinión que había que ahorcar a uno de los tres para que sirviera de ejemplo a los restantes; que se ejecutara al que por dos veces consecutivas había intentado cometer un asesinato armado de un hacha, y que en parte había logrado su fin, ya que el infeliz salvaje estaba tan malherido que no había muchas posibilidades de que se recobrara.

    Sin embargo, el gobernador español se mantuvo en la negativa. Un inglés les había salvado la vida, y él no consentiría jamás en condenar a muerte a un hombre de esa nacionalidad, aunque hubiese asesinado a la mitad de ellos; agregó que incluso si él mismo sucumbiera a manos de un inglés y le quedase tiempo para articular unas palabras, sería para pedir al resto que lo perdonara.

    Con tanta vehemencia insistió el gobernador que no se atrevieron a contradecirlo, y como la clemencia obtiene en seguida numerosos partidarios cuando es predicada con calor, todos terminaron por adoptar su partido. Quedaba sin embargo en pie la cuestión de impedir a aquellos villanos que pusieran su siniestro plan en ejecución, ya que tanto el gobernador como sus hombres veían necesario tomar medidas para preservar de tal peligro a la pequeña comunidad. Luego de un largo debate se decidió ante todo desarmar a los ingleses, y no permitirles usar en adelante fusiles, pólvora y balas, espadas ni arma alguna, tras lo cual se los arrojaría de la sociedad para que viviesen donde les pareciera mejor y en la forma que se les antojase. Quedó prohibido a los españoles y a ambos ingleses honestos que hablaran con los desterrados o tuviesen ningún contacto con ellos; les prohibieron acercarse a la residencia de la colonia y se les fijó el límite preciso, previniéndoles que apenas se advirtieran señales de que pretendían cometer algún desorden, tal como asolar, incendiar o cometer cualquier tropelía contra las plantaciones, el ganado o las empalizadas pertenecientes a la comunidad, serían muertos sin lástima, cazándolos como bestias salvajes dondequiera los hallaran.

    El gobernador, que era hombre de nobles sentimientos, consideró la sentencia y luego, volviéndose hacia los dos ingleses que los acompañaban, les dijo:

    —Escuchad, es preciso tener en cuenta que pasará mucho tiempo antes de que esos individuos puedan disponer de una cosecha propia y domesticar animales, y entretanto se morirán de hambre; necesario será, por consiguiente, proveer a su subsistencia.

    Se agregó entonces a la sentencia que los tres ingleses recibirían al ser expulsados una cantidad de grano que les alcanzara unos ocho meses tanto para alimentarse como para sembrar, a cuyo término podrían recoger su primera cosecha propia; además se les entregarían ocho cabras lecheras, cuatro machos y seis cabritos, tanto para que se alimentaran de ellos como para que iniciaran un rebaño. Se les darían igualmente las herramientas necesarias para el trabajo del plantío: seis hachuelas, un hacha, una sierra y otras cosas semejantes, pero no recibirían ninguno de esos instrumentos hasta tanto no juraran solemnemente que no los emplearían para herir a los españoles ni a los ingleses. Así los arrojaron de la sociedad, condenándolos a que vivieran por su cuenta y riesgo. Se alejaron sombríos y rebeldes, sin deseos de marcharse ni de quedarse,  pero  como  no  les quedaba otro remedio  que hacerlo se fueron fingiendo que les placia ir a vivir independientemente, ser dueños de su plantación y demás bienes. Al marcharse se les entregaron algunas provisiones, pero no armas.

    Cuatro o cinco días más tarde estuvieron de regreso en busca de más vituallas, e informaron al gobernador del sitio donde habían decidido instalarse, construir sus chozas e iniciar una plantación. Por cierto que el lugar era excelente, en la parte más remota de la isla, mirando hacia el N.E. y muy cerca del lugar donde conseguí hacer tierra después de haber sido arrastrado mar afuera, Dios sabe hasta dónde, en mi tentativa de circunnavegar la isla.

    Allí levantaron dos bonitas chozas utilizando la misma defensa que yo empleara en mi morada, es decir, protegiéndolas contra el flanco de una colina que aparecía rodeada por árboles en tres de sus lados; bastaba, pues, plantar algunos más para que las habitaciones quedaran totalmente ocultas por más que se las buscara. Querían pieles de cabra para que les sirviesen de abrigo y de lecho, y les fueron dadas. Luego que empeñaron su palabra de que en modo alguno perturbarían al resto de la colonia, o intentarían dañar sus plantaciones, les entregaron hachuelas y demás herramientas de las que podían desprenderse, así como guisantes, cebada y arroz para que sembraran; o sea, les proveyeron de todo menos de armas y municiones.

    Así apartados vivieron por espacio de seis meses, y recogieron su primera cosecha, la que resultó muy escasa porque el terreno cultivado era pequeño. Aquellos hombres tenían una inmensa tarea puesto que iniciaban el plantío con todas las fatigas imaginables; asimismo cuando se pusieron a hacer tablones o cacharros, su torpeza no les permitió obtener nada de provecho. Al llegar la estación de las lluvias, como carecían de una cueva en la colina no pudieron mantener seco el grano obtenido, y se vieron en peligro de perderlo todo. Estas contrariedades los tornó más humildes, y entonces acudieron a suplicar a los españoles que los ayudasen, lo que éstos aceptaron de buen grado. En cuatro días excavaron todos un gran agujero en la ladera rocosa, con capacidad suficiente para proteger de la lluvia el grano y otros alimentos. Era, sin embargo, un pobre almacén en comparación con el mío, en especial ahora que los españoles lo habían agrandado mucho, haciendo varios compartimientos nuevos en el interior de la cueva.

    Nueve meses después de los episodios narrados, una nueva locura se apoderó de aquellos tres bandidos que, sumada a sus anteriores hazañas, les trajo graves consecuencias y estuvo a un paso de ser la ruina de la colonia entera. Cansados los tres socios, por lo que parece, de la trabajosa vida que llevaban, y sin esperanzas de salir jamás de aquella situación, tuvieron la idea de emprender un viaje al continente de donde venían los salvajes e intentar apoderarse de algunos prisioneros entre los nativos que encontraran, trayéndolos consigo para tenerlos esclavos.

    El proyecto no habría sido tan descabellado de terminarse ahí; pero sus autores no eran capaces de proponer o hacer nada que no contuviese maldad en sí, ya sea en la intención o en el modo de ejecutarlo; verdaderamente es de creer, si se me permite dar mi opinión, que aquellos hombres estaban malditos de Dios.

    Pero vuelvo a mi relato sin más disgresiones. Los individuos acudieron una mañana a los españoles y solicitaron audiencia con palabras humildes. Atendidos de inmediato, manifestaron que estaban hartos de vivir en la forma que lo hacían, puesto que no eran lo bastante diestros para proveerse de las muchas cosas que necesitaban. Como carecían de ayuda, se veían condenados a morirse de hambre, por lo cual suplicaban a los españoles que les permitieran tomar una de las canoas en las cuales habían venido a la isla, así como armas y municiones adecuadas a su empresa, consistente en hacer la travesía en busca de mejor suerte, lo que entre otras cosas significaría para los demás colonos verse libres de darles provisiones continuamente.

    Los españoles sintieron la alegría que es de imaginar a la idea de verse libres de aquellos hombres, pero se apresuraron con toda nobleza a señalarles la segura desgracia que los esperaba en aquellas tierras. Agregaron que ellos habían pasado por tales pruebas en los mismos lugares, que podían afirmar sin ningún espíritu de profecía que apenas llegasen allá morirían de inanición o serían asesinados, por lo cual les rogaban que reflexionasen antes de decidirse.

    A esto replicaron audazmente los ingleses que lo mismo morirían de hambre quedándose en la isla, pues ni sabían ni querían trabajar, y que si en el continente perecían de inanición o asesinados no era para preocuparse puesto que no dejaban mujeres o hijos que los llorasen; en suma, insistieron en sus demandas, declarando que lo mismo se irían aunque no quisieran darles armas.

    Oyendo esto, los españoles replicaron cortésmente que si estaban dispuestos a embarcarse no se irían sin contar con elementos suficientes para defender sus vidas; cierto que les era penoso entregarles armas de fuego, pues apenas contaban con suficiente número para sí mismos, pero así y todo les darían dos mosquetes, además de una pistola, un machete y dos hachuelas, lo cual parecía suficiente para la expedición.

    Los ingleses aceptaron la oferta. Entonces, después que los colonos hornearon suficiente cantidad de pan para un mes y lo entregaron a los viajeros junto con carne fresca de cabra, una canasta grande de pasas, un tonel de agua dulce y un cabrito para matar durante el viaje, los tres se lanzaron a la aventura de cruzar el mar rumbo a una tierra situada por lo menos a cuarenta millas de distancia.

    La canoa era muy grande, con capacidad para quince o veinte hombres, por lo cual les era bastante trabajoso pilotearla; pero como el viento soplaba a favor y la marea los ayudaba, iniciaron felizmente el viaje. Habían hecho un mástil con una larga pértiga; y cuatro grandes pieles de cabra, secas y cosidas entre sí, formaban la vela. Llenos de entusiasmo se hicieron a la mar y los españoles los despidieron con un « ¡Buen viaje!», aunque cada uno de ellos estaba seguro de que no volvería a verlos jamás.

    Con frecuencia comentaban los colonos entre sí, incluso con los dos honestos ingleses que los acompañaban, qué tranquila y agradable era la vida ahora que aquellos turbulentos individuos se habían marchado. Jamás cruzó por la mente de ninguno de ellos la más remota idea de que los viajeros retornaran alguna vez, cuando he aquí que veintidós días más tarde uno de los ingleses que estaba trabajando lejos en su plantío vio repentinamente a tres desconocidos que se acercaban por aquel lado llevando fusiles al hombro.

    Como si se hubiera vuelto loco corrió el inglés a llevar la noticia al gobernador español, diciéndole que estaban en peligro por cuanto individuos desconocidos habían desembarcado en la isla y él ignoraba quiénes podían ser. El español reflexionó unos momentos, antes de hablar.

    — ¿Qué queréis decir —preguntó— con eso de que ignoráis  quiénes  son?  Indudablemente  se trata de salvajes.

    — ¡No, no! —dijo el inglés—. ¡Son hombres vestidos, con armas!

    —Pues entonces, ¿por qué afligirse así? —repuso el gobernador—. Si no son salvajes serán amigos, pues no hay nación cristiana en la tierra que no pueda hacernos más bien que mal.

    Mientras hablaban, los tres ingleses se presentaron por la parte exterior del bosque y gritaron para que los reconocieran. De inmediato cesó todo motivo de asombro, aunque los colonos no tardaron en sentirse nuevamente asombrados por otras razones, y sobre todo preocupados por no saber el motivo que había hecho regresar a aquellos individuos.

    Poco después se hallaban todos reunidos, y al interrogarlos sobre la travesía y sus incidentes los tres aventureros hicieron un breve resumen de cuanto les había pasado. Contaron que después de dos días de viaje o algo menos habían llegado a vista de tierra, pero no se atrevieron a desembarcar porque los nativos estaban alborotados al divisar la canoa y se preparaban con arcos y flechas a pelear contra ellos, por lo cual costearon hacia el norte durante seis o siete horas hasta llegar a un gran canal. Comprendieron entonces que la tierra divisada desde nuestra costa no era el continente, sino una isla, y remontando el canal descubrieron otra isla hacia la derecha, rumbo al norte, y varias más al oeste. Resueltos a desembarcar en alguna parte, se decidieron temerariamente a hacerlo en una de las que se hallaban al oeste, y pronto estuvieron en la costa. Encontraron nativos sumamente bondadosos y pacíficos, que de inmediato les ofrecieron raíces alimenticias y algo de pescado seco, manifestándose muy sociables. Las mujeres pareciendo tan deseosas como los hombres de que no les faltaran vituallas suficientes, se apresuraban a traerlas desde lejos sobre su cabeza. Permanecieron cuatro días en la isla, tratando de averiguar por medio de signos qué pueblos habitaban en una y otra parte; pronto comprendieron que en todas direcciones existían terribles y salvajes tribus, que se comían a los hombres, según les explicaron los indígenas. Ellos afirmaban no comer jamás carne humana, salvo cuando capturaban prisioneros en la guerra; terminaron por confesar que en esos casos hacían un gran festín y devoraban a los prisioneros.

    Los ingleses quisieron saber cuál era la última vez que una fiesta semejante había tenido lugar, y les dijeron que dos lunas antes, señalando la luna y luego mostrando dos dedos; que su gran rey había tomado doscientos prisioneros, los cuales eran engordados actualmente para la próxima fiesta. Como los ingleses se mostraran deseosos de ver a aquellos prisioneros, los indígenas interpretaron mal sus signos y creyeron que deseaban algunos para llevárselos en la canoa y devorarlos más adelante. Se apresuraron entonces a complacerlos, señalando primero hacia el poniente y luego al este, lo que quería decir que a la mañana siguiente, cuando saliera el sol, les traerían varios prisioneros. En efecto, aparecieron por la mañana con cinco mujeres y once hombres, regalándolos a los ingleses para que se los llevasen en la canoa, tal como nosotros llevaríamos igual número de vacas y bueyes a un puerto para avituallar a un navío.

    Por muy embrutecidos y perversos que aquellos hombres fuesen en su vida ordinaria, a la vista de eso sintieron que se le revolvía el estómago y no supieron qué hacer.

    Rehusar a los prisioneros hubiera sido la peor ofensa a los corteses salvajes que así los agasajaban, pero a la vez no sabían qué conducta adoptar. Luego de discutirlo entre ellos un rato, decidieron aceptar los prisioneros, y en compensación obsequiaron a los salvajes una hachuela, una vieja llave y un cuchillo, así como seis o siete balas que les llamaron mucho la atención aunque no podían comprender su objeto. Entonces, luego de atar las manos a la espalda de los desdichados prisioneros, los indígenas los arrastraron a la canoa de los ingleses.

    Tan pronto se hubo terminado esto los tres hombres se vieron en la necesidad de hacerse a la mar, ya que de lo contrario los dadores de tan generoso presente se hubiesen extrañado de que los viajeros no se dedicaran en seguida a comerse alguna de las víctimas, o bien, en retribución de atenciones, mataran a dos o tres e invitasen a los donantes a participar del festín.

    Luego de marcharse de la isla, con todas las demostraciones de afecto y gratitud que pueden prodigarse dos grupos que no entienden una sola palabra de cuantas se dicen mutuamente, los ingleses volvieron al mar con su canoa y se encaminaron directamente hacia la primera isla donde pusieron en libertad a ocho de sus prisioneros, ya que el total era excesivo para lo que ellos deseaban.

    Durante el viaje de regreso trataron de comunicarse de alguna manera con sus prisioneros, pero fue imposible hacerles comprender nada; cuanto les decían o les ofrecían era interpretado de inmediato como una señal de que iban a ser devorados. Los ingleses los desataron, pero los infelices se pusieron a gritar lastimeramente, en especial las mujeres, pues les parecía sentir ya el cuchillo en la garganta; el gesto de sus amos era para ellos clara señal de que su fin se avecinaba.

    Lo mismo ocurría si les daban de comer, ya que pensaban que los ingleses temían verlos adelgazar demasiado y que no sirvieran para el banquete; si detenían sus miradas por un momento en uno de ellos, el resto deducía que estaban analizando si se hallaba en condiciones de ser devorado. Incluso después que el viaje hubo concluido y aquellos salvajes recibieron buen trato en la isla, todavía esperaban constantemente ser víctimas del apetito de sus nuevos amos y servirles de almuerzo o cena.

    Cuando los tres aventureros terminaron el relato de su increíble aventura, los españoles les preguntaron dónde estaba su nueva familia. Al responderles que los habían encerrado momentáneamente en una de las chozas, y que venían a pedir algunos alimentos para ellos, todos los colonos tanto españoles como ingleses, y contando también al padre de Viernes, resolvieron ir allá para ver a los nuevos habitantes de la isla.

    Al entrar en la choza encontraron a los nativos sentados y con las manos ligadas, pues los ingleses habían adoptado esa precaución al desembarcar, para impedir que se apoderaran de la canoa en una tentativa por recobrar la libertad. Allí estaban, repito, sentados y completamente desnudos. Había tres hombres, fuertes y bien plantados, de excelentes proporciones que contarían de treinta a treinta y cinco años, y cinco mujeres, de las que dos tendrían entre treinta y cuarenta años, otras dos no pasaban de veinticinco y la última, una alta y hermosa doncella, contaría dieciséis o diecisiete. Las mujeres eran de muy buena presencia, tanto en formas como en facciones, salvo el color de la piel; dos de ellas, si hubiesen sido blancas, habrían pasado por muy hermosas mujeres aun en Londres, ya que sus figuras eran graciosas y tenían actitudes en extremo modestas, sobre todo cuando más tarde se les enseñó a andar vestidas y engalanadas, bien que lo que ellas llamaban galas apenas si merece tal nombre; pero se hablará más adelante de esto.

    Lo primero que hicieron los españoles fue enviar al padre de Viernes a que viera si entre aquellos salvajes había algún conocido suyo y si le era posible entender su idioma. El anciano los examinó detenidamente, pero todos ellos le resultaron desconocidos, y tampoco consiguieron entender una sola palabra de cuanto les dijo, ni siquiera sus signos, salvo una de las mujeres.

    Eso bastó para que se alcanzara el fin perseguido, haciendo saber a los salvajes que los hombres en cuyas manos habían caído eran cristianos que aborrecían la sola idea de comer carne humana, y que podían tener la seguridad de que estaban a salvo de todo peligro. Tan pronto como comprendieron aquello dieron señales de profundo regocijo, manifestándolo de la manera más extravagante y variada que pueda imaginarse, sobre todo porque había indígenas pertenecientes a distintos pueblos.

    La mujer que hacía de intérprete fue inducida a preguntarles en segundo término si se hallaban dispuestos a trabajar como criados de los hombres que venían de salvarles la vida. Al oír y entender la pregunta, todos se pusieron a danzar, y luego se precipitaron a recoger diversos objetos que había en el suelo y los pusieron sobre sus hombros, en señal de que estaban resueltos a trabajar para sus amos.

    Luego de discutirlo un poco, los cinco ingleses tomaron cada uno una mujer por esposa, y de esa manera principió para ellos una nueva forma de vida, mientras los españoles y el padre de Viernes seguían habitando en mi castillo, que agrandaron sobremanera en el interior. Los tres sirvientes apresados durante la batalla de los salvajes permanecían con ellos, y entre todos formaban como la capital de la colonia, proveyendo al resto de alimentos y acudiendo en su ayuda con todo cuanto tenían a su alcance para sacarlos de apuros.

    Lo extraordinario de este episodio es el hecho de que individuos tan díscolos, tan poco dispuestos a contemporizar, estuvieran completamente de acuerdo en lo que se refiere a las mujeres, y que dos de ellos no pretendieran quedarse con una de las nativas, siendo que había dos o tres que eran incomparablemente más agraciadas que las restantes. Eligieron sin embargo, el mejor procedimiento para evitar luchas, pues luego de encerrar a las cinco mujeres en una choza, se reunieron en la otra y tiraron suertes para decidir quién elegiría el primero.

    Aquel a quien la suerte designó fue a la choza donde esperaban las pobres y desvalidas mujeres y eligió la de su preferencia, siendo de notarse que a pesar de la ventaja en el sorteo se decidió por la más fea y de más edad del conjunto, lo que le valió no pocas bromas de sus compañeros y hasta de los españoles. Pero el hombre había pensado sensatamente, considerando que lo que principalmente se esperaba de aquellas mujeres era dedicación y trabajo y estuvo acertado porque aquélla fue la mejor de las cinco.

    Cuando las infelices se vieron colocadas en línea y elegidas una por una, el terror volvió a dominarlas y creyeron firmemente que al fin serían devoradas. El primero de los marineros ingleses eligió y quiso sacar a una de la tienda, pero las otras iniciaron un clamoreo desgarrador, aferrándose convulsivamente a la que era separada de ellas y despidiéndose con tales demostraciones de desesperación que hubiera ablandado el más duro corazón de la tierra. Fue imposible a los ingleses convencerlas de que en modo alguno iban a asesinarlas, hasta que buscando al padre de Viernes, le hicieron explicar a las mujeres que la única intención que tenían al elegir en esa forma era la de que se convirtiesen en sus esposas.

    Cuando, después de esas palabras tranquilizadoras, el espanto de las mujeres se calmó un poco, los ingleses ayudados por los españoles se pusieron a construir nuevas chozas para que cada pareja tuviese su casa, ya que las existentes estaban llenas con las herramientas, efectos y provisiones. Los tres bribones ocuparon un sitio más apartado, y los dos honestos más cerca, pero todos en el lado norte de la isla, de manera que continuaron separados como antes. Mi isla quedó por lo tanto poblada en tres sitios distintos, como si dijéramos que tres ciudades habían sido fundadas en ella.

    Es digno de hacer notar aquí que, como ocurre con frecuencia en el mundo (cuando los sabios fines de Dios se muestran de una manera para nosotros incomprensibles), los dos ingleses honestos tenían las esposas de menos méritos mientras los tres malvados, que valían menos que la soga para ahorcarlos, incapaces de hacer nada y mucho menos de reformarse o ser útiles a los demás, habían tenido la suerte de que les tocaran tres esposas inteligentes, industriosas y activas. No quiero decir con esto que las otras fuesen malas en cuanto a su carácter o costumbres, ya que todas se mostraban sumisas, humildes y llenas de mansedumbre, sino solamente que no podían compararse a las restantes en cuanto a inteligencia, habilidad y pulcritud.

    Otra observación merece hacerse aquí en homenaje a la aplicación y en reproche a la desidia y la negligencia. Cuando llegué yo a la isla y pude ver las mejoras practicadas, así como las plantaciones de las pequeñas colonias, observé que los dos ingleses habían sobrepasado en mucho a los otros tres. Cierto que ambos grupos tenían sembrado suficiente grano para proveer con holgura a sus necesidades ya que, como mi experiencia y las leyes de la naturaleza parecen demostrarlo, no había razón para plantar mayor cantidad de la necesaria; sin embargo, las diferencias que se observaban en los sembrados, en las empalizadas y muchos otros detalles, eran visibles de inmediato.

    Las esposas de los tres ingleses eran muy listas y hacendosas en sus casas; habían aprendido el modo de cocinar tal como se los enseñara uno de ellos que había sido ayudante de cocinero a bordo, y eran capaces de aderezar muy bien la comida para sus esposos, en tanto que las otras dos no consiguieron nunca aprender ese arte, por lo cual el esposo que sabía de cocina se ocupaba de la tarea. En cuanto a los tres ingleses, no hacían otra cosa que vagabundear, buscando huevos de tortuga, pescando y cazando; en una palabra, todo menos trabajar, por lo cual carecían de muchas cosas. Los dos más tesoneros vivían bien y con holgura, mientras los holgazanes lo hacían con grandes dificultades y miserablemente. Así ha de suceder, según creo, en cualquier parte del mundo.






	\chapter{La invasión de los caníbales}





    Llego ahora a un episodio distinto de todo lo ocurrido antes, tanto a los colonos como a mí mismo. He aquí el relato de lo acontecido.

    Una mañana muy temprano arribaron a la costa cinco o seis canoas de indios salvajes, llamadlos como queráis, y no hay duda de que la razón de su desembarco era la de devorar a algunos prisioneros. Este proceder era ya tan familiar a los españoles, así como a mis compatriotas, que no se afligían mayormente por ello, pues la experiencia les había demostrado que la sola precaución a tomar era la de mantenerse oculto todo el tiempo que durara la permanencia de los caníbales en la isla, tras lo cual podrían reanudar su vida corriente; estaba claro además que los salvajes seguían ignorando que la isla tuviese otros habitantes. Sabedores, pues, de su arribo, los colonos se apresuraron a comunicar la novedad a los de las otras plantaciones para que a su vez permanecieran a puertas cerradas, dejando solamente un observador que diera la buena nueva de que las canoas volvían a hacerse a la mar.

    Todo esto estaba muy bien pensado, pero un suceso desastroso vino a dar por tierra con las medidas tomadas e hizo conocer a los salvajes que había habitantes en la isla, lo cual trajo por lamentable consecuencia una catástrofe que estuvo a punto de asolar íntegramente la colonia. Ocurrió que apenas alejadas las canoas, los españoles se fueron a reconocer los alrededores, y algunos sintieron curiosidad por bajar a la playa y observar el sitio donde habían estado reunidos los caníbales. Allí, con la imaginable sorpresa, encontraron a tres salvajes profundamente dormidos en el suelo. Era de suponer que, después de hartarse con su bárbaro festín a la manera de las bestias, habían caído vencidos por el sueño y no se movieron cuando los otros retornaron a las canoas, salvo que hubiesen ido a recorrer los bosques, encontrando al regreso que los compañeros se habían marchado.

    Los españoles se quedaron pasmados ante la escena, sin saber a qué atinar. El gobernador español que los acompañaba y al cual pidieron consejo tampoco pudo ayudarlos a salir de dudas. Tenían bastantes esclavos para incorporar otros a la colonia, y en cuanto a matarlos allí mismo ninguno se sentía inclinado a hacerlo. El gobernador me dijo más tarde que les resultaba intolerable la idea de verter sangre inocente; para ellos, aquellos pobres nativos no eran culpables de daño alguno puesto que no habían invadido sus propiedades y carecían de motivo para arrebatarles la vida. Después de consultarse, resolvieron volver a sus escondites hasta que aquellos hombres se marcharan de la isla, pero entonces advirtió el gobernador que los salvajes carecían de canoa y que si se los dejaba vagabundear, terminarían descubriendo la presencia de los pobladores.

    Regresaron entonces a la playa donde los tres salvajes seguían profundamente dormidos, y decidieron despertarlos y hacerlos prisioneros. Así fue, con no poco espanto de los pobres nativos al verse atados, ya que al igual que las mujeres pensaban que no tardarían en ser asesinados y comidos. No hay duda de que aquellos hombres creen que todo el mundo hace lo que ellos, pero pronto se los arrancó de su error y se los condujo a lugar seguro.

    Fue una suerte que no los llevaran al castillo, es decir, a mi palacio bajo la colina, sino que primeramente los condujeron a la enramada donde tenían el centro de sus tareas rurales tales como cuidar del ganado y los plantíos; más tarde los trasladaron a la morada de los dos ingleses.

    Allí se los puso a trabajar, aunque no era mucha la tarea que para ellos había. Ignoro si existió negligencia de parte de sus guardianes, o si pensaron que los prisioneros no lograrían alejarse, pero el hecho es que uno de ellos se escapó, perdiéndose entre los bosques, donde fue imposible encontrarlo.

    Pronto se convencieron los colonos que el fugitivo había conseguido volver a su país embarcándose en alguna de las canoas que, llenas de salvajes arribaron tres o cuatro semanas más tarde y se marcharon a los dos días después de su habitual banquete. Este pensamiento los aterrorizó, pues era lógico pensar que el salvaje no tardaría en revelar a sus compatriotas que la isla estaba habitada. Como ya he observado antes, al salvaje no se le había dicho por fortuna cuántos hombres contaba la colonia y dónde vivían, así como tampoco había podido escuchar nunca un disparo de escopeta; habían tenido cuidado de mantener en secreto los demás refugios tales como la gran caverna en el valle, el abrigo que los dos ingleses habían hecho, y otras cosas.

    El primer testimonio de que el salvaje había dado la alarma lo tuvieron los colonos unos dos meses más tarde cuando seis canoas conteniendo siete u ocho hombres cada una vinieron a remo por el lado norte de la isla donde jamás acostumbraban desembarcar, y tocaron tierra una hora antes de la puesta del sol, en un excelente sitio a una milla de distancia de las chozas de los dos ingleses a cuyo cargo había quedado justamente el fugitivo. Como dijo más tarde el gobernador español, si todos los colonos hubiesen estado allí en ese momento, la victoria hubiera sido suya sin duda alguna y ni un solo salvaje hubiese escapado, pero la cosa era muy distinta para dos hombres frente a cincuenta. Los ingleses tuvieron la suerte de advertir las canoas cuando se hallaban a una legua de la costa, de modo que aún transcurrió una hora antes de que tocaran tierra, y como lo hicieron a una milla del sitio donde se alzaban las chozas, pasó otro rato antes de que llegasen a ellas. Con suficientes razones para creerse traicionados, lo primero que hicieron los colonos fue atar a los dos esclavos que les quedaban tras lo cual ordenaron a dos de los tres hombres que habían sido traídos junto con las mujeres y que les eran extraordinariamente fieles, que se llevaran consigo a los prisioneros así como a las mujeres y cuanto pudieran transportar; la orden era ocultar a todos en el refugio de los bosques del que ya he hablado, y atar allí a los dos salvajes manteniéndolos bien amarrados hasta que recibieran nuevas instrucciones.

    En segundo lugar, y advirtiendo que los enemigos se dirigían directamente hacia donde estaban sus habitaciones, los colonos abrieron los vallados tras los cuales estaban los rebaños de cabras y sacaron afuera el ganado, dejando a las cabras que retozaran a su gusto en los bosques a fin de que los caníbales las creyeran salvajes. Sin embargo el bribón que venía guiando a los invasores era demasiado astuto para creer tal cosa, y les había señalado con todo detalle el emplazamiento de las cabañas, pues se dirigieron sin vacilar a ellas.

    Una vez que los dos asustados colonos hubieron puesto en seguridad a sus esposas y sus bienes, enviaron al tercer esclavo que tenían a su servicio para que corriera a dar la noticia a los españoles y les reclamara auxilio. Tomando sus armas y municiones", se retiraron entonces al refugio del bosque donde estaban ya sus mujeres, y desde esa distancia trataron de observar la conducta de los salvajes.

    No se habían alejado mucho cuando, desde un altozano, vieron el pequeño ejército de los caníbales encaminarse directamente a las cabañas, advirtiendo un instante después que las chozas habían sido incendiadas para su cólera y desesperación, ya que aquella pérdida era gravísima, por lo menos durante algún tiempo. Permanecieron observando desde su refugio hasta advertir que los salvajes se dispersaban por los alrededores como bestias salvajes explorando cada sitio, y en toda forma imaginable, a la caza de algún botín y en especial de los moradores de las cabañas cuya existencia no podía ser ya puesta en duda por ellos.

    Viendo esto los ingleses, y al comprender que el refugio donde estaban no era ya muy seguro, pues en cualquier momento uno de los salvajes podía encaminarse hacia allí y tras él muchos otros, consideraron conveniente efectuar una segunda retirada apostándose media legua más atrás; creían ellos, como efectivamente aconteció, que cuanto más avanzaran los salvajes, más se dispersarían, y eligieron como segundo refugio la entrada de la parte más espesa del bosque, donde había un gran tronco de árbol hueco y corpulento, tras el cual tomaron posición ambos colonos resueltos a enfrentarse a lo que se presentara.

    No llevaban allí mucho tiempo cuando dos salvajes se acercaron corriendo en su dirección como si hubiesen sabido que estaban allí y se dispusieran a atacarlos; vieron que detrás venían otros tres, y más lejos un nuevo grupo de cinco, todos siguiendo el mismo camino; aparte de ésos divisaron siete u ocho que corrían en otra dirección, y en conjunto producían la impresión de cazadores que están dando una batida.

    Los pobres colonos se sintieron perplejos, no sabiendo si debían quedarse y mantener la posición o escapar al punto, pero después de un rápido debate comprendieron que si los salvajes continuaban batiendo los alrededores antes de que llegara auxilio, probablemente descubrirían el escondite en los bosques y se adueñarían de todo. Decidieron, pues, hacerles frente donde estaban, y si resultaban demasiados para contenerlos, trepar a lo alto del árbol desde el cual contaban con defenderse mientras les durasen las municiones y aunque fueran atacados por toda la horda enemiga, que contaba cerca de cincuenta hombres; salvo que a éstos se les ocurriera prender fuego al tronco. Ya resueltos, se trataba de decidir si dispararían sobre los dos primeros o si era preferible quedarse a la espera de los otros tres para atacar al grupo central, separando en esa forma a los dos primeros de los cinco últimos. Dejaron por fin pasar a los dos que venían en primer término, dispuestos a no atacarlos salvo en caso de que fueran descubiertos por ellos. Los indígenas, como si estuviesen de acuerdo, cambiaron de dirección alejándose un poco hacia otro lado del bosque, pero los tres que los seguían y los otros cinco marcharon directamente hacia el árbol como sospechando que los ingleses se escondían allí.

    Al verlos avanzar con tanta resolución, decidieron aprovechar que venían casi en línea, y apuntaron a uno después de otro para herir al menos a tres con el primer disparo. El que iba a tirar metió tres o cuatro balines en su mosquete y por un agujero del tronco a modo de tronera tuvo tiempo de apuntar cuidadosamente sin que lo viesen, esperando a que los salvajes estuvieran a treinta yardas del árbol y que el tiro resultara certero.

    El colono era demasiado buen tirador como para errar el blanco; aprovechando que los salvajes se acercaban uno detrás de otro y en fila, disparó alcanzando certeramente a dos de ellos. El primero cayó muerto de un tiro en la cabeza; el segundo, que era el indio fugitivo, fue alcanzado por una bala en el cuerpo y cayó aunque no muerto, mientras el tercero, que solamente había recibido un rasguño en el hombro, acaso de la misma bala que atravesara el cuerpo del segundo, se dejaba caer gritando y aullando de la manera más horrorosa.

    Los cinco que venían detrás, más asustados por el ruido que por el peligro mismo, se detuvieron al punto, ya que los bosques hacían el sonido mil veces más fuerte de lo que realmente era, multiplicándolo con el eco que venía de todas partes como con los chillidos de las aves que alzaban el vuelo gritando con toda la variedad posible de sonidos, lo mismo que me había sucedido a mí cuando disparé el balazo que fue quizás el primero que se escuchaba en aquella isla.

    Pronto volvió a reinar el silencio, y los indios, que todavía no habían comprendido de lo que se trataba, avanzaron temerariamente hasta el sitio donde yacían sus compañeros en tan miserable estado; allí, y sin darse cuenta de que iban a ser víctimas del mismo daño, los infelices se agruparon en torno del herido hablando todos a la vez y seguramente preguntándole quién y cómo lo habían herido; es de imaginar que el otro les contestó que el rayo y el trueno de los dioses habían matado a dos de ellos y alcanzado a herirlo. Digo que es de imaginar, por cuanto no lograban ver a ningún enemigo por los alrededores y al mismo tiempo desconocían las armas de fuego y sus mortíferos efectos; de ninguna manera podían concebir la muerte a distancia por medio de fuego y balas. De lo contrario es de suponer que no se hubieran quedado contemplando tan torpemente el triste destino de sus compañeros sin sentir alguna aprensión por el suyo propio.

    Nuestros dos hombres, aunque como me dijeron más tarde sentían verse obligados a matar a esos desdichados salvajes inconscientes del peligro, no podían sin embargo desperdiciar la ocasión ahora que los tenían en sus manos, y por eso el primero volvió a cargar su arma y luego de ponerse de acuerdo sobre cuáles serían sus blancos, hicieron una doble descarga que mató o hirió gravemente a cuatro enemigos; el quinto, paralizado de espanto aunque sin un rasguño, cayó como fulminado con los otros, de modo que nuestros hombres creyeron que habían alcanzado a matarlos a todos.

    Esta creencia los llevó a salir imprudentemente del refugio del árbol antes de tener la precaución de cargar otra vez las piezas; fue por cierto un grave error y sintieron no poca sorpresa cuando, al llegar al sitio, encontraron a cuatro hombres vivos, dos de ellos levemente heridos y otro sin la menor lesión. Esto los obligó a caer sobre el enemigo con la culata de los mosquetes, lanzándose primero sobre el salvaje causante de todo lo que sucedía y luego sobre otro, herido en la rodilla, a quienes libraron en un instante de sus dolores. El hombre que no había recibido heridas vino entonces hacia ellos y cayó de hinojos, alzando las manos juntas y haciendo toda clase de demostraciones suplicantes, mientras con lastimeros gemidos pedía que le perdonaran la vida; ellos, naturalmente, no entendieron una sola palabra de cuantas les dijo.

    Le ordenaron por fin que fuera a sentarse al pie de un árbol y uno de los ingleses, con un trozo de cordel que por feliz coincidencia llevaba en el bolsillo, le ató fuertemente los pies, así como también las manos a la espalda, y fue entonces a perseguir con toda la rapidez posible a los otros dos salvajes que dejaran pasar al principio, temerosos de que de un momento a otro descubrieran el escondite en el bosque donde habían hecho llevar a sus mujeres y los pocos bienes que poseían. Por un instante divisaron a los dos salvajes, pero estaban demasiado lejos; por fin, y con gran contento, vieron que los enemigos tomaban por un valle que iba hacia el mar, o sea el camino opuesto al que llevaba a su refugio. Satisfechos con esto retornaron al sitio donde habían dejado al prisionero que, como ya lo sospechaban, había sido entretanto libertado por sus camaradas, pues no vieron señales de él, y solamente hallaron al pie del árbol los pedazos del cordel con que lo ataran.

    Volvían ahora a sentirse preocupados, sin saber qué actitud tomar, ignorando si el enemigo andaba cerca y cuál era su número; resolvieron entonces acudir al refugio donde estaban sus mujeres para comprobar si nada les había ocurrido y tranquilizarlas, ya que imaginaban el espanto que estarían pasando. Cierto que los salvajes eran compatriotas de aquellas mujeres, pero ellas les tenían igualmente mucho miedo, y quizá todavía más por el hecho de que conocían harto bien sus costumbres.

    Al llegar al lugar vieron que los salvajes habían estado explorando el bosque muy cerca del escondite pero sin dar con él. Ciertamente que era inaccesible por los gruesos y juntos que aparecían los árboles, como ya se ha dicho, y nadie que no tuviera un guía sabedor del lugar hubiese podido descubrir aquel refugio. Los colonos hallaron todo sin novedad, salvo las mujeres, que sentían mucho miedo. Mientras permanecían allí tuvieron la alegría de que siete de los españoles acudieran en su ayuda, en tanto los otros diez con sus sirvientes y el viejo Viernes (quiero decir el padre de Viernes) habían formado un grupo para defender la enramada y el ganado y provisiones allí acumulados, en caso de que a los salvajes se les diera por explorar aquella parte de la isla, cosa a la que no se atrevieron.

    Con los siete españoles vino uno de los tres salvajes tomados prisioneros en la anterior ocasión, y también el salvaje a quien los dos ingleses habían dejado atado de pies y manos al pie de un árbol; ocurrió que la partida de españoles pasó por ese sitio, hallando el campo de batalla con los siete muertos, y luego de desatar al salvaje lo obligaron a que marchara con ellos hasta el escondite donde nuevamente fue amarrado en compañía de los otros dos esclavos que mantenían asegurados, pues eran de la partida del fugitivo que lograra huir de la isla.

    Aquellos prisioneros principiaron de inmediato a ser motivo de preocupación para los españoles, y tanto temían la posibilidad de que alguno se escapara, que en determinado momento resolvieron matarlos a todos, convencidos de que la seguridad de la colonia exigía aquella medida. El gobernador español se negó, sin embargo, a dar su consentimiento, ordenando en cambio que los prisioneros fueran llevados a la gran caverna del valle y mantenidos allí con una guardia de dos españoles que se encargarían a la vez de darles de comer, cosa que fue cumplida esa misma noche dejándolos atados de pies y manos en la caverna.

    Cuando se sintieron protegidos por los españoles, los dos ingleses recobraron de tal modo el coraje que no pudieron quedarse más tiempo inactivos en el refugio. Con las precauciones del caso resolvieron ir hasta su arruinada plantación, pero cuando les faltaba poco para llegar a ella y alcanzaron la línea de la costa, vieron con claridad a los salvajes que se embarcaban en sus canoas para hacerse a la mar.

    Al principio lo lamentaron, ya que no había tiempo de ponerse a tiro para hacerles una descarga de despedida, pero pronto se sintieron muy satisfechos de haberse librado de los enemigos.

    Por segunda vez se veían los pobres ingleses completamente arruinados y con sus bienes destruidos; el resto de la colonia decidió acudir en su ayuda mientras reconstruyeran las chozas y darles entretanto todo lo que necesitaran. Los otros tres ingleses, que no se destacaban sin embargo por su inclinación al bien, tan pronto se enteraron de lo ocurrido vinieron de inmediato (ya que viviendo mucho más al este no habían sabido nada de la lucha hasta que todo hubo cesado) y ofrecieron su ayuda y asistencia, trabajando con toda cordialidad durante muchos días hasta reconstruir las chozas y disponer lo necesario para sus habitantes; así en poco tiempo, ambos colonos pudieron reanudar su vida habitual.

    Dos días después tuvieron la satisfacción de encontrar tres canoas arrojadas a la costa, y a cierta distancia los cuerpos de dos salvajes ahogados, y recordando que la noche en que se marcharan de la isla había soplado un viento muy fuerte dedujeron que una tormenta había sorprendido a la flotilla en alta mar.

    Con todo, si algunos perecieron, es evidente que buen número alcanzó a salvarse llevando a los demás la noticia de lo ocurrido y de cuanto les había pasado en la isla, excitándolos así a intentar otro desembarco de la misma naturaleza que, por lo visto, resolvieron realizar con suficientes fuerzas para asolar cuanto se les presentara. Cierto que, fuera de lo que les había dicho el salvaje fugitivo acerca de los habitantes de la isla, nada podían ellos agregar por experiencia propia, ya que no alcanzaron a ver a ninguno; y como el primer informante había muerto, no tenían otros testigos que pudiesen confirmar su aserto.

    Pasaron cinco o seis meses antes de que tuvieran nuevas noticias de los salvajes, y durante ese tiempo nuestros hombres abrigaron esperanzas de que hubiesen olvidado su fracaso o bien que no se creyeran con fuerzas para desquitarse, cuando he aquí que repentinamente fueron invadidos por una formidable flota que no bajaba de veintiocho canoas atestadas de salvajes armados de arcos y flechas, pesadas mazas, espadas de madera y otras armas de guerra; tan grande era su número que al calcularlo los colonos cayeron en la más profunda consternación.

    Como la flota llegó a la costa por la tarde y desembarcaron en el extremo, oriental de la isla, los colonos tuvieron toda la noche para consultarse sobre lo que debían hacer. En primer término, y considerando que mantenerse completamente ocultos había sido hasta entonces su único medio de salvación, y mucho más ahora que el número de salvajes era tan grande, resolvieron derribar las dos chozas de los colonos ingleses y conducir su ganado a la caverna, pues tenían la seguridad de que apenas fuera de día los salvajes se encaminarían directamente hacia allá para repetir su anterior devastación, pese a que ahora habían tocado tierra a más de dos leguas de distancia.

    En segundo término retiraron las cabras que tenían en la vieja enramada, y que pertenecían a los españoles, tratando de borrar en lo posible toda huella de la presencia humana en esos parajes. Entonces, cuando vino el día, concentraron sus fuerzas en la plantación de los dos colonos, aguardando el avance enemigo. Ocurrió tal como lo presumían: los nuevos invasores, dejando las canoas en la extremidad oriental de la isla, corrieron a lo largo de la costa y luego directamente hacia aquel lugar, en número de unos doscientos cincuenta, según calcularon nuestros hombres. El ejército defensor era harto pequeño, y para colmo las armas ni siquiera alcanzaban. El total, según me parece, era el siguiente. Ante todo los hombres:



    Diecisiete españoles. Cinco ingleses.

    Uno, el viejo Viernes, o sea el padre de Viernes.

    Tres esclavos apresados junto con las mujeres, y que eran muy fieles.

    Tres esclavos que vivían con los españoles.



    Las armas que poseían eran las siguientes:

    Once mosquetes.

    Cinco pistolas.

    Tres escopetas de caza.

    Cinco mosquetes o escopetas de caza tomadas por mí a los marineros amotinados cuando los sometí.

    Dos espadas.

    Tres viejas alabardas.



    Los esclavos no recibieron ninguna arma de fuego, pero cada uno tenía una alabarda, o más bien un asta o bastón largo con una punta de hierro asegurada al extremo, además de una hachuela que llevaba en el costado; nuestros hombres tenían también un hacha cada uno. Dos de las mujeres, a quienes no se pudo impedir que participaran en el combate, llevaban arco y flechas que los españoles habían recogido después del episodio ya narrado cuando los salvajes combatieron entre sí; las dos mujeres tenían asimismo hachas al costado.

    El gobernador español, del cual he hablado tantas veces, era quien los comandaba, y William Atkins, que aunque altamente peligroso por su perversidad era el más temerario y valiente de los hombres, fue su teniente. Los salvajes corrieron al ataque como leones, y nuestros hombres los esperaron careciendo, para mayor desdicha, de una buena posición de defensa. Will Atkins, sin embargo, que en aquella ocasión demostró su valer, se había apostado con seis hombres detrás de un tupido matorral como guardia avanzada, con órdenes de dejar pasar a los primeros y disparar luego sobre el centro de los atacantes, apresurándose de inmediato a retroceder con toda la rapidez posible y, dando un rodeo por los bosques, volver por la retaguardia de los españoles que esperaban a su vez apostados detrás de un bosquecillo.

    Cuando aparecieron los salvajes, lanzándose al ataque en desordenados grupos, William Atkins dejó pasar a unos cincuenta de ellos; viendo luego venir al resto en grupo cerrado, ordenó a tres de sus hombres que tiraran, después de haber cargado los mosquetes con seis o siete balas de pistola cada uno. Cuántos alcanzaron a matar o herir no lo supieron, pero la consternación y la sorpresa de los salvajes fue indescriptible. Aterrados hasta lo indecible al oír tan espantoso estruendo, contemplaban caer a sus compañeros muertos o heridos sin poder precisar quiénes los atacaban. En medio de esa confusión, William Atkins y los otros tres tiraron nuevamente sobre el grupo más espeso, y menos de un minuto después vino una tercera descarga hecha por los tres primeros que ya habían tenido tiempo de cargar sus piezas.

    Si Atkins y su gente se hubieran retirado inmediatamente después de las descargas como se les había ordenado, o si el resto de los defensores hubiese estado a distancia conveniente para mantener un continuo fuego, los salvajes hubieran sido totalmente arrollados porque el terror que los dominaba venía principalmente de lo que suponían el fuego y el trueno de los dioses, ya que no veían a quienes los estaban hiriendo. Pero Will Atkins, quedándose para cargar otra vez las armas, les reveló la verdad de lo que ocurría; algunos salvajes, que a distancia habían alcanzado a divisarlos, cargaron sobre ellos, y aunque Atkins y los suyos dispararon dos o tres veces sus armas y mataron a cerca de veinte, retirándose después con toda la rapidez posible, los salvajes por su parte alcanzaron a herir al mismo Atkins y mataron a uno de sus compatriotas a flechazos, tal como más tarde eliminaron a un español y a uno de los esclavos indios que habían venido con las mujeres. Aquel esclavo, de una bravura extraordinaria, combatió desesperadamente, llegando a matar a cinco enemigos en lucha cuerpo a cuerpo sin otras armas que una de las alabardas y el hacha.

    Habiendo muertos dos hombres, y con Atkins herido, el grupo avanzado retrocedió a un terreno más alto en el interior del bosque; en cuanto a los españoles, después de tres descargas consecutivas, se replegaron a su turno. Tan grande era el número de los atacantes y tan encarnizados se mostraban —aunque tenían ya más de cincuenta muertos y muchos heridos— que se lanzaron nuevamente al ataque despreciando el peligro, y sus flechas llovieron como una nube. Era de admirar que aquellos guerreros heridos pero no imposibilitados para la lucha parecían enfurecerse más al sentirse lastimados y peleaban como bestias salvajes.

    Al retirarse los nuestros dejaron abandonados sus dos muertos el español y el inglés; entonces los salvajes, abalanzándose sobre los cadáveres, los despedazaron de la manera más horrible, rompiéndoles brazos, piernas y cabezas con golpes de maza y de espada, mostrando así su monstruoso salvajismo. Aunque advirtieron que los defensores habían retrocedido, no mostraron intenciones de perseguirlos sino que formaron una especie de círculo, lo que según parece constituye una costumbre entre ellos, y lanzaron un doble alarido en señal de victoria; tras de lo cual tuvieron sin embargo el disgusto de ver morir a varios de los suyos que yacían heridos y que no pudieron resistir la pérdida de sangre.

    Luego de reunir su pequeña fuerza en una eminencia, el gobernador español vio que Atkins, aunque herido, quería cargar de inmediato sobre los enemigos, y se dirigió a él diciéndole:

    —Señor Atkins, ya habéis visto cómo pelean los salvajes heridos; mejor será entonces dejarlos tranquilos hasta mañana, en que la pérdida de sangre los habrá debilitado y sentirán todo el dolor y el entumecimiento de sus llagas; de esa manera serán muchos menos los que tendremos que combatir.

    El consejo era bueno, pero a él respondió alegremente Will Atkins:

    —Es muy cierto, señor, pero a mí me pasará lo mismo que a ellos y es por eso que quisiera reanudar la lucha mientras me duren las fuerzas.

    — ¡Bravo, señor Atkins! —exclamó el español—. Habéis peleado valientemente y cumplido con vuestro deber; si mañana no estáis en condiciones de hacerlo nosotros os reemplazaremos, pero hasta entonces creo mejor esperar.

    Así se hizo, pero como la noche era de plenilunio y a su luz vieron que los salvajes permanecían en gran desorden en torno a sus muertos y heridos, decidieron por fin caer sobre ellos aprovechando ambas cosas y ver de hacerles una buena descarga antes de ser descubiertos. Esto resultó factible pues uno de los ingleses, en cuyo dominio se había desarrollado la lucha, los guió por entre los bosques y luego hacia el oeste de la costa, hasta que girando rumbo al sur los trajo tan cerca de los salvajes que, antes de que fuesen vistos u oídos, ocho hombres hicieron una descarga cerrada contra el grupo más numeroso de enemigos, ocasionando espantosa matanza. Medio minuto más tarde otros ocho hombres volvieron a tirar, habiendo puesto municiones en tal cantidad que gran número de salvajes resultaron muertos o heridos, y todo esto sin que alcanzaran a ver a quienes de tal modo los exterminaban ni el camino mejor para emprender la fuga.

    Con la mayor rapidez posible los españoles cargaron otra vez sus armas y luego, dividiéndose en tres cuerpos, resolvieron caer al mismo tiempo sobre el enemigo. Cada cuerpo contaba con ocho combatientes, lo que sumaba veinticuatro, de los cuales veintidós hombres y dos mujeres, las que, dicho sea de paso, pelearon denodadamente.

    Dividieron las armas de fuego entre ellos, y lo mismo hicieron con las alabardas y picas; hubieran querido que las mujeres permaneciesen en la retaguardia, pero éstas seguían dispuestas a morir junto a sus esposos. Formados así, salieron del bosque y se precipitaron sobre el enemigo lanzando terribles gritos con toda la fuerza de que eran capaces. Los salvajes se agruparon precipitadamente pero en la más espantosa confusión, al oír los alaridos de nuestros hombres que venían desde tres direcciones distintas. De haberlos visto, hubieran luchado sin duda y por cierto que, apenas llegaron a enfrentarse con ellos, hubo muchos disparos de flechas y el pobre padre de Viernes fue herido, aunque no de gravedad. Pero nuestros hombres no les dieron tiempo de rehacerse, pues corriendo hacia ellos desde tres sitios, descargaron al unísono sus armas y luego se precipitaron en lucha cuerpo a cuerpo armados con las culatas de sus mosquetes, las espadas, alabardas y hachuelas; tan bien las emplearon que muy pronto, con aullidos y gritos de desesperación, los salvajes se dispersaron a toda carrera y sin rumbo fijo, tratando de salvar la vida.

    Nuestros hombres estaban agotados de fatiga; en los dos combates habían matado o herido mortalmente a casi ciento ochenta enemigos. El resto, aterrado hasta perder completamente la cabeza, se dispersó por bosques y colinas con toda la rapidez que el miedo y sus ágiles pies podían prestarles; y como los nuestros no se preocuparon mucho por perseguirlos, terminaron por descender a la costa y reunirse en el sitio donde habían dejado las canoas. Sin embargo, otro desastre les esperaba allí, pues aquella noche sopló un furioso huracán del lado del mar impidiéndoles la menor tentativa de fuga. La tormenta duró la noche entera, y cuando vino la marea arrastró las canoas tan adentro de la costa que les costó infinito trabajo volver a botarlas, mientras no pocas se destrozaban contra las rocas de la playa o al chocar entre sí.

    Aunque muy contentos con la victoria, los nuestros apenas tuvieron descanso esa noche; habiéndose refrescado lo mejor posible, decidieron encaminarse al sitio donde debían estar los salvajes y ver en qué situación se encontraban. Pasaron naturalmente por el lugar donde se había librado la batalla, y hallaron a varios infelices que aún no habían muerto, aunque no había para ellos esperanza de vida. Fue aquél un penoso espectáculo para seres de alma generosa, ya que un hombre cabal, aunque obligado por la ley de la guerra a destruir a su oponente, no encuentra sin embargo, placer en su desgracia. Con todo, no les fue necesario adoptar providencia en este caso, pues los mismos salvajes que les servían de esclavos se apresuraron a rematar a aquellos heridos con sus hachas.

    Por fin, llegaron al sitio donde se encontraba congregada en la forma más lastimosa el resto del ejército salvaje, del cual quedaban aún cerca de cien hombres. Casi todos aparecían sentados en la arena, con el mentón apoyado en las rodillas y la cara cubierta por las manos en actitud de abatimiento.

    Cuando los nuestros estuvieron a dos tiros de mosquete, el gobernador español ordenó que se hicieran dos disparos sin bala, para alarmarlos. Deseaba averiguar por su reacción qué podía esperarse de ellos, es decir, si aún les quedaban deseos de pelear o estaban tan absolutamente abatidos que todo su ánimo se hubiera perdido y fuese entonces posible proceder en otra forma con ellos.

    La estratagema dio resultado, pues tan pronto los salvajes oyeron el primer disparo y vieron el resplandor del segundo, se levantaron con señales de profunda consternación, y como nuestros hombres avanzaban entretanto rápidamente hacia ellos, corrieron en confusión lanzando horribles alaridos y haciendo todos una especie de clamoreo cuyo sentido no comprendían los de nuestro bando por no haberlo oído jamás antes; y así, encaramándose por las colinas, los salvajes se dispersaron en el interior de la isla.

    Los colonos habían deseado al principio que el tiempo estuviese sereno a fin de que los derrotados salvajes huyeran al mar, pero no pensaron entonces que esto hubiera significado su pronto retorno en cantidades tan inmensas como para hacer inútil toda resistencia o, por lo menos, que las invasiones se hubieran repetido con tanta frecuencia como para desolar la isla y hacer morir de hambre a la colonia. Fue entonces cuando Will Atkins, quien a pesar de encontrarse herido permanecía junto a sus compañeros, demostró ser el mejor consejero de todos, opinando que debía aprovecharse la ventaja adquirida, situarse entre los fugitivos y sus canoas y privarlos en esa forma de toda posibilidad de que volviesen alguna vez a asolar la isla.

    Discutieron mucho esta idea, y algunos se oponían sosteniendo que era peligroso que los salvajes se refugiaran en los bosques y vivieran allí en constante amenaza para los colonos, obligándolos a salir a cazarlos como a fieras salvajes y cuidarse de todo movimiento así como mantener constante vigilancia sobre los plantíos, con el riesgo de que los rebaños fueran asolados y la vida, por fin, se convirtiera en un motivo de constante angustia.

    A esto repuso Will Atkins que resultaba preferible vérselas con cien salvajes que con cien pueblos, y que así como ahora era necesario destruir las canoas lo mismo habría que hacer más tarde con los hombres, a menos de resignarse a perecer a sus manos. En una palabra, mostró tan claramente la necesidad de lo que aconsejaba, que todos terminaron por convencerse; poniéndose entonces a la tarea buscaron leña seca en el bosque e intentaron incendiar las canoas, lo que no fue posible por el grado de humedad de la madera. De todos modos, el fuego alcanzó a carbonizar la parte superior, tornándolas completamente inútiles para hacerse a la mar. Cuando los indios vieron lo que hacían < con sus piraguas, algunos salieron de los bosques y acercándose todo lo posible a nuestros hombres se dejaron caer de rodillas y gritaron: « ¡Oa, oa! ¡ Waramoka!», así como otras palabras en su idioma que ninguno de los nuestros entendía; pero sí advirtieron los gestos suplicantes y quejumbrosos lamentos por los cuales pedían que no les estropearan las canoas, ofreciendo embarcarse al punto y no regresar nunca más.

    Los colonos se sentían ahora plenamente convencidos de que la única manera de salvar sus vidas y sus bienes estaba en impedir que uno solo de aquellos hombres se alejara rumbo a su nación; era evidente que si algún fugitivo alcanzaba a contar lo acontecido a sus compatriotas, la comunidad sería masacrada. Dándoles entonces a entender que no estaban dispuestos a apiadarse, siguieron destrozando las piraguas que la tormenta no había ya estropeado antes; al ver esto los salvajes dejaron escapar un horrible alarido que fue claramente escuchado por nuestros hombres, tras de lo cual se lanzaron a recorrer la isla como locos, tanto que nuevamente los colonos se encontraron sin saber qué medidas adoptar a su respecto.

    A pesar de toda su prudencia, los españoles no pensaron que mientras exasperaban de esa manera a aquellos hombres hubiera sido necesario mantener una guardia en las plantaciones. Cierto que habían salvado los rebaños de cabras y que los indios no dieron con el refugio principal, es decir, mi viejo castillo en la colina, como tampoco la caverna del valle, pero sí descubrieron mi plantación en la enramada y la redujeron a ruinas, arrancando los vallados y destrozando los plantíos, pisoteando el grano, rompiendo las viñas que estaban entonces casi en su punto y causando a la colonia un daño inmenso aunque sin lograr el menor provecho para sí mismos.

    Aunque los nuestros estaban listos para pelear con los salvajes dondequiera los hallasen, no se sentían sin embargo en condiciones de perseguirlos u organizar una cacería; aquellos hombres eran demasiado rápidos para ellos cuando los sorprendían y a su vez los nuestros no se atrevían a andar solos por miedo de ser repentinamente rodeados.

    Por suerte el enemigo carecía de armas, pues aunque los vencidos conservaban sus arcos habían ya gastado todas las flechas y no poseían material para reponerlas, faltándoles otras armas punzantes que pudieran emplear en su reemplazo.

    La miserable situación a que se veían reducidos era en verdad terrible, pero a la vez los colonos estaban sujetos por su culpa a un estado de cosas altamente peligroso; aunque sus refugios se habían salvado, la mayor parte de las provisiones resultó destruida así como arruinada la cosecha, de forma que no veían la manera de arreglárselas. Lo único a salvo era el ganado, que habían hecho llevar al valle donde estaba la caverna, y algo de grano que crecía en ese lugar; además tenían la plantación de los tres ingleses, William Atkins y sus compañeros, reducidos ahora a dos, por cuanto el otro había sido muerto de un flechazo que lo alcanzó justamente en la sien de tal modo que cayó fulminado; es digno de notarse que se trataba del mismo bárbaro individuo que hiriera de un hachazo a un pobre esclavo y que más tarde pretendiera asesinar a todos los españoles.

    Pienso que la situación de los colonos era por ese entonces peor que la mía en la época en que descubrí los granos de cebada y arroz y me puse a sembrarlos así como a domesticar animales; ellos estaban ahora acechados por lo que podríamos llamar cien lobos dispuestos a devorar cuanto encontraran, aunque difícilmente pudiesen llegar hasta sus personas.

    Lo primero que resolvieron luego de comprender claramente las circunstancias en que vivían fue tratar de que los salvajes quedaran acorralados en la parte más alejada de la isla por el lado S.O. a fin de que si nuevos grupos enemigos desembarcaban no se produjera un encuentro entre ellos; luego se dedicarían a perseguirlos y cazarlos diariamente, matando a todos los que pudieran hasta disminuir su número, y si por fin conseguían reducir por las buenas a los restantes y persuadirlos de que se entregaran buenamente, les darían grano y verían de enseñarles el modo de plantarlo para que vivieran de su trabajo.

    Conforme a esto principiaron a hostigarlos de tal modo, aterrorizándolos con los disparos de las armas de fuego, que pocos días más tarde bastaba hacer una descarga para que los salvajes cayeran al suelo como muertos aunque las balas no los hubiesen rozado; tan aterrados vivían que se fueron alejado más y más, siempre seguidos de cerca por nuestros hombres quienes, matando o hiriendo diariamente a algunos, los obligaron a permanecer ocultos en la profundidad de los bosques y hondonadas y reducidos a la peor miseria por falta de alimentos; muchos de ellos fueron hallados muertos en los bosques, y la carencia de heridas probaba que habían perecido de hambre.

    La contemplación de tan penosas escenas hizo sufrir a los colonos, que se sentían movidos a la piedad, en especial el gobernador español que era el más caballeresco y generoso espíritu que haya yo conocido en toda mi vida; fue él quien propuso que si era posible se apresara vivo a un salvaje para hacerle entender cuáles eran las intenciones de los atacantes y enviarlo luego como intermediario a fin de convencer a los restantes que aceptasen las condiciones impuestas y salvaran así sus vidas evitando a la vez nuevos daños a los colonos.

    No pasó mucho sin que cayera uno en sus manos; tan débil y medio muerto de hambre estaba que resultó fácil capturarlo. Se mostró muy hosco al principio negándose a comer y beber, pero como advirtiera con cuánta amabilidad se le trataba y que se le daban alimentos en vez de atormentarlo, comenzó a mostrarse más dócil y sumiso.

    Le trajeron entonces al anciano padre de Viernes para que hablase frecuentemente con él y le explicara cuan generosos serían los demás con todos ellos, insistiendo en que no sólo respetarían sus vidas sino que les entregarían una parte de la isla para que viviesen en ella, previa promesa de que no saldrían de los límites establecidos y no intentarían traspasarlos con intenciones dañinas. Le prometió que les darían suficiente grano para que tuviesen una plantación propia y no les faltara pan, y que hasta entonces recibirían comida de los colonos. Por fin, el padre de Viernes indicó al salvaje que volviese a reunirse con los suyos y les repitiera sus palabras, asegurándoles además que si no aceptaban de inmediato aquellos términos serían exterminados sin piedad.

    Los pobres infelices, completamente amansados y reducidos apenas a unos treinta y siete, aceptaron sin vacilar la propuesta rogando que se les diera algún alimento; al conocer esto, doce españoles y dos ingleses bien armados y seguidos de tres esclavos y del padre de Viernes, se encaminaron al sitio donde aquéllos se hallaban reunidos. Los esclavos indios llevaban consigo gran cantidad de pan, tortas de arroz secadas al sol y tres cabras vivas; se les ordenó a los salvajes que se congregaran junto a una colina, donde se sentaron a comer aquellas provisiones con profundo agradecimiento, mostrándose mucho más fieles a la palabra empeñada de lo que podría haberse pensado, ya que desde entonces y excepto cuando acudían a pedir vituallas e instrucciones jamás traspasaron los límites que les habían fijado, y allí vivían cuando yo arribé a la isla, por lo cual fui a conocerlos.

    Los españoles les enseñaron a plantar el grano y a hacer pan, el modo de criar cabras y ordeñarlas; de haber tenido mujeres consigo, pronto aquel grupo se hubiera convertido en un verdadero pueblo. Estaban confinados en una lengua de tierra, con elevados peñascos a sus espaldas y una llanura que iba en descenso hacia el mar, mirando al ángulo sudeste de la isla. Tenían tierra de sobra, la que era sumamente fértil; su dominio alcanzaba a medir una milla y medía de ancho por tres o cuatro de largo.

    Nuestros hombres les enseñaron a hacer azadas de madera, tal como yo había procedido antes; les entregaron doce hachuelas y tres o cuatro cuchillos, y así vivieron los salvajes convertidos en las criaturas más dóciles y humildes de que se tenga noticia.

    Después de esto la colonia gozó de absoluta tranquilidad en lo que respecta a los salvajes, hasta que llegué a visitarla unos dos años más tarde; no faltaban a veces algunas canoas de indios que arribaban a la costa para celebrar sus monstruosos festines triunfales, pero como había muchas naciones de caníbales, probablemente ignoraban la existencia de aquellos que arribaran antes que ellos, de manera que nunca mostraron intenciones de explorar la isla o averiguar el destino de sus compatriotas; por otra parte, aunque lo hubiesen intentado era casi imposible que dieran con ellos.

    Con esto me parece haber hecho una completa relación de todo lo sucedido en la colonia hasta mi regreso, por lo menos en cuanto a los episodios dignos de recuerdo. Los indios habían sido admirablemente civilizados por los colonos, y con frecuencia acudían éstos a visitarlos ya que habían prohibido bajo pena de muerte cruzar el límite a los salvajes, a fin de evitar que sus refugios fueran descubiertos por segunda vez.

    Hay algo digno de ser conocido, y es la forma en que enseñaron a los salvajes el arte de la cestería, tanto que bien pronto sobrepujaron a sus maestros; eran habilísimos en la forma de combinar y tejer el mimbre haciendo toda clase de cestas, cedazos, jaulas, armarios y otras cosas, tales como verdaderas sillas donde era posible sentarse, banquillos, camas y variedad de objetos que probaban su ingenio en dicha tarea una vez que habían recibido la iniciación de los colonos.

    Mi llegada a la isla fue particularmente grata a aquellas gentes, porque pudimos proveerlas con cuchillos, tijeras, azadas, palas, picos y todos los instrumentos que pudieran precisar para su trabajo.

    Con ayuda de tales herramientas se mostraron tan habilidosos que se animaron a construir sus viviendas, haciendo chozas verdaderamente bonitas; las paredes estaban tejidas con mimbres a manera de un enorme cesto, lo cual les daba un aspecto extraordinario, pero resultaban doblemente útiles contra el calor y los insectos. Tanto se maravillaron nuestros hombres al ver ese trabajo, que pidieron a los salvajes que hicieran lo mismo para ellos, de manera que cuando llegué a la isla y fui a visitar la colonia de los dos ingleses, a la distancia me pareció que estaban viviendo como abejas en una colmena. En cuanto a Will Atkins, que se había transformado en un hombre trabajador y atemperado, él mismo construyó su choza de mimbres con una destreza inigualable. Por cierto que este hombre mostraba suma habilidad en muchas cosas de las cuales no había tenido anteriormente noción. Se fabricó una fragua con dos fuelles de madera para avivar el fuego; hizo carbón para calentar la fragua, y con una de las alzaprimas de hierro consiguió un yunque bastante pasable, pudiendo entonces dedicarse a fabricar diversas cosas, pero especialmente ganchos, chapas, clavijas, cerrojos y goznes.

    No creo que en el mundo entero pudiera encontrarse una construcción de mimbre tan excelentemente realizada. En esa gran colmena vivían las tres familias, es decir, Will Atkins y su compañero, así como la viuda del que había muerto con sus tres hijos, a todos los cuales no les fue rehusado compartir cuanto poseían. Participaban en igual medida del grano, la leche y las pasas, y cuando mataban un cabrito o encontraban una tortuga en la costa igualmente recibían su parte, de manera que todos ellos vivían muy bien aunque no eran tan industriosos como los otros dos ingleses, cosa que ya ha sido observada anteriormente.

    Algo hay sin embargo que no puede aquí ser omitido, y es que en lo referente a religión no parece que aquellas gentes se preocuparan en lo más mínimo de profesarla. Cierto que frecuentemente venía a sus mentes la idea de que existe un Dios, pero esto a través del ordinario método de los marineros, es decir, jurando por Su nombre. En cuanto a las mujeres, pobres e ignorantes salvajes, sus almas no habían ganado gran cosa al ser tomadas en matrimonio por individuos a quienes llamaremos cristianos; esos hombres sabían bien poco de la existencia divina y eran por tanto incapaces de enseñar tales nociones a sus mujeres, o revelarles cualquier cosa concerniente a la religión.

    Lo más que puedo decir sobre el adelanto que lograron las mujeres con la compañía de los colonos es que aprendieron bastante bien el inglés; todos sus niños, que eran cerca de veinte en total, fueron también enseñados desde muy pequeños a hablar inglés, aunque al comienzo lo hacían de una manera chapurreada como sus madres. Ninguno de aquellos niños tenía más de seis años cuando llegué a la isla, ya que apenas habían transcurrido siete desde que trajeran a las mujeres, pero cada colono tenía ya varios hijos. Las madres se mostraban sumisas, hacendosas y trabajadoras, modestas y llenas de recato, así como dispuestas a auxiliarse mutuamente y muy respetuosas hacia sus amos —puesto que no debo llamarles esposos—. Sólo hubieran necesitado ser instruidas en los principios de la religión cristiana y legítimamente casadas, todo lo cual se logró felizmente más tarde por mi arribo a la isla, o, por lo menos, como consecuencia de mi llegada a ella.






	\chapter{Robinson organiza la colonia}





    Habiendo hecho así un relato de la colonia en general, y detenídome especialmente en mis cinco renegados ingleses, debo decir ahora algo de los españoles que formaban el grupo principal de la familia y en cuya historia hay también muchos episodios dignos de mencionarse.

    Tuve con ellos varias conversaciones acerca de la vida que llevaran mientras residieron entre los salvajes. Me dijeron con franqueza que no habían tenido oportunidad alguna de emplear su ingenio o su perseverancia en aquellas tierras, y que se habían sentido como un puñado de miserables, de abandonados, perdiendo así todo ánimo; incluso de haber tenido a mano los medios necesarios se habrían dejado vencer igualmente por la desesperación y el peso de sus miserias, pues sólo pensaban que el destino los condenaba a morir de hambre. Uno de ellos, hombre reflexivo e inteligente, me dijo, sin embargo, que estaba seguro de que habían cometido un error al pensar así, pues no es propio de hombres sensatos entregarse indefensos a la desgracia sino aprovechar en todo momento los auxilios que la razón ofrece, tanto para preservarse en el presente como para buscar la liberación en el futuro. Agregó que la pesadumbre es la pasión más inútil e insensata del mundo por cuanto sólo mira al pasado que es irrevocable y sin remedio, pero no se le ocurre encarar el porvenir ni comparte nada de lo que puede ser una salvación, sino que prefiere agregarse a la pena antes que buscarle remedio. Al decir esto me repitió un proverbio español que no podría yo citar con las palabras exactas, pero que recuerdo haber traducido entonces al inglés formando un proverbio mío:



    Si por la aflicción te afliges

    Aflicción doble te infliges.



    Pasó luego a comentar las mejoras que yo había podido llevar a cabo mientras duró mi soledad, mi incansable aplicación, según él la llamaba, y cómo había podido transformar una situación que era al comienzo mucho peor que la de ellos en otra mil veces más feliz que la suya. Me dijo que era digno de notarse que los ingleses muestran en la desgracia una mayor presencia de ánimo que cualquiera otra raza de las que él conocía; agregó que su infortunada nación, así como los portugueses, son los peores hombres del mundo para luchar contra el infortunio ya que su primera actitud ante el peligro, luego que los esfuerzos han fracasado, es la de la desesperación, dejarse abatir y aceptar la muerte sin siquiera reflexionar con detenimiento en los posibles remedios para tanta desgracia.

    Me describieron con palabras llenas de emoción cómo se maravillaron al ver tornar a su amigo y compañero de penurias a quien suponían devorado por las fieras de la más temible especie, es decir, por los salvajes caníbales. Con todo, se sorprendieron mucho más cuando él les hizo un relato de lo que le había sucedido diciéndoles que en tierras cercanas habitaba un cristiano dispuesto a contribuir con todas sus fuerzas y su generosidad a liberarlos del cautiverio.

    Relataron que habían sentido inmenso asombro al contemplar las provisiones de auxilio que yo les enviara, en especial los panes, ya que no habían visto uno solo desde su llegada a tan miserables tierras; cómo hicieron la señal de la cruz sobre aquel pan, bendiciéndolo cual si fuera enviado por el cielo, y de qué manera fortificó su ánimo el sabor de su masa y lo mismo las restantes cosas que les enviara para aliviarlos. Hubieran querido describirme la alegría que experimentaron al comprender que disponían de una embarcación y de pilotos que los llevarían al lugar de donde aquellos socorros venían, pero agregaron que carecían de palabras para expresar los transportes a que entonces se entregaron, conduciéndolos el exceso de alegría a desatinadas extravagancias próximas a la locura, y sin encontrar suficiente expansión a los sentimientos que los embargaban en aquellos momentos. Me dijeron que cada uno había reaccionado de distinta manera y mientras unos, a pesar de la inmensa alegría, rompían a llorar, otros parecían enloquecerse y algunos caían desmayados. Estas palabras me conmovieron mucho haciéndome recordar los transportes de Viernes cuando encontró a su padre, así como los arrebatos de aquellos desdichados que salvamos en mi barco cuando el navío en que viajaban se incendió en alta mar; también pensé en la alegría del capitán cuando se vio a salvo en el sitio donde daba por segura su muerte; e incluso recordé mi propia exaltación cuando después de veintiocho años de cautiverio supe que había un buque dispuesto a llevarme a mi patria. Todo aquello me tornaba más sensible al relato de los pobres españoles, y me afectó profundamente.

    Hecho ya el relato concerniente al estado de cosas en que los encontré, debo narrar ahora lo más importante entre lo que puede hacer para aquellas gentes y la situación en que los dejé al marcharme. Opinaban —y mi propia opinión coincidía con la suya— que ya no serían molestados por los salvajes. Incluso si los atacaban otra vez podrían defenderse eficazmente porque ahora estaban en doble número que antes, de manera que no sentían la menor preocupación al respecto. Inicié entonces una grave conversación con el español a quien llamaba gobernador, acerca de su permanencia en la isla; la verdad era que yo no había ido con intención de llevarme a ninguno de los colonos, de manera que tampoco me parecía justo embarcar a algunos dejando a otros, quienes acaso no se sentirían dispuestos a quedarse viendo así disminuida su fuerza.

    Por otra parte afirmé que había venido para asegurar su establecimiento en aquella tierra y no a despoblarla, revelándoles por fin que había traído conmigo diversas cosas necesarias para su vida y que había hecho gastos considerables para proveerlos de todo cuanto pudieran precisar allí, tanto para su comodidad como para su defensa; por fin les dije que había traído un número de personas que les serían útiles no sólo para aumentar la cantidad de pobladores, sino por las distintas profesiones a que aquéllos se dedicaban, lo cual les permitiría disponer de cosas que hasta ahora les faltaban y que sólo  con  ingenio  podían suplir.

    Cuando les manifesté esto lo hice estando todos reunidos, y antes de mostrarles los efectos que había llevado pregunté a cada uno si ya habían olvidado las animosidades que primeramente existieran entre ellos, y si estaban dispuestos a estrechar sus manos y comprometerse a una duradera amistad y a la unión de sus intereses, de tal modo que no hubiera en adelante malentendidos ni discordias.

    Con gran franqueza y no poco buen humor, Will Atkins declaró que bastantes aflicciones habían pasado como para atemperarlos a todos, y que el número de enemigos había sido suficiente para hacerlos a ellos amigos entre sí. Por su parte, se sentía dispuesto a vivir y morir en la colonia y estaba muy lejos de tener malas intenciones hacia los españoles. Reconoció que éstos sólo habían tomado contra él las medidas que su propio carácter díscolo tornaban necesarias, y que en lugar de ellos hubiera procedido en la misma forma y aun mucho más severamente. Agregó que si yo lo deseaba estaba dispuesto a pedirles perdón por todas las villanías y locuras que cometiera y se mostró ansioso por vivir en términos de franca amistad y unión con los demás colonos, comprometiéndose a emplear todas sus fuerzas en la tarea de convencerlos de ello. En cuanto a volver a Inglaterra, no le preocupaba haber faltado de ella todos esos años.

    Los españoles declararon por su parte que habían desarmado y excluido a Will Atkins y sus dos compañeros a causa de su perversa conducta, tal como me lo habían contado ya, insistiendo en que yo comprendiera las razones por las cuales habían procedido en esa forma. Agregaron que Will Atkins se había conducido tan valientemente en la gran batalla librada contra los salvajes y en muchas otras ocasiones posteriores, y desde entonces se había mostrado tan leal y tan preocupado por los intereses comunes, que habían olvidado completamente todo lo sucedido antes, pensando que Atkins merecía que se le devolvieran sus armas y se lo proveyera de lo necesario al igual que los demás. Habían tratado de testimoniarle su satisfacción confiándole el mando después del gobernador, y así como tenían entera confianza tanto en él como en sus compatriotas, se complacían en reconocer que había merecido esa confianza por las mismas razones que la logran los hombres honestos. Aprovecharon la oportunidad para repetirme insistentemente que jamás permitirían que algo los separase de aquellos compañeros.

    Luego de tan abiertas y francas declaraciones amistosas, fijamos el día siguiente para comer todos juntos y en verdad que celebramos un verdadero festín. Hice que el cocinero de a bordo bajara a tierra con su ayudante para preparar la comida, y el colono que había sido también ayudante de cocinero los ayudó. Desembarcamos seis trozos de carne de vaca y cuatro de cerdo, fuera de las provisiones del buque y todo lo necesario para preparar un ponche. Hice bajar también diez botellas de clarete francés y diez de cerveza de Inglaterra, bebidas que ni los españoles ni los ingleses habían probado durante muchos años, por lo cual imaginaba yo el placer que les causaría.

    Los españoles agregaron cinco cabritos al festín, y los cocineros los asaron; tres de ellos, bien envueltos, fueron remitidos a bordo para que los marineros pudiesen probar carne fresca así como nosotros habíamos llevado carne salada a tierra.

    Después del banquete, en el cual reinó la más simple y amistosa alegría, hice desembarcar el cargamento y, a fin de evitar toda disputa con motivo del reparto, principié por manifestar que alcanzaba sobradamente para todos, y que mi deseo era que cada uno recibiera una parte equivalente de prendas de vestir; es decir, igual número de ropas una vez que fueran confeccionadas. Comencé por distribuir tela suficiente para que cada uno pudiera hacerse cuatro camisas; y, por pedido de los españoles, elevé luego el número a seis. Es de imaginar la alegría que esto les causó después de tanto tiempo que carecían de esas prendas, habiendo casi olvidado su uso.

    Repartí entonces la fina tela inglesa de la que he hablado antes, para que cada cual pudiera llevar un vestido liviano, especie de bata que me pareció por lo suelta y cómoda lo más adecuado en aquel clima. Ordené que apenas se estropearan las ropas se las reemplazara por otras nuevas, según ellos lo dispusieran. Lo mismo en lo que respecta a zapatos, medias, sombreros y otras prendas.

    No alcanzo a describir la alegría y la satisfacción que se pintaba en las facciones de aquellos pobres hombres cuando vieron cómo me había preocupado por ellos y las cosas que les traía para ayudarlos. Me llamaron su padre, diciendo que un corresponsal como yo les hacía olvidar que estaban en un sitio desolado y en remotas tierras, y terminaron por comprometerse voluntariamente a no dejar jamás la isla sin mi consentimiento.

    Les presenté entonces a los nuevos colonos que había traído conmigo, especialmente el sastre, el herrero y los dos carpinteros, todos ellos sumamente necesarios a la comunidad; pero fue mi artífice general el que finalmente les resultó más útil que todo lo imaginable. El sastre, para demostrar de inmediato su celo, se puso a trabajar y con mi consentimiento cortó una camisa para cada uno. Su mejor obra sin embargo fue enseñar a las mujeres a coser y remendar, así como los distintos usos de la aguja, y obtener su ayuda para confeccionar las camisas destinadas a sus esposos y al resto de la colonia.

    En cuanto a los carpinteros, apenas hay que decir lo útiles que resultaron; luego de deshacer todos mis toscos y groseros utensilios fabricaron sólidas mesas, bancos, camas, armarios, alacenas, estantes, y todo lo que pudiera ser necesario allí.

    Para mostrarles cómo la Naturaleza es la primera en hacer buenos obreros* los llevé a que visitaran la casa de mimbres de Will Atkins, o casa-cesta, como yo la llamaba; los dos carpinteros declararon no haber visto jamás otro ejemplo de habilidad comparable a ése ni nada tan regularmente construido, por lo menos en su género. Uno de ellos, después de haber mirado mucho la choza y quedarse pensativo, se volvió a mí.

    —Estoy seguro —manifestó— que ese hombre no precisa de nuestra ayuda; lo único que necesita son herramientas.

    Hice entonces desembarcar mi provisión de herramientas y di a cada hombre una azada, una pala y un rastrillo, pues no teníamos arados; también entregué a cada sección de la comunidad un pico, un alzaprima, un hacha grande y una sierra, no dejando de advertir que cuando se rompieran o gastaran podrían ser reemplazadas con las que quedaban en un depósito general convenientemente provisto.

    Clavos, planchas, goznes, martillos, escoplos, cuchillos, tijeras, y toda suerte de herramientas y ferretería les fueron entregados en la cantidad necesaria y sin llevar cuenta, ya que ninguno trató de recibir más de lo que precisaba y muy insensato hubiera sido el que pensara en malgastar o perder aquellos utensilios; en fin, para uso del herrero dejé dos toneladas de hierro en bruto destinado a la forja.

    El almacén de pólvora y balas que les traje era tan abundante que se regocijaron mucho al verlo; ahora podían muy bien andar, como lo había hecho yo antaño, con un mosquete en cada hombro si se presentaba la ocasión, y tenían suficiente para pelear contra mil salvajes bastándoles sólo una pequeña ventaja en la posición, lo cual tampoco les faltaría llegado el caso.

    Traje conmigo a tierra al muchacho cuya madre había perecido de hambre, y asimismo a la doncella, una modesta, bien educada y religiosa muchacha, cuya conducta era tan intachable que todos tenían palabras de elogio para ella. Había llevado a bordo una vida bien triste, siendo la única mujer en nuestra compañía, pero lo soportó con paciencia. Después de un tiempo, y al ver lo bien dispuestas que estaban las cosas en la isla y cómo llevaban camino de prosperar, así como considerando que no tenían relaciones ni negocios en las Indias Orientales que justificaran tan largo viaje, el joven y la criada vinieron a pedirme que les diera permiso para quedarse en tierra y entrar a formar parte de lo que ellos llamaban mi familia.

    Acepté con muy buena voluntad, y les hice dar una pequeña extensión de terreno donde se levantaron tres casas rodeadas de un tejido de mimbre idéntico al

 que Will Atkins había puesto a la suya. Las chozas fueron dispuestas de tal modo que cada uno de ellos tenía una habitación aparte para vivir, y la choza del centro servía de almacén donde se guardaban sus efectos y era a la vez el comedor común. Entonces los otros dos ingleses decidieron trasladar allí sus habitaciones, y en esa forma la isla quedó dividida en dos colonias solamente, en la siguiente forma:

    Los españoles, con el viejo Viernes y los primeros sirvientes, vivían en mi vieja morada bajo la colina, la que constituía en una palabra la capital de la colonia. Habían alargado y extendido de tal manera la residencia, tanto dentro como fuera de la colina, que allí vivían a la vez con toda comodidad y a cubierto de peligros. Nunca hubo pueblo más pequeño en un bosque ni tan oculto en ningún lugar del mundo; pienso que mil hombres hubieran podido pasar un mes explorando la isla, y de no saber antes la existencia del sitio jamás hubiesen dado con él; los árboles eran espesos, estaban tan juntos y habían llegado de tal modo a entrelazarse que sólo derribándolos se hubiera notado la presencia de una habitación, salvo que se advirtieran las dos angostas entradas que había en él, lo que no era fácil. Uno de esos accesos principiaba en la orilla de la ensenada y tenía un largo de doscientas yardas; el otro era la escalera que he descrito ya otras veces. Había también un grande y espeso bosque en la cumbre de la colina, con una superficie de más de un acre, que creció al punto de ocultar enteramente el lugar, con un angosto acceso entre dos árboles casi imposible de descubrir.

    La segunda colonia era la de Will Atkins, en la que había cuatro familias inglesas, es decir, los colonos que yo dejara allí con sus mujeres e hijos, más tres salvajes esclavos; luego la viuda e hijos del inglés muerto por los indios, así como el muchacho y la criada, a la que dicho sea de paso casamos antes de abandonar la isla. Vivían también allí los dos carpinteros, el sastre que yo llevara como colonos y el herrero, hombre muy útil a la comunidad especialmente en su carácter de armero, ya que tenía a su cargo el arsenal; también estaba el otro muchacho a quien llamaba Juan Sabelotodo, que valía por veinte hombres, pues no sólo era altamente ingenioso sino alegre y jovial; fue a él a quien casamos con la virtuosa criada que acompañaba al jovencito en el barco cuya historia he narrado.

    Y ya que de matrimonios se trata, me siento llevado a decir algo del sacerdote francés que venía conmigo luego que lo salváramos del incendio de su barco. Aquel hombre era católico, y acaso pueda yo ofender a algunos si hago notar los extraordinarios detalles de la personalidad de este hombre a quien, antes de principar, debo definir con términos muy poco gratos para los protestantes. Pues aquel sacerdote era ante todo un papista; segundo, sacerdote papista, y tercero, sacerdote papista francés (1).

    La justicia me obliga, sin embargo, a hacer un fiel retrato de aquel sacerdote diciendo que era un hombre grave, atemperado, piadoso y profundamente devoto; su vida era irreprochable, su caridad grande, y casi todas sus acciones valían por ejemplos. ¿Qué puede entonces decirse de mi inclinación hacia él, si a pesar de su distinto credo poseía tales virtudes? Cierto que a salvo queda mi opinión, con la cual puede coincidir la de otros lectores, de que estaba en un error.

    Desde que principié a tratarlo luego que se manifestó dispuesto a ir conmigo a las Indias Orientales, me sentí atraído por su rara elocuencia; lo primero que hizo fue hablarme sobre religión en la forma más amable que pueda imaginarse.

    —Señor —me dijo—, no solamente me habéis salvado la vida después de Dios (y aquí se persignó), sino que me habéis admitido en el viaje que efectuáis, dándome acceso a vuestra compañía con la más exquisita cortesía y proporcionándome así la oportunidad de hablar libremente. Ahora bien, ya habéis advertido por mis hábitos cuál es mi comunión, así como yo deduzco por vuestra nacionalidad a cuál pertenecéis. Cierto que podría considerar mi deber, que por cierto lo es, de emplear la menor oportunidad que se me ofreciera para atraer todas las almas posibles al conocimiento de la verdad y llevarlas a abrazar la religión católica; pero como me encuentro aquí por vuestra generosidad y pertenezco ahora a vuestro pasaje me siento obligado, tanto en retribución a esa gentileza como por razones de simple conveniencia social y buenas maneras, a colocarme bajo vuestra autoridad. No entraré, por lo tanto, sin vuestra venia, en ningún debate que se refiera a asuntos religiosos en los que acaso disentiríamos, salvo que contara con vuestro permiso expreso.

    Me hizo luego un interesante relato de su vida que contenía episodios extraordinarios, contándome diversas aventuras que le acontecieran en los pocos años que llevaba recorriendo el mundo. Una de ellas es particularmente digna de mención: en el último viaje que realizara había tenido la desgracia de cambiar cinco veces de barco sin poder llegar en ninguna oportunidad a los lugares a los cuales esos navíos estaban destinados. Su primera idea era ir a la Martinica, y se embarcó en un navío que con tal rumbo partía de Saint Malo; arrastrados por el mal tiempo a Lisboa, el barco sufrió averías por tocar fondo en la desembocadura del río Tajo, viéndose precisado a dejar allí su cargamento. Hallando entonces un barco portugués que partía con destino a Madeira y estaba ya aparejado, aceptó embarcarse en él pensando que en Madeira pasaría al barco que viaja de allí a la Martinica. El capitán del navío portugués, hombre en demasía negligente, perdió el rumbo y el barco fue a parar a Fayal, donde por otra parte encontró excelente mercado para su cargamento que consistía en granos, y resolvió de inmediato no continuar viaje a Madeira, sino cargar sal en la isla de May y seguir a Terranova. Al sacerdote no le quedó más remedio que acompañar al barco e hizo un excelente viaje hasta llegar a los Bancos, como llaman al sitio donde se pesca; allí, al encontrar un barco francés que iba con destino a Quebec en el río del Canadá y de allí a la Martinica llevando provisiones, pensó que ésa era la oportunidad de completar su trayectoria. Sin embargo, al llegar a Quebec murió el capitán del barco y el navío no pudo seguir viaje; entonces tuvo que embarcarse en otro que volvía a Francia, el mismo que se quemó en alta mar y cuyo pasaje y tripulación salvamos; ahora por fin venía el sacerdote con nosotros rumbo a las Indias Orientales. En cinco viajes había sido desviado de su rumbo y se puede decir que los cinco formaban solamente uno, aparte de lo que más adelante tenga que relatar sobre aquel hombre.

    Pero no quiero continuar con disgresiones sin relación directa con mi relato; vuelvo ahora a nuestros asuntos en la isla. El sacerdote vino una mañana a verme (pues se alojaba con nosotros en tierra) y dio la casualidad que lo hiciera la misma mañana en que me disponía a visitar la colonia que los ingleses tenían en la parte más oriental de la isla. Vino, como digo, y me expresó con aire extremadamente grave que había esperado dos o tres días la oportunidad de conversar conmigo sobre algo que esperaba no me desagradaría oír, y que en cierta medida coincidía con mis intenciones de que la isla se convirtiera en un sitio próspero, cosa que tal vez ocurriera en un grado todavía mayor si lograban alcanzar para ese fin la bendición divina.

    Le contesté que iba a visitar las plantaciones de los ingleses y lo invité a que me acompañara, con lo cual tendríamos oportunidad de hablar durante el camino. Replicó que iría gustoso, pues justamente el asunto del que deseaba conversar conmigo se relacionaba con aquella colonia. Echamos a andar, y le pedí que expresara con toda llaneza y confianza lo que tenía que decirme.

    —Ante todo, señor —dijo el joven sacerdote—, tenéis aquí cuatro ingleses que han escogido mujeres entre los salvajes y las han tomado por esposas, teniendo ya de ellas muchos hijos sin haber contraído legítimo matrimonio como las leyes de Dios y del hombre requieren. Eso, tanto para unos como para otros, significa vivir como adúlteros, pues es de eso que se trata.

    Luego de una pausa, agregó:

    —Bien sé, señor, que a mis palabras responderéis que no había aquí sacerdote de ninguna comunión para llevar a cabo la ceremonia; que faltaba pluma, tinta y papel para redactar el contrato matrimonial. También sé lo que os ha dicho el gobernador español, o sea la promesa que exigió de aquellos hombres cuando eligieron a sus mujeres en el sentido de que las escogerían de común acuerdo y que vivirían por separado con ellas. Sin embargo, señor, nada de eso se parece a un matrimonio ya que falta ante todo el consentimiento de las esposas y sólo hubo acuerdo entre los hombres que querían evitar en esa forma nuevos motivos de querellas.

    »Con todo, la esencia del sacramento del matrimonio (como era católico romano lo llamaba así) no consiste solamente en el mutuo consentimiento de las partes en su carácter de marido y mujer, sino en la obligación formal y legal que existe en su contrato de considerarse como tales y así reconocerse en cualquier ocasión, obligando al hombre a apartarse de toda mujer y no participar de otra unión mientras ésta subsista; de la misma manera es su deber cuidar del sustento de su esposa e hijos en todo momento y honestamente. Por su parte, mutatis mutandis, la mujer contrae análogas obligaciones.

    »Ahora bien, señor, esos hombres pueden abandonar cuando les plazca a sus esposas así como desentenderse de sus hijos, desconociéndolos y prefiriendo a otra mujer para unirse a ella mientras la primera está aún viva.

    Y agregó con calor:

    — ¿Cómo creéis que Dios puede ser dignamente honrado en medio de esta libertad y esta licencia? Sois vos a quien cabe ejercitar sus poderes para poner fin a esta situación, y solamente a vos os corresponde hacerlo. Me sentí tan confundido que no entendí claramente lo que acababa de decirme, sino que imaginé que al expresar que «pusiera fin a esa situación» significaba que debía separar a las parejas y no permitirles más vivir juntas. Me apresuré a decirle que de ninguna manera podía llevar a cabo semejante cosa, que sólo serviría para perturbar la isla entera. Pareció sorprenderse de que lo hubiera comprendido tan mal.

    —No, caballero —dijo—, de ningún modo pretendo que los separéis, sino que los unáis en matrimonio legítimo ahora mismo. Bien sé que mi comunión no se concilia con sus usos y costumbres, aunque sería absolutamente válido aun para vuestras leyes, por lo cual pienso que vuestra intervención tendrá vigencia ante Dios y será considerada legal entre los hombres; quiero decir que debéis hacer firmar un contrato a esos hombres y mujeres, con todos los testigos necesarios, y semejante obligación será seguramente reconocida por todas las leyes de Europa.

    Me maravilló comprobar tanta piedad y tanto celo a través de esas palabras, la imparcialidad y tolerancia que contenía su discurso, así como el interés que se tomaba el sacerdote para la salvación de individuos de los cuales no tenía noción ni conocimiento alguno, buscando evitar que transgrediesen las leyes de Dios; por cierto que jamás había yo visto una cosa parecida. Pensando entonces en lo que acababa de decirme sobre el casamiento de los colonos por medio de un contrato escrito, lo cual me parecía muy bien, repliqué a mi interlocutor que sus palabras eran justas y revelaban una gran generosidad, por lo cual hablaría con aquellos hombres cuando llegásemos allá. No veía razón, agregué, por la cual los colonos pudieran sentir escrúpulos de dejarse casar en esa forma; bien sabía yo que ese documento tendría tanta validez como si un pastor de nuestra tierra hubiese efectuado en persona el matrimonio.

    Más tarde contaré cómo se llegó a una solución tocante a ese punto. Ahora, volviéndome a mi acompañante, le rogué que me formulara su segunda observación, no sin antes reconocer mi deuda por la primera y darle las gracias de todo corazón. Me dijo entonces que me hablaría con la misma franqueza que acababa de hacerlo y que esperaba ser escuchado del mismo modo.

    —Aunque vuestros súbditos ingleses —dijo, dándoles esa denominación— han vivido durante siete años con esas mujeres, enseñándoles a hablar el inglés y hasta a leerlo, aprovechando según lo he podido advertir su clara inteligencia y su capacidad de aprendizaje, sin embargo, hasta el día de hoy no les han enseñado absolutamente nada de lo que concierne a la religión cristiana. No, ni siquiera que existe un Dios, que hay un culto, y la manera en que Dios debe ser adorado; ni siquiera se han preocupado por demostrarles que su idolatría y adoración de quién sabe qué dioses son absolutamente falsas y absurdas.

    »Ahora bien, caballero —agregó—, aunque yo no reconozca vuestra religión ni vos la mía, a ambos debe alegrarnos que a los servidores del diablo y vasallos de su reino se les enseñen los principios generales de la religión cristiana, que oigan por lo menos la noción de que hay un Dios, un Redentor, que se enteren de la resurrección y de la vida futura, cosas en las cuales creemos en común. Pensad que por lo menos estarán mucho más cercanos incorporándose al seno de la verdadera iglesia que como lo están ahora profesando públicamente su idolatría y su culto infernal.

    — ¿Pero qué queréis que yo haga? —pregunté—. Ya sabéis que voy a marcharme de aquí.

    —Lo que os pido —dijo— es que me deis vuestro permiso  para  hablar  con esos pobres hombres  al respecto.

    —Os lo concedo de todo corazón —repuse— e incluso los obligaré a que reparen atentamente en todo lo que les digáis.

    —En cuanto a eso, debemos dejarlo librado a la gracia de Jesucristo, pero vuestra tarea consiste en asistirlos, darles ánimo, instruirlos; si me concedéis autorización y si recibo la gracia de Dios, no dudo que esas pobres almas ignorantes serán rescatadas y traídas al seno común del cristianismo, esa fe que todos nosotros abrazamos; y confío en lograrlo mientras permanezcáis aquí.

    Oyendo esto, afirmé:

    —No solamente os concedo mi permiso, sino que os doy mil gracias por ello.

    Lo que ocurrió como consecuencia de esta conversación será narrado más adelante. Pregunté luego a mi acompañante sobre la tercera observación o reproche que debía hacerme.

    —Pues bien —declaró—, se trata de algo de la misma naturaleza, y si me dejáis hacerlo os la diré con la llaneza de las anteriores. Se trata de esos pobres salvajes que son, si se les puede llamar así, vuestros súbditos por derecho de conquista. Hay un principio, caballero, que existe o debería existir entre todos los cristianos, sea cual fuere su particular iglesia o comunión, según el cual el conocimiento cristiano debe ser propagado por todos los medios posibles y en todas las ocasiones que se presenten. Es en base a ese principio que nuestra iglesia envía misioneros a Persia, a la India y a la China; es por eso que nuestro clero, aun el de jerarquía superior, se lanza voluntariamente a los más azarosos viajes, fijando su residencia entre los más temibles asesinos y salvajes, para enseñarles el conocimiento del verdadero Dios y conseguir que lleguen a abrazar la fe cristiana. Ahora bien, señor, tenéis aquí la oportunidad de convertir a treinta y seis o treinta y siete infelices salvajes, trayéndolos de la idolatría a la noción del verdadero Dios, su Hacedor y Redentor, al punto que me asombra que podáis desperdiciar semejante ocasión de hacer un bien digno de que un hombre le consagre su vida entera.

    Estas reflexiones me dejaron profundamente turbado, y no encontré una sola palabra que contestar. Ante mí había un representante del verdadero celo cristiano por Dios y su religión, fueran los que fuesen sus principios particulares.

    Al verme tan confundido, me miró afectuosamente. —Señor —dijo—, creedme que me afligiría mucho saber que algo de lo que os he dicho puede haberos ofendido.

    —No, no —dije yo—; si con alguien estoy ofendido es conmigo mismo. Creedme que me turba no solamente la idea de que jamás había cruzado antes por mi mente semejante responsabilidad, sino también el no saber cómo reparar esa falta. Bien sabéis, señor —agregué—, en qué circunstancias me encuentro. Mi destino es el de las Indias Orientales, en un barco fletado por comerciantes que recibirían un perjuicio injusto si su navío quedase detenido en esta isla por cuanto la tripulación es pagada y alimentada por los armadores. Cierto que tengo autorización para estar aquí doce días, y si me quedara más tiempo debería pagar tres libras esterlinas diarias por la demora. He prometido asimismo no prolongar la estadía por más de ocho días, y son ya trece los que llevo aquí, de manera que me siento incapaz de cumplir la tarea que me señaláis, salvo que decidiera quedarme otra vez en la isla. Pero pensad que si al navío le ocurriera una desgracia me encontraría nuevamente en la triste condición que viví al principio, y de la cual fui tan asombrosamente rescatado.

    Reconoció entonces que el viaje dificultaba la cuestión, pero insistió en dirigirse a mi conciencia, preguntándome si la salvación espiritual de esas treinta y siete almas no merecía que yo le dedicase cuanto tenía en el mundo.

    —Ciertamente, señor —declaré, sintiéndome menos penetrado que él de esa obligación—. Ser instrumento de Dios es algo admirable, así como traer paganos a la religión de Cristo. Pero desde que vos sois sacerdote y estáis entregado a una tarea que es justamente la propia de vuestra vocación, ¿cómo es que no preferís ofreceros en persona para cumplirla en vez de impulsarme a mí?

    Al oír estas palabras, pronunciadas a medida que caminábamos, se detuvo de improviso y encarándome, luego de hacer una profunda reverencia, dijo:

    —Agradezco a Dios y os agradezco, caballero, por darme un testimonio tan evidente de que soy llamado a cumplir esa obra santa. Si os sentís excusado de ella y deseáis que quede en mis manos, lo haré de todo corazón considerándola suficiente recompensa después de los azares y las penurias de un viaje tan desdichado como el mío, que sin embargo, me lleva por fin a tan hermosa tarea.

    Mientras me hablaba vi en su rostro una expresión como de éxtasis; brillaban sus ojos, y su rostro tan pronto palidecía como se arrebolaba, igual que el que sufre repentinos accesos de fiebre; en una palabra, parecía encendido por la alegría al verse enfrentado a semejante trabajo.

    —Sí, caballero —agregó—; daré las gracias todos los días de mi vida a Jesucristo y a la Santísima Virgen por permitirme ser humilde y dichoso instrumento de la salvación de esas pobres almas, aunque esto me valiera pasar toda la vida en la isla y no regresar nunca a mi patria. Y ahora, puesto que me hacéis el honor de confiarme tal misión, en razón de la cual rogaré a Dios por vuestra alma todos los días de mi existencia, tengo una simple petición que haceros.

    — ¿Cuál es ella?

    —Que me dejéis en compañía de vuestro criado Viernes para que sea mi intérprete ante los salvajes y me acompañe en la tarea; sin alguna ayuda sería imposible hablar a esos hombres o entender sus palabras.

    El pedido me llenó de preocupación, porque de ninguna manera quería yo separarme de Viernes. Tenía muchas razones para no hacerlo; ante todo, Viernes era el compañero de mis viajes y no solamente me había sido fiel en absoluto, sino que me quería de un modo extraordinario, por lo cual yo estaba resuelto a hacer cuanto pudiera por él en el caso de que me sobreviviera, lo que parecía probable. Además había hecho de Viernes un protestante y estaba seguro de que una distinta comunión lo confundiría lamentablemente; mientras viviera se negaría siempre a creer que su amo era un hereje y que moriría condenado. Todo aquello confundiría los principios religiosos del pobre muchacho, llevándolo quizá de nuevo a su antigua idolatría.

    Una súbita idea vino en mi auxilio para sacarme de la turbación en que me hallaba, y dije al joven sacerdote que me dolía mucho separarme de Viernes por cualquier motivo que fuese, bien que una tarea que él consideraba aún más preciosa que su vida no podía a mí parecerme menos importante que la simple separación de un sirviente. Sin embargo, agregué, estaba persuadido de que Viernes no consentiría de ninguna manera en alejarse de mi lado y yo no tenía derecho a decidirlo sin su previa aceptación, pues de lo contrario cometería una injusticia teniendo en cuenta la promesa y el compromiso de que jamás se separaría de mi lado a menos que se lo ordenase.

    Me pareció que esta negativa lo apenaba mucho, ya que quedaba sin medios para comunicarse con los salvajes y no comprendía una palabra de su lenguaje, y tampoco ellos del suyo. Para salvar esa dificultad le dije entonces que el padre de Viernes sabía hablar español —idioma que el sacerdote entendía— y podría servirle de intérprete. Esto lo animó mucho y ya nadie hubiese logrado disuadirlo de que no se quedara en la isla para convertir a los salvajes. La Providencia, sin embargo, lo dispuso de un modo distinto y mucho mejor.






	\chapter{La conversión de Will Atkins}





    Vuelvo ahora a las anteriores observaciones que me hiciera el sacerdote. Cuando llegamos adonde estaban los ingleses, los envié a buscar y luego de referirme a cuanto había hecho por ellos, las cosas que les había traído y la forma en que les fueran distribuidas —cosas que reconocían unánimemente y con profunda gratitud— principié a hablarles de la escandalosa vida que llevaban, haciéndoles un resumen de las cosas que a ese respecto me dijera el sacerdote. Después de demostrarles que esa vida era totalmente irreligiosa y que nada tenía de cristiana, les pregunté si anteriormente habían sido hombres casados o solteros.

    Me dieron prolijas explicaciones, por las cuales supe que dos de ellos habían quedado viudos mientras los tres restantes eran solteros. Les pregunté cómo habían tenido conciencia para tomar aquellas mujeres, llamarlas sus esposas y, sin embargo, no estar legítimamente casados con ellas.

    Como es natural me dieron la respuesta que yo había esperado, diciéndome que nadie había en la isla que pudiera casarlos, y que solamente se habían comprometido ante el gobernador a mantener a aquellas mujeres para siempre en carácter de esposas. Por lo que a ellos respecta, en base a la situación existente en la isla, se sentían tan legalmente casados como si un pastor hubiera efectuado la ceremonia con todas las formalidades del mundo.

    Repliqué que no había duda de que a los ojos de Dios estaban casados y con obligación de mantener a sus esposas y no abandonarlas jamás, pero que las leyes humanas eran distintas y por lo tanto en cualquier momento podrían ellos pretender que tal matrimonio no existía, abandonar a sus esposas y a sus hijos. Aquellas mujeres, pobres e ignorantes, sin amigos ni dinero, carecerían en adelante de todo socorro. Agregué entonces que hasta no sentirme seguro de que estaban dispuestos a proceder honestamente no haría nada por ellos, sino que todas mis preocupaciones estarían encaminadas solamente hacia las mujeres y los pequeñuelos. Por lo tanto, y mientras no me dieran alguna satisfacción de que verdaderamente se casarían con esas mujeres, no me parecía conveniente que continuaran su vida en común como esposos y esposas; todo eso era altamente escandaloso para los hombres y ofensivo para Dios, de quien no podrían esperar bendición alguna si continuaban en esa forma.

    Mi discurso produjo el esperado efecto. Todos ellos afirmaron, y en especial Will Atkins, que parecía hablar en nombre de los otros, que amaban mucho a sus esposas, tanto como si fueran nacidas en su propia patria, por lo cual jamás habían pensado en abandonarlas; agregaron que tenían la seguridad de que eran virtuosas y modestas y que se comportaban de la mejor manera tanto en lo que respecta a sus hijos como a sus esposos, por lo cual de ninguna manera se separarían de ellas. Will Atkins agregó luego por cuenta propia que si alguna vez le ofrecían rescatarlo de la isla, llevarlo a Inglaterra y hacerlo capitán del mejor navío de guerra de la flota, no aceptaría nada sin la autorización de llevar consigo a su esposa e hijos; y terminó diciendo que en ese caso, apenas encontrara un sacerdote a bordo, le pediría que los casara.

    Todo esto era lo que yo estaba deseando oír. El sacerdote no había venido conmigo, sino que permanecía en los alrededores, y para poner a prueba a Atkins le dije que había traído conmigo a un clérigo, de modo que si sus palabras eran sinceras yo podía casarlos a la mañana siguiente, por lo cual le pedía que considerara mis palabras y las hiciera saber a los otros. Me replicó que por lo que a él se refería no necesitaba pensar nada, pues estaba dispuesto a casarse y se sentía muy contento de que yo hubiese traído un ministro, y en cuanto a los demás descontaba que también se alegrarían.

    Le dije que mi amigo el sacerdote era un francés, y que como ignoraba nuestro idioma yo actuaría de intérprete. Ni siquiera me preguntó si aquel hombre era papista o protestante, que era lo que yo había estado temiendo, y tampoco se inquietaron más tarde por eso. Nos separamos entonces, volviendo yo junto a mi acompañante y Will Atkins a hablar con los suyos. Mi deseo era que el sacerdote no se encontrase por el momento con los colonos hasta que todo estuviese dispuesto y a punto, y me apresuré a contarle la respuesta que acababa de recibir.

    Antes que saliéramos de sus dominios vinieron a nosotros en grupo y me dijeron que acababan de considerar mis manifestaciones, alegrándose mucho de saber que había un sacerdote en la isla. Se sentían ansiosos por darme la satisfacción que les había demandado y querían casarse legítimamente tan pronto yo lo dispusiera así. Me repitieron que nada más alejado de sus mentes que la idea de abandonar a sus esposas, y que al elegirlas lo habían hecho sin la menor intención deshonesta.

    Los cité entonces para que vinieran a encontrarse conmigo a la mañana siguiente y les pedí que dedicaran el día a explicar a sus esposas el significado del contrato matrimonial, haciéndoles ver que no solamente estaba destinado a evitar todo escándalo, sino que les imponía mutuamente la obligación de conservarse fieles para siempre.

    Las mujeres comprendieron con facilidad el significado de la ceremonia y, como es de imaginar, se mostraron sumamente satisfechas. Ninguno faltó a la cita de la siguiente mañana en mi residencia, y entonces les presenté al sacerdote. Cierto que mi amigo carecía de la indumentaria adecuada, pues ni vestía la ropa de un ministro inglés ni la sotana habitual en los sacerdotes de Francia, sino que usaba un traje negro, a la manera de una casaca con una banda o faja que la asemejaba en parte a una indumentaria religiosa. En cuanto al idioma, yo fui su intérprete.

    La seriedad de su conducta, así como los escrúpulos que manifestó por casar a mujeres que no habían recibido el bautismo cristiano, dio a los colonos una elevada idea de su persona; después de eso ya no sintieron necesidad de averiguar si era o no un eclesiástico. Por un momento llegué a sentir miedo de que esos escrúpulos le impidieran casar a aquellas parejas. En efecto, a pesar del empeño que puse en que llevara a cabo la ceremonia se rehusó humildemente, pero con toda firmeza, manifestándome que de ningún modo casaría a los colonos antes de hablar separadamente con ellos. Al principio me mostré poco dispuesto a complacerlo, pero por fin se lo consentí con toda buena voluntad al advertir la sinceridad de su proceder y su intención.

    Fue en busca de los colonos y les hizo saber que yo lo había interiorizado de todas las circunstancias de su vida así como de su actual determinación; agregó que estaba dispuesto a realizar la ceremonia y casarlos tal como había sido mi deseo, pero que antes de hacerlo se tomaría la libertad de hablar con ellos.

    Dijo entonces que temía que fuesen cristianos harto indiferentes, con escaso conocimiento de Dios y su Providencia, y que era de imaginarse lo poco que habrían enseñado a sus esposas en materia religiosa. Exigía por lo tanto su promesa formal de que harían todos los esfuerzos posibles para que aquellas mujeres se convirtieran al cristianismo y recibieran de la mejor manera posible el conocimiento y la fe de Dios su Creador, así como que les enseñaran a adorar a Jesucristo, que las había redimido. De lo contrario, no los casaría porque no deseaba ser instrumento de una unión entre cristianos y salvajes, cosa contraria a todos los principios de la religión y prohibida de manera especial por las leyes de Dios.

    Escucharon muy atentamente aquellas palabras que yo les iba traduciendo a medida que eran pronunciadas, agregando aquí y allá alguna cosa para darles más énfasis y demostrarles hasta qué punto compartía las opiniones del sacerdote; por cierto que les hice distinguir claramente cuáles eran mis palabras y cuáles provenían del joven sacerdote.

    Contestaron entonces que lo que aquel caballero había dicho era cierto, que se consideraban muy malos cristianos y que nunca habían hablado de religión con sus esposas.

    — ¡Dios mío, señor! —exclamó Will Atkins—. ¿Qué podíamos enseñarles nosotros que no sabemos absolutamente nada? Y luego, si les hubiéramos hablado de Dios, de Jesucristo, del Cielo y el infierno, ellas se hubieran reído de nosotros y nos hubiesen preguntado cuál era nuestra creencia. De haberles contestado que creíamos en todas esas cosas de las cuales les hablábamos, tales como el paraíso para las almas buenas y el infierno para las malas, de inmediato nos habrían preguntado adonde pensábamos ir nosotros, gentes tan perversas y malvadas como verdaderamente somos. No, señor, eso hubiera servido solamente para causarles repugnancia hacia la religión, y además supongo que los que piensan enseñar a otros deben empezar por aprender ellos mismos alguna cosa.

    —Will Atkins —repliqué yo—, aunque temo que vuestras palabras contengan mucho de verdad, pienso sin embargo, que podéis tratar de explicar a vuestra esposa que vive en un error, que existe un Dios, una religión mejor que su idolatría; decirle que sus dioses no son más que ídolos, que no pueden oír ni hablar, que existe un inmenso Ser que hizo todas las cosas y que puede destruirlas con ese mismo poder; enseñarle que Dios premia el bien y castiga la maldad, y que todos seremos alguna vez juzgados por El a causa de nuestros actos. No sois tan ignorante puesto que hasta la Naturaleza misma os enseñará que todo eso es cierto, y me satisface que admitáis su verdad y lo creáis así.

    —Ciertamente, señor —repuso Atkins—, pero, ¿con qué cara he de presentarme a mi mujer para decirle todo eso, cuando inmediatamente me contestará que no puede ser verdad?

    — ¡Que no puede ser verdad! —exclamé yo—. ¿Qué queréis decir con eso?

    —Simplemente que ella me lo dirá así, señor; me dirá que el Dios de quien le hablo no puede ser justo ni tiene poder para premiar o castigar puesto que yo no he sido castigado y hundido en el infierno como lo merecía por malvado y perverso; mi esposa sabe bien que he sido un miserable, no solamente con ella, sino con todos los demás, y por lo tanto el hecho de verme todavía vivo contradice de raíz todo cuanto pueda yo explicarle, además de haber actuado siempre en forma opuesta a mi deber y a mis obligaciones.

    —Empiezo a creer de veras, Atkins —dije yo—, que lo que decís sea demasiado cierto.

    Y volviéndose entonces hacia el sacerdote que esperaba impaciente, le repetí las declaraciones que acababa de hacerme Atkins.

    — ¡Ah! —exclamó él al oírme—. Decidle que hay algo que puede transformarlo en el mejor ministro de la tierra ante su esposa, y que ese algo es el arrepentimiento; decidle que nadie puede enseñar el arrepentimiento mejor que un sincero penitente, y que luego que él mismo se haya arrepentido de sus pecados será un excelente predicador ante su esposa.

    Repetí esto a Atkins, que lo escuchó seriamente y que estaba, como pudimos advertirlo, profundamente conmovido por esas palabras. Pareció, sin embargo, querer evitar la continuación de esa escena y me dijo que deseaba hablar con su esposa, de manera que lo dejamos marcharse y seguimos con los otros. Observé que todos eran profundamente ignorantes en materia de religión, tal como lo fuera yo en la época en que abandoné la casa de mis padres; a pesar de esto ninguno dejó de prestar gran atención a lo que se les decía y prometieron formalmente hablar de ello a sus esposas a fin de conseguir por todos los medios que se convirtiesen al cristianismo.

    Cuando traduje esta respuesta al sacerdote, sonrió al escucharla y estuvo un largo rato callado. Por fin movió la cabeza.

    —Los que somos servidores de Cristo —dijo— no podemos ir más allá de instruir y exhortar, y cuando los hombres condescienden, aceptan el reproche y prometen cumplir lo que les pedimos, nada más podemos hacer nosotros. Es necesario aceptar su palabra, y sin embargo, caballero —prosiguió—, cualquiera sea la villanía que podáis conocer en la vida de ese hombre llamado Will Atkins, creedme que es el único sinceramente arrepentido entre todos ellos. Pienso que es un penitente de verdad y, aunque no desespero por el recuerdo de su vida pasada. No me cabe duda de que cuando hable de religión a su esposa, sus palabras valdrán asimismo para él, ya que muchas veces enseñar a los demás es la mejor manera de enseñarnos a nosotros mismos.

    Tras estas palabras, y contando, como se ha dicho, con la promesa de los colonos en el sentido de persuadir a sus esposas que abrazaran el cristianismo, el sacerdote casó a las tres parejas, pero Will Atkins y su mujer seguían ausentes. Al notar esto, el clérigo sintió curiosidad por saber adonde habría ido Atkins, y volviéndose a mí me dijo:

    —Os ruego, caballero, que salgamos juntos de este laberinto vuestro y vayamos a ver qué ocurre; me atrevo a asegurar que encontraremos a ese pobre hombre en un sitio u otro hablando seriamente con su mujer y principiando a darle algunas enseñanzas religiosas.

    A mí me parecía lo mismo, de manera que salimos juntos y lo llevé por un sendero que solamente yo conocía, donde los árboles estaban plantados a tan poca distancia unos de otros que resultaba difícil avanzar entre ellos y apenas se alcanzaba a divisar algo. Cuando llegamos a la extremidad del bosque vimos a Atkins y a su morena esposa sentados a la sombra de un árbol y hablando muy seriamente. Me detuve a esperar a mi clérigo, y luego de mostrarle dónde estaba la pareja nos quedamos un buen rato observándola.

    Notamos que él parecía empeñado en hacerle comprender algo, señalando hacia el sol y todos los lados del cielo, luego apuntando a la tierra, el mar y a sí mismo, para luego volverse hacia ella y los árboles del bosque.

    —Ya veis —dijo el sacerdote— cómo mis palabras han resultado ciertas. Ese hombre está instruyendo a su mujer, le está diciendo que Dios lo ha creado así como a ella, los cielos, la tierra, el mar, los bosques y sus árboles.

    —Así me parece —respondí.

    Vimos entonces a Will Atkins levantarse y caer en seguida de rodillas, con las manos alzadas. Imaginamos que estaba diciendo alguna cosa, pero nos hallábamos demasiado lejos para oír sus palabras. Luego de permanecer de hinojos un momento fue nuevamente a sentarse al lado de su mujer y se puso a hablarle. Observamos que ella parecía muy atenta, pero no alcanzamos a percibir si decía algo. Mientras Will Atkins permanecía de rodillas, pude observar cómo las lágrimas rodaban por las mejillas del sacerdote y yo mismo apenas logré contener el llanto. Desgraciadamente estábamos demasiado lejos para escuchar el diálogo que entre aquellos seres se desarrollaba.

    Como no podíamos acercarnos más por temor a perturbarlos, resolvimos seguir contemplando a distancia esa escena que, sin embargo, hablaba para nosotros con rara elocuencia y sin necesidad de voz alguna. Como he dicho, Atkins volvió a sentarse junto a su mujer hablándole con animación, y dos o tres veces vimos que la abrazaba tiernamente; en una oportunidad extrajo su pañuelo para secar los ojos de la esposa, y volvió luego a besarla en un transporte poco común. De pronto, luego que esa escena hubo durado un momento más, vimos de qué modo Atkins se enderezaba vivamente y tomando a su mujer de la mano para ayudarla a hacer lo mismo, avanzaban juntos uno o dos pasos y luego caían al unísono de rodillas, donde permanecieron inmóviles por espacio de dos minutos.

    Después que aquel pobre hombre y su esposa se levantaron, vimos que él seguía hablándole con gran entusiasmo, y conjeturamos por la actitud de su oyente que ella debía estar hondamente conmovida por sus palabras, ya que con frecuencia alzaba las manos, las llevaba luego contra el pecho y adoptaba, en fin, las actitudes que habitualmente señalan profunda atención e interés. Esto continuó durante un cuarto de hora, hasta que se marcharon juntos y ya no pudimos ver nada más.

    Aproveché ese intervalo para hablar con mi clérigo. Le dije cuánto me alegraba haber presenciado con él aquella escena, y que aunque por lo común me costaba mucho creer en tales conversiones creía que aquella pareja obraba, con mucha sinceridad a pesar de su ignorancia, y esperaba, por lo tanto, de aquel excelente principio, un final todavía más feliz.

    — ¿Y quién sabe —agregué— si estos dos, con ayuda de una cierta instrucción y del ejemplo no llegarán a influir sobre algunos de los otros?

    — ¡Algunos de los otros! —me replicó el sacerdote volviéndose con rapidez hacia JQÍ—. Decid más bien sobre todos; si esos dos salvajes, —puesto que a juzgar por lo que me habéis contado, él lo ha sido casi tanto como ella—, son capaces de abrazar a Jesucristo, jamás cejarán hasta haber convencido al resto, pues la verdadera religión es por esencia comunicativa y el que ha sido convertido al cristianismo nunca permitirá que en torno suyo quede algún pagano mientras pueda impedirlo.

    Admití que pensar así era cosa de buen cristiano, y además testimonio del gran celo y del buen corazón que yo había admirado en él.

    —Con todo, amigo mío —agregué—, ¿me permitiréis que os haga una pregunta al respecto? De, ninguna manera me sería posible poner reparo al afectuoso interés que habéis demostrado por convertir a esas personas al cristianismo y arrancarlas de su ignorancia. ¿Pero es que eso puede daros alguna alegría siendo que esas gentes se encuentran desde vuestro punto de vista fuera de la comunidad de la Iglesia católica, lo que equivale según vuestro credo a considerarlas privadas de salvación por su herejía? ¿No estáis obligado a considerarlas tan perdidas para Dios como si siguiesen practicando su paganismo?

    Me respondió entonces con profunda sencillez y caridad cristiana, diciéndome:

    —Caballero, soy católico apostólico romano y sacerdote de la orden de los benedictinos, por lo cual abrazo todos los principios de la fe romana. Sin embargo, si queréis creerme, puesto que no hablo por haceros cumplido ni en homenaje a mi situación y a vuestra amabilidad, os diré que no considero a los que como vos pertenecen a la reforma sin una cierta caridad. No me atrevería jamás a afirmar, aunque bien sé que. entre nosotros es opinión corriente, no me atrevería jamás a afirmar, como os digo, que no podréis salvaros. De ninguna manera pretendo limitar las mercedes de Cristo hasta el punto de creer que no habrá de recibiros en el seno de Su Iglesia, de una manera para nosotros incomprensible y que escapa a nuestro entendimiento. Y abrigo la esperanza de que vosotros tengáis la misma caridad con respecto a nosotros.

    Me llenó de asombro la sinceridad y tino de este piadoso papista, a la vez que me produjo admiración la claridad de su razonamiento. De inmediato vino a mí la idea de que si semejante criterio resultara universal, todos seríamos cristianos católicos cualquiera fuese nuestra iglesia o comunidad particular; el espíritu de la caridad podría pronto elevarnos a los principios más justos y verdaderos y, en una palabra, así como él pensaba que la verdadera caridad podía hacernos católicos a todos así le contesté que si los miembros de su iglesia mostraran igual moderación podrían ser todos protestantes; y con eso abandonamos el diálogo, pues evitábamos cualquier controversia.

    Me dirigí a él, sin embargo, empleando otro argumento, y le tomé la mano para decirle:

    —Amigo mío, quisiera yo que todo el clero de la Iglesia romana estuviera dorado de vuestra misma moderación, así como que compartiera vuestra caridad. Soy plenamente de vuestra opinión, pero fuerza me es declararos que si predicarais semejante doctrina en España o Italia, os entregarían prontamente a la Inquisición.

    —Puede ser —repuso él—. Ignoro lo que harían conmigo en Italia o en España, pero no me parece que demostraran ser mejores cristianos con semejante rigor, pues estoy seguro de que no existe la menor herejía en hablar de caridad.

    En fin, como Will Atkins y su esposa se habían ido nada teníamos ya que hacer en ese lugar y nos volvimos por el mismo sendero, hallando que nos esperaban para la ceremonia del matrimonio. Al ver esto pregunté al sacerdote si le parecía conveniente decir a Atkins que lo habíamos observado desde lejos, y me contestó que no le parecía prudente antes de hablar con él y escuchar sus declaraciones. Lo llamamos entonces aparte, sin que hubiese otros testigos presentes, y yo entablé con él la conversación que sigue:

    —Will Atkins —dije—, ¿qué educación habéis recibido? ¿Quién era vuestro padre?



    W. A.: Un hombre mucho mejor de lo que seré yo jamás, señor; tuve por padre a un eclesiástico.

    R. C: ¿Qué educación os dio?

    W. A.: El hubiera deseado instruirme bien, señor, pero yo despreciaba toda enseñanza, todo consejo, toda corrección, como bruto que era.

    R. C: Es verdad; Salomón ha dicho: «El que desdeña los reproches procede como los irracionales.»

    W. A.: Cierto, señor, y yo lo fui. Yo maté a mi padre, y os suplico por Dios que no habléis más de esto. ¡Yo fui el asesino de mi pobre padre!

    SACERDOTE: ¡Ah, un asesino!

    (La exclamación del sacerdote se produjo al traducirle yo las palabras de Atkins, y vi que palidecía. Sin duda creyó que Will había asesinado realmente a su padre.)R. C: No, no, caballero, no habéis entendido bien. Explicaos, Will Atkins. ¿Verdad que no matasteis a vuestro padre con vuestras propias manos?

    W. A.: No, señor, no corté su garganta pero sí el hilo de su tranquilidad abreviando sus días. Destrocé su corazón pagándole de la manera más ingrata y miserable el más tierno y afectuoso comportamiento que jamás haya tenido un padre o recibido un hijo.

    R. C: Bien, no os interrogué sobre vuestro padre para arrancaros esta confesión. Pido a Dios que os conceda el arrepentimiento por esa falta y os perdone ese y tantos otros pecados que habéis cometido. Si os hice la pregunta es porque he visto que, aunque no tenéis mucha educación, no sois sin embargo tan ignorante como otros acerca de lo que es bueno. He observado que sabéis de religión bastante más de lo que habéis puesto en práctica.

    W. A.: No sois vos, señor, quien me ha arrancado esa confesión sobre mi padre, ha sido mi conciencia. Cuando nos ponemos a recordar nuestra pasada vida, los pecados contra nuestros indulgentes padres son los primeros que se nos aparecen y las heridas por ellos ocasionadas las que dejan huella más honda, así como el peso de tales faltas será siempre mayor que el de todas nuestras faltas juntas.

    R. C: Creedme, Atkins, que apenas puedo soportar palabras que me alcanzan tan profundamente.

    W. A.: ¡Oh, no, señor! Me atrevo a decir que no sabéis nada de tales sentimientos de culpa.

    R. C: Sí, Atkins. Cada playa, cada colina, hasta cada árbol de esta isla son testigos de la angustia que sintiera mi alma por mi ingratitud y el mal pago que diera a un bueno y cariñoso padre, a un padre que se parecía mucho al vuestro por lo que me habéis dicho. Yo maté a mi padre al igual que vos, Will Atkins, pero pienso que mi arrepentimiento es insignificante en comparación con el vuestro.



    Hubiera agregado otras cosas, de haber podido dominar mi emoción. Al comprender que el arrepentimiento de ese pobre hombre era mucho más sincero que el mío estuve a punto de interrumpir el diálogo y marcharme, ya que me sorprendieron sus palabras y terminé por pensar que en vez de ser yo quien lo instruyera, aquel hombre era ahora mi maestro e instructor de la manera más sorprendente e inesperada.

    Confié todo esto al joven clérigo, quien se mostró muy emocionado y me respondió:

    — ¿No os había dicho, señor, que cuando este hombre se convirtiera terminaría por enseñarnos a todos? Os aseguro que si llega a ser un verdadero penitente, no habrá necesidad de mí en esta isla; él la transformará en una colonia de cristianos.

    Sintiéndome algo más calmado, renové entonces mi conversación con Will Atkins.

    —Decidme, Will —pregunté—. ¿Cómo es que recién ahora nace en vos el sentimiento de vuestras culpas?



    W. A.: Señor, me habéis confiado una tarea que ha sido como si una luz se encendiera en mi alma. Estuve hablando a mi esposa acerca de Dios y de la religión, a fin de cumplir vuestros deseos y convertirla al cristianismo. Y creedme, ha sido ella quien me ha predicado un sermón que no olvidaré mientras viva.

    R. C: No, no ha sido vuestra esposa quien os ha predicado, Atkins, sino que mientras buscabais argumentos religiosos para persuadirla, la conciencia se ha despertado en vos.

    W. A.: ¡Ah, sí, señor, y con una fuerza irresistible!

    R. C: Os ruego, Will, que nos reveléis qué ocurrió entre vos y vuestra esposa, de lo cual alguna noticia tengo ya.

    W. A.: Me será imposible haceros una narración detallada, señor. Me siento todavía bajo esa impresión y, sin embargo, carezco de palabras para explicarla. Pero, creedme, haya dicho ella una cosa u otra, y aunque me sea imposible haceros el relato, lo que puedo aseguraros es que resolví  enmendarme y  rehacer mi vida.

    R. C: Decidnos algo de ello. ¿Cómo principiasteis, Will? Ha sido éste un caso extraordinario, no me cabe duda. Admirable sermón habrá predicado vuestra esposa para conseguir de vos semejante propósito.

    W. A.: Veréis, señor. Ante todo le expliqué nuestras leyes acerca del matrimonio y las razones por las que hombres y mujeres deben obligarse a un contrato del cual ya nunca puedan en adelante desligarse; le dije' que en caso contrario sería imposible mantener el orden y la justicia, que los hombres se alejarían de sus esposas y abandonarían a sus hijos, tanto que la unión familiar no podría mantenerse ni tampoco determinar las herencias por descendencia legal.

    R. C: ¿Y qué dijo ella?

    W. A.: Respondió que le parecía muy bien y que era mucho mejor que en su país.

    R. C: ¿Pero le explicasteis qué significaba el matrimonio?

    W. A.: Sí, señor, así principió nuestro diálogo. Le pregunté si deseaba que nos casáramos de acuerdo con nuestras leyes. Quiso entonces saber cuáles eran esas leyes. Le contesté que el matrimonio había sido instituido por Dios, y a partir de entonces sostuvimos el más extraño dialogo que, a mi parecer, hayan tenido jamás marido y mujer.

    (NOTA: El diálogo entre Will Atkins y su esposa, tal como lo registré por escrito apenas lo hube escuchado de sus labios, es el que sigue):

    MUJER: ¡Instituido por vuestro Dios! ¿Pero entonces tener un Dios en vuestro país?

    W. A.: Sí, querida mía, Dios está en todos los países.

    MUJER: Vuestro Dios no estar en mi país; mi país tener el gran antiguo dios Benamuki.

    W. A.: Hija mía, yo no estoy capacitado para explicarte quién es Dios. Sólo puedo decirte que Dios está en el cielo, que hizo el cielo y la tierra, el mar y todo cuanto en ellos hay.

    MUJER: NO hacer la tierra; vuestro Dios no hacer toda la tierra, él no hacer mi país,

    (Will Atkins sonrió al oír aquella ingenua frase.)



    MUJER: NO reír. ¿Por qué reír de mí? No ser cosa de risa.



    (El reproche era harto justificado, por cuanto ella se mostraba al principio más serio que él en la discusión.)



    W. A.: Tienes  razón,  querida mía, no volveré a reírme.

    MUJER: ¿Por qué decir vuestro Dios hacer todo?

    W. A.: Porque es así, hija mía. Dios es el Hacedor de todo, de mí y de ti y de todas las cosas. El es el único Dios, el verdadero; no hay más Dios que El, y El vive para siempre en el cielo.

    MUJER: ¿Por qué no decirme antes esto?

    W. A.: Tienes mucha razón, pero he sido un miserable y un malvado, que no sólo olvidó darte a conocer la verdad de todo esto sino que prescindió de Dios durante toda la vida.

    MUJER: ¡Cómo! ¿Vos tener el gran Dios en vuestro país y vos no conocerlo? ¿No decir «Oh» a El? ¿No hacer cosas buenas para El? ¡No ser posible!

    W. A.: Desgraciadamente es así. Vivimos como si no hubiera Dios en los cielos o como si El no tuviera poder sobre la tierra.

    MUJER: ¿Pero por qué El dejaros vivir así? ¿Por qué no haceros buenos?

    W. A.: Por culpa nuestra.

    MUJER: Pero vos decirme El grande, El muy grande, El tener mucho poder; El poder matar cuando quiera. ¿Por qué El no mataros cuando vos no servirle, no decirle «Oh», no ser hombre bueno?

    W. A.: También tienes razón. El podría fulminarme ahora mismo como lo tengo merecido, porque no he sido nunca más que un miserable y un malvado. Pero Dios es generoso y no nos castiga en la medida en que lo merecemos.

    MUJER: Pero entonces, ¿vos no dar las gracias a Dios por eso?

    W. A.: No, por cierto que no. Nunca he dado las gracias a Dios por su Misericordia y nunca lo he temido por Su poder.

    MUJER: Entonces vuestro Dios no ser Dios, yo no creer en El, yo no creer que El ser uno, grande, tener mucho poder, mucha fuerza. El no mataros aunque vos encolerizarlo mucho.

    W. A.: ¡Ah! ¿Es que mi miserable vida habrá de impedirte ahora que creas en Dios? ¡Qué abominable criatura soy, y qué triste verdad la de que los crímenes de los cristianos son los que impiden la conversión de los gentiles!

    MUJER: ¿Cómo pensar yo que vosotros tener allí gran Dios (señalaba hacia el cielo) y vos, sin embargo, no hacer bien, no ser bueno? ¿El ver eso? ¿O El no ver nada?

    W. A.: Sí, sí; Dios conoce y ve todas las cosas. El oye y habla, ve cuanto hacemos y sabe lo que pensamos, aunque no pronunciemos palabras.

    MUJER: ¡Cómo! ¿El oíros jurar, maldecir, hablar grandes maldiciones?

    W. A.: Sí, sí; El oye todo.

    MUJER: ¿Dónde estar entonces el gran poder, la gran fuerza?

    W. A.: Dios es misericordioso, es todo cuanto podemos decir al respecto, y ello prueba que es el verdadero Dios. Piensa que es Dios, no hombre; y por eso no nos fulmina.



    (Aquí Will Atkins nos confesó que se había sentido invadido por el horror al pensar cómo había podido decir a su mujer tan claramente que Dios ve, oye y conoce los más secretos pensamientos del corazón así como cuanto hacemos, y a la vez se había atrevido a llevar a cabo todas las villanías de que era culpable.)



    MUJER: ¡Misericordioso! ¿Qué querer decir con eso?

    W. A.: El es nuestro Padre y Hacedor, y por eso se apiada de nosotros y nos preserva.

    MUJER: Pero si El nunca matar, nunca enojarse cuando vos hacer mal, entonces El no ser bueno o no ser poderoso.

    W. A.: Sí, sí, querida mía. El es infinitamente bueno, infinitamente grande y posee poder para castigarnos. A veces a fin de mostrar a los hombres Su justicia, deja caer Su cólera sobre muchos pecadores y los destruye mientras están entregados a sus crímenes, para que sirva de ejemplo.

    MUJER: Pero a vos no mataros todavía. ¿O El deciros tal vez a vos que no mataros y entonces vos poder hacer cosas malas? ¿El no estar enojado con vos y sí enojado con otros hombres?

    W. A.: No, no; mis pecados han sido cometidos abusando de Su bondad, y El sería infinitamente justo si me destruyese como lo ha hecho con otros.

    MUJER: Pero no mataros, no haceros morir. ¿Y vos no decir nada a eso, no estar agradecido a El por todo eso?

    W. A.: No soy más que un miserable desgraciado, tal es la verdad.

    MUJER: ¿Por qué El no haceros mejor? Vos decir que El ser vuestro Creador.

    W. A.: El me creó al igual que el resto del mundo; soy yo quien me he echado a perder y abusado de Su bondad, convirtiéndome en un monstruo abominable.

    MUJER: YO querer que Dios conocerme, vos hacer que El conocerme. Yo no encolerizarlo, yo no hacer malas cosas.



    (Atkins sintió, como me lo dijo luego, que su corazón desfallecía al escuchar a aquella pobre e ignorante criatura expresar su ingenuo deseo de conocer a Dios mientras él, pecador incorregible, no podía enseñar a su mujer una sola palabra sin que su propia conducta no desmintiera a sus ojos lo que deseaba participarle. Con harta claridad le había manifestado ella que no podía creer en un Dios que no destruía y aniquilaba la maldad y el crimen.)



    W. A.: Querida mía, lo que tú quieres decirme es que yo te haga conocer a Dios, y no que Dios te conozca a ti. El ya te conoce, así como el más profundo pensamiento de tu corazón.

    MUJER: ¡Cómo! ¿El saber lo que yo deciros ahora? ¿El saber que yo desear conocerlo? ¿Cómo conocer yo a mi Hacedor?

    W. A.: ¡Pobre criatura, El será quien te enseñe, yo no puedo nacerlo! Solamente le ruego que te enseñe a conocerlo, y que me perdone ser tan incapaz de acercarte a El. (Fue entonces cuando, como habíamos visto hacerlo, se arrodilló alzando las manos para rogar.)

    MUJER: ¿Por qué caer de rodillas? ¿Por qué levantar las manos? ¿Qué decir, a quién vos hablar? ¿Qué significar?

    W. A.: Querida mía, me humillo en señal de sumisión a mi Hacedor. Le digo «Oh» como vosotros lo hacéis y como decís que vuestros ancianos lo hacen con el ídolo Benamuki. Lo que he hecho es rogar a Dios.

    MUJER: ¿Para qué decir «Oh» a Dios?

    W. A.: Le he suplicado que abra tus ojos y tu entendimiento para que puedas conocerlo, a fin de que El te acepte en su seno.

    MUJER: ¿El poder hacer eso?

    W. A.: Sí, El lo puede todo.

    MUJER: ¿Y El escuchar lo que vos decirle?

    W. A.: Sí, porque Dios nos mandó que le rogásemos y prometió que nos escucharía.

    MUJER: ¿Mandaros rogar? ¿Cuándo mandaros eso? ¿Cómo? ¿Entonces vosotros oírlo hablar?

    W. A.: No, no lo escuchamos hablar, pero El se reveló de muchas maneras a nosotros.



    (Aquí una gran confusión se apoderó de Atkins al querer explicar a su esposa la revelación de Dios a través de Su palabra; pero por fin lo hizo de la siguiente manera):



    W. A.: Dios, en días ya lejanos, habló desde el cielo a algunos hombres por medio de palabras comprensibles; así fueron esos hombres iluminados por Su gracia, y pudieron escribir Sus leyes en un libro.

    MUJER: YO no comprender esto. ¿Dónde estar el libro?

    W. A.: ¡Ay, desdichada criatura, no lo tengo! Sin embargo abrigo la esperanza de que algún día hallaré uno para ti, y te ayudaré a leerlo.

    (La abrazó entonces con profundo afecto, sintiendo una aflicción desgarradora por carecer de una Biblia.)

    MUJER: ¿Pero cómo asegurar vos que Dios enseñar esos hombres a escribir libro?

    W. A.: Por el mismo principio que nos permite saber que El es Dios.

    MUJER: ¿Principio? ¿Cuál principio? ¿Cómo vosotros conocer a El?

    W. A.: Porque Dios solamente enseña y ordena lo que es bueno, justo y santo, con lo cual tiende a hacernos absolutamente buenos y felices. Además El prohíbe y nos manda evitar todo lo que sea malo, tanto lo malo en sí o en sus consecuencias.

    MUJER: ¡Ah, yo querer entender bien esto, yo verlo bien! Si El enseñar lo bueno, si El prohibir lo malo, premiar cosas buenas, castigar cosas malas, El crear todo, El dar todo, oír cuando yo decir «Oh» a El como vos hacer ahora, El hacerme buena si yo querer ser buena, El preservarme, no hacerme morir si yo querer ser buena; todo eso que vos decir El hacer, entonces El gran Dios, yo aceptar, yo pensar, yo creer que El ser gran Dios. ¡Yo decir «Oh» a El con vos, querido mío!



    Al oír estas palabras, el desdichado no pudo resistir por más tiempo, sino que levantándose invitó a su esposa a arrodillarse juntos, y entonces oró en voz alta, rogando a Dios que se diera a conocer a aquella criatura por medio de Su gracia. Le pidió asimismo que de ser posible se dignara alguna vez hacerle llegar una Biblia, a fin de que su esposa pudiera encontrar allí la Palabra de Dios y ser enseñada por ella a conocer a su Creador.

    Todo esto coincidía con lo que habíamos visto, cuando Atkins tomó de la mano a su esposa y la hizo caer de hinojos a su lado.

    Después de esto sostuvieron otras conversaciones demasiado extensas para consignarlas aquí. En el curso de las mismas la mujer pidió a Atkins la promesa de que reformara su vida, desde el momento que él mismo admitía que su existencia anterior había sido una abominable serie de provocaciones contra Dios; le suplicó que no lo ofendiera más, agregando que de lo contrario Dios lo mataría (como ella se expresaba) y entonces, quedándose sola, no podría jamás aprender a conocer al Creador. También agregó que no quería que su esposo sufriera después de muerto los castigos que él le había explicado.

    Este extraordinario relato nos afectó profundamente, en especial al joven sacerdote. Se manifestó maravillado por lo ocurrido, pero su aflicción fue grande al pensar que no le era posible hablar con la mujer. Ignoraba el idioma inglés para dirigirse a ella, y el enrevesado modo de hablar de aquella criatura hacía imposible que él pudiese entenderle algo. Volviéndose, sin embargo, a mí, me dijo que quería hacer por ella algo más que casarla, y como no le comprendí en el primer momento me explicó que su intención era bautizarla.

    Mientras se preparaba para llevar a cabo la ceremonia, le manifesté que era conveniente celebrar el acto con algunas precauciones, a fin de que el hombre no advirtiera que el sacerdote pertenecía a la Iglesia romana, debido a las desastrosas consecuencias que podía acarrearnos una discusión acerca de la religión a la cual estábamos convirtiendo a los demás. Me respondió que como no había allí capilla consagrada ni los elementos apropiados para la ceremonia, haría las cosas de tal manera que ni siquiera yo advertiría que se trataba de un católico como no fuese por mi anterior conocimiento.

    Así fue. Después de pronunciar como para sí mismo algunas palabras en latín, derramó sobre la cabeza de la mujer un vaso de agua, a tiempo que le decía en francés y en voz alta:

    —María (nombre que yo, en mi carácter de padrino, había impuesto a la mujer a pedido de su esposo), te bautizo en el nombre del Padre, del Hijo y del Espíritu Santo.

    Fue por lo tanto imposible para nadie advertir cuál era su particular comunión. Dijo luego una bendición en latín, pero Will Atkins no pareció advertir que ese idioma no era el francés, y por cierto no reparó en nada de esto.

    Tan pronto hubo concluido la ceremonia, procedimos a casar a la pareja; luego, volviéndose a Will Atkins y con un tono muy afectuoso, el sacerdote lo exhortó no solamente a perseverar en su excelente disposición sino apoyar las convicciones que se habían adueñado de su conciencia mediante el propósito de reformar su vida. Le aseguró que de nada valía sostener que estaba arrepentido si no renunciaba totalmente a la maldad, y le mostró cómo Dios lo había honrado al convertirlo en el instrumento por el cual su esposa había adquirido las primeras nociones de la religión cristiana, y que por lo tanto debía cuidarse de no deshonrar la gracia de Dios. Le dijo que si cometía ese crimen vería pronto cómo la que había sido pagana llegaba al seno de Dios mientras él, instrumento de esa conversión, era arrojado lejos.

    Agregó luego multitud de consejos a ambos esposos y después de invocar para ellos la bondad del Creador volvió a bendecirlos, con palabras que yo les traduje al inglés y tras las cuales concluyó la ceremonia. Pienso que aquél fue el más hermoso y grato día de que haya yo gozado en mi vida entera.

    Pero mi sacerdote no había terminado aún. Sus pensamientos volvían una y otra vez a la conversión de los treinta y siete salvajes, e insistió en quedarse en la isla para emprenderla. Traté entonces de convencerlo diciéndole en primer término que la empresa era en sí misma imposible y, luego, que tal vez hallara yo una manera de llevarla a cabo con buen éxito aunque no estuviera él en la isla. Todo esto se verá mejor más adelante.






	\chapter{El adiós a la isla}





    Solucionadas en esta forma las cuestiones de mi colonia, me disponía a embarcarme cuando el joven a quien, casi muerto de hambre, rescatara del buque, acudió a verme y me dijo estar enterado de que había un sacerdote en mi compañía, el cual acababa de casar a los colonos ingleses con las mujeres salvajes; venía a decirme que también él sabía de una pareja de cristianos que deberían ser casados antes de la partida del navío, y esperaba que eso no me desagradaría.

    Pensé en seguida que se refería a la joven que fuera criada de su difunta madre, por cuanto no había otra cristiana en la isla. Me apresuré entonces a pedirle que no se precipitara en tal unión movido solamente por el hecho de sentirse solo en el mundo, haciéndole notar que era dueño de una buena fortuna y que tenía excelentes amigos entre los cuales me contaba yo así como la joven criada; le dije que ésta no era solamente una muchacha de humilde condición y sin medios de fortuna, sino que había entre ambos una gran diferencia de edades, ya que ella tenía veintiséis o veintisiete años mientras él no pasaba los diecisiete o dieciocho. Muy posiblemente, y contando con mi ayuda, lograría escapar a la soledad en que ahora se encontraba y retornar a su patria, donde se arrepentiría amargamente de su elección; el mutuo disgusto que a ambos iba a ocasionarle su diferencia social redundaría en su mutua infelicidad.

    Me disponía a darle más consejos y me dijo, muy humildemente, que estaba equivocado en mis pensamientos ya que nada de eso había cruzado por su mente, demasiado absorbida por sus desventuras y la melancolía que éstas le causaban. Manifestó su alegría por verme tan bien dispuesto a ayudarlo a que algún día volviera a su patria, agregando que si se quedaba en la isla era solamente por lo largo y azaroso de mi viaje, el cual lo alejaría cada vez más de toda región familiar, y que momentáneamente su único deseo consistía en que yo le concediera una pequeña propiedad en la isla, uno o dos sirvientes y las provisiones más necesarias, con lo cual se establecería como plantador y esperaría la época en que mi retorno a Inglaterra, si se producía alguna vez, me permitiese llevarlo consigo. Agregó que si yo volvía solo a mi patria, confiaba en mi buena memoria y que me daría algunas cartas para sus amigos ingleses donde les contaría mis bondades para con él, así como el lugar del mundo y la situación en que lo había dejado. Terminó diciéndome que si alguna vez salía de la isla, tanto la plantación como todas las mejoras que él fuera capaz de introducir en ella me pertenecerían por entero, fuera cual fuese su valor.

    Sus palabras, pronunciadas con harta propiedad para un joven de tan pocos años, me agradaron mucho, así como su explicación de que ese matrimonio del cual me hablara al comienzo no se relacionaba con él. Le aseguré que si yo vivía lo bastante como para retornar a Inglaterra entregaría sus cartas y haría por él todo lo que fuese necesario, sin olvidar nunca la situación en que había quedado. Pero mi impaciencia era grande por conocer quiénes eran las personas que deseaban casarse, y entonces me explicó que se trataba de Susana, la criada, y de mi Juan Sabelotodo.

    Cuando pronunció sus nombres me sentí gratamente sorprendido por cuanto aquella alianza me pareció aconsejable desde todo punto de vista. Ya he hablado del carácter del novio, y en cuanto a la doncella era muy honesta, humilde y sencilla, así como profundamente religiosa. Muchas veces había notado su buen sentido, tenía una presencia sumamente agradable, hablaba con sobria claridad, y sin ser tímida en exceso tampoco era de las que se inmiscuyen en las cosas que no les son propias. Sumamente diestra y hacendosa; en toda tarea daba la impresión de que sería una excelente ama de casa y sin duda hubiera podido administrar sin dificultad la isla entera. Parecía darse clara cuenta de la forma en que debía proceder con cada uno de los que la rodeaban en la comunidad, y distinguirlos sin equivocarse.

    Dispuestas así las cosas, procedimos a casarlos ese mismo día; como yo hice ante el altar el papel de padre de la muchacha, también quise darle una dote concediéndole al igual que a su esposo un buen pedazo de tierra para la plantación. Esta alianza así como el pedido que me hiciera el jovencito de un terreno para plantíos me obligó a parcelar la isla entre todos los colonos a fin de que más tarde no se produjeran querellas por los respectivos dominios.

    Confié esta tarea a Will Atkins, que se había transformado en el más grave, sobrio y responsable individuo, demostrando su profunda devoción religiosa, por lo cual y en la medida que puedo afirmarlo ante semejante transformación, creo sinceramente que se había arrepentido de sus pecados.

    Dividió tan justamente las tierras y fue tanta la satisfacción de todos por el reparto, que sólo me pidieron una escritura general firmada de mi puño y letra, la cual me apresuré a redactar indicando claramente los límites y situación de cada una de las parcelas y asegurando en ella que concedía derecho de propiedad y legado a cada uno de los colonos por las fracciones que les había adjudicado, así como las mejoras que introdujeron en ellas, propiedad que sería extensiva a sus herederos. Reservé el resto de la isla como de mi exclusiva propiedad, percibiendo una cierta renta por cada plantación, la que empezaría a pagarse a partir de los once años y me sería entregada si la reclamaba personalmente o enviaba a un apoderado munido de una copia de dicha escritura.

    En cuanto al gobierno y leyes de la isla, les manifesté que no me sentía capaz de señalarles mejores principios que los nacidos de su propia reflexión. Solamente les exigí la promesa de que vivieran en paz y buena vecindad, y así me preparé para marcharme de mi colonia.

    Hay algo que no debo omitir, y es el hecho de que estando la isla constituida en una especie de nación en la que había mucho trabajo a realizar, resultaba absurdo que treinta y siete salvajes vivieran sin ocuparse de nada en un rincón de aquella tierra. Salvo la tarea de procurarse alimentos, cosa que a veces les daba bastante trabajo, no tenían propiedades que mejorar ni tarea en la cual ocuparse. Propuse entonces al gobernador español que fuera a entrevistarse con ellos en compañía del padre de Viernes, y les propusiera cambiar de sitio y dedicarse ya sea a plantar por cuenta propia o a trabajar en calidad de sirvientes, distribuidos entre las diversas familias de la isla, a cambio de lo cual serían alimentados y retribuidos por su tarea, sin considerárselos desde algún punto de vista como esclavos. Jamás permitiría yo la esclavitud de aquellos salvajes, pues se les había prometido la libertad por su capitulación y dichas condiciones no debían ser violadas.

    Aceptaron con entusiasmo la propuesta, y vinieron de inmediato en compañía del español, por lo cual les distribuimos tierras y plantíos que tres o cuatro aceptaron en seguida mientras el resto prefirió emplearse en calidad de sirvientes y trabajar para las familias de colonos. La comunidad, quedó, por lo tanto, organizada de la siguiente manera:

    Los españoles poseían mi antigua residencia, capital de la isla, y extendían sus plantaciones hacia el lado del arroyuelo que formaba la ensenada tantas veces descrita, llegando hasta mi enramada; a medida que aumentaban sus plantaciones las llevaban más al oeste.

    Los ingleses vivían en el sector noroeste, donde al comienzo se habían radicado Will Atkins y sus camaradas, avanzando luego hacia el sur y sudoeste hasta quedar a espaldas de los españoles. Cada plantación tenía un amplio terreno para expandirse si la oportunidad se presentaba, de manera que jamás pudieran producirse cuestiones por falta de espacio.

    Toda la región oriental de la isla quedó deshabitada a fin de que si los salvajes desembarcaban según su costumbre para celebrar sus bárbaros festines pudieran hacerlo sin encontrar oposición alguna; quedó establecido que si no molestaban a nadie tampoco ellos serían estorbados. Así debió ocurrir muchas veces, llegando a la isla y marchándose a poco, pues nunca supe que los plantadores fuesen otra vez atacados o invadidos por ellos.

    Recordé entonces haber dejado entrever a mi amigo el sacerdote que el problema de la conversión de los salvajes podía llevarse felizmente a cabo durante su ausencia, y le manifesté que tenía las mejores esperanzas de que así ocurriese ahora que los salvajes estaban viviendo en compañía de los colonos, con tal que cada uno de éstos cumpliera esa obligación con los sirvientes que le tocara dirigir...

    El sacerdote manifestó su conformidad, suponiendo que los cristianos cumplieran dicha obligación.

    —Pero —agregó— ¿cómo podemos estar seguros de que lo harán?

    Le propuse reunir a los colonos y señalarles sus obligaciones en conjunto, o bien hablar por separado con cada uno, lo que pareció más acertado. Nos dividimos entonces la tarea, dedicándose él a los españoles, que eran todos papistas, y haciéndolo yo con los ingleses, que profesaban el culto protestante. Les formulamos expresas recomendaciones, arrancándoles la promesa de que jamás establecerían diferencia alguna entre catolicismo y protestantismo, mientras exhortaran a los salvajes a convertirse a la fe cristiana, sino que se limitarían a enseñarles el conocimiento del verdadero Dios y de Jesucristo, nuestro Salvador. Asimismo prometieron que nunca sostendrían entre ellos disputas concernientes a cuestiones religiosas. Cuando llegué a casa de Will Atkins —si puedo llamar así a una construcción de mimbres como no creo que haya habido jamás otra en el mundo— encontré a Susana, la joven de quien ya he hablado, en compañía de la esposa de Atkins, pues ambas se habían hecho muy amigas. La prudente y religiosa joven no había tardado en perfeccionar la obra espiritual principiada por Will Atkins, y aunque no habían transcurrido aún cuatro días desde los episodios antes narrados, la recién bautizada salvaje se había transformado en una de las mujeres más cristianas que me haya sido dado ver u oír a lo largo de mis andanzas por el mundo.

    Antes de hacer mi visita, esa misma mañana se me había ocurrido que entre las cosas útiles que debía dejar a aquella gente no había incluido una Biblia, mostrándome mucho menos considerado hacia ellas de lo que fuera mi excelente amiga la viuda del capitán, cuando al enviarme desde Lisboa cien libras esterlinas, no olvidó agregar un paquete con tres Biblias y un libro de oraciones. Con todo, la caridad de aquella buena mujer alcanzó límites que ella jamás habría imaginado, pues sus libros estaban ahora destinados al consuelo y la enseñanza de aquellos que serían más capaces que yo de aprovechar sus doctrinas.

    Puse, pues, una de las Biblias en mi bolsillo y cuando llegué a casa de Will Atkins y éste me manifestó con profunda alegría que Susana y su esposa habían estado hablando de religión, le pregunté si todavía continuaban reunidas, a lo que repuso afirmativamente. Entramos entonces en la casa y hallamos a ambas mujeres absorbidas en su diálogo. Al vernos, Susana enrojeció y quiso marcharse en seguida, pero yo le rogué que permaneciera todavía, manifestándole que la tarea a la cual se había consagrado era digna de elogio y que sin duda Dios la bendeciría por ella.

    Hablamos un rato, y como advertí que carecían de todo libro, aunque me cuidé de preguntarlo, echando mano al bolsillo saqué una Biblia y la tendí a Atkins.

    —Tomad —dije—, os he traído el auxilio que acaso os faltaba.

    No creo que hombre alguno en el mundo haya sentido más gratitud de la que él manifestó por aquella Biblia, ni que haya habido jamás alegría mejor fundada. Aquel infeliz había sido el más disoluto, temerario y perverso individuo que pueda concebirse y, sin embargo, constituía un vivo ejemplo para quienes emprendan la educación de sus hijos, pues demostraba que los padres jamás deben desmayar en sus enseñanzas y admoniciones y tampoco desesperar del buen éxito de sus palabras aunque aquéllos se muestren obstinados, refractarios, y en apariencia totalmente insensibles a la educación que se les imparte. Si alguna vez Dios toca con su gracia la conciencia de hombres así, la fuerza de la educación recibida en la infancia vuelve a ellos de inmediato, probando que no se había perdido sino que solamente dormitaba durante los años de extravío para finalmente surgir y mostrar sus beneficios.

    También la joven se mostró muy contenta del regalo, aunque tanto ella como su joven amo tenían biblias a bordo entre sus efectos que aún no habían sido desembarcados. Y ahora que ya he dicho tantas cosas de esta muchacha, no puedo omitir un relato que nos concierne a ambos y que contiene cosas realmente instructivas y notables.

    He contado ya a qué penurias se vio reducida aquella infeliz muchacha, cómo su ama padeció hambre y murió a bordo del desventurado navío que abordamos en alta mar. He narrado asimismo que los tripulantes del barco, al verse reducidos al último extremo de necesidad, principiaron por dar insignificantes raciones a la dama, su hijo y la criada, terminando por abandonarlos completamente a su destino.

    Conversando un día sobre los sufrimientos que entonces habían pasado, le pregunté si era capaz de describir lo que se siente cuando se está a punto de morir de hambre. Me contestó que tal vez fuera posible hacerlo, y el relato que entonces escuché fue el mismo que consigno a continuación.

    —Al principio, señor —dijo Susana—, pasamos unos días en la mayor penuria sintiendo los efectos del hambre, y por fin nos vimos enteramente privados de alimentos, excepto azúcar, un poco de vino y algo de agua. El primer día que transcurrió sin haber probado yo comida alguna, sentí al atardecer una sensación de vacío y náusea en el estómago, y hacia la noche empecé a bostezar y a sentirme soñolienta. Me tendí en la cucheta que había en la cámara de popa y dormí unas tres horas, al cabo de las cuales desperté sintiéndome más aliviada, quizá por haber bebido antes de dormirme un buen vaso de vino. Cuando hubieron transcurrido tres horas de vigilia, a eso de las cinco de la madrugada, volví a sentir el vacío y la náusea en el estómago, por lo cual me acosté nuevamente, pero ya no me fue posible dormir, sintiéndome muy débil y mareada. Así pasó todo el segundo día, alternándose en mí la sensación de hambre, luego náusea y deseos de vomitar, de la manera más extraña. A la segunda noche, habiendo tenido que acostarme sin comer absolutamente nada y con un trago de agua por toda bebida, quedé dormida y soñé que estaba otra vez en las Barbadas, que veía el mercado repleto de toda clase de provisiones y que yo me apresuraba a comprar y llevaba a mi ama, comiendo todos con verdadero apetito.

    »Sentía en ese momento la placentera impresión que recibe aquel que termina con una excelente comida, pero al despertar creí desvanecerme de angustia al percibir la intensidad del hambre que sentía. Bebí entonces el último trago de vino que nos quedaba, echándole un poco de azúcar a fin de suplir en lo posible la falta de alimento. Como mi estómago estaba vacío y nada había en él para digerir, el único efecto del vino fue producirme un estado vecino a la embriaguez, haciéndome caer en un sopor inconsciente en el que permanecí como atontada durante mucho tiempo, según me explicaron más tarde.

    »En la mañana del tercer día, después de una noche de extraños y confusos sueños, durante los cuales más dormitaba que dormía, me desperté furiosa y exasperada por el hambre. Si la razón no hubiese venido en mi auxilio ayudándome a vencer ese estado de ánimo creo que de haber sido madre hubiese peligrado en esos momentos la vida de mi hijo. Esta desesperación duró unas tres horas, durante las cuales por dos veces me sentí enloquecer como los furiosos que vemos en el manicomio, y en verdad que debía parecerme a ellos según me lo describió más tarde mi joven amo, quien puede repetiros las mismas cosas.

    »En uno de esos accesos de desesperación, no sé si por el balanceo del barco o porque me resbalé de improviso, el hecho es que caí golpeándome la cara contra un ángulo de la cama donde yacía mi señora y el golpe me produjo una abundante hemorragia nasal. El grumete trajo entonces una jofaina en la cual dejé verter mi sangre durante largo rato, y poco a poco me fui serenando hasta que la violencia de la fiebre cedió y las manifestaciones más terribles del hambre se aplacaron.

    »Volví de pronto a sentirme mal, y aunque quise vomitar nada había en mi estómago para hacerlo. Todavía continuó algún rato la hemorragia, hasta que repentinamente me desvanecí y todos pensaron que había muerto. Recobré más tarde el sentido, con el más horrible dolor imaginable en el estómago; es imposible describirlo: no se asemejaba a un cólico sino que la necesidad de alimentos parecía roerme desde adentro. Hacia la noche me sentí mejor, salvo el incesante deseo de comer, una ansiedad que presumo ha de parecerse a lo que siente una mujer encinta. Bebí algo de agua con azúcar, pero mi estómago rechazaba su sabor dulce y la vomité en seguida; traté entonces de beber agua sin azúcar y pude tolerarla, tras lo cual me tendí en el lecho rogando a Dios que por piedad me llevara consigo. Llena de esperanza ante la idea de la muerte, dormité un rato y al despertar desfalleciente por la falta de alimentos, me pareció como si ya estuviera muerta. Encomendé entonces mi alma al Señor, sintiendo a la vez el ardiente deseo de que alguien viniera a arrojarme al mar.

    »Mientras todo esto ocurría mi ama estaba al lado mío, expirante a mi parecer, pero la verdad era que soportaba el hambre con mucha más paciencia que yo, luego de haber dado a su hijo el último trozo de pan que le quedaba. Mi joven amo se negaba a comerlo, pero ella se lo mandó y creo que fue el alimento el que finalmente le salvó la vida.

    »Volví a dormirme al amanecer, y al despertar caí en una violenta crisis de llanto, tras de la cual sentí de nuevo las torturas del hambre. Me puse furiosa del modo más horrible, y creo que si mi ama hubiera estado ya muerta, a pesar de todo mi cariño hacia ella habría comido su carne con tanto gusto y despreocupación como si hubiese sido de un animal. Hasta recuerdo que una o dos veces estuve a punto de morderme el brazo. Por fin descubrí la jofaina donde había vertido la sangre que derramara la noche anterior y corriendo hacia ella tragué su contenido con tal apuro y con una avidez tan inmensa como si me maravillara de que nadie se hubiese fijado antes en él o si temiera que otro pudiera intentar arrebatármelo.

    »Tan pronto hube bebido, y a pesar de que la sola idea de lo que había hecho me llenaba de horror, sentí que el hambre se aplacaba algo y fui a tomar un trago de agua, sintiéndome mucho mejor y más tranquila por espacio de unas horas. Estábamos ya en el cuarto día de martirio, y así me mantuve hasta el caer de la noche, en el plazo de unas tres horas, volví a pasar por todos los estados ya descritos, o sea náusea, sueño, hambre rabiosa, dolor de estómago, nuevamente furia por comer, nuevas náuseas, accesos de locura, llanto, otra vez una furia incontenible, y así cambiando cada cuarto de hora hasta que mis pocas fuerzas quedaron extenuadas. Por la noche me dejé caer rendida, poniendo todo mi consuelo en la esperanza de morir antes de que fuera otra vez de día.

    »Durante la noche no dormí nada, pero ya el hambre se había convertido en enfermedad y sentí violentísimos cólicos y retortijones, ocasionados por la presencia en las entrañas de aire en vez de alimentos. Así permanecí hasta la mañana, cuando me llamaron a la realidad los gritos y lamentaciones de mi joven amo, quien aseguraba que su madre había muerto. Alcancé a enderezarme un poco, pues carecía de fuerzas para abandonar mi postura, advirtiendo que mi señora no estaba todavía muerta, aunque apenas daba señales de vida.

    »Sentí entonces tales convulsiones en el estómago por falta de alimento, que me sería imposible describirlas, y los arrebatos a que me llevaba el hambre fueron tan horribles que sólo las torturas de la muerte pueden dar una imagen de su violencia. Fue entonces cuando oí a los marineros gritar en la cubierta: "¡Una vela, una vela!", a la vez que lanzaban alaridos como si se hubieran vuelto locos.

    »Me fue imposible abandonar el lecho, lo mismo que mi señora; en cuanto a mi joven amo, estaba tan enfermo que le creí ya muerto, de manera que no pudimos abrir la puerta del camarote ni formarnos una idea clara de la causa de aquellas exclamaciones. Hacía dos días que no cambiábamos una palabra con nadie de la tripulación, pues nos habían manifestado que no tenían un solo bocado en todo el barco; más tarde supimos que ellos nos habían dado por muertos.

    »En tan espantosa situación nos hallábamos, señor, cuando vos fuisteis enviado para salvarnos la vida; las circunstancias en las cuales nos encontrasteis las conocéis mejor que yo misma.

    Tal fue el relato de aquella joven, y contiene tantos detalles acerca de lo que representa morir por inanición, que nunca lo hubiese imaginado y me resultó sumamente interesante. Pienso que se trata de una narración ajustada a la realidad, por cuanto coincidía en buena parte con lo que al respecto me contara el joven pasajero. Su relato no fue sin embargo tan detallado y conmovedor como el de la doncella, fuera del hecho de que su madre lo había salvado de morir a expensas de su propia vida. La criada, por el hecho de ser más fuerte y robusta que su ama, ya anciana y de constitución débil, había luchado más con la muerte y, por tanto, alcanzó a sentir más intensamente los tormentos del hambre. Está claro, como de la narración anterior se desprende, que si nuestro barco no hubiese encontrado providencialmente a aquellos seres en pocos días habrían muerto todos, a menos que una parte se hubiese devorado a la otra, y aun así de poco les habría servido, pues se encontraban a quinientas leguas de cualquier tierra, sin posibilidad alguna de auxilio salvo el que milagrosamente pudimos nosotros proporcionarles. Pero todo esto es dicho de paso, y vuelvo ahora a las disposiciones adoptadas con respecto a mis colonos.

    Ante todo es conveniente observar que, por muchos motivos, no me pareció necesario hacerles saber la existencia del balandro que trajera desarmado con la idea de aparejarlo en la isla. Había advertido —por lo menos a mi llegada— tales gérmenes de discordia entre aquellos hombres, que si les dejaba el balandro era evidente que al primer disgusto se produciría una separación y algunos se marcharían a otras tierras o se harían piratas, y la isla terminaría convirtiéndose en una guarida de bandoleros en vez de ser una plantación en que vivían gentes religiosas y honestas. Tampoco quise dejarles las dos piezas de artillería que trajera, así como los cañones que a mi pedido embarcara mi sobrino. Me parecía necesario pertrecharlos para una guerra defensiva contra cualquiera que pretendiese invadir sus dominios, pero no darles armas que les sirviesen para convertirse en atacantes o que les inspirase la idea de lanzarse a la conquista de otras regiones, lo cual finalmente sólo traería la ruina y destrucción sobre ellos y sobre sus planes. Guardé entonces el balandro, reservándolo igual que los cañones para que prestasen servicio en otra forma, como será contado más adelante.

    He terminado mi narración concerniente a la isla. Dejé a sus habitantes convenientemente instalados y en las mejores condiciones, embarcándome el cinco de mayo después de permanecer entre ellos veinticinco días. Como todos se mostraban resueltos a quedarse hasta que yo quisiera venir a buscarlos, les prometí hacerles llegar desde Brasil mayor número de socorros en cuanto se me presentase la oportunidad; en especial proyectaba remitirles algún ganado, tal como ovejas, cerdos y vacunos, ya que las dos vacas y los terneros que habíamos traído de Inglaterra tuvieron que ser muertos en alta mar por lo prolongado de nuestro viaje y el haberse terminado la avena que llevábamos a bordo.

    Al día siguiente, y luego de una salva de cinco cañonazos a modo de saludo, nos hicimos a la vela, arribando a la isla de Todos los Santos, en Brasil, luego de veintidós días de navegación durante los cuales no nos ocurrió nada de extraordinario salvo el episodio que paso a narrar. Llevábamos tres días de viaje con buen tiempo, cuando una corriente que nos hacía derivar hacia el E.N.E., llevándonos hacia una bahía o golfo en el continente, nos apartó algo de nuestro camino. Una o dos veces distinguieron nuestros hombres tierra hacia el este, pero si se trataba del continente o de islas no pudimos verificarlo en modo alguno.

    Al tercer día, hacia el anochecer, estando muy sereno el mar y bonacible, vimos el agua en dirección a las tierras cubierta de alguna capa o mancha negra. Durante largo rato no pudimos distinguir de qué se trataba hasta que el segundo, trepando a una de las cofas y mirando en aquella dirección con el catalejo, gritó que era una armada.

    No entendí al principio lo que quería decir con aquello de la armada, y hablé con demasiada brusquedad, llamándole tonto o algo parecido.

    —Señor —respondió—, no os enfadéis porque en verdad es una armada, o mejor aún, una flota, ya que alcanzo a distinguir no menos de mil canoas y pronto las veréis vos también, pues vienen remando rápidamente hacia aquí.

    Me sentí bastante alarmado al oír esto, así como mi sobrino el capitán, que había oído referir en la isla terribles historias acerca de los salvajes; como nunca había navegado antes por esos mares no sabía qué pensar de lo que nos esperaba, y recuerdo que dos o tres veces me dijo que sin duda seríamos devorados. Debo admitir por mi parte que, con la falta de viento y la corriente que nos llevaba en dirección a tierra, empezaba a temer lo peor. Le dije, sin embargo, que no se asustara sino que mandase echar el ancla tan pronto tuviéramos la seguridad de que iba a producirse el combate.

    El tiempo se mantenía sereno y la flotilla avanzaba rápidamente hacia nosotros de manera que ordené anclar de inmediato y aferrar las velas. Dije a la tripulación que el único peligro estaba en que aquellos salvajes intentasen prender fuego al barco, por lo cual era preciso arriar los botes y tenerlos listos, uno a popa y otro a proa, a la espera de lo que pudiera ocurrir. Los hombres que tripularan los botes debían estar provistos de velas mojadas y de cubos para sofocar todo principio de incendio que los salvajes ocasionaran en la estructura exterior del barco. Así los esperamos y poco rato más tarde estaban ya cerca de nosotros, constituyendo uno de los espectáculos más espantosos que pudiera imaginar un cristiano. Mi segundo se había equivocado grandemente al estimar el número de canoas, pues el total resultó ser de ciento veintiséis, pero asimismo resultaba una flota temible por cuanto algunas de las piraguas estaban tripuladas por dieciséis o diecisiete hombres, y las más pequeñas contenían seis o siete.

    Cuando se pusieron a tiro, los salvajes dieron la impresión de sentirse llenos de asombro y, maravilla, ante la contemplación de algo que jamás habían visto antes. Por lo que advertimos más tarde, tampoco ellos sabían qué actitud adoptar respecto a nosotros. Se aproximaron empero temerariamente como si tuvieran intención de rodear el navío, pero gritamos a los tripulantes de nuestros botes que no les dejaran acercarse demasiado.

    Esta orden produjo contra nuestros deseos un combate contra los salvajes, porque seis o siete de sus canoas se acercaron tanto a nuestra chalupa que los tripulantes de ésta les hicieron seña de que retrocedieran; entendieron perfectamente, pero mientras se retiraban les enviaron cerca de cincuenta tiros de flecha, uno de los cuales hirió gravemente a un marinero de la chalupa.

    Pese a lo ocurrido, mandé que no se disparase contra los salvajes por ningún motivo, y entretanto bajamos algunos tablones al bote, donde el carpintero los dispuso de manera de parapeto que protegiera a los tripulantes si los atacantes disparaban una nueva andanada.

    Una media hora más tarde vimos que avanzaban en montón por el lado de popa, acercándose tanto que pudimos distinguir fácilmente de qué gentes se trataba aunque no alcanzáramos a comprender su propósito. Me bastó mirarlos para reconocer en ellos a mis viejos amigos los salvajes, los mismos que tanto me habían dado que hacer en la isla. Vimos que remaban alejándose un poco; después avanzaron por un costado del barco y llegaran a ponerse tan cerca que los oíamos hablar con toda claridad. Ordené a mis hombres que se mantuviesen ocultos para evitar que nuevas flechas fuesen disparadas y que alistaran los fusiles para cualquier evento, tras lo cual dije a Viernes que, desde la borda, preguntara a aquellos salvajes en su idioma qué querían de nosotros, cosa que cumplió de inmediato. No sé si lo entendieron o no, pero tan pronto les hubo hablado vimos que seis de ellos, tripulantes de la canoa más próxima a nosotros, apartaban su embarcación y volviéndose nos mostraban sus desnudas espaldas. Si aquello era un desafío o una provocación, lo ignoro, salvo que se tratara de una muestra de desprecio o bien una señal convenida con los demás. Lo cierto es que Viernes gritó entonces que iban a disparar, y desgraciadamente para él, ¡desdichado muchacho!, unas trescientas flechas volaron sobre nosotros matando al pobre Viernes que para mi desesperación era el único que les servía de blanco. El infeliz fue alcanzado por no menos de tres dardos, y otros cayeron cerca de donde estaba, lo que prueba su poca puntería.

    Tan enfurecido me sentí con la pérdida de mi antiguo criado, compañero de tantas desventuras y tanta soledad, que ordené de inmediato cargar cinco cañones con metralla y otros cuatro con bala, y les enviamos una andanada como seguramente no habían recibido en su vida. Se hallaban a unas cincuenta varas del navío cuando tiramos sobre ellos, de manera que los artilleros pudieron apuntar con tanta precisión que probablemente cada tiro hundió por lo menos cuatro o cinco canoas. No puedo decir cuántos salvajes matamos o cuántos resultaron heridos, pero por cierto que espanto y apuro semejantes no se vio jamás antes. Trece a catorce canoas estaban volcadas, con sus tripulantes nadando alrededor; el resto, enloquecido de espanto, remaba con toda la rapidez posible sin ocuparse de recoger aquellos cuyas piraguas se habían hundido o hacían agua a causa de nuestras balas. Supongo que muchos se ahogaron allí mismo, y nuestros hombres recogieron a uno de aquellos infelices que luchó más de una hora por mantenerse a flote y salvar la vida.

    La metralla de los cañones debió sin duda matar y herir a muchos, pero no llegamos a averiguarlo con certidumbre, porque escaparon a tal velocidad que tres horas más tarde apenas si divisábamos unas tres canoas retrasadas, y del resto no tuvimos más noticias, porque levantándose viento esa misma tarde levamos el ancla y seguimos viaje hacia Brasil.

    Cierto que conservamos un prisionero a bordo, pero el salvaje era tan hosco que se negaba a hablar o a comer, tanto que creímos estaba dispuesto a dejarse morir de hambre. Hallé, sin embargo, la manera de disuadirlo, pues ordené que lo bajaran a la chalupa y le hicieran entender que si no hablaba volvería a ser arrojado al mar. Ni siquiera así quiso doblegarse, y entonces los marineros lo arrojaron al agua, y se alejaron en seguida; el salvaje, que nadaba como un pez, siguió de cerca a la chalupa gritando en su idioma palabras que los marineros no podían comprender. Por fin condescendieron a recogerlo de nuevo a bordo, y entonces se mostró mucho más tratable. De más está decir que nuestra intención no había sido en ningún momento el que el salvaje se ahogara.

    Seguimos nuestro viaje, sintiéndome yo el más desconsolado de los hombres por la muerte de mi pobre Viernes, y hasta estuve tentado de volverme a la isla en busca de alguno de los que allí quedaran para que me acompañase en adelante, pero eso no podía ser, de manera que proseguimos el viaje. Transcurrió un tiempo antes de que lográramos hacer comprender a nuestro prisionero las más simples cosas, pero nuestros hombres le enseñaron por fin algo de inglés y su carácter se hizo más sumiso. Le preguntamos entonces de qué país venía, pero nada sacamos en claro de sus respuestas, porque sus palabras eran tan guturales y hablaba con voz tan profunda y ronca que no podíamos entenderle absolutamente nada. Fueron todos de opinión de que un lenguaje semejante lo mismo podía hablarse con una mordaza, a juzgar por el sonido; tampoco pudimos percibir que empleara para vocalizar los dientes o la lengua, y tampoco el paladar y los labios, sino que formaba sus palabras lo mismo que un cuerno de caza forma su sonido, sin modulación alguna.

    Más tarde, cuando le enseñamos un poco de inglés, nos dijo finalmente que en aquella ocasión iban a librar una gran batalla conducidos por sus reyes. Le preguntamos cuántos reyes tenían, y respondió que habían cinco «nación» (no le pudimos hacer entender los plurales) y que las cinco estaban aliadas con otras dos. Quisimos saber por qué se habían acercado a nuestro navío.

    —Para hacer gran maravilla ver —respondió.

    Vale la pena observar que todos los nativos, así como los oriundos de África, no pueden hablar el inglés que han aprendido sin agregar dos «e» al final de las palabras donde nosotros sólo empleamos una, y además hacen caer el acento sobre dichas palabras, como «makeé» en vez de «make», etc.; y resulta imposible desarraigar ese hábito en ellos. Yo mismo recuerdo el trabajo que tuve para corregir la pronunciación de Viernes, bien que al final obtuve buen resultado.

    Y ahora que he vuelto a nombrar al pobre muchacho, debo dedicarle mi último adiós. ¡Fiel e infortunado Viernes! Lo sepultamos con toda solemnidad y del mejor modo posible, encerrando su cuerpo en un ataúd y arrojándolo al mar. Ordené que disparasen once cañonazos en su honor, y así terminó la existencia del más agradecido, fiel, honesto y cariñoso criado que jamás haya tenido hombre alguno.

    Con viento favorable seguimos nuestra ruta a Brasil doblamos el cabo San Agustín y tres días más tarde anclávamos en la bahía de Todos los Santos, antiguo lugar de mi liberación y sitio de donde provino mi buena o mala suerte.

    Nunca barco alguno llegó a estas tierras con menos negocios a ventilar que el nuestro, y sin embargo no fue sin gran dificultad que pudimos comunicarnos con tierra firme. No me valió mi socio el plantador, aún vivo y gozando gran prestigio en la región, así como tampoco mis dos comerciantes apoderados; ni siquiera la fama de mi asombrosa aventura en la isla me sirvió de pasaporte para entrar en el país. Mi socio recordó entonces que yo había enviado quinientos moidores al prior del monasterio de los agustinos, así como doscientos setenta y dos para ser repartidos entre los pobres, y rogó al prior que intercediera ante el gobierno para que se me dejase desembarcar en compañía del capitán y otra persona, aparte de ocho marineros, sin que nadie más tocara tierra. La expresa condición que se nos imponía era la de que de ninguna manera intentáramos desembarcar mercancías, así como cualquier otra persona que careciera de especial permiso.

    Se mostraron tan estrictos con nosotros en lo que respecta al desembarco de mercancías que con extrema dificultad pude conseguir bajar a tierra tres bultos de efectos ingleses tales como telas finas, paños y algo de lienzo que había traído para obsequiar a mi asociado.

    Era aquel hombre un caballero generoso y cordial, bien que al igual que yo provenía de humilde condición. Aunque desconocía mi intención de hacerle un regalo, se apresuró a mandar a bordo un presente compuesto de provisiones frescas, vinos, confituras, que valían más de treinta moidores, incluyendo algo de tabaco y cuatro hermosas medallas de oro. Yo, por mi parte, estaba en condiciones de retribuir adecuadamente con mi obsequio que, como he dicho, consistía en géneros finos, paños ingleses, encajes y telas de Holanda. Le hice llegar asimismo un valor aproximado de cien libras esterlinas en efectos parecidos, con otro destino, y le pedí que mandara armar el balandro que había traído conmigo desde Inglaterra para el uso de mi colonia, a fin de que en él fuesen enviados los auxilios prometidos.

    Pocos días más tarde, y de acuerdo con mi encargo, el balandro estuvo totalmente aparejado, puesto que sólo había que juntar las piezas. Di entonces a quien habría de capitanearlo las instrucciones adecuadas para reconocer la isla y, como más tarde supe por noticias de mis socios, la encontró sin dificultades. Pronto estuvo el balandro repleto con el pequeño cargamento que enviaba a la colonia, y uno de nuestros marineros que había bajado a tierra conmigo me ofreció ir en el navío y establecerse en mi isla, si yo le daba una carta para el gobernador español por la cual le concediera tierras a fin de iniciar una plantación, algunas ropas e instrumentos necesarios para ese trabajo que por lo visto entendía muy bien, pues me aseguró que había sido antiguo plantador en Maryland y bucanero para más detalles.

    Animé a aquel hombre concediéndole cuanto me pedía y como condición le confié el salvaje que apresamos y que sería un esclavo en condición de prisionero de guerra; escribí entonces al gobernador español ordenándole que concediera al nuevo colono una participación análoga a la del resto de la comunidad.

    Cuando todo estuvo listo para la partida, mi socio me manifestó que conocía a un hombre excelente, plantador portugués, que se veía en dificultades con la Iglesia.

    —Yo no sé en realidad qué le ocurre —me dijo—, pero tengo para mí que es un hereje en el fondo de su alma y se ha visto obligado a ocultarse por temor a la Inquisición.

    Me explicó que aquel hombre se sentiría feliz de poder escapar con su esposa y sus dos hijas, y que si yo lo autorizaba a trasladarse a mi isla y le concedía una plantación, él por su parte le suministraría elementos para empezar, ya que los agentes de la Inquisición habían confiscado todos sus bienes y fortuna, dejándole sólo algunos efectos domésticos y dos esclavos.

    —Aunque discrepo con sus principios —declaró mi socio— no quisiera verlo caer en las manos de sus enemigos porque seguramente sería quemado vivo.

    Accedí a la petición, agregando esa nueva familia a mi inglés, y ocultamos al plantador, su esposa e hijas a bordo de nuestro barco hasta que el balandro levó anclas para hacerse a la mar, y como ya habíamos hecho trasladar los pocos bienes de aquel hombre, trasbordamos a la familia cuando nos encontrábamos fuera de las aguas de la bahía.

    Nuestro marinero pareció muy complacido con su flamante socio. En realidad sus recursos eran casi iguales; utensilios y herramientas, muchos preparativos y una hacienda en perspectiva, pero nada más que eso para principiar. Llevaban sin embargo algo que valía más que todo el resto, es decir, instrucciones para el cultivo de la caña de azúcar, así como plantas de caña cuyo cuidado el plantador portugués conocía muy bien.

    Entre las distintas cosas que enviaba yo a mis colonos, hice embarcar en el balandro tres vacas lecheras con cinco terneros, unos veintidós cerdos y tres caballos.

    Tres mujeres portuguesas formaron parte del pasaje, con destino a mi isla, donde irían a reunirse con los españoles. Escribí una carta recomendando a éstos que se casaran tres de ellos con aquellas mujeres y que se mostraran considerados hacia ellas. Hubiera podido enviar más mujeres, pero recordé que el pobre portugués perseguido tenía dos hijas, siendo sólo cinco españoles los que deseaban casarse en la isla, pues el resto tenía esposas, aunque en lejanas tierras.

    El cargamento arribó sin novedad y fue, como podéis imaginar, recibido con inmensa alegría por mi colonia, que ahora se acrecentaba con los nuevos habitantes hasta contar entre sesenta y setenta, sin los niños que eran ya muy numerosos. En Londres hallé cartas de todos ellos, que habían sido remitidas vía Lisboa y que pude leer a mi retorno a Inglaterra, cosa de la que hablaré más adelante.

    He terminado con esto mi relato acerca de la isla y lo concerniente a ella; quienes continúen leyendo el resto de mi narración harán bien en apartar su mente por completo de aquella tierra, conformándose solamente con la historia de las locuras de un anciano que no supo adquirir experiencia ni por sus propias desgracias ni por las ajenas; que no abatido luego de casi cuarenta años de miseria y decepciones y no satisfecho con una propiedad superior a sus esperanzas, era incapaz de sentar cabeza ante la aflicción y las catástrofes más espantosas.






	\chapter{La aventura de Madagascar}





    Yo no tenía más razones para lanzarme a un viaje a las Indias Orientales de las que pudiera tener un hombre libre, que no ha cometido crimen alguno, para presentarse al alcaide de la cárcel de Newgate y pedirle ser encerrado con los presos y puesto a pan y agua. Si hubiera fletado en Inglaterra un barco de poco tonelaje para ir directamente a mi isla, llevando como cargamento los mismos auxilios para los colonos que embarqué en el navío que me trajo; si me hubiese apresurado a solicitar del gobierno un derecho de propiedad de la isla que sólo quedaría sujeta a la Corona de Inglaterra; si, llevando conmigo armas y municiones, criados y pobladores para establecer y tomar firme posición de mi dominio, lo hubiese fortificado en nombre de Inglaterra, acrecentando al mismo tiempo la población, como era bien simple de hacer; si entonces me hubiera establecido allí, enviando de vuelta al navío con un cargamento de excelente arroz, cosa posible en un plazo de seis meses, con órdenes a mis amigos europeos para que volviesen a cargarlos con otros efectos necesarios a la colonia; en fin, se hubiera hecho todas esas cosas quedándome en persona en la isla, entonces hubiese actuado como un hombre de buen sentido. Pero yo estaba dominado por un espíritu errante y me burlaba de todos los beneficios. Parecíame bastante ser el amo de toda aquella gente que había puesto en la isla, y conducirme ante ella con la arrogancia y la majestad de un antiguo monarca patriarcal. Pensaba haber cumplido mi deber enviándoles socorros, como si hubiese sido el padre de aquella familia, así como era el fundador de las plantaciones. Pero jamás cruzó por mi mente colonizar en nombre de algún gobierno o nación, o reconocer una determinada soberanía así como incluir a mis colonos en calidad de súbditos de una nación u otra. Ni siquiera me ocupé en dar nombre a la isla sino que la dejé tal como la encontrara, sin dueño real, con una población privada de todo gobierno y disciplina que no fuesen lo que mi deseo les imponía, cuando en realidad, y aunque yo tuviese influencia sobre aquellas gentes en mi carácter de bienhechor, carecía de verdadero poder y autoridad para tomar decisiones y dar órdenes en uno u otro sentido, órdenes que ellos cumplían solamente por voluntario consentimiento.

    Con todo, si hubiese quedado allá las cosas hubieran marchado bastante bien, pero como me fui para no regresar jamás, las últimas noticias que de la isla tuve me llegaron por intermedio de mi socio, quien había enviado tiempo después otro balandro a la colonia y me escribió una carta al respecto; carta que yo leí recién cinco años más tarde. En ella me contaba que la colonia declinaba y que los pobladores se quejaban de su excesiva permanencia en la isla; Will Atkins había muerto y cinco de los españoles habían acabado por marcharse. Aunque los salvajes no los molestaron mucho, sin embargo tuvieron varias escaramuzas con ellos. Por fin, los colonos suplicaron a mi socio que me escribiera recordándome la promesa empeñada en sacarlos alguna vez de la isla, porque todos deseaban ver una vez más su patria antes de morir.

    Pero por aquel entonces me había lanzado yo a perseguir nuevas quimeras, y quien quiera saber de mí deberá acompañarme a través de una nueva serie de locuras, temeridades y arriesgadas aventuras. No es el momento de detenerme a considerar las razones o el absurdo de mi conducta, sino que continúo con mi historia. Estaba embarcado para un cierto viaje, y ese viaje es el que ahora he de proseguir.

    Agregaré solamente que mi excelente amigo y piadoso sacerdote se separó de mí en Brasil. Un barco zarpaba rumbo a Lisboa y me pidió consentimiento para embarcarse en él, aunque como me dijo parecía destinado aún a no alcanzar nunca el final de su viaje. ¡Cuánto mejor habría sido para mí si lo hubiese acompañado en su regreso!

    Del Brasil pusimos proa hacia el Cabo de Buena Esperanza, y tuvimos un buen viaje, rumbeando casi continuamente hacia el SE. Sufrimos una que otra tormenta así como vientos contrarios, pero estaba escrito que mis desastres marítimos habían concluido y que todas las desventuras que me esperaban acontecieran en tierra, lo que basta para demostrar que la tierra sirve tanto como el mar de azote y castigo cuando el Cielo, que dirige el orden de los acontecimientos, la elige para ello.

    Nuestro barco hacía un viaje comercial, y llevaba a bordo un sobrecargo, quien debería decidir todas las operaciones una vez que hubiesen llegado al Cabo. Sólo cierto número de días estaba permitido al barco permanecer en cada puerto, según lo estipulaba el contrato; todo esto no me concernía en absoluto por cuanto eran asuntos a decidir entre mi sobrino el capitán y el sobrecargo mencionado.

    Nos quedamos en el Cabo tiempo suficiente para renovar la provisión de agua dulce, y seguimos inmediatamente rumbo a la costa de Coromandel. Se nos había informado que un navío de guerra francés de cincuenta cañones así como dos barcos mercantes de gran calado se habían hecho a la vela con rumbo a las Indias, y como yo sabía que estábamos en guerra con Francia sentía no poca aprensión, pero por lo visto siguieron su camino y no tuvimos más noticias de aquellos barcos.

    Indicaré brevemente los puertos y sitios que tocamos, así como los episodios acontecidos mientras íbamos de uno a otro. Ante todo arribamos a la isla de Madagascar donde, aunque los nativos son fieros y traidores, y están muy bien armados con lanzas y flechas que disparan con una precisión asombrosa pudimos sin embargo trabar relaciones sumamente amistosas, por lo menos al comienzo.

    Los nativos nos trataron amigablemente, y a cambio de algunas baratijas que les dimos, tales como cuchillos y tijeras, nos trajeron once bueyes de tamaño mediano pero de carne excelente, que aceptamos para tener algo de carne fresca durante la travesía y salar el resto.

    Luego de avituallarnos nos fue preciso permanecer todavía algunos días en esa tierra, y yo, que fui siempre lo bastante curioso para explorar todos los rincones del mundo adonde me conducía mi destino, aprovechaba para desembarcar lo más seguido posible. Una tarde, estando en la costa oriental de la isla, decidimos bajar a tierra; los nativos, que dicho sea de paso, eran allí harto numerosos, vinieron en tropel hacia nosotros y se detuvieron a cierta distancia para contemplarnos. Como hasta ese momento habíamos recibido trato amistoso de su parte y nuestro tráfico comercial era continuo, no sentimos ningún temor. Sin embargo, al ver aquella muchedumbre cortamos tres ramas de un árbol y las clavamos a cierta distancia de donde estábamos, lo que en aquel país equivale a una señal de tregua y amistad. Es costumbre que cuando una parte ha cumplido con esa ceremonia, la otra hace lo mismo en señal de que acepta la tregua de amistad. Hay, con todo, una condición expresa en eso, y es que uno no debe traspasar los límites fijados por las tres ramas de los nativos, así como ellos tampoco intentan hacerlo con el límite contrario. Las dos partes están entonces perfectamente seguras detrás de sus ramas, y el espacio existente entre un límite y otro es terreno neutral que sirve de mercado para comerciar, hacer intercambios y trabar amistad. Se sobreentiende que cuando se entra en el sector neutral hay que hacerlo sin arma alguna, y en cuanto a los nativos, jamás se acercan allí sin antes dejar sus jabalinas y lanzas junto a las tres ramas. Si alguna violencia es entonces cometida se apresuran a correr en dirección a sus ramas recogen las armas y la tregua queda rota.

    Aquella tarde, al desembarcar, ocurrió que gran cantidad de nativos vinieron a nuestro encuentro mostrándose como siempre amistosos y cordiales. Nos trajeron diversas clases de alimentos a cambio de los cuales les dimos las baratijas que traíamos, y las mujeres indígenas, por su parte, nos ofrecieron raíces y leche así como otros productos que nos resultaban útiles. Todo parecía tranquilo, y terminamos por levantar una especie de choza con ramas de árboles, dispuestos a pernoctar en la costa.

    No sé en realidad la causa de mi desazón, pero no me sentía tan dispuesto como mis compañeros a quedarme allí toda la noche. Nuestro bote había quedado anclado a tiro de piedra, con dos hombres a bordo para vigilarlo, e hice que uno de ellos viniera a reunirse con los demás; tomando luego ramas para abrigarme en el bote, tendí la vela en el fondo de la embarcación y pasé allí toda la noche, protegido por las ramas.

    A eso de las dos de la madrugada oímos de improviso los alaridos de uno de nuestros hombres, gritando que por Dios acercáramos el bote para que pudiesen embarcar porque de lo contrario serían todos asesinados. En el mismo instante oí los disparos de cinco mosquetes, número total de armas que tenían, y la descarga se repitió por tres veces, ya que parece que aquí los indígenas no se asustaban tanto con las armas de fuego como los salvajes americanos de quienes mucho he hablado.

    Todo esto sucedía sin que termináramos de darnos clara cuenta de lo que pasaba, hasta que despertándonos completamente por tan terribles gritos, ordené que el bote fuera llevado a tierra mientras, armados con los tres fusiles que había a bordo, nos disponíamos a defender a nuestros compañeros.

    Tan pronto acercamos la embarcación a tierra pudimos comprobar el apuro en que estaban aquellos hombres, pues lanzándose a la playa se hundieron en el agua en su ansia de llegar lo antes posible al bote, perseguidos de cerca por unos trescientos o cuatrocientos nativos. Los nuestros eran en total nueve, pero sólo cinco tenían fusiles mientras el resto debía arreglarse con pistolas y espadas, de muy poca utilidad en la emergencia.

    Embarcamos a siete de los nuestros con mucho trabajo, pues dos de ellos estaban malheridos. Lo peor fue que mientras permanecíamos en la borda ayudando a trepar a los fugitivos, los indígenas nos hicieron correr un riesgo igual al que aquéllos sufrieran en tierra, pues nos lanzaron tal lluvia de flechas que nos vimos precisados a parapetarnos con los bancos y dos o tres tablas sueltas que, por milagro o buena fortuna, teníamos providencialmente a bordo.

    Con todo, de haber sido de día y con la extraordinaria puntería que según parece tienen aquellos nativos, por poco que hubieran podido distinguirnos, es difícil que nos hubiéramos salvado de sus flechas. A la luz de la luna alcanzamos vagamente a verlos agrupados en la orilla, de donde nos arrojaban sus flechas y dardos. Ya para entonces habíamos alistado nuestras armas, y les hicimos una descarga que a juzgar por los alaridos que vinieron de tierra alcanzó a herir a unos cuantos. No se movieron sin embargo de allí; se quedaron en línea de batalla hasta el amanecer, probablemente con la intención de ejercitar entonces su puntería.

    Permanecimos en tal situación sin poder levar el ancla ni desplegar la vela porque hubiéramos debido enderezarnos en el bote, ofreciéndoles un blanco tan seguro como lo es un pájaro en un árbol para quien le dispara con municiones.

    Hicimos entonces señales de auxilio al barco, que estaba anclado una milla más lejos. Mi sobrino el capitán, oyendo nuestras descargas y al notar, por medio de su catalejo, la situación en que nos encontrábamos, comprendió perfectamente lo ocurrido y ordenando levar anclas aproximó el barco a tierra todo lo posible, al mismo tiempo que otra chalupa tripulada por diez hombres acudía a socorrernos. Gritamos a los del otro bote que no se acercaran demasiado, explicándoles el peligro existente, pero vinieron lo mismo cerca de nosotros y uno de los hombres se arrojó al agua llevando el extremo de un cable de remolque; como nadaba en el espacio existente entre nuestro bote y el suyo, situado más allá, estaba a cubierto de las flechas, y llegó felizmente junto a nosotros, que nos apresuramos a asegurar el remolque y luego de soltar nuestro cable y abandonar el ancla fuimos llevados poco a poco más allá del alcance de las flechas, sin que un solo instante nos atreviéramos a abandonar nuestra barricada.

    Tan pronto nos apartamos de la línea del buque, dejándole la playa en descubierto, vimos que se ponía de costado y descargaba de inmediato una andanada de balines, metralla de hierro y plomo y proyectiles semejantes, aparte de las balas mayores que hicieron terribles estragos entre los nativos.

    Una vez a bordo y fuera de peligro, nos pusimos a indagar la causa de lo acontecido, y fue precisamente el sobrecargo quien me pidió lo hiciera, pues él había estado varias veces en aquellas regiones y aseguraba firmemente que los nativos jamás hubieran roto una tregua pactada sin que de nuestra parte hubiese existido provocación. Por fin se supo que una anciana, que traspusiera su línea de ramas para vendernos leche, había traído a una joven de acompañante, la cual a su vez ofrecía algunas raíces o legumbres. Mientras la anciana, de la que no sabemos si era o no su madre, vendía la leche a algunos hombres, uno de los nuestros se condujo atrevidamente con la moza, por lo cual la vieja mujer se enfureció al punto. El marinero, despreciando sus amenazas, se llevó a la muchacha fuera de su vista y en dirección a los bosques, siendo a esa hora casi de noche. La anciana se marchó entonces sola, pero como puede imaginarse sus protestas y gritos se renovaron entre los nativos, los cuales en las primeras horas de la noche formaron aquel considerable ejército con el que estuvieron a un paso de destruirnos a todos.

    Uno de nuestros hombres cayó muerto al instante, atravesado por una lanza que le dispararon en momentos en que salía de la improvisada choza. Todos los restantes pudieron salvarse menos el marinero causante de aquella gresca, el cual pagó harto caro su conducta hacia la muchacha y del cual no tuvimos noticias durante largo tiempo. Dos días más tarde, aunque teníamos viento favorable, insistimos en acercarnos a la costa y hacer señales a nuestro compañero, pero en vano el bote costeó varias veces en una y otra dirección aquella tierra, pues nada vimos u oímos de él, por lo cual tuvimos que abandonar la búsqueda pensando que si al fin y al cabo él era el único en sufrir por lo sucedido la pérdida no resultaba tan grande.

    Con todo no me sentí satisfecho hasta no aventurarme una vez más para tratar de descubrir algún indicio de su suerte. Era ya la tercera noche a contar desde la batalla, y deseaba saber exactamente qué daño habíamos alcanzado a hacer entre los nativos y cuál era la situación de su bando. Por miedo a que nos atacaran, desembarcamos en la oscuridad; pero yo hubiera debido tener la seguridad de que los hombres que venían conmigo estaban dispuestos a obedecer mis órdenes en una emergencia tan aventurada y peligrosa, en la cual nos lanzábamos sin mayor conocimiento ni planes anticipados.

    Formamos el cuerpo de desembarco con veinte marineros decididos, aparte del sobrecargo y yo. Dos horas antes de medianoche tocamos tierra en el mismo sitio donde los indios se habían agrupado en batalla la noche de la alarma. Mi intención, al desembarcar en ese punto, era cerciorarme de si los nativos habían abandonado el lugar y si quedaban huellas del daño que pudiéramos haberles ocasionado. Tal vez, de hacer uno o dos prisioneros, pudiésemos luego canjearlos por nuestro marinero.

    Silenciosamente descendimos del bote, y dividí a los hombres en dos grupos tomando el comando de uno de ellos mientras el contramaestre dirigía el otro. No vimos moverse nada, y tampoco escuchamos el menor rumor mientras avanzábamos, un grupo algo separado del otro, hacia el campo de batalla. Tan oscura era la noche que repentinamente nuestro contramaestre tropezó con algo que resultó ser un cadáver, cayendo sobre él. Esto los hizo detenerse, comprendiendo que habían llegado al sitio donde estuvieran los nativos, y a la espera de que yo me les reuniese. Decidimos quedarnos allí hasta que saliera la luna, que no podía tardar más de una hora, y hacer entonces un reconocimiento de los estragos causados por la andanada. Contamos treinta y dos cuerpos yacentes, de los cuales dos conservaban aún vida. A algunos cadáveres les faltaba un brazo o una pierna, y los había también decapitados; en cuanto a los heridos imaginamos que se los habrían llevado los sobrevivientes.

    Cuando hubimos concluido lo que nos pareció un reconocimiento completo decidí que ya era hora de volvernos al barco, pero entonces el contramaestre y sus hombres me notificaron su intención de hacer una visita a la aldea de los indígenas, donde aquellos perros (según su expresión) habitaban. Me invitaron a que los acompañara asegurándome que si dábamos con el sitio, cosa que en su fantasía les parecía muy fácil, recogeríamos un gran botín y tal vez encontráramos a Thomas Jeffery, que tal era el nombre del extraviado marinero.

    Si aquellos hombres me hubieran solicitado autorización para ir a la aldea, mi negativa hubiera sido terminante como puede imaginarse, ya que les hubiera mandado volver al bote, sabedor de que no podíamos lanzarnos a semejante aventura siendo responsables de un navío que estaba a nuestro cargo y cuyo viaje dependía de la vida de aquellos hombres; pero como se limitaron a notificarme su intención de hacer el viaje, y me invitaron a acompañarlos, me rehusé terminantemente y levantándome del sitio donde estaba descansando me dispuse a volver al bote.

    Uno o dos de los hombres de mi grupo principiaron a importunarme para que fuésemos con los otros, y como yo repetí mi terminante negativa, se pusieron a refunfuñar, murmurando que ellos no estaban a mis órdenes y que irían si les daba la gana.

    — ¡Ven, Jack! —dijo uno—. ¿Quieres acompañarme? Me voy con ellos.

    Jack repuso que iría, otro lo imitó y luego otro más; en una palabra, todos me abandonaron menos uno, a quien pude persuadir de que se quedara a mi lado, así como un grumete que había permanecido en el bote. El sobrecargo y yo, acompañados del marinero fiel, volvimos entonces a la embarcación donde, según dije a los que se marchaban, nos quedaríamos a esperarlos para ver de salvar a los que volviesen con vida; les aseguré que emprendían una locura tan rematada que su destino sería probablemente el mismo que el de Thomas Jeffery.

    Como buenos marineros, me confesaron que estaban segurísimos de volver sin novedad, que se cuidarían mucho, etc. Y se fueron, pese a que los insté a que reflexionaran acerca del navío y el viaje a realizar, que sus vidas no les pertenecían sino que en cierta medida eran dependientes del destino del barco; agregué que si algo les pasaba el buque se perdería por falta de tripulantes y jamás podrían ellos rendir justa cuenta de su acción ante Dios ni ante los hombres. Agregué muchas otras cosas en este tono, pero lo mismo hubiera sido hablar al palo mayor del barco; estaban tan entusiasmados con su proyecto que se limitaron a pedirme que no me enfadara y que adoptarían las necesarias precauciones a fin de estar de vuelta antes de una hora.

    Según afirmaban, la aldea indígena no distaba más de una milla de la costa, aunque resultó más tarde que había por lo menos dos millas.

    La cosa es que se marcharon, y aunque su tentativa era una verdadera locura, preciso es reconocer en su homenaje que la emprendieron con tanta prudencia como valor. Iban muy bien armados, pues cada uno llevaba un fusil o mosquete, una bayoneta y una pistola. Algunos tenían anchos machetes; otros, sables, y el contramaestre, así como otros dos marineros, iban provistos de hachuelas de mano. Aparte de todo eso llevaban entre todos trece granadas de mano. Jamás partida más temeraria y mejor equipada se puso en marcha para intentar el más alevoso de los golpes.

    Al iniciar la expedición su motivo principal era el saqueo, pues todos abrigaban la esperanza de encontrar oro en la aldea. Una circunstancia inesperada, empero, los llenó de deseos de venganza y los convirtió en criaturas demoníacas. Al llegar a las escasas chozas indígenas que imaginaban era una aldea se sintieron grandemente decepcionados, pues eran sólo doce cabañas. En cuanto a la aldea en sí, desconocían su importancia y situación. Se consultaron sin que por largo rato pudiesen llegar a ponerse de acuerdo; si caían sobre aquellos moradores sería preciso degollarlos a todos y había diez probabilidades contra una de que aprovechando la oscuridad nocturna, y pese a que había luna, algún nativo se escapara llevando la alarma a la aldea, con lo cual se verían atacados por un ejército de indios. Por otra parte, si proseguían adelante dejando a los moradores que continuasen su tranquilo sueño, ¿cómo se las arreglarían para localizar la aldea?

    Con todo la segunda era la mejor idea, y decidieron adoptarla dispuestos a encontrar de un modo u otro la ubicación del pueblo. Avanzaron un poco y dieron con una vaca atada a un árbol, la que les pareció que podría resultar un guía excelente, por cuanto aquel animal pertenecía con seguridad a la aldea y si la desataban se encaminaría derechamente en dirección a ella. En caso de que la vaca volviese atrás la dejarían irse, pero si echaba a andar hacia adelante todo estaba en seguirla de cerca. Cortaron de inmediato la soga de lianas entretejidas y notaron que la vaca se ponía en marcha hacia adelante de tal manera que los guió directamente hasta la aldea que, según dijeron más tarde, contaba con más de doscientas chozas o cabañas, en algunas de la cuales vivían varias familias juntas.

    Encontraron todo en completa y silenciosa calma, tal como el sueño y la seguridad de que ningún enemigo los acechaba podían dar a aquellas gentes. Un segundo conciliábulo tuvo entonces lugar, resolviendo por fin dividirse en tres cuerpos que procederían a incendiar la aldea por tres lados a la vez, y así que los nativos saliesen huyendo los apresarían y atarían de inmediato; si alguno intentaba resistirse, no hace falta decir lo que le esperaba, tras lo cual se dedicarían en común a saquear las chozas que no se hubieran quemado.

    Antes de poner en práctica su plan resolvieron marchar silenciosamente a través de la aldea a fin de apreciar su tamaño e importancia, y decidir si podrían o no atreverse a intentar el asalto. Así lo hicieron, persistiendo en su primera idea, pero mientras se animaban mutuamente a la tarea tres de ellos que marchaban un poco adelantados los llamaron en alta voz diciéndoles que habían encontrado a Tom Jeffery. Corrieron al lugar, comprobando que en efecto el desdichado estaba allí, pues lo encontraron degollado y completamente desnudo, colgando de un árbol, al que lo habían suspendido por un brazo.

    Casi al lado del árbol había una cabaña indígena, donde estaban reunidos dieciséis o diecisiete de los principales nativos que se empeñaran en lucha con nosotros, así como dos o tres de los heridos por la andanada de metralla; los expedicionarios los oyeron hablarse unos a otros, prueba de que permanecían despiertos, pero no pudieron darse cuenta de su número.

    Tanto los enardeció la vista de su torturado camarada, que juraron allí mismo vengarlo sangrientamente, sin conceder cuartel a ninguno de los nativos que cayeran en sus manos. Se pusieron de inmediato a la tarea, aunque no tan alocadamente como su rabia y exaltación podría haber hecho pensar. Ante todo buscaron aquello que más fácilmente ardiera, pero después de inspeccionar la aldea notaron que no había necesidad de preocuparse por cuanto la mayoría de las cabañas eran bajas y estaban techadas con juncos o espadañas, que abundan en la región. Hicieron entonces una especie de mechas o pez griega como se las llama, humedeciendo un poco de pólvora en la palma de la mano, y un cuarto de hora más tarde el pueblo empezaba a arder por cuatro o cinco lados, y en especial aquella choza donde había un grupo de nativos despiertos. Tan pronto sintieron el calor de las llamas, locos de espanto, se lanzaron fuera para salvar sus vidas, pero su triste destino los aguardaba allí en la puerta donde los atacantes los hicieron retroceder. El contramaestre en persona mató a uno o dos con su hachuela, y como la choza era grande y él no quería entrar, pidió una granada de mano y la arrojó en el interior, cosa que al principio solamente asustó a los indígenas, pero al producirse la explosión hubo un horroroso estrago entre ellos.

    En resumen, la mayoría de los indios que se encontraban allí resultaron muertos o heridos por la granada, excepto dos o tres que se lanzaron a la puerta, donde los esperaban el contramaestre y dos hombres, armados de bayonetas, y sucumbieron allí mismo. Había sin embargo otra habitación en la choza donde el príncipe o rey se encontraba en compañía de otros nativos. Cercados totalmente, aquellos infelices se vieron precisados a quedarse en la choza que era ya una llama viva, hasta que el techo cayendo sobre el grupo los quemó o ahogó a todos.

    Antes de que esto sucediera, los atacantes no dispararon sus armas para no despertar a los demás, hasta tener la seguridad de que podrían dominarlos. Con todo el incendio había ya sembrado la alarma y entonces los nuestros consideraron conveniente mantenerse reunidos porque el fuego era tan pavoroso, ardían con tanta facilidad aquellas frágiles chozas combustibles, que apenas podían soportar quedarse en los espacios que había entre una y otra, a la vez que sus planes los obligaban a mantenerse al lado del incendio para matar a los nativos. Tan pronto como el fuego arrojaba de sus viviendas a los indígenas, los atacantes los esperaban en la puerta a fin de exterminarlos, gritándose unos a otros para darse ánimo que recordaran lo ocurrido a Thomas Jeffery.

    Mientras esto acontecía, yo estaba lleno de intranquilidad especialmente cuando distinguí las llamaradas que venían del pueblo, incendio que a causa de la oscuridad reinante me parecía estar más cercano de la costa y casi a mi lado.

    Mi sobrino, el capitán, también alarmado por aquel incendio y temiendo que algo grave me ocurriese, no sabía qué hacer ni qué decisión adoptar. Su alarma aumentó al oír disparos, pues en ese momento los nuestros empezaban a utilizar sus piezas. Mil pensamientos cruzaban por su mente referentes a mí y al sobrecargo, y se desesperaba pensando qué podría ser de nosotros. Por fin, aunque era una temeridad disponer de nuevos hombres, pero incapaz de quedarse ignorando qué nos ocurría en tierra, ordenó arriar otro bote y mandando en persona trece hombres se apresuró a venir en nuestro auxilio.

    Se sorprendió mucho al encontrarme en el bote acompañado del sobrecargo y los dos marineros, y aunque lo alegró sabernos sin novedad, su ansiedad por averiguar lo que ocurría más lejos se mantuvo invariable; por otra parte los disparos continuaban y el fuego iba en aumento, de modo que para cualquier hombre hubiera resultado imposible refrenar la curiosidad de todos por saber qué había ocurrido, así como su inquietud por la suerte de los compañeros.

    Tan poco me valió ahora amonestar a mi sobrino como anteriormente a los expedicionarios. Me dijo que quería ir allá, y que si algo lamentaba era haber dejado hombres a bordo ya que jamás consentiría que sus marineros muriesen por falta de auxilio; prefería, agregó, perder el barco, el viaje y la vida, y con esas palabras se puso en marcha.

    Naturalmente no me fue posible quedarme atrás ahora que estaba seguro de que no lo detendría en su propósito. El capitán ordenó a dos marineros que volviesen de inmediato al barco para buscar otros doce hombres y que lo hicieran en la pinaza mientras la chalupa permanecía anclada, de manera que a su regreso quedaran seis hombres en custodia de los botes a tiempo que los otros seis se unían a nosotros. Solamente dieciséis hombres restaban en el barco, ya que la tripulación entera se componía de sesenta y cinco hombres, de los cuales dos habían muerto en la pelea causante de todos estos males.

    Ya puestos en marcha es de imaginarse en el apuro que lo haríamos; el fuego nos guiaba, y no nos cuidamos de seguir un camino sino que marchamos en línea recta al lugar del incendio. Si el estampido de los disparos había empezado por sorprendernos, los alaridos de los infelices nativos nos helaron de espanto. Debo confesar que jamás he participado del saqueo de una ciudad, ni de su conquista por asalto. Había oído hablar de cómo Oliverio Cromwell tomó Drogheda, en Irlanda, pasando a degüello hombres, mujeres y niños; y también cómo el conde de Tilly, al saquear la ciudad de Magdeburgo, permitió asesinar a veintidós mil personas de ambos sexos; sin embargo, no tenía una idea clara de lo que podía ser aquello, y por eso me resulta imposible describir la horrible impresión que me causó escuchar aquellos clamores.

    Seguimos adelante sin embargo y pronto estuvimos ante la aldea, aunque el fuego tornaba imposible todo intento de penetración. Lo primero que vimos fueron las ruinas de una choza, o más bien sus cenizas, porque estaba enteramente consumida; delante de ellas, claramente visibles a la luz de las llamas, yacían los cadáveres de cuatro hombres y tres mujeres; nos pareció reconocer también uno o dos cuerpos confundidos con el fuego. Había allí las huellas de una venganza atrozmente bárbara, de una furia que iba más allá de lo humano, y hubo un momento en que creímos imposible que nuestros hombres hubiesen sido capaces de hacer una cosa así o pensamos que si verdaderamente eran los culpables todos ellos merecían ser sentenciados a muerte.

    Pero esto no era todo; el fuego crecía cada vez más y los gritos parecían aumentar en la misma proporción, de manera que estábamos totalmente confundidos y turbados. Avanzamos un poco y entonces, ante nuestra estupefacción, vimos tres mujeres, completamente desnudas y lanzando espantosos alaridos, que corrían como si tuviesen alas, y tras ellas dieciséis o diecisiete nativos, poseídos del mismo espanto y perseguidos por tres de los carniceros ingleses —ya que no puedo darles otro nombre— que, furiosos al advertir que no podían darles alcance dispararon sobre ellos alcanzando a matar a un nativo que cayó a poca distancia de nosotros. Cuando el resto se dio cuenta de nuestra presencia, considerándonos sus enemigos al igual que los otros y sin duda dispuestos a asesinarlos, lanzaron horrorosos alaridos, especialmente las mujeres, y dos de ellas cayeron como fulminadas por el terror.

    Mi alma desfalleció ante la contemplación de semejante escena, y creí que la sangre se me helaba en las venas. Pienso que si los tres marineros ingleses hubiesen continuado la persecución de los nativos y venido hacia nosotros, habría ordenado a los nuestros que los mataran. Hicimos lo posible porque aquellas pobres gentes comprendieran que no teníamos intenciones contra ellos, y entonces se acercaron con las manos alzadas, profiriendo enternecedores lamentos y pidiéndonos que les salváramos la vida, lo que prometimos al punto. De inmediato se fueron agrupando en montón, detrás nuestro, como si fuésemos una barricada de defensa. Hice que mis hombres formaran un pelotón, con orden de no herir a nadie, pero recomendándoles que tratasen de apoderarse de alguno de los nuestros y averiguaran qué espíritu diabólico se había posesionado de ellos y qué pretendían hacer. Hice que les transmitieran mi orden de suspender inmediatamente la masacre, asegurándoles que si persistían hasta que fuese de día, por lo menos cien mil nativos vendrían a atacarlos.

    Hecho esto me fui a ver a los fugitivos, acompañado solamente por dos hombres, y me encontré ante un espectáculo verdaderamente espantoso. Algunos nativos tenían los pies terriblemente quemados por correr sobre las brasas, otros en cambio mostraban quemaduras en las manos. Una mujer, que cayera en medio de las llamas, estaba casi quemada cuando alcanzó a salir de aquel infierno, y dos o tres de los nativos mostraban profundos tajos en las espaldas y los muslos, ocasionados por nuestros hombres, que los habían perseguido; por fin vi a uno atravesado de un balazo y que murió estando yo allí.

    Hubiera querido enterarme del motivo de todo aquello, pero me resultó imposible comprender una sola palabra de cuantas me decían, aunque por sus señales observé que la mayoría estaba tan ajena como yo a las causas de la matanza. Me sentí tan aterrado por ese vandálico asalto que no pude quedarme más tiempo allí sino que volviendo a mis hombres me dispuse a internarme en la aldea misma, costara lo que costase, y poner así término a lo que estaba ocurriendo. Tan pronto me reuní a ellos les participé mi resolución y les ordené que me siguieran, cuando justamente en ese momento aparecieron cuatro de los nuestros con el contramaestre a la cabeza, pisoteando los cuerpos que habían asesinado y totalmente cubiertos de sangre y de polvo. Parecían estar buscando más gente que masacrar cuando los nuestros les gritaron con todas sus fuerzas, y por fin los oyeron y vinieron hacia donde nosotros permanecíamos a la espera.

    Tan pronto el contramaestre nos reconoció, se puso a lanzar gritos de triunfo pensando sin duda que veníamos a plegarnos a sus designios. Sin darme tiempo a que le dijese una palabra, exclamó:

    — ¡Capitán, digno capitán, me alegro de que hayáis venido! ¡Aún no hemos exterminado más que a la mitad de esos salvajes perros del infierno! ¡Quiero matar tantos como pelos tenía en la cabeza el pobre Tom! ¡Hemos jurado no perdonar a ninguno, y creedme que extirparemos a la raza entera de la tierra!

    Y con esto quiso seguir corriendo, perdido el aliento y sin detenerse a escuchar una sola palabra de nuestra parte.

    Por fin, alzando la voz de manera que pudiese hacerlo callar, le grité:

    — ¡Bestia feroz! ¿Qué vais a hacer? ¡No he de permitir que se toque a una sola de estas criaturas bajo pena de muerte! ¡Por vuestra vida os aconsejo deteneros y volver aquí inmediatamente, o sois hombre muerto en este mismo instante!

    —Pero, señor —replicó él—, ¿es que no estáis enterado de lo que han hecho, para obrar en esa forma? Si necesitáis un motivo que justifique nuestra acción, venid conmigo.

    Y  me mostró el cadáver degollado que colgaba del árbol.

    Confieso que esto me indignó y que en otras circunstancias hubiera aceptado la venganza, pero pensé que habían ya llevado su rabia demasiado lejos, y recordé las palabras de Jacob a sus hijos Simeón y Leví: «Maldita sea su cólera porque ha sido feroz; y maldita su ira, porque ha sido cruel.»

    Con todo una nueva tarea me esperaba, porque cuando los hombres que venían conmigo vieron el cadáver me costó tanto trabajo contenerlos como si se tratara de los otros. Hasta mi sobrino se dejó arrastrar por la cólera, y manifestó en alta voz que su única preocupación era el temor de que los nativos terminaran por superar a los nuestros, porque en cuanto a los de la aldea no creía que uno solo mereciera salvarse, ya que se habían complacido en el asesinato de un desdichado y debían por lo tanto ser tenidos por criminales.

    Al oír estas palabras ocho de mis hombres capitaneados por el contramaestre se lanzaron a completar su sangrienta obra; comprendiendo entonces que estaba más allá de mis fuerzas el detenerlos, me alejé triste y pensativo porque no podía soportar el espectáculo y mucho menos los horribles alaridos que proferían los desventurados que caían en sus manos.

    Acompañado solamente por el sobrecargo y dos hombres volví adonde estaban los botes. Admito que fue una locura de mi parte aventurarme casi solo como si no existiera peligro, ya que era casi de día y la alarma se había propagado por toda la región. Más de cuarenta guerreros armados de lanzas y arcos se habían juntado en el caserío ya mencionado donde había unas doce o trece chozas; pero por suerte seguí un camino directo hacia el mar, evitando ir por aquel lado, y cuando llegamos a la costa era ya pleno día. Embarcándome volví a bordo sin perder tiempo, ordenando que la pinaza regresara para asistir a los nuestros en cuanto pudieran necesitarla.

    Había advertido al llegar a las embarcaciones que el incendio estaba casi extinguido, así como que los gritos habían cesado. Sin embargo, cuando media hora más tarde subí al buque escuché una descarga cerrada y pude ver una gruesa columna de humo. Según supe más tarde, esto ocurría cuando los nuestros cayeron sobre el grupo que se había juntado en el caserío, matando unos diecisiete nativos e incendiando todas las chozas, aunque sin tocar a las mujeres y los niños.

    Para ese entonces los marineros habían vuelto a la costa en la pinaza, justamente cuando los nuestros empezaban a reaparecer en la playa. Llegaron por grupos, sin constituir dos cuerpos regulares sino dispersos y moviéndose desordenadamente, con tanta displicencia que una pequeña fuerza enemiga hubiera podido arrollarlos en un instante.

    Sin embargo, el espanto por ellos despertado se había difundido en la comarca entera, y los nativos estaban a tal punto aterrados y confundidos que acaso cien de ellos hubiesen huido al ver a cinco de los nuestros. En toda la acción no hubo un solo nativo que efectuara una defensa apropiada; quedaron tan sorprendidos del repentino fuego y el sorpresivo ataque en plena oscuridad, que no habían sabido cómo oponerse al asalto, ni siquiera cómo escapar, pues si lo hacían en una dirección encontraban a un contingente y lo mismo si elegían otro camino, terminando por morir a manos de los atacantes.

    De los nuestros resultaron todos ilesos, salvo uno que se dislocó un pie y otro sufrió profundas quemaduras en una mano.

    Me sentía profundamente irritado con la tripulación, pero especialmente con mi sobrino, que había procedido en forma enteramente opuesta a su deber como capitán del barco, olvidando la responsabilidad que significaba aquel viaje y apresurándose a estimular más que a contener la rabia de su hombres en empresa tan cruel y sangrienta. Mi sobrino aceptó respetuosamente mis reproches, pero me dijo que al ver el cadáver del infeliz marinero asesinado de una manera tan inhumana había perdido el dominio de sí mismo y no fue ya dueño de sus pasiones. Admitió que en su carácter de capitán no hubiera debido dejarse arrastrar en esa forma, pero que al fin y al cabo era un hombre y sus arrebatos lo habían llevado más allá de lo razonable.






	\chapter{Traficantes y piratas}





    Por muy justa que la venganza pareciera a nuestra tripulación, yo seguí sintiéndome contrario a ella y sostuve muchas veces que Dios maldeciría nuestro viaje. Les dije que la sangre derramada los convertía en verdaderos asesinos, y que aunque los nativos habían matado a Tom Jeffery no era menos cierto que él había sido el provocador.

    La primera desgracia nos ocurrió en el Golfo Pérsico, cuando cinco de nuestros hombres que se habían aventurado a desembarcar en el lado árabe del golfo fueron rodeados por los naturales que asesinaron a unos y a otros los condujeron prisioneros en calidad de esclavos. El resto de los que iban con ellos no alcanzó a rescatarlos y tuvo el tiempo preciso para retornar al bote. Aquello me pareció un justo castigo del Cielo y no dejé de manifestarlo así a la tripulación, pero el contramaestre me dijo entonces muy exaltado que a su juicio yo me estaba excediendo en mis vituperios y censuras más allá de lo que podía justificarlas.

    Con todo, mis frecuentes amonestaciones a propósito de lo ocurrido tuvieron al fin peores consecuencias de las que yo había imaginado. El contramaestre, que había sido aquella vez el cabecilla, se me acercó con insolencia un día y me dijo que estaba cansado de que yo hiciera continuas referencias a lo sucedido, así como injustas acusaciones, y que en aquella oportunidad era yo quien me había conducido muy mal con la tripulación y con él en especial. Agregó que yo viajaba como pasajero del buque, sin ninguna autoridad a bordo, por lo cual no estaban obligados a soportar mis opiniones, pero que todos se sentían intranquilos pensando que acaso tuviera algún designio en contra de ellos, tal como denunciarlos cuando volviéramos a Inglaterra. Por lo tanto, si no me comprometía a modificar mi actitud así como a dejar de inmiscuirme en sus asuntos, él abandonaría el barco, ya que no se consideraba seguro navegando en aquel navío y en mi compañía.

    Escuché con paciencia aquel discurso y le dije que en verdad yo me había opuesto a la masacre de Madagascar, como continuaría llamándola siempre, y que había manifestado libremente mi pensamiento en cuanta oportunidad se presentó, aunque sin acusarlo a él más que a los restantes hombres. Por cierto que no tenía mando a bordo ya que jamás había pretendido hacer uso de alguna autoridad y solamente ejercitaba mi libertad de opinión y de palabra en aquellas cosas que a todos nos concernían. Agregué que mi interés especial en aquel viaje no era de su incumbencia, pero que siendo yo propietario de una parte considerable del cargamento ello me daba amplio derecho a hablar incluso con más fuerza de la empleada hasta entonces, cosa a la que él ni nadie podía oponerse. Como me exalté un poco al decirle estas cosas, y, el contramaestre me escuchó sin discutir mayormente, terminé por pensar que la cosa había concluido allí. Estábamos ya en la rada de Bengala y deseoso de conocer aquel lugar desembarqué en un bote acompañado del sobrecargo y dispuesto a distraerme. Por la tarde me preparaba a volver cuando vino uno de los marineros a decirme que no me tomase la molestia de ir hasta el bote por cuanto tenían órdenes de no llevarme otra vez a bordo. Es de imaginar la sorpresa que me produjo aquel insolente mensaje, y de inmediato pregunté al marinero quién le había ordenado que me transmitiese tal orden. Contestó que era el patrón de la chalupa, a lo cual le dije que tornara a manifestarle que me había repetido el mensaje y que yo no tenía ninguna respuesta que dar.

    Me apresuré a ir en busca del sobrecargo para narrarle lo ocurrido, agregando que preveía el estallido de un motín a bordo y por lo tanto le rogaba fuese de inmediato al barco en una canoa india para avisar con tiempo al capitán. Sin embargo, hubiera podido evitarme todo aquello, pues antes de decir una palabra al sobrecargo las cosas  habían ya sucedido  a bordo.

    En efecto, el contramaestre, el artillero, el carpintero y, en una palabra, todos los oficiales de inferior categoría, apenas me hube alejado en el bote subieron al alcázar y pidieron hablar con el capitán. Allí, tomando la palabra el contramaestre, que hablaba muy galanamente y que pronunció un largo discurso, le fue repetido al capitán todo lo que aquel hombre me dijera anteriormente, agregando que como yo había desembarcado pacíficamente la tripulación deseaba evitar violencias conmigo, pero que de no haberse producido tal circunstancia se hubieran visto obligados a desembarcarme por la fuerza.

    Agregaron que desde el momento que se habían enganchado en el navío para servir bajo su mando estaban bien dispuestos a hacerlo con lealtad, pero que si yo rehusaba dejar el barco o si él no me obligaba a hacerlo, entonces ellos lo abandonarían, negándose a proseguir el viaje bajo su mando. Y al pronunciar la palabra «todos» el contramaestre miró en dirección al palo mayor (lo cual probablemente era una señal convenida de antemano) y los marineros que allí estaban congregados gritaron a coro:

    — ¡Todos, sí, todos!

    Mi sobrino el capitán era hombre valeroso y de gran presencia de ánimo. Se sorprendió de pronto, como es natural pero les dijo calmosamente que debían darle tiempo a que reflexionara sobre la cuestión, aparte de que ninguna medida podía ser adoptada antes de que hubiese conferenciado conmigo. Empleó entonces diversos argumentos para tratar de convencerlos de lo irrazonable y lo injusto de su actitud, pero todo fue en vano; los amotinados pronunciaron el juramento, mientras se estrechaban las manos, de abandonar en conjunto el barco si no recibían la seguridad de que yo no volvería a embarcarme ni proseguiría con ellos aquel viaje.

    La situación era harto penosa para mi sobrino, que se sentía obligado hacia mí y no sabía cuál camino seguir. Se puso entonces a hablarles caballerosamente, diciéndoles que yo tenía importantes intereses en aquel barco y su travesía, por lo cual, en justicia, no podía expulsarme de algo que era mi propia casa. Dijo que hacer tal cosa hubiese equivalido a imitar la actitud del famoso pirata Kidd que, provocando un motín a bordo, abandonó al capitán del barco en una isla desierta y se escapó con el navío. Les dijo que si cambiaban de buque no podrían retornar nunca a Inglaterra sin que les costase cara su conducta, que el barco era mío y que a él le estaba vedado expulsarme de él. En suma, prefería perder la embarcación y malograr el viaje antes que cometer semejante falta, de manera que hiciesen lo que les viniera en gana. Les aseguró que iría a tierra para conferenciar conmigo, e invitó al contramaestre a que lo acompañara, diciendo que acaso las cosas pudieran arreglarse de esa manera.

    Rechazaron unánimemente la propuesta, repitiendo que no querían tener nada que ver conmigo en adelante, ya fuera a bordo o en tierra, y que si embarcaba nuevamente abandonarían de inmediato el navío.

    —Muy bien —declaró el capitán—. Si todos sois de esa opinión, iré a tierra para hablar con él.

    Y así lo hizo, llegando a mi lado poco después que el marinero me transmitiera el mensaje del patrón de la chalupa.

    Me produjo gran alegría ver llegar a mi sobrino, porque había tenido miedo que los tripulantes lo apresaran por violencia, se hicieran a la vela y escaparan con el barco. Allí, en tierra extraña, hubiese quedado yo abandonado y sin recursos, colocado en una situación todavía peor que cuando arribé solitario a la isla.

    Para mi gran satisfacción los amotinados no habían llegado a tales extremos, y cuando mi sobrino me repitió las palabras que venía de escuchar a bordo, la forma en que se habían juramentado, estrechándose las manos, para abandonar juntos el barco si yo intentaba retornar a él, me apresuré a decirle que no se preocupara por ello, pues estaba dispuesto a quedarme en tierra. Solamente le pedía que hiciera desembarcar mis efectos así como suficiente cantidad de dinero para que pudiese elegir un camino de vuelta a Inglaterra.

    Tal decisión fue motivo de gran pesar para mi sobrino, pero no había sin embargo otro remedio que adaptarse a las circunstancias. Volvió, pues, a su barco y satisfizo a los tripulantes diciéndoles que su tío se había inclinado ante las exigencias y acababa de pedir que le enviasen sus efectos personales. En pocas horas quedó resuelto todo, los marinos retornaron a sus deberes y yo me quedé solo y pensando qué camino tomaría ahora.

    Me hallé así en la más remota parte del mundo, como creo que muy bien puedo denominarla, ya que se encontraba a tres mil leguas marinas más lejos de Inglaterra que mi antigua isla. Cierto es que podía viajar por tierra desde allí cruzando el país del Gran Mongol hasta Surate, por mar a Basora y el Golfo Pérsico y luego siguiendo la ruta de las caravanas por el desierto de Arabia hasta Aleppo y Scanderoon (1) , desde donde podría trasladarme en barco a Italia, y luego otra vez por tierra a Francia. Todo esto, puesto el línea recta, alcanzaría el diámetro total de la Tierra y hasta creo que si se midiera lo sobrepasaría en mucho.

    Quedaba otro camino a tomar, cual era el de embarcarme en alguno de los navíos ingleses que llegarían a Bengala provenientes de Acheen —en la isla de Sumatra— y hacer el viaje directo a Inglaterra. Pero por desgracia había llegado hasta esas tierras sin tener relación alguna con la Compañía Inglesa de las Indias Orientales, de manera que sería muy difícil abandonarlas sin su permiso a menos de tener gran influencia ante los capitanes de aquellos barcos o los agentes de la Compañía, para todos los cuales era yo un extraño en absoluto.

    Tuve entonces el particular placer —para decirlo irónicamente— de observar la partida del navío que se marchaba sin mí. No creo que muchos hombres en mis circunstancias hayan recibido alguna vez un tratamiento parecido, excepto en el caso de aquellos piratas que se apoderan de un navío tras abandonar en tierra a los que se niegan a ser cómplices de sus maldades; y lo ocurrido me parecía un camino abierto para que llegasen a tales excesos.

    Mi sobrino me había dejado dos sirvientes o, más bien, un acompañante y un criado. El primero era el empleado del contador de a bordo, a quien comprometió para que se quedase a mi lado, y el otro era uno de sus criados. Busqué entonces un buen alojamiento en casa de una dama inglesa donde vivían varios comerciantes, de los cuales dos eran franceses, dos italianos, o más bien judíos, y un inglés. Fui muy bien tratado en esta hospedería, y para que no pudiera decirse que me había apresurado en mis medidas permanecí en ese lugar más de nueve meses considerando cuál camino seguiría y cómo debía arreglármelas. Era dueño de valiosas mercaderías inglesas, así como de cantidad considerable de dinero. Mi sobrino me había provisto de mil piezas de a ocho y una letra de crédito por mayor cantidad en caso de que me fuera necesario, a fin de que no me viese en apuros pasara lo que pasase. Dispuse rápidamente de las mercaderías, vendiéndolas con ventaja, y tal como lo había pensado desde un comienzo empleé el dinero en comprar algunos hermosos diamantes que, de todas las cosas, eran las más convenientes, ya que me permitían llevar conmigo mi entera fortuna.

    Después de larga permanencia, y diversos proyectos de viaje, ninguno de los cuales me agradó, el comerciante inglés que se alojaba en la misma casa y con el cual había contraído una sólida amistad vino a verme una mañana.

    —Compatriota —me dijo—, tengo un proyecto que participaros, el cual se adapta muy bien a mis deseos, y acaso pueda resultaros aceptable, una vez que lo hayáis considerado atentamente.

    »Nos encontramos aquí —agregó—, vos por accidente y yo por mi propia voluntad, en una parte del mundo muy alejada de nuestra patria; es, sin embargo, un país donde los que entienden de comercio e intercambio pueden ganar grandes sumas. Si agregáis mil libras a mil que tengo yo dispuestas, adquiriremos el primer barco que nos parezca conveniente, del cual vos seréis capitán y yo traficante, y nos iremos a hacer un viaje comercial a la China. ¿Qué ganamos, decidme, con quedarnos aquí? El mundo entero se mueve, girando incesantemente; todas las criaturas de Dios, tanto celestes como terrenas, son inteligentes y están atareadas; ¿por qué habríamos nosotros de quedarnos ociosos? El mundo es de los hombres y no de los zánganos. ¿Por qué sumarnos a estos últimos?

    Me agradó mucho su propuesta, especialmente al serme expresada con tanta buena voluntad y de la manera más amistosa. No diré que yo podía, por mi libertad e independencia, ser el hombre apropiado para abrazar una proposición comercial de esa naturaleza; por el contrario, el comercio no era en modo alguno mi elemento. Pero acaso pueda asegurar que si el comercio no lo era, en cambio sí la aventura y la vida errante, por lo cual toda oportunidad de ver alguna parte de la tierra aún no visitada anteriormente no podía dejar de surtir efecto en mi espíritu.

    Pasó un tiempo antes de que pudiéramos encontrar un buque que nos conviniera. Cuando dimos con él se presentó la dificultad de enganchar marineros ingleses, por lo menos en número suficiente para que mandaran al resto de la tripulación que reclutáramos. Después de mucho dimos con un segundo, un contramaestre y un artillero ingleses; también encontramos un carpintero holandés y tres marinos portugueses, pensando que bastarían para dirigir la restante tripulación, formada por marineros hindúes.

    Son tantos los viajeros que han escrito el relato de sus travesías que resultaría bien poco interesante que yo hiciera una detallada narración sobre los lugares donde tocamos puerto, así como de sus habitantes. Dejo esas cosas a otros y remito al lector a los diarios de viaje de algunos ingleses, muchos de los cuales han sido publicados y otros se anuncian frecuentemente. Me basta con decir que navegamos hasta Acheen, en la isla de Sumatra, y allí pusimos proa a Siam, donde hicimos intercambio de algunas mercancías por opio y arak; el primero de estos productos se cotizaba a alto precio entre los chinos y en aquel entonces era muy codiciado en el país. Nos trasladamos entonces a Suskan, luego de un largo viaje, que nos llevó ocho meses, y retornamos a Bengala sintiéndome yo muy satisfecho de la aventura.

    He notado con frecuencia que nuestros compatriotas ingleses admiran la forma en que los oficiales que la compañía manda a la India, así como los comerciantes que se radican en el país, logran amasar grandes fortunas y vuelven a veces a la patria con sesenta, setenta o cien mil libras esterlinas. Pero no es de asombrar semejante cosa, como habrá de verse en adelante, cuando se considera la cantidad de puertos y factorías donde existe para ellos el libre comercio, y mucho menos puede maravillarse quien sepa que en todos aquellos lugares donde entran barcos ingleses existe constante demanda por los artículos provenientes de cualquier nación, lo cual asegura apreciables ganancias en el trueque así como mercado seguro para los productos locales, que se venden muy bien en ultramar.

    En resumen, hicimos un excelente viaje y yo gané tanto dinero con mi primera aventura, así como tan amplia experiencia acerca de la manera de aumentarlo aún más, que si hubiese tenido veinte años menos habría sentido la tentación de radicarme allí sin buscar más lejos los medios de hacer fortuna. ¿Pero qué podía significar esto para un hombre de más de sesenta años, suficientemente rico y que se había lanzado a viajar más por el insalvable deseo de ver mundo que por la ambición de acrecentar sus bienes? Pienso que es con toda verdad que llamo insaciable a mi afán de viajes, pues en verdad lo era. Cuando vivía en mi hogar me dominaban las ansias de ver el mundo; y ahora que me encontraba tan lejos soñaba con volver a mi patria. Lo repito: ¿qué podía importarme aquel comercio?

    Naturalmente, mi amigo el traficante pensaba de muy distinta manera. No lo digo con el propósito de poner en evidencia mi punto de vista, por cuanto reconozco que el suyo era el más justo y el más natural en la vida de un comerciante que, lanzado a remotas tierras, debe consagrar sus afanes a aquello que le produzca la mayor cantidad de dinero. Mi nuevo amigo se atenía a lo práctico del viaje y se hubiera contentado, como un caballo de posta, con acudir siempre a la misma posada una y otra vez, suponiendo que en ella encontrase el provecho que buscaba. Yo en cambio tenía de los viajes la idea que se forma un muchacho alocado y aventurero, que no siente el menor interés por ver dos veces una misma cosa.

    Pero no era esto todo. Aunque mis deseos de volver al hogar crecían cada vez más, no podía decidirme acerca del mejor camino a tomar para la vuelta. En los intervalos de esos conciliábulos, mi amigo, que estaba siempre a la caza de negocios, me propuso un viaje a las Islas de las Especias para traer un cargamento de clavos de olor de Manila o de aquellas regiones. Sabía yo que los holandeses trafican en aquellas islas aunque la posesión corresponde parcialmente a los españoles; no nos era preciso aventurarnos tan lejos sino hasta las tierras donde aquéllos carecen del poder que tienen en Batavia, Ceilán y otros sitios.

    Los preparativos se completaron rápidamente, aunque la mayor dificultad fue convencerme para que emprendiera el viaje. Con todo, como nada se ofrecía y yo hallaba agrado en navegar y dedicarme a un tráfico que producía tan pingues como fáciles ganancias, decidí embarcarme en la travesía en lugar de estarme sin hacer nada, lo que, dado mi carácter, era para mí una verdadera desgracia. Tocamos Borneo así como muchas otras islas cuyos nombres he olvidado, y cinco meses después nos hallábamos de regreso. Vendimos las especias, principalmente clavo y algo de nuez moscada, a los mercaderes persas que se ocupaban en distribuirlas en el golfo; y como obtuvimos ganancias en proporción de cinco a uno es de imaginar el excelente beneficio de la operación.

    Mientras liquidábamos muestras cuentas, mi amigo me miró sonriendo.

    —Y bien —dijo como tratando de provocarme en lo que yo tenía de indolente—. ¿No es esto mejor que andar dando vueltas como un hombre ocioso y perder el tiempo contemplando la ignorancia y los errores de los paganos?

    —Por cierto que sí, amigo mío —repuse—, y empiezo a sentirme aficionado al comercio. Pero dejadme que os diga de paso una cosa, y es que no sabéis de lo que yo soy capaz. Si alguna vez venciera mi indolencia y me lanzara a correr por el mundo, pese a lo viejo que estoy os obligaría a moveros a la par mía hasta agotaros; creedme que si me decidiese a hacerlo no os daría un momento de reposo.

    Pero abreviemos estas consideraciones. Poco después entró en el puerto un barco holandés procedente de Batavia. Era un pequeño navío de cabotaje, no un mercante europeo, de unas doscientas toneladas; la tripulación declaró haber pasado tantas enfermedades en el viaje que el capitán no tenía bastantes hombres para hacerse de nuevo a la mar. Permanecieron en Bengala y sea porque hubiese tenido ya suficiente ganancias o deseara por algún motivo volver prontamente a Europa, el capitán hizo anunciar que el barco estaba en venta. La noticia vino a mí antes que a mi socio, y sentí en seguida deseos de comprar aquel buque, de manera que de inmediato marché a hablar con él.

    Lo pensó un rato, porque no era hombre precipitado, y después de meditarlo bien, me dijo:

    —Es un barco un poco grande, pero lo compraremos lo mismo.

    Así fue, y después de habernos puesto de acuerdo con el capitán tomamos posesión del navío. Hecho esto pensamos en conservar los tripulantes a fin de agregarlos a los que ya teníamos bajo contrato y salir inmediatamente de viaje; desgraciadamente, como les había sido entregada su participación en las ganancias se dispersaron rápidamente y no pudimos encontrar a ninguno. Averiguamos dónde podían estar y por fin llegamos a saber que se habían ido por tierra a Agrá, la ciudad que es residencia del Gran Mogol y que desde allí pensaban viajar hasta Surate por agua al Golfo Pérsico.

    Pocas cosas pudieron dolerme más que el haber perdido así la oportunidad de irme con ellos. Semejante expedición, y en esa compañía, hubiera sido a la par que segura sumamente entretenida, adaptándose perfectamente a mis dos mayores deseos: ver mundo y volver a mi patria. Sin embargo, días más tarde me sirvió de consuelo saber qué clase de gentes eran aquéllas. Parece que el hombre a quien llamaban capitán era solamente el artillero; mientras traficaban, al desembarcar en alguna isla habían sido atacados por los malayos, que mataron al capitán y a tres hombres por lo cual los sobrevivientes, once en total, habían decidido escaparse con el buque, lo que efectivamente hicieron navegando en él hasta la bahía de Bengala no sin antes abandonar al piloto y a otros cinco hombres en tierra, de todo lo cual se sabrá más a su debido tiempo.

    En fin, cualquiera sea la forma por la cual llegaron a adueñarse del buque, lo cierto es que nosotros lo adquirimos con toda honestidad y nuestro único descuido fue no examinar las cosas más de cerca. No se nos ocurrió interrogar a los marineros, cosa que de haberse llevado a cabo hubiera revelado seguramente alguna contradicción en sus relatos y nos hubiera hecho entrar en sospechas acerca de su verdadera situación. El pretendido capitán nos había mostrado una escritura de venta del navío a nombre de un tal Emmanuel Clostershoven, o cosa parecida, todo lo cual era probablemente una falsificación. El mismo se hacía llamar por dicho nombre, y nada nos inducía a dudar de él, de manera que procediendo tal vez con demasiada ligereza y sin la más mínima sospecha de la verdad cerramos trato como he contado.

    Contratamos aquí y allá algunos marineros ingleses y holandeses, decidiéndonos a efectuar el segundo viaje al sudeste en busca de clavo y otras especias, es decir, a las Filipinas y Molucas. Como no quiero llenar esta parte de mi relato con bagatelas cuando lo que hace falta por contar es tan notable, diré solamente que transcurrieron seis años en continuos viajes de puerto en puerto, traficando siempre con excelentes resultados; transcurría ahora el último año en que yo viajaba junto a mi socio, y nos trasladábamos en el buque mencionado hacia la China, con previa escala en Siam para comprar arroz.

    Los vientos contrarios nos obligaron a hacer muchos rodeos en los estrechos de Malaca y entre sus islas, y apenas habíamos conseguido escapar de tan peligrosas aguas cuando descubrimos que el buque hacía agua sin que nos fuera posible, a pesar de todos nuestros empeños, descubrir dónde estaba abierta la vía. Era necesario refugiarse lo antes posible en un punto, y mi socio, que conocía aquellas zonas mejor que yo, ordenó al capitán que pusiera proa hacia el río de Cambodge. Debo hacer notar de paso que yo había designado capitán al piloto inglés, un tal mister Thompson, ya que no me agradaba serlo en persona de un segundo navío. En cuanto al río, se encuentra en el lado norte del gran golfo o bahía que conduce hasta Siam.

    Mientras nos hallábamos allí, bajando con frecuencia a tierra para avituallarnos, se me acercó cierto día un inglés. Era, a lo que parecía, segundo artillero a bordo de un buque de la Compañía Inglesa de las Indias Orientales que anclava en el mismo río y cerca de la ciudad de Cambodge. Ignoraba yo qué podía traer ese hombre cerca de mí; el hecho es que se acercó y me habló en nuestro idioma.

    —Señor —dijo—, sois un extraño para mí como yo para vos, pero tengo algo que deciros que os concierne muy de cerca.

    Lo miré fijamente un momento, pensando que acaso lo conocía, pero no era así.

    —Si se trata de algo que me concierne —dije—, sin que por lo visto nada tenga que ver con vos, ¿qué os mueve a decírmelo?

    —Me mueve —respuso el hombre— un inminente peligro al que estáis sujeto y del cual, por lo que veo, no tenéis conocimiento alguno.

    —Ignoro ese peligro —observé—, salvo que mi navío tiene un rumbo de agua que no puedo descubrir. Mi intención es hacerlo entrar mañana en dique seco para examinar el casco.

    —Pues bien, señor —dijo aquel hombre—. Con rumbo o sin él, lo encontréis o no, espero que no cometáis el desatino de varar vuestro buque después de lo que voy a deciros. ¿Es que ignoráis, señor, que el pueblo de Cambodge se halla quince leguas río arriba? Hay allí dos grandes navíos ingleses, cinco leguas hacia este lado, así como tres holandeses.

    — ¡Bueno! ¿Y qué me importa a mí eso?

    — ¡Cómo, señor! —exclamó él—. ¿Es que un hombre lanzado a una aventura como la vuestra entra en un puerto sin averiguar primero qué barcos hay allí y si está en condiciones de resistirlos? ¿O por ventura creéis que la batalla puede favoreceros?

    Al oír estas palabras, que más me divirtieron que asustaron, por cuanto no alcanzaba a entender su significado preciso, me volví a él diciéndole:

    —Caballero, os pido que os expliquéis. No puedo imaginar qué razón puede haber para sentir miedo de cualquier compañía naviera o de barcos holandeses. Navego con mi licencia en regla, de manera que nada pueden echarme en cara.

    Me miró con el aire de un hombre a la vez complacido y enojado. Sonriendo, y después de una pausa, me dijo:

    —Muy bien, señor, si os consideráis seguro, arriesgaos entonces. Lamento que vuestro destino parezca cegaros ante el buen consejo, pero tened por seguro que si no os hacéis inmediatamente a la mar en la próxima marea seréis atacado por cinco chalupas llenas de hombres armados. Si sois apresados, os colgarán como piratas y recién más tarde se examinarán vuestros descargos. Yo pensé, señor —agregó—, haber tenido mejor acogida a cambio de un servicio de tal importancia como el que os he hecho.

    —Jamás sería yo ingrato —repliqué— ante un servicio cualquiera o ante un hombre que se muestre cordial conmigo. Sin embargo, excede mi comprensión el porqué de ese intento contra mi persona. En fin, puesto que afirmáis que no hay tiempo que perder y que se trama contra mí alguna perfidia, me embarcaré inmediatamente y nos haremos a la mar siempre que mis hombres consigan tapar la vía de agua o de lo contrario estemos dispuestos a nadar. Pero, señor —agregué—, ¿es que tendré que marcharme ignorando la causa de lo que sucede? ¿No podéis vos aclararme esas tinieblas?

    —Lo que puedo es relataros una parte de esa historia, señor —repuso—, y un marinero holandés que está aquí conmigo tal vez sea capaz de completarla más tarde aunque apenas haya tiempo para ello. En resumen, he aquí lo que sucede, y vos debéis conocer muy bien la primera parte o sea que estabais con ese buque en Sumatra y que allí fue muerto por los malayos vuestro capitán con tres hombres. Los que quedasteis a bordo, o alguno de ellos, decidieron apoderarse del buque y huir con él. Tal es la historia y os puedo asegurar que si os capturan os ahorcarán como piratas, porque es sabido que los barcos mercantes no pierden el tiempo en escrúpulos legales cuando apresan a piratas.

    —Ahora sí habláis en inglés llano —dije— y os lo agradezco. En verdad nada sé de cuanto me contáis porque el barco ha venido a nuestras manos de la manera más honesta y legítima, y sin embargo, la situación que me habéis descrito con tanta franqueza me obliga a defenderme y estar en guardia.

    —No, señor —replicó entonces el hombre—. No digáis estar en guardia, ya que la mejor defensa es ponerse fuera de peligro. Si en algo estimáis vuestra vida y la de vuestros hombres, haceos a la vela apenas venga la pleamar. Tenéis toda una marea de ventaja y estaréis ya demasiado lejos antes de que ellos lleguen aquí, pues piensan aprovechar la pleamar y mientras recorren las veinte millas que los separan de este puerto vos ganaréis cerca de dos horas sobre ellos por la diferencia de marea, sin contar la ventaja de la distancia. Además vendrán en chalupas y seguramente no se atreverán a seguiros por el mar, sobre todo si sopla algo de viento.

    — ¡Pues bien! —dije yo—. Nos habéis prestado un gran servicio, y quisiera saber en cuánto estimáis vuestra recompensa.

    —Señor —me respondió—, no creo que pueda hablaros de recompensa alguna por cuanto aún no estáis bien convencido de la verdad que os he hecho saber. Con todo os haré una oferta. Se me deben diecinueve meses de paga a bordo del «...», en que he venido de Inglaterra, y siete meses al holandés que me acompaña; si nos garantizáis esos sueldos, nos iremos con vos, y si nada ocurre de lo que os he prevenido, nada pediremos entonces, pero si os convencéis de que nos debéis la vida, así como las de quienes están en vuestro navío, entonces os dejaremos fijar personalmente la recompensa.

    Acepté de inmediato la propuesta, y acompañado de los dos hombres me trasladé al punto a bordo. Cuando llegué junto al navío, mi socio, que estaba en el puente, se inclinó desde el alcázar y me gritó con grandes muestras de alegría:

    — ¡Hola, hola! ¡Hemos cerrado el rumbo! ¡Hemos cerrado el rumbo!

    — ¿De veras? —exclamé—. ¡Alabado sea Dios! ¡A levar anclas ahora mismo!— ¿Levar anclas?  ¿Qué queréis decir?  ¿Os ocurre algo?

    —Nada de preguntas —dije—. Manos a la obra todo el mundo y a levar anclas sin perder un minuto. Lleno de sorpresa, mi socio llamó al capitán, quien dio las órdenes necesarias, y aunque la marea no había alcanzado aún la pleamar, aprovechando una suave brisa nos apresuramos a alejarnos. Me reuní entonces con mi socio en la cámara y le referí todo lo sucedido, tras lo cual hicimos venir a los dos hombres que nos repitieron la misma cosa.

    Esto nos llevó bastante tiempo, y antes de que hubiéramos terminado vino un marinero a la cámara diciendo que el capitán lo enviaba a comunicarnos que éramos perseguidos.

    — ¡Perseguidos! —exclamé—. ¿Y por quiénes? —Por cinco balandros o chalupas —repuso el marinero— cargados de hombres.

    — ¡Perfectamente! —dije—. Entonces había algo de cierto en el relato.

    Ordené reunir a la tripulación y le manifesté que existía el plan de apoderarse de nuestro barco y apresarnos como piratas, por lo cual quería saber si permanecían de nuestro lado para defender el buque. Llenos de entusiasmo me respondieron a coro que estaban dispuestos a morir por nosotros.

    Interrogué al capitán sobre la mejor manera de enfrentar al enemigo, ya que estaba dispuesto a resistir hasta el último momento. Me manifestó que a su parecer convenía mantenerlos a distancia todo el tiempo posible con el tiro de nuestros cañones, y si lograban acercarse entonces hacerles fuego con las piezas pequeñas; por fin, si aún persistían en el ataque, nos refugiaríamos debajo del puente, que acaso no pudieran forzar por falta de material adecuado.

    Mientras tanto, el artillero había recibido órdenes de asestar dos cañones, uno a proa y otro a popa, desocupar el puente y poner como carga balas de mosquete, metralla y cuanta pieza de hierro hubiese a mano, alistándonos así para el combate. Seguíamos rumbo a alta mar aprovechando el viento favorable y podíamos ver a la distancia las chalupas, cinco embarcaciones grandes que nos perseguían con todo el trapo que habían podido desplegar.

    Dos de las chalupas, que con ayuda del catalejo pudimos ver que eran inglesas, sobrepasaron a las otras por más de dos leguas de ventaja, adelantándose tanto que no tardaron en acercarse a nosotros. Cuando comprendimos que había llegado el momento, disparamos un cañonazo sin metralla para intimarles que se pusieran al pairo, y alzamos la bandera de paz en señal de que deseábamos parlamentar. Siguieron sin embargo la persecución hasta que estuvieron a tiro y entonces quitamos el pabellón blanco del cual no habían hecho caso alguno, y reemplazándolo por otro rojo, descargamos un cañonazo. Siguieron acercándose pese a todo, hasta hallarse a tan poca distancia que podíamos hacernos entender por medio de una bocina, y entonces les gritamos que se alejaran o los hundiríamos.

    Fue en vano; arreciando en la persecución intentaron guarecerse a nuestra popa con probable intención de lanzarse al abordaje por el alcázar. Viendo que estaban resueltos a pelear llenos de confianza por los refuerzos que venían de más lejos, ordené virar a fin de presentarles un lado y disparamos una andanada de cinco cañonazos, uno de los cuales había sido tan bien asestado que se llevó íntegramente la popa de la chalupa que venía segunda, obligando a sus hombres a recoger las velas y correr a la proa para impedir que se hundiera. Allí quedó, habiendo recibido el suficiente castigo, pero como la otra chalupa se empeñaba en perseguirnos le apuntamos con todos los cañones sin perder tiempo.

    Mientras esto ocurría, una de las chalupas retrasadas, adelantándose a sus compañeros, llegó hasta la embarcación desmantelada para ayudarla, y pudimos ver que salvaba la tripulación. Gritamos nuevamente a los de la primera chalupa, ofreciéndoles parlamentar y saber qué querían de nosotros, pero no obtuvimos respuesta a tiempo que la embarcación se acercaba aún más a nuestra popa. Al ver esto, el artillero, que era hombre sumamente diestro, juntó sus dos cañones móviles en la popa y tiró simultáneamente, pero erró el tiro y los de la chalupa lanzaron gritos de triunfo a la vez que agitaban sus gorras y se acercaban más y más al navío. Preparándose con toda rapidez, el artillero disparó nuevamente y uno de los tiros, aunque erró a la chalupa misma cayó entre los hombres y pudimos comprobar que había causado grandes estragos entre ellos. Sin preocuparnos por ello, viramos de bordo y al presentarles la aleta de proa descargamos tres cañonazos que partieron la chalupa en pedazos, en especial el timón y una parte de la popa, que saltaron lejos. Aquello produjo gran desorden entre los enemigos, quienes se apresuraron a aferrar las velas. Para completar su desgracia nuestro cañonero disparó otros dos cañonazos sobre ellos y aunque no puedo asegurar dónde alcanzaron a la embarcación vimos que se hundía rápidamente y que algunos hombres estaban ya nadando en su entorno. Al ver esto ordené arriar la pinaza que habíamos tenido lista a tal fin, con orden de recoger a quienes fuera posible y retornar de inmediato a bordo, pues entretanto las otras chalupas seguían acercándose. Nuestros marineros cumplieron la orden y alcanzaron a salvar a tres hombres, uno de los cuales estaba ya medio ahogado y tardó mucho en volver en sí. Tan pronto los izamos a bordo nos hicimos mar afuera con el velamen desplegado, siendo visible que las otras chalupas, luego de acercarse al lugar de combate, abandonaban la persecución.

    Nos vimos así desembarazados de un peligro que, aun sin conocer la causa del mismo, parecía haber sido más grande de lo que yo creyera al comienzo. Me apresuré a ordenar que cambiáramos el rumbo a fin de que nadie pudiese sospechar la dirección que tomábamos. Navegamos hacia el oriente, muy lejos de las rutas de los barcos europeos, tanto de los que viajaban hacia China como a cualquier otro punto de las zonas comerciales europeas.






	\chapter{El barco fatídico}





    Cuando estuvimos en alta mar volvimos a interrogar a los dos marineros, preguntándoles cuál podía ser el significado de lo acontecido. El holandés nos reveló al punto el secreto, diciéndonos que el individuo que nos vendiera el barco no era más que un ladrón y que se había apoderado del navío como un pirata, huyendo con él. Nos contó entonces cómo el capitán, cuyo nombre no recuerdo, fue traidoramente asesinado por los nativos de la costa de Malaca, junto con tres de los suyos; el holandés y otros cuatro marineros se internaron en los bosques, donde anduvieron largo tiempo errantes, hasta que de una manera milagrosa pudo él escapar de allí y nadar hacia un navío holandés que, pasando cerca de la costa en su viaje a China, acababa de enviar un bote a la costa en procura de agua dulce. Nos dijo que no se había atrevido a salir a aquella parte de la costa donde había varado la chalupa, sino que esperó a que fuera de noche para lanzarse al agua un poco más lejos nadando hasta la embarcación, donde fue por fin recogido y enviado al barco.

    Más tarde, el marinero llegó a Batavia donde encontró a dos tripulantes de su barco, que habiendo abandonado al resto acababan de llegar. Le contaron que los otros, luego de fugarse a Bengala, habían vendido el barco a unos piratas que ahora se dedicaban con él a abordar buques mercantes, habiendo conseguido apoderarse de un navío inglés y dos holandeses, todos ellos cargados con muchas riquezas.

    Esta última parte del relato nos concernía directamente, aunque sabíamos bien que era falsa; como muy acertadamente opinó mi socio, de haber caído en manos de nuestros perseguidores y con semejante acusación contra nosotros hubiera sido vano tratar de defenderse o esperar recibir cuartel de su parte. Nuestros acusadores hubiesen sido jueces al mismo tiempo, y de ellos no hubiésemos podido esperar más que lo que la rabia les dictara y el apasionamiento irracionable pusiera en ejecución. Mi socio agregó que le parecía conveniente volver a Bengala sin hacer ninguna escala intermedia, y una vez allí probar claramente nuestra situación, dónde nos encontrábamos cuando el navío arribó a puerto, a quién lo compramos, así como otros detalles. Y, lo que contaba todavía más, si la necesidad nos llevaba a enfrentarnos con verdaderos jueces, tendríamos la seguridad de recibir el tratamiento adecuado y no  ser  primero  ahorcados para que nos juzguen después.

    Al principio estuve de acuerdo con mi socio, pero después de meditarlo un tiempo le dije que me parecía excesivamente arriesgado volvernos a Bengala,  por cuanto estábamos al otro lado del estrecho de Malaca, y si la alarma cundía era seguro que se pondrían al acecho en ambas salidas del mismo, tanto los holandeses de Batavia como los ingleses de aquellos lados; si llegaban a apresarnos después de haber huido de ellos, no sería necesaria otra evidencia para que nos condenasen de inmediato y encontraríamos el peor de los fines. Pedí opinión al marinero inglés, quien me dijo que mis palabras eran justas y que seguramente seríamos apresados en el estrecho.

    Todo esto me obligaba a pensar ansiosamente en la manera de huir, aunque no alcanzaba a descubrir el camino apropiado ni algún lugar donde refugiarnos. Viéndome tan desanimado, mi socio, que al comienzo había estado más inquieto que yo, quiso darme ánimos luego de describirme los distintos puertos que había en aquellas costas me dijo que a su juicio convenía poner proa hacia la Cochinchina o la bahía de Tonkín; de allí sería fácil trasladarnos a Macao, ciudad antaño en poder de los portugueses y donde aún residían muchas familias de origen europeo, agregando que los frailes misioneros acostumbraban dirigirse allá antes de seguir su ruta hacia la China.

    Resolvimos, pues, tomar ese rumbo, y luego de un accidentado y tedioso viaje en el cual la falta de provisiones nos hizo pasar grandes penurias alcanzamos una mañana a divisar la costa. Pensando en las circunstancias por las cuales acabábamos de pasar, y el peligro del cual habíamos escapado, quedó resuelto echar el ancla en la boca de un riacho que sin embargo contaba con suficiente profundidad, y luego, ya sea viajando por tierra o en la pinaza del barco, averiguar si en el puerto había algún navío del cual pudiésemos temer algo. Esta medida fue nuestra salvación, pues aunque en la bahía de Tonkín no encontramos en ese momento ningún barco, a la mañana siguiente entraron en el puerto dos navíos holandeses; y un tercero que no enarbolaba pabellón alguno, pero que supusimos holandés, pasó a unas dos leguas de la costa, rumbo a la China. Por la tarde aparecieron dos navíos ingleses que seguían la misma ruta y así llegamos a sentirnos rodeados de enemigos en todas direcciones.

    Nos hallábamos en un sitio salvaje y bárbaro cuyos habitantes eran todos ladrones por hábito y profesión; cierto que no teníamos mayor necesidad de entrar en tratos con ellos como no fuera para procurarnos algunas provisiones, pero asimismo nos vimos varias veces en dificultades para evitar incidencias.

    Ya fue dicho antes que nuestro barco tenía una vía de agua que no habíamos logrado localizar hasta que, inesperadamente, pudo ser tapada en el preciso momento en que estábamos a punto de ser apresados por los navíos ingleses y holandeses que había en Siam. Con todo, notando que el barco no se encontraba en condiciones necesarias para reanudar la navegación, resolvimos vararlo aprovechando nuestra permanencia en este lugar, y luego de alijarlo del escaso cargamento que llevábamos hacer una detenida inspección al casco para descubrir los rumbos.

    Habiendo aligerado el navío y puesto los cañones y demás cosas transportables a un lado, tratamos de tumbarlo sobre la playa para que nos fuera posible ver el casco, pero luego de pensarlo mejor renunciamos al proyecto por cuanto no encontramos ningún sitio de la costa que se prestara a ejecutarlo.

    Los habitantes de esa región, que jamás habían presenciado un espectáculo parecido, se acercaron llenos de asombro a la playa y viendo al barco tan inclinado y casi tumbándose sobre la playa (sin advertir a nuestros hombres que estaban trabajando en el casco por el lado de afuera, embarcados en los botes y colgados de un andamio) dedujeron que sin duda el barco había sido arrojado y varado allí por alguna tormenta.

    Apoyados en esa suposición, dos o tres horas más tarde, se presentaron en unas doce lanchas, algunas de las cuales contenían diez tripulantes, con evidente intención de subir al barco y saquearlo; en caso de encontrar a alguno de nosotros a bordo, probablemente pensaban conducirnos en calidad de esclavos ante su rey o como le llamaran, pues ignorábamos qué clase de gobernante tenían.

    Tan pronto llegaron junto al barco y empezaron a dar vueltas alrededor, nos descubrieron trabajando activamente por la parte exterior del casco, rascándolo unos, mientras otros lo embreaban y calafateaban al modo que todo marino sabe hacerlo.

    Se quedaron quietos, observándonos un rato, y no alcanzamos a comprender qué intenciones traían, aunque nos alarmaron un poco. Para no ser tomados de sorpresa buscamos la manera de que algunos hombres entraran en el barco y alcanzaron armas y municiones a aquellos que seguían trabajando por la parte de afuera, a fin de que si se presentaba motivo, pudieran contar con qué defenderse. Nunca hubo medida más acertada, porque un cuarto de hora más tarde, después de consultarse entre ellos, los naturales llegaron a la conclusión de que efectivamente se trataba de un naufragio y que estábamos tratando de poner el barco a flote o bien nos disponíamos a salvarnos en nuestras chalupas. Cuando vieron que trasladábamos las armas en los botes imaginaron que pretendíamos salvar parte del cargamento, y como al mismo tiempo se hicieron a la idea de que todo aquello era ya de su pertenencia —incluso nuestras personas—, se precipitaron en orden de batalla contra nuestros hombres.

    Los marineros, un poco asustados al ver el crecido número de sus oponentes y la mala posición en que estaban para pelear, lanzaron gritos preguntándonos qué debíamos hacer. Ordené entonces a los que estaban trabajando suspendidos en los andamios que dejasen caer las tablas y treparan a bordo sin perder tiempo, a la vez que mandaba a los que estaban en los botes que dieran la vuelta para subir a su vez por el lado más bajo. Los pocos que permanecíamos a bordo nos pusimos con todas nuestras fuerzas a enderezar el navío, pero pronto vimos que ni los marineros del andamio ni los de las chalupas podían cumplir mis órdenes por cuanto los cochinchinos se les arrojaron encima y mientras dos de sus embarcaciones cercaban nuestras chalupas sus tripulantes procedían a tomar prisioneros a los nuestros.

    Al primero que echaron mano fue a un marino inglés, muchacho robusto y decidido, quien dueño de un mosquete prefirió tirarlo al fondo de la barca en lugar de disparar con él, lo cual me pareció una completa locura. Sin embargo, sabía lo que hacía mejor de lo que yo hubiese podido enseñarle, porque con sus manos libres aferró al pagano que pretendía apresarlo y lo hizo pasar de su bote al nuestro, donde, sujetándole por las orejas, le golpeó con tal fuerza la cabeza contra los caperoles que lo mató instantáneamente. Entretanto un holandés que estaba a su lado levantó el mosquete y empleando la culata hizo tales molinetes con ella que derribó a cinco enemigos que pretendían asaltar el bote. Cierto que de poco servía esto contra treinta o cuarenta individuos que ignorando todo temor porno darse cuenta del peligro que corrían empezaban a asaltar la chalupa, donde sólo había cinco hombres para defenderla. Un episodio que nos hizo reír mucho terminó sin embargo dando una completa victoria a nuestros hombres.

    Nuestro carpintero había estado preparándose para embrear el casco del buque así como las costuras en toda la porción calafateada para impedir el rumbo del agua, y tenía en el bote dos calderas, una con pez hirviendo y la otra conteniendo una mezcla de resina, sebo, aceite y demás ingredientes que para tales trabajos se emplean. Por su parte, el ayudante del carpintero empuñaba un gran cucharón de hierro con el cual iba alcanzando a los que trabajaban en el andamio el caliente líquido. Dos de los enemigos saltaron al bote justamente en donde estaba aquel hombre, o sea en la escotilla de proa, y él los recibió con una cucharada de pez hirviente a manera de saludo, quemándoles de tal manera, por cuanto estaban medio desnudos, que se pusieron a rugir como fieras y sin poder resistir el dolor de las quemaduras, se tiraron al agua. El carpintero, que había visto la escena, gritó entonces:

    — ¡Muy bien, Jack; dales otro poco!

    Y adelantándose tomó un estropajo y luego de sumergirlo en el caldero de pez se puso a rociar de tal manera a los atacantes, ayudado por su compañero, que al poco no hubo un solo hombre en los tres botes atacantes que no hubiese recibido quemaduras, algunos de forma verdaderamente horrible, por lo cual aullaban y se retorcían de la manera más espantosa y como jamás creo haber visto antes. Vale la pena observar que aunque el dolor hace exhalar gritos a todo el mundo, cada nacionalidad tiene su manera particular de quejarse y los gritos que profieren son tan distintos entre sí como su lenguaje. No encuentro, para dar una idea de lo que eran esos gritos, un mejor nombre que el de aullidos, y no recuerdo nada que se pareciera más a lo que escuché antaño en los bosques, en la frontera del Languedoc, cuando los lobos hambrientos nos rondaban.

    Mientras esto sucedía mi socio y yo, dirigiendo al resto de los marineros que se hallaban a bordo, habíamos conseguido con gran habilidad ir enderezando el barco, y a poco nos fue posible asestar los cañones en su sitio. El artillero me pidió entonces que mandara retirar la chalupa para que quedara campo libre y pudiese él descargar sus piezas, pero llamándole a mi lado le dije que no hiciera fuego, ya que el carpintero se estaba arreglando perfectamente sin su auxilio y ordené en cambio que el cocinero calentara otra cantidad de pez. El enemigo, aterrorizado con lo que le costara su primera tentativa, no se atrevió a repetirla; los que se hallaban más lejos, al ver que el barco se enderezaba y empezaba a flotar nuevamente, terminaron sin duda por admitir su error y abandonar la empresa, convenciéndose de que aquello no había sido lo que creían. Así concluyó esta refriega tan divertida para nosotros, y luego de llevar a bordo algo de arroz, pan y legumbres, así como dieciséis cerdos que teníamos apartados desde hacía dos días, resolvimos no quedarnos más, sino reanudar nuestro viaje a fin de que el ataque no se repitiera, ya que con seguridad seríamos rodeados poco después por tal cantidad de aquellos vagabundos que acaso el caldero de pez no fuese bastante para dispersarlos.

    Por la noche subimos todo a bordo, y a la mañana siguiente nos hallábamos dispuestos a hacernos a la vela. Entretanto, anclados a cierta distancia de la costa y listos a todo evento, no nos preocupábamos mayormente de que se presentara cualquier enemigo. Al día siguiente, terminadas las tareas de reparación y seguros de que el buque no tenía ya vías de agua, salimos mar afuera. Nos hubiera agradado entrar en la bahía de Tonkín para informarnos acerca de los navíos holandeses que allí habían anclado; sin embargo, no nos atrevimos a acercarnos, pues habíamos divisado varios barcos que parecían encaminarse en la misma dirección. Pusimos, pues, rumbo al N.E. hacia la isla de Formosa, tan temerosos de ser avistados por un marino mercante inglés u holandés como cualquiera de éstos tiene miedo de serlo por un buque de guerra argelino en el Mediterráneo.

    De Formosa navegamos hacia el norte, manteniéndonos a cierta distancia de la costa de China hasta tener la seguridad de que habíamos dejado atrás todos los puertos chinos donde navíos europeos acostumbraban entrar. Estábamos resueltos a no caer en sus manos si ello era posible, especialmente en aquel país donde, de acuerdo con nuestras presentes circunstancias, estaríamos enteramente perdidos. Tan grande era mi temor de ser apresado por alguno de los navíos que creo firmemente haber preferido en aquel entonces caer en manos de la Inquisición española.

    Nos hallábamos a 30{\grado} de latitud y en consecuencia acordamos entrar en el primer puerto comercial que halláramos. Cuando nos acercamos a la costa, vino hacia nosotros un bote y en él un anciano piloto portugués, que al advertir que el nuestro era un barco europeo, deseaba ofrecernos sus servicios, de lo cual mucho nos alegramos haciéndole subir inmediatamente a bordo. Cuando estuvo con nosotros y sin siquiera preguntarnos hacia dónde pensábamos ir, despidió al bote que lo había traído.

    Pensando que aquel piloto nos llevaría al lugar donde nos pareciera mejor, me puse a hablar con él acerca de un viaje al golfo de Nankín, que se encuentra en la parte más septentrional de la costa china. El anciano, sonriendo, me dijo que conocía muy bien el golfo, pero quiso saber qué pensábamos hacer nosotros allí.

    Le contesté que vender nuestro cargamento y comprar en cambio porcelanas, zarazas, seda cruda, té y sedas estampadas, así como otras cosas, con lo cual emprenderíamos regreso a nuestro punto de partida. Me dijo que en ese caso el puerto más conveniente era Macao, donde no dejaríamos de encontrar excelente mercado para el opio, y podríamos comprar a nuestra vez toda clase de mercancías chinas a precios tan baratos como los de Nankín.

    No pudimos disuadir al anciano de su idea, en la cual se mostraba sumamente empecinado, le dije que además de comerciantes éramos caballeros ávidos de viajar, por lo cual sentíamos deseos de ver la gran ciudad de Pekín y la famosa corte del monarca chino.

    —Pues entonces —dijo el anciano—, es conveniente ir a Ning-Po, desde donde, remontando cinco leguas el río que allí se vuelca en el mar, podréis llegar al gran canal, verdadero río navegable que atraviesa el corazón del vasto imperio chino, cruza los otros ríos y salva algunas alturas considerables por medio de esclusas y compuertas hasta llegar a la misma ciudad de Pekín después de un viaje de casi doscientas setenta leguas.

    —Eso está muy bien, señor portugués —repuse yo—, pero no es lo que nos conviene ahora. Deseamos saber si podéis llevarnos a la ciudad de Nankín y si de allí es posible viajar más tarde a Pekín.

    Nos contestó afirmativamente, agregando que un barco holandés acababa de pasar poco antes llevando la misma ruta. Esto me asustó un poco, pues los barcos holandeses eran ahora nuestro terror y hubiésemos preferido encontrarnos con el mismísimo diablo (siempre que no se apareciese con una figura demasiado horrible) antes que con un navío de esa bandera. De ninguna manera estábamos en condiciones de hacerles frente ya que los barcos que hacen el tráfico son de gran tonelaje y poseían, por lo tanto, mucho más armamento que el nuestro.

    El anciano debió advertir mi confusión al nombrarme el paso del navío holandés, pues me dijo:

    —Caballeros, no hay razón para que os preocupéis por la cercanía de ese barco; supongo que Islanda no está en guerra con vuestra nación.

    —No lo está —dije yo—, pero nadie sabe las libertades que puedan llegar a tomarse los hombres cuando están fuera del alcance de la ley.

    — ¡Cómo! Si no sois piratas, ¿por qué habríais de sentir temor? De ninguna manera se atreverán a molestar a un pacífico barco mercante.

    Creo que si toda la sangre de mi cuerpo no afluyó a mis mejillas al escuchar aquella palabra, fue sólo por algún obstáculo puesto en mis venas por la misma naturaleza, pero sentí la más grande turbación imaginable, tanto que a pesar de mis esfuerzos por disimularla el viejo piloto la advirtió de inmediato.

    —Caballero —dijo—, veo que os sentís un tanto alterado por causa de mis palabras. Os ruego entonces que adoptéis simplemente el camino que os parezca más conveniente en la seguridad de que os ayudaré en todo lo que pueda.

    — ¡Ah, señor! —repuse—. Es cierto que me siento algo confundido acerca del rumbo a seguir, y lo que acabáis de decirme acerca de los piratas en estos mares, ya que no estamos en condiciones de hacerles frente; bien veis nuestro pobre armamento y cuan escasa es nuestra tripulación.

    —No os aflijáis por eso, señor —repuso él—. En estos últimos quince años no he oído decir ni una sola vez que hubiese piratas por estos lados, salvo un navío que fue visto el mes pasado, según me contaron, en la bahía de Siam. Tened, empero, la seguridad que ha debido dirigirse al sur, aparte de que se trata de un navío pequeño y poco adecuado para esas actividades. No fue construido para ser armado en corso sino que su perversa tripulación se apoderó de él luego que el capitán y algunos de sus hombres perecieron a manos de los malayos en la isla de Sumatra o sus inmediaciones.

    — ¿Es posible...? —exclamé fingiendo completa ignorancia sobre lo ocurrido—. ¿Asesinaron a su capitán?

    —No —repuso el piloto—, pero como no tardaron en apoderarse del buque y huir en él, es creencia general de que lo traicionaron y quizá buscaron los medios de que los malayos le mataran.

    —Pues bien, entonces merecen la muerte como si hubieran sido los mismos asesinos.

    —Fuera de toda duda —asintió el anciano— y creed que serán ejecutados tan pronto como un barco holandés o inglés les aprese, pues los capitanes se han comprometido a no darles cuartel si llegan a caer en sus manos.

    —Sin embargo —observé yo—, desde que según creéis el pirata se ha alejado de estos mares, ¿cómo esperan capturarlo?

    —En efecto, se supone que el navío ha huido, pero como os dije hace un momento el mes pasado estaba en la bahía de Siam, en el río Cambodge, siendo descubierto por algunos holandeses que habían pertenecido a su tripulación y fueron abandonados en tierra cuando los otros escaparon para hacerse piratas. Algunos barcos mercantes ingleses y holandeses que fondeaban río arriba estuvieron a punto de apresarlo; por cierto que si las chalupas que encabezaban el ataque —agregó— hubiesen sido bien apoyadas por las restantes, con toda seguridad lo habrían tomado por asalto, pero eran solamente dos y los del barco viraron de bordo y las desmantelaron antes que las otras estuvieran a igual distancia. Como se alejara luego a toda vela, las restantes chalupas no fueron capaces de seguirlo y así escapó. Con todo, hay una descripción tan exacta del navío que se tiene la seguridad de reconocerlo dondequiera que lo encuentren, y hay promesa formal de no conceder cuartel a ningún hombre de la tripulación, incluido el capitán sino colgarlos a todos de la antena de su barco.

    — ¡Cómo! —exclamé—. ¿Los ejecutarán a todos con justicia o sin ella? ¿Los ahorcarán primero para juzgarlos después?

    — ¡Oh, caballero, no hay necesidad de ser tan estrictos con miserables como ésos! Basta con atarlos de dos a dos espalda contra espalda, y tirarlos al mar, ya que no merecen otra cosa.

    Como sabía que el anciano estaba en mi poder abordo y que no podría causarme daño alguno, le interrumpí bruscamente para decirle:

    —Pues bien, señor; ésa es justamente la causa por la cual quiero que nos llevéis a Nankín y no a Macao o a cualquier otra parte del país donde haya navíos ingleses u holandeses. Ya veo que estáis bien enterado de que los capitanes de tales barcos son una pandilla orgullosa e insolente, incapaz de discriminar sobre lo que es la justicia y conducirse de acuerdo con las leyes de Dios y la naturaleza. Tan envanecidos están en su profesión que haciendo mal uso de sus poderes intentan proceder como asesinos para castigar a los que consideran ladrones; no vacilan en ofender a hombres falsamente acusados y los declaran culpables sin haber hecho la menor averiguación. ¡Ah, creedme que espero vivir bastante para obligar a algunos de ellos a rendirme cuenta de sus actos y tal vez a enseñarles cómo debe administrarse justicia! Sí, les demostraré que ningún hombre debe ser tratado como un criminal hasta que no se presente la clara evidencia de su crimen y la seguridad que es el culpable.

    Y entonces hice al anciano piloto la confesión de que nuestro barco era el que andaban persiguiendo, le relaté la escaramuza que había sostenido con las chalupas y cuánta impericia y cobardía demostraron en ella. Luego de explicarle la compra del barco y cómo los holandeses nos ayudaron, agregué las razones que me asistían para sospechar que la historia del asesinato del capitán a manos de los malayos era falsa, así como la acusación de que los tripulantes del barco se habían entregado a la piratería. Le dije finalmente que antes de atacarnos por sorpresa, obligándonos a defender nuestras vidas, aquellos capitanes mercantes debieron asegurarse primeramente de si tenían derecho y motivo para hacerlo, por lo cual la sangre derramada en la lucha caía sobre ellos y de ninguna manera sobre nosotros, que obrábamos en legítima defensa.

    Profundamente asombrado quedó el anciano al escucharme y me aseguró que hacíamos muy bien en encaminarnos hacia el norte del país, agregando que su consejo era el de vender el barco en China, cosa factible, y comprar o construir otro en ese país.

    —Cierto —agregó— que no será un navío tan bueno, pero sí suficiente como para llevaros a todos, así como vuestros efectos, de vuelta a Bengala o el sitio que prefiráis.

    Le aseguré que seguiría su consejo tan pronto entráramos en algún puerto donde hubiera un barco adecuado o un comprador para el nuestro. Me dijo que en Nankín se encontraban siempre individuos dispuestos a adquirir un navío, y que para el retorno lo más apropiado era un junco chino, todo lo cual estaba él dispuesto a procurarme apenas llegáramos.

    —Muy bien, señor —le dije—, pero reparad en que si nuestro barco era ya tan bien conocido como me habéis dicho, su traspaso puede ser causa de que algún inocente se vea envuelto en un terrible conflicto y tal vez resulte asesinado a sangre fría. Pensad que donde quieran encuentren el barco no vacilarán en declarar culpables a quienes estén a bordo y probablemente la tripulación entera sea miserablemente asesinada.

    —También buscaré una medida de impedir que tal cosa ocurra —dijo el anciano—, por cuanto conozco a los capitanes de quien con tanta verdad os habéis expresado. A medida que pasen los entrevistaré a todos a fin de que se aclare el malentendido y pueda demostrarles que estaban enteramente equivocados, ya que si los antiguos tripulantes del barco se marcharon con él, eso no prueba que se dedicaran a la piratería, y en segundo término, la actual tripulación del navío no es aquélla, sino una distinta, reclutada después de que vosotros comprasteis honestamente el barco para vuestros viajes comerciales. Estoy persuadido de que les convenceré y que en el futuro procederán con más tino y cautela.

    —Muy bien —dije yo—. ¿Y les daréis, señor, un mensaje de mi parte?

    —Lo haré con gusto —me respondió— si me lo entregáis por escrito para que pueda probar que viene de vos y no es un invento mío.

    Contesté afirmativamente y tomando papel y pluma me puse a detallar un detallado informe sobre el ataque de que había sido víctima por parte de las chalupas, la pretendida razón del mismo y el injusto y cruel designio que llevaba al realizarlo. Agregué, dirigiéndome a los comandantes responsables de aquel atropello, que no solamente debían sentirse avergonzados de su acción, sino que en el futuro, si llegaban alguna vez a Inglaterra y yo vivía aún para saberlo, les haría pagar cara su insolencia a menos que las leyes de mi país estuvieran en desuso cuando volviera a él.

    El anciano piloto leyó una y otra vez el documento y me preguntó si me responsabilizaba por él. Le dije que así lo haría mientras algo me quedara en el mundo, y a la espera de que la oportunidad se presentara de cumplir mi promesa en Inglaterra. Sin embargo, jamás hubo ocasión de que el piloto fuese portador de aquella carta, por cuanto no regresó más a su antigua residencia.

    Mientras en tal forma conversábamos, seguíamos navegando rumbo a Nankín y después de trece días de viaje anclamos en el extremo sudoeste del gran golfo del mismo nombre. Allí, y de manera casual, vine a saber que dos barcos holandeses habían llegado a puerto antes que nosotros y que no había manera de escapar de ellos si proseguíamos en su dirección. Consulté a mi socio, que estaba tan afligido como yo y hubiese querido desembarcar a salvo en cualquier parte. Yo conservaba algo más de serenidad y pregunté al anciano piloto si no habría algún puerto o ensenada donde echar anclas para emprender privadamente negociaciones con los compradores chinos, sin peligro de vernos atacados por el enemigo. Me aconsejó navegar unas cuarenta y dos leguas hacia el sur rumbo a un pequeño puerto llamado Quinchang, donde los misioneros hacían habitualmente escala viniendo de Macao en su camino a las regiones chinas para evangelizar a los naturales. Me aseguró que allí no anclaban barcos europeos, y que una vez seguros y en tierra, sería más fácil considerar la ruta a seguir. Me advirtió que no era lugar para comerciantes, salvo ciertas épocas del año, en que se efectuaba allí una especie de feria donde los mercaderes japoneses acudían para traficar con productos chinos.

    Nos pareció conveniente navegar hacia ese sitio cuyo nombre acaso no alcanzo a escribir correctamente porque lo he olvidado; el librito donde, juntamente con otros sitios y puertos, había escrito ese nombre, se estropeó en el agua a causa de un accidente que relataré en su debido tiempo. Pero sí recuerdo que los comerciantes japoneses y chinos con los cuales tuvimos relación le daban un nombre distinto del empleado por nuestro piloto portugués, quien, repito, lo pronunciaba Quinchang.

    Como estábamos todos de acuerdo en dirigirnos a ese punto, levamos anclas al siguiente día después de bajar solamente dos veces a tierra para renovar nuestras provisiones de agua dulce; en ambas ocasiones los naturales se mostraron muy amables con nosotros y nos trajeron diversas cosas para vender, tales como alimentos, plantas, raíces, té, arroz, y algunas aves; y todo lo cobraban a buen precio.

    Tardamos cinco días en arribar al otro puerto a causa de vientos contrarios, pero nos pareció lugar seguro y fue con gran alegría y hasta puedo decir que con reconocimiento que desembarqué resuelto, al mismo tiempo que mi socio, a no

 poner nunca más los pies a bordo de aquel fatídico navío, siempre que nos fuera posible solucionar de cualquier modo nuestra presente situación. Me es preciso declarar aquí que de todas las circunstancias de la vida que me hayan sido dadas a conocer, ninguna hace más desdichado a un hombre que sentirse constantemente atemorizado.

    Tanto mi socio como yo no pasábamos una sola noche sin soñar con cuerdas y con los pañoles de las vergas, es decir, con patíbulos; si no era eso, se trataba de luchas en las cuales caíamos prisioneros, de asesinar o ser asesinados. Una noche me puse tan furioso en mis sueños, viendo que los holandeses nos abordaban, que creyendo que golpeaba a uno de los asaltantes descargué tales puñetazos contra el tabique de mi camarote que me lastimé las manos, quebrándome los nudillos y desgarrando de tal manera la piel de los dedos que me desperté por efecto de los golpes, y hasta creí que perdería dos dedos.

    Uno de los temores más oprimentes que sentía era pensar en las crueldades que con nosotros cometerían los holandeses si caíamos en sus manos. Recordaba lo acontecido en Amboina (1) y se me ocurría pensar que acaso los holandeses nos torturaran como allí habían hecho con nuestros compatriotas, obligando a algunos hombres, por la fuerza insoportable del sufrimiento, a confesar crímenes de los cuales jamás habían sido culpables. Capaces de hacernos admitir hasta que éramos piratas, no vacilarían en sentenciarnos a muerte con todas las apariencias de verdadera justicia; probablemente se sentirían tentados a proceder de esa manera en vista de la ganancia que les daría el buque y su cargamento, todo lo cual valía no menos de cuatro o cinco mil libras.

    Estas ideas nos atormentaban sin darnos un momento de reposo. Apenas nos deteníamos a pensar que los capitanes de navío carecen de autoridad para proceder de tal manera y que si nos entregábamos a ellos no podrían someternos a torturas o matarnos sin ser responsables de sus actos y verse obligados a rendir cuenta de ellos cuando tornaran a su país. Nada de eso nos satisfacía; ¿qué ventaja hubiera sido para nosotros el que más tarde les pidieran cuenta de su proceder? Y si éramos sacrificados por ellos, ¿de qué nos serviría que alguna vez nuestros asesinos recibieran el condigno castigo?

    Pero así como la ansiedad pesaba insoportablemente sobre nosotros mientras estábamos en alta mar, así ahora nos sentimos llenos de satisfacción apenas tocamos tierra firme. Mi socio me contó un sueño que había tenido en el cual se veía soportando sobre la espalda un terrible peso que le era necesario subir hasta la montaña. En momentos en que se sentía desfallecer, el piloto portugués había llegado para librarlo del fardo, a tiempo que la montaña desaparecía y el terreno se tornaba sumamente liso y llano delante de él. Y así era en efecto, ya que todos nos sentíamos repentinamente aliviados de un peso abrumador. Por mi parte, sentí aligerarse mi corazón de tan dura carga que ya me era imposible continuar soportando, y como he dicho resolvimos en común no embarcamos nunca más en aquel navío.






	\chapter{Viaje a través de la China}





    Apenas desembarcamos, el anciano piloto, que era ahora nuestro amigo, buscó alojamiento y depósito para nuestros efectos, encontrándolos por fin en el mismo lugar. Se trataba de una especie de cabaña con una casa adyacente, todo ello construido de bambú y rodeado de una empalizada de altas cañas que impedían el acceso a los rateros que, por lo que supimos, abundaban mucho en el país. Los magistrados nos concedieron también una modesta guardia, y en nuestra puerta teníamos siempre a un centinela armado de una alabarda o especie de pica, al cual dábamos diariamente una pinta de arroz y una moneda que valía unos tres peniques; en esa forma nuestros bienes estaban bien asegurados.

    La feria o mercado que periódicamente se efectuaba en aquel lugar habíase realizado tiempo atrás, pero sin embargo supimos que tres o cuatro juncos permanecían en el río y que dos barcos japoneses cargados de mercaderías compradas en China aún no se habían hecho a la vela, permaneciendo los comerciantes japoneses en tierra firme.

    Lo primero que buscó nuestro anciano piloto portugués fue vincularnos a tres misioneros católicos que vivían en el pueblo y que habían pasado un tiempo convirtiendo a las gentes al cristianismo. Nosotros pensábamos que no habían logrado gran resultado con su prédica y que los ya convertidos eran pésimos cristianos, pero naturalmente nada de eso nos concernía. Uno de los misioneros era francés y se llamaba el padre Simón; era hombre alegre y bien dispuesto, de conversación franca y libre, que no daba la impresión de seriedad ni tenía el aire grave de los otros dos, uno de los cuales era portugués y el otro genovés. El padre Simón, sumamente servicial y de maneras sencillas, resultaba excelente compañero; los otros, mucho más reservados, parecían rígidos y austeros y se aplicaban empecinadamente al trabajo que allí los llevara, es decir, mezclarse entre las gentes y tratar de obtener poco a poco su confianza y aprecio.

    El misionero francés había recibido, según parece, órdenes del superior de la misión para encaminarse a Pekín, real sede del emperador chino, y sólo esperaba la llegada de otro sacerdote que venía desde Macao para emprender con él el viaje. Apenas lo habíamos conocido cuando ya me invitaba a que fuésemos juntos, diciéndome que me haría conocer todas la admirables cosas de aquel poderoso imperio, y entre otras la ciudad más grande del mundo, una ciudad que de acuerdo con sus palabras, no podía ser igualada por París y Londres juntas.

    Se refería a Pekín que, no tengo reparo en admitirlo, es una enorme ciudad densamente poblada; pero como yo miraba aquellas cosas con ojos distintos que los demás, mi opinión sobre ellas será dada en su oportunidad, cuando en el curso de este relato y este viaje llegue la ocasión  de  hablar  en  particular  de  ellas.

    Vuelvo ahora a nuestro amigo el misionero. Cenando un día con él, y sintiéndonos todos sumamente alegres, me mostré algo inclinado a acompañarlo en su viaje, por lo cual se puso a apremiarme y lo mismo a mi socio, tratando de persuadirnos para que diéramos nuestro consentimiento.

    —Pero, padre Simón —dijo entonces mi socio—, ¿por qué deseáis tanto nuestra compañía? Bien sabéis que somos herejes y por lo tanto no podéis ni amarnos ni tener gusto en estar con nosotros.

    — ¡Oh! —respondió el misionero—. Tal vez con el tiempo lleguéis a ser buenos católicos. Mi tarea aquí es la de convertir a los paganos, ¿y quién sabe si no me será posible convertiros también a vosotros?

    —Muy bien, padre —dije yo—. Eso quiere decir que nos iréis predicando durante todo el viaje.

    — ¡Oh, no seré tan fastidioso como para eso! —replicó él—. Nuestra religión no nos priva de buenos modales y además —agregó— somos aquí casi compatriotas si nos comparamos al sitio en que nos vemos reunidos. Sois hugonotes y yo católico, pero con todo tenemos el cristianismo en común, y por otra parte somos caballeros y podemos alternar en el viaje sin causarnos mutuamente molestias.

    Me gustaron mucho sus palabras que me hicieron recordar a aquel otro sacerdote que había dejado en el Brasil. Sin embargo, el carácter del padre Simón difería del de aquel joven clérigo; aunque de ninguna manera podía ser acusado de ligereza censurable, carecía de aquel profundo celo cristiano, aquella piedad y hondo sentido religioso que poseía el otro eclesiástico del cual tantas veces he hablado.

    Pero dejemos un momento al padre Simón, aunque él no nos dejara a nosotros y siguiera solicitando nuestra compañía para el viaje; otra cosa debíamos solucionar ante todo, ya que aún nos quedaban por liquidar nuestras mercaderías y también el barco, y empezábamos a afligirnos al ver que aquel sitio era muy poco importante en materia de comercio. En una oportunidad estuve a punto de embarcarme con destino al río de Kilam y a la ciudad de Nankin, pero la Providencia pareció más que nunca presente en su intervención para salvarme. Lo primero fue que el viejo piloto portugués nos trajo a un comerciante del Japón que se mostró interesado en saber qué productos queríamos vender, y empezó comprándonos el cargamento de opio por el cual recibimos excelente precio pagado en oro y al peso, parte en monedas de su país y parte en pequeñas cuñas de oro, cada una de las cuales pesaba diez u once onzas. Mientras discutíamos la venta del opio se me ocurrió que tal vez aquel hombre quisiera comprarnos el barco, y ordené al intérprete que le propusiera la venta. A las primeras palabras se encogió de hombros, pero días más tarde volvió acompañado de uno de los misioneros a manera de intérprete y me dijo que tenía una propuesta que hacerme, la cual consistía en lo siguiente: nos había comprado gran cantidad de mercaderías antes de que se le ofreciera el barco en venta por lo cual no le quedaba suficiente dinero para pagar su precio; ahora bien, si yo permitía que la tripulación del barco continuase a bordo él me tomaría en arriendo para ir al Japón, desde donde lo enviaría a las Islas Filipinas con un nuevo cargamento cuyo producto bastaría para pagar el flete total de aquellos viajes; entonces, a su regreso, estaría en condiciones de comprar el buque.

    Escuché atentamente la proposición, y de súbito me sentí invadido por mis ansias errantes, tanto que al punto me pareció posible hacer en persona el viaje hasta las Filipinas y de allí embarcarme a los mares del Sur. Pregunté al comerciante japonés si no estaría dispuesto a arrendar el buque hasta las Islas Filipinas y dejarnos allí, pero me contestó negativamente, declarando que no tendría cómo regresar con su cargamento, y que en cambio me proponía llevarnos al Japón cuando volviera el navío. Esta idea tampoco me pareció mala y me manifesté dispuesto a emprender el viaje; pero mi socio, más sensato que yo, me persuadió de no embarcarme recordándome los peligros del mar y también de los japoneses, de quienes me dijo que son individuos falsos, crueles y traidores; aparte de eso estaban los riesgos derivados de los españoles de las Filipinas, aún más falsos, crueles y traidores que los otros.

    Pero abreviemos este largo rodeo para llegar a su conclusión. Lo primero que debíamos hacer era consultar al capitán del barco y a sus hombres si estaban dispuestos a viajar al Japón. Mientras nos ocupábamos en ello vino a verme el joven que me dejara mi sobrino como compañero de viaje y me manifestó que a su juicio aquel viaje se anunciaba excelente y lleno de posibilidades de ganar dinero. Se sentiría, agregó, muy contento si yo me embarcaba a tal efecto, pero si decidía permanecer en tierra deseaba mi consentimiento para ir en calidad de comerciante o como a mí me pareciera mejor. Terminó diciéndome que si alguna vez retornaba a Inglaterra y me encontraba con vida en mi patria, me daría detallada cuenta de los resultados obtenidos, que podría considerar como de mi pertenencia.

    Me disgustó mucho la idea de separarme de aquel muchacho, pero considerando las favorables posibilidades que se le presentaban y que se trataba de un joven lleno de las mejores condiciones para aprovecharlas, decidí darle mi consentimiento, pero antes le manifesté que consultaría a mi socio y le daría una respuesta al día siguiente.

    Discutimos la cuestión con mi socio, quien me hizo una generosa oferta.

    —Ya sabéis —me dijo— que el barco ha sido una fuente de desgracias para nosotros y que ambos hemos resuelto no embarcarnos nunca más en él; por lo tanto, si vuestro mayordomo (como él le llamaba) se aventura en ese viaje, le dejaré mi parte del buque para que la aproveche lo mejor que pueda, y si vivimos para encontrarnos alguna vez en Inglaterra y él alcanza a tener éxito en su empresa, nos entregará la mitad del beneficio obtenido con el flete del barco, y la otra mitad será suya.

    En vista de que mi socio, quien nada tenía que ver con aquel muchacho, le hacía semejante oferta, yo estaba obligado por lo menos a imitarlo. Y como la tripulación se mostró dispuesta a viajar con él concedimos en propiedad la mitad del barco, recibiendo de su puño y letra un documento por el cual debería rendirnos cuenta de la otra parte; y pronto zarpó para el Japón.

    El comerciante japonés demostró ser un hombre honesto y cumplidor; no sólo lo protegió en el Japón sino que le hizo extender una licencia para que pudiese bajar a tierra, cosa que por lo común no pueden lograrlos europeos; le pagó puntualmente el flete, enviándolo luego a las Filipinas con un cargamento de porcelanas japonesas y chinas y un sobrecargo por cuenta suya quien, luego de traficar con los españoles, trajo de regreso mercaderías europeas así como gran cantidad de clavo y otras especias. No solamente recibió a satisfacción el precio del flete sino que, negándose al regreso a vender el barco al comerciante japonés, éste lo proveyó con un cargamento de géneros por su cuenta. Con ellos, más algún dinero y especias que le pertenecían, el joven inglés volvió a Manila donde pudo vender ventajosamente su cargamento a los españoles.

    Después de vincularse muy bien en Manila consiguió que su barco fuese declarado libre y el gobernador de Manila se lo arrendó para viajar a Acapulco, en América, sobre la costa mejicana, dándole asimismo licencia para que pudiera desembarcar y viajar hasta Méjico de donde le sería posible retornar a Europa con todos sus hombres a bordo de un buque español.

    Tuvo una excelente navegación hasta Acapulco, allí vendió por fin su barco y obtuvo el permiso necesario para viajar por tierra a Portobelo, de donde halló medio para pasar a Jamaica con todos sus bienes, y unos ocho años más tarde llegó a Inglaterra lleno de riquezas, de lo cual se hablará en su lugar.

    Pero volvamos a lo que concierne al buque y a su tripulación, y entre ello a considerar qué recompensa daríamos a los dos hombres que tan oportunamente nos habían advertido del peligro que nos amenazaba en el río de Cambodge. La verdad es que nos habían hecho un señalado servicio y merecían ser pagados por él, bien que, dicho sea de paso, fueran un par de redomados bribones. Ambos habían creído firmemente la historia de que éramos piratas y nos habíamos escapado con el buque, de manera que vinieron a nosotros no solamente para traicionar a quienes intentaban apresarnos sino con la intención de embarcarse, en un navío dedicado, según creían, a piratear. Uno de ellos terminó confesando más tarde que solamente la esperanza de enriquecerse por medio del pillaje lo había decidido a cometer esa acción. Con todo, sus servicios no habían sido pequeños, y como además yo les había prometido mostrarme generoso ordené en primer lugar pagarles la cantidad que según sus declaraciones les debían a bordo de sus respectivos barcos; el inglés recibió el sueldo de diecinueve meses, y el holandés de siete; además les regalé a cada uno cierta cantidad de dinero en oro, que aunque pequeña, los llenó de contento. A continuación hice que el inglés tomara el puesto de artillero del barco, ya que el nuestro había ascendido a segundo piloto contador; al holandés lo nombré contramaestre, y los dos parecieron muy contentos con esto y demostraron más tarde sus excelentes condiciones como marinos, ya que se trataba de enérgicos y recios individuos.

    Nos encontrábamos ahora en la costa de China. Si antes, en Bengala, me había sentido desterrado y a remota distancia de mi hogar, cuando" en realidad tenía muchos caminos para retornar, ¿qué podía decir ahora que me hallaba mil leguas más lejos que antes de mi patria, privado de toda perspectiva y de todo medio para tornar a ella?

    Lo único que nos quedaba era aguardar unos cuatro meses hasta que se efectuara allí otra feria donde sería posible comprar diversas manufacturas de la región y acaso encontrásemos algún junco chino o un barquichuelo de Tonkín que estuviera en venta y pudiese llevarnos junto con nuestros bienes al lugar que decidiéramos.

    Me pareció una buena idea y resolví esperar; por otra parte, como nuestras personas no eran sospechosas, tal vez si algún navío inglés u holandés entraba en el puerto sería posible embarcar en él nuestras mercaderías y sacar pasaje para cualquier punto de la India, siempre más próximo a nuestra patria.

    Alimentando estas esperanzas resolvimos continuar allí, pero para combatir el tedio hicimos dos o tres expediciones al interior del país. En primer lugar empleamos diez días viajando a Nankín, ciudad digna de ser visitada y de la cual se asegura que posee un millón de habitantes, cosa en la que no creo. Muy bien construida, con calles regulares que se cortan en ángulo recto, tiene una apariencia sumamente agradable.

    Sin embargo, cuando comparo la miserable población de aquellas regiones con la nuestra, y pienso en sus edificios, su modo de vivir, su religión y gobierno, así como sus bienes y lo que algunos llaman su gloria, debo confesar que apenas si me parece digno de mencionarlos en estos relatos a fin de que no pierdan el tiempo quienes los lean más adelante.

    Tanto sus fuerzas como su grandeza, la navegación, economía y comercio, son imperfectos e insignificantes en comparación con los europeos. Lo mismo en cuanto a sus conocimientos, el aprendizaje y la profundidad que alcanzan en las ciencias; cierto que poseen globos y esferas, así como un superficial conocimiento de las matemáticas, pero cuando uno inquiere más profundamente en sus conocimientos, ¡cuan poca visión demuestran sus más aventajados estudiosos! Nada saben sobre el movimiento de los cuerpos celestes, y tan groseramente ignorantes se muestran que cuando se produce un eclipse solar piensan que un gran dragón ha arrebatado al sol para llevárselo y se ponen a hacer estruendo con todos los tambores y calderos que hay en el país para asustar al monstruo, lo mismo que lo haríamos nosotros para encerrar un enjambre de abejas.

    Sentía yo el deseo de visitar la ciudad de Pekín de la cual tanto había oído hablar, y el padre Simón me importunaba diariamente para que llevara a cabo el viaje. Por fin, estando resuelta su partida por cuanto el otro misionero que lo acompañaría acababa de llegar de Macao, fue necesario decidir si iríamos o no con él, de manera que consulté el caso con mi socio diciéndole que dejaba la respuesta librada a su elección. Contestó por la afirmativa y por lo tanto nos preparamos a efectuar el viaje.

    Iniciamos la jornada con la gran ventaja de estar seguros del camino, ya que fuimos admitidos en la comitiva de uno de los mandarines, especie de virrey y alto magistrado de aquella provincia en la cual tenía su residencia. Aquellos funcionarios despliegan gran pompa en todo momento, viajando con numeroso séquito y recibiendo el continuo homenaje del pueblo que, muchas veces, les debe el encontrarse tan empobrecido, ya que por todas las regiones que atraviesan en el viaje deben avituallar tanto al mandarín como a su comitiva.

    Me llamó la atención muy particularmente que mientras permanecimos con él recibimos suficientes provisiones para nosotros y nuestros caballos, productos entregados en las regiones que íbamos atravesando y que pertenecían al noble, pero al mismo tiempo se nos obligaba a pagar por cada cosa que nos daban de acuerdo con los precios locales, de lo cual se encargaba el administrador del mandarín, especie de comisario que recibía diariamente el dinero que nos exigían. De manera que aquel viaje en la comitiva del mandarín, aunque fuera para nosotros un alto favor, no lo era menos en lo que a él respecta, sino por el contrario una continua fuente de provecho si se considera que cerca de treinta personas viajaban de la misma manera junto a nosotros bajo la protección de aquella escolta que más merece el nombre de convoy; repito que era un pingüe provecho para él desde que las provisiones no le costaban absolutamente nada, siendo obligadamente cedidas por los pobladores y recibiendo él en cambio nuestro dinero en pago de las diarias raciones.

    Veinticinco días empleamos en el viaje a Pekín, a través de un país infinitamente populoso pero muy poco y mal cultivado; la economía y el modo de vivir eran miserables, pese a que los chinos se jactan mucho de la diligencia de su pueblo. Sí, miserable era todo aquello en especial para nosotros que, acostumbrados a otra manera de vivir, podemos compararla o pensar lo que sería tener que someterse a esa exigencia; sin embargo, para aquellos pobres hombres que no conocen otra cosa, ha de resultar tolerable. El orgullo de los chinos es muy grande y solamente lo excede la pobreza, que se agrega a lo que yo llamo su miseria; incluso me veo obligado a pensar que los desnudos salvajes de América viven más felizmente que estos hombres, porque como nada tienen nada desean; éstos son orgullosos e insolentes, mientras en realidad sólo son mendigos y ganapanes. Su ostentación es inexpresable, y se manifiesta especialmente en el modo de vestirse, en sus casas y el afán que tienen de rodearse de multitud de sirvientes y esclavos; y finalmente en el desprecio, ridículo en alto grado, que demuestran hacia cualquiera que no pertenezca a su pueblo.

    Debo admitir que viajé posteriormente con mucho más agrado en los desiertos e inmensas soledades de la Gran Tartaria; sin embargo, los caminos que cruzábamos ahora estaban bien pavimentados y excelentemente mantenidos, por lo que resultaba muy cómodo viajar por ellos, aunque era profundamente desagradable contemplar a aquel altanero, imperioso e insolente pueblo en medio de la más grosera y crasa ignorancia, por cuanto su tan afamado ingenio no es otra cosa. Mi amigo el padre Simón y yo nos divertíamos con mucha frecuencia al observar la mezcla de mendicidad y orgullo que poseían aquellas gentes.

    Una vez, por ejemplo, llegando a la casa de un caballero rural —como le llamaba el padre Simón— a unas diez leguas más allá de la ciudad de Nankín, tuvimos el honor de cabalgar por espacio de dos millas en compañía del propietario, cuyo séquito era verdaderamente quijotesco, mezcla perfecta de pompa y miseria. El traje de este monigote hubiera sido adecuadísimo para un bufón; era de una indiana muy sucia, abigarrado como el ropaje de un juglar y lleno de adornos tales como mangas colgantes, borlitas y cuchillas por todas partes; se cubría con una capa de tafetán grasienta como la de un carnicero y que daba prueba de que su señoría era de un exquisito desaseo. Montaba un caballo enteco, flaco, hambriento y que cojeaba, como los que en Inglaterra se venden por treinta o cuarenta chelines; dos esclavos le seguían a pie a fin de hacer andar a la desgraciada cabalgadura. El noble señor tenía un látigo en la mano y con él azotaba a la bestia en la cabeza al mismo tiempo que los esclavos lo hacían en las ancas; así anduvo al lado nuestro, seguido de diez o doce sirvientes, dirigiéndose desde la ciudad a su residencia campestre a una media legua delante de nosotros. Como andábamos despacio, aquel extravagante caballero se nos adelantó y pronto nos detuvimos en un villorrio para descansar durante una hora.

    Cuando llegamos cerca de su residencia campestre vimos al grande hombre instalado en una especie de jardín, tomando su refrigerio. Se lo distinguía fácilmente y nos dieron a entender que cuanto más lo mirásemos, más satisfecho y complacido quedaría. Estaba sentado bajo un árbol, especie de palmera enana que proyectaba sombra sobre su cabeza por el lado sur, pese a lo cual habían plantado debajo del árbol una gran sombrilla que quedaba muy bien. El noble se había sentado, o más bien reclinado, en una gran silla de brazos, pues era hombre corpulento y muy pesado, mientras dos esclavas le traían su alimento. Otras dos estaban a su lado destinadas a un servicio que, según pienso, pocos caballeros en Europa aceptarían de sus criados; en efecto, una de ellas alimentaba al señor con una cuchara mientras la otra sostenía el plato con una mano y con la otra iba retirando la comida que caía sobre la barba de su señoría y en su capa de tafetán; entretanto, el gran bruto pensaba seguramente que no era digno de él emplear sus propias manos en menesteres que los mismos monarcas y reyes prefieren hacer personalmente antes que ser ofendidos por los torpes dedos de sus sirvientes.

    En cuanto al mandarín en cuyo séquito viajábamos, era respetado como un rey; aparecía rodeado de sus caballeros y atendido en todo con tal pompa que fue muy poco lo que alcancé a ver de él a la distancia; observé sin embargo que las cabalgaduras de su comitiva valían mucho menos que cualquier caballo de carga en Inglaterra; no obstante, estaban tan enjaezados con mantas, adornos y otros oropeles que uno no alcanzaba a saber si eran robustos o flacos y, en una palabra, apenas podíamos distinguir la cabeza y las patas.

    Mi corazón estaba ahora libre de sus preocupaciones, y como todas las perplejidades e inconvenientes que he relatado habían terminado para mí, aquel viaje me resultó sumamente grato, ya que ningún pensamiento desagradable vino a turbarlo; tampoco me ocurrió nada de malo en él, salvo que al vadear un riacho cayó mi caballo haciéndome comprar el suelo, según suele decirse; en una palabra, me arrojó al agua. El riacho no era hondo pero bastó para empaparme, y si menciono el incidente es porque allí se estropeó un libro de anotaciones donde había estampado los nombres de diversas personas y lugares que deseaba recordar más adelante. Como no tomé la debida precaución, la humedad invadió las páginas, y las palabras terminaron por borrarse, perdiendo así los nombres de muchas partes visitadas en el viaje que relato. Por fin llegamos a Pekín. Conmigo viajaba solamente el joven que mi sobrino me dejara en calidad de sirviente y que se había conducido con gran lealtad y diligencia; mi socio, por su parte, llevaba consigo a un criado que era además pariente suyo. En cuanto al piloto portugués que se mostraba ansioso por visitar la corte, le costeamos los gastos del viaje por el gusto de su compañía y además porque era nuestro intérprete; entendía muy bien el idioma del país y hablaba buen francés y algo de inglés. Por cierto que aquel anciano nos fue sumamente útil en todas partes; no llevábamos una semana en Pekín cuando vino riéndose a vernos.

    — ¡Ah, señor inglés! —exclamó—. Tengo algo que deciros que alegrará vuestro corazón.

    — ¿Alegrar mi corazón? —dije yo—. ¿Y qué puede ser? No conozco a nadie en este país que pueda producirme ni alegría ni pesar en semejante grado.

    —Sí, sí —insistió el piloto en su inglés chapurreado—. A vos os alegrará, pero a mí entristecerme.

    Esta frase me llenó de curiosidad.

    — ¿Y por qué habría de entristeceros a vos? —pregunté.

    —Porque —me respondió— me habéis traído después de un viaje de veinticinco días y me dejáis que me vuelva solo. ¿Y cómo me las arreglaré yo para regresar a mi puerto sin un barco, sin un caballo, sin pecune?

    Así llamaba al dinero, y el pésimo latín que sembraba frecuentemente en la conversación nos divertía mucho.

    En una palabra, me explicó que una gran caravana de comerciantes polacos y moscovitas se disponía a partir de Pekín cuatro o cinco semanas más tarde, con intención de trasladarse por tierra a Moscú, y él suponía que yo no iba a perder tal oportunidad para marcharme con ella y, por lo tanto, dejarlo solo para el retorno. Confieso que sus noticias me sorprendieron al comienzo y que se expandió por mi alma una secreta alegría como jamás sintiera anteriormente y que me sería imposible describir. Durante un rato fui incapaz de responder una sola palabra al anciano, pero por fin me volví hacia él.

    — ¿Cómo habéis sabido eso? —pregunté—. ¿Estáis seguro de que es verdad?

    —Sí —respondió—, por cuanto encontré esta mañana en la calle a un viejo amigo mío, un armenio o griego, como vosotros les llamáis. Viene de Astrakán y se disponía a seguir viaje a Tonkín, donde en otros tiempos trabamos relación, pero acaba de cambiar de planes y está resuelto a incorporarse a la caravana para viajar a Moscú y luego, por el curso del río Volga, a Astrakán.

    —Muy bien, señor —dije yo entonces—, no abriguéis ningún temor de quedaros solos aquí. Si existe ese camino para mi regreso a Inglaterra, vuestra será la culpa si os volvéis solo a Macao.

    Nos pusimos entonces a consultar lo que debíamos hacer, y pregunté a mi socio qué pensaba de las novedades que trajera el piloto y si le convenían para el curso de sus negocios. Me dijo que estaba en completa libertad de tomar esa ruta por cuanto había dejado sus cosas tan bien arregladas al salir de Bengala, y sus bienes en tan buenas manos, que a fin de completar el excelente viaje que habíamos hecho hasta Pekín estaba dispuesto a comprar sedas chinas, tanto crudas como manufacturadas, que merecieran llevarse a Inglaterra para vender allá, pudiendo retornar él a Bengala más" tarde con un barco de la Compañía.

    Esto resuelto, convinimos que si nuestro piloto portugués deseaba viajar con nosotros pagaríamos sus gastos hasta Moscú, o bien hasta Inglaterra, si estaba dispuesto a proseguir la ruta. Por cierto que no era aquélla una generosidad excesiva, e incluso debíamos hacer más por él ya que los servicios que nos había prestado lo merecían sobradamente. Piloto en alta mar, había sido nuestro corredor e intermediario en tierra, y la relación que nos facilitara con el comerciante japonés valía cientos de libras que estaban ahora en nuestro poder. Nos consultamos al respecto, y manifestándonos dispuestos a gratificar como era justicia a aquel hombre, y ansiosos por tenerlo aún en nuestra compañía, ya que resultaba sumamente útil en muchas ocasiones, decidimos hacerle entrega de una cantidad en metálico que, según mis cálculos, llegaba a ciento setenta y cinco libras esterlinas, además de pagarle todos los gastos del viaje incluyendo su cabalgadura, excepto los que demandara un segundo caballo que llevaba para sus efectos.

    Llamamos entonces al piloto para comunicarle lo que acabábamos de resolver. Le manifesté que había escuchado su queja de que dejaríamos volverse solo pero que por el contrario estábamos dispuestos a no permitir tal cosa. Partiríamos con la caravana llevándolo en nuestra compañía, y, por lo tanto, deseábamos conocer su pensamiento.

    Movió la cabeza al oírme, declarando que era una larga travesía y que él carecía de pecune para efectuarla, lo mismo que para alimentarse una vez llegase a destino. Entonces le interrumpimos diciéndole que habíamos pensado en todo eso y decidido hacer algo que le probara cuan reconocidos le estábamos por los servicios prestados y a la vez cuánto nos agradaba su compañía. Pasamos a explicarle la resolución tomada poco antes, el dinero que le entregaríamos para que dispusiera de él a su voluntad, asegurándole que todos los gastos del viaje quedaban por nuestra cuenta si estaba dispuesto a acompañarnos, y que le dejaríamos sano y salvo —no mediando contingencias imprevisibles— en Moscú o Inglaterra, debiendo ocuparse él tan sólo del traslado de sus efectos.

    Escuchó mis palabras con la expresión de un hombre arrobado, y nos dijo que iría a nuestro lado hasta el fin del mundo. Nos dispusimos por tanto a iniciar juntos la travesía; nuestros preparativos corrieron parejos con los muchos otros comerciantes que se incorporaban a la caravana, pero en vez de estar listos en cinco semanas transcurrieron cuatro meses y varios días antes de que nos halláramos dispuestos a emprender el viaje.

    A comienzos de febrero (según nuestro calendario) salimos de Pekín. Mi socio y el anciano piloto habían regresado al puerto donde desembarcáramos primero a fin de vender algunos efectos que quedaron almacenados en él; entretanto, yo, acompañado de un comerciante chino a quien había conocido en Nankín, me encaminé a dicha ciudad, donde adquirí noventa piezas de finísimo damasco además de doscientas piezas de otras hermosas sedas de varias clases, algunas bordadas en oro, volviendo con todo ese cargamento a Pekín antes de que lo hiciera mi socio. Aparte de eso adquirimos gran cantidad de seda cruda y otras mercancías; solamente en esos géneros nuestro cargamento tenía un valor de tres mil quinientas libras esterlinas, lo cual, agregado a una cantidad de té, algunas indianas finas y tres camellos cargados de clavo y nuez moscada, formaba una caravana de dieciocho camellos aparte de los que montábamos nosotros. Sumando dos o tres caballos de refresco y dos cargados de provisiones, nuestra parte en la caravana estaba constituida por veintiséis camellos y caballos.

    La comitiva era muy grande y creo recordar que no bajaba de trescientos a cuatrocientos caballos, con más de ciento veinte hombres excelentemente armados y dispuestos a todo evento, ya que así como las caravanas del Oriente están sujetas a los ataques de los árabes, aquí el peligro lo constituyen los tártaros. No son, sin embargo, tan peligrosos como aquéllos, ni se muestran tan bárbaros después de lograr una victoria.

    La caravana estaba constituida por hombres de diversas nacionalidades, pero principalmente moscovitas, contándose unos sesenta comerciantes y vecinos de aquella ciudad, bien que algunos eran de Livonia. Para nuestra particular satisfacción encontramos a cinco escoceses que parecían ser comerciantes muy avezados y poseedores de gran fortuna.

    Al cumplirse la primera jornada los guías, que eran cinco, convocaron a todos los caballeros y comerciantes, es decir, a todos los viajeros menos sus criados, para celebrar lo que ellos llamaban un gran consejo. En dicho gran consejo cada uno entregó cierta cantidad de dinero para formar un fondo común destinado a pagar el forraje —que no hubiera sido posible obtener de otra manera—, así como el salario de los guías, compra de caballos y demás gastos por el estilo. Se procedió luego a «organizar las jornadas», según su expresión, esto es, a nombrar capitanes y oficiales que nos comandarían en caso de ataque, repartiéndose los distintos turnos de mando. Esta organización militar no era en lo más mínimo innecesaria ni superflua en semejante travesía, como pronto podrá comprobarse.

    El camino que seguimos nos hizo conocer un país densamente poblado por gentes que se dedican a la alfarería y también a preparar las tierras con que se hace la porcelana china. Mientras avanzábamos, nuestro piloto portugués, que siempre tenía algo divertido para decirnos, vino a mí con aire socarrón declarando que deseaba hacerme conocer la cosa más rara del país. Agregó que después del desagrado que yo manifestara tantas veces en la China, por lo menos tendría en compensación la oportunidad  de  ver  algo  único  en  el mundo.

    Me sentí lleno de curiosidad por saber qué era aquello, y por fin me reveló que se trataba de la residencia de un noble, construida íntegramente con productos de la China.

    —Pues bien —dije yo—, me imagino que todas las construcciones se harán aquí con los materiales que existen a mano. ¿Qué tiene eso de raro?

    —No me habéis entendido —replicó él—. Quiero decir que es una casa hecha con lo que en Inglaterra llamáis «producto de China», y que en nuestro país denominamos porcelana.

    — ¿Es posible? ¿Y qué tamaño tiene esa casa? ¿No podríamos llevarla dentro de una caja y a lomo de camello? Si es posible, la compraremos.

    — ¡A lomo de camello! —gritó el piloto, levantando las manos—. ¡Pero si en ella vive una familia de treinta personas!

    Me sentí todavía más deseoso de verla, y cuando por fin llegamos cerca encontré una casa de madera recubierta de yeso, como diríamos en mi país; sin embargo, aquel yeso era verdadera porcelana china o más bien una capa de la tierra especial que allí se usa para obtener la porcelana.

    El exterior, que brillaba al reflejar la luz solar, estaba esmaltado y tenía un hermosísimo aspecto de blancura deslumbrante; se veían figuras azules, idénticas a las que tienen las grandes porcelanas chinas que se llevan a Inglaterra, y parecían tan firmes como si hubiesen sido cocidas. En cuanto al interior, en lugar de frisos las paredes aparecían adornadas con hileras de mosaicos, pintados y cocidos, iguales a los que en mi país llamamos azulejos, pero hechos de la más fina porcelana y recubiertos de hermosísimas figuras que mostraban extraordinaria variedad de colorido en el cual no faltaba el oro; varios azulejos componían una imagen, y habían sido unidos con tal habilidad —empleando un mortero del mismo material— que no se advertían las junturas por más que se las buscara. Los pisos de la casa estaban hechos del mismo material y tan sólidos como los pisos de piedra cocida que usamos en muchos lugares de Inglaterra tales como Lincolnshire, Nottinghamshire y Leicestershire; los pisos eran sumamente lisos pero no cocidos ni pintados, excepto algunas habitaciones pequeñas íntegramente recubiertas de los mosaicos mencionados. Los techos y, en una palabra, toda la superficie de la casa, aparecían formados por el mismo material y hasta el tejado estaba recubierto de mosaicos, solamente que éstos eran de un negro brillante.

    En verdad que se trataba de una casa de porcelana; el nombre podía serle aplicado muy justamente, y si no hubiese tenido que proseguir viaje me habría agradado quedarme allí unos días para examinarla en detalle. Me aseguraron que tenía fuentes y piscinas íntegramente recubiertas de aquellos mosaicos, y que en los jardines había largas hileras de estatuas de porcelana.

    Como ésta es una de la particularidades más notables de la China, bien pueden permitirse destacar en ella, pero de lo que no tengo duda es que se destacan aún más en los relatos que sobre tal arte hacen, algunos tan increíbles que no me animo a transcribirlos en la seguridad de que son absolutamente falsos. Recuerdo que me dijeron, entre otras cosas, que un artista había hecho un barco de porcelana con todos sus aparejos, mástiles y velas del mismo material, lo bastante grande para contener cincuenta hombres. Si hubieran agregado que el barco, después de botado al agua, fue capaz de navegar hasta el Japón, entonces mi paciencia se habría concluido; pero como me daba perfecta cuenta de que aquel relato era una completa mentira de quien me lo hacía, me limité a sonreír sin decir una sola palabra.

    El curioso espectáculo me retrasó unas dos horas, quedando rezagado de la caravana, por lo cual el comandante de turno me multó en tres chelines, agregando que si nos hubiéramos encontrado a tres días de viaje más allá de la muralla en vez de hallarnos todavía dentro de sus límites, me hubiera aplicado una multa cuatro veces mayor y exigido que pidiera perdón ante el gran consejo. Prometí entonces ser más puntual, y en realidad pronto tuve motivos para comprender que nuestra seguridad exigía que permaneciéramos unidos durante la travesía.

    Dos días después atravesamos la Gran Muralla levantada por los chinos como fortificación contra los tártaros. Es aquélla una enorme construcción que se extiende sobre colinas y montañas siguiendo innecesariamente una ruta donde las rocas son tan abruptas y los precipicios tales que ningún enemigo podría pasarlos y en caso de ser capaz de hacerlo, no valdría una muralla para detenerlo. Nos aseguraron que su longitud es casi de mil millas inglesas pero que el país, en línea recta, mide unas quinientas, defendidas por aquella pared que describe numerosas curvas y sinuosidades. Tiene una altura de veinticuatro pies y en algunas partes alcanza igual espesor.

    Permanecí junto a la muralla por espacio de una hora, sin contravenir las órdenes porque la caravana empleó ese tiempo en cruzar los portones; la miraba por todos lados, de cerca y de lejos, hasta donde alcanzaba mi vista. Nuestro guía, que había alabado incesantemente la construcción, estaba ansioso por escuchar mi parecer y entonces declaré que la consideraba un excelente recurso para mantener alejados a los tártaros. No entendió el exacto sentido de mis palabras, pero las tomó por un cumplido, lo cual hizo reír a nuestro viejo piloto.

    — ¡Oh, señor inglés! —exclamó—. Vos habláis en colores.

    — ¡En colores! ¿Qué queréis decir con eso?

    —Que vuestras palabras son blancas en un sentido y negras en otro; parecen amables, pero son irónicas. Habéis dicho al guía que la muralla es buena para mantener alejados a los tártaros, de donde deduzco que queréis decir que solamente sirve para alejar a dicho pueblo, pero que de nada valdría frente a otro. Yo os comprendo bien, señor inglés —agregó—, pero el señor chino os entiende a su propia manera.

    —Pues bien —dije a mi vez—, ¿es que pensáis que esta muralla podría resistir a un ejército europeo con buena artillería, o que se mantendría en pie si fuera atacada por nuestros ingenieros secundados por dos compañías de minadores? ¿No creéis que la echarían abajo en menos de diez días para que el ejército invadiera el país, o bien la harían volar con cimientos y todo sin que quedara ni rastro de ella?

    — ¡Ah! —repuso él—. ¡Lo sé muy bien!

    El guía chino se mostraba sumamente ansioso por entender lo que yo acababa de decir, y autoricé a mi intérprete para que le tradujera mis palabras unos días más tarde cuando saliéramos de las fronteras de su país y se dispusiera a dejar la caravana. Cuando escuchó entonces mis manifestaciones se quedó muy callado durante el resto del trayecto, y, mientras estuvo con nosotros no volvimos a escucharle ningún relato sobre el poder de los chinos y sus grandezas.






	\chapter{Bandoleros e idolatras}





    Después de haber transpuesto aquella gran inutilidad llamada Muralla, que me recordaba la que los romanos habían construido en el Northumberland contra los pictos, empezamos a advertir que el país se despoblaba cada vez más y que las gentes parecían concentrarse en ciudades fortificadas a fin de estar al abrigo de los merodeos y las depredaciones de los tártaros, que se lanzan a robar en grandes números y no pueden ser contenidos por las pobres gentes del campo.

    Fue allí que advertimos la necesidad de mantenernos constantemente unidos, ya que divisamos muchas partidas de tártaros que nos acechaban. Sin embargo, cuando alcancé a observarlos con más detenimiento me asombré de que el Imperio chino hubiese podido alguna vez ser conquistado por aquellos despreciables individuos, mera horda de salvajes que atacan sin guardar orden alguno, ignoran toda disciplina y tácticas de guerra.

    Sus caballos son animales entecos y hambrientos, sin adiestramiento alguno y que no sirven de nada; nos convencimos de ello el primer día que les vimos, cosa que ocurrió después de haber penetrado en la parte más vasta del territorio. Ocurrió que nuestro jefe de turno concedió permiso a dieciséis de los nuestros para que fuésemos de caza, como llaman ellos al hecho de apoderarse de carneros y ovejas. Merece sin embargo el nombre de caza, porque aquellos animales son los más salvajes y veloces que haya yo visto, sólo que no pueden correr mucho tiempo y el cazador está seguro de alcanzarlos apenas principia la persecución, ya que para colmo se presentan en rebaños de treinta o cuarenta y, a la manera que es natural en las ovejas, se mantienen constantemente juntos en la huida.

    Mientras nos dedicábamos a esa rara cacería dimos con una partida de unos cuarenta tártaros. Ignoro si estaban también dedicados a la caza de carneros o andaban a la busca de otra clase de presa. Tan pronto nos vieron uno de ellos sopló con fuerza en una especie de cuerno que dejó escapar un sonido tan salvaje que jamás había yo escuchado nada parecido y que, dicho sea de paso, preferiría no volver a escuchar. Imaginamos que aquélla era una señal para que los otros tártaros se les reunieran, y así fue, pues en menos de diez minutos una partida de cuarenta o cincuenta hombres apareció a la distancia, mas cuando llegaron habíamos procedido ya en la forma que ha de verse. Se hallaba entre nosotros uno de los comerciantes escoceses de Moscú, y tan pronto escuchó la llamada del cuerno nos dijo que lo único que quedaba por hacer era cargar inmediatamente sobre los tártaros. Luego de reunimos en una línea nos preguntó si estábamos resueltos, y como le contestamos afirmativamente, partimos al galope contra el enemigo. Los tártaros se habían quedado contemplándonos como verdadera horda sin orden ni disciplina; tan pronto nos vieron avanzar hacia ellos dispararon sus flechas que, por fortuna, erraron el tiro. Parece que calcularon mal la distancia aunque no la dirección, ya que las flechas cayeron un poco más adelante de donde veníamos al galope, pero disparadas con tanto acierto que si nos hubiésemos encontrado veinte yardas más adelante muchos habrían resultado heridos y acaso muertos.

    Nos detuvimos al punto; y aunque gran distancia nos separaba de la partida, disparamos una descarga contra ella enviándoles balas de plomo a cambio de sus dardos, y luego nos precipitamos otra vez a la carga, sable en mano, pues así lo mandaba el valeroso escocés que nos dirigía. Aquel hombre era tan sólo un comerciante, pero se condujo con tal vigor y decisión, y se mostró a la vez tan frío y sereno, que jamás vi hombre más capacitado para ejercer el mando en tales circunstancias. Tan pronto estuvimos cerca disparamos nuestras pistolas a quemarropa y luego los atropellamos, pero ellos huyeron en la confusión más completa. Su única tentativa fue hecha sobre nuestra ala derecha, donde tres tártaros se detuvieron a combatir llamando a la vez a los otros para que volvieran; tenían una especie de cimitarra en la mano y los arcos les colgaban del hombro. Nuestro comandante, sin pedir a nadie que lo siguiera, galopó en línea recta hacia ellos, y mientras derribaba a uno del caballo con un culatazo del mosquete, mató al segundo de un pistoletazo, con lo cual el tercero salió huyendo y el combate terminó allí mismo.

    Desgraciadamente, mientras esto acontecía nuestros carneros habíanse perdido de vista. Salimos ilesos de la lucha, pero los tártaros tuvieron por lo menos cinco muertos. No sabíamos si otros estarían heridos, mas lo que vimos claramente fue que la segunda partida de tártaros se asustó tanto con el estampido de nuestras armas que no hizo la menor tentativa de cargar contra nosotros.

    Esto tuvo lugar mientras viajábamos aún en territorio chino, y por eso los tártaros no se mostraban tan osados como lo fueron más adelante. Cinco días después entramos en un inmenso y desolado desierto que atravesamos en una marcha de tres días con sus noches, viéndonos obligados a llevar agua suficiente en grandes odres y acampar por las noches tal como he oído que se acostumbra en los desiertos de Arabia.

    Pregunté a quién pertenecían aquellos dominios y me fue dicho que era una especie de frontera que podía ser llamada Tierra de Nadie. Formaba parte del Gran Karakatay o Gran Tartaria, región que aparecía como sometida a China, pero ésta no se preocupaba de limpiar de bandoleros aquella región que era considerada el peor desierto en el mundo entero, pese a que más tarde tuvimos que atravesar otros de mayor extensión.

    Mientras cruzábamos por aquellas soledades, que confesaré me impresionaron mucho al comienzo, vimos dos o tres veces algunas pequeñas partidas de tártaros dedicados al parecer a sus propios asuntos, y que no demostraron malas intenciones para nosotros. Como el hombre que se encontró con el diablo, si nada tenían ellos que decirnos nosotros les pagamos de igual modo y los dejamos en paz.

    Una vez, sin embargo, un grupo se acercó mucho a la caravana, deteniéndose luego a observarnos. No sabíamos si estaban considerando la conveniencia de atacarnos o no, por lo cual luego de desfilar cerca de ellos organizamos un grupo de cuarenta hombres para que constituyeran una retaguardia lista a todo evento, dejando que la caravana avanzara una media milla más allá. Poco después los vimos alejarse, pero sin embargo nos saludaron con cinco flechas, una de las cuales hirió malamente a un caballo al punto que nos vimos precisados a abandonar al pobre animal, quien hubiera necesitado la ayuda de un veterinario. Imaginamos que podrían dispararnos nuevas flechas con buena puntería, pero aquella vez no vimos más tártaros ni sus dardos.

    Transcurrió cerca de un mes en el curso del cual nuestro viaje se efectuó por caminos no tan malos como los anteriores, aunque todavía estábamos en los dominios del Emperador de China. Casi todos los caminos pasaban por pueblos fortificados contra las incursiones de los tártaros. Al llegar a una de esas poblaciones (a dos jornadas y media de nuestra próxima etapa, la ciudad de Naun) quise comprar un camello que frecuentemente son ofrecidos en venta a lo largo de esa ruta, así como caballos, ya que muchas caravanas que atraviesan la región los tornan necesarios. La persona a quien pedí que me procurara el animal se dispuso a ir en persona a buscarlo, pero yo, como un insensato, me empeciné en ir con él. El lugar se encontraba a unas dos millas fuera del pueblo, y era allí donde al parecer "mantenían los camellos y caballos protegidos por una guardia.

    Nuestro viejo piloto se agregó a mí y fuimos los tres a pie, ansiando cambiar un poco nuestro modo de movernos. Llegamos al lugar que era un terreno deprimido y pantanoso, cercado con una pared de piedras amontonadas sin mortero o argamasa, y con el aspecto de un cuartel; en la entrada había una pequeña guardia de soldados chinos. Elegimos el camello, y luego de convenir el precio nos volvimos, llevando al animal el chino que había venido con nosotros, cuando he aquí que de repente aparecieron cinco tártaros a caballo. Dos de ellos se precipitaron sobre el chino y le arrebataron el camello, mientras los otros tres nos encaraban al piloto y a mí viendo que estábamos desarmados. En efecto, yo no tenía otra arma que mi espada, que muy poco podía defenderme contra tres jinetes. Alzándola, sin embargo, contuve al primero de los atacantes, que retrocedió, por cuanto son insignes cobardes, pero el segundo, atacándome de lado, me descargó tal golpe en la cabeza que no supe más hasta recobrar el sentido tiempo después, sin alcanzar a comprender qué me había pasado; el hecho es que me desplomé pesadamente en el suelo.

    El anciano piloto, eficiente como en todas las ocasiones, tenía una pistola en su bolsillo, cosa que ni los tártaros ni yo habíamos sospechado; bien que de haberlo sabido no nos hubiesen acometido, pues los cobardes son sólo temerarios cuando no hay peligro.

    Viéndome caer desvanecido, el anciano se precipitó furiosamente sobre el bandolero que me había golpeado, y aferrándole el brazo con una mano, hizo tal presión que lo obligó a inclinarse hacia él, disparándole entonces un pistoletazo en la cabeza que le dejó muerto en el acto. Volviéndose al tártaro que nos había detenido, y antes de que pudiera reaccionar de su estupor y atacarlo, le soltó un terrible mandoble con una cimitarra que llevaba siempre consigo, y aunque no acertó a darle asestó un tajo en la cabeza de su caballo, cortándole de raíz una oreja así como un gran trozo de carne. El pobre animal, enloquecido por la herida, se desbocó inmediatamente, aunque su jinete estaba firme en la silla, lanzándose al galope con tal rapidez que el piloto no pudo herir por segunda vez. A cierta distancia, luego de levantarse sobre las patas traseras, el caballo terminó por despedir de su montura al tártaro y caer sobre él.

    En este intervalo, el pobre chino, que había perdido el camello y carecía de armas, vino corriendo hacia nosotros. Viendo entonces al tártaro caído y al caballo que lo aplastaba, corrió hacia él y quitándole una tosca arma que aquél llevaba a la cintura, especie de hacha aunque en realidad no merecía tal nombre, la alzó sobre su enemigo hendiéndole el cráneo de un golpe. Entretanto nuestro piloto se las entendía con el tercer tártaro: al ver que no se escapaba como había esperado, pero que tampoco venía a provocar lucha sino que permanecía inmóvil y como a la espera, el anciano hizo lo mismo, pero apresurándose entretanto a cargar otra vez su pistola. No bien el tártaro advirtió su movimiento, ya fuera porque se asustó pensando que se trataba de otra arma cargada o por otra cosa, el hecho es que huyó desalado dejando al piloto, mi campeón como me complací en llamarle más tarde, dueño del campo y victorioso.

    Para entonces yo estaba volviendo en mí. Mi primera idea al despertar fue que había estado durmiendo apaciblemente, pero luego me pregunté qué había ocurrido y por qué estaba tirado en el suelo. Ya recobrados los sentidos, sentí agudo dolor aunque no alcanzaba todavía a localizarlo. Me llevé la mano a la cabeza y la retiré ensangrentada, sintiendo que el dolor crecía; entonces, casi instantáneamente, recobré la memoria y supe todo lo que había acontecido.

    Salté sobre mis pies buscando la espada, pero ya no había enemigos a la vista. Encontré un tártaro muerto y su caballo muy quieto a su lado, y mirando más lejos alcancé a ver, mi campeón y salvador, quien luego de inspeccionar la labor del chino, retornaba con su cimitarra en la mano. Viéndome de pie, el anciano vino corriendo y me abrazó con inexplicable alegría porque había temido que yo hubiera muerto. Al ver la sangre, pensó que estaba herido, pero pronto comprendimos que no era gran cosa, sino lo que habitualmente se llama la cabeza rota. Ni siquiera más tarde me sentí demasiado dolorido por el golpe o el lugar donde lo recibiera, y en dos o tres días estuve perfectamente bien.

    No obtuvimos gran provecho de aquella victoria, pues al fin y al cabo perdimos un camello. Lo más notable fue, sin embargo, que al retornar al pueblo el hombre quiso recibir el valor del camello. Yo me negué a pagárselo y fue necesario comparecer ante el juez chino del lugar, lo que en Inglaterra llamaríamos acudir al juez de paz. Para ser justo con él debo admitir que se condujo con gran prudencia e imparcialidad, pues luego de escuchar a ambas partes preguntó gravemente al chino que había ido conmigo a comprar el camello si era mi sirviente.

    —No soy sirviente —repuso él—, pero acompañé al extranjero.

    — ¿A pedido de quién?

    —A pedido del extranjero.

    —Entonces —sentenció el juez—, tú eras el sirviente del extranjero en ese momento, y si el camello fue entregado al sirviente fue por tanto entregado al amo, quien debe pagar por él.

    Confieso que la cosa estaba clara y que no cabía decir una palabra en contra. Admirando tan justa manera de reaccionar, así como la imparcialidad del juicio, pagué con buena voluntad el camello y mandé a buscar otro. Observad que he dicho mandé. No fui en persona a elegir otro animal; me bastaba con lo sucedido.

    La ciudad de Naun está en la frontera del imperio chino. La llaman una ciudad fortificada, y en cuanto a las necesidades de aquellas tierras lo es ciertamente, ya que puedo afirmar sin temor que todos los tártaros de Karakatay —algunos millones según creo— la atacarían en vano con sus arcos y flechas. Pero en cuanto a su solidez en el caso de ser sitiada con cañones, provocaría la risa de aquellos que entienden de la materia. Nos encontrábamos, como ya he dicho, a unas dos jornadas de aquella ciudad cuando nos alcanzaron los mensajeros enviados ex profeso a todos los puntos del camino para advertir a los viajeros y sus caravanas que se detuvieran a esperar una escolta que se les enviaría, ya que un ejército de tártaros que alcanzaba el insólito número de diez mil hombres acababa de ser visto por aquellos lados, a unas treinta millas más allá de la ciudad.

    Muy malas noticias eran aquéllas para nosotros, pero significaban un atinado proceder por parte del gobernador y nos alegramos mucho sabiendo que seríamos protegidos por una guardia. Dos días más tarde, en efecto, recibimos doscientos soldados enviados desde una guarnición china situada a nuestra izquierda, y luego otros trescientos provenientes de Naun. Protegidos por ellos avanzamos sin temor, precedidos por los soldados de Naum, con una retaguardia constituida por los doscientos hombres de la guarnición y nosotros formando las alas a fin de que la caravana estuviese rodeada por todas partes. Dispuestos a la batalla, nos sentíamos capaces de pelear contra los diez mil tártaros mongoles si les daba por enfrentarnos, pero cuando aparecieron al día siguiente vimos que las cosas eran muy distintas.

    Era de mañana temprano cuando, al salir de un pueblecito muy bien situado llamado Changu, tuvimos que vadear un río. Si los tártaros hubieran sabido algo de táctica. Hubiesen podido atacarnos entonces con ventaja, aprovechando el momento en que la caravana había cruzado el río y la retaguardia estaba aún en la otra orilla; sin embargo, no vimos huellas de ellos.

    Tres horas más tarde, entrando en un desierto de unas quince millas o más de extensión, vimos repentinamente la nube de polvo que levantaban nuestros enemigos aproximándose a galope tendido y espoleando sus cabalgaduras.

    Los chinos que formaban la vanguardia, y que tan jactanciosos se habían mostrado el día anterior, empezaron a vacilar y mirar con frecuencia hacia atrás, lo que en un soldado constituye invariablemente signo de que se apresta a emprender la fuga. El anciano piloto era de mi parecer, y viéndome cerca declaró:

    —Señor inglés, es necesario dar ánimo a esos soldados o serán los causantes de nuestra ruina. Estoy seguro de que si los tártaros cargan sobre nosotros, no esperarán a recibir el choque.

    —Pienso como vos —repuse—, ¿pero qué podemos hacer?

    — ¿Hacer? Todo está en enviar a cincuenta de los nuestros a que flanqueen cada ala e infundan coraje. Si se sienten acompañados por hombres valerosos pelearán bien, pero de lo contrario darán en seguida la espalda al enemigo.

    Galopé hacia donde estaba nuestro jefe y le transmití lo que acababa de ocurrírsenos; como le pareciera bien, cincuenta de los nuestros se colocaron sobre el ala derecha y otros tantos en la izquierda, mientras el resto constituía una línea de reserva. Así avanzamos, dejando a los otros doscientos chinos que constituyeran un segundo ejército destinado a proteger los camellos. Si era necesario, cien de aquellos hombres acudirían a reforzar nuestra reserva de cincuenta hombres.

    Los tártaros avanzaron en hordas innumerables, cuyo número no podría dar aunque era por lo menos de diez mil. Una patrulla avanzó en reconocimiento de nuestras líneas y atravesó el terreno frente a nosotros. Al advertir que estaban a distancia de tiro, nuestro jefe ordenó que las dos alas avanzaran con suma rapidez y les hicieran una descarga cruzada, lo cual se efectuó de inmediato. Se alejaron entonces a todo galope, probablemente para informar a los otros de la recepción que acababa de serles brindada; y no me cabe duda de que ese saludo les enfrió notablemente la sangre, pues el ejército hizo alto como para deliberar, y dando después media vuelta abandonó su designio y no supimos más de él. Es de imaginar la alegría que nos causó semejante retirada, ya que nos habíamos sentido muy poco seguros de nuestras probabilidades contra un número tan abrumador de enemigos.

    Dos días después arribamos a la ciudad de Naun o Naum. Agradecimos al gobernador el cuidado que había tenido de nosotros e hicimos una colecta por valor de unas cien coronas que repartimos entre los soldados que nos habían escoltado, quedándonos todo un día en el lugar. Se trataba de una guarnición donde se concentraban novecientos hombres y la razón de tal defensa era que antaño las fronteras moscovitas se encontraban mucho más cercanas al fuerte que en la actualidad. Parece que los rusos abandonaron más tarde aquellos territorios en una extensión de doscientas millas al oeste de Naun por considerarlos desolados e impropios para los cultivos, fuera de que su alejamiento los tornaba difíciles de defender; conviene decir aquí que aún nos hallábamnos a más de dos mil millas de la Moscovia propiamente dicha.

    Siguiendo el viaje, cruzamos varios grandes ríos y dos horrorosos desiertos, uno de los cuales insumió dieceséis días de viaje, mereciendo como he dicho que se llamara la Tierra de Nadie. El 13 de abril llegamos por fin a las fronteras del dominio moscovita. Creo que la primera ciudad, pueblo o fortaleza —como quiera llamársele— perteneciente al Zar de Moscovia era el llamado Argunsk, en la orilla izquierda del río Argun.

    No pude menos de manifestar la profunda satisfacción que me causaba haber llegado por fin a un país de cristianos o, por lo menos, a un país gobernado por cristianos. Cierto que en mi opinión apenas merecen los moscovitas tal denominación, aunque pretendan serlo y a su manera se muestren sumamente devotos.

    Saludé entonces al bravo comerciante escocés de quien he hablado más arriba y tomándole la mano exclamé:

    — ¡Bendito sea el Señor! ¡Por fin estamos otra vez entre cristianos!

    Sonriéndose, me contestó:

    —No os regocijéis tan pronto, compatriota. Estos moscovitas son una rara especie de cristianos. Ya veréis que aparte del nombre, pasarán varios meses de viaje sin que descubráis el espíritu del cristianismo en esta tierra.

    —De todas maneras —observé—, mejor es eso que el paganismo y la adoración de demonios.

    —Os diré —declaró mi compañero—, que exceptuando a los soldados rusos de las guarniciones, así como algunos habitantes de las ciudades que encontraremos de paso, todo el resto del país en una extensión superior a mil millas en redondo está poblado por los más ignorantes y peores paganos imaginables.

    Y así era, efectivamente.

    Seguimos avanzando más allá del río Argun en cómodas jornadas, y tuvimos oportunidad de mostrarnos agradecidos al Zar de Moscovia por el cuidado puesto en levantar ciudades e instalar guarniciones en todos los lugares posibles, desde los cuales los soldados mantienen la vigilancia a semejanza de los fortines puestos antaño por los romanos en los remotos rincones de su imperio —incluso algunos en Gran Bretaña para seguridad del comercio y el alojamiento de los viajeros. Así empecé a advertir lo que me habían prevenido; a cualquier sitio que llegásemos, bien que en los pueblos y las guarniciones tanto el gobernador como los soldados eran rusos y cristianos, el resto de los habitantes profesaba el paganismo, hacía sacrificios a los ídolos y adoraba al sol, la luna y las estrellas, o a todos los astros del cielo. Pienso que de todos los paganos que me haya sido dado conocer éstos eran los más bárbaros, sólo que no comían carne humana como los salvajes americanos.

    Algunas muestras de esa barbarie pudimos encontrar en el territorio situado entre Argunsk —donde penetramos en los dominios moscovitas— y una ciudad a la vez rusa y tártara llamada Nertchinsk, territorio en el que hay grandes desiertos y bosques que nos llevaron veinte días hasta recorrerlos. En un pueblo cercano a la segunda ciudad nombrada sentí curiosidad por averiguar la manera de vivir de aquellas gentes, que es por cierto la más brutal e insoportable que pueda imaginarse. Parece que aquel día iban a celebrar un sacrificio; encontramos, puesto sobre un tronco de árbol, el más horrible ídolo concebible. Era de madera, tan espantoso como el diablo, o por lo menos como llegamos a imaginarnos su fealdad. Tenía una cabeza que apenas se asemejaba a la forma humana; orejas iguales a cuernos de macho cabrío; ojos del tamaño de las monedas que llamamos coronas; una nariz curvada como un cuerno retorcido y la boca, abierta en forma de cuadrilátero a la manera de un león, con horribles dientes curvos como el pico de los loros. Aparecía vestido con las ropas más inmundas que imaginarse pueda; tenía una especie de chaqueta de piel de oveja con la lana hacia afuera, y en la cabeza un gran gorro tártaro del cual salían dos cuernos. Medía ocho pies de alto y no tenía pies o piernas, así como ninguna proporción comparable a un cuerpo humano.

    Este espantapájaros había sido emplazado en las afueras del villorrio, y cuando me acerqué había cerca de él dieciséis o diecisiete individuos que no sé si eran hombres o mujeres, porque no hacen distinción en sus ropas ni en sus cabellos; yacían tirados de boca contra el suelo, en torno al monstruoso ídolo, y no vi que ninguno hiciera el más mínimo movimiento como si fuesen pedazos de madera igual que su ídolo. Hasta llegué a pensarlo seriamente, pero al acercarme un poco más se enderezaron de pronto lanzando una especie de aullido semejante al de una jauría de mastines, tras lo cual se alejaron como si los ofendiera que los hubiésemos molestado. Poco más allá del ídolo, y en la puerta de una especie de choza o tienda hecha con pieles de oveja y de cabra, vimos a tres carniceros, o lo que nos parecieron tales. Cuando nos acercamos algo más notamos que tenían largos cuchillos en las manos y en el centro de la tienda alcanzamos a distinguir tres ovejas degolladas así como un novillo. Parece que tales eran los sacrificios que hacían al horroroso ídolo, y aquellos tres hombres eran sus sacerdotes.

    En cuanto al grupo de individuos posternados, se trataba de los que habían ofrecido el sacrificio y al llegar nosotros estaban entregados a sus plegarias al ídolo.

    Confieso que me sentí más indignado al ver tanta estupidez y advertir la torpe adoración de aquel espantajo que todo cuanto viera antes en mi vida; cabalgando entonces hacia la imagen de aquel monstruo o como se quiera llamarle partí en dos pedazos con mi espada el bonete que tenía en la cabeza, de tal modo que le quedó colgando de los cuernos. Uno de los hombres que venía conmigo aferró entonces la chaqueta de piel de oveja que cubría al ídolo y se la arrancó a tirones en el mismo instante en que un horrible clamoreo se difundía por todo el villorrio y por lo menos trescientos habitantes se precipitaban en nuestra dirección; nos apresuramos, pues, a ponernos a salvo, observando que muchos estaban armados de arco y flechas, pero con toda la intención de hacerles una nueva visita.

    Nuestra caravana permaneció tres días en el pueblo, que estaba a cuatro millas del lugar, a fin de proveerse de algunos caballos que nos hacían falta, ya que el mal estado de los caminos y el cruce de los desiertos habían estropeado muchas cabalgaduras. Me quedaba suficiente tiempo para poner en ejecución mis designios y comuniqué mi proyecto al comerciante escocés de cuyo coraje tenía testimonio suficiente como he narrado más arriba. Le dije lo que había visto y la indignación que me causaba pensar que la naturaleza humana pudiera llegar a semejante grado de degeneración. Afirmé que estaba resuelto, siempre que encontrase a cuatro o cinco hombres bien armados dispuestos a secundarme, a volver para destruir aquel vil y abominable ídolo, demostrando así a las gentes que ni siquiera tenía poder para defenderse a sí mismo y mucho menos merecía que lo adorasen, le elevaran plegarias y le rindiesen homenajes y sacrificios.

    Al oírme, mi interlocutor se echó a reír.

    —Vuestro celo es excelente —me dijo—, ¿pero qué os proponéis con esa expedición?

    — ¡Qué me propongo! —exclamé—. ¡Pues vindicar el honor de Dios que es insultado por esa diabólica adoración!

    — ¿Y cómo podéis vindicar el honor de Dios si esa gente no llega a darse cuenta de la intención que os ha movido, salvo que pudierais hablarles y convencerles? Aun así os atacarán y os batirán, porque os aseguro que son gentes resueltas, especialmente cuando del culto de sus dioses se trata.

    — ¿No podemos llevar a cabo mi proyecto durante la noche —pregunté— y dejarles por escrito las razones que nos han guiado, redactadas en su propio idioma?

    — ¡Por escrito! —exclamó el escocés—. ¡Pero si no hay un solo hombre en cinco naciones de las suyas que sea capaz de entender una carta ni leer una línea en cualquier lenguaje y menos en el suyo!

    — ¡Maldita ignorancia! —exclamé—. Y sin embargo, señor, me siento resuelto a ejecutar mi plan. Tal vez el instinto o la naturaleza los lleve a deducir de lo sucedido su propia ignorancia y la brutalidad en que están sumidos al adorar tan horrible ídolo.

    —Considerad esto, señor —me dijo entonces mi interlocutor—. Si vuestro celo os mueve en tal forma llevad a ejecución vuestro designio, pero es mi deber advertiros que esos pueblos o naciones se encuentran bajo el dominio del Zar de Moscovia y reducidos por la fuerza; si hacéis semejante cosa con ellos, hay diez probabilidades contra una de que acudan al gobernador de Nertchinsk y luego de quejarse exigirán satisfacción; si él se niega a darla, las mismas probabilidades existen de que se produzca una revuelta que conduzca a una nueva guerra contra todos los tártaros de estas regiones.

    Admito que sus palabras me obligaron a pensar más serenamente la cosa, pero una y otra vez volvía a tocar la misma cuerda y durante todo el día me sentí deseoso de llevar a la práctica mi proyecto. Hacia la noche, el comerciante escocés me encontró accidentalmente mientras paseábamos por el pueblo, y pidió hablar conmigo.

    —Mucho me temo —dijo— haberos disuadido de vuestra idea, y eso me ha preocupado todo el día, ya que aborrezco los ídolos y la idolatría tanto como vos.

    —En verdad —repuse— que me habéis contenido algo en mis ansias de llevar a cabo esa idea, pero no del todo, pues sigo creyendo que intentaré la aventura antes de que salgamos del pueblo, aunque más tarde me entreguen a ellos para aplacar su rabia.

    — ¡Oh, no! —exclamó él—. ¡Dios no quiera que seáis jamás entregado a semejante caterva de monstruos! Pero no lo harán, ciertamente, porque equivaldría a que os asesinaran.

    — ¿De  veras?   ¿Y  qué  haría  esa  gente  conmigo?

    — ¿Qué haría? —dijo él—. Os contaré cómo trataron a un pobre ruso que también les reprochó como vos su idolatría y a quien apresaron después de herir de un flechazo, impidiéndole que escapara. Luego de desnudarlo y amarrarlo fuertemente lo subieron a lo alto de su ídolo, lo rodearon y se pusieron a disparar sobre él tantas flechas como podía contener su cuerpo. Más tarde lo quemaron, sin sacarle antes las flechas, a manera de sacrificio al ídolo.

    — ¿Era ese mismo ídolo?

    —Sí, el mismo.

    —Pues bien —dije yo—, voy a relataros una historia.

    Le conté lo ocurrido a nuestros hombres de Madagascar, y cómo incendiaron y saquearon una aldea indígena, asesinando a hombres, mujeres y niños por la muerte de uno de ellos. Cuando hube terminado la narración, opiné que la misma cosa debía hacerse con ese villorrio.

    Escuchó mi relato atentamente, pero al oírme decir que lo mismo merecían los de ese poblado, respondió:

    —Os equivocáis grandemente, porque lo que os he dicho no sucedió aquí sino a casi cien millas de este lugar. Sin embargo, se trataba del mismo ídolo, ya que lo llevan en procesión de pueblo en pueblo.

    —Perfectamente —dije—. Entonces hay que castigar al ídolo y creedme que lo haré si alcanzo a vivir esta noche.

    Al verme tan resuelto, terminó por aprobar mi plan y me rogó que no fuese solo, ofreciéndose a acompañarme con otro de sus compatriotas, individuo enérgico y valeroso.

    —Es un hombre —agregó— tan notable por su celo religioso como el mejor que podríais imaginar para que os secunde en contra de esas demoníacas adoraciones paganas.

    Me presentó a su camarada, un escocés llamado capitán Richardson, a quien hice el fiel relato de todo cuanto había presenciado y también de lo que intentaba poner en práctica. El capitán me dijo que iría conmigo, aunque le costara la vida, y por fin decidimos hacer la expedición los tres solos. Yo lo había propuesto antes a mi socio, pero se rehusó diciéndome que estaba dispuesto a acompañarme hasta el fin cuando se tratara de defender mi vida, pero que esa aventura estaba fuera de su línea de conducta. Por lo tanto, nos resolvimos a ir solos, llevando también a mi criado, y poner mi designio en práctica esa noche a las doce, con todo el secreto necesario.

    Más tarde, sin embargo, después de reflexionarlo mejor, decidimos aplazar la empresa hasta la noche siguiente, ya que la caravana volvería a ponerse en marcha por la mañana y suponíamos que el gobernador, sabiéndonos lejos y fuera del alcance de su poder, no se animaría a entregarnos para satisfacción de las gentes. El comerciante escocés, que se mostraba ahora tan resuelto en el proyecto como en su ejecución, me trajo ropas a la usanza tártara, hechas de piel de oveja, con un gorro, arco y flechas, mientras obtenía iguales atavíos para él y su compañero, a fin de que nadie pudiera deducir por nuestro aspecto a qué nacionalidad pertenecíamos.

    Pasamos la primera noche haciendo una mezcla con algunas sustancias combustibles, aguardiente, pólvora y otros ingredientes. Habiéndonos provisto de cierta cantidad de brea en un recipiente, una hora después del anochecer salimos rumbo al lugar convenido.

    Llegamos al sitio a eso de las once de la noche, notando que las gentes no parecían en lo más mínimo inquietas por la seguridad de su ídolo. Era una noche nublada, aunque la luna alcanzaba a verter suficiente luz para mostrarnos el emplazamiento del ídolo, en el mismo sitio y forma en que lo viéramos la primera vez. Todos parecían haberse ido a dormir, pero en la gran choza o tienda donde viéramos a los tres sacerdotes que confundimos con carniceros observamos que había luz y al acercarnos más escuchamos voces como de unas cinco o seis personas que dialogaban.

    Si procedíamos a aplicar nuestra pez griega al ídolo para incendiarlo, los moradores de la choza saldrían inmediatamente a fin de salvar la imagen de su destrucción. ¿Cuál sería entonces nuestra actitud? Pensamos al principio transportar al ídolo más lejos para incendiarlo con seguridad: pero al tratar de levantarlo encontramos que pesaba demasiado y nos quedamos llenos de perplejidad. El segundo escocés era de opinión que incendiáramos la choza y cuando sus moradores se lanzaran fuera los golpeáramos hasta privarlos del conocimiento. Yo me manifesté contrario a dicho plan, pues me repugnaba la idea de matar alguno pudiendo evitarlo.

    —Pues bien —dijo entonces el comerciante escocés—, ésta es mi idea: trataremos de tomarlos prisioneros, atarles fuertemente las manos a la espalda y obligarlos a que presencien la destrucción de su ídolo.

    Por fortuna, teníamos suficiente bramante o cuerda fina con la cual habíamos atado nuestras materias incendiarias, de manera que nos dispusimos a atacar a aquellos hombres con el menor ruido posible. Golpeamos la puerta y las cosas ocurrieron a gusto nuestro, pues uno de los sacerdotes acudió a abrir. Nos apoderamos de él al punto tapándole la boca y le*atamos las manos a la espalda arrastrándolo hasta donde se hallaba el ídolo y allí procedimos a amordazarlo para que no pudiese exhalar el menor quejido, atándole los pies y abandonándolo en el suelo.

    Dos de nosotros permanecimos en la puerta esperando que otro saliera a averiguar qué había pasado, pero como transcurría el tiempo y nuestro tercer compañero había vuelto a reunirse con nosotros, otra vez llamamos suavemente a la puerta. Dos de ellos salieron juntos e hicimos la misma cosa a sus expensas, viéndonos sin embargo obligados a llevarlos entre los tres para dejarlos amordazados junto al ídolo; cuando volvimos encontramos a dos hombres a la puerta y un tercero entre ellos, pero más adentro. Sujetamos a los dos primeros, atándoles rápidamente, mas el tercero alcanzó a echarse atrás y exhalar un grito, al tiempo que el comerciante escocés entraba tras él y encendiendo una composición que habíamos hecho y que servía para producir un espeso humo maloliente, la arrojó dentro de la choza. Entretanto, el otro escocés y mi criado, encargándose de los dos prisioneros ya bien atados, los llevaban en dirección al ídolo y los dejaban allí para que se maravillaran de que su dios no acudiera a salvarlos; de inmediato los nuestros se nos reunieron en la choza.

    La mixtura que arrojáramos había llenado la choza con tanto humo que sus ocupantes estaban ya medio sofocados; tiramos entonces al interior un saquito de cuero cuyo contenido se inflamó como una vela haciéndonos ver a cuatro personas, dos hombres y dos mujeres, que por lo visto habían estado celebrando alguno de sus diabólicos sacrificios. Tan aterrados quedaron que parecían muertos, quietos en el suelo y temblando como azogados, incapaces de hablar a causa del humo que los ahogaba.

    Los apresamos sin dificultad, atándolos igual que a los restantes y sin el menor ruido. Debo agregar que antes de eso los habíamos sacado de la choza, pues apenas podíamos resistir la sofocación del humo. Hecho esto los llevamos al sitio donde estaba el ídolo y nos pusimos a trabajar en la imagen. Ante todo la untamos, ropas incluidas, con pez y otra sustancia que habíamos traído y que era una mezcla de sebo y azufre. Luego de rellenar los ojos, oídos y boca del ídolo con buena cantidad de pólvora, envolvimos finalmente el gorro o bonete con una gran porción de pez griega. Reuniendo en seguida todas las sustancias combustibles que nos quedaban, miramos en derredor en busca de otros elementos para formar una hoguera, hasta que mi criado recordó haber visto en la choza un haz de forraje seco, no recuerdo si paja o juncos. En compañía de uno de los escoceses fueron a buscarlo, volviendo con grandes brazadas que dispusimos a cierta distancia. Entonces, después de desatar los pies a los prisioneros y quitarles las mordazas, los trajimos cerca de su ídolo y le pegamos fuego.

    Transcurrió un cuarto de hora hasta que la pólvora de los ojos, boca y orejas hizo explosión, deformando todo rasgo de la imagen y dejando al ídolo convertido en un humeante tronco de madera. Acercamos entonces el forraje seco para que terminara de consumirlo, y pensamos emprender la retirada, pero el comerciante escocés nos detuvo diciéndonos:

    —No, debemos quedarnos; de lo contrario esos pobres infelices, en su desesperación, se arrojarán a la hoguera para quemarse junto con su ídolo.

    Permanecimos, pues, montando guardia hasta que sólo hubo cenizas, recién nos marchamos dejando en libertad a aquellas gentes.

    Por la mañana nos incorporamos a nuestros compañeros, que estaban muy ocupados ultimando los preparativos para la jornada, y nadie hubiese podido sospechar que habíamos pasado la noche en otro lugar que nuestros lechos, donde es de imaginarse que se quedan los viajeros cuyas fuerzas serán empleadas en la jornada diurna.






	\chapter{Invernada en Siberia}





    Aquel curioso episodio no terminó allí. Al siguiente día una gran multitud compuesta no solamente de campesinos sino por los habitantes de numerosísimos poblados cercanos, se presentó a las puertas del pueblo y del modo más insultante exigió al gobernador ruso plena satisfacción por la afrenta hecha a sus sacerdotes y la destrucción de su gran Cham-Chi-Thaungu, nombre que daban a la monstruosa imagen objeto de su culto. Los habitantes de Nertchinsk se mostraron consternados al ver esto, pues aseguraban que los tártaros no bajaban de treinta mil y que en pocos días alcanzarían a reunir cien mil hombres armados.

    El gobernador ruso envió representantes para que aplacaran los ánimos y se apresuró a decir a los peticionarios las palabras más amables que puedan imaginarse. Les aseguró que ignoraba lo ocurrido, que ni un alma en su guarnición había estado ausente de ella y por lo tanto nadie podía allí ser considerado culpable, pero que si ellos descubrían al causante de la injuria estaba dispuesto a castigarlo severamente.

    A esto le contestaron con altanería que el país entero reverenciaba al gran Cham-Chi-Thaungu, morador del sol, y que ningún mortal se hubiese atrevido a cometer semejante ofensa salvo algunos cristianos descreídos —como les llamaban habitualmente según supe luego—. Por lo tanto, agregaron, se consideraban a partir de ese momento en guerra contra todos los rusos, que eran cristianos y en consecuencia descreídos.

    Empleando toda su paciencia, y sin querer un rompimiento que fuera causa de guerra con aquellos tártaros (ya que el zar le había encargado que gobernara aquel país conquistado con toda la bondad posible), el gobernador insistió en sus amables palabras y por fin terminó diciéndoles que una caravana había salido aquella mañana rumbo a Rusia, y tal vez entre sus componentes se encontrara el culpable del ultraje, por lo cual ordenaría una investigación si eso los satisfacía. Sus manifestaciones aplacaron algo a aquellas gentes, y el gobernador envió mensajeros para que nos enteraran de lo acontecido haciéndonos saber que si alguien en la caravana era el culpable debía apresurarse a emprender la fuga, pero aun cuando nadie tuviese relación con aquel episodio nos aconsejaba apresurarnos lo más posible mientras él se las arreglaba para entretener a los enfurecidos habitantes.

    La conducta del gobernador fue generosa en extremo, mas cuando los mensajeros arribaron con sus noticias nadie entre nosotros sabía nada de lo sucedido, y en lo que respecta a los verdaderos culpables no eran objeto de la menor sospecha y no se les preguntó nada. El jefe de la caravana recogió, sin embargo, la advertencia del gobernador y nos ordenó apresurar la marcha de modo que anduvimos dos días con sus noches sin hacer ninguna parada de importancia, hasta que por fin arribamos a un pueblo llamado Plotbus. Al otro día de dejarlo a nuestra espalda, algunas nubes de polvo que se advertían a la distancia señalaron claramente la presencia de perseguidores.

    Habíamos entrado en el desierto, pasando por un gran lago llamado Schaks-Oser, cuando divisamos un gran cuerpo de caballería en la orilla del lago que mira hacia el norte, mientras nosotros seguíamos con rumbo al oeste. Notamos que también ellos tomaban dicho rumbo, aunque habían supuesto que seguiríamos por la orilla norte mientras afortunadamente para nosotros, preferimos la otra. Por espacio de dos días no los volvimos a ver.

    Al tercer día se dieron cuenta de su error o bien alguien los enteró de nuestra posición, pues al atardecer los vimos avanzar al galope contra nosotros.

    Nos tranquilizó sin embargo haber encontrado un excelente lugar para construir un campamento y pasar la noche, pues aunque estábamos en los comienzos del desierto sabíamos que en una extensión de más de quinientas millas careceríamos de toda población donde alojarnos.

    Nadie, salvo nosotros tres, conocía el verdadero motivo de que fuésemos así perseguidos, pero como aquel desierto es merodeado frecuentemente por partidas de tártaros mongoles las caravanas no dejan nunca de precaverse contra un posible ataque nocturno de su parte, y por la frecuencia con que nos había ocurrido algo parecido tales precauciones no sorprendían a nadie.

    Acampamos, pues, a fin de pernoctar, pero antes de haber completado nuestros preparativos el enemigo se presentó. No se lanzaron sobre nosotros como bandidos —que era lo que esperábamos— sino que enviaron tres mensajeros para exigir la entrega de los hombres que habían ofendido a los sacerdotes y quemado al dios Cham-Chi-Thaungu, a fin de castigarlos a su vez con la muerte de fuego. Afirmaron que a cambio de dicha entrega se alejarían sin dañar a nadie, pero de lo contrario asaltarían la caravana y le pegarían fuego.

    Nuestros compañeros se miraron angustiados al escuchar aquellas amenazas y empezaron a observarse unos a otros buscando algún rostro en el cual se delatara la culpabilidad. Pero la repuesta fue «nadie»; nadie había cometido aquella ofensa.

    El jefe de la caravana afirmó bajo palabra que estaba seguro de la inocencia de todos los miembros de la expedición ya que se trataba de pacíficos comerciantes que viajaban por negocios; les aseguró que no habíamos hecho daño ni a ellos ni a nadie y que por lo tanto era mejor que buscasen a los culpables en otro lado pues con nosotros se equivocaban. Por fin les hizo saber que no deberían molestarnos más, y que si éramos atacados nos defenderíamos.

    Esta respuesta estuvo muy lejos de satisfacerlos y un gran número de ellos se presentó por la mañana en las inmediaciones del campamento. Viendo sin embargo la excelente posición que ocupábamos no se atrevieron a avanzar más allá del arroyuelo que corría cerca, donde se fueron concentrando en tal número que nos espantaron; los que calculaban con más discreción no veían menos de diez mil. Allí se quedaron un rato observándonos y luego exhalado un horrible alarido enviaron una lluvia de flechas que no alcanzaron a dañar a nadie, pues nos habíamos parapetado detrás de nuestros equipajes.

    Poco después los vimos hacer un movimiento hacia la derecha, como si se dispusieran a flanquearnos, pero entonces un cosaco de Jerawena, hombre astuto y resuelto, habló con nuestro jefe y le dijo:

    —Yo me encargo de desviar a toda esa gente en dirección a Shilka.

    Era aquélla una ciudad situada a cuatro o cinco jornadas hacia el sur, y más bien en sentido contrario al punto donde habíamos llegado nosotros. Tomando su arco y flechas, así como su caballo, el cosaco galopó a nuestra retaguardia como si estuviera por volverse a Nertchinsk, tras de lo cual hizo un gran rodeo y se presentó en el ejército tártaro como si lo hiciera ex profeso para comunicarle noticias; contó que los culpables de la destrucción de Cham-Chi-Thaungu viajaban en dirección a Shilka con una caravana de descreídos (como llamó a los cristianos según la denominación común en esas tierras), agregando que esos mismos individuos tenían la intención de quemar el dios Schal-Isar, reverenciado por los tungusos.

    Como este individuo era de raza tártara y hablaba perfectamente su idioma, los engañó tan bien que le creyeron a pies juntillas y de inmediato se lanzaron como un huracán en dirección a Shilka que según dije se encontraba a unas cinco jornadas de donde estábamos; tres horas más tarde ya los habíamos perdido de vista y jamás volvimos a oír hablar de ellos, por lo cual no sabemos si alcanzaron a llegar al pueblo denominado Shilka.

    Con toda felicidad arribamos a la ciudad de Jerawena, donde había una guarnición de moscovitas y allí permanecimos descansando cinco días ya que la caravana entera estaba fatigada por aquellas duras jornadas y la falta de tranquilidad y reposo durante las noches.

    Partiendo de esa ciudad entramos en un espantoso desierto que nos llevó veintitrés días de marcha para atravesarlo.

    Nos habíamos provisto de algunas tiendas a fin de pasar las noches en ellas, y el jefe de la caravana hizo comprar dieciséis carros o furgones del país para transportar el agua y las provisiones. Por las noches, el círculo formado por los carros constituía nuestra defensa, y aunque se hubiesen presentado los tártaros en gran número no habrían logrado buen éxito en sus ataques.

    Es de imaginarse el ansia que tendría yo de descansar después de semejante travesía. Mientras la efectuábamos vimos abundancia de cazadores de martas cebellinas, partidas de habitantes de la Tartaria Mongólica a la cual pertenece este desierto y que con frecuencia atacan a las caravanas más pequeñas. Nunca vimos gran cantidad de ellos, y aunque nos llenaban de curiosidad las pieles de marta que habían obtenido fue imposible hablar con ellos porque no se animaban a acercarse y nosotros no nos atrevimos a abandonar el grueso de la caravana para salirles al encuentro.

    Cruzado el desierto, entramos en país muy habitado; hallamos pueblos y castillos donde por orden del zar de Moscovia había guarniciones estables que protegían a las caravanas y defendían el país contra los tártaros que de lo contrario hubiesen tornado peligrosas las travesías.

    El gobierno del zar había dado órdenes tan estrictas para la salvaguardia de las caravanas que apenas se recibían noticias de que partidas de tártaros merodeaban por la región las guarniciones enviaban destacamentos para escoltar a los viajeros de etapa en etapa.

    Así fue como el gobernador de Udinsk, a quien tuve oportunidad de visitar por mediación del comerciante escocés que lo conocía, nos ofreció una guardia de cincuenta hombres para que nos escoltaran hasta la próxima ciudad.

    Yo había pensado al principio que a medida que nos fuéramos acercando a Europa hallaríamos regiones más habitadas y gentes de mayor civilización, en lo cual me engañé completamente.

    Cuando cruzamos el país de los tungusos vimos las mismas señales de paganismo y barbarie, si no peor, que observáramos anteriormente. Como eran pueblos conquistados y reducidos por los moscovitas no se mostraban tan peligrosos, pero en cuanto a grosería de modales, idolatría y politeísmo, no creo que ningún otro pueblo en el mundo entero haya conseguido jamás superarlos.

    Todos ellos se visten con pieles de animales y sus casas están construidas del mismo material. No se puede distinguir a una mujer de un hombre, pues se asemejan en los modales y en las ropas. Al llegar el invierno, cuando la tierra se cubre de nieve, se meten en chozas subterráneas, especie de bóvedas comunicadas entre sí por galerías.

    Si los tártaros tenían a su Cham-Chi-Thaungu para toda una región, aquí se encontraban ídolos en cada choza y en cada cueva. Aparte de eso adoran a las estrellas, al sol y las aguas, así como a la nieve. En una palabra, rinden culto a todo lo que no aciertan a entender (y entienden muy pocas cosas) de manera que cualquier objeto, por poco que tenga de sorprendente, les parece propicio a la adoración.

    Pero no es mi deseo hablar de pueblos y países salvo que mi propia historia tenga directa relación con ellos. Nada de particular me ocurrió en esas regiones cuya extensión, a partir del desierto ya citado, calculo en unas cuatrocientas millas de las cuales la mitad están ocupadas por otro desierto cuya travesía nos llevó doce días de penurias, sin casas ni árboles para protegernos y viéndonos obligados a llevar con nosotros el agua y las provisiones indispensables. Pasado el desierto y luego de otras dos jornadas, llegamos a Ienisseisk, una ciudad rusa situada sobre el gran río Ienissei. Allí nos dijeron que ese río señala la división de Europa y Asia, aunque nuestros cartógrafos no concuerdan con esa idea. De lo que no cabe duda es que se trata del límite de la antigua Siberia, que constituye una simple provincia del vasto imperio moscovita y sin embargo es tan glande como todo el imperio germánico.

    Aun en estas regiones, y exceptuando las guarniciones rusas, observé que el paganismo y la ignorancia dominaban. El territorio entre los ríos Obi y Ienissei es pagano y sus habitantes tan bárbaros corno" los más remotos tártaros; no creo que en esto sean sobrepujados por ningún pueblo de Asia o de América.

    Me pareció —y así lo dije a los gobernadores moscovitas con quienes tuve oportunidad de departir— que aquellos paganos no están, por el solo hecho de hallarse bajo la dominación rusa, más próximos al cristianismo ni más capacitados para entenderlo. Todos me dieron la razón, asegurándome sin embargo que aquel problema no les concernía y que si el zar esperaba convertir a sus súbditos siberianos, tungusos o tártaros debía enviar misioneros entre ellos y no soldados; algunos agregaron, con una sinceridad que yo no había esperado, que el zar parecía interesarse más en que aquellos hombres fuesen vasallos que cristianos.

    Para llegar al Obi, desde el otro río, atravesamos un país salvaje y desolado del que no puedo decir que sea estéril sino que las gentes no lo aprovechan; con buena administración se convertiría en la más agradable y fértil de las comarcas.

    Todos los habitantes que vimos eran paganos, excepto aquellos provenientes de la Rusia; conviene aquí decir que éste es el país —a ambas márgenes del Obi— donde son desterrados los criminales que no han merecido sentencia de muerte, y que resulta casi imposible huir de esa región.

    Nada tengo que contar sobre mis asuntos personales hasta llegar a Tobolsk, ciudad capital de Siberia donde hube de vivir algún tiempo a causa de lo que paso a relatar.

    Llevábamos casi siete meses de viaje y el invierno principiaba a hacerse sentir, por lo cual mi socio y yo sostuvimos una conferencia sobre nuestros intereses y pensamos en la conveniencia de decidir el itinerario futuro ya que nuestro destino era Inglaterra y no Moscú. Nos habían hablado de trineos y de renos que nos llevarían sobre la nieve mientras durara el invierno, pues allí existen tales medios de transporte y resulta casi increíble el hecho de que aprovechando esos trineos los rusos prefieren viajar en invierno y no en verano. Pueden recorrer en ellos grandes distancias tanto de noche como de día, y como la nieve se hiela y endurece formando una capa uniforme, todo se convierte en una superficie continua, colinas, valles, ríos y lagos, de una solidez de piedra.

    Pero yo no tenía por qué lanzarme a semejante travesía invernal. Repito que mi destino era Inglaterra y no Moscú y dos rutas se abrían a mi elección. Podía seguir con la caravana hasta Jaroslaw, de ahí encaminarme hacia el oeste para alcanzar Narva y el Golfo de Finlandia y seguir por mar o tierra hasta Danzig, donde esperaba vender con buena ganancia mi cargamento de productos chinos.

    Si elegía el otro camino, después de separarme de la caravana en un pueblecito situado sobre el Duina, tendría seis días de viaje fluvial hasta Arcángel, donde con seguridad conseguiría embarcarme para Inglaterra, Holanda o Hamburgo.

    Pronto advertí, sin embargo, que cualquiera de las dos rutas hubiera sido desastrosa en invierno. Yendo a Danzig habría encontrado helado el Báltico, sin posibilidad de navegar por él; viajar por tierra hacia aquellas regiones resultaba aún menos seguro que entre los tártaros de Mongolia. Lo mismo podía decirse sobre el viaje a Arcángel en pleno mes de octubre; sin duda no había allí ningún barco, pues nadie se queda en invierno en esa latitud y hasta los comerciantes que viven durante el verano en la ciudad se retiran a invernar a Moscú. Solamente encontraría frío, escasez de provisiones y todo un invierno a pasar en un pueblo desolado.

    Después de pensarlo bien me resolví por fin a abandonar la caravana y quedarme durante el invierno allí donde me encontraba, o sea Tobolsk, en Siberia, ciudad situada a cincuenta y ocho grados de latitud; tenía así la seguridad de disponer de tres cosas imprescindibles para una invernada: abundancia de alimentos y de provisiones, casa abrigada con suficiente combustible y excelente compañía. De todo esto daré minucioso detalle en su debido lugar.

    Me encontraba ahora en un clima muy distinto del de mi querida isla, donde nunca sentí frío salvo cuando estuve enfermo de fiebres, y por el contrario me costaba soportar el peso de la ropa en los hombros y jamás encendía fuego si no era al aire libre y por la necesidad de cocer mis alimentos. Me mandé hacer tres gruesos trajes, así como unos abrigos que me llegaban hasta los pies y se abotonaban en las muñecas, enteramente forrados en pieles para que conservaran el calor.

    En cuanto a una casa conveniente, debo confesar que no me agrada el método inglés de encender fuego en cada habitación y en chimeneas abiertas, las cuales, una vez que el fuego se ha consumido, sólo sirven para que el aire se enfríe tanto como el del exterior.

    Por el contrario, luego de alquilar un departamento en una excelente casa de la ciudad, hice que construyeran en el centro de mis seis habitaciones una chimenea a manera de hornilo, como una verdadera estufa. El caño que recibía el humo iba en dirección contraria a la abertura que servía para alimentar el fuego, y en esa forma todas las habitaciones tenían calor equilibrado sin que se viera el fuego tal como se hace para calentar los baños en Inglaterra.

    En esta forma gozábamos siempre de idéntica temperatura en todas las habitaciones, y aunque afuera hiciese frío riguroso en la casa había un ambiente agradable, sin verse fuego alguno ni soportar la incomodidad del humo. Lo más extraordinario de todo fue encontrar grata compañía en aquella región tan bárbara, la más septentrional de toda Europa, cerca del Mar Glacial, y a pocos grados de diferencia con Nueva Zembla. Ya he dicho sin embargo que en esta región eran desterrados los reos de estado y la ciudad estaba llena de nobles, príncipes, caballeros, coroneles y la corte de Moscovia. Se encontraba allí el famoso príncipe Galitzin, el anciano general Robostisky y muchas otras personas distinguidas, así como no pocas damas.

    Por intermedio del comerciante escocés —quien dicho sea de paso se separó de mí en esta ciudad— trabé relación con varios de aquellos caballeros, algunos pertenecientes a la más alta nobleza, y durante las largas noches de invierno recibí con frecuencia sus muy gratas visitas. Recuerdo que una noche hablaba con el príncipe «X...», uno de los ex ministros de estado del zar, que fuera desterrado a Siberia, cuando la conversación recayó sobre mi persona. Me había estado narrando con abundancia de detalles la grandeza, magnificencia, dominios y absoluto poder del Emperador de las Rusias. Lo interrumpí entonces para manifestarle que yo era un príncipe aún más poderoso que el zar de Moscovia, aunque mis dominios no fuesen tan dilatados ni tan numerosa mi población. El grande de Rusia me contempló con algo de sorpresa, y fijando su mirada en mí empezó a preguntarse qué quería yo decir con aquello.

    Le aseguré que su asombro cesaría una vez que le explicara mi posición, y ante todo le hice saber que disponía en absoluto de la vida y fortuna de todos mis súbditos, pero no obstante mi omnipotencia no había en mis tierras una sola persona que se manifestara contraria a mi gobierno.

    Al oírme movió la cabeza, murmurando que ciertamente excedía yo en eso al zar de Moscovia. Agregué entonces que todas las tierras de mi dominio eran de mi propiedad privada, por lo cual los súbditos eran solamente arrendatarios y eso mientras a mí me pareciera bien; que estaban dispuestos a luchar por mí hasta la última gota de su sangre y que jamás tirano alguno —porque reconocía yo serlo— fue tan universalmente amado y a la vez tan terriblemente temido por sus vasallos.

    Después de continuar un rato con tan divertidos enigmas políticos dije la verdad a mis oyentes contándoles en detalle la historia de mi vida en la isla y cómo después de gobernarme a mí mismo llegué a hacerlo con mis colonos según lo he contado detalladamente.

    El relato les interesó vivamente, en especial al príncipe, quien, con un suspiro, me dijo que la verdadera grandeza de una vida consiste en llegar a ser el dueño de uno mismo, y que jamás habría él cambiado una condición como la mía con la del zar de las Rusias. Agregó que personalmente había encontrado más dicha en el retiro de su obligado destierro que antaño en el poder y autoridad que había gozado en la corte de su amo y señor el zar. Pensaba que la mayor sabiduría humana estaba en amoldarse a las circunstancias y conservar la serenidad interior en medio de las peores tempestades exteriores.

    —Y no creáis, señor —agregó—, que trato con estas ideas de adaptarme en modo alguno a mis presentes circunstancias, que acaso algunos consideren miserables; creed que de ninguna manera querría yo volver a la corte aunque el zar, mi señor, me llamara para devolverme mi antigua grandeza. No quisiera hacerlo al igual que mi alma, el día en que sea liberada de su prisión corporal y alcance a gustar su celestial condición, no querrá volver a la cárcel de carne y sangre que la aprisiona ahora, y cambiar el cielo por el polvo y la miseria de las cosas humanas.

    Pronunció aquellas palabras con profundo calor, manifestando tanta emoción y vehemencia en su acento y en su rostro que no tuve duda de la sinceridad que las inspiraba y de los sentimientos de su alma.

    No quiero extenderme en demasía detallando la muy grata conversación que sostuve con aquel hombre, en el curso de la cual me hizo conocer que su espíritu estaba inspirado por un conocimiento profundo de las cosas y acrecentado tanto por la religión como por la sabiduría, al extremo de que el desprecio que manifestaba por las miserias terrenales era legítimo y sincero, manteniéndose invariable hasta el fin, como se verá en el relato que haré de inmediato.

    Llevaba yo en la ciudad ocho meses, y aquel invierno me parecía intenso y mortificante, con un frío tan riguroso que me impedía asomarme fuera sin envolverme antes en pieles, llevando una especie de máscara sobre el rostro, o más bien una caperuza con un orificio para respirar y dos para ver. La poca luz diurna que tuvimos durante tres meses no excedía de cinco horas o seis como máximo.

    Como el tiempo estaba sin embargo despejado, y la nieve cubría totalmente el suelo, nunca era completamente oscuro.

    Nuestros caballos vivían en muy malas condiciones, medio muertos de hambre y en cuevas subterráneas. En cuanto a nuestros sirvientes —pues habíamos ajustado a tres para que nos atendieran así como a nuestros caballos— constantemente nos veíamos precisados a darles fricciones en las manos y pies para impedir que los atacase la gangrena y los perdiesen.

    Cierto que dentro de la casa hacía calor, pues teníamos todo cerrado, las ventanas eran pequeñas y con cristales dobles. Nos alimentábamos principalmente de carne de ciervo, secada y salada en la debida época. Comíamos buen pan, sólo que lo cocían de una manera semejante a la de los bizcochos, pescado seco de muchas clases así como algo de carne de oveja y también de búfalo, que es muy sabrosa. Parece que todas las reservas de provisiones son dispuestas y curadas en el verano; bebíamos aguardiente mezclado con agua en vez de coñac, y en lugar de vino, hidromiel, que allá es de excelente calidad. Los cazadores, que se animan a salir en cualquier época, nos traían con frecuencia carne fresca de venado, muy sabrosa y nutritiva, y otras veces carne de oso que no nos gustaba tanto. Poseíamos abundante reserva de té, con el cual agasajábamos a los amigos que he mencionado, y en esa forma vivíamos muy bien y agradablemente si se tiene en cuenta las circunstancias.

    Llegó el mes de marzo y empezaron a alargarse los días, mejorándose el tiempo, por lo cual los demás viajeros empezaron a alistar trineos y a prepararse para la partida. Mis medidas estaban sin embargo tomadas y como seguía decidido a viajar por la ruta de Arcángel y no vía Moscú no me preocupé en lo más mínimo sabiendo que los barcos del sur jamás se aproximan a aquellas altas latitudes antes de mayo o junio, de manera que si llegaba a Arcángel a comienzos de agosto, tendría tiempo de sobra para embarcarme. Vi, pues, cómo todos los restantes viajeros se iban marchando, hasta que no quedó ninguno en la ciudad.






	\chapter{El retorno a la patria}





    Hacia fines de mayo empecé a organizar mi partida y fue entonces cuando, al ocuparme en los preparativos, empecé a reflexionar sobre todos aquellos hombres desterrados por el zar a Siberia, quienes vivían allí en la más completa libertad de movimientos, y a preguntarme por qué no aprovechaban la oportunidad para fugarse a alguna parte del mundo donde pudieran encontrarse mejor. Lleno de perplejidad quise averiguar qué razones podían detenerlos en semejante empresa.

    Mi asombro cesó sin embargo cuando lo manifesté a la persona de quien ya he hablado, quien me dio la siguiente respuesta:

    —Considerad, caballero —dijo—, el lugar en el cual nos hallamos y, en segundo término, las condiciones en que vivimos; pensad también en la calidad de aquellos que soportan este destierro. Estamos rodeados por cosas mucho más seguras que barrotes y cerrojos; por el norte se extiende un océano innavegable al cual jamás se aventura un navío. Aunque dispusiéramos de un barco  o una chalupa,   ¿adonde iríamos con él?

    »En cuanto a los otros caminos —agregó— obligan a recorrer más de mil millas en los dominios del zar, y las tierras no son transitables más que por los caminos reales, atravesando pueblos donde existen guarniciones; forzosamente seríamos descubiertos o moriríamos de hambre en caso de elegir otras rutas, de manera que ya veis lo vano de semejante tentativa.

    Me quedé silencioso, pensando que ciertamente era aquélla una prisión donde los desterrados estaban tan seguros como si los hubieran encarcelado en Moscú. Mientras seguía entregado a tales reflexiones se me ocurrió que acaso pudiera ser yo un instrumento de liberación para aquel excelente amigo, y que aunque su evasión resultara muy arriesgada me animaría a intentarla. Una noche le participé mis pensamientos, diciéndole que me parecía fácil llevarlo en mi viaje ya que nadie lo vigilaba especialmente en la ciudad y yo pensaba marchar por la ruta de Arcángel y no de Moscú. Además, como era considerado miembro de una caravana, no estaba obligado de ningún modo a detenerme o pernoctar en los pueblos de la ruta sino que podía levantar mi campamento en cualquier parte del camino que me placiera; en esa forma, agregué, al llegar sin inconvenientes a Arcángel lo haría subir de inmediato a un barco holandés o inglés en el cual estaría a salvo apenas nos hiciéramos a la vela.

    Me escuchó, atentamente, mirándome con verdadero transporte mientras le hablaba, y pude leer en su rostro que cuanto le decía lo emocionaba profundamente. Cambiaba a cada instante de color, sus ojos se enrojecieron y tan precipitadamente debía latir su corazón que se notaba en sus facciones; no le fue posible contestar inmediatamente y se limitó a abrazarme mientras yo esperaba su respuesta.

    — ¡Cuan desvalidos somos, pobres seres sin defensa —exclamó—, que hasta nuestros más sinceros actos de amistad pueden resultar trampas tendidas para perdernos o instrumentos de la tentación! Mi querido amigo —agregó—, vuestra oferta es tan sincera, demuestra tanta bondad y desinterés, así como voluntad de hacerme un bien, que debería ser yo harto ignorante del mundo si no me maravillara al escucharla y no reconociera la profunda deuda que tal proposición me hace contraer hacia vos. Decidme, ¿creéis que era sincero cuando tantas veces os manifestaba mi desprecio por las cosas mundanas? ¿Pensasteis que os hablaba con el corazón en la mano y que verdaderamente había logrado aquí un grado tal de paz y felicidad capaz de hacerme despreciar cuanto pudiera darme el mundo y sus grandezas? ¿De veras me creísteis cuando os dije que jamás volvería a mi tierra aunque el zar me llamara a su corte y me devolviera mi grandeza y mis dominios? ¿Verdaderamente visteis en mí a un hombre que hablaba con  sinceridad,  o  sólo  a  un  hipócrita jactancioso?

    Se detuvo, como si quisiera conocer mi opinión, pero advertí que había callado porque la agitación de su espíritu no lo dejaba proseguir con su discurso; aquel gran corazón estaba demasiado conmovido y agitado para conservar el don de expresarse en palabras. Confieso que al escucharlo me sentí profundamente asombrado tanto por las palabras como por lo que demostraban sobre aquel nombre, pero insistí sin embargo con toda clase de argumentos para convencerlo de que debía buscar su libertad. Le dije que mirara mi propuesta como una puerta abierta por el mismo Cielo para su liberación, así como una intimación de la Providencia que rige el destino de todas las cosas en procura de su mayor bien y de que se convirtiera en un hombre útil para el mundo.

    Para entonces ya se había recobrado un poco.

    —¿Y cómo sabéis, señor —me preguntó—, si vuestra proposición es una llamada del Cielo y no, por el contrario, la tentación de un poder muy distinto, el cual trataría de mostrarme con los más brillantes colores la felicidad de mi liberación para tenderme una trampa y arrastrarme a la ruina? Vedme aquí donde estoy, libre de todo deseo de retornar a mi antigua y deleznable grandeza; allá, en cambio, quién sabe si la semilla del orgullo, de la ambición, la avaricia y la pompa que están siempre latentes en la naturaleza humana, no revivirían para arraigarse en mí y dominarme como antaño. Entonces este feliz prisionero, a quien veis ahora dueño de la libertad de su alma, volvería a ser el miserable esclavo de sus sentidos a causa de su excesiva libertad. ¡Ah, mi querido amigo, dejadme en este bendito destierro, alejado de todos los crímenes de la vida! ¡No me llevéis a adquirir una ilusión de libertad a expensas de la verdadera libertad interior, a expensas de una dicha futura que preveo ahora, pero que allá perdería pronto de vista! Pensad que no soy más que un hombre, un pobre hombre de carne y hueso, con pasiones e impulsos prontos a arrebatarme como le ocurre a todo ser humano. ¡Oh, no seáis a la vez mi amigo y mi tentador!

    Si antes me había sorprendido, ahora permanecí como atontado y en silencio, mirándolo y sintiendo crecer en mi interior la admiración que por él experimentaba. Era tan grande la lucha que se libraba en su alma que a pesar del tiempo extremadamente frío su rostro aparecía bañado en sudor, y comprendí entonces que le era necesario desahogar libremente sus pensamientos. Le dije unas pocas palabras, agregando que le daría tiempo para que considerara mi propuesta antes de volver a entrevistarlo,  y me marché a mi casa.

    Unas dos horas más tarde oí que alguien andaba en la puerta y fui a abrirla, pero él ya lo había hecho por la parte de afuera y entrado en mi casa.

    —Querido amigo —dijo—, creed que hace un rato llegasteis a trastornarme, pero ya me he recobrado y os pido que no toméis a mal que no acepte vuestra proposición. Os aseguro que no paso por alto la generosidad que contiene y demuestra, y he venido expresamente para manifestaros tal cosa. Pero a la vez quiero deciros que he conseguido triunfar sobre mí mismo.

    —Alteza —repuse—, confío en que al proceder así estéis bien seguro de que no lo hacéis en contra de una intimación del Cielo.

    —Caballero —dijo entonces—, si hubiese sido un llamado celeste, el mismo poder hubiera influido sobre mí para que lo aceptara. Por el contrario, espero y tengo la seguridad de que es mi negativa la que se apoya en un designio del Cielo, y al separarnos me queda la infinita satisfacción de que os apartáis de un hombre que sigue siendo íntegro, si no libre.

    No me quedaba más que rendirme ante tal cosa, aparte de repetir que mi intención no había tenido otro fin que el de serle útil. Me abrazó afectuosamente, asegurándome que en ningún momento había dudado de ello y que lo recordaría mientras viviera. A continuación me hizo un valioso regalo en pieles de marta cebellina, obsequio que me parecía excesivo recibir de un hombre en tales circunstancias, pero que resultó imposible rehusar por su firme insistencia.

    A la mañana siguiente envié a mis criados para que llevasen a Su Alteza un regalo de té, dos piezas de damasco chino y cuatro pequeñas cuñas de oro del Japón que no pesaban más de seis onzas, lo cual en conjunto era mucho menos valioso que su regalo de pieles, que al llegar a Inglaterra me fue tasado en doscientas libras esterlinas.

    Mi amigo aceptó el té, así como una de las piezas de damasco y un lingote de oro que tenía muy finamente impreso el sello del Japón, recibiéndolo a causa de la rareza de este grabado. No quiso aceptar nada más y me mandó decir por mis criados que deseaba hablar conmigo.

    Al llegar a su casa me recordó lo acontecido entre nosotros rogándome que no intentara en modo alguno doblegar su voluntad, pero, desde que yo le había hecho tan generosa oferta, deseaba saber si estaría dispuesto a mantenerla para otra persona por cuya suerte se interesaba sobremanera.

    Le respondí que no me sentía tan inclinado a mantener mi proposición para otro que no fuera él ya que lo había hecho por el particular aprecio que me merecía y porque me hubiese llenado de contento ser el instrumento de su liberación. Sin embargo, desde que era un pedido suyo, le rogaba que me hiciese saber el nombre de su protegido, agregando que pensaría entonces la conducta a seguir y que confiaba en que no se disgustaría conmigo si mi respuesta no era la que esperaba. Supe entonces que se trataba de su hijo, quien, aunque yo no lo conocía, se encontraba en su misma situación de desterrado a unas doscientas millas sobre la orilla opuesta del Obi. Declaró que si le daba mi consentimiento lo enviaría a buscar al punto.

    Naturalmente respondí sin vacilar que estaba dispuesto a ayudarlo, agregando que si procedía en tal forma era solamente por él, ya que advirtiendo la imposibilidad de convencerlo de que me acompañara deseaba demostrarle mi aprecio en la persona de su hijo. Muchas otras cosas le dije, pero sería tedioso repetirlas aquí. Al otro día envió a buscar a su hijo, quien estuvo de vuelta con el mensajero unos veinte días más tarde, trayendo seis o siete caballos cargados de pieles finas y que sumaban grandísimo valor. Los sirvientes hicieron entrar los caballos en la ciudad, pero el joven caballero quedó fuera hasta la noche, entrando entonces de incógnito en nuestra casa, donde su padre me lo presentó para que nos pusiéramos de acuerdo sobre la manera de viajar y las necesidades de aquella travesía.

    Había yo comprado considerable cantidad de martas, zorros negros y armiños, así como otras pieles muy finas, trocándolas por una parte de los productos adquiridos en la China, especialmente clavo y nuez moscada, de los cuales vendí la mayor parte y el resto en Arcángel, a un precio muy superior al que hubiese podido conseguir en Londres. Mi socio, que atendía preferentemente a los beneficios ya que tal era su negocio, estaba sumamente contento de nuestra permanencia en Tobolsk a causa de las buenas ganancias allí logradas.

    Fue a comienzos de junio que partí de aquella remota ciudad, la cual es sin duda poco conocida en el mundo, ya que se halla tan alejada de toda ruta comercial que apenas hay ocasión de mencionarla. Formábamos ahora una pequeña caravana con unos veinticinco caballos y camellos en total, y todos figuraban como de mi pertenencia, aunque mi nuevo huésped era dueño de once animales. Resultaba perfectamente natural que llevara conmigo más sirvientes que antes, y el joven señor se pasaba por mi mayordomo; yo debía ser considerado un gran señor, aunque no me preocupé de averiguar lo que pensaban a mi respecto.

    Enfrentamos ante todo el peor y más dilatado de los desiertos que halláramos en todo aquel largo viaje, pero nos consolaba la seguridad de que no encontraríamos partidas de tártaros o bandoleros, ya que jamás cruzan a este lado del Obi o lo hacen raramente; sin embargo, pronto advertimos que las cosas no eran así.

    Mi joven señor llevaba consigo a un fiel criado moscovita o más bien siberiano, quien conocía al dedillo la ruta y nos condujo por caminos privados que nos evitaron tener que cruzar los principales pueblos y ciudades del camino tales como Tuimen, Solikamsk y algunas otras, ya que las guarniciones moscovitas que allí tienen asiento se muestran muy estrictas en el examen de los pasajeros así como en la revisión de equipajes, para impedir que alguno de los nobles en Siberia pueda fugarse hacia Moscovia. Ahora bien, como avanzábamos evitando las ciudades, nuestras jornadas se efectuaban a través de un verdadero desierto y debíamos pernoctar en tiendas en vez de gozar de cómodos alojamientos en las ciudades. El joven señor se manifestó contrario a tal cosa y no quiso permitir que nos privásemos de tales descansos, por lo cual quedó convenido que permanecería en las afueras refugiado en los bosques en compañía de su criado, y que volveríamos a encontrarnos en sitios convenidos de antemano.

    Entrábamos ahora en Europa, después de cruzar el río Kama que en esa latitud es límite entre Europa y Asia: la primera ciudad que hallamos del lado europeo fue Solikamsk, palabra que significa la gran ciudad del Kama. Pensábamos al entrar en esos territorios que encontraríamos grandes diferencias en los habitantes tanto en lo referente a sus maneras y costumbres como en materia religiosa y comercial. Si embargo, nos equivocamos, pues a medida que atravesábamos aquel vasto desierto (que según algunos tiene seiscientas millas de largo en ciertas partes, aunque sólo doscientas media en nuestra ruta) las gentes que vimos no se diferenciaban mucho de los tártaros mongoles; casi todas eran paganas, apenas algo mejor que los salvajes de América. Sus casas y aldeas aparecían sembradas de ídolos y ellos vivían de la manera más bárbara, salvo en las ciudades antes mencionadas y sus pueblos adyacentes. Allí había cristianos pertenecientes a lo que llaman Iglesia Griega, pero su religión está mezclada con supersticiones que en algunos sitios apenas se diferencia de la hechicería y la magia.

    Mientras atravesábamos aquellas florestas, y cuando pensábamos hallarnos a salvo de todo peligro, estuvimos sin embargo a punto de ser asaltados, saqueados y a la vez muertos por una pandilla de bandoleros. Ignoro de qué país procedían, si eran bandas de errantes ostiacos —vecinos de los tártaros— o gentes salvajes de las riberas del Obi, a menos que se tratara de cazadores siberianos de martas cebellinas. El hecho es que montaban a caballo, estaban armados de arco y flechas y no eran menos de cuarenta y cinco hombres. Se nos acercaron repentinamente hasta ponerse a dos tiros de mosquete, y sin decirnos palabra formaron un círculo en torno nuestro observándonos en esa forma por dos veces consecutivas, hasta que por fin se apostaron justamente en nuestro camino. De inmediato nos tendimos en línea delante de nuestros camellos, sin contar con más de dieciséis hombres para nuestra defensa; así aprestados enviamos al criado siberiano del joven señor para que parlamentase con aquellos individuos. Su amo en persona le ordenó que averiguase la procedencia de esas gentes, pues no estaba poco intranquilo temiendo que se tratara de tropas siberianas enviadas en su» persecución.

    El criado se aproximó llevando bandera de parlamento, y les habló en distintos idiomas (o más bien en distintos dialectos) sin conseguir entender una sola palabra de lo que le contestaron. Por eso, y luego que le hicieran signos de que no se acercara demasiado, mostrándole claramente su intención de disparar sobre él si lo hacía, el hombre regresó sin haber averiguado nada, aunque declaró que por sus vestidos le daban la impresión de ser tártaros calmucos o un grupo disgregado de alguna horda circasiana, imaginando que debía haber más cantidad en aquel vasto desierto cosa extraña porque jamás había oído que alcanzaran a subir tanto hacia el norte.

    Todo esto no era ningún consuelo para nosotros, pero nada más podía hacerse sin embargo. A nuestra mano izquierda y a un cuarto de milla se alzaba un bosquecillo con árboles muy juntos y próximos al camino. Resolví de inmediato que debíamos refugiarnos a ese abrigo, fortificándonos lo mejor posible, ya que en primer lugar los árboles serían excelente protección contra las flechas, y luego impedirían a los atacantes cargar contra nosotros a caballo. Fue en realidad mi anciano piloto portugués quien tuvo la idea, como siempre ocurría en casos de grave peligro y en los cuales se mostraba el más apto para encontrar la salida favorable. Avanzamos de inmediato con toda la rapidez posible y ganamos el bosquecillo mientras los tártaros (o ladrones, ya que no sé cómo llamarles) se mantenían inmóviles y sin intentar atacarnos por la retaguardia.

    Al llegar a nuestro refugio comprobamos con gran satisfacción que se alzaba en un terreno pantanoso, y que a un lado había un manantial del cual nacía un arroyuelo que, algo más adelante, se unía a otro de aspecto semejante formando —como supimos luego— las cabeceras de un gran río denominado Wirtzka. Los árboles que crecían en torno al manantial no pasaban de doscientos, pero como eran muy corpulentos y próximos unos a otros, tan pronto nos colocamos detrás tuvimos magnífica defensa contra los enemigos, cuya única probabilidad estaba en desmontar y atacarnos cuerpo a cuerpo. Para que esto les fuese difícil, nuestro infatigable portugués cortó gruesas ramas y las dejó colgando, sin arrancarlas del todo, de un árbol a otro hasta formar una empalizada en torno a nuestras líneas.

    Permanecimos allí algunas horas, espiando los movimientos del enemigo, que sin embargo se mantenía inactivo. Cuando faltaban unas horas para la noche los vimos marchar de improviso al ataque, notando que algunos otros se les habían agregado hasta el punto de constituirse una fuerza de ochenta jinetes, varios de los cuales nos dieron la impresión de ser mujeres. Se acercaron hasta quedar a medio tiro del bosquecillo, y entonces disparamos un mosquete sin balas y les preguntamos en ruso qué querían de nosotros y les mandamos alejarse. Como no parecían entender nada de lo que les dijimos, avanzaron con redoblada furia directamente hacia nuestro refugio, sin imaginarse que nos habíamos protegido con semejante barricada. Nuestro anciano piloto era ahora capitán como antes ingeniero, y nos mandó que no disparásemos hasta que no los tuviésemos a tiro de pistola, a fin de hacer buena puntería. Le pedimos que diera en persona la orden de fuego y no lo hizo hasta que el enemigo estuvo casi junto a  nosotros,  y entonces  disparamos  nuestras  armas.

    Tan bien habíamos podido hacer puntería, o la Providencia dirigió nuestro fuego tan certeramente, que matamos a catorce y herimos a muchos otros, así como a numerosos caballos; cada uno había cargado su arma por lo menos con tres balas.

    Terriblemente sorprendidos por nuestro fuego, retrocedieron en confusión unas cien yardas, tiempo que aprovechamos para cargar nuevamente las piezas. Como habíamos calculado la distancia a que se encontraban, les hicimos una salida en el curso de la cual capturamos cuatro o cinco caballos cuyos jinetes habían muerto. Acercándonos a los cadáveres comprobamos que se trataba de tártaros, aunque no pudimos precisar de qué región y cómo habían podido llegar a una distancia tan enorme de sus tierras.

    Una hora más tarde parecieron dispuestos a reanudar el ataque y rondaron el bosque buscando algún punto débil. Como advirtieron que los estábamos esperando en todas direcciones, se retiraron de nuevo y nos dedicamos a permanecer atentos toda la noche, y sin movernos de allí. Podéis imaginaros lo mal que habremos dormido, ya que empleamos la noche en reforzar nuestras barricadas así como las entradas del bosque, y apostamos centinelas constantemente alertas.

    La luz del día nos trajo un triste descubrimiento. El enemigo, a quien suponíamos descorazonado por su anterior fracaso, había crecido en número y estaba ahora compuesto por no menos de trescientos hombres, quienes habían alzado once o doce tiendas como si tuvieran intención de sitiarnos. El campamento se hallaba en campo abierto, a unos tres cuartos de milla de nuestro fuerte. Es de suponer lo que nos sorprendería nuestro descubrimiento y debo confesar que en ese momento me consideré perdido con todo lo que llevaba. Por cierto que la pérdida de mis bienes no me afligía mucho, bien que fuesen considerables, pero sí la idea de caer en manos de aquellos bárbaros justamente hacia el final de mi largo viaje y después de sobrellevar tantos azares y tantas dificultades, casi a la vista del puerto donde esperábamos la salvación y libertad. En cuanto a mi socio, estaba abiertamente furioso y declaraba que la pérdida de sus bienes significaría su ruina, prefiriendo por su parte morir en la batalla antes que de hambre, por lo cual era de parecer que empeñásemos el combate.

    El joven señor, gallardo caballero como el que más, era también de la opinión de que se luchara, mientras el anciano piloto sostenía la conveniencia de seguir la resistencia donde nos encontrábamos. En este debate transcurrió el día, pero al anochecer descubrimos que el número de enemigos iba en aumento. Tal vez, como estaban divididos en varias partidas que acechaban su presa, los primeros habían enviado mensajeros para que convocasen a los otros a fin de distribuirse el botín. No nos atrevíamos a pensar en cuánto se elevaría el número cuando llegara el día, de manera que empecé a preguntar a las gentes que habíamos traído de Tobolsk si existía algún otro camino privado o atajo por el cual pudiésemos fugarnos durante la noche, encontrando acaso un refugio en cualquier pueblo o refuerzos que nos auxiliaran en el desierto.

    El siberiano criado del joven señor, nos dijo que si estábamos dispuestos a evitar el combate y escaparnos, él nos llevaría por un camino que iba hacia el norte, rumbo al río Petrov, garantizándonos que los tártaros no lo advertirían; nos dijo, sin embargo, que su amo le había asegurado que jamás se retiraría, prefiriendo quedarse y combatir.

    A esto le respondí que estaba grandemente equivocado acerca de las intenciones de su amo, quien era demasiado sensato para luchar por el solo gusto de hacerlo, agregando que de la manifiesta bravura de su señor tenía yo harta prueba con su comportamiento anterior, y que indudablemente no querría él obligar a diecisiete o dieciocho hombres a pelear con quinientos, salvo que una necesidad inevitable así lo dispusiera, de manera que si existía un camino abierto para nuestra salvación no nos quedaba más que seguirlo durante la noche.

    Contestó entonces que si su amo le daba órdenes en ese sentido, se dejaría matar antes de faltar a su cumplimiento. Seguros de esto, pronto convencimos a su señor que se lo mandase, hablándole en privado, y nos dispusimos a llevar a efecto la fuga.

    Tan pronto oscureció encendimos una hoguera en nuestro campamento y la mantuvimos constantemente alimentada, dejándola de tal modo dispuesta que continuara encendida la noche entera para hacer creer a los tártaros que seguíamos allí. Cuando oscureció y pudimos ver las estrellas (pues nuestro guía no se atrevía a salir antes) cargamos los camellos y caballos y seguimos a aquel hombre que, según comprendí, se guiaba por la estrella polar, ya que no había otra orientación en aquel terreno tan llano.

    Después de andar rápidamente durante dos horas notamos que empezaba a aclarar a causa de la luna naciente, lo cual nos produjo no poca inquietud. A las seis de la mañana estábamos a cuarenta millas de distancia, aunque preciso sea decir que casi reventamos los caballos. Hallamos un pueblo ruso llamado Kermazinskoy (¿Kertchemskoy?) donde pudimos descansar sin tener aquel día más noticias de los tártaros calmucos. Dos horas antes del anochecer partimos nuevamente, viajando hasta las ocho de la mañana siguiente aunque ya no con el sigilo de antes. A eso de las siete cruzamos un riacho llamado Kirtza para arribar luego a un gran pueblo habitado por rusos, sumamente populoso y llamado Ozomoys (?). Oímos allí que muchas partidas u hordas de calmucos habían merodeado por el desierto, pero que ya nos encontrábamos alejados de todo peligro, cosa que nos tranquilizó no poco, como es de imaginar. Adquirimos algunos caballos de refresco, y para tomarnos el descanso que necesitábamos decidimos permanecer allí cinco días; mi socio y yo acordamos entretanto premiar al fiel siberiano que nos trajera sanos y salvos con un  valor  de diez pistolas.

    Cinco días más tarde llegamos a Veuslima (?) sobre el río Vichegda, que desagua en el Duina, y nos hallamos felizmente al final de nuestro viaje por tierra, pues el río es navegable y en siete días de barca podríamos estar en Arcángel. Ante todo nos dirigimos a Lawrensoy (¿Jarensk?), llegando el 3 de julio, donde compramos dos botes para transportar las mercancías y una barca para nosotros, embarcándonos el día 7 y arribando sanos y salvos el 18 a Arcángel después de un año, cinco meses y tres días de viaje en el que se incluye la permanencia de ocho meses y días en Tobolsk.

    Seis semanas nos vimos obligados a permanecer en aquel puerto a la espera de que arribara algún navío, y acaso la permanencia hubiese sido mayor a no presentarse un barco hamburgués antes que los ingleses. Pronto decidimos que Hamburgo podía ser tan buen mercado como Londres para la venta de nuestras mercaderías y sacamos pasaje en aquel navío. Puestos ya mis efectos a bordo, era harto natural que enviase a mi mayordomo para que los vigilara, de manera que el joven señor tuvo oportunidad de quedarse escondido en el barco, sin bajar una sola vez a tierra mientras permanecimos allí; fue una buena precaución, pues él temía que algún comerciante de Moscú alcanzara a reconocerlo y dar la alarma.

    Salimos de Arcángel el 20 de agosto del mismo año, y luego de un viaje no del todo malo llegamos a Elba el 13 de setiembre. Aquí mi socio y yo hallamos excelente mercado para nuestras mercaderías, tanto los originarios de la China como las pieles finas siberianas. Al liquidar el producto por las ventas, mi parte ascendió a tres mil cuatrocientas setenta y cinco libras esterlinas, dieciséis chelines y tres peniques, a pesar de los muchos gastos que habíamos tenido durante el largo y accidentado viaje. Conviene decir que en esa suma se incluyen unas seiscientas libras esterlinas que constituían el valor de los diamantes comprados por mí en Bengala.

    El joven señor se despidió allí de nosotros, remontando el Elba a fin de llegar a la corte de Viena donde estaba resuelto a buscar amparo y desde la cual podría comunicarse con los amigos de su padre que aún vivían. No se marchó sin manifestarme en todas las formas posibles su gratitud por los servicios que yo le prestara, así como por la conducta observada hacia el príncipe su padre.

    En conclusión: después de vivir cuatro meses en Hamburgo y viajar de allí a La Haya, me embarqué en un paquebote y llegué a Londres el 10 de enero de 1705, después de permanecer ausente de Inglaterra durante diez años y nueve meses.

    Ahora, resuelto a no reanudar las andanzas, me apresto a emprender un viaje mucho más extenso que todos los otros, habiendo vivido setenta y dos años de una existencia infinitamente accidentada y aprendido a conocer por ella el valor del sosiego y la bendición de concluir en paz mis días.




\Fin

\end{document}

